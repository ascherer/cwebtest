\input cwebmac
\datethis

\N{1}{1}Intro. This program, hacked from {\mc ACHAIN4}, finds all
canonical addition chains of minimum length for a given integer.

There are two command-line parameters. First is a file that
contains values of $l(n)$, as output by the previous program.
Then comes the desired integer $n$.

\Y\B\4\D$\\{nmax}$ \5
\T{10000000}\C{ should be less than $2^{24}$ on a 32-bit machine }\par
\Y\B\8\#\&{include} \.{<stdio.h>}\6
\&{unsigned} \&{char} \|l[\\{nmax}];\6
\&{int} \|a[\T{128}]${},{}$ \|b[\T{128}];\6
\&{unsigned} \&{int} ${}\\{undo}[\T{128}*\T{128}];{}$\6
\&{int} \\{ptr};\C{ this many items of the \PB{\\{undo}} stack are in use }\6
\&{struct} ${}\{{}$\1\6
\&{int} \\{lbp}${},{}$ \\{ubp}${},{}$ \\{lbq}${},{}$ \\{ubq}${},{}$ \|r${},{}$ %
\\{ptrp}${},{}$ \\{ptrq};\2\6
${}\}{}$ \\{stack}[\T{128}];\6
\&{int} \\{tail}[\T{128}]${},{}$ \\{outdeg}[\T{128}]${},{}$ \\{outsum}[%
\T{128}]${},{}$ \\{limit}[\T{128}];\6
\&{int} \\{down}[\\{nmax}];\C{ a navigation aid discussed below }\6
\&{FILE} ${}{*}\\{infile};{}$\7
\\{main}(\&{int} \\{argc}${},\39{}$\&{char} ${}{*}\\{argv}[\,]){}$\1\1\2\2\6
${}\{{}$\1\6
\&{register} \&{int} \|i${},{}$ \|j${},{}$ \|n${},{}$ \|p${},{}$ \|q${},{}$ %
\|r${},{}$ \|s${},{}$ \\{ubp}${},{}$ \\{ubq}${}\K\T{0},{}$ \\{lbp}${},{}$ %
\\{lbq}${},{}$ \\{ptrp}${},{}$ \\{ptrq};\6
\&{int} \\{lb}${},{}$ \\{nn};\7
\X2:Process the command line\X;\6
\X7:Initialize the \PB{\\{down}} table\X;\6
\&{for} ${}(\|n\K\T{1};{}$ ${}\|n\Z\\{nn};{}$ ${}\|n\PP){}$\5
${}\{{}$\1\6
\X3:Input the next value, \PB{\|l[\|n]}\X;\6
\X8:Update the \PB{\\{down}} links\X;\6
\4${}\}{}$\2\6
\X4:Backtrack through all solutions\X;\6
\4${}\}{}$\2\par
\fi

\M{2}\B\X2:Process the command line\X${}\E{}$\6
\&{if} ${}(\\{argc}\I\T{3}){}$\5
${}\{{}$\1\6
${}\\{fprintf}(\\{stderr},\39\.{"Usage:\ \%s\ infile\ n\\}\)\.{n"},\39\\{argv}[%
\T{0}]);{}$\6
${}\\{exit}({-}\T{1});{}$\6
\4${}\}{}$\2\6
${}\\{infile}\K\\{fopen}(\\{argv}[\T{1}],\39\.{"r"});{}$\6
\&{if} ${}(\R\\{infile}){}$\5
${}\{{}$\1\6
${}\\{fprintf}(\\{stderr},\39\.{"I\ couldn't\ open\ `\%s}\)\.{'\ for\ reading!%
\\n"},\39\\{argv}[\T{1}]);{}$\6
${}\\{exit}({-}\T{2});{}$\6
\4${}\}{}$\2\6
\&{if} ${}(\\{sscanf}(\\{argv}[\T{2}],\39\.{"\%d"},\39{\AND}\\{nn})\I\T{1}\V%
\\{nn}<\T{3}\V\\{nn}\G\\{nmax}){}$\5
${}\{{}$\1\6
${}\\{fprintf}(\\{stderr},\39\.{"The\ number\ `\%s'\ was}\)\.{\ supposed\ to\
be\ betw}\)\.{een\ 3\ and\ \%d!\\n"},\39\\{argv}[\T{2}],\39\\{nmax}-\T{1});{}$\6
${}\\{exit}({-}\T{3});{}$\6
\4${}\}{}$\2\par
\U1.\fi

\M{3}\B\X3:Input the next value, \PB{\|l[\|n]}\X${}\E{}$\6
$\\{lb}\K\\{fgetc}(\\{infile})-\.{'\ '}{}$;\C{ \PB{\\{fgetc}} will return a
negative value after EOF }\6
\&{if} ${}(\\{lb}<\T{0}\V(\|n>\T{1}\W\\{lb}>\|l[\|n-\T{1}]+\T{1})){}$\5
${}\{{}$\1\6
${}\\{fprintf}(\\{stderr},\39\.{"Input\ file\ has\ the\ }\)\.{wrong\ value\ (%
\%d)\ for}\)\.{\ l[\%d]!\\n"},\39\\{lb},\39\|n);{}$\6
${}\\{exit}({-}\T{4});{}$\6
\4${}\}{}$\2\6
${}\|l[\|n]\K\\{lb}{}$;\par
\U1.\fi

\N{1}{4}The interesting part.

\Y\B\4\X4:Backtrack through all solutions\X${}\E{}$\6
$\|a[\T{0}]\K\|b[\T{0}]\K\T{1},\39\|a[\T{1}]\K\|b[\T{1}]\K\T{2};{}$\6
${}\|n\K\\{nn},\39\\{lb}\K\|l[\|n];{}$\6
\&{for} ${}(\|i\K\T{0};{}$ ${}\|i\Z\\{lb};{}$ ${}\|i\PP){}$\1\5
${}\\{outdeg}[\|i]\K\\{outsum}[\|i]\K\T{0};{}$\2\6
${}\|a[\\{lb}]\K\|b[\\{lb}]\K\|n;{}$\6
\&{for} ${}(\|i\K\T{2};{}$ ${}\|i<\\{lb};{}$ ${}\|i\PP){}$\1\5
${}\|a[\|i]\K\|a[\|i-\T{1}]+\T{1},\39\|b[\|i]\K\|b[\|i-\T{1}]\LL\T{1};{}$\2\6
\&{for} ${}(\|i\K\\{lb}-\T{1};{}$ ${}\|i\G\T{2};{}$ ${}\|i\MM){}$\5
${}\{{}$\1\6
\&{if} ${}((\|a[\|i]\LL\T{1})<\|a[\|i+\T{1}]){}$\1\5
${}\|a[\|i]\K(\|a[\|i+\T{1}]+\T{1})\GG\T{1};{}$\2\6
\&{if} ${}(\|b[\|i]\G\|b[\|i+\T{1}]){}$\1\5
${}\|b[\|i]\K\|b[\|i+\T{1}]-\T{1};{}$\2\6
\4${}\}{}$\2\6
\X9:Try to fix the rest of the chain, and output all the solutions\X;\par
\U1.\fi

\M{5}One of the key operations we need is to increase \PB{\|p} to the smallest
element $p'>p$ that has $l[p']<s$, given that $l[p]<s$. Since
$l[p+1]\le l[p]+1$, we can do this quickly by first setting $p\gets p+1$;
then, if $l[p]=s$, we set $p\gets\PB{\\{down}}[p]$, where \PB{\\{down}[\|p]} is
the
smallest $p'>p$ that has $l[p']<l[p]$.

The links \PB{\\{down}[\|p]} can be prepared as we go, starting them off at $%
\infty$
and updating them whenever we learn a new value of \PB{\|l[\|n]}.

Instead of using infinite links, however, we can save space by
temporarily letting $\PB{\\{down}}[p]=p''$ in such cases, where $p''$ is the
largest element {\it less than\/} $p$ whose \PB{\\{down}} link is effectively
infinite. These temporary links tell us exactly what we need to know during
the updating process. And we can distinguish them from ``real'' \PB{\\{down}}
links by pretending that $\PB{\\{down}}[p]=\infty$ whenever \PB{$\\{down}[\|p]%
\Z\|p$}.

\Y\B\4\X5:Given that \PB{$\|l[\|p]<\|s$}, increase \PB{\|p} to the next such
element\X${}\E{}$\6
${}\{{}$\1\6
${}\|p\PP;{}$\6
\&{if} ${}(\|l[\|p]\E\|s){}$\1\5
${}\|p\K(\\{down}[\|p]>\|p\?\\{down}[\|p]:\\{nmax});{}$\2\6
\4${}\}{}$\2\par
\U11.\fi

\M{6}\B\X6:Given that \PB{$\|l[\|p]\G\|s$}, increase \PB{\|p} to the next
element with \PB{$\|l[\|p]<\|s$}\X${}\E{}$\6
\&{do}\5
${}\{{}$\1\6
\&{if} ${}(\\{down}[\|p]>\|p){}$\1\5
${}\|p\K\\{down}[\|p];{}$\2\6
\&{else}\5
${}\{{}$\1\6
${}\|p\K\\{nmax}{}$;\5
\&{break};\6
\4${}\}{}$\2\6
\4${}\}{}$\5
\2\5
\&{while} ${}(\|l[\|p]\G\|s){}$;\par
\Us10\ET11.\fi

\M{7}\B\X7:Initialize the \PB{\\{down}} table\X${}\E{}$\6
\&{for} ${}(\|n\K\T{1};{}$ ${}\|n\Z\\{nn};{}$ ${}\|n\PP){}$\1\5
${}\\{down}[\|n]\K\|n-\T{1}{}$;\2\par
\U1.\fi

\M{8}I can't help exclaiming that this little algorithm is quite pretty.

\Y\B\4\X8:Update the \PB{\\{down}} links\X${}\E{}$\6
\&{if} ${}(\|l[\|n]<\|l[\|n-\T{1}]){}$\5
${}\{{}$\1\6
\&{for} ${}(\|p\K\\{down}[\|n];{}$ ${}\|l[\|p]>\|l[\|n];{}$ ${}\|p\K\|q){}$\1\5
${}\|q\K\\{down}[\|p],\39\\{down}[\|p]\K\|n;{}$\2\6
${}\\{down}[\|n]\K\|p;{}$\6
\4${}\}{}$\2\par
\U1.\fi

\M{9}\B\X9:Try to fix the rest of the chain, and output all the solutions\X${}%
\E{}$\6
$\\{ptr}\K\T{0}{}$;\C{ clear the \PB{\\{undo}} stack }\6
\&{for} ${}(\|r\K\|s\K\\{lb};{}$ ${}\|s>\T{2};{}$ ${}\|s\MM){}$\5
${}\{{}$\1\6
\&{if} ${}(\\{outdeg}[\|s]\E\T{1}){}$\1\5
${}\\{limit}[\|s]\K\|a[\|s]-\\{tail}[\\{outsum}[\|s]]{}$;\5
\2\&{else}\1\5
${}\\{limit}[\|s]\K\|a[\|s]-\T{1}{}$;\C{ the max feasible \PB{\|p} }\2\6
\&{if} ${}(\\{limit}[\|s]>\|b[\|s-\T{1}]){}$\1\5
${}\\{limit}[\|s]\K\|b[\|s-\T{1}];{}$\2\6
\X10:Set \PB{\|p} to its smallest feasible value, and \PB{$\|q\K\|a[\|s]-\|p$}%
\X;\6
\&{while} ${}(\|p\Z\\{limit}[\|s]){}$\5
${}\{{}$\1\6
\X14:Find bounds $(\PB{\\{lbp}},\PB{\\{ubp}})$ and $(\PB{\\{lbq}},\PB{%
\\{ubq}})$ on where \PB{\|p} and \PB{\|q} can be inserted; but go to \PB{%
\\{failpq}} if they can't both be accommodated\X;\6
${}\\{ptrp}\K\\{ptr};{}$\6
\&{for} ( ; ${}\\{ubp}\G\\{lbp};{}$ ${}\\{ubp}\MM){}$\5
${}\{{}$\1\6
\X16:Put \PB{\|p} into the chain at location \PB{\\{ubp}}; \PB{\&{goto} %
\\{failp}} if there's a problem\X;\6
\&{if} ${}(\|p\E\|q){}$\1\5
\&{goto} \\{happiness};\2\6
\&{if} ${}(\\{ubq}\G\\{ubp}){}$\1\5
${}\\{ubq}\K\\{ubp}-\T{1};{}$\2\6
${}\\{ptrq}\K\\{ptr};{}$\6
\&{for} ( ; ${}\\{ubq}\G\\{lbq};{}$ ${}\\{ubq}\MM){}$\5
${}\{{}$\1\6
\X18:Put \PB{\|q} into the chain at location \PB{\\{ubq}}; \PB{\&{goto} %
\\{failq}} if there's a problem\X;\6
\4\\{happiness}:\5
\X12:Put local variables on the stack and update outdegrees\X;\6
\&{goto} \\{onward};\C{ now \PB{\|a[\|s]} is covered; try to cover \PB{$\|a[%
\|s-\T{1}]$} }\6
\4\\{backup}:\5
${}\|s\PP;{}$\6
\&{if} ${}(\|s>\\{lb}){}$\1\5
\&{goto} \\{impossible};\2\6
\X13:Restore local variables from the stack and downdate outdegrees\X;\6
\&{if} ${}(\|p\E\|q){}$\1\5
\&{goto} \\{failp};\2\6
\4\\{failq}:\5
\&{while} ${}(\\{ptr}>\\{ptrq}){}$\1\5
\X15:Undo a change\X;\2\6
\4${}\}{}$\C{ end loop on \PB{\\{ubq}} }\2\6
\4\\{failp}:\5
\&{while} ${}(\\{ptr}>\\{ptrp}){}$\1\5
\X15:Undo a change\X;\2\6
\4${}\}{}$\C{ end loop on \PB{\\{ubp}} }\2\6
\4\\{failpq}:\5
\X11:Advance \PB{\|p} to the next smallest feasible value, and set \PB{$\|q\K%
\|a[\|s]-\|p$}\X;\6
\4${}\}{}$\C{ end loop on \PB{\|p} }\2\6
\&{goto} \\{backup};\6
\4\\{onward}:\5
\&{continue};\6
\4${}\}{}$\C{ end loop on \PB{\|s} }\2\6
\X20:Print a solution\X;\6
\&{goto} \\{backup};\6
\4\\{impossible}:\par
\U4.\fi

\M{10}At this point we have \PB{$\|a[\|k]\K\|b[\|k]$} for all $r\le k\le\PB{%
\\{lb}}$.

\Y\B\4\X10:Set \PB{\|p} to its smallest feasible value, and \PB{$\|q\K\|a[\|s]-%
\|p$}\X${}\E{}$\6
\&{if} ${}(\|a[\|s]\AND\T{1}){}$\5
${}\{{}$\C{ necessarily \PB{$\|p\I\|q$} }\1\6
\4\\{unequal}:\5
\&{if} ${}(\\{outdeg}[\|s-\T{1}]\E\T{0}){}$\1\5
${}\|q\K\|a[\|s]/\T{3}{}$;\5
\2\&{else}\1\5
${}\|q\K\|a[\|s]\GG\T{1};{}$\2\6
\&{if} ${}(\|q>\|b[\|s-\T{2}]){}$\1\5
${}\|q\K\|b[\|s-\T{2}];{}$\2\6
${}\|p\K\|a[\|s]-\|q;{}$\6
\&{if} ${}(\|l[\|p]\G\|s){}$\5
${}\{{}$\1\6
\X6:Given that \PB{$\|l[\|p]\G\|s$}, increase \PB{\|p} to the next element with
\PB{$\|l[\|p]<\|s$}\X;\6
${}\|q\K\|a[\|s]-\|p;{}$\6
\4${}\}{}$\2\6
\4${}\}{}$\5
\2\&{else}\5
${}\{{}$\1\6
${}\|p\K\|q\K\|a[\|s]\GG\T{1};{}$\6
\&{if} ${}(\|l[\|p]\G\|s){}$\1\5
\&{goto} \\{unequal};\C{ a rare case like \PB{$\|l[\T{191}]\K\|l[\T{382}]$} }\2%
\6
\4${}\}{}$\2\6
\&{if} ${}(\|p>\\{limit}[\|s]){}$\1\5
\&{goto} \\{backup};\2\6
\&{for} ( ; ${}\|r>\T{2}\W\|a[\|r-\T{1}]\E\|b[\|r-\T{1}];{}$ ${}\|r\MM){}$\1\5
;\2\6
\&{if} ${}(\|p>\|b[\|r-\T{1}]){}$\5
${}\{{}$\C{ now \PB{$\|r<\|s$}, since \PB{$\|p\Z\|b[\|s-\T{1}]$} }\1\6
\&{while} ${}(\|p>\|a[\|r]){}$\1\5
${}\|r\PP{}$;\C{ this step keeps \PB{$\|r<\|s$}, since \PB{$\|a[\|s-\T{1}]\K%
\|b[\|s-\T{1}]$} }\2\6
${}\|p\K\|a[\|r],\39\|q\K\|a[\|s]-\|p;{}$\6
\4${}\}{}$\5
\2\&{else} \&{if} ${}(\|q<\|p\W\|q>\|b[\|r-\T{2}]){}$\5
${}\{{}$\1\6
\&{if} ${}(\|a[\|r]\Z\|a[\|s]-\|b[\|r-\T{2}]){}$\1\5
${}\|p\K\|a[\|r],\39\|q\K\|b[\|s]-\|p;{}$\2\6
\&{else}\1\5
${}\|q\K\|b[\|r-\T{2}],\39\|p\K\|a[\|s]-\|q;{}$\2\6
\4${}\}{}$\2\par
\U9.\fi

\M{11}\B\X11:Advance \PB{\|p} to the next smallest feasible value, and set %
\PB{$\|q\K\|a[\|s]-\|p$}\X${}\E{}$\6
\&{if} ${}(\|p\E\|q){}$\5
${}\{{}$\1\6
\&{if} ${}(\\{outdeg}[\|s-\T{1}]\E\T{0}){}$\1\5
${}\|q\K(\|a[\|s]/\T{3})+\T{1}{}$;\C{ will be decreased momentarily }\2\6
\&{if} ${}(\|q>\|b[\|s-\T{2}]){}$\1\5
${}\|q\K\|b[\|s-\T{2}]{}$;\5
\2\&{else}\1\5
${}\|q\MM;{}$\2\6
${}\|p\K\|a[\|s]-\|q;{}$\6
\&{if} ${}(\|l[\|p]\G\|s){}$\5
${}\{{}$\1\6
\X6:Given that \PB{$\|l[\|p]\G\|s$}, increase \PB{\|p} to the next element with
\PB{$\|l[\|p]<\|s$}\X;\6
${}\|q\K\|a[\|s]-\|p;{}$\6
\4${}\}{}$\2\6
\4${}\}{}$\5
\2\&{else}\5
${}\{{}$\1\6
\X5:Given that \PB{$\|l[\|p]<\|s$}, increase \PB{\|p} to the next such element%
\X;\6
${}\|q\K\|a[\|s]-\|p;{}$\6
\4${}\}{}$\2\6
\&{if} ${}(\|q>\T{2}){}$\5
${}\{{}$\1\6
\&{if} ${}(\|a[\|s-\T{1}]\E\|b[\|s-\T{1}]){}$\5
${}\{{}$\C{ maybe \PB{\|p} has to be present already }\1\6
\4\\{doublecheck}:\5
\&{while} ${}(\|p<\|a[\|r]\W\|a[\|r-\T{1}]\E\|b[\|r-\T{1}]){}$\1\5
${}\|r\MM;{}$\2\6
\&{if} ${}(\|p>\|b[\|r-\T{1}]){}$\5
${}\{{}$\1\6
\&{while} ${}(\|p>\|a[\|r]){}$\1\5
${}\|r\PP;{}$\2\6
${}\|p\K\|a[\|r],\39\|q\K\|a[\|s]-\|p{}$;\C{ possibly \PB{$\|r\K\|s$} now }\6
\4${}\}{}$\5
\2\&{else} \&{if} ${}(\|q>\|b[\|r-\T{2}]){}$\5
${}\{{}$\1\6
\&{if} ${}(\|a[\|r]\Z\|a[\|s]-\|b[\|r-\T{2}]){}$\1\5
${}\|p\K\|a[\|r],\39\|q\K\|b[\|s]-\|p;{}$\2\6
\&{else}\1\5
${}\|q\K\|b[\|r-\T{2}],\39\|p\K\|a[\|s]-\|q;{}$\2\6
\4${}\}{}$\2\6
\4${}\}{}$\2\6
\&{if} ${}(\\{ubq}\G\|s){}$\1\5
${}\\{ubq}\K\|s-\T{1};{}$\2\6
\&{while} ${}(\|q\G\|a[\\{ubq}+\T{1}]){}$\1\5
${}\\{ubq}\PP;{}$\2\6
\&{while} ${}(\|q<\|a[\\{ubq}]){}$\1\5
${}\\{ubq}\MM;{}$\2\6
\&{if} ${}(\|q>\|b[\\{ubq}]){}$\5
${}\{{}$\1\6
${}\|q\K\|b[\\{ubq}],\39\|p\K\|a[\|s]-\|q;{}$\6
\&{if} ${}(\|a[\|s-\T{1}]\E\|b[\|s-\T{1}]){}$\1\5
\&{goto} \\{doublecheck};\2\6
\4${}\}{}$\2\6
\4${}\}{}$\2\par
\U9.\fi

\M{12}\B\X12:Put local variables on the stack and update outdegrees\X${}\E{}$\6
$\\{tail}[\|s]\K\|q,\39\\{stack}[\|s].\|r\K\|r;{}$\6
${}\\{outdeg}[\\{ubp}]\PP,\39\\{outsum}[\\{ubp}]\MRL{+{\K}}\|s;{}$\6
${}\\{outdeg}[\\{ubq}]\PP,\39\\{outsum}[\\{ubq}]\MRL{+{\K}}\|s;{}$\6
${}\\{stack}[\|s].\\{lbp}\K\\{lbp},\39\\{stack}[\|s].\\{ubp}\K\\{ubp};{}$\6
${}\\{stack}[\|s].\\{lbq}\K\\{lbq},\39\\{stack}[\|s].\\{ubq}\K\\{ubq};{}$\6
${}\\{stack}[\|s].\\{ptrp}\K\\{ptrp},\39\\{stack}[\|s].\\{ptrq}\K\\{ptrq}{}$;%
\par
\U9.\fi

\M{13}\B\X13:Restore local variables from the stack and downdate outdegrees%
\X${}\E{}$\6
$\\{ptrq}\K\\{stack}[\|s].\\{ptrq},\39\\{ptrp}\K\\{stack}[\|s].\\{ptrp};{}$\6
${}\\{lbq}\K\\{stack}[\|s].\\{lbq},\39\\{ubq}\K\\{stack}[\|s].\\{ubq};{}$\6
${}\\{lbp}\K\\{stack}[\|s].\\{lbp},\39\\{ubp}\K\\{stack}[\|s].\\{ubp};{}$\6
${}\\{outdeg}[\\{ubq}]\MM,\39\\{outsum}[\\{ubq}]\MRL{-{\K}}\|s;{}$\6
${}\\{outdeg}[\\{ubp}]\MM,\39\\{outsum}[\\{ubp}]\MRL{-{\K}}\|s;{}$\6
${}\|q\K\\{tail}[\|s],\39\|p\K\|a[\|s]-\|q,\39\|r\K\\{stack}[\|s].\|r{}$;\par
\U9.\fi

\M{14}After the test in this step is passed, we'll have \PB{$\\{ubp}>\\{ubq}$}
and \PB{$\\{lbp}>\\{lbq}$}.

\Y\B\4\X14:Find bounds $(\PB{\\{lbp}},\PB{\\{ubp}})$ and $(\PB{\\{lbq}},\PB{%
\\{ubq}})$ on where \PB{\|p} and \PB{\|q} can be inserted; but go to \PB{%
\\{failpq}} if they can't both be accommodated\X${}\E{}$\6
\&{if} ${}(\|l[\|p]\G\|s){}$\1\5
\&{goto} \\{failpq};\2\6
${}\\{lbp}\K\|l[\|p];{}$\6
\&{while} ${}(\|b[\\{lbp}]<\|p){}$\1\5
${}\\{lbp}\PP;{}$\2\6
\&{if} ${}((\|p\AND\T{1})\W\|p>\|b[\\{lbp}-\T{2}]+\|b[\\{lbp}-\T{1}]){}$\5
${}\{{}$\1\6
\&{if} ${}(\PP\\{lbp}\G\|s){}$\1\5
\&{goto} \\{failpq};\2\6
\4${}\}{}$\2\6
\&{if} ${}(\|a[\\{lbp}]>\|p){}$\1\5
\&{goto} \\{failpq};\2\6
\&{for} ${}(\\{ubp}\K\\{lbp};{}$ ${}\|a[\\{ubp}+\T{1}]\Z\|p;{}$ ${}\\{ubp}%
\PP){}$\1\5
;\2\6
\&{if} ${}(\\{ubp}\E\|s-\T{1}){}$\1\5
${}\\{lbp}\K\\{ubp};{}$\2\6
\&{if} ${}(\|p\E\|q){}$\1\5
${}\\{lbq}\K\\{lbp},\39\\{ubq}\K\\{ubp};{}$\2\6
\&{else}\5
${}\{{}$\1\6
${}\\{lbq}\K\|l[\|q];{}$\6
\&{if} ${}(\\{lbq}\G\\{ubp}){}$\1\5
\&{goto} \\{failpq};\2\6
\&{while} ${}(\|b[\\{lbq}]<\|q){}$\1\5
${}\\{lbq}\PP;{}$\2\6
\&{if} ${}(\|a[\\{lbq}]<\|b[\\{lbq}]){}$\5
${}\{{}$\1\6
\&{if} ${}((\|q\AND\T{1})\W\|q>\|b[\\{lbq}-\T{2}]+\|b[\\{lbq}-\T{1}]){}$\1\5
${}\\{lbq}\PP;{}$\2\6
\&{if} ${}(\\{lbq}\G\\{ubp}){}$\1\5
\&{goto} \\{failpq};\2\6
\&{if} ${}(\|a[\\{lbq}]>\|q){}$\1\5
\&{goto} \\{failpq};\2\6
\&{if} ${}(\\{lbp}\Z\\{lbq}){}$\1\5
${}\\{lbp}\K\\{lbq}+\T{1};{}$\2\6
\&{while} ${}((\|q\LL(\\{lbp}-\\{lbq}))<\|p){}$\1\6
\&{if} ${}(\PP\\{lbp}>\\{ubp}){}$\1\5
\&{goto} \\{failpq};\2\2\6
\4${}\}{}$\2\6
\&{for} ${}(\\{ubq}\K\\{lbq};{}$ ${}\|a[\\{ubq}+\T{1}]\Z\|q\W(\|q\LL(\\{ubp}-%
\\{ubq}-\T{1}))\G\|p;{}$ ${}\\{ubq}\PP){}$\1\5
;\2\6
\4${}\}{}$\2\par
\U9.\fi

\M{15}The undoing mechanism is very simple: When changing \PB{\|a[\|j]}, we
put \PB{$(\|j\LL\T{24})+\|x$} on the \PB{\\{undo}} stack, where \PB{\|x} was
the former value.
Similarly, when changing \PB{\|b[\|j]}, we stack the value \PB{$(\T{1}\LL%
\T{31})+(\|j\LL\T{24})+\|x$}.

\Y\B\4\D$\\{newa}(\|j,\|y)$ \5
$\\{undo}[\\{ptr}\PP]\K(\|j\LL\T{24})+\|a[\|j],\39\|a[\|j]\K{}$\|y\par
\B\4\D$\\{newb}(\|j,\|y)$ \5
$\\{undo}[\\{ptr}\PP]\K(\T{1}\LL\T{31})+(\|j\LL\T{24})+\|b[\|j],\39\|b[\|j]%
\K{}$\|y\par
\Y\B\4\X15:Undo a change\X${}\E{}$\6
${}\{{}$\1\6
${}\|i\K\\{undo}[\MM\\{ptr}];{}$\6
\&{if} ${}(\|i\G\T{0}){}$\1\5
${}\|a[\|i\GG\T{24}]\K\|i\AND\T{\^ffffff};{}$\2\6
\&{else}\1\5
${}\|b[(\|i\AND\T{\^3fffffff})\GG\T{24}]\K\|i\AND\T{\^ffffff};{}$\2\6
\4${}\}{}$\2\par
\U9.\fi

\M{16}At this point we know that $a[ubp]\le p\le b[ubp]$.

\Y\B\4\X16:Put \PB{\|p} into the chain at location \PB{\\{ubp}}; \PB{\&{goto} %
\\{failp}} if there's a problem\X${}\E{}$\6
\&{if} ${}(\|a[\\{ubp}]\I\|p){}$\5
${}\{{}$\1\6
${}\\{newa}(\\{ubp},\39\|p);{}$\6
\&{for} ${}(\|j\K\\{ubp}-\T{1};{}$ ${}(\|a[\|j]\LL\T{1})<\|a[\|j+\T{1}];{}$ ${}%
\|j\MM){}$\5
${}\{{}$\1\6
${}\|i\K(\|a[\|j+\T{1}]+\T{1})\GG\T{1};{}$\6
\&{if} ${}(\|i>\|b[\|j]){}$\1\5
\&{goto} \\{failp};\2\6
${}\\{newa}(\|j,\39\|i);{}$\6
\4${}\}{}$\2\6
\&{for} ${}(\|j\K\\{ubp}+\T{1};{}$ ${}\|a[\|j]\Z\|a[\|j-\T{1}];{}$ ${}\|j%
\PP){}$\5
${}\{{}$\1\6
${}\|i\K\|a[\|j-\T{1}]+\T{1};{}$\6
\&{if} ${}(\|i>\|b[\|j]){}$\1\5
\&{goto} \\{failp};\2\6
${}\\{newa}(\|j,\39\|i);{}$\6
\4${}\}{}$\2\6
\4${}\}{}$\2\6
\&{if} ${}(\|b[\\{ubp}]\I\|p){}$\5
${}\{{}$\1\6
${}\\{newb}(\\{ubp},\39\|p);{}$\6
\&{for} ${}(\|j\K\\{ubp}-\T{1};{}$ ${}\|b[\|j]\G\|b[\|j+\T{1}];{}$ ${}\|j%
\MM){}$\5
${}\{{}$\1\6
${}\|i\K\|b[\|j+\T{1}]-\T{1};{}$\6
\&{if} ${}(\|i<\|a[\|j]){}$\1\5
\&{goto} \\{failp};\2\6
${}\\{newb}(\|j,\39\|i);{}$\6
\4${}\}{}$\2\6
\&{for} ${}(\|j\K\\{ubp}+\T{1};{}$ ${}\|b[\|j]>\|b[\|j-\T{1}]\LL\T{1};{}$ ${}%
\|j\PP){}$\5
${}\{{}$\1\6
${}\|i\K\|b[\|j-\T{1}]\LL\T{1};{}$\6
\&{if} ${}(\|i<\|a[\|j]){}$\1\5
\&{goto} \\{failp};\2\6
${}\\{newb}(\|j,\39\|i);{}$\6
\4${}\}{}$\2\6
\4${}\}{}$\2\6
\X17:Make forced moves if \PB{\|p} has a special form\X;\par
\U9.\fi

\M{17}If, say, we've just set \PB{$\|a[\T{8}]\K\|b[\T{8}]\K\T{132}$}, special
considerations apply,
because the only addition chains of length~8 for 132 are
$$\eqalign{
&1,2,4,8,16,32,64,128,132;\cr
&1,2,4,8,16,32,64,68,132;\cr
&1,2,4,8,16,32,64,66,132;\cr
&1,2,4,8,16,32,34,66,132;\cr
&1,2,4,8,16,32,33,66,132;\cr
&1,2,4,8,16,17,33,66,132.\cr}$$
The values of \PB{\|a[\T{4}]} and \PB{\|b[\T{4}]} must therefore be 16; and
then, of course,
we also must have \PB{$\|a[\T{3}]\K\|b[\T{3}]\K\T{8}$}, etc. Similar reasoning
applies
whenever we set $a[j]=b[j]=2^j+2^k$ for $k\le j-4$.

Such cases may seem extremely special. But my hunch is that they are
important, because efficient chains need such values. When we try
to prove that no efficient chain exists, we want to show that
such values can't be present. Numbers with small \PB{\|l[\|p]} are harder
to rule out, so it should be helpful to penalize them.

\Y\B\4\X17:Make forced moves if \PB{\|p} has a special form\X${}\E{}$\6
$\|i\K\|p-(\T{1}\LL(\\{ubp}-\T{1}));{}$\6
\&{if} ${}(\|i\W((\|i\AND(\|i-\T{1}))\E\T{0})\W(\|i\LL\T{4})<\|p){}$\5
${}\{{}$\1\6
\&{for} ${}(\|j\K\\{ubp}-\T{2};{}$ ${}(\|i\AND\T{1})\E\T{0};{}$ ${}\|i\MRL{{%
\GG}{\K}}\T{1},\39\|j\MM){}$\1\5
;\2\6
\&{if} ${}(\|b[\|j]<(\T{1}\LL\|j)){}$\1\5
\&{goto} \\{failp};\2\6
\&{for} ( ; ${}\|a[\|j]<(\T{1}\LL\|j);{}$ ${}\|j\MM){}$\1\5
${}\\{newa}(\|j,\39\T{1}\LL\|j);{}$\2\6
\4${}\}{}$\2\par
\U16.\fi

\M{18}At this point we had better not assume that $a[ubq]\le q\le b[ubq]$,
because \PB{\|p} has just been inserted. That insertion can mess up the
bounds that we looked at when \PB{\\{lbq}} and \PB{\\{ubq}} were computed.

\Y\B\4\X18:Put \PB{\|q} into the chain at location \PB{\\{ubq}}; \PB{\&{goto} %
\\{failq}} if there's a problem\X${}\E{}$\6
\&{if} ${}(\|a[\\{ubq}]\I\|q){}$\5
${}\{{}$\1\6
\&{if} ${}(\|a[\\{ubq}]>\|q){}$\1\5
\&{goto} \\{failq};\2\6
${}\\{newa}(\\{ubq},\39\|q);{}$\6
\&{for} ${}(\|j\K\\{ubq}-\T{1};{}$ ${}(\|a[\|j]\LL\T{1})<\|a[\|j+\T{1}];{}$ ${}%
\|j\MM){}$\5
${}\{{}$\1\6
${}\|i\K(\|a[\|j+\T{1}]+\T{1})\GG\T{1};{}$\6
\&{if} ${}(\|i>\|b[\|j]){}$\1\5
\&{goto} \\{failq};\2\6
${}\\{newa}(\|j,\39\|i);{}$\6
\4${}\}{}$\2\6
\&{for} ${}(\|j\K\\{ubq}+\T{1};{}$ ${}\|a[\|j]\Z\|a[\|j-\T{1}];{}$ ${}\|j%
\PP){}$\5
${}\{{}$\1\6
${}\|i\K\|a[\|j-\T{1}]+\T{1};{}$\6
\&{if} ${}(\|i>\|b[\|j]){}$\1\5
\&{goto} \\{failq};\2\6
${}\\{newa}(\|j,\39\|i);{}$\6
\4${}\}{}$\2\6
\4${}\}{}$\2\6
\&{if} ${}(\|b[\\{ubq}]\I\|q){}$\5
${}\{{}$\1\6
\&{if} ${}(\|b[\\{ubq}]<\|q){}$\1\5
\&{goto} \\{failq};\2\6
${}\\{newb}(\\{ubq},\39\|q);{}$\6
\&{for} ${}(\|j\K\\{ubq}-\T{1};{}$ ${}\|b[\|j]\G\|b[\|j+\T{1}];{}$ ${}\|j%
\MM){}$\5
${}\{{}$\1\6
${}\|i\K\|b[\|j+\T{1}]-\T{1};{}$\6
\&{if} ${}(\|i<\|a[\|j]){}$\1\5
\&{goto} \\{failq};\2\6
${}\\{newb}(\|j,\39\|i);{}$\6
\4${}\}{}$\2\6
\&{for} ${}(\|j\K\\{ubq}+\T{1};{}$ ${}\|b[\|j]>\|b[\|j-\T{1}]\LL\T{1};{}$ ${}%
\|j\PP){}$\5
${}\{{}$\1\6
${}\|i\K\|b[\|j-\T{1}]\LL\T{1};{}$\6
\&{if} ${}(\|i<\|a[\|j]){}$\1\5
\&{goto} \\{failq};\2\6
${}\\{newb}(\|j,\39\|i);{}$\6
\4${}\}{}$\2\6
\4${}\}{}$\2\6
\X19:Make forced moves if \PB{\|q} has a special form\X;\par
\U9.\fi

\M{19}\B\X19:Make forced moves if \PB{\|q} has a special form\X${}\E{}$\6
$\|i\K\|q-(\T{1}\LL(\\{ubq}-\T{1}));{}$\6
\&{if} ${}(\|i\W((\|i\AND(\|i-\T{1}))\E\T{0})\W(\|i\LL\T{4})<\|q){}$\5
${}\{{}$\1\6
\&{for} ${}(\|j\K\\{ubq}-\T{2};{}$ ${}(\|i\AND\T{1})\E\T{0};{}$ ${}\|i\MRL{{%
\GG}{\K}}\T{1},\39\|j\MM){}$\1\5
;\2\6
\&{if} ${}(\|b[\|j]<(\T{1}\LL\|j)){}$\1\5
\&{goto} \\{failq};\2\6
\&{for} ( ; ${}\|a[\|j]<(\T{1}\LL\|j);{}$ ${}\|j\MM){}$\1\5
${}\\{newa}(\|j,\39\T{1}\LL\|j);{}$\2\6
\4${}\}{}$\2\par
\U18.\fi

\M{20}\B\X20:Print a solution\X${}\E{}$\6
\&{for} ${}(\|j\K\T{0};{}$ ${}\|j\Z\\{lb};{}$ ${}\|j\PP){}$\1\5
${}\\{printf}(\.{"\ \%d"},\39\|a[\|j]);{}$\2\6
\\{printf}(\.{"\\n"});\par
\U9.\fi

\N{1}{21}Index.


\fi


\inx
\fin
\con
