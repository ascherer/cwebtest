\input cwebmac
\hypertextrue\srcloctrue
\datethis
\input epsf

\def\adj{\mathrel{\!\mathrel-\mkern-8mu\mathrel-\mkern-8mu\mathrel-\!}}
   % adjacent vertices
\def\dadj{\mathrel{\!\mathrel-\mkern-8mu\mathrel-\mkern-12mu\to\!}}

\everypar{\looseness=-1}


\N[11 skew-ternary-calc.w]{1}{1}Introduction. This program does calculations
with skew ternary trees,
and exhibits the corresponding nonseparable planar graphs.
It implements some simple algorithms that I discovered in November, 2013,
based on ideas of Alberto Del Lungo, Francesco Del Ristoro, and
Jean-Guy Penaud [{\sl Theoretical Computer Science\/ \bf233} (2000),
201--215]. I wrote it in order to learn more about the seemingly magical
properties of this amazing correspondence.

I apologize for having no time to provide a better user interface, or to
give more extensive commentary. Ideally an interactive system should
be written, with excellent graphics to display and manipulate
the trees and graphs in an intuitive fashion;
color should be used to exhibit at least some of the
fascinating patterns that are present, etc. I'm hoping that some
reader will be motivated to write such an ``app,'' because it will
certainly be a fabulously instructive toy.

I will at least try to define and explain the basics in this document.
A ternary tree is either
empty, or it consists of a root node and three ternary trees
called the left, middle, and right subtrees of the root. The
roots of those subtrees are called the left, middle, and right children
of the root node. (This definition is strictly analogous to the
familiar definition of binary trees, in Section 2.3 of my book
{\sl Fundamental Algorithms}.)

Furthermore we extend the ternary tree by
placing ``buds'' in the positions of empty subtrees. And we also
introduce a bud at the top, attached to the root node. In this way every node,
including the root, is attached to exactly four other nodes or buds;
and every bud is attached to exactly one other node or bud.
We will give labels to each node and to each bud, in order to exhibit
fine details of the structure.

An extended ternary tree with $n$ nodes always has $2n+2$ buds.
(Notice that this result, which is easily proved by induction on~$n$,
holds in particular when the ternary tree is empty.
In that case, $n=0$ and there simply are two buds joined to each other.)

The embedding of such a tree in the plane leads to
a family of $2n+2$ extended ternary trees that are
``cyclically equivalent,'' as illustrated below.
There's one such tree for each bud,
obtained by placing that bud at the top and letting everything else
``hang down'' from it in.
Each of these trees has a different
root bud, but not necessarily a different root, because
different buds can lead to distinct trees with the same root node.

The $2n+2$ buds can always be paired up into $n+1$ groups of two. Indeed,
we can find the mate of any bud by proceeding on a unique path away
from that bud, always taking the middle branch whenever there are
three choices for the next step, and continuing
until another bud is encountered.

Every node and every bud is assigned a {\it rank\/} in the following
natural way: The root node and root bud have rank zero; and the
left, middle, and right children of a rank~$r$ node
have ranks $r-1$, $r$, and $r+1$, respectively.

A {\it skew ternary tree\/} is a ternary tree for which all nodes
have nonnegative rank.

For example, the skew ternary tree shown here has 6 nodes and 7 pairs of
buds. Ranks are shown in red.
Notice that there's one bud of rank $-1$ for every node of rank~0.
$$\vcenter{\epsfbox{skew-ternary-calc.1}}\qquad\qquad
\vcenter{\epsfbox{skew-ternary-calc.3}}$$

\fi

\M[80 skew-ternary-calc.w]{2}Fact: {\sl Every family of $2n+2$ cyclically
equivalent ternary trees
includes exactly four skew ternary trees.}

Moreover, this theorem---which
is the main reason for the existence of this program---has an
astonishingly simple proof.
The idea is to consider the $n-1$ edges that go between nodes of
the ternary, and to treat each edge $\.U \adj \.V$ as a pair
of arcs $\.U \dadj \.V$ and $\.V \dadj \.U$. That gives us
$2n-2$ arcs. And there's a natural way to match those arcs to $2n-2$
of the $2n+2$ buds,
by means of $2n-2$ noncrossing filaments as illustrated here:
$$\vcenter{\epsfbox{skew-ternary-calc.2}}$$
More precisely, imagine an ant, named Alice, who crawls around the periphery of
the tree. Alice starts in state $-2$, just to the right of bud number~0.
She increases her state by~1 whenever she passes a bud; and
she decreases her state by~1 whenever she passes an arc. Then she will
be in state $-2+(2n+2)-(2n-2)=+2$ when she returns to its starting point:
$$\vcenter{\epsfbox{skew-ternary-calc.4}}$$
And if she keeps on crawling, she will repeat the same pattern, but
with her state increased by~4.

The key fact is that Alice is in state $k$ whenever she reaches a bud
of rank~$k$---except for the starting bud, when she's in state $\pm2$.
Thus the unmatched buds correspond to the skew ternary trees; in this
example the trees whose roots hang down from buds 0, 4, 5, and 6
will have no nodes of negative rank. Conversely, a ternary
tree that begins at a matched bud will have at least one buds of rank $<-1$,
so it will have at least one node of rank $<0$.

{\mc QED}.

\smallskip
(A reader who understands this proof will also be able to show
that {\sl every family of $4n+2$ cyclically equivalent {\it quinary trees\/}
includes exactly six skew quinary trees.} And so on.)

\fi

\M[117 skew-ternary-calc.w]{3}The four skew ternary trees of a cyclic family
turn out to have
remarkable properties. Let's look at the state transitions that Alice
would encounter by starting at each of the four unmatched buds:
$$\vcenter{\halign{#\hfil\cr
\epsfxsize=.5\hsize \epsfbox{skew-ternary-calc.5}\cr\noalign{\smallskip}
\epsfxsize=.5\hsize \epsfbox{skew-ternary-calc.6}\cr\noalign{\smallskip}
\epsfxsize=.5\hsize \epsfbox{skew-ternary-calc.7}\cr\noalign{\smallskip}
\epsfxsize=.5\hsize \epsfbox{skew-ternary-calc.8}\cr}}$$
The corresponding skew ternary trees,
which we might as well show without their buds, are
$$T=\vcenter{\epsfbox{skew-ternary-calc.10}}\;;\qquad
T^+=\vcenter{\epsfbox{skew-ternary-calc.11}}\;;\qquad
T^{++}=\vcenter{\epsfbox{skew-ternary-calc.12}}\;;\qquad
T^{+++}=\vcenter{\epsfbox{skew-ternary-calc.13}}\;.$$
Notice the notation used here, based on a well-defined operator
$T\mapsto T^+$ that takes one skew ternary tree to another.
Since $T^{++++}=T$, we also abbreviate $T^{+++}$ as $T^-$;
and $T^{++}$ can also be called $T^{--}$.

\fi

\M[136 skew-ternary-calc.w]{4}One of the first goals of this program will be to
compute the
``conjugates'' $T^+$, $T^{++}$, and $T^{+++}=T^-$, given a
skew ternary tree~$T$. That tree is specified on the command line,
as a sequence of four-character arguments \.{abcd}: The first character,
\.a, names a node; the next three characters name that node's children,
using `\.-' for an empty child. For example, the tree $T$ above could
be specified by the six command-line arguments
$$\.{A-BD} \qquad
\.{B--C} \qquad
\.{C---} \qquad
\.{DE-F} \qquad
\.{E---} \qquad
\.{F---}$$
in some order. There should be one argument
for each node. The program parses the arguments and
checks to make sure that they actually do define a skew ternary tree.

\fi

\M[153 skew-ternary-calc.w]{5}OK, we now know the definition of skew ternary
trees, and it's time
to begin coding. Here's the structure of the program as a whole:

\Y\B\8\#\&{include} \.{<stdio.h>}\6
\8\#\&{include} \.{<stdlib.h>}\6
\8\#\&{include} \.{<string.h>}\6
\X7:Type definitions\X\6
\X8:Global variables\X\6
\X66:Assertion failure subroutine\X\6
\X11:Subroutines\X\7
\\{main}(\&{int} \\{argc}${},\39{}$\&{char} ${}{*}\\{argv}[\,]){}$\1\1\2\2\6
${}\{{}$\1\6
\&{register} \&{int} \|c${},{}$ \|i${},{}$ \|j${},{}$ \|k${},{}$ \|p;\7
\X6:Process the command line\X;\6
\X16:Find and print the three conjugates of \PB{\|T}\X;\6
\X49:Find and print the corresponding planar maps\X;\6
\4${}\}{}$\2\par
\fi

\M[171 skew-ternary-calc.w]{6}We don't deal with the empty tree; there must be
at least one node.

\Y\B\4\X6:Process the command line\X${}\E{}$\6
\&{if} ${}(\\{argc}\E\T{1}){}$\5
${}\{{}$\1\6
${}\\{fprintf}(\\{stderr},\39\.{"Usage:\ \%s\ node\_1\ no}\)\.{de\_2\ ...\ node%
\_n\\n"},\39\\{argv}[\T{0}]);{}$\6
${}\\{exit}({-}\T{1});{}$\6
\4${}\}{}$\2\6
\X9:Parse the arguments; report a problem and \PB{\\{exit}} if they don't
define a skew tree\X;\par
\U5.\fi

\N[181 skew-ternary-calc.w]{1}{7}Parsing. First things first: We gotta get the
tree into memory
in a convenient form. The basic data structure has \PB{\\{left}}, \PB{%
\\{middle}},
\PB{\\{right}}, and \PB{\\{parent}} fields in each node, and a few other things
we'll need as we go along. Buds are represented by negative integers;
other links are to a node's 8-bit character code.

(At most 64 different visible character codes are used in this
implementation, namely \PB{\.{'*'}} and the 63 from \PB{\.{'@'}} to \PB{\.{'%
\~'}}.)

\Y\B\4\D$\\{sentinel}$ \5
\T{999}\par
\B\4\D$\\{maxcodes}$ \5
\T{64}\par
\Y\B\4\X7:Type definitions\X${}\E{}$\6
\&{typedef} \&{struct} \&{node\_struct} ${}\{{}$\1\6
\&{int} \\{left};\C{ the left child }\6
\&{int} \\{middle};\C{ the middle child }\6
\&{int} \\{right};\C{ the right child }\6
\&{int} \\{parent};\C{ the parent }\6
\&{int} \\{rank};\C{ set to \PB{\\{sentinel}} at input time; later is the
actual rank }\2\6
${}\}{}$ \&{node};\7
\&{typedef} \&{struct} \&{bud\_struct} ${}\{{}$\1\6
\&{int} \\{parent};\C{ the parent }\6
\&{int} \\{rank};\C{ the actual rank }\6
\&{int} \\{stepno};\C{ step number in the state chart (see below) }\2\6
${}\}{}$ \&{bud};\par
\A14.
\U5.\fi

\M[208 skew-ternary-calc.w]{8}\B\X8:Global variables\X${}\E{}$\6
\&{node} \\{inputnode}[\T{256}];\C{ actually only nodes \PB{\.{'@'}} thru \PB{%
\.{'\~'}} are used }\6
\&{bud} \\{inputbud}[\T{512}];\C{ data for bud number \PB{\|k} is stored in %
\PB{$\\{inputbud}[{-}\|k]$} }\6
\&{int} \\{buds};\C{ this many buds have been created so far }\6
\&{int} \|n;\C{ the number of nodes in the tree }\par
\As15, 30\ETs41.
\U5.\fi

\M[214 skew-ternary-calc.w]{9}The initial setup is straightforward, although a
bit tedious.

\Y\B\4\D$\\{abort0}(\\{message},\\{code})$ \6
${}\{{}$\5
\1${}\\{fprintf}(\\{stderr},\39\.{"\%s!\\n"},\39\\{message});{}$\6
\\{exit}(\\{code});\5
${}\}{}$\2\par
\B\4\D$\\{abort1}(\\{message},\|j,\\{code})$ \6
${}\{{}$\5
\1${}\\{fprintf}(\\{stderr},\39\.{"Bad\ arg\ (\%s):\ \%s!\\n}\)\.{"},\39%
\\{argv}[\|j],\39\\{message});{}$\6
\\{exit}(\\{code});\5
${}\}{}$\2\par
\B\4\D$\\{abort2}(\\{message},\|j,\|c,\\{code})$ \6
${}\{{}$\5
\1${}\\{fprintf}(\\{stderr},\39\.{"Bad\ arg\ (\%s):\ Node\ }\)\.{'\%c'\ \%s!%
\\n"},\39\\{argv}[\|j],\39\|c,\39\\{message});{}$\6
\\{exit}(\\{code});\5
${}\}{}$\2\par
\Y\B\4\X9:Parse the arguments; report a problem and \PB{\\{exit}} if they don't
define a skew tree\X${}\E{}$\6
\&{for} ${}(\|j\K\T{1};{}$ ${}\|j<\\{argc};{}$ ${}\|j\PP){}$\5
${}\{{}$\1\6
\&{if} ${}(\\{strlen}(\\{argv}[\|j])\I\T{4}){}$\1\5
${}\\{abort1}(\.{"Must\ be\ four\ charac}\)\.{ters\ long"},\39\|j,\39{-}%
\T{10});{}$\2\6
${}\|c\K\\{argv}[\|j][\T{0}];{}$\6
\&{if} ${}(\|c<\.{'@'}\V\|c>\.{'\~'}){}$\1\5
${}\\{abort2}(\.{"is\ not\ permitted"},\39\|j,\39\|c,\39{-}\T{15});{}$\2\6
\&{if} ${}(\\{inputnode}[\|c].\\{rank}){}$\1\5
${}\\{abort2}(\.{"has\ already\ been\ de}\)\.{fined"},\39\|j,\39\|c,\39{-}%
\T{11});{}$\2\6
${}\\{inputnode}[\|c].\\{rank}\K\\{sentinel};{}$\6
${}\|p\K\\{argv}[\|j][\T{1}];{}$\6
\&{if} ${}(\|p\I\.{'-'}){}$\5
${}\{{}$\1\6
\&{if} ${}(\|p<\.{'@'}\V\|p>\.{'\~'}){}$\1\5
${}\\{abort2}(\.{"is\ not\ permitted"},\39\|j,\39\|p,\39{-}\T{16});{}$\2\6
\&{if} ${}(\\{inputnode}[\|p].\\{parent}){}$\1\5
${}\\{abort2}(\.{"already\ has\ a\ paren}\)\.{t"},\39\|j,\39\|p,\39{-}%
\T{12});{}$\2\6
${}\\{inputnode}[\|c].\\{left}\K\|p,\39\\{inputnode}[\|p].\\{parent}\K\|c;{}$\6
\4${}\}{}$\2\6
${}\|p\K\\{argv}[\|j][\T{2}];{}$\6
\&{if} ${}(\|p\I\.{'-'}){}$\5
${}\{{}$\1\6
\&{if} ${}(\|p<\.{'@'}\V\|p>\.{'\~'}){}$\1\5
${}\\{abort2}(\.{"is\ not\ permitted"},\39\|j,\39\|p,\39{-}\T{17});{}$\2\6
\&{if} ${}(\\{inputnode}[\|p].\\{parent}){}$\1\5
${}\\{abort2}(\.{"already\ has\ a\ paren}\)\.{t"},\39\|j,\39\|p,\39{-}%
\T{13});{}$\2\6
${}\\{inputnode}[\|c].\\{middle}\K\|p,\39\\{inputnode}[\|p].\\{parent}\K\|c;{}$%
\6
\4${}\}{}$\2\6
${}\|p\K\\{argv}[\|j][\T{3}];{}$\6
\&{if} ${}(\|p\I\.{'-'}){}$\5
${}\{{}$\1\6
\&{if} ${}(\|p<\.{'@'}\V\|p>\.{'\~'}){}$\1\5
${}\\{abort2}(\.{"is\ not\ permitted"},\39\|j,\39\|p,\39{-}\T{18});{}$\2\6
\&{if} ${}(\\{inputnode}[\|p].\\{parent}){}$\1\5
${}\\{abort2}(\.{"already\ has\ a\ paren}\)\.{t"},\39\|j,\39\|p,\39{-}%
\T{14});{}$\2\6
${}\\{inputnode}[\|c].\\{right}\K\|p,\39\\{inputnode}[\|p].\\{parent}\K\|c;{}$\6
\4${}\}{}$\2\6
\4${}\}{}$\2\6
${}\|n\K\\{argc}-\T{1};{}$\6
\X10:Introduce the buds and compute the ranks\X;\par
\U6.\fi

\M[264 skew-ternary-calc.w]{10}We need to locate the root, which should be the
unique input node
that has no parent. Then we'll attach bud \PB{${-}\T{1}$} to it.

Buds $-2k-1$ and $-2k-2$ are mates; hence the mate of bud \PB{\|x} is bud \PB{$%
\|x\XOR\T{1}$}.

\Y\B\4\D$\\{root}$ \5
$\\{inputbud}[\T{1}].{}$\\{parent}\par
\Y\B\4\X10:Introduce the buds and compute the ranks\X${}\E{}$\6
\&{for} ${}(\|j\K\.{'@'};{}$ ${}\|j\Z\.{'\~'};{}$ ${}\|j\PP){}$\5
${}\{{}$\1\6
\&{if} ${}(\\{inputnode}[\|j].\\{rank}\W\R\\{inputnode}[\|j].\\{parent}){}$\5
${}\{{}$\1\6
\&{if} (\\{root})\5
${}\{{}$\1\6
${}\\{fprintf}(\\{stderr},\39\.{"Nodes\ '\%c'\ and\ '\%c'}\)\.{\ cannot\ both\
be\ root}\)\.{s!\\n"},\39\\{root},\39\|j);{}$\6
${}\\{exit}({-}\T{20});{}$\6
\4${}\}{}$\2\6
${}\\{root}\K\|j;{}$\6
\4${}\}{}$\2\6
\&{if} ${}(\\{inputnode}[\|j].\\{parent}\W\R\\{inputnode}[\|j].\\{rank}){}$\5
${}\{{}$\1\6
${}\\{fprintf}(\\{stderr},\39\.{"No\ data\ was\ supplie}\)\.{d\ for\ node\ `%
\%c'!\\n"},\39\|j);{}$\6
${}\\{exit}({-}\T{21});{}$\6
\4${}\}{}$\2\6
\4${}\}{}$\2\6
\&{if} ${}(\R\\{root}){}$\1\5
${}\\{abort0}(\.{"There's\ no\ root"},\39{-}\T{21});{}$\2\6
${}\\{inputbud}[\T{1}].\\{rank}\K{-}\T{2}{}$;\C{ bud number 1 is the ``root
bud,'' above the root node }\6
\\{setmate}(\\{root});\C{ locate and define its mate }\6
${}\\{fillbuds}(\\{root},\39\T{0});{}$\6
\X13:Check that we've filled out the whole tree\X;\par
\U9.\fi

\M[293 skew-ternary-calc.w]{11}The \PB{\\{setmate}} subroutine allocates two
new buds. Its parameter names
the node where we discovered the existence of such buds.

In most cases, the mate is reached by going upward through middle
links, then crossing from left to right (or vice versa) and going downward
through middle links.

\Y\B\4\X11:Subroutines\X${}\E{}$\6
\&{void} \\{setmate}(\&{int} \|p)\1\1\2\2\6
${}\{{}$\1\6
\&{register} \&{int} \|q${},{}$ \|d;\7
${}\\{buds}\MRL{+{\K}}\T{2},\39\|q\K\T{1}-\\{buds};{}$\6
\&{if} ${}(\\{inputbud}[\\{buds}-\T{1}].\\{parent}){}$\5
${}\{{}$\1\6
\&{if} ${}(\\{buds}>\T{2}){}$\1\5
\\{confusion}(\.{"bud\ parent\ already\ }\)\.{set"});\2\6
\&{goto} \\{downward\_mid};\6
\4${}\}{}$\2\6
${}\\{inputbud}[\\{buds}-\T{1}].\\{parent}\K\|p;{}$\6
\4\\{upward}:\5
\&{if} ${}(\\{inputnode}[\|p].\\{middle}\E\|q){}$\5
${}\{{}$\1\6
${}\|q\K\|p,\39\|p\K\\{inputnode}[\|p].\\{parent};{}$\6
\&{goto} \\{upward};\6
\4${}\}{}$\2\6
\&{if} ${}(\\{inputnode}[\|p].\\{left}\E\|q){}$\5
${}\{{}$\1\6
${}\|q\K\|p,\39\|p\K\\{inputnode}[\|p].\\{right},\39\|d\K\T{1};{}$\6
\&{goto} \\{downward};\6
\4${}\}{}$\2\6
\&{if} ${}(\\{inputnode}[\|p].\\{right}\E\|q){}$\5
${}\{{}$\1\6
${}\|q\K\|p,\39\|p\K\\{inputnode}[\|p].\\{left},\39\|d\K{-}\T{1};{}$\6
\&{goto} \\{downward};\6
\4${}\}{}$\2\6
\\{confusion}(\.{"supposed\ parent\ nod}\)\.{e\ not\ apparent"});\6
\4\\{downward\_mid}:\5
${}\|q\K\|p,\39\|p\K\\{inputnode}[\|p].\\{middle},\39\|d\K\T{0};{}$\6
\4\\{downward}:\5
\&{if} ${}(\|p<\T{0}){}$\1\5
${}\\{abort0}(\.{"Mate\ mixup"},\39{-}\T{25});{}$\2\6
\&{if} ${}(\|p>\T{0}){}$\1\5
\&{goto} \\{downward\_mid};\2\6
\&{if} ${}(\|d>\T{0}){}$\1\5
${}\\{inputnode}[\|q].\\{right}\K{-}\\{buds};{}$\2\6
\&{else} \&{if} ${}(\|d<\T{0}){}$\1\5
${}\\{inputnode}[\|q].\\{left}\K{-}\\{buds};{}$\2\6
\&{else}\1\5
${}\\{inputnode}[\|q].\\{middle}\K{-}\\{buds};{}$\2\6
${}\\{inputbud}[\\{buds}].\\{parent}\K\|q;{}$\6
\4${}\}{}$\2\par
\As12, 17, 18, 19, 21, 24, 32, 33, 42\ETs55.
\U5.\fi

\M[331 skew-ternary-calc.w]{12}The main work of filling buds and setting ranks
is done by a straightforward
recursive procedure \PB{\\{fillbuds}}, which traverses the ternary tree in
preorder.

\Y\B\4\X11:Subroutines\X${}\mathrel+\E{}$\6
\&{void} \\{fillbuds}(\&{int} \|p${},\39{}$\&{int} \|r)\1\1\2\2\6
${}\{{}$\1\6
\&{if} ${}(\|r<\T{0}){}$\5
${}\{{}$\1\6
${}\\{fprintf}(\\{stderr},\39\.{"Not\ properly\ skewed}\)\.{:\ rank(\%c)=-1!%
\\n"},\39\|p);{}$\6
${}\\{exit}({-}\T{30});{}$\6
\4${}\}{}$\2\6
${}\\{inputnode}[\|p].\\{rank}\K\|r;{}$\6
\&{if} ${}(\\{inputnode}[\|p].\\{left}>\T{0}){}$\1\5
${}\\{fillbuds}(\\{inputnode}[\|p].\\{left},\39\|r-\T{1});{}$\2\6
\&{else}\5
${}\{{}$\1\6
\&{if} ${}(\\{inputnode}[\|p].\\{left}\E\T{0}){}$\1\5
${}\\{inputnode}[\|p].\\{left}\K{-}\\{buds}-\T{1},\39\\{setmate}(\|p);{}$\2\6
${}\\{inputbud}[{-}\\{inputnode}[\|p].\\{left}].\\{rank}\K\|r-\T{1};{}$\6
\4${}\}{}$\2\6
\&{if} ${}(\\{inputnode}[\|p].\\{middle}>\T{0}){}$\1\5
${}\\{fillbuds}(\\{inputnode}[\|p].\\{middle},\39\|r);{}$\2\6
\&{else}\5
${}\{{}$\1\6
\&{if} ${}(\\{inputnode}[\|p].\\{middle}\E\T{0}){}$\1\5
${}\\{inputnode}[\|p].\\{middle}\K{-}\\{buds}-\T{1},\39\\{setmate}(\|p);{}$\2\6
${}\\{inputbud}[{-}\\{inputnode}[\|p].\\{middle}].\\{rank}\K\|r;{}$\6
\4${}\}{}$\2\6
\&{if} ${}(\\{inputnode}[\|p].\\{right}>\T{0}){}$\1\5
${}\\{fillbuds}(\\{inputnode}[\|p].\\{right},\39\|r+\T{1});{}$\2\6
\&{else}\5
${}\{{}$\1\6
\&{if} ${}(\\{inputnode}[\|p].\\{right}\E\T{0}){}$\1\5
${}\\{inputnode}[\|p].\\{right}\K{-}\\{buds}-\T{1},\39\\{setmate}(\|p);{}$\2\6
${}\\{inputbud}[{-}\\{inputnode}[\|p].\\{right}].\\{rank}\K\|r+\T{1};{}$\6
\4${}\}{}$\2\6
\4${}\}{}$\2\par
\fi

\M[358 skew-ternary-calc.w]{13}We've prepared a skewed ternary tree by filling
in all the missing
fields. But if the input is, say, `\.{A---} \.{B-B-}', the tree we've prepared
won't contain all of the given nodes, because of a cycle. Thus we must
make sure that the number of buds found is $2n+2$.

\Y\B\4\X13:Check that we've filled out the whole tree\X${}\E{}$\6
\&{if} ${}(\\{buds}\I\|n+\|n+\T{2}){}$\1\5
${}\\{abort0}(\.{"The\ input\ contains\ }\)\.{a\ cycle"},\39{-}\T{66}){}$;\2\par
\U10.\fi

\N[366 skew-ternary-calc.w]{1}{14}The state chart.
Now that we've got the tree in memory, we can emulate Alice's moves.
This program gathers more information than is absolutely needed, just in case
the extra data will help me psych out some structural properties.

\Y\B\4\X7:Type definitions\X${}\mathrel+\E{}$\6
\&{typedef} \&{struct} \&{step\_struct} ${}\{{}$\1\6
\&{int} \\{rank};\C{ rank on entry }\6
\&{int} \\{first};\C{ bud being passed, or arc's initial node }\6
\&{int} \\{second};\C{ arc's final node (in the second case) }\6
\&{int} \\{match};\C{ bud being matched (in the second case) }\2\6
${}\}{}$ \&{step};\par
\fi

\M[379 skew-ternary-calc.w]{15}\B\X8:Global variables\X${}\mathrel+\E{}$\6
\&{step} ${}\\{chart}[\T{4}*\\{maxcodes}]{}$;\C{ the state chart, of length
$4n$ }\6
\&{int} \\{steps};\C{ the current number of entries in \PB{\\{chart}} }\6
\&{int} \\{stack}[\T{256}];\C{ buds currently unmatched }\6
\&{int} \\{stacked};\C{ the number of such buds }\par
\fi

\M[385 skew-ternary-calc.w]{16}\B\X16:Find and print the three conjugates of %
\PB{\|T}\X${}\E{}$\6
\X20:Create the state chart\X;\6
\X23:Print the tree with all buds shown\X;\6
\X22:Print the conjugates from the state chart\X;\par
\U5.\fi

\M[390 skew-ternary-calc.w]{17}The state chart is created from a recursive
routine \PB{\\{createsteps}},
analogous to \PB{\\{fillbuds}}.

\Y\B\4\X11:Subroutines\X${}\mathrel+\E{}$\6
\&{void} ${}\\{branch}(\&{int},\39\&{int}){}$;\C{ see below }\6
\&{void} \\{budstep}(\&{int});\C{ see below }\7
\&{void} \\{createsteps}(\&{int} \|p)\1\1\2\2\6
${}\{{}$\1\6
\&{register} \&{int} \|q;\7
${}\|q\K\\{inputnode}[\|p].\\{left};{}$\6
\&{if} ${}(\|q>\T{0}){}$\1\5
${}\\{branch}(\|p,\39\|q);{}$\2\6
\&{else}\1\5
${}\\{budstep}({-}\|q);{}$\2\6
${}\|q\K\\{inputnode}[\|p].\\{middle};{}$\6
\&{if} ${}(\|q>\T{0}){}$\1\5
${}\\{branch}(\|p,\39\|q);{}$\2\6
\&{else}\1\5
${}\\{budstep}({-}\|q);{}$\2\6
${}\|q\K\\{inputnode}[\|p].\\{right};{}$\6
\&{if} ${}(\|q>\T{0}){}$\1\5
${}\\{branch}(\|p,\39\|q);{}$\2\6
\&{else}\1\5
${}\\{budstep}({-}\|q);{}$\2\6
\4${}\}{}$\2\par
\fi

\M[409 skew-ternary-calc.w]{18}\B\D$\\{offset}$ \5
\T{2}\C{ difference between \PB{\\{stacked}} and the current rank }\par
\Y\B\4\X11:Subroutines\X${}\mathrel+\E{}$\6
\&{void} \\{budstep}(\&{int} \|b)\1\1\2\2\6
${}\{{}$\C{ chart gains a bud }\1\6
${}\\{chart}[\\{steps}].\\{first}\K\|b,\39\\{chart}[\\{steps}].\\{rank}\K%
\\{stacked}-\\{offset};{}$\6
\&{if} ${}(\\{chart}[\\{steps}].\\{rank}\I\\{inputbud}[\|b].\\{rank}){}$\1\5
\\{confusion}(\.{"rank\ offense\ b"});\2\6
${}\\{inputbud}[\|b].\\{stepno}\K\\{steps};{}$\6
${}\\{steps}\PP,\39\\{stack}[\\{stacked}\PP]\K\|b;{}$\6
\4${}\}{}$\2\par
\fi

\M[419 skew-ternary-calc.w]{19}\B\X11:Subroutines\X${}\mathrel+\E{}$\6
\&{void} \\{branch}(\&{int} \|p${},\39{}$\&{int} \|q)\1\1\2\2\6
${}\{{}$\C{ chart passes from one arc to its dual }\1\6
${}\\{chart}[\\{steps}].\\{first}\K\|p,\39\\{chart}[\\{steps}].\\{second}\K\|q,%
\39\\{chart}[\\{steps}].\\{rank}\K\\{stacked}-\\{offset};{}$\6
\&{if} ${}(\\{chart}[\\{steps}].\\{rank}\I\\{inputnode}[\|q].\\{rank}){}$\1\5
\\{confusion}(\.{"rank\ offense\ q"});\2\6
${}\\{chart}[\\{steps}].\\{match}\K\\{stack}[\MM\\{stacked}];{}$\6
${}\\{steps}\PP;{}$\6
\\{createsteps}(\|q);\6
${}\\{chart}[\\{steps}].\\{first}\K\|q,\39\\{chart}[\\{steps}].\\{second}\K\|p,%
\39\\{chart}[\\{steps}].\\{rank}\K\\{stacked}-\\{offset};{}$\6
\&{if} ${}(\\{chart}[\\{steps}].\\{rank}\I\\{inputnode}[\|q].\\{rank}+\T{2}){}$%
\1\5
\\{confusion}(\.{"rank\ offense\ p"});\2\6
${}\\{chart}[\\{steps}].\\{match}\K\\{stack}[\MM\\{stacked}];{}$\6
${}\\{steps}\PP;{}$\6
\4${}\}{}$\2\par
\fi

\M[432 skew-ternary-calc.w]{20}\B\X20:Create the state chart\X${}\E{}$\6
$\\{chart}[\T{0}].\\{rank}\K{-}\T{2},\39\\{chart}[\T{0}].\\{first}\K\T{1},\39%
\\{steps}\K\T{1};{}$\6
${}\\{stack}[\T{0}]\K\T{1},\39\\{stacked}\K\\{offset}-\T{1};{}$\6
\\{createsteps}(\\{root});\6
\&{if} ${}(\\{stacked}\I\T{2}+\\{offset}){}$\1\5
\\{confusion}(\.{"mismatched"});\2\6
\&{if} ${}(\\{steps}\I\T{4}*\|n){}$\1\5
\\{confusion}(\.{"total\ steps"});\2\par
\U16.\fi

\M[439 skew-ternary-calc.w]{21}Conversely, given the state chart, there's a
simple recursive routine
that prints a tree beginning after an unmatched bud.

Interestingly, the nodes of the tree are reported in postorder although
they are encountered in preorder.

\Y\B\4\X11:Subroutines\X${}\mathrel+\E{}$\6
\&{void} \\{printfam}(\&{int} \|p)\1\1\2\2\6
${}\{{}$\1\6
\&{register} \&{int} \|q;\6
\&{int} \|l${},{}$ \|m${},{}$ \|r;\7
\&{if} ${}(\\{steps}\E\T{4}*\|n){}$\1\5
${}\\{steps}\K\T{0};{}$\2\6
${}\|q\K\\{chart}[\\{steps}\PP].\\{second};{}$\6
\&{if} ${}(\|q\E\T{0}){}$\1\5
${}\|l\K\.{'-'};{}$\2\6
\&{else}\1\5
${}\|l\K\|q,\39\\{printfam}(\|q),\39\\{steps}\PP;{}$\2\6
\&{if} ${}(\\{steps}\E\T{4}*\|n){}$\1\5
${}\\{steps}\K\T{0};{}$\2\6
${}\|q\K\\{chart}[\\{steps}\PP].\\{second};{}$\6
\&{if} ${}(\|q\E\T{0}){}$\1\5
${}\|m\K\.{'-'};{}$\2\6
\&{else}\1\5
${}\|m\K\|q,\39\\{printfam}(\|q),\39\\{steps}\PP;{}$\2\6
\&{if} ${}(\\{steps}\E\T{4}*\|n){}$\1\5
${}\\{steps}\K\T{0};{}$\2\6
${}\|q\K\\{chart}[\\{steps}\PP].\\{second};{}$\6
\&{if} ${}(\|q\E\T{0}){}$\1\5
${}\|r\K\.{'-'};{}$\2\6
\&{else}\1\5
${}\|r\K\|q,\39\\{printfam}(\|q),\39\\{steps}\PP;{}$\2\6
${}\\{printf}(\.{"\ \%c\%c\%c\%c"},\39\|p,\39\|l,\39\|m,\39\|r);{}$\6
\4${}\}{}$\2\par
\fi

\M[465 skew-ternary-calc.w]{22}The algorithm here is very cute, so I let the
reader have the
fun of deciphering it.

\Y\B\4\X22:Print the conjugates from the state chart\X${}\E{}$\6
$\\{chart}[\T{4}*\|n].\\{second}\K\\{sentinel},\39\\{chart}[\T{4}*\|n].%
\\{first}\K\\{root};{}$\6
\&{for} ${}(\|j\K\T{1};{}$ ${}\|j<\T{4};{}$ ${}\|j\PP){}$\5
${}\{{}$\1\6
\&{for} ${}(\|i\K\T{0};{}$ ${}\|i<\|j;{}$ ${}\|i\PP){}$\1\5
\\{printf}(\.{"+"});\2\6
\\{printf}(\.{":"});\6
\&{for} ${}(\|k\K\\{steps}\K\\{inputbud}[\\{stack}[\|j+\\{offset}-\T{2}]].%
\\{stepno}+\T{1};{}$ ${}\\{chart}[\|k].\\{second}\E\T{0};{}$ ${}\|k\PP){}$\1\5
;\2\6
${}\\{printfam}(\\{chart}[\|k].\\{first});{}$\6
\\{printf}(\.{"\\n"});\6
\&{if} ${}(\\{chart}[\\{steps}].\\{first}\I\\{stack}[\|j+\\{offset}-\T{2}]){}$%
\1\5
\\{confusion}(\.{"bad\ end\ of\ cycle"});\2\6
\4${}\}{}$\2\par
\U16.\fi

\M[479 skew-ternary-calc.w]{23}\B\X23:Print the tree with all buds shown\X${}%
\E{}$\6
\\{print\_tree}(\\{root});\6
\\{printf}(\.{"\\n"});\par
\U16.\fi

\M[483 skew-ternary-calc.w]{24}\B\X11:Subroutines\X${}\mathrel+\E{}$\6
\&{void} \\{print\_tree}(\&{int} \|p)\1\1\2\2\6
${}\{{}$\C{ prints node \PB{\|p}'s subtree in preorder }\1\6
\&{register} \&{int} \|i;\7
\&{for} ${}(\|i\K\T{0};{}$ ${}\|i<\\{inputnode}[\|p].\\{rank}+\T{8};{}$ ${}\|i%
\PP){}$\1\5
\\{printf}(\.{"."});\2\6
${}\\{printf}(\.{"\ \%c:"},\39\|p);{}$\6
\&{if} ${}(\\{inputnode}[\|p].\\{left}<\T{0}){}$\1\5
${}\\{printf}(\.{"\%3d"},\39{-}\\{inputnode}[\|p].\\{left});{}$\2\6
\&{else}\1\5
${}\\{printf}(\.{"\ \ \%c"},\39\\{inputnode}[\|p].\\{left});{}$\2\6
\&{if} ${}(\\{inputnode}[\|p].\\{middle}<\T{0}){}$\1\5
${}\\{printf}(\.{"\%3d"},\39{-}\\{inputnode}[\|p].\\{middle});{}$\2\6
\&{else}\1\5
${}\\{printf}(\.{"\ \%c\ "},\39\\{inputnode}[\|p].\\{middle});{}$\2\6
\&{if} ${}(\\{inputnode}[\|p].\\{right}<\T{0}){}$\1\5
${}\\{printf}(\.{"\%3d\\n"},\39{-}\\{inputnode}[\|p].\\{right});{}$\2\6
\&{else}\1\5
${}\\{printf}(\.{"\ \%c\\n"},\39\\{inputnode}[\|p].\\{right});{}$\2\6
\&{if} ${}(\\{inputnode}[\|p].\\{left}>\T{0}){}$\1\5
${}\\{print\_tree}(\\{inputnode}[\|p].\\{left});{}$\2\6
\&{if} ${}(\\{inputnode}[\|p].\\{middle}>\T{0}){}$\1\5
${}\\{print\_tree}(\\{inputnode}[\|p].\\{middle});{}$\2\6
\&{if} ${}(\\{inputnode}[\|p].\\{right}>\T{0}){}$\1\5
${}\\{print\_tree}(\\{inputnode}[\|p].\\{right});{}$\2\6
\4${}\}{}$\2\par
\fi

\N[506 skew-ternary-calc.w]{1}{25}The quad-edge data structure for planar maps.
Turning from trees to more complex graphs drawn in the plane,
we now implement some beautiful data structures that were introduced
by Leo Guibas and Jorge Stolfi in {\sl ACM Transactions on Graphics\/
\bf4} (1985), 74--123.

The best way to understand their ``quad-edge structure'' is to consider
a small example. The normal way to draw a planar graph with, say, vertices
$\{1,2,3,4\}$ and edges $\{a,b,c,d,e,f\}$ and faces $\rm\{I,II,III,IV\}$
is to connect the vertices by lines for the edges, and to name the
faces in the enclosed regions:
$$\vcenter{\epsfbox{skew-ternary-calc.21}}\eqno({*})$$
Inside a computer, however, the best way to represent the topology of
this diagram is to construct a more elaborate structure, which can be
regarded as annotating the graph~$(*)$ and embedding it in a richer graph:
$$\vcenter{\epsfbox{skew-ternary-calc.20}}\eqno({**})$$
Vertices (red) and faces (green) have been replaced by oriented cycles,
which all travel counterclockwise, except that the outermost cycle
runs clockwise. (That cycle would run the other way if we drew it on the
equator of a sphere and looked at it from the south pole, while viewing
the rest of the map from the north pole.)

The oriented cycles in $(**)$ have
little connectors that we shall call ``pips.''
The cycle for a vertex $v$ of degree~$d$ has $d$ pips,
which indicate all of the edges adjacent to~$v$, in
counterclockwise order. Similarly, the cycle for a face indicates all
of the edges surrounding that face, as we march counterclockwise around it.

One advantage of a representation like $(**)$ is the fact that it nicely
represents also the {\it dual\/} graph, in which vertices become faces, faces
become vertices, and edges ``rotate'' by $90^\circ$. For example, the dual
of $(*)$ is the planar graph
$$\vcenter{\epsfbox{skew-ternary-calc.22}}\,.\eqno({*{*}*})$$

\fi

\M[541 skew-ternary-calc.w]{26}Notice that each of the edges $\{a,b,c,d,e,f\}$
in $(*)$ appears as
a vertex in~$(**)$. Every such vertex has degree~4, and it connects
to four pips via lines numbered 0, 1, 2, and~3 in clockwise order.
The pips on lines 0 and 2 always
belong to vertex cycles; the pips on lines 1 and 3
always belong to face cycles. We could change the numbers $(0,1,2,3)$
to $(2,3,0,1)$, respectively, at each edge-vertex, without changing
the meaning of this diagram; the actual numbering of these lines isn't
important. But their cyclic ordering is crucial, and so is
their evenness or oddness.

I should point out that two planar graphs are considered to be
essentially the same if they are topologically equivalent
when drawn on the surface of a sphere. They should represent the
same decomposition of the sphere's surface into regions delimited
by the given edges, in the sense that we could transform one drawing into
the other, smoothly and without trickery. In particular, we could
redraw $(*)$ in three equivalent ways by choosing any of the other
faces to be exterior:
$$\def\epsfsize#1#2{.8#1}
\vcenter{\epsfbox{skew-ternary-calc.23}}\qquad
\vcenter{\epsfbox{skew-ternary-calc.24}}\qquad
\vcenter{\epsfbox{skew-ternary-calc.25}}$$
Each of these graphs corresponds precisely
to the vertices, edges, faces, and pips of~$(**)$;
and so are three variants of~$(*{*})$! Butleft-right reflection
would give a different graph in this case, because $(*)$ has no symmetry.

\fi

\M[569 skew-ternary-calc.w]{27}Suppose there are $m$ edges. Diagram $(**)$ can
also be regarded as a
{\it permutation\/} of the $4m$ pips, expressed in cycle form, namely
$$\alpha=(a_2d_2c_2b_2)(d_0e_2)(c_0e_0f_0)(a_0b_0f_2)
(a_1f_1e_3d_3)(c_3d_1e_1)(b_3c_1f_3)(a_3b_1).$$
These cycles correspond to
vertices (1), (2), (3), (4) and faces (I), (II), (III), (IV);
for instance, `$(a_0b_0f_2)$' describes the cycle
$a_0\dadj b_0\dadj f_2\dadj a_0$ for vertex~4 in~$(**)$.

Guibas and Stolfi noted that the
permutations $\alpha$ obtained from planar maps in this way
have a very important property called the backup axiom:
{\sl If $\alpha$ takes $u_{i+1}\mapsto v_j$},
where $u$ and~$v$ are edges of the planar graph being represented and
where their subscripts are treated modulo~4, {\sl then $\alpha$ also
takes $v_{j+1}\mapsto u_i$}. For example, $a_2\mapsto d_2$ and
$d_3\mapsto a_1$ in $\alpha$. Notice that $a_2$ and $d_2$ are vertex pips,
but $d_3$ and $a_1$ are face pips.

The backup axiom can be formulated in terms of permutations, using the
special ``rotation'' permutation
$$\rho=(a_0a_1a_2a_3)(b_0b_1b_2b_3)(c_0c_1c_2c_3)(d_0d_1d_2d_3)
(e_0e_1e_2e_3)(f_0f_1f_2f_3);$$
namely, it is equivalent to saying that $\alpha\rho\alpha\rho$ is the
identity permutation. After moving from any pip by
applying $\alpha$ and then $\rho$, we can
back up to our original state by applying $\alpha$ and~$\rho$ again.
This principle underlies the efficiency of a quad-edge structure, because
we can easily move in the clockwise or counterclockwise direction
around any vertex or around any face; we needn't traverse the whole cycle
to find our predecessor.

Continuing our example, we have
$$\alpha\rho=
(a_0b_1)(a_1f_2)(a_2d_3)(a_3b_2)(b_0f_3)(b_3c_2)
(c_0e_1)(c_1f_0)(c_3d_2)(d_0e_3)(d_1e_2)(e_0f_1).$$
In general $\alpha\rho$ will always be
a permutation of order~2, which takes every vertex pip into some face pip.
Therefore $\alpha\rho$ consists entirely of two-cycles; it's a matching
between the $2m$ vertex pips and the $2m$ face pips.

Similarly, $\rho\alpha$ always consists of two-cycles; in our case it's
$$\rho\alpha=
(a_0f_1)(a_1d_2)(a_2b_1)(a_3b_0)(b_3f_2)(b_2c_1)
(c_3e_0)(c_0f_3)(c_2d_1)(d_3e_2)(d_0e_1)(e_3f_0).$$
We'll have $(u_iv_j)$ in $\rho\alpha$ if and only if $(u_{i+1}v_{j+1})$
is in $\alpha\rho$, because of the backup axiom. For example,
$\rho\alpha$ contains $(a_1d_2)$ which $\alpha\rho$ contains $(a_2d_3)$.

The regions outside the cycles in $(**)$ all have four sides. For example,
there's a region near the top right whose corner pips in counterclockwise
order are $(d_3,d_2,a_2,a_1)$. The backup axiom explains this fact.
Moreover, it tells us that the
pips at opposite corners, such as $\{d_3,a_2\}$ and $\{d_2,a_1\}$, are the
pips matched by $\alpha\rho$ and $\rho\alpha$.

\fi

\M[625 skew-ternary-calc.w]{28}Thus the quad-edge data structure for a planar
graph with $m$ edges
essentially consists of $4m$ pointers, which tell us how to move
from one pip to another. These pointers form a permutation of the
pips, where the permutation takes vertex pips into vertex pips and
face pips into face pips. It also satisfies the backup axiom.

Guibas and Stolfi also observed that every planar graph without
isolated vertices can be constructed
by repeatedly performing a single primitive operation.
This operation, called {\it splice},
exchanges two vertex pips and two face pips. (Well, there's
also a primitive operation that initializes the entire data structure.
To get things rolling, we begin with
a set of $m$ edges that define $m$ two-vertex components;
thus we have $2m$ vertices initially, each of degree~1.
After the initialization, we can proceed to splice until we've got the
graph we want.)

The best way to understand splicing is---guess what---to look at
a small example. So let's construct the pip-permutation~$\alpha$
above, and the planar graph above, by a sequence of splices.

We begin with the initial permutation
$$\alpha_0\beta_0\gamma_0\delta_0\varepsilon_0\varphi_0;$$
here $\alpha_0=(a_0)(a_2)(a_1a_3)$,
$\beta_0=(b_0)(b_2)(b_1b_3)$, \dots, and
$\varphi_0=(f_0)(f_2)(f_1f_3)$
each specify a two-vertex graphs~$K_2$ disjoint from the others.
The sub-permutation $\alpha_0$ corresponds to the two-vertex
graph whose edge is named~$a$, and the other sub-permutations are similar.

Two edges $a$ and $b$ belong to the same component of a graph
if and only if there's a sequence $(d_1,d_2,\ldots,d_r)$, for some~$r$,
such that $\alpha\rho^{d_1}\alpha\rho^{d_2}\ldots\alpha\rho^{d_r}$
takes $a_0\mapsto b_0$. If the graph has $c$ components, we can best
think of it as a set of $c$ connected graphs, each of which is drawn on
the surface of a separate sphere; thus each component has its own
``exterior face.'' According to a famous theorem of Euler,
the number of vertices plus the number of faces
is always equal to the number of edges plus $2c$.

A splice $\sigma$ consists of applying two swap operations;
in other words, $\sigma$ has the form $(u_iv_j)(r_ks_l)$. Here $i$ and $j$ are
even (hence $u_i$ and $v_j$ are vertex pips), while $k$ and $l$ are odd (hence
$r_k$ and $s_l$ are face pips). If $u_i$ and $v_j$ lie in different
cycles of~$\alpha$, then they lie in the same cycle of $\alpha\sigma$; in fact,
they are obtained by pasting the cycles together in a straightforward way:
$$(u_ix_1\ldots x_p)(v_jy_1\ldots y_q)(u_iv_j)
=(u_ix_1\ldots x_pv_jy_1\ldots y_q).$$
For example, if $u$ and $v$ are edges of different components,
it's clear that $u_i$ and $v_j$ must lie in different cycles;
and in that case the net effect is to paste two disconnected components
of a graph together, with two vertices coalescing into one.

If $u$ and $v$ are edges of the same component, but $u_i$ and $v_j$ aren't
in the same cycle, then we're allowed to splice them together only if
$u_{i+1}$ and $v_{j+1}$ are pips of the same face. Otherwise
we couldn't merge the vertices of $u_i$ and $v_j$ without making lines
cross. (That's a consequence of Euler's theorem, mentioned above;
we'd be drawing the component on a torus, not a sphere.)

Going the other way, suppose $u_i$ and $v_j$ do lie
in the same cycle of~$\alpha$. Then
the splice operation splits the graph into two parts, making two
copies of the vertex whose pips included $u_i$ and $v_j$; one copy
gets some of the adjacent edges, the other copy gets the rest.
Actually, however, we won't need to do splices of this kind;
it's possible to form any desired planar graph without isolated vertices
merely by pasting
vertices together, decreasing the number of vertices with each splice.

Similar remarks apply to the cycles of face pips, depending on whether
or not $r_k$ and $s_l$ belong to the same cycle of~$\alpha$. A swap
operation between elements of different cycles always merges the
cycles; a swap operation between elements of a single cycle always
splits that cycle.

The value of $(r_ks_l)$ is completely determined by the value of
$(u_iv_j)$ by the following important {\it splice rule\/}:
If~$\alpha$ takes $u_{i+1}\mapsto x_p$
and $v_{j+1}\mapsto y_q$, then $r_k=x_p$ and $s_l=y_q$. This rule
is necessary so that the backup axiom is preserved; we can't paste
vertices together or split them apart unless we do an exactly
consistent thing with respect to the faces. And fortunately, the splice
rule is also sufficient for maintaining the backup condition.

\fi

\M[711 skew-ternary-calc.w]{29}Armed with all this information, we're ready at
last to construct
the example map~$(*)$ from the initial permutation
$\alpha_0\beta_0\gamma_0\delta_0\varepsilon_0\varphi_0$
in a sensible way, without resorting to trial and error.

That map has both $a$ and $b$ attached to vertex~4, with pips
$a_0$ and $b_0$ in the vertex ring for~4 in $(**)$. So we can start
by splicing $\alpha_0$ and $\beta_0$ together; that means
$u_i=a_0$ and~$v_j=b_0$. The splice rule
now tells us that $\sigma=(a_0b_0)(a_3b_3)$, because $\alpha_0\beta_0$ takes
$a_1\mapsto a_3$ and $\beta_1\mapsto b_3$. Thus we obtain
$$\alpha_1=\alpha_0\beta_0\sigma_1=\alpha_0\beta_0(a_0b_0)(a_3b_3)=
(a_0b_0)(a_2)(b_2)(a_1b_3b_1a_3).$$
This permutation represents a path of length 2, consisting of edges $a$ and $b$
joined by a common vertex whose pips are $a_0$ and $b_0$.
The other two vertices, which have degree~1 because they're endpoints of
the path, have the respective pips $a_2$ and $b_2$.
The reader is encouraged to draw the corresponding graph of vertices,
edges, faces, and pips, so that these ideas become crystal clear.
(This graph will be analogous to $(**)$, but it will be considerably
simpler because there are only three vertices, two edges, and one face.)

Next let's join {\it those\/} two vertices together; we're allowed to
do that, even though they belong to the same component,
because $a_3$ and $b_3$ belong to the same cycle. The result, with
$\sigma_2=(a_2b_2)(a_1b_1)$, is
$$\alpha_2=\alpha_1\sigma_2=(a_0b_0)(a_2b_2)(a_1b_3)(a_3b_1),$$
representing a cycle of length 2 between the vertices $(a_0b_0)$ and
$(a_2b_2)$. There are two faces, $(a_1b_3)$ and $(a_3b_1)$, either
of which can be considered to be the exterior face.

In a similar way we can build up a {\it three\/}-cycle from $c$, $d$, and $e$:
Letting $\sigma_3=(c_2d_2)(c_1d_1)$,
$\sigma_4=(d_0e_2)(d_3e_1)$, and $\sigma_5=(c_0e_0)(c_3e_3)$, we obtain
$$\eqalign{
\gamma_1&=\gamma_0\delta_0\sigma_3=
(c_0)(c_2d_2)(d_0)(c_1c_3d_1d_3);\cr
\gamma_2&=\gamma_1\varepsilon_0\sigma_4=
(c_0)(c_2d_2)(d_0e_2)(e_0)(c_1c_3d_1e_1e_3d_3);\cr
\gamma_3&=\gamma_2\varphi\sigma_5=
(c_0e_0)(c_2d_2)(d_0e_2)(c_1e_3d_3)(c_3d_1e_1).\cr
}$$

Now we can hook $f$ to vertex $(c_0e_0)$, using $\sigma_6=(c_0f_0)(e_3f_3)$:
$$\gamma_4=\gamma_3\sigma_6=(c_0e_0f_0)(c_2d_2)(d_0e_2)(f_2)
(c_1f_3f_1e_3d_3)(c_3d_1e_1).$$

The two remaining components can be joined together, merging
vertices $(a_2b_2)$ and $(c_2d_2)$ appropriately with
$\sigma_7=(b_2d_2)(a_1c_1)$:
$$\alpha_3=\alpha_2\gamma_4\sigma_7=
(a_2d_2c_2b_2)(a_0b_0)(c_0e_0f_0)(d_0e_2)(f_2)
(a_1b_3c_1f_3f_1e_3d_3)(a_3b_1)(c_3d_1e_1).$$
And the final coup de gr\^ace hooks $(f_2)$ to $(a_0b_0)$, with
$\sigma_8=(f_2a_0)(f_1b_3)$:
$$\alpha_4=\alpha_3\sigma_8=
(a_0b_0f_2)(a_2d_2c_2b_2)(c_0e_0f_0)(d_0e_2)
(a_1f_1e_3d_3)(a_3b_1)(b_3c_1f_3)(c_3d_1e_1).$$

Yes, this is indeed the permutation of $(**)$. We have
$$\alpha=\alpha_0\beta_0\gamma_0\delta_0\varepsilon_0\varphi_0\,
\sigma_1\sigma_2\sigma_3\sigma_4\sigma_5\sigma_6\sigma_7\sigma_8.$$
One of the
main reasons I wrote this program was because I knew that a computer
could do these calculations almost instantly, without making
silly mistakes.

\fi

\M[778 skew-ternary-calc.w]{30}So let's write that code. Each pip is
conveniently represented
by its subscript plus four times the ASCII code of the edge name.
For example, $a_3$ would be \PB{$(\.{'a'}\LL\T{2})+\T{3}$}, which is 391
because
\PB{$\.{'a'}\K\T{97}$}.

We maintain both $\alpha$ and its inverse $\alpha^-$ in memory,
because both representations are useful. If \PB{\|p} is a pip with
\PB{$\\{alpha}[\|p]\K\|q$}, then \PB{$\|q\K\\{alphainv}[\|p]$}.
(There isn't really a need for both representations, however,
because the backup axiom $\alpha^-=\rho\alpha\rho$ always holds.)

\Y\B\4\D$\\{pip}(\|u,\|i)$ \5
$((\|u)\LL\T{2})+{}$(\|i)\par
\B\4\D$\\{pip\_edge}(\|p)$ \5
$((\|p)\GG\T{2}{}$)\par
\B\4\D$\\{pip\_sub}(\|p)$ \5
$((\|p)\AND\T{\^3}{}$)\par
\B\4\D$\\{rot}(\|p)$ \5
$(((\|p)+\T{1})\XOR(((\|p)\XOR((\|p)+\T{1}))\AND{-}\T{4}){}$)\C{ $\rho$ }\par
\B\4\D$\\{irot}(\|p)$ \5
$(((\|p)-\T{1})\XOR(((\|p)\XOR((\|p)-\T{1}))\AND{-}\T{4}){}$)\C{ $\rho^-$ }\par
\Y\B\4\X8:Global variables\X${}\mathrel+\E{}$\6
\&{int} ${}\\{alpha}[\T{4}*\T{256}]{}$;\C{ the permutation of pips describing
the current planar map }\6
\&{int} ${}\\{alphainv}[\T{4}*\T{256}]{}$;\C{ its inverse }\6
\&{int} \\{verts};\C{ the current number of vertices }\par
\fi

\M[801 skew-ternary-calc.w]{31}We'll permute only the pips for a special {\it
root edge\/}
and for edges that correspond to nodes in the skew ternary tree that was
input. The latter nodes are identifiable because they have left children.
(They also have middle and root children. But we don't really care
about the children's identities, only their existence.)
The root edge is called \.*.

\Y\B\4\X31:Create the initial permutation\X${}\E{}$\6
\&{for} ${}(\|k\K\.{'*'},\39\\{verts}\K\T{0};{}$ ${}\|k\Z\.{'\~'};{}$ ${}\|k%
\PP){}$\1\6
\&{if} ${}(\|k\E\.{'*'}\V\\{inputnode}[\|k].\\{left}){}$\5
${}\{{}$\1\6
${}\\{alpha}[\\{pip}(\|k,\39\T{0})]\K\\{alphainv}[\\{pip}(\|k,\39\T{0})]\K%
\\{pip}(\|k,\39\T{0});{}$\6
${}\\{alpha}[\\{pip}(\|k,\39\T{1})]\K\\{alphainv}[\\{pip}(\|k,\39\T{1})]\K%
\\{pip}(\|k,\39\T{3});{}$\6
${}\\{alpha}[\\{pip}(\|k,\39\T{2})]\K\\{alphainv}[\\{pip}(\|k,\39\T{2})]\K%
\\{pip}(\|k,\39\T{2});{}$\6
${}\\{alpha}[\\{pip}(\|k,\39\T{3})]\K\\{alphainv}[\\{pip}(\|k,\39\T{3})]\K%
\\{pip}(\|k,\39\T{1});{}$\6
${}\\{verts}\MRL{+{\K}}\T{2};{}$\6
\4${}\}{}$\2\2\6
\&{if} ${}(\\{verts}\I\T{2}*(\|n+\T{1})){}$\1\5
\\{confusion}(\.{"initial\ vertex\ coun}\)\.{t"});\2\par
\Us49\ET62.\fi

\M[818 skew-ternary-calc.w]{32}The \PB{\\{splice}} subroutine is given the
addresses of two vertex pips
that are supposed to be interchanged. It figures out the two
corresponding face pips, using the splice rule mentioned above.

We do not worry about the ``legality'' of a splice, in the sense
of preserving planarity, because we'll use \PB{\\{splice}} only to
reduce the number of vertices. Any illegal usage would cause the
final number of faces to be incompatible with Euler's criterion.

(Well, there's an exception: In one place below I will splice two
pips apart that are adjacent in their vertex cycle. To compensate,
I'll increase \PB{\\{verts}} by~2.)

\Y\B\4\X11:Subroutines\X${}\mathrel+\E{}$\6
\&{void} \\{splice}(\&{int} \|p${},\39{}$\&{int} \|q)\1\1\2\2\6
${}\{{}$\1\6
\&{register} \&{int} \|r${},{}$ \|s;\7
\&{if} ${}((\|p\AND\T{1})+(\|q\AND\T{1})){}$\1\5
\\{confusion}(\.{"attempt\ to\ splice\ f}\)\.{ace\ pips"});\2\6
${}\|r\K\\{alphainv}[\|p],\39\|s\K\\{alphainv}[\|q];{}$\6
${}\\{alphainv}[\|p]\K\|s,\39\\{alphainv}[\|q]\K\|r;{}$\6
${}\\{alpha}[\|s]\K\|p,\39\\{alpha}[\|r]\K\|q;{}$\6
${}\|p\K\\{alpha}[\\{rot}(\|p)],\39\|q\K\\{alpha}[\\{rot}(\|q)]{}$;\C{ now swap
the appropriate faces }\6
${}\|r\K\\{alphainv}[\|p],\39\|s\K\\{alphainv}[\|q];{}$\6
${}\\{alphainv}[\|p]\K\|s,\39\\{alphainv}[\|q]\K\|r;{}$\6
${}\\{alpha}[\|s]\K\|p,\39\\{alpha}[\|r]\K\|q;{}$\6
${}\\{verts}\MM;{}$\6
\4${}\}{}$\2\par
\fi

\M[845 skew-ternary-calc.w]{33}Here's a cute subroutine that displays all
relevant information about
the planar graph by printing $\alpha$'s cycles. First come the pips
of the vertex cycles, then the pips of the face cycles, one line at a time.

The program also counts the number of vertices and faces, so that
it can use Euler's formula to report the number of components (assuming
planarity).

\Y\B\4\X11:Subroutines\X${}\mathrel+\E{}$\6
\&{void} \\{print\_alpha}(\&{void})\1\1\2\2\6
${}\{{}$\1\6
\&{register} \&{int} \|c${},{}$ \|f${},{}$ \|p${},{}$ \|q${},{}$ \|r${},{}$ %
\|t${},{}$ \|v;\7
\X34:Print and count the vertex cycles\X;\6
\&{if} ${}(\|v\I\\{verts}){}$\1\5
\\{confusion}(\.{"vertex\ count"});\2\6
\X35:Print and count the face cycles\X;\6
${}\|c\K(\|v-(\|n+\T{1})+\|f)\GG\T{1};{}$\6
${}\\{printf}(\.{"(Altogether\ \%d\ vert}\)\.{ices,\ \%d\ edges,\ \%d\ f}\)%
\.{aces,\ \%d\ component\%s}\)\.{.)\\n"},\39\|v,\39\|n+\T{1},\39\|f,\39\|c,\39%
\|c\E\T{1}\?\.{""}:\.{"s"});{}$\6
\4${}\}{}$\2\par
\fi

\M[864 skew-ternary-calc.w]{34}The idea is to find a cycle leader (the least %
\PB{\|p} whose cycle hasn't
already been printed), and to print its cycle, until all cycles for
even-numbered pips have been found.

\Y\B\4\X34:Print and count the vertex cycles\X${}\E{}$\6
\\{printf}(\.{"Vertices:\\n"});\6
${}\|v\K\T{0},\39\|p\K\\{pip}(\.{'*'},\39\T{0}),\39\|t\K\T{2}*(\|n+\T{1});{}$\6
\&{while} (\T{1})\5
${}\{{}$\1\6
\&{for} ( ; ${}\\{alpha}[\|p]\Z\T{0}\W\|t;{}$ ${}\|p\MRL{+{\K}}\T{2}){}$\5
${}\{{}$\1\6
\&{if} ${}(\\{alpha}[\|p]<\T{0}){}$\5
${}\{{}$\C{ we've temporarily negated it, see Algorithm 1.3.3I }\1\6
${}\\{alpha}[\|p]\K{-}\\{alpha}[\|p],\39\|t\MM;{}$\6
\4${}\}{}$\2\6
\4${}\}{}$\2\6
\&{if} ${}(\|t\E\T{0}){}$\1\5
\&{break};\C{ \PB{\|t} unprocessed pips remain }\2\6
\&{for} ${}(\|q\K\|p,\39\|r\K\\{alpha}[\|q];{}$ ${}\|r>\T{0};{}$ ${}\|q\K\|r,%
\39\|r\K\\{alpha}[\|q]){}$\5
${}\{{}$\1\6
${}\\{printf}(\.{"\ \%c\%d"},\39\\{pip\_edge}(\|r),\39\\{pip\_sub}(\|r));{}$\6
${}\\{alpha}[\|q]\K{-}\|r;{}$\6
\4${}\}{}$\2\6
\\{printf}(\.{"\\n"});\6
${}\|v\PP;{}$\6
\4${}\}{}$\2\par
\U33.\fi

\M[887 skew-ternary-calc.w]{35}Exactly the same idea works for odd-numbered
pips, of course.

\Y\B\4\X35:Print and count the face cycles\X${}\E{}$\6
\\{printf}(\.{"Faces:\\n"});\6
${}\|f\K\T{0},\39\|p\K\\{pip}(\.{'*'},\39\T{1}),\39\|t\K\T{2}*(\|n+\T{1});{}$\6
\&{while} (\T{1})\5
${}\{{}$\1\6
\&{for} ( ; ${}\\{alpha}[\|p]\Z\T{0}\W\|t;{}$ ${}\|p\MRL{+{\K}}\T{2}){}$\5
${}\{{}$\1\6
\&{if} ${}(\\{alpha}[\|p]<\T{0}){}$\5
${}\{{}$\C{ we've temporarily negated it, see Algorithm 1.3.3I }\1\6
${}\\{alpha}[\|p]\K{-}\\{alpha}[\|p],\39\|t\MM;{}$\6
\4${}\}{}$\2\6
\4${}\}{}$\2\6
\&{if} ${}(\|t\E\T{0}){}$\1\5
\&{break};\C{ \PB{\|t} unprocessed pips remain }\2\6
\&{for} ${}(\|q\K\|p,\39\|r\K\\{alpha}[\|q];{}$ ${}\|r>\T{0};{}$ ${}\|q\K\|r,%
\39\|r\K\\{alpha}[\|q]){}$\5
${}\{{}$\1\6
${}\\{printf}(\.{"\ \%c\%d"},\39\\{pip\_edge}(\|r),\39\\{pip\_sub}(\|r));{}$\6
${}\\{alpha}[\|q]\K{-}\|r;{}$\6
\4${}\}{}$\2\6
\\{printf}(\.{"\\n"});\6
${}\|f\PP;{}$\6
\4${}\}{}$\2\par
\U33.\fi

\N[908 skew-ternary-calc.w]{1}{36}The building blocks of planar graphs.
Any connected multigraph is built up in a straightforward treelike way from
so-called blocks (aka biconnected components or nonseparable graphs),
attached together via so-called
articulation points (aka cut vertices).
We exclude the trivial cases where a block has fewer than two vertices;
in other words, we exclude the empty graph, the one-vertex graph~$K_1$,
and the multigraph that consists of a single self-loop.

A nontrivial biconnected planar graph is said to be {\it rooted\/}
when we place an arrow on one of its edges, converting that edge
to a directed arc from $u$ to~$v$ called the root edge.
Vertex~$u$ is called the root; and we draw the graph so that
the root edge lies on the path that travels counterclockwise
around the exterior face. (In other words, the exterior face lies
on your right, if you move from $u$ to $v$.)

\def\RNBPM/{{\mc RNBPM}}
A rooted, nontrivial biconnected planar map (henceforth ``\RNBPM/'')
is an equivalence class of rooted, nontrivial biconnected planar graphs,
where two such graphs are said to be equivalent if they're topo\-logically
the same when drawn on a sphere as discussed earlier. Thus, each
\RNBPM/ can be characterized by its $\alpha$~permutation, except for
renaming of the edges and except for adding 2~(mod~4) to the subscripts of
any selected subset of the edges.

Suppose $r$ is the root edge, and suppose the other edges of
the exterior cycle are $e^1$, $e^2$, \dots,~$e^p$. We will define
things so that the root vertex cycle contains the pip~$r_0$, hence the
exterior face cycle contains the pip~$r_3$. We will also
define pip numbers so that $\alpha$ takes $r_0\mapsto e^1_2$,
$e^1_0\mapsto e^2_2$, \dots, $e^p_0\mapsto r_2$, so that
the exterior face cycle is $(e^p_3\ldots e^2_3e^1_3r_3)$.
Exception: If $p=0$, that cycle is of course $(r_1r_3)$.

\fi

\M[943 skew-ternary-calc.w]{37}The simplest \RNBPM/ consists of just the root
edge. Otherwise
we can build up any \RNBPM/ recursively in a simple way:
Removal of the root edge leaves a graph with $m\ge1$ blocks
(hence $m-1$ articulation points); consequently each of those blocks
becomes an \RNBPM/ once we identify its root edge.
We choose to take the root as the first edge encountered
on the exterior face of the full graph, in counterclockwise order.
We also take note of the first edge that is {\it not\/} exterior in
the full graph, thereby
making the block ``doubly rooted.'' Then the original \RNBPM/ is
easily reconstructed from its doubly rooted blocks.

Once again we crave an example. Our previous graph $(*)$ will be
an \RNBPM/ if we choose any edge~$u$ as a designated root edge,
and if we consider the pip $u_3$ to be
on its exterior face. But that example is too simple to reveal
the general situation; so let's consider something a bit more complex:
$$\vcenter{\epsfbox{skew-ternary-calc.40}}\eqno(\dag)$$
Here $m=4$ and the exterior face has $p=7$ other edges; its cycle is therefore
$(r_3d^2_3d^1_3c^3_3c^2_3c^1_3b^1_3a^1_3)$. Furthermore the
interior face touching~$r$ is
$(r_1a^3_3a^2_3b^1_1c^7_3c^6_3c^5_3c^4_3d^3_3)$.
If we remove edge~$r$,
three articulation points spring up that subdivide the remaining graph into
four blocks, having exterior edges identified by the letters
$\{a,b,c,d\}$. Block $b$ is just an isthmus, but the other blocks have
been built up in turn from smaller constituents. Those larger blocks
have been shaded in this diagram, because they may contain complicated
interior structure that is invisible from the outside.

The four blocks can be regarded as \RNBPM/s, having the respective
root edge pips $a^1_3$, $b^1_3$, $c^1_3$, and $d^1_3$.
And they're also doubly rooted, because we specify nonroot exterior
vertex pips $a^2_2$, $b^1_0$, $c^4_2$, $d^3_2$ that tell us how to
hook them together. If $\alpha$, $\beta$, $\gamma$, and~$\delta$ are
the permutations corresponding to those blocks, and if
$\omega=(r_0)(r_2)(r_1r_3)$ is the permutation for edge~$r$,
the permutation for the
whole \RNBPM/ is
$\alpha\beta\gamma\delta\omega\,\sigma_1\sigma_2\sigma_3\sigma_4\sigma_5$,
where
$$\sigma_1=(d^3_2r_2)(d^2_3r_1),\quad
\sigma_2=(c^4_2d^1_2)(c^3_3d^3_3),\quad
\sigma_3=(b^1_0c^1_2)(b^1_3c^7_3),\quad
\sigma_4=(a^2_2b^1_2)(a^1_3b^1_1),\quad
\sigma_5=(a^1_2r_0)(r_3a^3_3)
$$
are the appropriate splicings.

\fi

\M[992 skew-ternary-calc.w]{38}Here, for handy reference, are the smallest %
\RNBPM/s and their
canonical permutations:
$$\def\\#1{$\vcenter{\medskip\epsfbox{skew-ternary-calc.3#1}\medskip}$}
\vcenter{\halign{\hfil\\#\hfil&\qquad$#$\hfil\cr
0&(r_0)(r_2)(r_3r_1)\cr
1&(r_0a_2)(a_0r_2)(r_3a_3)(a_1r_1)\cr
2&(r_0a_2b_0)(a_0r_2b_2)(r_3a_3)(a_1b_1)(b_3r_1)\cr
3&(r_0a_2)(a_0b_2)(b_0r_2)(r_3b_3a_3)(a_1b_1r_1)\cr
4&(r_0a_2c_2b_0)(a_0r_2b_2c_0)(r_3a_3)(a_1c_3)(c_1b_1)(b_3r_1)\cr
5&(r_0a_2c_0)(a_0b_2)(b_0r_2c_2)(r_3b_3a_3)(a_1b_1c_1)(c_3r_1)\cr
6&(r_0a_2c_0)(a_0r_2b_2)(b_0c_2)(r_3a_3)(a_1b_1c_1)(c_3b_3r_1)\cr
7&(r_0a_2c_0)(a_0b_2c_2)(b_0r_2)(r_3b_3a_3)(a_1c_1)(c_3b_1r_1)\cr
8&(r_0a_2)(a_0b_2c_0)(b_0r_2c_2)(r_3b_3a_3)(b_1c_1)(a_1c_3r_1)\cr
9&(r_0a_2)(a_0b_2)(b_0c_2)(c_0r_2)(r_3c_3b_3a_3)(r_1a_1b_1c_1)\cr
}}$$

\fi

\N[1008 skew-ternary-calc.w]{1}{39}Planar maps, conform\`emente \`a Jacquard et
Schaeffer.
We return now to our main theme of skew ternary trees.

At the very beginning I mentioned that Del Lungo et al found an
intriguing correspondence between skew ternary trees and \RNBPM/s.
They found it after first having invented the idea of skew ternary
trees, and conjecturing that the number of such trees with $n$ nodes
is precisely the number of \RNBPM/s with $n$ nonroot edges.

Benjamin Jacquard and Gilles Schaeffer responded to that conjecture
by finding an ingenious correspondence that is quite different
from the one discovered almost simultaneously by Del Lungo et al.
[See {\sl Journal of Combinatorial Theory\/ \bf A83}
(1998), 1--20.] Naturally I wondered if the two correspondences are
somehow related, so I decided to implement both of them in this program.

According to their construction, an \RNBPM/ such as $(\dag)$ is
represented by a skew ternary tree of the form
$$\vcenter{\epsfbox{skew-ternary-calc.41}}\quad\lower15pt\hbox{,}$$
where $A'$, $B'$, $C'$, and $D'$ represent the doubly rooted \RNBPM/s
of the $m=4$ blocks that arise when edge~$r$ is removed.
Thus the chart corresponding to their representation will have the form
$$\vcenter{\epsfbox{skew-ternary-calc.42}}\quad\raise10pt\hbox{,}$$
where $A^*$, $B^*$, $C^*$, and $D^*$ represent the subtrees $A'$, $B'$,
$C'$, and $D'$ in some fashion.

In this particular example $B'$ is empty, because component $b$ has
only a single edge in~$(\dag)$; thus $B^*$ is simply a ``$+1$ step''
for the bud $\overline2$. But the subtrees $A'$, $C'$, and $D'$ are
nonempty (and they might in fact be extremely complicated).

The Jacquard--Schaeffer construction also has the property that the total
number of rank~0 nodes is always exactly~$p$, the number of nonroot edges on
the exterior face of the given \RNBPM/. Consequently the subtrees
$D'$ and $C'$ will contain nodes $d^2$, $c^3$, and $c^2$ of rank~0;
but $A'$ won't contain any such nodes.

\fi

\M[1045 skew-ternary-calc.w]{40}To complete the construction, we need to
explain how to represent
a doubly rooted \RNBPM/. Consider, for example, the skew ternary
tree $T$ that appeared in the introductory sections
at the very beginning of this program: The \RNBPM/ corresponding to
$T$ can be used to build larger \RNBPM/s in three different
ways, because $T$ has three nodes \.A, \.B, and \.E of rank~0.

It turns out that the ``buds and charts'' method
discussed above provides a nice way to encode the second root.
The idea is to use one of the three cyclic variants that begin at
a bud of rank~$-1$ (namely at bud 1, 2, or~4). Those trees
have respectively 2, 1, and 0 nodes of rank~$-1$, and no nodes of
rank~$-2$; so they can safely be used as the right subtree $T'$
of a node that has rank~0.

For example, the three possibilities for $T^*$ in this example
have the following respective charts:
$$\vcenter{\halign{#\hfil\cr
\epsfxsize=.5\hsize \epsfbox{skew-ternary-calc.81}\cr\noalign{\smallskip}
\epsfxsize=.5\hsize \epsfbox{skew-ternary-calc.82}\cr\noalign{\smallskip}
\epsfxsize=.5\hsize \epsfbox{skew-ternary-calc.84}\cr}}$$
One way to form this is to start at the bud in question and continue
creating the chart cyclically until that bud occurs again. The
we delete both appearances, and replace it by matching arcs
from an assumed parent node \.T to the new subtree root and back.
(It follows that a subtree $T'$ of $n$ nodes has a chart $T^*$ of length
$4n+1$,
even when $n=0$.)

Conversely, it's easy to reconstruct $T$ from any of these
shifted variants~$T^*$, by undoing the process: First we delete the
matching arcs that enclose the whole; then we replace the bud that
was deleted. Finally we wind back the cycle until creating a
bud with rank $-2$ for the first time. (That bud
will be cloned from the rightmost bud of rank~$+2$.)

Incidentally, one can show by induction that the number of nodes of odd rank
in the skew ternary tree is equal to the number of faces in the corresponding
\RNBPM/, minus~2, according to this construction.
And the number of nodes of even rank is the number of
nonroot vertices.

\fi

\M[1086 skew-ternary-calc.w]{41}Our principal goal is to take a given skew
ternary tree and to
construct the corresponding \RNBPM/, but computing the quad-tree
permutation of that planar map. The tree will be given in
chart form. I won't be stingy with memory, so I'll keep a stack
of the various charts that arise during the recursion.

A tree with \PB{\|n} nodes will produce an \RNBPM/ with $n$ edges
in addition to the root edge.

\Y\B\4\X8:Global variables\X${}\mathrel+\E{}$\6
\&{step} ${}\\{chartstack}[\\{maxcodes}][\T{4}*\\{maxcodes}]{}$;\C{ (only the %
\PB{\\{first}} and \PB{\\{second}} fields of these entries are used) }\6
\&{step} ${}\\{tmpchart}[\T{4}*\\{maxcodes}];{}$\6
\&{int} ${}\\{stk}[\\{maxcodes}*\\{maxcodes}]{}$;\C{ stack of subtrees waiting
to be processed }\6
\&{int} \\{curbud};\C{ the bud whose tree is being mapped (see below) }\par
\fi

\M[1102 skew-ternary-calc.w]{42} The \PB{\\{rnbpm\_js}} routine constructs the
Jacquard--Schaeffer
\RNBPM/ for \PB{\\{chartstack}[\|s]}
with root edge~\PB{\|r}. A third parameter, \PB{\|h}, tells the current height
of the
auxiliary stack \PB{\\{stk}}.

The value of \PB{\\{stk}[\|h]} is also supposed to identify
the root of the skew ternary tree whose chart is in \PB{\\{chartstack}[\|s]}.
(The name of the root doesn't appear in the chart when the tree has
only one node, hence we need this extra contextual information.)

\Y\B\4\X11:Subroutines\X${}\mathrel+\E{}$\6
\&{void} \\{rnbpm\_js}(\&{int} \|s${},\39{}$\&{int} \|r${},\39{}$\&{int} \|h)\1%
\1\2\2\6
${}\{{}$\1\6
\&{register} \&{int} \|i${},{}$ \|j${},{}$ \|k${},{}$ \|l${},{}$ \|m${},{}$ %
\|p${},{}$ \|q${},{}$ \|t${},{}$ \\{tt}${},{}$ \\{apip}${},{}$ \\{steps};\7
\X43:Determine the number \PB{\|m} of initial rank~0 nodes, and stack them\X;\6
${}\\{apip}\K\\{pip}(\|r,\39\T{2}){}$;\C{ pip for attaching blocks }\6
\&{while} (\|m)\5
${}\{{}$\1\6
${}\|m\MM,\39\|t\K\\{stk}[\|h+\|m];{}$\6
\X44:Copy the tree $T$ underlying the next $T^*$ to \PB{$\\{chartstack}[\|s+%
\T{1}]$}\X;\6
\&{if} ${}(\|m\W\|l\G\T{0}\W(\\{chartstack}[\|s][\\{steps}].\\{first}\I\|t\V%
\\{chartstack}[\|s][\\{steps}].\\{second}\I\\{stk}[\|h+\|m-\T{1}])){}$\1\5
\\{confusion}(\.{"arc\ bracketing"});\2\6
${}\\{steps}\PP;{}$\6
\&{if} ${}(\|l<\T{0}){}$\1\5
\X45:Handle the case of an empty tree\X\2\6
\&{else}\1\5
\X46:Build the \RNBPM/ for \PB{$\\{chartstack}[\|s+\T{1}]$}\X;\2\6
\X47:Splice the new \RNBPM/ to the previous fragment\X;\6
\4${}\}{}$\2\6
\X48:Splice everything into a cycle\X;\6
\4${}\}{}$\2\par
\fi

\M[1132 skew-ternary-calc.w]{43}\B\X43:Determine the number \PB{\|m} of initial
rank~0 nodes, and stack them\X${}\E{}$\6
\&{if} ${}(\\{chartstack}[\|s][\T{0}].\\{second}){}$\1\5
\\{confusion}(\.{"no\ root\ bud"});\2\6
\&{for} ${}(\|m\K\T{1},\39\\{steps}\K\T{1};{}$  ; ${}\|m\PP){}$\5
${}\{{}$\1\6
\&{if} ${}(\\{chartstack}[\|s][\\{steps}\PP].\\{second}){}$\1\5
\\{confusion}(\.{"non\ skew"});\2\6
${}\\{stk}[\|h+\|m]\K\\{chartstack}[\|s][\\{steps}\PP].\\{second};{}$\6
\&{if} ${}(\\{stk}[\|h+\|m]\E\T{0}){}$\1\5
\&{break};\2\6
\4${}\}{}$\2\par
\U42.\fi

\M[1140 skew-ternary-calc.w]{44}At this point we're poised to look at the steps
of $T^*$, where
$T$ is the subtree that corresponds to edge \PB{$\|t\K\\{stk}[\|h+\|m]$}. If
$T^*$ is the
trivial one-edge \RNBPM/, we set \PB{$\|l\K{-}\T{1}$}; otherwise we set \PB{%
\|l} to the
number of nodes in $T^*$ that have rank~0 (which is also the number of buds
that have rank~$-1$).

In the second case, the subchart
$T^*$ is easily identified because it
begins with a downward step from \PB{\|t} to \PB{\\{tt}}
and ends with a downward step from \PB{\\{tt}} to \PB{\|t}. (These downward
steps occur first from rank~1 down to rank~0, then from rank~3 down
to rank~2; so they are reminiscent of the German text for quoted text,
which begins with \lower1ex\hbox{''} and ends with ``$\,$!)
We delete those steps and shift the others cyclically backward,
in order to deduce $T$ from $T^*$ as explained above.

\Y\B\4\X44:Copy the tree $T$ underlying the next $T^*$ to \PB{$\\{chartstack}[%
\|s+\T{1}]$}\X${}\E{}$\6
\&{if} ${}(\\{chartstack}[\|s][\\{steps}].\\{second}\E\T{0}){}$\1\5
${}\|l\K{-}\T{1};{}$\2\6
\&{else}\5
${}\{{}$\1\6
${}\\{tt}\K\\{chartstack}[\|s][\\{steps}].\\{second};{}$\6
\&{if} ${}(\\{chartstack}[\|s][\\{steps}\PP].\\{first}\I\|t){}$\1\5
\\{confusion}(\.{"wrong\ parent"});\2\6
${}\\{tmpchart}[\T{0}].\\{first}\K\\{tmpchart}[\T{0}].\\{second}\K\T{0}{}$;\C{
dummy bud }\6
\&{for} ${}(\|j\K\T{1},\39\|q\K\|l\K\T{0};{}$ ${}\\{chartstack}[\|s][%
\\{steps}].\\{second}\I\|t;{}$ ${}\\{steps}\PP,\39\|j\PP){}$\5
${}\{{}$\1\6
${}\\{tmpchart}[\|j]\K\\{chartstack}[\|s][\\{steps}];{}$\6
\&{if} ${}(\|q\E\T{2}){}$\1\5
${}\|k\K\|j{}$;\C{ remember the location of the last bud with rank 2 }\2\6
\&{else} \&{if} ${}(\|q\E{-}\T{1}){}$\1\5
${}\|l\PP{}$;\C{ count the buds of rank $-1$ }\2\6
\&{if} ${}(\\{tmpchart}[\|j].\\{second}){}$\1\5
${}\|q\MM{}$;\5
\2\&{else}\1\5
${}\|q\PP{}$;\C{ \PB{\|q} is the rank }\2\6
\4${}\}{}$\2\6
\&{if} ${}(\\{chartstack}[\|s][\\{steps}].\\{first}\I\\{tt}){}$\1\5
\\{confusion}(\.{"right\ bracket"});\2\6
\&{for} ${}(\|i\K\|k;{}$ ${}\|i<\|j;{}$ ${}\|i\PP){}$\1\5
${}\\{chartstack}[\|s+\T{1}][\|i-\|k]\K\\{tmpchart}[\|i];{}$\2\6
\&{for} ${}(\|i\K\T{0};{}$ ${}\|i<\|k;{}$ ${}\|i\PP){}$\1\5
${}\\{chartstack}[\|s+\T{1}][\|i+\|j-\|k]\K\\{tmpchart}[\|i];{}$\2\6
\4${}\}{}$\2\6
${}\\{steps}\PP{}$;\par
\U42.\fi

\M[1174 skew-ternary-calc.w]{45}The \RNBPM/ for an empty tree is simply the
unadorned root edge \PB{\|r}.
We've initialized that edge already (although I could have
initialized it here, instead).

\Y\B\4\X45:Handle the case of an empty tree\X${}\E{}$\6
$\|p\K\\{pip}(\|t,\39\T{1}){}$;\par
\U42.\fi

\M[1181 skew-ternary-calc.w]{46}There's a better way to do this step, because
we can identify the pip \PB{\|p}
directly while copying the chart. But I didn't have time to
stop and figure it out.

\Y\B\4\X46:Build the \RNBPM/ for \PB{$\\{chartstack}[\|s+\T{1}]$}\X${}\E{}$\6
${}\{{}$\1\6
${}\\{stk}[\|h+\|m]\K(\\{chartstack}[\|s+\T{1}][\T{2}].\\{second}\?%
\\{chartstack}[\|s+\T{1}][\T{2}].\\{first}:\\{chartstack}[\|s+\T{1}][\T{3}].%
\\{second}\?\\{chartstack}[\|s+\T{1}][\T{3}].\\{first}:\\{tt});{}$\6
${}\\{rnbpm\_js}(\|s+\T{1},\39\|t,\39\|h+\|m);{}$\6
\&{for} ${}(\|p\K\\{alphainv}[\\{pip}(\|t,\39\T{3})];{}$ \|l; ${}\|l\MM){}$\1\5
${}\|p\K\\{alphainv}[\|p];{}$\2\6
\4${}\}{}$\2\par
\U42.\fi

\M[1195 skew-ternary-calc.w]{47}\B\X47:Splice the new \RNBPM/ to the previous
fragment\X${}\E{}$\6
$\\{splice}(\\{irot}(\|p),\39\\{apip});{}$\6
${}\\{apip}\K\\{pip}(\|t,\39\T{2}){}$;\par
\U42.\fi

\M[1199 skew-ternary-calc.w]{48}\B\X48:Splice everything into a cycle\X${}\E{}$%
\6
$\\{splice}(\\{pip}(\|r,\39\T{0}),\39\\{apip}){}$;\par
\U42.\fi

\M[1202 skew-ternary-calc.w]{49}Okay, \PB{\\{rnbpm\_js}} is finished.
Here's how we apply it to each of the four skew ternary
trees of interest.

\Y\B\4\X49:Find and print the corresponding planar maps\X${}\E{}$\6
\&{for} ${}(\|j\K\T{0};{}$ ${}\|j<\T{4};{}$ ${}\|j\PP){}$\5
${}\{{}$\1\6
\\{printf}(\.{"---\ JS\ map\ for\ T"});\6
\&{for} ${}(\|i\K\T{0};{}$ ${}\|i<\|j;{}$ ${}\|i\PP){}$\1\5
\\{printf}(\.{"+"});\2\6
\\{printf}(\.{"\ ---\\n"});\6
\X31:Create the initial permutation\X;\6
\&{for} ${}(\|i\K\\{inputbud}[\\{stack}[\|j+\\{offset}-\T{2}]].\\{stepno},\39%
\|k\K\T{0};{}$ ${}\|i<\T{4}*\|n;{}$ ${}\|i\PP,\39\|k\PP){}$\1\5
${}\\{chartstack}[\T{0}][\|k]\K\\{chart}[\|i];{}$\2\6
\&{for} ${}(\|i\K\T{0};{}$ ${}\|i<\\{inputbud}[\\{stack}[\|j+\\{offset}-%
\T{2}]].\\{stepno};{}$ ${}\|i\PP,\39\|k\PP){}$\1\5
${}\\{chartstack}[\T{0}][\|k]\K\\{chart}[\|i];{}$\2\6
${}\\{stk}[\T{0}]\K\\{inputbud}[\\{stack}[\|j+\\{offset}-\T{2}]].\\{parent};{}$%
\6
${}\\{rnbpm\_js}(\T{0},\39\.{'*'},\39\T{0});{}$\6
\\{print\_alpha}(\,);\6
\4${}\}{}$\2\par
\A62.
\U5.\fi

\M[1221 skew-ternary-calc.w]{50}Unfortunately, I must report serious
disappointment with this correspondence
between \RNBPM/s and skew ternary trees, because its ``$T^*$ method'' of
keeping \PB{\|l} exterior nonroot edges of a sub-\RNBPM/ in the larger
\RNBPM/ actually keeps the {\it last\/} $l$ such edges, not the
edges that follow the root! Therefore the correspondence between
nodes and edges is extremely weird, and it doesn't have any apparent
significance for understanding the graph structure.

For example, the tree that corresponds to $(\dag)$ turns out to be
quite crazy:
$$\vcenter{\epsfbox{skew-ternary-calc.14}}$$
The nodes of rank zero are $a^1$, $b^1$, $c^1$, $d^1$, $d^3$, $c^5$,
$c^6$, $c^7$! They're equinumerous with the nonroot edges of $(\dag)$,
but that's almost the only good thing we can say about them.

The problem appears to be unfixable, because the rank~0 nodes in
$A^*$ are widely separated from node $a^1$.

\fi

\N[1239 skew-ternary-calc.w]{1}{51}Planar maps, conformemente Del Lungo et al.
The paper by Del Lungo, Del Ristoro, and Penaud presented a
completely different way to associate \RNBPM/s with skew ternary
trees, based on a completely different recursive decomposition.
We shall call it the DDP correspondence.

\def\join#1{\buildrel #1 \over\bowtie}
Instead of building an \RNBPM/ from $m$ other doubly labeled \RNBPM/s,
as in~$(\dag)$, the DDP correspondence relies on an interesting
{\it binary\/} operation `$\join c$', which forms an \RNBPM/ from
just two others, $S$ and $T$, where $S$ is doubly rooted but
$T$ is just singly rooted. The following picture illustrates
this operation:
$$\vcenter{\epsfbox{skew-ternary-calc.50}}\eqno(\ddag)$$
Here $S$ has three edges $a^1$, $a^2$, $a^3$ on its exterior
face, besides its root edge; and edge $a^2$ is the second root
(distinguishable by the fact that the first root edge has no name).
Similarly, $T$ has four exterior edges $b^1$, $b^2$, $b^3$, $b^4$,
and it is singly rooted. If the root edge of~$T$ runs from
vertex~$u$ to vertex~$v$, and if $S$'s main root points to
vertex~$s$ while its second root points to vertex~$w$,
the operation attaches the two \RNBPM/s by (i)~making $u$ and~$w$ coincide;
(ii)~erasing $T$'s root edge, and (iii)~introducing a new edge $c$ from
$s$ to~$v$.

The smallest cases of $\join c$ need special care: If $T$ consists simply
of its root edge, we simply add a new edge~$c$ from $s$ to~$w$.
If $S$ consists simply of its root edge, we consider that it is
doubly rooted with the second root the reverse of the first.
In the latter case the net effect is to take $T$ and split its
root into two pieces, the second of which is~$c$.

The green shading in $(\ddag)$, as in $(\dag)$, indicates that complicated
structure might exist within the interiors of $S$ and $T$.
Such structure is, however, irrelevant
as we continue to build larger structures. Notice that when $S$ isn't
simply a root edge, one face of~$T$ gains one or more edges
when $T$'s root edge is removed; but $T$'s interior structure
doesn't ``leak out.''

\fi

\M[1279 skew-ternary-calc.w]{52}This construction is equivalent to making three
splices in the
the quad-edge permutations that correspond to $S$ and $T$.
Let's assume that $S$'s root edge is called $r$,
so that its exterior face in this example is the cycle $(r_3a^3_3a^2_3a^1_3)$.
The auxiliary root edge is $a^2$; hence $w$ is the vertex cycle
that contains $a^2_0$. (If $S$ were simply the root edge~$r$,
its exterior face would be $(r_3r_1)$, and $w$ would be $(r_2)$.)
We may also assume that $T$'s root edge is called $c$; hence
its exterior face cycle is $(c_3b^4_3b^3_3b^2_3b^1_3)$,
$u=(c_2\ldots{})$, and $v=(c_0\ldots{})$.

\smallskip
Step (i) of the operation corresponds to splicing with $(c_2a^3_2)$;
this attaches $S$ to~$T$.
Step (ii) then corresponds to splicing with $(c_2x)$, where
$x=c_2\alpha$ is the first pip counterclockwise from $c$ in
the cycle for $u=w$. This leaves edge $c$ ``dangling'':
$$\hbox{\rm(i)}\quad\vcenter{\epsfbox{skew-ternary-calc.51}}\;;\qquad
\hbox{\rm(ii)}\quad\vcenter{\epsfbox{skew-ternary-calc.52}}$$
Finally, a splice with $(c_2a^1_2)$ produces $S\join c T$.

\fi

\M[1300 skew-ternary-calc.w]{53}The canonical permutation representations are
now different, because
edge labels are assigned in a different order.
Here, again for handy reference, is the new list for all cases
with at most three nonroot edges---showing also the skew ternary
trees that we are about to construct for them:
$$\def\\#1{$\vcenter{\medskip\epsfbox{skew-ternary-calc.9#1}\medskip}$}
\def\tree#1{\vcenter{\def\epsfsize##1##2{.8##1}%
\epsfbox{skew-ternary-calc.10#1}}}
\def\btree#1{\vcenter{\def\epsfsize##1##2{.8##1}%
\epsfbox{skew-ternary-calc.20#1}}}
\vcenter{\halign{\hfil\\#\hfil&\qquad$#$\hfil&\qquad$#$\cr
0&(r_0)(r_2)(r_3r_1)&\tree0\cr
1&(r_0a_2)(a_0r_2)(r_3a_3)(a_1r_1)&\tree1=\tree0\join A\tree0\cr
2&(r_0b_2a_2)(b_0r_2a_0)(r_3b_3)(b_1a_3)(a_1r_1)&
\tree2=\btree1\join B\tree0\cr
3&(r_0a_2)(a_0b_2)(b_0r_2)(r_3b_3a_3)(a_1b_1r_1)&
\tree3=\tree0\join A\btree0\cr
4&(r_0c_2b_2a_2)(r_2a_0b_0c_0)(r_3c_3)(c_1b_3)(b_1a_3)(a_1r_1)&
\tree4=\btree2\join C\tree0\cr
5&(r_0b_2a_2)(r_2a_0c_0)(b_0c_2)(r_3c_3b_3)(r_1a_1)(b_1c_1a_3)&
\tree5=\btree1\join B\btree5\cr
6&(r_0c_2a_2)(r_2b_0c_0)(a_0b_2)(r_3c_3)(r_1a_1b_1)(c_1b_3a_3)&
\tree6=\btree4\join C\tree0\cr
7&(r_0c_2a_2)(c_0b_2a_0)(b_0r_2)(r_3b_3c_3)(c_1a_3)(a_1b_1r_1)&
\tree7=\btree3\join C\tree0\cr
8&(r_0a_2)(a_0b_2c_2)(r_2c_0b_0)(r_3b_3a_3)(b_1c_3)(a_1c_1r_1)&
\tree8=\tree0\join A\btree6\cr
9&(r_0a_2)(a_0b_2)(b_0c_2)(c_0r_2)(r_3c_3b_3a_3)(r_1a_1b_1c_1)&
\tree9=\tree0\join A\btree7\cr
}}$$

\fi

\M[1331 skew-ternary-calc.w]{54}The DDP correspondence between an \RNBPM/ $T$
and its
skew ternary tree $\widehat T$ is designed to have
two key properties, both of which can be observed in
the examples just shown: (1)~The nodes of rank~0, from top to bottom,
correspond to the nonroot edges that touch the root vertex, in
counterclockwise order. (2)~The buds of
rank~0 that follow the last node of rank~0, in preorder,
correspond to the nonroot edges of the exterior cycle, in counterclockwise
order.

And the recursive rule to define the correspondence is amazingly simple:
The simplest \RNBPM/ (which has nothing but a root edge) corresponds to the
empty tree (which has two buds, both of rank~0, only one of which is actually
considered to be meaningful in property (2)).
Otherwise the tree corresponding to $S\join c T$ is obtained by
(i)~finding $\widehat S$ and $\widehat T$;
(ii)~computing $\widehat T^+$ using the cyclic rotation operation
on skew ternary trees at the beginning of this program;
(iii)~replacing the rank~0 bud of $\widehat S$ that
corresponds to $S$'s second root by a new node $c$ whose right
child is $\widehat T^+$.

To invert this rule, notice that we can recover $S$, $T$, and~c
from the resulting skew ternary tree, because $c$ is the last
node of rank~0 (in preorder); then if $c$'s right subtree is
$R$, we have $R=\widehat T^+$, hence $\widehat T=R^-$; and $\widehat S$ is
obtained by removing $c$ and~$R$. From $\widehat S$ and $\widehat T$
we know $S$ and~$T$, recursively. And the second root
in~$S$ corresponds to the parent node of~$c$.

\fi

\M[1361 skew-ternary-calc.w]{55}Here then is a subroutine that implements what
was just said.
The \PB{\\{rnbpm\_ddp}} routine constructs the
\RNBPM/ for \PB{\\{chartstack}[\|s]}
with root edge~\PB{\|r}. The chart is followed by a special step
for which the \PB{\\{second}} field is \PB{\\{sentinel}} and the \PB{\\{first}}
field
is the name of the root.

\Y\B\4\X11:Subroutines\X${}\mathrel+\E{}$\6
\&{void} \\{rnbpm\_ddp}(\&{int} \|s${},\39{}$\&{int} \|r)\1\1\2\2\6
${}\{{}$\1\6
\&{register} \&{int} \|c${},{}$ \|i${},{}$ \|j${},{}$ \\{jj}${},{}$ \|k${},{}$ %
\|p${},{}$ \|q${},{}$ \\{rr}${},{}$ \|t${},{}$ \\{steps}${},{}$ \\{parent};\7
\X56:Find the last node, \PB{\|c}, in preorder that has rank~0, and its parent~%
\PB{\|p}\X;\6
\X57:Copy $R^-$ into \PB{$\\{chartstack}[\|s+\T{1}]$}, where $R$ is the right
subtree of~$c$\X;\6
\X58:Recursively build the \RNBPM/ for $T$\X;\6
\X59:Copy the rest of the tree into \PB{$\\{chartstack}[\|s+\T{1}]$}\X;\6
\X60:Recursively build the \RNBPM/ for $S$\X;\6
\X61:Hook everything together with three magic splices\X;\6
\4${}\}{}$\2\par
\fi

\M[1379 skew-ternary-calc.w]{56}The steps of the chart follow preorder.

\Y\B\4\X56:Find the last node, \PB{\|c}, in preorder that has rank~0, and its
parent~\PB{\|p}\X${}\E{}$\6
\&{if} ${}(\\{chartstack}[\|s][\T{0}].\\{second}){}$\1\5
\\{confusion}(\.{"no\ root\ bud"});\2\6
\&{for} ${}(\\{steps}\K\T{1},\39\|q\K{-}\T{1};{}$ ${}\\{chartstack}[\|s][%
\\{steps}].\\{second}\I\\{sentinel};{}$ ${}\\{steps}\PP){}$\5
${}\{{}$\1\6
\&{if} ${}(\|q\E{-}\T{1}){}$\1\5
${}\|j\K\\{steps};{}$\2\6
\&{if} ${}(\\{chartstack}[\|s][\\{steps}].\\{second}\E\T{0}){}$\1\5
${}\|q\PP{}$;\5
\2\&{else}\1\5
${}\|q\MM;{}$\2\6
\4${}\}{}$\2\6
\&{if} ${}(\|q\I\T{2}){}$\1\5
\\{confusion}(\.{"bad\ rank\ at\ end"});\2\6
${}\|c\K\\{chartstack}[\|s][\|j-\T{1}].\\{second};{}$\6
\&{if} ${}(\|c\E\T{0}){}$\5
${}\{{}$\C{ \PB{\|c} is the root of the charted tree }\1\6
\&{if} ${}(\|j\I\T{1}){}$\1\5
\\{confusion}(\.{"parentless\ rank\ -1\ }\)\.{bud\ not\ at\ beginning}\)\.{"});%
\2\6
${}\|c\K\\{chartstack}[\|s][\\{steps}].\\{first},\39\|p\K\T{0};{}$\6
\4${}\}{}$\5
\2\&{else}\1\5
${}\|p\K\\{chartstack}[\|s][\|j-\T{1}].\\{first};{}$\2\6
\&{if} ${}(\\{chartstack}[\|s][\|j+\T{1}].\\{second}){}$\1\5
\\{confusion}(\.{"not\ the\ last\ zero"});\2\par
\U55.\fi

\M[1395 skew-ternary-calc.w]{57}If $c$'s right child is just a bud, the subtree
$R$ is empty and
it corresponds to the empty tree. Otherwise $R$ is bracketed in
the chart by arcs from $c$ to its root node and back again,
just as the subtree $T^*$ was bracketed in the procedure \PB{\\{rnbpm\_js}}.
In this case the copying task is simpler than it was before, because
we needn't count zeros.

\Y\B\4\X57:Copy $R^-$ into \PB{$\\{chartstack}[\|s+\T{1}]$}, where $R$ is the
right subtree of~$c$\X${}\E{}$\6
$\\{jj}\K\|j-\T{1},\39\\{steps}\K\|j+\T{2};{}$\6
${}\\{rr}\K\\{chartstack}[\|s][\\{steps}].\\{second};{}$\6
\&{if} (\\{rr})\5
${}\{{}$\C{ \PB{\\{rr}} is the root of a nonempty subtree $R$ }\1\6
\&{if} ${}(\\{chartstack}[\|s][\\{steps}\PP].\\{first}\I\|c){}$\1\5
\\{confusion}(\.{"wrong\ parent"});\2\6
${}\\{tmpchart}[\T{0}].\\{first}\K\\{tmpchart}[\T{0}].\\{second}\K\T{0}{}$;\C{
dummy bud }\6
\&{for} ${}(\|j\K\T{1},\39\|q\K\T{0};{}$ ${}\\{chartstack}[\|s][\\{steps}].%
\\{second}\I\|c;{}$ ${}\\{steps}\PP,\39\|j\PP){}$\5
${}\{{}$\1\6
${}\\{tmpchart}[\|j]\K\\{chartstack}[\|s][\\{steps}];{}$\6
\&{if} ${}(\|q\E\T{2}){}$\1\5
${}\|k\K\|j{}$;\C{ remember the location of the last bud with rank 2 }\2\6
\&{if} ${}(\\{tmpchart}[\|j].\\{second}){}$\1\5
${}\|q\MM{}$;\5
\2\&{else}\1\5
${}\|q\PP{}$;\C{ \PB{\|q} is the rank }\2\6
\4${}\}{}$\2\6
\&{if} ${}(\\{chartstack}[\|s][\\{steps}].\\{first}\I\\{rr}){}$\1\5
\\{confusion}(\.{"right\ bracket"});\2\6
\&{for} ${}(\|i\K\|k;{}$ ${}\|i<\|j;{}$ ${}\|i\PP){}$\1\5
${}\\{chartstack}[\|s+\T{1}][\|i-\|k]\K\\{tmpchart}[\|i]{}$;\C{ shift to $R^-$
}\2\6
\&{for} ${}(\|i\K\T{0};{}$ ${}\|i<\|k;{}$ ${}\|i\PP){}$\1\5
${}\\{chartstack}[\|s+\T{1}][\|i+\|j-\|k]\K\\{tmpchart}[\|i];{}$\2\6
${}\\{chartstack}[\|s+\T{1}][\|j].\\{second}\K\\{sentinel};{}$\6
${}\\{chartstack}[\|s+\T{1}][\|j].\\{first}\K(\\{chartstack}[\|s+\T{1}][\T{2}].%
\\{second}\?\\{chartstack}[\|s+\T{1}][\T{2}].\\{first}:\\{chartstack}[\|s+%
\T{1}][\T{3}].\\{second}\?\\{chartstack}[\|s+\T{1}][\T{3}].\\{first}:%
\\{rr});{}$\6
\4${}\}{}$\2\6
${}\\{steps}\PP{}$;\par
\U55.\fi

\M[1423 skew-ternary-calc.w]{58}\B\X58:Recursively build the \RNBPM/ for $T$%
\X${}\E{}$\6
\&{if} (\\{rr})\1\5
${}\\{rnbpm\_ddp}(\|s+\T{1},\39\|c){}$;\2\par
\U55.\fi

\M[1426 skew-ternary-calc.w]{59}If \PB{\|c} is the root of the tree, then \PB{%
\|p} is zero, subtree $\widehat S$
is empty, and nothing needs to be done. Otherwise the tree that
corresponds to $\widehat S$ is obtained by simply leaving out \PB{\|c} and $R$.
In the latter case, \PB{\\{jj}} points to the arc from \PB{\|p} to~\PB{\|c},
and \PB{\\{steps}}
points to the arc from \PB{\|c} back to~\PB{\|p}.

\Y\B\4\X59:Copy the rest of the tree into \PB{$\\{chartstack}[\|s+\T{1}]$}\X${}%
\E{}$\6
\&{if} (\|p)\5
${}\{{}$\1\6
\&{for} ${}(\|i\K\T{0};{}$ ${}\|i<\\{jj};{}$ ${}\|i\PP){}$\1\5
${}\\{chartstack}[\|s+\T{1}][\|i]\K\\{chartstack}[\|s][\|i];{}$\2\6
${}\\{chartstack}[\|s+\T{1}][\|i].\\{first}\K\\{chartstack}[\|s+\T{1}][\|i].%
\\{second}\K\T{0},\39\|i\PP{}$;\C{ bud for \PB{\|c} }\6
\&{for} ${}(\\{steps}\PP;{}$  ; ${}\\{steps}\PP,\39\|i\PP){}$\5
${}\{{}$\1\6
${}\\{chartstack}[\|s+\T{1}][\|i]\K\\{chartstack}[\|s][\\{steps}];{}$\6
\&{if} ${}(\\{chartstack}[\|s+\T{1}][\|i].\\{second}\E\\{sentinel}){}$\1\5
\&{break};\2\6
\4${}\}{}$\2\6
\4${}\}{}$\2\par
\U55.\fi

\M[1442 skew-ternary-calc.w]{60}\B\X60:Recursively build the \RNBPM/ for $S$%
\X${}\E{}$\6
\&{if} (\|p)\1\5
${}\\{rnbpm\_ddp}(\|s+\T{1},\39\|r){}$;\2\par
\U55.\fi

\M[1445 skew-ternary-calc.w]{61}Finally we obey the three-step splicing
protocol for $\join c$ that was
described above. Some tricky maneuvering is necessary in the degenerate
cases.

\Y\B\4\X61:Hook everything together with three magic splices\X${}\E{}$\6
\&{if} ${}(\\{rr}\E\T{0}){}$\5
${}\{{}$\C{ $\widehat T$ is empty }\1\6
\&{if} (\|p)\1\5
${}\\{splice}(\\{pip}(\|c,\39\T{0}),\39\\{alpha}[\\{pip}(\|p,\39\T{0})]);{}$\2\6
\&{else}\1\5
${}\\{splice}(\\{pip}(\|c,\39\T{0}),\39\\{pip}(\|r,\39\T{2}));{}$\2\6
\4${}\}{}$\5
\2\&{else}\5
${}\{{}$\1\6
\&{if} (\|p)\1\5
${}\\{splice}(\\{pip}(\|c,\39\T{2}),\39\\{alpha}[\\{pip}(\|p,\39\T{0})]);{}$\2\6
\&{else}\1\5
${}\\{splice}(\\{pip}(\|c,\39\T{2}),\39\\{pip}(\|r,\39\T{2}));{}$\2\6
${}\\{splice}(\\{pip}(\|c,\39\T{2}),\39\\{alpha}[\\{pip}(\|c,\39\T{2})]);{}$\6
${}\\{verts}\MRL{+{\K}}\T{2}{}$;\C{ because we spliced two pips from the same
vertex }\6
\4${}\}{}$\2\6
${}\\{splice}(\\{pip}(\|c,\39\T{2}),\39\\{irot}(\\{alphainv}[\\{pip}(\|r,\39%
\T{3})])){}$;\par
\U55.\fi

\M[1461 skew-ternary-calc.w]{62}Okay, \PB{\\{rnbpm\_ddp}} is finished.
Here's how we apply it to each of the four skew ternary
trees of interest.

\Y\B\4\X49:Find and print the corresponding planar maps\X${}\mathrel+\E{}$\6
\&{for} ${}(\|j\K\T{0};{}$ ${}\|j<\T{4};{}$ ${}\|j\PP){}$\5
${}\{{}$\1\6
\\{printf}(\.{"---\ DDP\ map\ for\ T"});\6
\&{for} ${}(\|i\K\T{0};{}$ ${}\|i<\|j;{}$ ${}\|i\PP){}$\1\5
\\{printf}(\.{"+"});\2\6
\\{printf}(\.{"\ ---\\n"});\6
\X31:Create the initial permutation\X;\6
\&{for} ${}(\|i\K\\{inputbud}[\\{stack}[\|j+\\{offset}-\T{2}]].\\{stepno},\39%
\|k\K\T{0};{}$ ${}\|i<\T{4}*\|n;{}$ ${}\|i\PP,\39\|k\PP){}$\1\5
${}\\{chartstack}[\T{0}][\|k]\K\\{chart}[\|i];{}$\2\6
\&{for} ${}(\|i\K\T{0};{}$ ${}\|i<\\{inputbud}[\\{stack}[\|j+\\{offset}-%
\T{2}]].\\{stepno};{}$ ${}\|i\PP,\39\|k\PP){}$\1\5
${}\\{chartstack}[\T{0}][\|k]\K\\{chart}[\|i];{}$\2\6
${}\\{chartstack}[\T{0}][\|k].\\{first}\K\\{inputbud}[\\{stack}[\|j+\\{offset}-%
\T{2}]].\\{parent};{}$\6
${}\\{chartstack}[\T{0}][\|k].\\{second}\K\\{sentinel};{}$\6
${}\\{rnbpm\_ddp}(\T{0},\39\.{'*'});{}$\6
\\{print\_alpha}(\,);\6
\4${}\}{}$\2\par
\fi

\M[1481 skew-ternary-calc.w]{63}Here, for instance, is the \RNBPM/ that
corresponds to the example
skew ternary tree~$T$ at the very beginning of this program,
according to the DDP correspondence:
$$\vcenter{\epsfbox{skew-ternary-calc.110}}$$
The $\alpha$ permutation is
$$(e_2b_2a_2r_0)(f_0d_0e_0r_2)(c_0d_2f_2a_0)(c_2b_0)
(a_1f_1r_1)(e_3r_3)(b_1c_1a_3)(e_1d_3c_3b_3)(f_3d_1).$$

\fi

\M[1489 skew-ternary-calc.w]{64}One can show that the number of nodes of even
rank is the number
of interior faces in the DDP correspondence;
the number of nodes of odd rank is the number of vertices, minus~2.

(The actual rank of each vertex node and each face node
can in fact be ``read off'' from the \RNBPM/, if it is examined in an
appropriate depth-first search order, because of results
mentioned below.)

\fi

\M[1498 skew-ternary-calc.w]{65}Indeed, this program led to a huge surprise,
because much, much more
is true. In every case that I had examined by hand, in my first explorations
of the DDP correspondence, I noticed that {\sl the four
\RNBPM/s obtained from $T$, $T^+$, $T^{++}$, and $T^{+++}$ are
dual graphs!} I wrote this program in order to check that conjecture
on examples that were too large to study reliably by hand; and I
found that the conjecture was always verified, even when I looked
at large random instances.

More precisely, if we call the four pip permutations
$\alpha_0$, $\alpha_1$, $\alpha_2$, and $\alpha_3$, when the DDP
correspondence has been applied to four conjugate skew ternary trees,
I found that $\alpha_k$ was equal to $\alpha_0\hat\rho^k$ in every
case that I computed. Here $\hat\rho$ is the permutation
$(r_1r_3)\rho(r_1r_3)$; it's like $\rho$ except that it {\it decreases\/}
subscripts of $r$ while increasing the subscripts of all the other
edges (modulo~4). For example, the alpha permutations for $T^+$, $T^{++}$, and
$T^{+++}$ are the following ``clones'' of the alpha permutation for
$T$ given above:
$$\displaylines{
(e_1b_1a_1r_1)(f_3d_3e_3r_3)(c_3d_1f_1a_3)(c_1b_3)
(a_0f_0r_2)(e_2r_0)(b_0c_0a_2)(e_0d_2c_2b_2)(f_2d_0)\cr
(e_0b_0a_0r_2)(f_2d_2e_2r_0)(c_2d_0f_0a_2)(c_0b_2)
(a_3f_3r_3)(e_1r_1)(b_3c_3a_1)(e_3d_1c_1b_1)(f_1d_3)\cr
(e_3b_3a_3r_3)(f_1d_1e_1r_1)(c_1d_3f_3a_1)(c_3b_1)
(a_2f_2r_0)(e_0r_2)(b_2c_2a_0)(e_2d_0c_0b_0)(f_0d_2)\cr
}$$

Evidence for that conjecture was overwhelming, so I asked several
experts for help. And it turned out, by extraordinary luck, that I had
chosen exactly the right person to ask, namely Gilles Schaeffer.
He told me that, after writing the paper with Jacquard that was
cited above, he continued to do research about
connections between trees and planar maps. The result was his
Ph.D. dissertation, {\sl Conjugation d'arbres et cartes combinatoires
al\'eatores\/} (l'Universit\'e Bordeaux~I, 1998); and when I downloaded
that thesis I found it to be an amazingly rich
compendium of deep new results, covering many topics in addition
to \RNBPM/s (most of which he chose not to publish elsewhere).
In particular, on pages 65--67 he sketched a $(2n+2)$-to-4 correspondence
between ternary trees and \RNBPM/s, which amounts to an independent
discovery of the DDP correspondence, although he did not explicitly
mention any connection with skew ternary trees.

Schaeffer's remarkable construction on those three pages explains everything:
It can be used to show
that {\sl Alice can actually construct the corresponding planar
graph ``online'' as she is walking around the tree!}

Namely, we can add four downward steps to the state chart, and
assign new labels to the steps, as follows (illustrated for the
skew ternary tree in the introduction):
$$\vcenter{\epsfbox{skew-ternary-calc.9}}$$
An upward step is labeled $x_i$, where $i$ is the rank (modulo~4)
at the beginning of the step
and $x$ is the name of the node attached to this bud. A downward step
is labeled $y_j$, where $j$ is the rank plus two (modulo~4) at the
beginning of the step and $y$ is the name of the node to which we
are descending. The final four steps are labeled $r_2$, $r_3$,
$r_0$, and $r_1$. Thus we've assigned $4n+4$ labels, one for each pip
in the permutation; and the pips have been matched up in pairs
$\{x_i,y_j\}$. Every such pair has the meaning that
$\alpha$ maps $x_i\mapsto y_{j-1}$ and $y_j\mapsto x_{i-1}$, where
the subscripts are treated modulo~4.

The validity of this rule is readily proved by induction, because
it is an {\it extremely\/} strong induction hypothesis. And
my conjecture about the way duality affects the permutations
$\alpha_k$ is an immediate consequence.

\fi

\M[1568 skew-ternary-calc.w]{66}Hey, we're finished---except for a final
routine, which we fondly hope
will never be invoked.

\Y\B\4\X66:Assertion failure subroutine\X${}\E{}$\6
\&{void} \\{confusion}(\&{char} ${}{*}\\{id}){}$\1\1\2\2\6
${}\{{}$\C{ an assertion has failed }\1\6
${}\\{fprintf}(\\{stderr},\39\.{"This\ can't\ happen\ (}\)\.{\%s)!\\n"},\39%
\\{id});{}$\6
${}\\{exit}({-}\T{666});{}$\6
\4${}\}{}$\2\par
\U5.\fi

\N[1577 skew-ternary-calc.w]{1}{67}Index.
\fi

\inx
\fin
\con
