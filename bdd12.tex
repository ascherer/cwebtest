\input cwebmac
\srcloctrue
\datethis


\N[4 bdd12.w]{1}{1}Intro. This program is the twelfth in a series of
exploratory studies by
which I'm attempting to gain first-hand experience with OBDD structures, as I
prepare Section 7.1.4 of {\sl The Art of Computer Programming}.
Again it's quite different from its predecessors: This one implements a
new approach to finding an optimum ordering for the variables of a given
Boolean function or set of functions.
The new approach is based on QDDs (quasi-reduced BDDs),
which undergo the ``jump-up'' operation but not the normal
synthesis operations of a traditional BDD package. It requires no
hashing or garbage collection.

The given function is specified explicitly by generating its QDD.
As a demonstration, we implement the $2^m$-way multiplexer
$M_m(x_1,\ldots,x_m;x_{m+1},\ldots,x_{m+2^m})$ here,
because it exhibits great extremes of BDD size under different orderings.
But with \.{CWEB} change files any other function can be substituted.

If desired, optimization will be restricted to a subrange of the possible
levels, keeping variables at the top and bottom of the BDD in place.
We assume that \PB{$\\{botvar}-\\{topvar}$} is at most 24.

\Y\B\4\D$\\{mm}$ \5
\T{2}\C{ order of MUX in this demonstration version }\par
\B\4\D$\\{nn}$ \5
$(\\{mm}+(\T{1}\LL\\{mm}){}$)\C{ the number of Boolean variables }\par
\B\4\D$\\{outs}$ \5
\T{1}\C{ the number of Boolean functions to be simultaneously optimized }\par
\B\4\D$\\{interval}$ \5
\T{1000}\C{ make this larger to suppress progress reports }\par
\B\4\D$\\{worksize}$ \5
$(\T{1}\LL\T{20}{}$)\C{ must be at most \PB{$\T{1}\LL\T{20}$} in this
implementation }\C{ the jump-up work area will have \PB{$\T{3}*\\{worksize}$}
octabytes }\par
\B\4\D$\\{topvar}$ \5
\T{1}\C{ first variable whose order will be varied (must be \PB{$\G$ \T{1}} }%
\par
\B\4\D$\\{botvar}$ \5
\\{nn}\C{ last variable whose order will be varied (must be \PB{$\Z$ \\{nn}} }%
\par
\B\4\D$\\{nnn}$ \5
$(\\{botvar}+\T{1}-\\{topvar}{}$)\C{ variables being permuted (at most 25) }\Y%
\par
\B\4\D$\|o$ \5
$\\{mems}\PP{}$\C{ count a memory access to an octabyte }\par
\B\4\D$\\{oo}$ \5
$\\{mems}\MRL{+{\K}}{}$\T{2}\C{ or two }\par
\B\4\D$\\{ooo}$ \5
$\\{mems}\MRL{+{\K}}{}$\T{3}\C{ or three }\par
\B\4\D$\\{oooo}$ \5
$\\{mems}\MRL{+{\K}}{}$\T{4}\C{ or four, wow }\par
\Y\B\8\#\&{include} \.{<stdio.h>}\6
\8\#\&{include} \.{<stdlib.h>}\6
\X2:Type definitions\X\6
\X3:Global variables\X\7
\&{unsigned} \&{long} \&{long} \\{mems};\7
\X7:Subroutines\X\7
\\{main}(\,)\1\1\2\2\6
${}\{{}$\1\6
\&{register} \&{int} \|h${},{}$ \|i${},{}$ \|j${},{}$ \|k${},{}$ \|l${},{}$ %
\\{lo}${},{}$ \\{hi}${},{}$ \\{jj}${},{}$ \\{kk}${},{}$ \\{var}${},{}$ %
\\{cycle};\6
\&{octa} \|x;\7
\X4:Initialize everything\X;\6
\&{for} ${}(\\{cycle}\K\T{1};{}$ ${}\\{cycle}<\T{1}\LL(\\{nnn}-\T{1});{}$ ${}%
\\{cycle}\PP){}$\5
${}\{{}$\1\6
\&{if} ${}(\\{cycle}\MOD\\{interval}\E\T{0}){}$\5
${}\{{}$\1\6
${}\\{printf}(\.{"Beginning\ cycle\ \%d\ }\)\.{(\%llu\ mems\ so\ far)\\n}\)%
\.{"},\39\\{cycle},\39\\{mems});{}$\6
\\{fflush}(\\{stdout});\6
\4${}\}{}$\2\6
\X21:Do the jump-up for the current \PB{\\{cycle}}\X;\6
\4${}\}{}$\2\6
\X27:Figure out an optimum order\X;\6
${}\\{printf}(\.{"Altogether\ \%llu\ mem}\)\.{s.\\n"},\39\\{mems});{}$\6
\4${}\}{}$\2\par
\fi

\M[62 bdd12.w]{2}\B\X2:Type definitions\X${}\E{}$\6
\&{typedef} \&{unsigned} \&{long} \&{long} \&{octa};\C{ an octabyte }\par
\A8.
\U1.\fi

\N[65 bdd12.w]{1}{3}QDD representation. The quasi-reduced binary decision
decision
for a Boolean function of $n$ variables has $q_k$ nodes on level~$k$,
for $0\le k\le n$, one for every distinct subfunction that can arise
by hard-wiring constant values for the initial variables
$(x_1,\ldots,x_k)$. The sequence $(q_0,\ldots,q_n)$ is
called the function's {\it quasi-profile}.

The maximum value of $q_k$ is $\min(t2^k,2^{2^{n-k}})$ when there are
$t$ output functions of $n$ input variables. In particular,
the maximum quasi-profile when $n=25$ and $t=1$ is
$$(1,2,4,\ldots,2^{19},2^{20},2^{16},2^8,2^4,2^2,2);$$
its potentially biggest element, $2^{20}$, occurs when $k=20$.

In this implementation the nodes on level $k$ are numbered from 0 to
$q_k-1$, and we store them consecutively in a big array \PB{\\{node}[\|k]}.
Each node contains three fields, (\PB{\\{lo}},\PB{\\{hi}},\PB{\\{dep}}), packed
into a
64-bit word. The \PB{\\{lo}} and \PB{\\{hi}} fields, 20 bits each, point to
nodes
at level~$k+1$; the \PB{\\{dep}} field, which occupies the other 24 bits,
represents the set of variables on which this subfunction depends.
Level $n$ is special; it has the two ``sink'' nodes 0 and~1,
which are represented by \PB{$(\T{0},\T{0},\T{0})$} and \PB{$(\T{1},\T{1},%
\T{0})$}, respectively.

For example, suppose $n=3$ and $f(x_1,x_2,x_3)=(\bar x_1\land x_2)
\lor(x_1\land x_3)$. The nodes at level~2, which correspond to
branching on~$x_3$, are
\PB{$\\{node}[\T{2}][\T{0}]\K(\T{0},\T{0},\T{0})$},
\PB{$\\{node}[\T{2}][\T{1}]\K(\T{1},\T{1},\T{0})$}, and
\PB{$\\{node}[\T{2}][\T{2}]\K(\T{0},\T{1},\T{\^4})$}, representing the
respective subfunctions 0, 1,~$x_3$.
The nodes at level~1, which correspond to branching on~$x_2$, are
\PB{$\\{node}[\T{1}][\T{0}]\K(\T{0},\T{1},\T{\^2})$} and
\PB{$\\{node}[\T{1}][\T{1}]\K(\T{2},\T{2},\T{\^4})$}, representing the
subfunctions $x_2$ and $x_3$.
And there's one node at level~0, namely
\PB{$\\{node}[\T{0}][\T{0}]\K(\T{0},\T{1},\T{\^7})$}.

\Y\B\4\X3:Global variables\X${}\E{}$\6
\&{octa} ${}{*}\\{node}[\\{nnn}+\T{1}]{}$;\C{ the node arrays }\6
\&{int} ${}\\{qq}[\\{nnn}+\T{1}]{}$;\C{ the quasi-profile: $\PB{%
\\{qq}}[k]=q_{k-topvar+1}$ }\par
\As9, 13, 15, 20, 24\ETs28.
\U1.\fi

\M[103 bdd12.w]{4}To launch this structure, we first need to allocate the \PB{%
\\{node}} arrays.
They are actually created only for levels \PB{$\\{topvar}-\T{1}$} through \PB{%
\\{botvar}}.

\Y\B\4\X4:Initialize everything\X${}\E{}$\6
\&{for} ${}(\|k\K\T{0};{}$ ${}\|k\Z\\{nnn};{}$ ${}\|k\PP){}$\5
${}\{{}$\1\6
${}\|j\K\\{worksize},\39\\{kk}\K\|k+\\{topvar}-\T{1};{}$\6
\&{if} ${}(\\{nn}-\\{kk}<\T{5}\W\|j>\T{1}\LL(\T{1}\LL(\\{nn}-\\{kk}))){}$\1\5
${}\|j\K\T{1}\LL(\T{1}\LL(\\{nn}-\\{kk}));{}$\2\6
\&{if} ${}(\\{kk}<\T{20}\W((\\{worksize}/\\{outs})\GG\\{kk})>\T{0}\W(\|j>%
\\{outs}*(\T{1}\LL\\{kk}))){}$\1\5
${}\|j\K\\{outs}*(\T{1}\LL\\{kk});{}$\2\6
${}\\{node}[\|k]\K{}$(\&{octa} ${}{*}){}$ \\{malloc}${}(\|j*\&{sizeof}(%
\&{octa}));{}$\6
\&{if} ${}(\R\\{node}[\|k]){}$\5
${}\{{}$\1\6
${}\\{fprintf}(\\{stderr},\39\.{"I\ couldn't\ allocate}\)\.{\ \%d\ octabytes\
for\ no}\)\.{de[\%d]!\\n"},\39\|j,\39\|k);{}$\6
${}\\{exit}({-}\T{2});{}$\6
\4${}\}{}$\2\6
\4${}\}{}$\2\par
\As6, 16\ETs22.
\U1.\fi

\M[118 bdd12.w]{5}The packing policy is slightly unusual because I'm squeezing
65 bits
into 64. The \PB{\\{dep}} field represents the state of up to 25 variables,
but only 24 bits are available.

Not to worry: The least significant bit of a 25-bit \PB{\\{dep}} would truly
be of the least possible significance in every way, because it would be
zero everywhere except on level~0. And that bit on level~0 is almost
never examined.

Therefore I don't store the least significant bit of \PB{\\{dep}}. I store only
the value \PB{$\\{dp}\K\\{dep}\GG\T{1}$}, and I treat level~0 as a special
case.
This saves time even when $n<25$.

(If I wanted to handle more than 25 variables, I could use a more
elaborate scheme in which the packing conventions vary from level to level.
But 25 is plenty for me at the moment.)

\Y\B\4\D$\\{pack}(\\{lo},\\{hi},\\{dp})$ \5
$(((((\&{octa})(\\{dp})\LL\T{20})+(\\{lo}))\LL\T{20})+(\\{hi}){}$)\par
\B\4\D$\\{lofield}(\|x)$ \5
$(((\|x)\GG\T{20})\AND\T{\^fffff}{}$)\par
\B\4\D$\\{hifield}(\|x)$ \5
$((\|x)\AND\T{\^fffff}{}$)\par
\B\4\D$\\{depfield}(\|x)$ \5
$(((\|x)\GG\T{40})\LL\T{1}{}$)\par
\B\4\D$\\{extrabit}(\|k,\|j)$ \5
$(\|k\E\T{0}\?\\{lofield}(\\{node}[\T{0}][\|j])\I\\{hifield}(\\{node}[\T{0}][%
\|j]):\T{0}{}$)\par
\fi

\M[141 bdd12.w]{6}Once the \PB{\\{node}} arrays exist, we can set up the
initial QDD.
Here I implement the familiar function
$$M_m(x_1,\ldots,x_m;x_{m+1},\ldots,x_{m+2^m})\;=\;
x_{m+1+(x_1\ldots x_m)_2}.$$
(In fact, the example $f(x_1,x_2,x_3)$ above is $M_1(x_1;x_2,x_3)$.)

Only the \PB{\\{lo}} and \PB{\\{hi}} fields are initialized here, because we
will use
the reduction routine to compute appropriate \PB{\\{dep}}s.

The following code assumes that \PB{$\\{topvar}\K\T{1}$} and \PB{$\\{botvar}\K%
\\{nn}$}.

\Y\B\4\X4:Initialize everything\X${}\mathrel+\E{}$\6
\&{for} ${}(\|k\K\T{0};{}$ ${}\|k<\\{mm};{}$ ${}\|k\PP){}$\5
${}\{{}$\1\6
\&{for} ${}(\|j\K\T{0};{}$ ${}\|j<\T{1}\LL\|k;{}$ ${}\|j\PP){}$\1\5
${}\\{node}[\|k][\|j]\K\\{pack}(\|j+\|j,\39\|j+\|j+\T{1},\39\T{0});{}$\2\6
${}\\{qq}[\|k]\K\T{1}\LL\|k;{}$\6
\4${}\}{}$\2\6
\&{for} ${}(\|j\K\T{0};{}$ ${}\|j<\T{1}\LL\\{mm};{}$ ${}\|j\PP){}$\1\6
\&{if} ${}(\|j\E\T{0}){}$\1\5
${}\\{node}[\\{mm}][\|j]\K\\{pack}(\T{0},\39\T{1},\39\T{0});{}$\2\6
\&{else}\1\5
${}\\{node}[\\{mm}][\|j]\K\\{pack}(\|j+\T{1},\39\|j+\T{1},\39\T{0});{}$\2\2\6
${}\\{qq}[\\{mm}]\K\T{1}\LL\\{mm};{}$\6
\&{for} ${}(\|k\K\\{mm}+\T{1};{}$ ${}\|k\Z\\{nn};{}$ ${}\|k\PP){}$\5
${}\{{}$\1\6
\&{for} ${}(\|j\K\T{0};{}$ ${}\|j<\\{nn}+\T{2}-\|k;{}$ ${}\|j\PP){}$\1\6
\&{if} ${}(\|j<\T{2}){}$\1\5
${}\\{node}[\|k][\|j]\K\\{pack}(\|j,\39\|j,\39\T{0});{}$\2\6
\&{else} \&{if} ${}(\|j\E\T{2}){}$\1\5
${}\\{node}[\|k][\|j]\K\\{pack}(\T{0},\39\T{1},\39\T{0});{}$\2\6
\&{else}\1\5
${}\\{node}[\|k][\|j]\K\\{pack}(\|j-\T{1},\39\|j-\T{1},\39\T{0});{}$\2\2\6
${}\\{qq}[\|k]\K\\{nn}+\T{2}-\|k;{}$\6
\4${}\}{}$\C{ N.B.: now \PB{$\\{qq}[\\{nn}]\K\T{2}$}, and the sink nodes have
been initialized }\2\6
\X14:Compute the \PB{\\{dep}} fields of the initial QDD\X;\par
\fi

\M[172 bdd12.w]{7}For diagnostic purposes, I might want to pretty-print the
current QDD.

\Y\B\4\X7:Subroutines\X${}\E{}$\6
\&{void} \\{print\_level}(\&{int} \|k)\1\1\2\2\6
${}\{{}$\1\6
\&{register} \&{int} \|j;\7
${}\\{printf}(\.{"level\ \%d\ (x\%d):\\n"},\39\|k,\39\\{map}[\|k]+%
\\{topvar});{}$\6
\&{for} ${}(\|j\K\T{0};{}$ ${}\|j<\\{qq}[\|k];{}$ ${}\|j\PP){}$\1\5
${}\\{printf}(\.{"\ \%d,\%d\ \%x\\n"},\39\\{lofield}(\\{node}[\|k][\|j]),\39%
\\{hifield}(\\{node}[\|k][\|j]),\39\\{depfield}(\\{node}[\|k][\|j])+%
\\{extrabit}(\|k,\39\|j));{}$\2\6
\4${}\}{}$\2\7
\&{void} \\{print\_qbdd}(\&{void})\1\1\2\2\6
${}\{{}$\1\6
\&{register} \&{int} \|k;\7
\&{for} ${}(\|k\K\\{topvar}-\T{1};{}$ ${}\|k<\\{botvar};{}$ ${}\|k\PP){}$\1\5
\\{print\_level}(\|k);\2\6
\4${}\}{}$\2\par
\As10\ET11.
\U1.\fi

\N[189 bdd12.w]{1}{8}Reduction. Restructurings of the QDD are implemented with
the help
of a work area, capable of holding twice as many nodes
as needed on any particular level.

In the work area, nodes appear in octabytes of type \PB{\\{pair}}, which
simply have two integer fields, \PB{\|l} and~\PB{\|r}.

Two large arrays, \PB{\\{work}} and \PB{\\{head}}, comprise the work area.
A node with fields \PB{$(\\{lo},\\{hi})$} is represented in a \PB{\\{pair}}
that has
\PB{$\|l\K\\{hi}$}; this pair, which appears in the \PB{\\{work}} array,
is part of a linked list that begins \PB{\\{head}[\\{lo}]}. Links appear
in the \PB{\|r} fields.

The work area is intended for nodes that are destined to become
part of \PB{\\{node}[\|k]}, for some level~\PB{\|k}. The \PB{\\{lo}} and
\PB{\\{hi}} fields, which are addresses of nodes at the next level,
must therefore be less than \PB{$\\{qq}[\|k+\T{1}]$}.

\Y\B\4\X2:Type definitions\X${}\mathrel+\E{}$\6
\&{typedef} \&{struct} ${}\{{}$\5
\1\&{unsigned} \&{int} \|l${},{}$ \|r;\5
\2${}\}{}$ \&{pair};\par
\fi

\M[210 bdd12.w]{9}\B\X3:Global variables\X${}\mathrel+\E{}$\6
\&{pair} ${}\\{work}[\T{2}*\\{worksize}],{}$ \\{head}[\\{worksize}];\C{ the big
workspace }\par
\fi

\M[213 bdd12.w]{10}We use \PB{$\|j+\T{1}$} to link to entry \PB{\|j}, so that
null links are
distinguishable from links to entry~0.

\Y\B\4\X7:Subroutines\X${}\mathrel+\E{}$\6
\&{void} \\{print\_work}(\&{int} \|k)\1\1\2\2\6
${}\{{}$\1\6
\&{register} \&{int} \\{lo}${},{}$ \\{hi};\7
${}\\{printf}(\.{"Current\ workspace\ f}\)\.{or\ level\ \%d:\\n"},\39\|k);{}$\6
\&{for} ${}(\\{lo}\K\T{0};{}$ ${}\\{lo}<\\{qq}[\|k+\T{1}];{}$ ${}\\{lo}\PP){}$%
\1\6
\&{for} ${}(\\{hi}\K\\{head}[\\{lo}].\|r;{}$ \\{hi}; ${}\\{hi}\K\\{work}[%
\\{hi}-\T{1}].\|r){}$\1\5
${}\\{printf}(\.{"\ \%d,\%d\\n"},\39\\{lo},\39\\{work}[\\{hi}-\T{1}].\|l);{}$\2%
\2\6
\4${}\}{}$\2\par
\fi

\M[225 bdd12.w]{11}The key subroutine we need is called \PB{\\{reduce}}. It
finds the
unique combinations \PB{$(\\{lo},\\{hi})$} in the
work area, and places them into \PB{\\{node}[\|k]} with proper \PB{\\{dep}}
fields.
It resets \PB{\\{qq}[\|k]} to the number of distinct nodes found.

The \PB{\\{reduce}} routine also sets \PB{\\{clone}[\|j]} to the new address of
the
node that was originally represented in \PB{\\{work}[\|j]}.

``Bucket sorting'' is the basic idea here.
When we're working on list \PB{\\{head}[\\{lo}]}, we set \PB{$\\{head}[\\{hi}].%
\|l$} to
the address of an existing node \PB{$(\\{lo},\\{hi})$}, if \PB{\\{reduce}} has
already created such a node.

\Y\B\4\X7:Subroutines\X${}\mathrel+\E{}$\6
\&{void} \\{reduce}(\&{int} \|k)\1\1\2\2\6
${}\{{}$\1\6
\&{register} \&{int} \\{lo}${},{}$ \\{hi}${},{}$ \\{dep}${},{}$ \|p${},{}$ %
\\{nextp}${},{}$ \|q;\7
${}\|q\K\T{0}{}$;\C{ the number of nodes created so far }\6
\&{for} ${}(\|o,\39\\{lo}\K\T{0};{}$ ${}\\{lo}<\\{qq}[\|k+\T{1}];{}$ ${}\\{lo}%
\PP){}$\5
${}\{{}$\1\6
\&{for} ${}(\|o,\39\|p\K\\{head}[\\{lo}].\|r;{}$ \|p; ${}\|p\K\\{nextp}){}$\5
${}\{{}$\1\6
${}\|o,\39\\{nextp}\K\\{work}[\|p-\T{1}].\|r;{}$\6
${}\\{hi}\K\\{work}[\|p-\T{1}].\|l{}$;\C{ the \PB{\\{hi}} field of a node with
the current \PB{\\{lo}} }\6
\&{if} ${}(\|o,\39\\{head}[\\{hi}].\|l){}$\1\5
${}\|o,\39\\{clone}[\|p-\T{1}]\K\\{head}[\\{hi}].\|l-\T{1}{}$;\C{ \PB{$(\\{lo},%
\\{hi})$} already exists }\2\6
\&{else}\1\5
\X12:Create a new entry in \PB{\\{node}[\|k]}\X;\2\6
\4${}\}{}$\2\6
\&{for} ${}(\|p\K\\{head}[\\{lo}].\|r;{}$ \|p; ${}\|p\K\\{nextp}){}$\5
${}\{{}$\1\6
${}\|o,\39\\{nextp}\K\\{work}[\|p-\T{1}].\|r,\39\\{hi}\K\\{work}[\|p-\T{1}].%
\|l;{}$\6
${}\|o,\39\\{head}[\\{hi}].\|l\K\T{0}{}$;\C{ clean up }\6
\4${}\}{}$\2\6
${}\|o,\39\\{head}[\\{lo}].\|r\K\T{0}{}$;\C{ reset the \PB{\\{lo}} list to
empty }\6
\4${}\}{}$\2\6
${}\|o,\39\\{qq}[\|k]\K\|q;{}$\6
\4${}\}{}$\2\par
\fi

\M[259 bdd12.w]{12}\B\X12:Create a new entry in \PB{\\{node}[\|k]}\X${}\E{}$\6
${}\{{}$\1\6
\&{if} ${}(\|q\G\\{worksize}){}$\5
${}\{{}$\1\6
${}\\{fprintf}(\\{stderr},\39\.{"Sorry,\ level\ \%d\ of\ }\)\.{the\ QDD\ is\
too\ big\ (}\)\.{worksize=\%d)!\\n"},\39\|k,\39\\{worksize});{}$\6
${}\\{exit}({-}\T{3});{}$\6
\4${}\}{}$\2\6
\&{if} ${}(\\{lo}\I\\{hi}){}$\1\5
${}\\{oo},\39\\{dep}\K\\{depfield}(\\{node}[\|k+\T{1}][\\{lo}]\OR\\{node}[\|k+%
\T{1}][\\{hi}])\OR(\T{1}\LL\|k);{}$\2\6
\&{else}\1\5
${}\|o,\39\\{dep}\K\\{depfield}(\\{node}[\|k+\T{1}][\\{lo}]);{}$\2\6
${}\|o,\39\\{node}[\|k][\|q]\K\\{pack}(\\{lo},\39\\{hi},\39\\{dep}\GG\T{1});{}$%
\6
${}\|o,\39\\{clone}[\|p-\T{1}]\K\|q;{}$\6
${}\|o,\39\\{head}[\\{hi}].\|l\K\PP\|q;{}$\6
\4${}\}{}$\2\par
\U11.\fi

\M[274 bdd12.w]{13}\B\X3:Global variables\X${}\mathrel+\E{}$\6
\&{int} ${}\\{clone}[\T{2}*\\{worksize}]{}$;\C{ the new node addresses }\par
\fi

\M[277 bdd12.w]{14}Let's illustrate \PB{\\{reduce}} by using it to tidy up the
initial QDD.
The main point of interest is that we must remap the pointers on level~$k-1$
after level~$k$ has been reduced.

\Y\B\4\D$\\{makework}(\|j,\\{lo},\\{hi})$ \5
$\\{work}[\|j].\|l\K\\{hi},\39\\{work}[\|j].\|r\K\\{head}[\\{lo}].\|r,\39%
\\{head}[\\{lo}].\|r\K\|j+{}$\T{1}\par
\Y\B\4\X14:Compute the \PB{\\{dep}} fields of the initial QDD\X${}\E{}$\6
\&{for} ${}(\|k\K\\{nnn}-\T{1};{}$  ; ${}\|k\MM){}$\5
${}\{{}$\1\6
\&{for} ${}(\|o,\39\|j\K\T{0};{}$ ${}\|j<\\{qq}[\|k];{}$ ${}\|j\PP){}$\5
${}\{{}$\1\6
${}\|o,\39\\{lo}\K\\{lofield}(\\{node}[\|k][\|j]);{}$\6
${}\\{ooo},\39\\{makework}(\|j,\39\\{lo},\39\\{hifield}(\\{node}[\|k][%
\|j]));{}$\6
\4${}\}{}$\2\6
\\{reduce}(\|k);\6
\&{if} ${}(\|k\E\T{0}){}$\1\5
\&{break};\2\6
\&{for} ${}(\|o,\39\|j\K\T{0};{}$ ${}\|j<\\{qq}[\|k-\T{1}];{}$ ${}\|j\PP){}$\5
${}\{{}$\1\6
${}\|o,\39\|x\K\\{node}[\|k-\T{1}][\|j];{}$\6
${}\\{ooo},\39\\{node}[\|k-\T{1}][\|j]\K\\{pack}(\\{clone}[\\{lofield}(\|x)],%
\39\\{clone}[\\{hifield}(\|x)],\39\T{0});{}$\6
\4${}\}{}$\2\6
\4${}\}{}$\2\par
\U6.\fi

\N[297 bdd12.w]{1}{15}Jumping. Our chief goal is to study what happens when the
variables
of our function are permuted. We'll see later that only $2^{n-1}$
of the $n!$ permutations need to be examined, and that each of them
can be obtained from its predecessor by a simple ``jump-up'' operation.

The jump-up operation $k\to j$, where $k>j$, takes the variable
at level~$k$ of the branching structure and moves it up to level~$j$,
sliding the previous occupants of levels $(j,j+1,\ldots,k-1)$
down one notch. For example, suppose $n=7$ and we do jump-up $5\to2$.
Before the operation, levels 0 thru~6 represent branching on
variables $(x_1,x_2,x_3,x_4,x_5,x_6,x_7)$, respectively; after jumping, they
represent branching on $(x_1,x_2,x_6,x_3,x_4,x_5,x_7)$.

The \PB{\\{dep}} fields are always based on the assumption that level $k$
branches
on variable $x_{k+1}$. An auxiliary \PB{\\{map}} table records the results of
previous jumps; level~$k$ actually branches on variable
$x_{map[k]+topvar}$ of the original unpermuted function. We also maintain
a \PB{\\{bitmap}} array, where \PB{\\{bitmap}[\|k]} contains the bits that
encode the set
$\{\PB{\\{map}}[0],\ldots,\PB{\\{map}}[k-1]\}$.

\Y\B\4\X3:Global variables\X${}\mathrel+\E{}$\6
\&{int} \\{map}[\\{nnn}];\C{ the current permutation }\6
\&{int} \\{bitmap}[\\{nnn}];\C{ its initial dependency sets }\par
\fi

\M[321 bdd12.w]{16}\B\X4:Initialize everything\X${}\mathrel+\E{}$\6
\&{for} ${}(\|j\K\|k\K\T{0};{}$ ${}\|j<\\{nnn};{}$ ${}\|k\MRL{+{\K}}\T{1}\LL%
\|j,\39\|j\PP){}$\1\5
${}\|o,\39\\{map}[\|j]\K\|j,\39\\{bitmap}[\|j]\K\|k{}$;\2\par
\fi

\M[324 bdd12.w]{17}The basic idea for jumping up is quite simple. We
essentially make
two copies of levels $j$ thru~$k-1$, one for the case when $x_{k+1}=0$
and one for the case when $x_{k+1}=1$. Those copies are moved to
levels $j+1$ thru~$k$, and reduced to eliminate duplicates.
Finally, every original node on level $j$ is replaced by a node
that branches on $x_{k+1}$. The \PB{\\{lo}} branch of this new node is
to the 0-copy of the old node, and the \PB{\\{hi}} branch is to the 1-copy.

Because of the remapping that takes place here, the \PB{\\{dep}} fields
in levels less than~\PB{\\{jj}} are no longer correct. But we will do
jump-ups in a controlled manner, so that those \PB{\\{dep}} fields are
never actually examined.

\Y\B\4\X17:Jump up from \PB{\\{kk}} to \PB{\\{jj}}\X${}\E{}$\6
$\|o,\39\\{var}\K\\{map}[\\{kk}];{}$\6
\&{for} ${}(\|k\K\\{kk};{}$ ${}\|k>\\{jj};{}$ ${}\|k\MM){}$\5
${}\{{}$\1\6
\X18:Make two copies of level \PB{$\|k-\T{1}$} in the work area\X;\6
\\{reduce}(\|k);\6
${}\\{oo},\39\\{map}[\|k]\K\\{map}[\|k-\T{1}];{}$\6
${}\\{oo},\39\\{bitmap}[\|k]\K\\{bitmap}[\|k-\T{1}]\OR(\T{1}\LL\\{var});{}$\6
\4${}\}{}$\2\6
\X19:Remake level \PB{\\{jj}}\X;\par
\U21.\fi

\M[347 bdd12.w]{18}Simple but cool list processing does the trick here.

\Y\B\4\X18:Make two copies of level \PB{$\|k-\T{1}$} in the work area\X${}\E{}$%
\6
\&{for} ${}(\|o,\39\|j\K\T{0};{}$ ${}\|j<\\{qq}[\|k-\T{1}];{}$ ${}\|j\PP){}$\5
${}\{{}$\1\6
${}\|o,\39\|x\K\\{node}[\|k-\T{1}][\|j],\39\|l\K\\{lofield}(\|x),\39\|h\K%
\\{hifield}(\|x);{}$\6
\&{if} ${}(\|k\E\\{kk}){}$\5
${}\{{}$\1\6
${}\\{oo},\39\\{lo}\K\\{lofield}(\\{node}[\|k][\|l]),\39\\{hi}\K\\{lofield}(%
\\{node}[\|k][\|h]);{}$\6
${}\\{ooo},\39\\{makework}(\|j+\|j,\39\\{lo},\39\\{hi});{}$\6
${}\\{lo}\K\\{hifield}(\\{node}[\|k][\|l]),\39\\{hi}\K\\{hifield}(\\{node}[%
\|k][\|h]);{}$\6
${}\\{ooo},\39\\{makework}(\|j+\|j+\T{1},\39\\{lo},\39\\{hi});{}$\6
\4${}\}{}$\5
\2\&{else}\5
${}\{{}$\1\6
${}\\{oo},\39\\{lo}\K\\{clone}[\|l+\|l],\39\\{hi}\K\\{clone}[\|h+\|h];{}$\6
${}\\{ooo},\39\\{makework}(\|j+\|j,\39\\{lo},\39\\{hi});{}$\6
${}\\{oo},\39\\{lo}\K\\{clone}[\|l+\|l+\T{1}],\39\\{hi}\K\\{clone}[\|h+\|h+%
\T{1}];{}$\6
${}\\{ooo},\39\\{makework}(\|j+\|j+\T{1},\39\\{lo},\39\\{hi});{}$\6
\4${}\}{}$\2\6
\4${}\}{}$\2\par
\U17.\fi

\M[365 bdd12.w]{19}As mentioned earlier, we needn't worry about the integrity
of the
\PB{\\{dep}} fields in levels less than \PB{\\{jj}}.

\Y\B\4\X19:Remake level \PB{\\{jj}}\X${}\E{}$\6
\&{for} ${}(\|o,\39\|j\K\T{0};{}$ ${}\|j<\\{qq}[\\{jj}];{}$ ${}\|j\PP){}$\5
${}\{{}$\1\6
${}\\{oo},\39\\{lo}\K\\{clone}[\|j+\|j],\39\\{hi}\K\\{clone}[\|j+\|j+\T{1}];{}$%
\6
${}\\{ooo},\39\\{makework}(\|j,\39\\{lo},\39\\{hi});{}$\6
\4${}\}{}$\2\6
\\{reduce}(\\{jj});\6
${}\|o,\39\\{map}[\\{jj}]\K\\{var};{}$\6
\&{if} (\\{jj})\1\6
\&{for} ${}(\|o,\39\|j\K\T{0};{}$ ${}\|j<\\{qq}[\\{jj}-\T{1}];{}$ ${}\|j\PP){}$%
\5
${}\{{}$\1\6
${}\|o,\39\|x\K\\{node}[\\{jj}-\T{1}][\|j];{}$\6
${}\\{ooo},\39\\{node}[\\{jj}-\T{1}][\|j]\K\\{pack}(\\{clone}[\\{lofield}(%
\|x)],\39\\{clone}[\\{hifield}(\|x)],\39\T{0});{}$\6
\4${}\}{}$\2\2\par
\U17.\fi

\N[380 bdd12.w]{1}{20}The algorithm. OK, we've now got a beautiful
infrastructure to work with.
But the reader may well wonder why we've been building it.

Perhaps I should have presented this program in a top-down way,
starting with the statement of the problem (which is to find the optimum
ordering of variables) and then explaining how to reduce that problem
to list processing.

Well then, to start at the beginning, we can note that the main problem
can be reduced to smaller problems of the same kind. Namely,
for any given subset $X$ of the variables $\{x_1,\ldots,x_n\}$,
where $X$ has $k$ elements,
we can ask what ordering of those variables minimizes the profile of the
first $k$ levels, when those variables are required to
be tested first. (Here I'm talking about the real profile
$(b_0,\ldots,b_n)$, not the quasi-profile $(q_0,\ldots,q_n)$.)

If we've solved that problem for all subsets $X$ of size~$k<n$, we can
solve it for all subsets of size~$k+1$, as follows: Find a
QDD in which the elements of~$X$ appear on the levels 0 through~$k-1$.
For each $x\notin X$, count the number of elements on level~$k$
that depend on~$x$; that is the value of $b_k$ that will appear in
the profile of any BDD in which the elements of~$X$ are tested first.
Add that value to the minimum cost of $X$ on the previous levels,
thereby getting a candidate for the minimum cost of $X\cup x$.
After trying all $X$ and all~$x$, we'll know the minimum costs for
all $k+1$-element subsets.

Notice that every QDD has information about $n$ different subsets
at once, namely the subsets of elements on its initial $k$ levels
for $1\le k\le n$. Therefore, instead of working only on small
subsets before large ones, this program gathers data for all sizes
simultaneously; we can determine the actual minimum costs later.

(A completely different strategy, which uses branch-and-bound
methods to avoid subsets that are known to be nonoptimum,
will work better on many Boolean functions. I plan to implement that
method too, for comparison purposes. But the method considered
here is advantageous for the hard functions that can make branch-and-bound
examine too many branches.)

For each variable $x_{j+1}$ and each $k$-element set $X$ with
$x_{j+1}\notin X$, we will set \PB{\|b[\|X][\|j]} to the profile element $b_k$
that would occur when $x_{j+1}$ is at level~$k$ of a BDD
and when the variables of $X$ occupy levels
0 through~$k-1$. The subset \PB{\|X} is encoded as a binary number,
$\sum\{2^{i-1}\mid x_i\in X\}$.

\Y\B\4\X3:Global variables\X${}\mathrel+\E{}$\6
\&{int} ${}\|b[\T{1}\LL\\{nnn}][\\{nnn}]{}$;\C{ transition data for BDDs }\par
\fi

\M[431 bdd12.w]{21}Now I must explain a beautiful pattern: There's a simple way
to produce
all the QDDs we need, using a relatively short sequence of jump-ups.

In essence, we want a sequence of permutations with the
property that every $k$-element subset of $\{1,\ldots,n\}$ appears
as the first $k$ elements of some permutation, and such that we can
get from each permutation to its successor by a single jump-up.
There's a nice way to do this with a sequence of length $2^{n-1}$:
When $n=1$, the permutation is simply `1'. When $n>1$, take the
sequence for $n-1$ and place $n$ at the bottom; then jump $n$ up
to the top; then use the sequence for $n-1$ on the {\it lower\/}
$n-1$ elements. This idea turns out to be equivalent to the following:
The $k$th jump-up is $\nu k+\rho k\to\nu k-1$, for $1\le k<2^{n-1}$.
(In this formula
$\nu k$ denotes the number of 1s in the binary representation of~$k$,
and $\rho k$ denotes the number of 0s at the right end of that representation.)
For example, when $n=4$ the jumps are $1\to0$, $2\to0$, $2\to1$, $3\to0$,
$2\to1$, $3\to1$, $3\to2$; the permutations are
$$\thinmuskip=\thickmuskip
1234,2134,3214,3124,4312,4132,4213,4231.$$
The idea is that all $k$-combinations that don't contain~$n$ appear
in the first half; those that do contain~$n$ appear in the second half.

\Y\B\4\X21:Do the jump-up for the current \PB{\\{cycle}}\X${}\E{}$\6
\&{for} ${}(\\{jj}\K\T{0},\39\\{kk}\K\T{1},\39\|k\K\\{cycle};{}$ ${}(\|k\AND%
\T{1})\E\T{0};{}$ ${}\\{kk}\PP){}$\1\5
${}\|k\MRL{{\GG}{\K}}\T{1}{}$;\C{ compute $\rho k$ }\2\6
\&{for} ${}(\|k\MRL{\AND{\K}}\|k-\T{1};{}$ \|k; ${}\\{jj}\PP,\39\\{kk}\PP){}$\1%
\5
${}\|k\MRL{\AND{\K}}\|k-\T{1}{}$;\C{ compute $\nu k$ }\2\6
\X17:Jump up from \PB{\\{kk}} to \PB{\\{jj}}\X;\6
\&{for} ${}(\|k\K\\{jj}+\T{1};{}$ ${}\|k\Z\\{kk};{}$ ${}\|k\PP){}$\1\5
\X23:Gather statistics from level \PB{\|k} of the current QDD\X;\2\par
\U1.\fi

\M[461 bdd12.w]{22}We also need to gather statistics from the initial QDD.

\Y\B\4\X4:Initialize everything\X${}\mathrel+\E{}$\6
\X26:Gather statistics from level 0 of the current QDD\X;\6
\&{for} ${}(\|k\K\T{1};{}$ ${}\|k<\\{nnn};{}$ ${}\|k\PP){}$\1\5
\X23:Gather statistics from level \PB{\|k} of the current QDD\X;\2\par
\fi

\M[468 bdd12.w]{23}Every node in \PB{\\{node}[\|k]} is marked with a \PB{%
\\{dep}} field, telling which
of the remaining variables it depends on. More precisely, if the \PB{\\{dep}}
contains the bit \PB{$\T{1}\LL\|i$}, there's a dependency on $x_j$, where \PB{$%
\|j\K\\{map}[\|i]$}.

There might be lots and lots of nodes, so I'd like to examine each \PB{\\{dep}}
field only once. (In fact, I could have gathered the stats while doing
\PB{\\{reduce}} during the jump-up; so I won't charge any mems for
fetching the node here.) There are three bytes in the \PB{\\{dp}} field,
so I'll count how many nodes have each possible pattern in each
of those bytes. Usually only one or two of the bytes can be nonzero;
so the routine breaks into six rather tedious cases.

\Y\B\4\X23:Gather statistics from level \PB{\|k} of the current QDD\X${}\E{}$\6
${}\{{}$\1\6
\&{switch} ${}(((\\{nnn}-\T{2})\GG\T{3})*\T{4}+((\|k-\T{1})\GG\T{3})){}$\5
${}\{{}$\1\6
\4\&{case} \T{2}${}*\T{4}+\T{0}{}$:\5
\&{for} ${}(\|j\K\T{0};{}$ ${}\|j<\T{256};{}$ ${}\|j\PP){}$\1\5
${}\\{ooo},\39\\{count1}[\|j]\K\\{count2}[\|j]\K\\{count3}[\|j]\K\T{0};{}$\2\6
\&{for} ${}(\|o,\39\|j\K\T{0};{}$ ${}\|j<\\{qq}[\|k];{}$ ${}\|j\PP){}$\5
${}\{{}$\1\6
${}\|x\K\\{node}[\|k][\|j]{}$;\C{ mems not charged, see above }\6
${}\\{oo},\39\\{count1}[\|x\GG\T{56}]\PP;{}$\6
${}\\{oo},\39\\{count2}[(\|x\GG\T{48})\AND\T{\^ff}]\PP;{}$\6
${}\\{oo},\39\\{count3}[(\|x\GG\T{40})\AND\T{\^ff}]\PP;{}$\6
\4${}\}{}$\2\6
;\5
\&{break};\6
\4\&{case} \T{2}${}*\T{4}+\T{1}{}$:\5
\&{for} ${}(\|j\K\T{0};{}$ ${}\|j<\T{256};{}$ ${}\|j\PP){}$\1\5
${}\\{oo},\39\\{count1}[\|j]\K\\{count2}[\|j]\K\T{0};{}$\2\6
\&{for} ${}(\|o,\39\|j\K\T{0};{}$ ${}\|j<\\{qq}[\|k];{}$ ${}\|j\PP){}$\5
${}\{{}$\1\6
${}\|x\K\\{node}[\|k][\|j]{}$;\C{ mems not charged, see above }\6
${}\\{oo},\39\\{count1}[\|x\GG\T{56}]\PP;{}$\6
${}\\{oo},\39\\{count2}[(\|x\GG\T{48})\AND\T{\^ff}]\PP;{}$\6
\4${}\}{}$\2\6
;\5
\&{break};\6
\4\&{case} \T{2}${}*\T{4}+\T{2}{}$:\5
\&{for} ${}(\|j\K\T{0};{}$ ${}\|j<\T{256};{}$ ${}\|j\PP){}$\1\5
${}\|o,\39\\{count1}[\|j]\K\T{0};{}$\2\6
\&{for} ${}(\|o,\39\|j\K\T{0};{}$ ${}\|j<\\{qq}[\|k];{}$ ${}\|j\PP){}$\5
${}\{{}$\1\6
${}\|x\K\\{node}[\|k][\|j]{}$;\C{ mems not charged, see above }\6
${}\\{oo},\39\\{count1}[\|x\GG\T{56}]\PP;{}$\6
\4${}\}{}$\2\6
;\5
\&{break};\6
\4\&{case} \T{1}${}*\T{4}+\T{0}{}$:\5
\&{for} ${}(\|j\K\T{0};{}$ ${}\|j<\T{256};{}$ ${}\|j\PP){}$\1\5
${}\\{oo},\39\\{count2}[\|j]\K\\{count3}[\|j]\K\T{0};{}$\2\6
\&{for} ${}(\|o,\39\|j\K\T{0};{}$ ${}\|j<\\{qq}[\|k];{}$ ${}\|j\PP){}$\5
${}\{{}$\1\6
${}\|x\K\\{node}[\|k][\|j]{}$;\C{ mems not charged, see above }\6
${}\\{oo},\39\\{count2}[(\|x\GG\T{48})\AND\T{\^ff}]\PP;{}$\6
${}\\{oo},\39\\{count3}[(\|x\GG\T{40})\AND\T{\^ff}]\PP;{}$\6
\4${}\}{}$\2\6
;\5
\&{break};\6
\4\&{case} \T{1}${}*\T{4}+\T{1}{}$:\5
\&{for} ${}(\|j\K\T{0};{}$ ${}\|j<\T{256};{}$ ${}\|j\PP){}$\1\5
${}\|o,\39\\{count2}[\|j]\K\T{0};{}$\2\6
\&{for} ${}(\|o,\39\|j\K\T{0};{}$ ${}\|j<\\{qq}[\|k];{}$ ${}\|j\PP){}$\5
${}\{{}$\1\6
${}\|x\K\\{node}[\|k][\|j]{}$;\C{ mems not charged, see above }\6
${}\\{oo},\39\\{count2}[(\|x\GG\T{48})\AND\T{\^ff}]\PP;{}$\6
\4${}\}{}$\2\6
;\5
\&{break};\6
\4\&{case} \T{0}${}*\T{4}+\T{0}{}$:\5
\&{for} ${}(\|j\K\T{0};{}$ ${}\|j<\T{256};{}$ ${}\|j\PP){}$\1\5
${}\|o,\39\\{count3}[\|j]\K\T{0};{}$\2\6
\&{for} ${}(\|o,\39\|j\K\T{0};{}$ ${}\|j<\\{qq}[\|k];{}$ ${}\|j\PP){}$\5
${}\{{}$\1\6
${}\|x\K\\{node}[\|k][\|j]{}$;\C{ mems not charged, see above }\6
${}\\{oo},\39\\{count3}[(\|x\GG\T{40})\AND\T{\^ff}]\PP;{}$\6
\4${}\}{}$\2\6
;\5
\&{break};\6
\4${}\}{}$\2\6
\&{for} ${}(\|j\K\|k;{}$ ${}\|j<\\{nnn};{}$ ${}\|j\PP){}$\1\5
\X25:Gather stats for variable \PB{\|j} at level \PB{\|k}\X;\2\6
\4${}\}{}$\2\par
\Us21\ET22.\fi

\M[521 bdd12.w]{24}\B\X3:Global variables\X${}\mathrel+\E{}$\6
\&{int} \\{count1}[\T{256}]${},{}$ \\{count2}[\T{256}]${},{}$ \\{count3}[%
\T{256}];\C{ bit pattern counts }\par
\fi

\M[524 bdd12.w]{25}\B\X25:Gather stats for variable \PB{\|j} at level \PB{\|k}%
\X${}\E{}$\6
${}\{{}$\1\6
${}\|l\K\T{1}\LL((\|j-\T{1})\AND\T{\^7}),\39\|i\K\T{0};{}$\6
\&{if} ${}(\|j<\T{9}){}$\5
${}\{{}$\1\6
\&{for} ${}(\|i\K\T{0},\39\|h\K\|l;{}$ ${}\|h<\T{256};{}$ ${}\|h\K(\|h+\T{1})%
\OR\|l){}$\1\5
${}\|o,\39\|i\MRL{+{\K}}\\{count3}[\|h];{}$\2\6
\4${}\}{}$\5
\2\&{else} \&{if} ${}(\|j<\T{17}){}$\5
${}\{{}$\1\6
\&{for} ${}(\|i\K\T{0},\39\|h\K\|l;{}$ ${}\|h<\T{256};{}$ ${}\|h\K(\|h+\T{1})%
\OR\|l){}$\1\5
${}\|o,\39\|i\MRL{+{\K}}\\{count2}[\|h];{}$\2\6
\4${}\}{}$\5
\2\&{else}\1\6
\&{for} ${}(\|i\K\T{0},\39\|h\K\|l;{}$ ${}\|h<\T{256};{}$ ${}\|h\K(\|h+\T{1})%
\OR\|l){}$\1\5
${}\|o,\39\|i\MRL{+{\K}}\\{count1}[\|h];{}$\2\2\6
${}\\{oooo},\39\|b[\\{bitmap}[\|k]][\\{map}[\|j]]\K\|i;{}$\6
\4${}\}{}$\2\par
\U23.\fi

\M[535 bdd12.w]{26}At this point we have the initial QDD, with \PB{$\\{map}[%
\|j]\K\|j$} for all \PB{\|j}.

\Y\B\4\X26:Gather statistics from level 0 of the current QDD\X${}\E{}$\6
\&{for} ${}(\|j\K\T{0};{}$ ${}\|j<\\{qq}[\T{0}];{}$ ${}\|j\PP){}$\5
${}\{{}$\1\6
${}\|o,\39\|i\K\\{depfield}(\\{node}[\T{0}][\|j])+\\{extrabit}(\T{0},\39%
\|j);{}$\6
\&{for} ${}(\|k\K\T{0};{}$ ${}\|k<\\{nnn};{}$ ${}\|k\PP){}$\1\5
${}\|o,\39\|b[\T{0}][\|k]\MRL{+{\K}}(\|i\GG\|k)\AND\T{1};{}$\2\6
\4${}\}{}$\2\par
\U22.\fi

\M[543 bdd12.w]{27}At last comes the final reckoning: We can compute the
minimum cost
that can be achieved when subset \PB{\|X} occurs at the top of the BDD,
for $1\le X<2^n$.

While we're at it, we might as well compute the maximum cost too.

\Y\B\4\X27:Figure out an optimum order\X${}\E{}$\6
\&{for} ${}(\|k\K\T{1};{}$ ${}\|k<\T{1}\LL\\{nnn};{}$ ${}\|k\PP){}$\5
${}\{{}$\C{ \PB{\|k} represents a subset \PB{\|X} }\1\6
${}\|h\K\T{1}\LL\\{nnn}{}$;\C{ infinite cost }\6
\&{for} ${}(\|j\K\T{0},\39\|i\K\T{1};{}$ ${}\|j<\\{nnn};{}$ ${}\|j\PP,\39\|i%
\MRL{{\LL}{\K}}\T{1}){}$\1\6
\&{if} ${}(\|k\AND\|i){}$\5
${}\{{}$\1\6
${}\\{oo},\39\|l\K\\{cost}[\|k\XOR\|i]+\|b[\|k\XOR\|i][\|j]{}$;\C{ the cost if
$x_{j+1}$ comes last }\6
\&{if} ${}(\|l<\|h){}$\1\5
${}\|h\K\|l,\39\\{lo}\K\|j;{}$\2\6
\4${}\}{}$\2\2\6
${}\\{oo},\39\\{cost}[\|k]\K\|h,\39\\{routing}[\|k]\K\\{lo};{}$\6
\4${}\}{}$\2\6
${}\\{printf}(\.{"Optimum\ ordering\ (c}\)\.{ost\ \%d+externals)\ ca}\)\.{n\ be%
\ achieved\ thus:\\}\)\.{n"},\39\\{cost}[(\T{1}\LL\\{nnn})-\T{1}]);{}$\6
\&{for} ${}(\|j\K\\{nnn}-\T{1},\39\|k\K(\T{1}\LL\\{nnn})-\T{1};{}$ \|k; ${}\|j%
\MM,\39\|k\MRL{{\XOR}{\K}}\T{1}\LL\\{routing}[\|k]){}$\1\5
${}\\{printf}(\.{"\ level\ \%d,\ x\%d\ (\%d)}\)\.{\\n"},\39\|j,\39\\{routing}[%
\|k]+\\{topvar},\39\|b[\|k\XOR(\T{1}\LL\\{routing}[\|k])][\\{routing}[%
\|k]]);{}$\2\6
\&{for} ${}(\|k\K\T{1};{}$ ${}\|k<\T{1}\LL\\{nnn};{}$ ${}\|k\PP){}$\5
${}\{{}$\C{ \PB{\|k} represents a subset \PB{\|X} }\1\6
${}\|h\K\T{0};{}$\6
\&{for} ${}(\|j\K\T{0},\39\|i\K\T{1};{}$ ${}\|j<\\{nnn};{}$ ${}\|j\PP,\39\|i%
\MRL{{\LL}{\K}}\T{1}){}$\1\6
\&{if} ${}(\|k\AND\|i){}$\5
${}\{{}$\1\6
${}\|l\K\\{cost}[\|k\XOR\|i]+\|b[\|k\XOR\|i][\|j]{}$;\C{ the cost if $x_{j+1}$
comes last }\6
\&{if} ${}(\|l>\|h){}$\1\5
${}\|h\K\|l,\39\\{lo}\K\|j;{}$\2\6
\4${}\}{}$\2\2\6
${}\\{cost}[\|k]\K\|h,\39\\{routing}[\|k]\K\\{lo};{}$\6
\4${}\}{}$\2\6
${}\\{printf}(\.{"Pessimum\ ordering\ (}\)\.{cost\ \%d+externals)\ c}\)\.{an\
be\ achieved\ thus:}\)\.{\\n"},\39\\{cost}[(\T{1}\LL\\{nnn})-\T{1}]);{}$\6
\&{for} ${}(\|j\K\\{nnn}-\T{1},\39\|k\K(\T{1}\LL\\{nnn})-\T{1};{}$ \|k; ${}\|j%
\MM,\39\|k\MRL{{\XOR}{\K}}\T{1}\LL\\{routing}[\|k]){}$\1\5
${}\\{printf}(\.{"\ level\ \%d,\ x\%d\ (\%d)}\)\.{\\n"},\39\|j,\39\\{routing}[%
\|k]+\\{topvar},\39\|b[\|k\XOR(\T{1}\LL\\{routing}[\|k])][\\{routing}[%
\|k]]){}$;\2\par
\U1.\fi

\M[577 bdd12.w]{28}\B\X3:Global variables\X${}\mathrel+\E{}$\6
\&{int} ${}\\{cost}[\T{1}\LL\\{nnn}]{}$;\C{ the optimum node count for each
bitmap }\6
\&{char} ${}\\{routing}[\T{1}\LL\\{nnn}]{}$;\C{ the variable to put last in the
optimum order }\par
\fi

\N[581 bdd12.w]{1}{29}Index.
\fi

\inx
\fin
\con
