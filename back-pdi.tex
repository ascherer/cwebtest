\input cwebmac

\N{1}{1}Intro. This program finds all ``perfect digital invariants'' of
order~$m$
in the decimal system, namely all integers that satisfy $\pi_m x=x$,
where $\pi_m$ takes an integer into the sum of the $m$th powers of its digits.

It can be shown without difficulty that such integers have at most $m+1$
digits.
Indeed, if $10^p\le x<10^{p+1}$ we have $\pi_m x<10^p$ whenever $p>m$.
(The proof follows from the fact that $(m+1)9^m<10^{m+1}$.)

It's an interesting backtrack program, in which I successively choose the
digits $9\ge x_1\ge x_2\ge\cdots\ge x_{m+1}\ge0$ that will be the
digits of~$x$ (in some order). Lower bounds and upper bounds on $x$
are sufficiently sharp to rule out lots of cases before very many
of those digits have been specified. (And if $m$ is small,
I could even run through {\it all\/} such sequences of digits, because there
are only $m+10\choose 9$ of them. That's about 2.5 billion when $m=40$.)

The only high-precision arithmetic needed here is addition. I implement it
with binary-coded decimal representation (15 digits per octabyte),
using bitwise techniques as suggested in exercise 7.1.3--100.

Memory references (mems) are counted as if an optimizing compiler were
doing things like inlining subroutines, and as if the distribution arrays were
packed into a single octabyte. I actually keep the elements unpacked, to keep
debugging simple.

\Y\B\4\D$\\{maxm}$ \5
\T{1000}\par
\B\4\D$\\{maxdigs}$ \5
$(\T{1}+(\\{maxm}/\T{15}){}$)\C{ octabytes per binary-coded decimal number }\par
\B\4\D$\|o$ \5
$\\{mems}\PP{}$\par
\B\4\D$\\{oo}$ \5
$\\{mems}\MRL{+{\K}}{}$\T{2}\par
\Y\B\8\#\&{include} \.{<stdio.h>}\6
\8\#\&{include} \.{<stdlib.h>}\6
\&{int} \|m;\C{ command-line parameter }\6
\&{typedef} \&{unsigned} \&{long} \&{long} \&{ull};\6
\&{ull} \\{mems};\6
\&{ull} \\{nodes};\6
\&{ull} \\{thresh}${}\K\T{10000000000}{}$;\C{ reporting time }\6
\&{ull} ${}\\{profile}[\\{maxm}+\T{3}];{}$\6
\&{int} \\{count};\6
\&{int} \\{vbose};\C{ level of verbosity }\7
\X8:Global variables\X;\6
\X4:Subroutines\X;\7
\\{main}(\&{int} \\{argc}${},\39{}$\&{char} ${}{*}\\{argv}[\,]){}$\1\1\2\2\6
${}\{{}$\1\6
\&{register} \&{int} \|j${},{}$ \|k${},{}$ \|l${},{}$ \|p${},{}$ \|r${},{}$ %
\|t${},{}$ \\{pd}${},{}$ \\{alt}${},{}$ \\{blt}${},{}$ \\{xl}${},{}$ %
\\{change};\7
\X3:Process the command line\X;\6
\X7:Precompute the power tables\X;\6
\X11:Backtrack through all cases\X;\6
${}\\{fprintf}(\\{stderr},\39\.{"Altogether\ \%d\ solut}\)\.{ions\ for\ m=\%d\
(\%llu\ }\)\.{nodes,\ \%llu\ mems).\\n}\)\.{"},\39\\{count},\39\|m,\39%
\\{nodes},\39\\{mems});{}$\6
\&{if} (\\{vbose})\1\5
\X2:Print the profile\X;\2\6
\4${}\}{}$\2\par
\fi

\M{2}\B\X2:Print the profile\X${}\E{}$\6
${}\{{}$\1\6
${}\\{fprintf}(\\{stderr},\39\.{"Profile:\ \ \ \ \ \ \ \ \ \ 1}\)\.{\\n"});{}$\6
\&{for} ${}(\|k\K\T{2};{}$ ${}\|k\Z\|m+\T{2};{}$ ${}\|k\PP){}$\1\5
${}\\{fprintf}(\\{stderr},\39\.{"\%19lld\\n"},\39\\{profile}[\|k]);{}$\2\6
\4${}\}{}$\2\par
\U1.\fi

\M{3}\B\X3:Process the command line\X${}\E{}$\6
\&{if} ${}(\\{argc}<\T{2}\V\\{sscanf}(\\{argv}[\T{1}],\39\.{"\%d"},\39{\AND}%
\|m)\I\T{1}){}$\5
${}\{{}$\1\6
${}\\{fprintf}(\\{stderr},\39\.{"Usage:\ \%s\ m\ [profil}\)\.{e]\ [verbose]\
[extrav}\)\.{erbose]\\n"},\39\\{argv}[\T{0}]);{}$\6
${}\\{exit}({-}\T{1});{}$\6
\4${}\}{}$\2\6
${}\\{vbose}\K\\{argc}-\T{2};{}$\6
\&{if} ${}(\|m<\T{2}\V\|m>\\{maxm}){}$\5
${}\{{}$\1\6
${}\\{fprintf}(\\{stderr},\39\.{"Sorry,\ m\ should\ be\ }\)\.{between\ 2\ and\ %
\%d,\ no}\)\.{t\ \%d!\\n"},\39\\{maxm},\39\|m);{}$\6
${}\\{exit}({-}\T{2});{}$\6
\4${}\}{}$\2\6
${}\\{mdigs}\K\T{1}+(\|m/\T{15}){}$;\par
\U1.\fi

\N{1}{4}Tricky arithmetic. I've got to deal with biggish numbers and inspect
their
decimal digits. But I'm using a binary computer and I don't want to
be repeatedly dividing by powers of~10. So I have an addition routine
that computes (say) the sum of hexadecimal-coded numbers
\PB{\T{\^344159959}} and \PB{\T{\^271828043}}, giving \PB{\T{\^615988002}} as
if
the  numbers were decimal instead.

\Y\B\4\X4:Subroutines\X${}\E{}$\6
\&{void} \\{add}(\&{ull} ${}{*}\|p,\39{}$\&{ull} ${}{*}\|q,\39{}$\&{ull} ${}{*}%
\|r){}$\1\1\2\2\6
${}\{{}$\C{ add \PB{\|p} to \PB{\|q}, giving \PB{\|r} }\1\6
\&{register} \&{int} \|k${},{}$ \|c;\6
\&{register} \&{ull} \|t${},{}$ \|w${},{}$ \|x${},{}$ \|y;\7
\&{for} ${}(\|k\K\|c\K\T{0};{}$ ${}\|k<\\{mdigs};{}$ ${}\|k\PP){}$\1\5
\X5:Add \PB{$\|c+{*}(\|p+\|k)$} to \PB{${*}(\|q+\|k)$}, giving \PB{${*}(\|r+%
\|k)$} and carry~\PB{\|c}\X;\2\6
\&{if} (\|c)\5
${}\{{}$\C{ this shouldn't happen }\1\6
${}\\{fprintf}(\\{stderr},\39\.{"Overflow!\\n"});{}$\6
${}\\{exit}({-}\T{999});{}$\6
\4${}\}{}$\2\6
\4${}\}{}$\2\par
\As6\ET10.
\U1.\fi

\M{5}It's interesting that I must add \PB{\|c} to \PB{\|x} here, {\it not\/}
to~\PB{\|y}.
Otherwise the nondecimal digit \.{a} might appear in the result.

\Y\B\4\X5:Add \PB{$\|c+{*}(\|p+\|k)$} to \PB{${*}(\|q+\|k)$}, giving \PB{${*}(%
\|r+\|k)$} and carry~\PB{\|c}\X${}\E{}$\6
${}\{{}$\1\6
${}\|o,\39\|x\K{*}(\|p+\|k)+\|c{}$;\C{ \PB{\|x} might have a nondecimal digit
now }\6
${}\|o,\39\|y\K{*}(\|q+\|k)+\T{\^666666666666666}{}$;\C{ no cross-digit carries
occur }\6
${}\|t\K\|x+\|y;{}$\6
${}\|w\K(\|t\XOR\|x\XOR\|y)\AND\T{\^1111111111111110}{}$;\C{ this is where
cross-digit carries happen }\6
${}\|w\K(\|w\XOR\T{\^1111111111111110})\GG\T{3};{}$\6
${}\|t\MRL{-{\K}}\|w+(\|w\LL\T{1}){}$;\C{ subtract 6 where there were no
carries }\6
${}\|o,\39{*}(\|r+\|k)\K\|t\AND\T{\^fffffffffffffff};{}$\6
${}\|c\K\|t\GG\T{60};{}$\6
\4${}\}{}$\2\par
\U4.\fi

\M{6}At the beginning of this program, I need a table of $0^m$, $1^m$, $2^m$,
\dots,~$9^m$. So why not compute it via addition?

\Y\B\4\X4:Subroutines\X${}\mathrel+\E{}$\6
\&{void} \\{kmult}(\&{int} \|k${},\39{}$\&{ull} ${}{*}\|a){}$\1\1\2\2\6
${}\{{}$\C{ multiply \PB{\|a} by $k$ }\1\6
\&{switch} (\|k)\5
${}\{{}$\1\6
\4\&{case} \T{8}:\5
${}\\{add}(\|a,\39\|a,\39\|a);{}$\6
\4\&{case} \T{4}:\5
${}\\{add}(\|a,\39\|a,\39\|a);{}$\6
\4\&{case} \T{2}:\5
${}\\{add}(\|a,\39\|a,\39\|a){}$;\5
\&{break};\6
\4\&{case} \T{6}:\5
${}\\{add}(\|a,\39\|a,\39\|a);{}$\6
\4\&{case} \T{3}:\5
${}\\{add}(\|a,\39\|a,\39\|z){}$;\5
${}\\{add}(\|a,\39\|z,\39\|a){}$;\5
\&{break};\6
\4\&{case} \T{5}:\5
${}\\{add}(\|a,\39\|a,\39\|z){}$;\5
${}\\{add}(\|z,\39\|z,\39\|z){}$;\5
${}\\{add}(\|a,\39\|z,\39\|a){}$;\5
\&{break};\6
\4\&{case} \T{9}:\5
${}\\{add}(\|a,\39\|a,\39\|z){}$;\5
${}\\{add}(\|z,\39\|z,\39\|z){}$;\5
${}\\{add}(\|z,\39\|z,\39\|z){}$;\5
${}\\{add}(\|a,\39\|z,\39\|a){}$;\5
\&{break};\6
\4\&{case} \T{7}:\5
${}\\{add}(\|a,\39\|a,\39\|z){}$;\5
${}\\{add}(\|a,\39\|z,\39\|z){}$;\5
${}\\{add}(\|z,\39\|z,\39\|z){}$;\5
${}\\{add}(\|a,\39\|z,\39\|a){}$;\5
\&{break};\6
\4\&{case} \T{0}:\5
\&{case} \T{1}:\5
\&{break};\6
\4${}\}{}$\2\6
\4${}\}{}$\2\par
\fi

\M{7}\B\X7:Precompute the power tables\X${}\E{}$\6
\&{for} ${}(\|k\K\T{1};{}$ ${}\|k<\T{10};{}$ ${}\|k\PP){}$\5
${}\{{}$\1\6
${}\\{table}[\T{1}][\|k][\T{0}]\K\|k;{}$\6
\&{for} ${}(\|j\K\T{2};{}$ ${}\|j\Z\|m;{}$ ${}\|j\PP){}$\1\5
${}\\{kmult}(\|k,\39\\{table}[\T{1}][\|k]){}$;\C{ compute $k^m$ }\2\6
\&{for} ${}(\|j\K\T{2};{}$ ${}\|j\Z\|m+\T{1};{}$ ${}\|j\PP){}$\1\5
${}\\{add}(\\{table}[\T{1}][\|k],\39\\{table}[\|j-\T{1}][\|k],\39\\{table}[%
\|j][\|k]){}$;\C{ compute $j\cdot k^m$ }\2\6
\4${}\}{}$\2\par
\U1.\fi

\M{8}\B\X8:Global variables\X${}\E{}$\6
\&{int} \\{mdigs};\C{ our multiprecision arithmetic routine uses this many
octabytes }\6
\&{ull} ${}\\{table}[\\{maxm}+\T{2}][\T{10}][\\{maxdigs}]{}$;\C{ precomputed
tables of $j\cdot k^m$ }\6
\&{ull} \|z[\\{maxdigs}];\C{ temporary buffer for bignums }\par
\A14.
\U1.\fi

\M{9}Here's a macro that delivers a given digit (nybble) of a mutlibyte number.

\Y\B\4\D$\\{nybb}(\|a,\|p)$ \5
$(\&{int})((\|a[\|p/\T{15}]\GG(\T{4}*(\|p\MOD\T{15})))\AND\T{\^f}{}$)\par
\fi

\M{10}When debugging, or operating verbosely, I want to see all digits of a
multiprecise number, with a vertical bar just before digit number~\PB{\|t}.

\Y\B\4\X4:Subroutines\X${}\mathrel+\E{}$\6
\&{void} \\{printnum}(\&{ull} ${}{*}\|a,\39{}$\&{int} \|t)\1\1\2\2\6
${}\{{}$\1\6
\&{register} \&{int} \|k;\7
\&{for} ${}(\|k\K\|m;{}$ ${}\|k\G\T{0};{}$ ${}\|k\MM){}$\5
${}\{{}$\1\6
\&{if} ${}(\|t\E\|k){}$\1\5
${}\\{fprintf}(\\{stderr},\39\.{"|"});{}$\2\6
${}\\{fprintf}(\\{stderr},\39\.{"\%d"},\39\\{nybb}(\|a,\39\|k));{}$\6
\4${}\}{}$\2\6
\4${}\}{}$\2\par
\fi

\N{1}{11}The algorithm. This program has the overall structure of a typical
backtrack program, with a few twists. One of those twists is the
state parameter~\PB{\\{pd}}, which is nonzero when the move at level~\PB{$\|l-%
\T{1}$} was forced.
(Such cases are rare, but important.)

\Y\B\4\X11:Backtrack through all cases\X${}\E{}$\6
\4\\{b1}:\5
\X15:Initialize the data structures\X;\6
\4\\{b2}:\5
${}\\{profile}[\|l]\PP,\39\\{nodes}\PP;{}$\6
\X12:Report the current state, if \PB{$\\{mems}\G\\{thresh}$}\X;\6
\&{for} ${}(\|k\K\T{0};{}$ ${}\|k<\T{10};{}$ ${}\|k\PP){}$\5
${}\{{}$\1\6
${}\\{pdist}[\|l][\|k]\K\\{pdist}[\|l-\T{1}][\|k];{}$\6
${}\\{dist}[\|l][\|k]\K\\{dist}[\|l-\T{1}][\|k]+(\|k\E\\{xl}\?\T{1}:\T{0});{}$\6
\4${}\}{}$\2\6
${}\\{oo},\39\\{oo}{}$;\C{ two mems to copy \PB{\\{pdist}} and \PB{\\{dist}},
which could have been packed }\6
\&{if} (\\{pd})\1\5
\X22:Absorb a forced move\X\2\6
\&{else}\5
${}\{{}$\1\6
\&{if} ${}(\|r\E\T{0}){}$\1\5
\&{goto} \\{b5};\C{ we haven't room to accept a new digit \PB{\\{xl}} }\2\6
${}\|r\MM,\39\\{add}(\\{sig}[\|l-\T{1}],\39\\{table}[\T{1}][\\{xl}],\39\\{sig}[%
\|l]);{}$\6
\4${}\}{}$\2\6
\&{if} ${}(\|l>\|m+\T{1}){}$\1\5
\X17:Print a solution and \PB{\&{goto} \\{b5}}\X;\2\6
\4\\{b3}:\5
\&{if} ${}(\\{vbose}>\T{1}){}$\1\5
${}\\{fprintf}(\\{stderr},\39\.{"Level\ \%d,\ trying\ \%d}\)\.{\ (\%lld\ mems)%
\\n"},\39\|l,\39\\{xl},\39\\{mems});{}$\2\6
\X18:If there's an easy way to prove that $x_l$ can't be \PB{$\Z$ \\{xl}}, \PB{%
\&{goto} \\{b5}}\X;\6
\4\\{move}:\5
\X16:Advance to the next level with $x_l=\PB{\\{xl}}$ and \PB{\&{goto} \\{b2}}%
\X;\6
\4\\{b4}:\5
\&{if} (\\{xl})\5
${}\{{}$\1\6
${}\\{xl}\MM;{}$\6
${}\|o,\39\\{pd}\K\\{pdist}[\|l][\\{xl}]{}$;\C{ \PB{\\{dist}[\|l][\\{xl}]} was
zero }\6
\&{goto} \\{b3};\6
\4${}\}{}$\2\6
\4\\{b5}:\5
\&{if} ${}(\MM\|l){}$\5
${}\{{}$\1\6
${}\|o,\39\\{pd}\K\\{pdsave}[\|l];{}$\6
\&{if} (\\{pd})\1\5
\&{goto} \\{b5};\2\6
\X21:Restore the previous state at level \PB{\|l}\X;\6
\&{goto} \\{b4};\6
\4${}\}{}$\2\par
\U1.\fi

\M{12}\B\X12:Report the current state, if \PB{$\\{mems}\G\\{thresh}$}\X${}\E{}$%
\6
\&{if} ${}(\\{mems}\G\\{thresh}){}$\5
${}\{{}$\1\6
${}\\{thresh}\MRL{+{\K}}\T{10000000000};{}$\6
${}\\{fprintf}(\\{stderr},\39\.{"After\ \%lld\ mems:"},\39\\{mems});{}$\6
\&{for} ${}(\|k\K\T{2};{}$ ${}\|k\Z\|l;{}$ ${}\|k\PP){}$\1\5
${}\\{fprintf}(\\{stderr},\39\.{"\ \%lld"},\39\\{profile}[\|k]);{}$\2\6
${}\\{fprintf}(\\{stderr},\39\.{"\\n"});{}$\6
\4${}\}{}$\2\par
\U11.\fi

\M{13}The purpose of backtrack level \PB{\|l} is to compute the $l$th largest
digit,
$x_l$, of a solution~$x$, assuming that $x_1$, \dots,~$x_{l-1}$ have already
been specified.

The main idea is to compute bounds $a_l$ and $b_l$ such that $a_l\le x\le b_l$
must be valid, whenever $x_1$, \dots,~$x_{l-1}$ have the given values and
$x_l$ is at most a given threshold value~\PB{\\{xl}}. Those bounds, like all of
the
multiprecise numbers in this computation, are $(m+1)$-digit numbers whose
individual digits are $a_{lm}\ldots a_{l0}$ and $b_{lm}\ldots b_{l0}$.
They share a common prefix $p_m\ldots p_{t+1}$ of length $m+1-t$; thus
if $a_l<b_l$ we have $0\le t\le m$ and $a_{lt}<b_{lt}$.

The main point is that each of the digits in the multiset
$P=\{p_m,\ldots,p_{t+1}\}$ must appear in~$x$, and so must each of the
digits in the multiset $D=\{d_1,\ldots,d_{t-1}\}$. Therefore we know that
each of the digits in $S=P\cup D$ must be present in any solution~$x$.
(Recall that if $d$ appears $a$ times in a multiset~$A$ and $b$ times in
a multiset~$B$, then it appears $\max(a,b)$ times in $A\cup B$.)

The digit \PB{\|d} occurs \PB{\\{dist}[\|l][\|d]} times in $D$ and \PB{%
\\{pdist}[\|l][\|d]} times in~$P$.
If \PB{$\|d>\\{xl}$} we must have \PB{$\\{pdist}[\|l][\|d]\Z\\{dist}[\|l][%
\|d]$}.
If \PB{$\|d\K\\{xl}$} we set $\PB{\\{pd}}=\max(0,\PB{\\{pdist}}[l][d]-\PB{%
\\{dist}}[l][d])$.
Thus, if \PB{\\{xl}} occurs thrice in $D$ but only
once in~$P$, we have $pd=0$; but if \PB{\\{xl}} occurs thrice in $P$ but only
once in~$D$, we have $pd=2$. In the latter case we must choose $x_l=\PB{%
\\{xl}}$
and also $x_{l+1}=\PB{\\{xl}}$.

Let \PB{\|r} be the number of unknown digits of \PB{\|x}. (When \PB{$\\{pd}\K%
\T{0}$}, this is
$m+1$ minus $\vert S\vert$, the number of known digits.) If
$a_{lt}<b_{lt}<\PB{\\{xl}}$,
we know that $r>0$ and that one of the unknown digits lies between
$a_{lt}$ and~$b_{lt}$, inclusive.

When \PB{\\{xl}} decreases, the bounds get tighter, hence the prefix can
become longer. And that's good.

These are the key facts governing our bounds $a_l$ and $b_l$. In order
to do the computations conveniently we maintain the sum of known digits,
$\PB{\\{sig}}[l]=\sum_{k=0}^{xl-1}\PB{\\{pdist}}[l][k]\cdot k^m+
\sum_{k=xl}^9\PB{\\{dist}}[l][k]\cdot k^m+\PB{\\{pd}}\cdot\PB{\\{xl}}^m$.

\fi

\M{14}\B\X8:Global variables\X${}\mathrel+\E{}$\6
\&{int} ${}\\{dist}[\\{maxm}+\T{1}][\T{16}],{}$ ${}\\{pdist}[\\{maxm}+\T{1}][%
\T{16}];{}$\6
\&{ull} ${}\|a[\\{maxm}+\T{1}][\\{maxdigs}],{}$ ${}\|b[\\{maxm}+\T{1}][%
\\{maxdigs}],{}$ ${}\\{sig}[\\{maxm}+\T{1}][\\{maxdigs}];{}$\6
\&{int} ${}\|x[\\{maxm}+\T{1}],{}$ ${}\\{rsave}[\\{maxm}+\T{1}],{}$ ${}%
\\{tsave}[\\{maxm}+\T{1}],{}$ ${}\\{pdsave}[\\{maxm}+\T{1}]{}$;\par
\fi

\M{15}\B\X15:Initialize the data structures\X${}\E{}$\6
$\|l\K\T{1};{}$\6
${}\\{pd}\K\\{pdsave}[\T{1}]\K\T{0};{}$\6
${}\\{alt}\K\T{0},\39\\{blt}\K\T{9};{}$\6
${}\|t\K\|m,\39\|r\K\|m+\T{1};{}$\6
${}\\{xl}\K\T{9};{}$\6
${}\\{profile}[\T{1}]\K\T{1};{}$\6
\&{goto} \\{b3};\C{ I really don't want to do step \PB{\\{b2}} at root level! }%
\par
\U11.\fi

\M{16}\B\X16:Advance to the next level with $x_l=\PB{\\{xl}}$ and \PB{\&{goto} %
\\{b2}}\X${}\E{}$\6
$\\{oo},\39\\{tsave}[\|l]\K\|t,\39\\{rsave}[\|l]\K\|r;{}$\6
${}\|o,\39\\{pdsave}[\|l]\K\\{pd};{}$\6
${}\|o,\39\|x[\|l\PP]\K\\{xl};{}$\6
\&{goto} \\{b2};\par
\U11.\fi

\M{17}\B\X17:Print a solution and \PB{\&{goto} \\{b5}}\X${}\E{}$\6
${}\{{}$\1\6
${}\\{count}\PP;{}$\6
${}\\{printf}(\.{"\%d:\ "},\39\\{count});{}$\6
\&{for} ${}(\|k\K\T{1};{}$ ${}\|k\Z\|m+\T{1};{}$ ${}\|k\PP){}$\1\5
${}\\{printf}(\.{"\%d"},\39\|x[\|k]);{}$\2\6
\\{printf}(\.{"->"});\6
\&{for} ${}(\|k\K\|m;{}$ ${}\|k\G\T{0};{}$ ${}\|k\MM){}$\1\5
${}\\{printf}(\.{"\%d"},\39\\{nybb}(\\{sig}[\|l],\39\|k));{}$\2\6
\\{printf}(\.{"\\n"});\6
\&{goto} \\{b5};\6
\4${}\}{}$\2\par
\U11.\fi

\M{18}When this code is performed, \PB{\\{sig}[\|l]} and \PB{\\{dist}[\|l]} and
\PB{\\{pdist}[\|l]}
are supposed to be up to date, as well as \PB{\\{xl}}, \PB{\|t}, \PB{\|r}, \PB{%
\\{alt}}, and \PB{\\{blt}}.

\Y\B\4\X18:If there's an easy way to prove that $x_l$ can't be \PB{$\Z$ %
\\{xl}}, \PB{\&{goto} \\{b5}}\X${}\E{}$\6
\4\\{loop}:\5
\&{if} ${}(\|t\G\T{0}){}$\5
${}\{{}$\1\6
${}\\{change}\K\T{0};{}$\6
\X19:Recompute $a_l$ and $b_l$\X;\6
\&{if} ${}(\\{vbose}>\T{2}){}$\5
${}\{{}$\1\6
${}\\{fprintf}(\\{stderr},\39\.{"\ a="});{}$\6
${}\\{printnum}(\|a[\|l],\39\|t);{}$\6
${}\\{fprintf}(\\{stderr},\39\.{",b="});{}$\6
${}\\{printnum}(\|b[\|l],\39\|t);{}$\6
${}\\{fprintf}(\\{stderr},\39\.{"\\n"});{}$\6
\4${}\}{}$\2\6
\&{if} (\\{change})\1\5
\&{goto} \\{loop};\C{ either $a_l$ or $b_l$ or both can be improved }\2\6
\&{while} ${}(\\{alt}\E\\{blt}){}$\1\5
\X20:Increase the current prefix, or \PB{\&{goto} \\{b5}}\X;\2\6
\&{if} (\\{change})\1\5
\&{goto} \\{loop};\2\6
\4${}\}{}$\2\par
\U11.\fi

\M{19}The numbers \PB{\\{alt}} and \PB{\\{blt}} just past the prefix give
important constraints
on what the future can bring. If we can improve them, we can often
improve them further yet, and possibly even extend the prefix.

\Y\B\4\X19:Recompute $a_l$ and $b_l$\X${}\E{}$\6
\&{if} ${}(\\{blt}<\\{xl}){}$\5
${}\{{}$\1\6
\&{if} ${}(\|r\E\T{0}){}$\1\5
\&{goto} \\{b5};\2\6
${}\\{add}(\\{sig}[\|l],\39\\{table}[\T{1}][\\{alt}],\39\|a[\|l]){}$;\C{ $a_l%
\gets\PB{\\{sig}}[l]+\PB{\\{alt}}^m$ }\6
${}\\{add}(\\{sig}[\|l],\39\\{table}[\T{1}][\\{blt}],\39\|b[\|l]);{}$\6
${}\\{add}(\|b[\|l],\39\\{table}[\|r-\T{1}][\\{xl}],\39\|b[\|l]){}$;\C{ $b_l%
\gets\PB{\\{sig}}[l]+\PB{\\{blt}}^m+(r-1)\cdot\PB{\\{xl}}^m$ }\6
\4${}\}{}$\5
\2\&{else}\5
${}\{{}$\1\6
\&{for} ${}(\|k\K\T{0};{}$ ${}\|k<\\{mdigs};{}$ ${}\|k\PP){}$\1\5
${}\\{oo},\39\|a[\|l][\|k]\K\\{sig}[\|l][\|k]{}$;\C{ $a_l\gets\PB{\\{sig}}[l]$
}\2\6
${}\\{add}(\\{sig}[\|l],\39\\{table}[\|r][\\{xl}],\39\|b[\|l]){}$;\C{ $b_l\gets%
\PB{\\{sig}}[l]+r\cdot\PB{\\{xl}}^m$ }\6
\4${}\}{}$\2\6
\&{if} ${}(\|o,\39\\{alt}\I\\{nybb}(\|a[\|l],\39\|t)){}$\5
${}\{{}$\1\6
\&{if} ${}(\\{alt}>\\{nybb}(\|a[\|l],\39\|t)){}$\5
${}\{{}$\1\6
${}\\{fprintf}(\\{stderr},\39\.{"Confusion\ (a\ decrea}\)\.{sed)!\\n"});{}$\6
${}\\{exit}({-}\T{13});{}$\6
\4${}\}{}$\2\6
${}\\{alt}\K\\{nybb}(\|a[\|l],\39\|t);{}$\6
\&{if} ${}(\\{blt}<\\{xl}){}$\1\5
${}\\{change}\K\T{1};{}$\2\6
\4${}\}{}$\2\6
\&{if} ${}(\|o,\39\\{blt}\I\\{nybb}(\|b[\|l],\39\|t)){}$\5
${}\{{}$\1\6
\&{if} ${}(\\{blt}<\\{nybb}(\|b[\|l],\39\|t)){}$\5
${}\{{}$\1\6
${}\\{fprintf}(\\{stderr},\39\.{"Confusion\ (b\ increa}\)\.{sed)!\\n"});{}$\6
${}\\{exit}({-}\T{14});{}$\6
\4${}\}{}$\2\6
${}\\{blt}\K\\{nybb}(\|b[\|l],\39\|t);{}$\6
\&{if} ${}(\\{blt}<\\{xl}){}$\1\5
${}\\{change}\K\T{1};{}$\2\6
\4${}\}{}$\2\par
\U18.\fi

\M{20}Here's the most delicate (and most important) part, as we've learned
another digit of~\PB{\|x}.

Incidentally, here's an interesting example of a ``flowchart'' where
a \PB{\&{goto}} statement seems necessary without repeating code. Consider
two conditions $A$ and $B$, and two actions $\alpha$ and $\beta$.
If $A$ and $B$, we want to do $\alpha$ then~$\beta$;
if $A$ and not $B$, we want to do nothing;
if not~$A$, we want to do $\beta$. Without a \PB{\&{goto}} I must either
evaluate $A$ twice (as in
`if (not $A$) or $B$ then (if $A$ do $\alpha$; do~$\beta$)')
or code $\beta$ twice (as in
`if $A$ then (if $B$ do $\alpha$ and $\beta$) else do $\beta$').

\Y\B\4\X20:Increase the current prefix, or \PB{\&{goto} \\{b5}}\X${}\E{}$\6
${}\{{}$\1\6
${}\|o,\39\|p\K\\{pdist}[\|l][\\{blt}];{}$\6
\&{if} ${}(\\{blt}\G\\{xl}){}$\5
${}\{{}$\1\6
\&{if} ${}(\|o,\39\|p<\\{dist}[\|l][\\{blt}]){}$\1\5
\&{goto} \\{okay};\C{ a ``necessary'' \PB{\&{goto}}! }\2\6
\&{if} ${}(\\{blt}>\\{xl}){}$\1\5
\&{goto} \\{b5};\C{ oops, we've already saturated that digit }\2\6
${}\\{pd}\K\|p+\T{1}-\\{dist}[\|l][\\{blt}]{}$;\C{ \PB{\\{pd}} becomes
positive, if it wasn't already }\6
\4${}\}{}$\2\6
\&{if} ${}(\MM\|r<\T{0}){}$\1\5
\&{goto} \\{b5};\2\6
${}\\{add}(\\{sig}[\|l],\39\\{table}[\T{1}][\\{blt}],\39\\{sig}[\|l]){}$;\C{
newly known digit less than \PB{\\{xl}} }\6
\4\\{okay}:\5
${}\|o,\39\\{pdist}[\|l][\\{blt}]\K\|p+\T{1};{}$\6
${}\|t\MM,\39\\{change}\K\T{1};{}$\6
\&{if} ${}(\|t<\T{0}){}$\1\5
\&{break};\2\6
${}\\{oo},\39\\{alt}\K\\{nybb}(\|a[\|l],\39\|t),\39\\{blt}\K\\{nybb}(\|b[\|l],%
\39\|t);{}$\6
\4${}\}{}$\2\par
\U18.\fi

\M{21}\B\X21:Restore the previous state at level \PB{\|l}\X${}\E{}$\6
$\\{oo},\39\|t\K\\{tsave}[\|l],\39\|r\K\\{rsave}[\|l];{}$\6
\&{if} ${}(\|t\G\T{0}){}$\1\5
${}\\{oo},\39\\{alt}\K\\{nybb}(\|a[\|l],\39\|t),\39\\{blt}\K\\{nybb}(\|b[\|l],%
\39\|t);{}$\2\6
\&{else}\1\5
${}\\{alt}\K\\{blt}\K\T{9};{}$\2\6
${}\|o,\39\\{xl}\K\|x[\|l]{}$;\par
\U11.\fi

\M{22}When \PB{\\{dist}} is ``catching up'' with \PB{\\{pdist}}, we don't
change \PB{\\{sig}},
because a digit that occurred in the prefix was already accounted for;
we knew that an \PB{\\{xl}} would be coming, and it has finally arrived.
(Also \PB{\|t} and \PB{\|r} remain unchanged.)

\Y\B\4\X22:Absorb a forced move\X${}\E{}$\6
${}\{{}$\1\6
\&{if} ${}(\\{vbose}>\T{1}){}$\1\5
${}\\{fprintf}(\\{stderr},\39\.{"Level\ \%d,\ that\ \%d\ w}\)\.{as\ forced%
\\n"},\39\|l,\39\\{xl});{}$\2\6
\&{for} ${}(\|k\K\T{0};{}$ ${}\|k<\\{mdigs};{}$ ${}\|k\PP){}$\1\5
${}\\{oo},\39\\{sig}[\|l][\|k]\K\\{sig}[\|l-\T{1}][\|k];{}$\2\6
\&{if} ${}(\MM\\{pd}){}$\1\5
\&{goto} \\{move};\2\6
\4${}\}{}$\2\par
\U11.\fi

\N{1}{23}Index.
\fi

\inx
\fin
\con
