\input cwebmac
\datethis
\input epsf
\def\lab#1:#2/{$\.{#1}_{#2}$}


\N{1}{1}Introduction. The purpose of this program is to enumerate polyominoes
of up to 30 cells. The method---possibly new?---is based on a sequence of
diagonal slices through the shape, proceeding from the upper left to
the lower right. For example, the curious polyomino
$$\epsfbox{polyomino.1}$$
has 20 diagonal slices, which we will call
$$\vbox{\halign{\lab#/,\hfil\cr
1002:2\cr
1002:4\cr
1002:6\cr
10002:8\cr
10223004:13\cr
112022003:19\cr
1010101002:24\cr
100010010023:29\cr
1000100010022:34\cr
10001000100202:39\cr
100010000102002:44\cr
10200103001405446:54\cr
100022033003004:61\cr
10230340000005:67\cr
1220020000003:72\cr
100010000002:75\cr
1000002:77\cr
100023:80\cr
12344:85\cr
1111:89\cr
}}$$
respectively.
(This polyomino obviously has more than 30 cells, but large examples will help
clarify the concepts needed in the program below.)

Each slice name consists of a string of hexadecimal digits, beginning and
ending with a nonzero digit; it also has a numeric subscript (in decimal).
The subscript counts the cells that lie on and above this diagonal slice.
The nonzero hexadecimal digits represent cells in the current diagonal;
such cells have the same digit if and only if they are rookwise connected
as a consequence of the cells seen so far.

The main virtue of such an approach is that many polyominoes have identical
slice names, hence they are essentially equivalent to each other with respect
to the lower part of the diagram. For example, $2^{29}$ of the fixed
30-ominoes have the simple slice sequence \lab1:1/, \lab1:2/,
\dots,~\lab1:30/. The total number of possible slices will therefore be
substantially smaller than the total number of possible polyominoes.

This program enumerates polyominoes in fixed position, because the task
of correcting such counts to take account of symmetries takes much
less time.

Actually this program doesn't do the whole job of enumeration;
it only outputs the edges of a certain directed acyclic graph.
Another program reads that graph and computes the number of
paths through it.

\Y\B\4\D$\\{nmax}$ \5
\T{30}\C{ the size of polyominoes being counted, must not exceed 30 }\par
\Y\B\8\#\&{include} \.{<stdio.h>}\6
\X10:Type definitions\X\6
\X6:Global variables\X\6
\X2:Subroutines\X\7
\\{main}(\&{int} \\{argc}${},\39{}$\&{char} ${}{*}\\{argv}[\,]){}$\1\1\2\2\6
${}\{{}$\1\6
\X20:Local variables\X;\6
\X4:Scan the command line\X;\6
\X19:Initialize\X;\6
\X49:Compute\X;\6
\X62:Print the results\X;\6
\4${}\}{}$\2\par
\fi

\M{2}\B\X2:Subroutines\X${}\E{}$\6
\&{void} \\{panic}(\&{char} ${}{*}\\{mess}){}$\1\1\2\2\6
${}\{{}$\1\6
${}\\{fprintf}(\\{stderr},\39\.{"\%s!\\n"},\39\\{mess});{}$\6
${}\\{exit}({-}\T{1});{}$\6
\4${}\}{}$\2\par
\As5, 15, 35, 52, 53, 54\ETs59.
\U1.\fi

\M{3}The base name of the output file should be given as a command-line
argument. This name will actually be extended by \.{.0}, \.{.1}, \dots,
as explained below, because there might be an enormous amount of output.

\fi

\M{4}\B\X4:Scan the command line\X${}\E{}$\6
\&{if} ${}(\\{argc}\I\T{2}){}$\5
${}\{{}$\1\6
${}\\{fprintf}(\\{stderr},\39\.{"Usage:\ \%s\ outfilena}\)\.{me\\n"},\39%
\\{argv}[\T{0}]);{}$\6
\4${}\}{}$\2\6
${}\\{base\_name}\K\\{argv}[\T{1}]{}$;\par
\U1.\fi

\N{1}{5}Connectivity. By definition, a polyomino is a rookwise connected
set of cells. We'll want to restrict consideration to slices that can
actually lead to a 30-omino; this means that we reject any slice for which
one cannot connect all the so-far-unconnected pieces with \PB{$\T{30}-\|m$}
cells that lie below the \PB{\|m} cells already accounted for.

Fortunately a simple algorithm is available to compute the
minimum number of cells needed for connection. For example,
a hexadecimal pattern like \.{1002} needs 5 such cells, and \.{10203} needs~6.
A~more complex pattern like \.{1002000301} also needs~6; and in general
the subpattern $\.1\.0^{g_1}\.2\.0^{g_2}\ldots\.k\.0^{g_k}\.1$ needs
$\sum_{j=1}^k (2g_j+\nobreak1) - \max_{j=1}^k (2g_j+1)$,
after which that entire
subpattern can effectively be replaced by the single digit~\.1.

Equal digits are always nested, in the sense that we cannot have
a pattern like `\.{...1...2...1...2...}'. Therefore a simple
stack-like approach suffices for the computation, given the
hexadecimal digits \PB{\|a[\T{0}]}, \PB{\|a[\T{1}]}, \dots,~\PB{$\|a[\\{len}-%
\T{1}]$}.

\Y\B\4\X2:Subroutines\X${}\mathrel+\E{}$\6
\&{int} \\{conn\_distance}(\&{int} \\{len}${},\39{}$\&{char} ${}{*}\|a){}$\1\1%
\2\2\6
${}\{{}$\1\6
\&{register} \&{int} \|j${},{}$ \|k${},{}$ \|m${},{}$ \\{acc}${}\K\T{0};{}$\7
${}\\{stk}[\T{0}]\K\|a[\T{0}],\39\|m\K\T{0};{}$\6
\&{for} ${}(\|j\K\T{1};{}$ ${}\|j<\\{len};{}$ ${}\|j\PP){}$\5
${}\{{}$\1\6
\&{for} ${}(\|k\K\T{1};{}$ ${}\|a[\|j]\E\T{0};{}$ ${}\|j\PP){}$\1\5
${}\|k\MRL{+{\K}}\T{2};{}$\2\6
${}\\{acc}\MRL{+{\K}}\|k;{}$\6
\&{if} ${}(\|a[\|j]>\\{stk}[\|m]){}$\5
${}\{{}$\1\6
${}\\{dst}[\|m]\K\|k;{}$\6
${}\\{stk}[\PP\|m]\K\|a[\|j];{}$\6
\4${}\}{}$\5
\2\&{else}\5
${}\{{}$\1\6
\&{while} ${}(\|a[\|j]<\\{stk}[\|m]){}$\5
${}\{{}$\1\6
${}\|m\MM;{}$\6
\&{if} ${}(\\{dst}[\|m]>\|k){}$\1\5
${}\|k\K\\{dst}[\|m];{}$\2\6
\4${}\}{}$\2\6
\&{if} ${}(\|a[\|j]\I\\{stk}[\|m]){}$\1\5
\\{panic}(\.{"Oops,\ the\ program\ l}\)\.{ogic\ is\ screwed\ up"});\2\6
${}\\{acc}\MRL{-{\K}}\|k;{}$\6
\4${}\}{}$\2\6
\4${}\}{}$\2\6
\&{return} \\{acc};\6
\4${}\}{}$\2\par
\fi

\M{6}\B\X6:Global variables\X${}\E{}$\6
\&{char} \\{stk}[\T{16}]${},{}$ \\{dst}[\T{16}];\C{ stacks for component
numbers and distances }\par
\As13, 14, 18, 21, 31, 34, 37, 40, 48, 51, 58\ETs63.
\U1.\fi

\M{7}The \PB{\\{conn\_distance}} of an initial, totally disconnected slice like
\lab102300400056:6/, having $k$ nonzero digits and length~$l$, is $2l-1-k$.
Thus we see that the set of all 30-ominoes has exactly $2^{14}$ possible
initial states, corresponding to the bits of the odd numbers less
than~$2^{15}$.

(Also, if we ignore connectivity but consider only the on-off pattern of
cells, the same set of $2^{14}$ patterns accounts for all interior slices.
The reason is that \PB{\\{conn\_distance}} also computes the minimum number of
cells either above or below {\it or both\/} that will connect up a
given pattern: Folding does not decrease connectivity.

\Y\B\4\D$\\{maxlen}$ \5
$((\\{nmax}+\T{1})\GG\T{1}{}$)\C{ usable patterns won't be longer than this }%
\par
\Y\B\4\X7:Place the initial slices\X${}\E{}$\6
\&{for} ${}(\|k\K\T{1};{}$ ${}\|k<(\T{1}\LL\\{maxlen});{}$ ${}\|k\MRL{+{\K}}%
\T{2}){}$\1\5
\X47:Place the initial slice corresponding to the binary number \PB{\|k}\X;\2%
\par
\U49.\fi

\N{1}{8}Successive slices. Before we choose data structures for the main part
of the
computation, let's look at the key problem that faces us, namely the
determination of all feasible slices that can follow a given slice.
Ignoring the subscripts for the moment, what are the possible successors of a
slice like, say, \.{1020032014}?

That hypothetical slice has 6 cells, namely 6 nonzero hexadecimal digits,
and they are adjacent to 10 cells in the slice that
comes immediately below. We must occupy at least one cell that is adjacent
to each of the component classes \.1, \.2, \.3, and \.4, lest the whole
polyomino be disconnected. Choosing the cell between the \.3 and the~\.2,
and/or the cell between the \.1 and the~\.4,
will kill two birds with one stone; so
we can get by with as few as 2 cells in the slice that follows \.{1020032014},
and in such a case its pattern will be \.{1002}.
The principle of inclusion and exclusion tells us how many ways there are to
fulfill the connectivity constraint, namely
$$2^{10}-2^6-2^6-2^8-2^8+2^2+2^4+2^5+2^5+2^4+2^6-2^1-2^1-2^3-2^3+2^0=529.$$

For each of these ways to occupy the 10 adjacent cells, we can also add
new cells that are not connected to any of the previous ones. For example,
we could put a cell midway between the \.2 and the \.3 in \.{2003};
we could also occupy cells that lie off to the left or the right.

The calculations for outlying cells are the same for all slices: A pattern
of $k$ cells that extends $l$ positions left of the cells adjacent to a given
slice adds $2l-k$ to the \\{conn\_distance}, plus a constant that depends only
on the position of the leftmost occupied adjacent cell.
A similar situation applies at the right.

It's important to notice that some successors of a slice will occur
more than once. For example, there are five ways to go from \lab1011:8/ to
\lab1:9/. This is one of the reasons I have such high hopes for the slice
method.

\fi

\M{9}Several slices will have the same pattern of digits but different
subscripts. In such cases both slices have the same successors, except for
cutoffs based on the subscripts.

Closer study of this situation reveals, in fact, that each pattern in a slice
with subscript~$m$ occurs also with subscript~$m+1$, unless its
\PB{\\{conn\_distance}} equals \PB{$\\{nmax}-\|m$}. The reason is that we could
have added one
more cell above the topmost slice.

Therefore each pattern has a definite ``lifetime'': It is born at a certain
level~$m$, and it dies after level \PB{$\\{nmax}-\|d$}, where $d$ is its
\PB{\\{conn\_distance}}.

Our job then is to consider every pattern that is born at level~$m$ and to
compute all successor patterns of cost at most \PB{$\\{nmax}-\|m$}, where
``cost'' is the
cell-count~$w$ plus the connection distance~$d$. Such a successor will be born
at level $m+w$. We output this information to a file, so that a postprocessor
can rapidly count the number of paths through the network of possible slices.

If \PB{\\{nmax}} is not too large, we could easily build the network ourselves
and
avoid any postprocessing stage. For example, the author's first attempt at
such a program enumerated all fixed polynomials of size at most \PB{$\\{nmax}\K%
\T{15}$}
in less than half a second. And even when \PB{\\{nmax}} is 25, the
number of patterns turns out to be less than 300,000 and the number of
slices less than 600,000. But the number of arcs between slices is
19 million, and these numbers grow exponentially as \PB{\\{nmax}} increases.

\fi

\N{1}{10}Data structures. If you've been reading this commentary sequentially
instead
of hypertextually, you will now understand that the task of designing
efficient data structures for our network of slices is
quite interesting, although elementary.

Our program will go through the list of all slices with subscript~$m$ and
successively generate their successors. I have a hunch that the total number
of different slices will fit comfortably in memory. (In theory, a pattern
of $k$ cells can appear with ${2k\choose k}{1\over k+1}$ different sequences
of connection numbers, but in practice most of those sequences never arise. For
example, a pattern like \.{1213} or \.{10213} is impossible.)

Therefore each pattern like \.{100201} has its own \&{pattern} node in the
program below; that node can be addressed via a hash table.

The first digit of a pattern is always \.1, so we can omit it. We won't need
patterns of length more than 15, so each pattern can be represented as
a hexadecimal number with 14 digits (and trailing zeros). An additional
byte is prepended, containing the pattern length;
thus \.{100201} is actually represented by the 64-bit hexadecimal number
\PB{\T{\^0600201000000000}}. To make this program work on 32-bit machines,
a special type is declared in which we use 8 bytes instead of 16 nybbles.

\Y\B\4\D$\\{length}(\\{pk})$ \5
$\\{pk}.\\{bytes}{}$[\T{0}]\par
\Y\B\4\X10:Type definitions\X${}\E{}$\6
\&{typedef} \&{struct} ${}\{{}$\1\6
\&{unsigned} \&{char} \\{bytes}[\T{8}];\2\6
${}\}{}$ \&{patkey};\par
\As11\ET12.
\U1.\fi

\M{11}In the present program I'm also putting an \PB{\\{aux}} field into each
pattern
node, with the aim of eliminating a potentially long search when deciding
whether a pattern has appeared before as a successor. This field takes
up space, but it probably saves enough time to make it worthwhile.

\Y\B\4\X10:Type definitions\X${}\mathrel+\E{}$\6
\&{typedef} \&{struct} \&{patt\_struct} ${}\{{}$\1\6
\&{patkey} \\{key};\C{ the hexadecimal pattern of connection digits }\6
\&{struct} \&{succ\_struct} ${}{*}\\{aux}{}$;\C{ reference from a predecessor }%
\6
\&{struct} \&{patt\_struct} ${}{*}\\{link}{}$;\C{ chain pointer used by \PB{%
\\{lookup}} }\6
\&{struct} \&{patt\_struct} ${}{*}\\{next}{}$;\C{ previous pattern with the
same birthdate }\2\6
${}\}{}$ \&{patt};\par
\fi

\M{12}Each successor to the current pattern
has a table entry telling its weight, date of death,
and the number of ways in which it succeeds.

\Y\B\4\X10:Type definitions\X${}\mathrel+\E{}$\6
\&{typedef} \&{struct} \&{succ\_struct} ${}\{{}$\1\6
\&{patt} ${}{*}\\{pat}{}$;\C{ the successor pattern }\6
\&{char} \\{weight};\C{ the number of cells it contains }\6
\&{char} \\{degree};\C{ the replication number }\6
\&{char} \\{death};\C{ level at which it last appears }\2\6
${}\}{}$ \&{succ};\par
\fi

\M{13}Here I use a generous upper bound on the number of possible successors of
any pattern that might arise.

\Y\B\4\D$\\{succ\_size}$ \5
$(\\{maxlen}*(\T{1}\LL\\{maxlen}){}$)\par
\Y\B\4\X6:Global variables\X${}\mathrel+\E{}$\6
\&{succ} \\{succ\_table}[\\{succ\_size}];\C{ successors of the current pattern
}\6
\&{succ} ${}{*}\\{succ\_ptr}{}$;\C{ the first unused slot in \PB{\\{succ%
\_table}} }\par
\fi

\M{14}The pattern table is the big memory hog; when \PB{$\\{nmax}\K\T{30}$},
more than
three million patterns are involved.

\Y\B\4\D$\\{patt\_size}$ \5
\T{3002000}\C{ must exceed the number of patterns generated }\par
\Y\B\4\X6:Global variables\X${}\mathrel+\E{}$\6
\&{patt} \\{patt\_table}[\\{patt\_size}];\C{ the patterns }\6
\&{patt} ${}{*}\\{patt\_list}[\\{nmax}+\T{1}]{}$;\C{ lists of patterns sorted
by birthdate }\6
\&{int} ${}\\{patt\_count}[\\{nmax}+\T{1}]{}$;\C{ lengths of those lists }\6
\&{patt} ${}{*}\\{patt\_ptr}\K\\{patt\_table}{}$;\C{ the first unused slot in %
\PB{\\{patt\_table}} }\6
\&{patt} ${}{*}\\{bad\_patt}\K\\{patt\_table}+\\{patt\_size}-\T{1}{}$;\C{ the
first unusable slot }\par
\fi

\M{15}The \PB{\\{lookup}} routine finds a node given its pattern.
I'm using separate chains because I want patterns to be encoded as consecutive
numbers in the output.

Parameters \PB{\\{len}} and \PB{\|a} are the pattern length and digits, as in
the
\PB{\\{conn\_distance}} routine. Parameter \PB{\|m} is the birthdate of the
pattern, if it
happens to be new.

\Y\B\4\X2:Subroutines\X${}\mathrel+\E{}$\6
\&{patt} ${}{*}{}$\\{lookup}(\&{int} \\{len}${},\39{}$\&{char} ${}{*}\|a,\39{}$%
\&{int} \|m)\1\1\2\2\6
${}\{{}$\1\6
\&{patkey} \\{key};\6
\&{register} \&{int} \|j${},{}$ \|l;\6
\&{register} \&{unsigned} \&{int} \|h;\6
\&{register} \&{patt} ${}{*}\|p;{}$\6
\&{register} \&{unsigned} \&{char} ${}{*}\|q;{}$\7
\X17:Pack and hash the \PB{\\{key}} from \PB{\|a} and \PB{\\{len}}\X;\6
${}\|p\K\\{hash\_table}[\|h];{}$\6
\&{if} ${}(\R\|p){}$\1\5
${}\\{hash\_table}[\|h]\K\\{patt\_ptr};{}$\2\6
\&{else}\5
\1\&{while} (\T{1})\5
${}\{{}$\1\6
\&{for} ${}(\|j\K\T{0},\39\|q\K{\AND}(\|p\MG\\{key}.\\{bytes}[\T{0}]);{}$ ${}%
\|j<\|l;{}$ ${}\|j\PP,\39\|q\PP){}$\1\6
\&{if} ${}({*}\|q\I\\{key}.\\{bytes}[\|j]){}$\1\5
\&{goto} \\{mismatch};\2\2\6
\&{return} \|p;\C{ successful search, the key matches }\6
\4\\{mismatch}:\5
\&{if} ${}(\R\|p\MG\\{link}){}$\5
${}\{{}$\1\6
${}\|p\MG\\{link}\K\\{patt\_ptr}{}$;\5
\&{break};\6
\4${}\}{}$\2\6
${}\|p\K\|p\MG\\{link};{}$\6
\4${}\}{}$\2\2\6
\X16:Insert a new pattern into \PB{\\{patt\_table}} and return\X;\6
\4${}\}{}$\2\par
\fi

\M{16}\B\X16:Insert a new pattern into \PB{\\{patt\_table}} and return\X${}%
\E{}$\6
\&{if} ${}(\\{patt\_ptr}\E\\{bad\_patt}){}$\1\5
\\{panic}(\.{"Pattern\ memory\ over}\)\.{flow"});\2\6
${}\\{patt\_ptr}\MG\\{key}\K\\{key};{}$\6
${}\\{patt\_ptr}\MG\\{next}\K\\{patt\_list}[\|m];{}$\6
${}\\{patt\_list}[\|m]\K\\{patt\_ptr};{}$\6
${}\\{patt\_count}[\|m]\PP;{}$\6
\&{return} \\{patt\_ptr}${}\PP{}$;\par
\U15.\fi

\M{17}``Universal hashing'' (TAOCP exercise 6.4--72) is used to get a good hash
function, because most of the key bits are zero.

\Y\B\4\D$\\{hash\_width}$ \5
\T{20}\C{ lg of hash table size }\par
\B\4\D$\\{hash\_mask}$ \5
$((\T{1}\LL\\{hash\_width})-\T{1}{}$)\par
\Y\B\4\X17:Pack and hash the \PB{\\{key}} from \PB{\|a} and \PB{\\{len}}\X${}%
\E{}$\6
$\|a[\\{len}]\K\T{0},\39\|h\K\\{len}\LL(\\{hash\_width}-\T{4});{}$\6
\&{for} ${}(\|l\K\T{1};{}$ ${}\|l+\|l\Z\\{len};{}$ ${}\|l\PP){}$\5
${}\{{}$\1\6
${}\\{key}.\\{bytes}[\|l]\K(\|a[\|l+\|l-\T{1}]\LL\T{4})+\|a[\|l+\|l];{}$\6
${}\|h\MRL{+{\K}}\\{hash}[\|l][\\{key}.\\{bytes}[\|l]];{}$\6
\4${}\}{}$\2\6
${}\\{length}(\\{key})\K\\{len};{}$\6
${}\|h\MRL{\AND{\K}}\\{hash\_mask}{}$;\par
\U15.\fi

\M{18}\B\X6:Global variables\X${}\mathrel+\E{}$\6
\&{patt} ${}{*}\\{hash\_table}[\\{hash\_mask}+\T{1}]{}$;\C{ heads of the chains
}\6
\&{unsigned} \&{int} \\{hash}[\T{8}][\T{256}];\C{ random bits for universal
hashing }\par
\fi

\M{19}The random number generator used here doesn't have to be of sensational
quality.

\Y\B\4\X19:Initialize\X${}\E{}$\6
$\|m\K\T{314159265};{}$\6
\&{for} ${}(\|j\K\T{1};{}$ ${}\|j<\T{8};{}$ ${}\|j\PP){}$\1\6
\&{for} ${}(\|k\K\T{0};{}$ ${}\|k<\T{256};{}$ ${}\|k\PP){}$\5
${}\{{}$\1\6
${}\|m\K\T{69069}*\|m+\T{1};{}$\6
${}\\{hash}[\|j][\|k]\K\|m\GG(\T{32}-\\{hash\_width});{}$\6
\4${}\}{}$\2\2\par
\As24\ET55.
\U1.\fi

\M{20}\B\X20:Local variables\X${}\E{}$\6
\&{register} \&{int} \|j${},{}$ \|k${},{}$ \|l${},{}$ \|m;\par
\As23\ET45.
\U1.\fi

\N{1}{21}Computing the successors. Now let's turn to the details of the
procedure sketched earlier. We will want some special data structures
for that, in addition to the major structures used for patterns.

The cells of a possible successor slice will be numbered from 0 to 49,
with cell 16 being adjacent-to-and-left-of the initial \.1 in the
slice whose successors are being found. (Any number exceeding
$3\times\PB{\\{nmax}}/2$ will do in place of~50; we will have fewer than
15 new elements to the left of the pattern and fewer than 15 to the right,
hence we have plenty of elbow room.) Cell $j$ will be occupied if and only if
\PB{\\{occ}[\|j]} is nonzero. Cell~0 is permanently unoccupied.

The cells adjacent to the previous pattern are \PB{\\{adjcell}[\T{0}]}, \PB{%
\\{adjcell}[\T{1}]},
\dots, terminating with~0. The cells interior to but not adjacent to
the previous pattern are \PB{\\{intcell}[\T{0}]}, \PB{\\{intcell}[\T{1}]}, %
\dots, again terminating
with~0. The nonadjacent cells to the left of the pattern are 15, 14, \dots,
and the nonadjacent cells to the right are \PB{\\{rightend}}, \PB{$%
\\{rightend}+\T{1}$}, \dots.
The array \PB{\\{dig}} contains the hexadecimal pattern digits.
The array \PB{\\{touched}} tells how many occupied cells are adjacent to a
given connected component. Finally, there's an array \PB{\\{first}}
with a slightly tricky meaning: \PB{$\\{first}[\|j]\K\|k$} if \PB{\\{dig}[\|j]}
was
the leftmost appearance of component~\PB{\|k}.

For example, with pattern \.{1020032014}, we have
\PB{$\\{dig}[\T{16}]\K\T{1}$}, \PB{$\\{dig}[\T{17}]\K\T{0}$}, \PB{$\\{dig}[%
\T{18}]\K\T{2}$}, \dots, \PB{$\\{dig}[\T{25}]\K\T{4}$};
also \PB{$\\{adjcell}[\T{0}]\K\T{16}$}, \PB{$\\{adjcell}[\T{1}]\K\T{17}$}, %
\dots, \PB{$\\{adjcell}[\T{9}]\K\T{26}$}, \PB{$\\{adjcell}[\T{10}]\K\T{0}$};
and \PB{$\\{intcell}[\T{0}]\K\T{20}$}, \PB{$\\{intcell}[\T{1}]\K\T{0}$}, \PB{$%
\\{rightend}\K\T{27}$}.
The values of \PB{\\{first}} are zero except that \PB{$\\{first}[\T{16}]\K%
\T{1}$},
\PB{$\\{first}[\T{18}]\K\T{2}$}, \PB{$\\{first}[\T{21}]\K\T{3}$}, and \PB{$%
\\{first}[\T{25}]\K\T{4}$}.
If \PB{$\\{occ}[\T{14}]\K\\{occ}[\T{16}]\K\\{occ}[\T{19}]\K\\{occ}[\T{20}]\K%
\\{occ}[\T{25}]\K\T{1}$} and other entries of \PB{\\{occ}} are
zero, we will have \PB{$\\{touched}[\T{1}]\K\T{2}$}, \PB{$\\{touched}[\T{2}]\K%
\T{1}$}, \PB{$\\{touched}[\T{3}]\K\T{0}$}, and
\PB{$\\{touched}[\T{4}]\K\T{1}$}.

\Y\B\4\X6:Global variables\X${}\mathrel+\E{}$\6
\&{char} \\{dig}[\T{50}];\C{ component numbers in previous slice }\6
\&{char} \\{adjcell}[\T{17}];\C{ cells adjacent to the previous slice }\6
\&{char} \\{intcell}[\T{13}];\C{ nonadjacent cells between adjacent ones }\6
\&{char} \\{first}[\T{50}];\C{ initial appearances of components that mustn't
die }\6
\&{char} \\{occ}[\T{50}];\C{ is this cell occupied? }\6
\&{char} \\{touched}[\T{16}];\C{ occupied cells adjacent to components }\6
\&{char} \\{appeared}[\T{16}];\C{ auxiliary record of component appearances }\6
\&{char} \\{rightend};\C{ the smallest nonadjacent cell at the right of the
pattern }\6
\&{char} \\{leftbound}${},{}$ \\{rightbound};\C{ first and last occupied cells
}\par
\fi

\M{22}Given a pattern \PB{\|p} whose successors need to be found,
we begin by initializing the structures just mentioned.

Later we will mention a \PB{\\{leader}} table, which might as well be
initialized while we're setting up the other things.
Any cell that is not adjacent to the
previous slice should have \PB{$\\{leader}[\|j]\K\|j$}.

The program in this step does not clear \PB{\\{dig}[\\{rightend}]} to zero.
No harm is done, because subsequent steps never look at \PB{\\{dig}[\|j]} for
\PB{$\|j\G\\{rightend}$}.

\Y\B\4\X22:Unpack and massage the pattern \PB{$\|p\MG\\{key}$}\X${}\E{}$\6
$\|l\K\\{length}(\|p\MG\\{key});{}$\6
\&{for} ${}(\|j\K\T{2};{}$ ${}\|j\Z\|l;{}$ ${}\|j\MRL{+{\K}}\T{2}){}$\5
${}\{{}$\1\6
${}\|k\K\|p\MG\\{key}.\\{bytes}[(\|j\GG\T{1})];{}$\6
${}\\{dig}[\|j+\T{15}]\K\|k\GG\T{4},\39\\{dig}[\|j+\T{16}]\K\|k\AND\T{\^f};{}$\6
\4${}\}{}$\2\6
${}\\{dig}[\|l+\T{16}]\K\T{0};{}$\6
${}\\{rightend}\K\|l+\T{17};{}$\6
\&{for} ${}(\|j\K\\{rightend};{}$ ${}\|j\Z\T{31};{}$ ${}\|j\PP){}$\1\5
${}\\{leader}[\|j]\K\|j;{}$\2\6
\&{for} ${}(\|j\K\T{1};{}$ ${}\|j\Z\|l;{}$ ${}\|j\PP){}$\1\5
${}\\{touched}[\|j]\K\\{appeared}[\|j]\K\T{0};{}$\2\6
\&{for} ${}(\|j\K\T{16},\39\|k\K\|l\K\T{0};{}$ ${}\|j<\\{rightend};{}$ ${}\|j%
\PP){}$\5
${}\{{}$\1\6
${}\\{first}[\|j]\K\T{0};{}$\6
\&{if} (\\{dig}[\|j])\5
${}\{{}$\1\6
\&{if} ${}(\\{dig}[\|j-\T{1}]\E\T{0}){}$\1\5
${}\\{adjcell}[\|k\PP]\K\|j;{}$\2\6
${}\\{adjcell}[\|k\PP]\K\|j+\T{1};{}$\6
\&{if} ${}(\R\\{appeared}[\\{dig}[\|j]]){}$\1\5
${}\\{first}[\|j]\K\\{dig}[\|j],\39\\{appeared}[\\{dig}[\|j]]\K\T{1};{}$\2\6
\4${}\}{}$\2\6
\&{else} \&{if} ${}(\\{dig}[\|j-\T{1}]\E\T{0}){}$\1\5
${}\\{intcell}[\|l\PP]\K\|j,\39\\{leader}[\|j]\K\|j;{}$\2\6
\4${}\}{}$\2\6
${}\\{adjcell}[\|k]\K\\{intcell}[\|l]\K\T{0}{}$;\par
\U27.\fi

\M{23}\B\X20:Local variables\X${}\mathrel+\E{}$\6
\&{register} \&{patt} ${}{*}\|p{}$;\par
\fi

\M{24}\B\X19:Initialize\X${}\mathrel+\E{}$\6
$\\{dig}[\T{16}]\K\T{1};{}$\6
\&{for} ${}(\|j\K\T{1};{}$ ${}\|j<\T{16};{}$ ${}\|j\PP){}$\1\5
${}\\{leader}[\|j]\K\|j;{}$\2\6
\&{for} ${}(\|j\K\T{31};{}$ ${}\|j<\T{50};{}$ ${}\|j\PP){}$\1\5
${}\\{leader}[\|j]\K\|j{}$;\2\par
\fi

\M{25}A setting of the \PB{\\{occ}} array for adjacent cells is valid if and
only
if each component is adjacent to at least one occupied cell. The
simple algorithm in this step moves from one valid setting to the
colexicographically next one, or does a \PB{\&{break}} if the last valid
setting has been considered.

It is convenient to set \PB{\\{touched}[\T{0}]} to such a large value that it
cannot become zero.

\Y\B\4\X25:Move to the next valid pattern of adjacent cells, or \PB{\&{break}}%
\X${}\E{}$\6
$\\{touched}[\T{0}]\K\T{128};{}$\6
\&{for} ${}(\|k\K\T{0};{}$  ; ${}\|k\PP){}$\5
${}\{{}$\1\6
${}\|j\K\\{adjcell}[\|k];{}$\6
\&{if} (\\{occ}[\|j])\1\5
${}\\{touched}[\\{dig}[\|j-\T{1}]]\MM,\39\\{touched}[\\{dig}[\|j]]\MM,\39%
\\{occ}[\|j]\K\T{0};{}$\2\6
\&{else}\1\5
\&{break};\2\6
\4${}\}{}$\2\6
\&{if} ${}(\R\|j){}$\1\5
\&{break};\C{ all were occupied, but now \PB{\\{occ}} is entirely zero }\2\6
${}\\{touched}[\\{dig}[\|j-\T{1}]]\PP,\39\\{touched}[\\{dig}[\|j]]\PP,\39%
\\{occ}[\|j]\K\T{1};{}$\6
\X26:Move up to the next valid setting\X;\par
\U27.\fi

\M{26}We have essentially added 1 in binary notation, clearing \PB{\\{occ}}
bits
to zero when ``carrying.'' Now we might have to reset some of them
in order to keep components alive.

(This computation is done also when we're getting started. Then \PB{\\{occ}} is
identically zero and \PB{\|k} is at the end of the list of adjacent cells.
In that case it finds the colexicographically smallest valid configuration.)

\Y\B\4\X26:Move up to the next valid setting\X${}\E{}$\6
\&{for} ${}(\|k\MM;{}$ ${}\|k\G\T{0};{}$ ${}\|k\MM){}$\5
${}\{{}$\1\6
${}\|j\K\\{adjcell}[\|k];{}$\6
\&{if} ${}(\R\\{touched}[\\{first}[\|j]]){}$\1\5
${}\\{touched}[\\{dig}[\|j-\T{1}]]\PP,\39\\{touched}[\\{dig}[\|j]]\PP,\39%
\\{occ}[\|j]\K\T{1};{}$\2\6
\4${}\}{}$\2\par
\Us25\ET27.\fi

\M{27}Fans of top-down programming will have noticed that we've recently been
working bottom-up. Now let's get back in balance by giving an outline of
the successor generation process.

When this code is performed, \PB{\|m} will be the current slice's subscript,
namely
the number of cells on and above the slice whose successor is being
found.

The canonization process below will set \PB{\|l} to the cost of the new
pattern.

\Y\B\4\X27:Generate all successors to \PB{\|p}\X${}\E{}$\6
${}\{{}$\1\6
${}\\{succ\_ptr}\K\\{succ\_table};{}$\6
\X22:Unpack and massage the pattern \PB{$\|p\MG\\{key}$}\X;\6
${}\\{touched}[\T{0}]\K\T{128};{}$\6
\X26:Move up to the next valid setting\X;\6
\&{while} (\T{1})\5
${}\{{}$\1\6
\X41:Canonize the new pattern based on adjacent occupied cells\X;\6
\&{if} ${}(\|m+\|l>\\{nmax}){}$\1\5
\&{goto} \\{move};\C{ prune it away, it makes only big polyominoes }\2\6
\X44:Insert the new pattern into the successor list\X;\6
\X28:Run through all patterns of nonadjacent cells that might be relevant\X;\6
\4\\{move}:\5
\X25:Move to the next valid pattern of adjacent cells, or \PB{\&{break}}\X;\6
\4${}\}{}$\2\6
\X46:List also the null successor, if appropriate\X;\6
\4${}\}{}$\2\par
\U49.\fi

\M{28}\B\X28:Run through all patterns of nonadjacent cells that might be
relevant\X${}\E{}$\6
\&{while} (\T{1})\5
${}\{{}$\1\6
\X29:Run through all patterns of nonadjacent cells at the left\X;\6
\4\\{advance}:\5
\&{for} ${}(\|k\K\T{0};{}$  ; ${}\|k\PP){}$\5
${}\{{}$\1\6
${}\|j\K\\{intcell}[\|k];{}$\6
\&{if} (\\{occ}[\|j])\1\5
${}\\{occ}[\|j]\K\T{0};{}$\2\6
\&{else}\1\5
\&{break};\2\6
\4${}\}{}$\2\6
\&{if} ${}(\R\|j){}$\1\5
\&{break};\2\6
${}\\{occ}[\|j]\K\T{1};{}$\6
\X42:Canonize the new pattern based on adjacent and interior occupied cells\X;\6
\&{if} ${}(\|m+\|l>\\{nmax}){}$\1\5
\&{goto} \\{advance};\2\6
\X44:Insert the new pattern into the successor list\X;\6
\4${}\}{}$\2\par
\U27.\fi

\M{29}Here I make use of Mathematics, although it saves only a little
computation: When the \&{while} loop in this section ends,
\PB{\\{occ}[\|j]} will be zero
for $\PB{\\{leftbound}}<j<16$, because of a property of the \PB{\\{conn%
\_distance}}
function that was mentioned earlier.

\Y\B\4\X29:Run through all patterns of nonadjacent cells at the left\X${}\E{}$\6
$\\{save\_leftbound}\K\\{leftbound};{}$\6
\&{while} (\T{1})\5
${}\{{}$\1\6
\X30:Run through all patterns of nonadjacent cells at the right\X;\6
\&{for} ${}(\|j\K\T{15};{}$  ; ${}\|j\MM){}$\5
${}\{{}$\1\6
\&{if} (\\{occ}[\|j])\1\5
${}\\{occ}[\|j]\K\T{0};{}$\2\6
\&{else}\1\5
\&{break};\2\6
\4${}\}{}$\2\6
${}\\{occ}[\|j]\K\T{1};{}$\6
\&{if} ${}(\|j<\\{leftbound}){}$\1\5
${}\\{leftbound}\K\|j;{}$\2\6
\&{if} ${}(\\{rightbound}-\\{leftbound}\G\\{maxlen}){}$\1\5
\&{break};\2\6
\X43:Canonize the new pattern based on all occupied cells\X;\6
\&{if} ${}(\|m+\|l>\\{nmax}){}$\1\5
\&{break};\2\6
\X44:Insert the new pattern into the successor list\X;\6
\4${}\}{}$\2\6
${}\\{occ}[\\{leftbound}]\K\T{0}{}$;\C{ this clears out the whole left end }\6
${}\\{leftbound}\K\\{save\_leftbound}{}$;\par
\U28.\fi

\M{30}\B\X30:Run through all patterns of nonadjacent cells at the right\X${}%
\E{}$\6
$\\{save\_rightbound}\K\\{rightbound};{}$\6
${}\\{rightbound}\K\\{rightend};{}$\6
${}\\{occ}[\\{rightend}]\K\T{1};{}$\6
\&{while} (\T{1})\5
${}\{{}$\1\6
\&{if} ${}(\\{rightbound}-\\{leftbound}\G\\{maxlen}){}$\1\5
\&{break};\2\6
\X43:Canonize the new pattern based on all occupied cells\X;\6
\&{if} ${}(\|m+\|l>\\{nmax}){}$\1\5
\&{break};\2\6
\X44:Insert the new pattern into the successor list\X;\6
\&{for} ${}(\|j\K\\{rightend};{}$  ; ${}\|j\PP){}$\5
${}\{{}$\1\6
\&{if} (\\{occ}[\|j])\1\5
${}\\{occ}[\|j]\K\T{0};{}$\2\6
\&{else}\1\5
\&{break};\2\6
\4${}\}{}$\2\6
${}\\{occ}[\|j]\K\T{1};{}$\6
\&{if} ${}(\|j>\\{rightbound}){}$\1\5
${}\\{rightbound}\K\|j;{}$\2\6
\4${}\}{}$\2\6
${}\\{occ}[\\{rightbound}]\K\T{0}{}$;\C{ this clears out the whole right end }\6
${}\\{rightbound}\K\\{save\_rightbound}{}$;\par
\U29.\fi

\M{31}\B\X6:Global variables\X${}\mathrel+\E{}$\6
\&{char} \\{save\_leftbound}${},{}$ \\{save\_rightbound};\C{ boundaries within
the previous pattern }\par
\fi

\N{1}{32}Canonization. Once a sequence of occupied cells has been proposed, we
need
to represent it in canonical form as a sequence of component digits. For
example, if we occupy the four cells marked \.x in
$$\vbox{\halign{\hfil\.{#}\hfil\cr
1020032014\cr
\noalign{\vskip-3pt}
x00000x00xx\cr}}$$
then the components \.{23} and \.{14} are merged, so the new digits are
\.{10000020011}. (A canonical sequence always numbers the components
\.1, \.2, \dots~in order as they appear from left to right.)

Our first task is trivial: We occasionally need to locate the leftmost and
rightmost occupied cells.

\Y\B\4\X32:Find the proper \PB{\\{leftbound}} and \PB{\\{rightbound}}\X${}\E{}$%
\6
\&{for} ${}(\|j\K\T{16};{}$ ${}\R\\{occ}[\|j];{}$ ${}\|j\PP){}$\1\5
;\2\6
${}\\{leftbound}\K\|j;{}$\6
\&{for} ${}(\|j\K\\{rightend}-\T{1};{}$ ${}\R\\{occ}[\|j];{}$ ${}\|j\MM){}$\1\5
;\2\6
${}\\{rightbound}\K\|j{}$;\par
\Us41\ET42.\fi

\M{33}Our next job is more interesting: After changing the status of
adjacent cells, we need to merge components that are being joined.

A simple ``union--find'' algorithm is appropriate for this task.
Each occupied cell will point to the smallest cell in its class,
and the cells of a class are also circularly linked.

However, we make a quick exit to \PB{\\{done}} if the cost is obviously so high
that the rest of the calculation cannot succeed.

\Y\B\4\X33:Find the new classes, or \PB{\&{goto} \\{done}}\X${}\E{}$\6
\&{for} ${}(\|j\K\\{leftbound}-\T{1},\39\|l\K\|m;{}$ ${}\|j\Z\\{rightbound};{}$
${}\|j\PP){}$\5
${}\{{}$\1\6
\&{if} (\\{occ}[\|j])\1\5
${}\|l\PP,\39\\{circ}[\|j]\K\\{leader}[\|j]\K\|j;{}$\2\6
\&{if} (\\{dig}[\|j])\5
${}\{{}$\1\6
\&{if} (\\{occ}[\|j])\1\5
${}\\{appeared}[\\{dig}[\|j]]\K\|j;{}$\2\6
\&{else} \&{if} ${}(\\{occ}[\|j+\T{1}]){}$\1\5
${}\\{appeared}[\\{dig}[\|j]]\K\|j+\T{1};{}$\2\6
\4${}\}{}$\2\6
\4${}\}{}$\2\6
\&{if} ${}(\|l>\\{nmax}){}$\1\5
\&{goto} \\{done};\C{ pointless to continue }\2\6
\&{for} ${}(\|j\K\\{leftbound}-\T{1};{}$ ${}\|j\Z\\{rightbound};{}$ ${}\|j%
\PP){}$\1\6
\&{if} ${}((\|k\K\\{dig}[\|j])){}$\5
${}\{{}$\1\6
\&{if} ${}(\\{occ}[\|j]\W\\{leader}[\|j]\I\\{leader}[\\{appeared}[\|k]]){}$\1\5
${}\\{merge}(\\{leader}[\|j],\39\\{leader}[\\{appeared}[\|k]]);{}$\2\6
\&{if} ${}(\\{occ}[\|j+\T{1}]\W\\{leader}[\|j+\T{1}]\I\\{leader}[\\{appeared}[%
\|k]]){}$\1\5
${}\\{merge}(\\{leader}[\|j+\T{1}],\39\\{leader}[\\{appeared}[\|k]]);{}$\2\6
\4${}\}{}$\2\2\par
\U41.\fi

\M{34}\B\X6:Global variables\X${}\mathrel+\E{}$\6
\&{char} \\{circ}[\T{50}];\C{ circular links for component classes }\6
\&{char} \\{leader}[\T{50}];\C{ class representatives }\par
\fi

\M{35}\B\X2:Subroutines\X${}\mathrel+\E{}$\6
\&{void} \\{merge}(\&{int} \|j${},\39{}$\&{int} \|k)\1\1\2\2\6
${}\{{}$\1\6
\&{register} \&{int} \|p${},{}$ \|t;\7
\&{if} ${}(\|j<\|k){}$\5
${}\{{}$\1\6
\&{for} ${}(\|p\K\|t\K\\{circ}[\|k];{}$ ${}\|p\I\|k;{}$ ${}\|p\K\\{circ}[%
\|p]){}$\1\5
${}\\{leader}[\|p]\K\|j;{}$\2\6
${}\\{leader}[\|p]\K\|j;{}$\6
${}\\{circ}[\|p]\K\\{circ}[\|j],\39\\{circ}[\|j]\K\|t;{}$\6
\4${}\}{}$\5
\2\&{else}\5
${}\{{}$\1\6
\&{for} ${}(\|p\K\|t\K\\{circ}[\|j];{}$ ${}\|p\I\|j;{}$ ${}\|p\K\\{circ}[%
\|p]){}$\1\5
${}\\{leader}[\|p]\K\|k;{}$\2\6
${}\\{leader}[\|p]\K\|k;{}$\6
${}\\{circ}[\|p]\K\\{circ}[\|k],\39\\{circ}[\|k]\K\|t;{}$\6
\4${}\}{}$\2\6
\4${}\}{}$\2\par
\fi

\M{36}The \PB{\\{newpat}} array, which contains the canonical component
numbers starting with~\.1, can now be written down without further ado.

\Y\B\4\X36:Establish the \PB{\\{newpat}} array\X${}\E{}$\6
\&{for} ${}(\|j\K\\{leftbound},\39\|k\K\|l\K\T{0};{}$ ${}\|j\Z%
\\{rightbound};{}$ ${}\|j\PP,\39\|k\PP){}$\1\6
\&{if} ${}(\R\\{occ}[\|j]){}$\1\5
${}\\{newpat}[\|k]\K\T{0};{}$\2\6
\&{else} \&{if} ${}(\\{leader}[\|j]<\|j){}$\1\5
${}\\{newpat}[\|k]\K\\{class}[\\{leader}[\|j]];{}$\2\6
\&{else}\1\5
${}\\{newpat}[\|k]\K\\{class}[\|j]\K\PP\|l{}$;\2\2\par
\U39.\fi

\M{37}\B\X6:Global variables\X${}\mathrel+\E{}$\6
\&{char} \\{class}[\T{50}];\C{ canonical component number }\par
\fi

\M{38}We aren't done yet, however. The reverse of a pattern is
computationally equivalent to the pattern itself; so we gain
a factor of roughly two (in both time and space)
by switching to the reverse pattern when
it is lexicographically smaller.

\Y\B\4\X38:Establish the \PB{\\{backpat}} array\X${}\E{}$\6
\&{for} ${}(\|j\K\\{rightbound},\39\|k\K\|l\K\T{0};{}$ ${}\|j\G%
\\{leftbound};{}$ ${}\|j\MM,\39\|k\PP){}$\1\6
\&{if} ${}(\R\\{occ}[\|j]){}$\1\5
${}\\{backpat}[\|k]\K\T{0};{}$\2\6
\&{else} \&{if} ${}(\\{class}[\\{leader}[\|j]]\AND\T{\^10}){}$\1\5
${}\\{backpat}[\|k]\K\\{class}[\\{leader}[\|j]]\AND\T{\^f};{}$\2\6
\&{else}\1\5
${}\\{backpat}[\|k]\K\PP\|l,\39\\{class}[\\{leader}[\|j]]\K\|l+\T{\^10}{}$;\2\2%
\par
\U39.\fi

\M{39}Here then is how we compute the cost of a purported successor.
(Again we bypass computation when a detailed calculation would be fruitless.)

\Y\B\4\X39:Determine the cost of the new pattern\X${}\E{}$\6
\&{for} ${}(\|j\K\\{leftbound},\39\|l\K\T{0};{}$ ${}\|j\Z\\{rightbound};{}$ ${}%
\|j\PP){}$\1\6
\&{if} (\\{occ}[\|j])\1\5
${}\|l\PP;{}$\2\2\6
\&{if} ${}(\|m+\|l\Z\\{nmax}){}$\5
${}\{{}$\1\6
${}\\{weight}\K\|l;{}$\6
\X36:Establish the \PB{\\{newpat}} array\X;\6
\X38:Establish the \PB{\\{backpat}} array\X;\6
${}\\{len}\K\|k{}$;\C{ at this point \PB{\|k} is the pattern length }\6
${}\\{bestpat}\K\\{newpat};{}$\6
\&{for} ${}(\|j\K\T{1};{}$ ${}\|j<\|k;{}$ ${}\|j\PP){}$\1\6
\&{if} ${}(\\{newpat}[\|j]<\\{backpat}[\|j]){}$\1\5
\&{break};\2\6
\&{else} \&{if} ${}(\\{newpat}[\|j]>\\{backpat}[\|j]){}$\5
${}\{{}$\1\6
${}\\{bestpat}\K\\{backpat};{}$\6
\&{break};\6
\4${}\}{}$\2\2\6
${}\|k\K\\{conn\_distance}(\|k,\39\\{bestpat});{}$\6
${}\|l\K\\{weight}+\|k;{}$\6
\4${}\}{}$\2\par
\Us41, 42\ETs43.\fi

\M{40}\B\X6:Global variables\X${}\mathrel+\E{}$\6
\&{char} \\{newpat}[\T{16}];\C{ canonical sequence for the new pattern }\6
\&{char} \\{backpat}[\T{16}];\C{ canonical sequence for its reversal }\6
\&{char} ${}{*}\\{bestpat}{}$;\C{ the lexicographically smaller }\6
\&{int} \\{weight};\C{ the number of occupied cells }\par
\fi

\M{41}\B\X41:Canonize the new pattern based on adjacent occupied cells\X${}%
\E{}$\6
\X32:Find the proper \PB{\\{leftbound}} and \PB{\\{rightbound}}\X;\6
\&{if} ${}(\\{rightbound}-\\{leftbound}\G\\{maxlen}){}$\5
${}\{{}$\1\6
${}\|l\K\\{nmax}{}$;\5
\&{goto} \\{done};\C{ too long, so we make the cost huge  }\6
\4${}\}{}$\2\6
\X33:Find the new classes, or \PB{\&{goto} \\{done}}\X;\6
\X39:Determine the cost of the new pattern\X;\6
\4\\{done}:\par
\U27.\fi

\M{42}At this point the new classes of adjacent cells have already been
determined.
Interior cells cannot make \PB{$\\{rightbound}-\\{leftbound}\G\\{maxlen}$}.

\Y\B\4\X42:Canonize the new pattern based on adjacent and interior occupied
cells\X${}\E{}$\6
\X32:Find the proper \PB{\\{leftbound}} and \PB{\\{rightbound}}\X;\6
\X39:Determine the cost of the new pattern\X;\par
\U28.\fi

\M{43}And at this point the \PB{\\{leftbound}} and \PB{\\{rightbound}} are
already known.

\Y\B\4\X43:Canonize the new pattern based on all occupied cells\X${}\E{}$\6
\X39:Determine the cost of the new pattern\X;\par
\Us29\ET30.\fi

\N{1}{44}Loose ends. We have finished the complicated decision-making that goes
into listing all successors of a given slice, but we still haven't
actually generated any successors. Now we're ready to do that simple task,
thereby unmasking the mystery of \PB{\\{aux}}.

\Y\B\4\X44:Insert the new pattern into the successor list\X${}\E{}$\6
$\|q\K\\{lookup}(\\{len},\39\\{bestpat},\39\|m+\\{weight});{}$\6
\&{if} ${}(\|q\MG\\{aux}){}$\1\5
${}\|q\MG\\{aux}\MG\\{degree}\PP{}$;\C{ been there, done that }\2\6
\&{else}\5
${}\{{}$\1\6
${}\|s\K\\{succ\_ptr}\PP;{}$\6
${}\|s\MG\\{pat}\K\|q;{}$\6
${}\|q\MG\\{aux}\K\|s;{}$\6
${}\|s\MG\\{degree}\K\T{1};{}$\6
${}\|s\MG\\{weight}\K\\{weight};{}$\6
${}\|s\MG\\{death}\K\\{nmax}-\|l+\\{weight};{}$\6
\4${}\}{}$\2\par
\Us27, 28, 29\ETs30.\fi

\M{45}\B\X20:Local variables\X${}\mathrel+\E{}$\6
\&{register} \&{patt} ${}{*}\|q;{}$\6
\&{register} \&{succ} ${}{*}\|s{}$;\par
\fi

\M{46}The special successor pattern \PB{$\NULL$}, of cost 0, is added to the
list
if slice \PB{\|p} had only one component. This will be true if and only
if \PB{\\{appeared}[\T{2}]} is zero. It means, ``We can stop now if we like,
having
generated a polyomino of weight~\PB{\|m}.''

\Y\B\4\X46:List also the null successor, if appropriate\X${}\E{}$\6
\&{if} ${}(\R\\{appeared}[\T{2}]){}$\5
${}\{{}$\1\6
${}\|s\K\\{succ\_ptr}\PP;{}$\6
${}\|s\MG\\{pat}\K\NULL,\39\|s\MG\\{degree}\K\T{1},\39\|s\MG\\{weight}\K\T{0},%
\39\|s\MG\\{death}\K\|m;{}$\6
\4${}\}{}$\2\par
\U27.\fi

\M{47}Placing initial slices is complicated by the fact that we want to gain
a factor of two by symmetry. Thus if \PB{\|k} is palindromic, we start with
pattern \PB{\|k} itself; otherwise we consider \PB{\|k} and its reflection but
with double weight. The latter case is essentially the same as a degree-2
transition from the null state.

The total number of initial transitions, which is also the total number of
slices that will appear at level \PB{\\{nmax}}, is
$$2^{t-2}+\cases{ 2^{t/2}-1,&if $t$ is even,\cr
\noalign{\smallskip} 2^{(t-1)/2}+2^{(t-3)/2}-1,&if $t$ is odd,\cr}$$
where $t=\PB{\\{maxlen}}=\lceil\PB{\\{nmax}}/2\rceil$. For example, when \PB{%
\\{nmax}} is 30
this number is 8383.

\Y\B\4\X47:Place the initial slice corresponding to the binary number \PB{\|k}%
\X${}\E{}$\6
${}\{{}$\1\6
${}\|m\K\|k;{}$\6
\&{for} ${}(\|j\K\|l\K\T{0};{}$ \|m; ${}\|j\PP,\39\|m\MRL{{\GG}{\K}}\T{1}){}$\1%
\6
\&{if} ${}(\|m\AND\T{1}){}$\1\5
${}\\{newpat}[\|j]\K\PP\|l;{}$\2\6
\&{else}\1\5
${}\\{newpat}[\|j]\K\T{0};{}$\2\2\6
${}\\{len}\K\|j;{}$\6
${}\\{weight}\K\|l;{}$\6
\&{for} ${}(\|j\MM;{}$ ${}\|j\G\T{0};{}$ ${}\|j\MM,\39\|m\PP){}$\1\6
\&{if} (\\{newpat}[\|j])\1\5
${}\\{backpat}[\|m]\K\|l+\T{1}-\\{newpat}[\|j];{}$\2\6
\&{else}\1\5
${}\\{backpat}[\|m]\K\T{0};{}$\2\2\6
${}\\{mult}\K\T{1};{}$\6
\&{for} ${}(\|j\K\T{1};{}$ ${}\|j<\\{len};{}$ ${}\|j\PP){}$\1\6
\&{if} ${}(\\{newpat}[\|j]<\\{backpat}[\|j]){}$\5
${}\{{}$\1\6
${}\\{mult}\K\T{2};{}$\6
\&{break};\6
\4${}\}{}$\5
\2\&{else} \&{if} ${}(\\{newpat}[\|j]>\\{backpat}[\|j]){}$\1\5
\&{goto} \\{bypass};\2\2\6
\X57:Record an initial transition to \PB{\\{newpat}} with degree \PB{\\{mult}}%
\X;\6
\4\\{bypass}:\6
\4${}\}{}$\2\par
\U7.\fi

\M{48}\B\X6:Global variables\X${}\mathrel+\E{}$\6
\&{int} \\{len};\C{ the pattern length }\6
\&{int} \\{mult};\C{ its multiplicity }\par
\fi

\M{49}The heart of the computation is, of course, the process of generating the
non-initial slices.

\Y\B\4\X49:Compute\X${}\E{}$\6
\X7:Place the initial slices\X;\6
\&{for} ${}(\|m\K\T{1};{}$  ; ${}\|m\PP){}$\5
${}\{{}$\1\6
${}\\{printf}(\.{"\ \%d\ new\ patterns\ on}\)\.{\ level\ \%d\ (\%d,\%d)\\n"},%
\39\\{patt\_count}[\|m],\39\|m,\39\\{patt\_ptr}-\\{patt\_table},\39%
\\{arcs});{}$\6
\X56:Record the arrival of a new \PB{\|m}\X;\6
\&{if} ${}(\|m\E\\{nmax}){}$\1\5
\&{break};\2\6
\&{for} ${}(\|p\K\\{patt\_list}[\|m];{}$ \|p; ${}\|p\K\|p\MG\\{next}){}$\5
${}\{{}$\1\6
\X27:Generate all successors to \PB{\|p}\X;\6
\X50:Output all transitions from \PB{\|p}\X;\6
\4${}\}{}$\2\6
\4${}\}{}$\2\par
\U1.\fi

\M{50}\B\X50:Output all transitions from \PB{\|p}\X${}\E{}$\6
\X60:Record \PB{\|p} as the current predecessor\X;\6
\&{for} ${}(\|s\K\\{succ\_table};{}$ ${}\|s<\\{succ\_ptr};{}$ ${}\|s\PP){}$\5
${}\{{}$\1\6
\X61:Record a transition to \PB{\|s}\X;\6
\&{if} ${}(\|s\MG\\{pat}){}$\1\5
${}\|s\MG\\{pat}\MG\\{aux}\K\NULL;{}$\2\6
\4${}\}{}$\2\par
\U49.\fi

\N{1}{51}Output. Finally we must deal with transition records,
which are sent to a file for subsequent processing.
That file might be huge,
so it is generated in a compact binary format.
Each sequence of transitions from a pattern is specified by one word
identifying that pattern followed by one word for each successor pattern.

In fact several gigabytes of output are generated when \PB{$\\{nmax}\K\T{30}$},
and my Linux system frowns on files of length greater than
$2^{31}-1=2147483647$. Therefore this program breaks the output up
into a sequence of files called \.{foo.0}, \.{foo.1}, \dots, each
at most one large gigabyte in size. (That's one GGbyte${}=2^{30}$~bytes.)

If the special variable \PB{\\{verbose}} is nonzero, transitions are
also displayed in symbolic form on standard output.

\Y\B\4\D$\\{filelength\_threshold}$ \5
\T{\^10000000}\C{ in tetrabytes }\par
\Y\B\4\X6:Global variables\X${}\mathrel+\E{}$\6
\&{int} \\{verbose}${}\K\T{0}{}$;\C{ set nonzero for debugging }\6
\&{FILE} ${}{*}\\{out\_file}{}$;\C{ the output file }\6
\&{unsigned} \&{int} \\{buf};\C{ place for binary output }\6
\&{int} \\{words\_out};\C{ the number of tetrabytes output in current output
file }\6
\&{int} \\{file\_extension};\C{ the number of GGbytes output }\6
\&{char} ${}{*}\\{base\_name},{}$ \\{filename}[\T{100}];\par
\fi

\M{52}\B\X2:Subroutines\X${}\mathrel+\E{}$\6
\&{void} \\{open\_it}(\,)\1\1\2\2\6
${}\{{}$\1\6
${}\\{sprintf}(\\{filename},\39\.{"\%.90s.\%d"},\39\\{base\_name},\39\\{file%
\_extension});{}$\6
${}\\{out\_file}\K\\{fopen}(\\{filename},\39\.{"wb"});{}$\6
\&{if} ${}(\R\\{out\_file}){}$\5
${}\{{}$\1\6
${}\\{fprintf}(\\{stderr},\39\.{"I\ can't\ open\ file\ \%}\)\.{s"},\39%
\\{filename});{}$\6
\\{panic}(\.{"\ for\ output"});\6
\4${}\}{}$\2\6
${}\\{words\_out}\K\T{0};{}$\6
\4${}\}{}$\2\par
\fi

\M{53}\B\X2:Subroutines\X${}\mathrel+\E{}$\6
\&{void} \\{close\_it}(\,)\1\1\2\2\6
${}\{{}$\1\6
\&{if} ${}(\\{fclose}(\\{out\_file})\I\T{0}){}$\1\5
\\{panic}(\.{"I\ couldn't\ close\ th}\)\.{e\ output\ file"});\2\6
${}\\{printf}(\.{"[\%d\ bytes\ written\ o}\)\.{n\ file\ \%s.]\\n"},\39\T{4}*%
\\{words\_out},\39\\{filename});{}$\6
\4${}\}{}$\2\par
\fi

\M{54}\B\X2:Subroutines\X${}\mathrel+\E{}$\6
\&{int} \\{out\_it}(\,)\1\1\2\2\6
${}\{{}$\1\6
\&{if} ${}(\\{words\_out}\E\\{filelength\_threshold}){}$\5
${}\{{}$\1\6
\\{close\_it}(\,);\6
${}\\{file\_extension}\PP;{}$\6
\\{open\_it}(\,);\6
\4${}\}{}$\2\6
${}\\{words\_out}\PP;{}$\6
\&{return} \\{fwrite}${}({\AND}\\{buf},\39{}$\&{sizeof}(\&{unsigned} %
\&{int})${},\39\T{1},\39\\{out\_file})\E\T{1};{}$\6
\4${}\}{}$\2\par
\fi

\M{55}\B\X19:Initialize\X${}\mathrel+\E{}$\6
\\{open\_it}(\,);\par
\fi

\M{56}How should we encode the binary output? Each transition has a
multiplicity,
and the multiplicity can get as large as~30.
Therefore we will devote 5~bits to that piece of information.
We can also give special meaning to the code numbers 0 and 31
if they happen to appear in the high-order 5 bits of a 32-bit word.

(At first I thought the maximum multiplicity was 16, because of examples like
\lab1020304050607:7/${}\to{}$\lab1:30/ or
\lab11223344556677:21/${}\to{}$\lab1:30/ or
\lab101010101010101:29/${}\to{}$\lab1:30/.
But then I realized that there are 28 ways
to go from \lab1:1/ to \lab10203040506007:8/ or to \lab10203040506077:9/,
because of the way we collapse symmetric slices together.
Still later I encountered the examples
\lab11:3/${}\to{}$\lab10002003004005:8/,
\lab11:3/${}\to{}$\lab10020003004005:8/,
\lab1011:7/${}\to{}$\lab100000200003:10/,
\lab1111:7/${}\to{}$\lab100000200003:10/.
I~believe these are the only four cases of degree $\ge30$, but the
program now checks explicitly to make sure that I haven't miscalculated again.)

For the main body of information, we can take advantage of the fact that
new patterns arise consecutively. Thus if bit~6 is~0, it means,
``The successor is the next new pattern; here are its birth and death dates.''
But if bit~6 is~1 it means, ``The low-order 26 bits are the serial number of
the successor pattern, whose birth and death dates you already know.''

\Y\B\4\D$\\{new\_pred\_code}$ \5
\T{0}\C{ high 5 when new slice is the predecessor }\par
\B\4\D$\\{new\_level\_code}$ \5
\T{31}\C{ high 5 when $m$ increases }\par
\Y\B\4\X56:Record the arrival of a new \PB{\|m}\X${}\E{}$\6
$\\{buf}\K(\\{new\_level\_code}\LL\T{27})+\|m;{}$\6
\&{if} ${}(\R\\{out\_it}(\,)){}$\1\5
\\{panic}(\.{"Bad\ write\ of\ newlev}\)\.{el\ message"});\2\par
\U49.\fi

\M{57}The first pattern has serial number 1, not 0, because we let 0 stand for
the sink vertex.

\Y\B\4\D$\\{patt\_code}(\|q)$ \5
$(((\|q)-\\{patt\_table})+\T{1}{}$)\par
\Y\B\4\X57:Record an initial transition to \PB{\\{newpat}} with degree \PB{%
\\{mult}}\X${}\E{}$\6
$\|l\K\\{nmax}-\\{conn\_distance}(\\{len},\39\\{newpat});{}$\6
${}\|q\K\\{lookup}(\\{len},\39\\{newpat},\39\\{weight});{}$\6
\&{if} ${}(\\{patt\_code}(\|q)\I\PP\\{prev\_pat}){}$\1\5
\\{panic}(\.{"Out\ of\ sync"});\2\6
\&{if} (\\{verbose})\5
${}\{{}$\1\6
${}\\{printf}(\.{"-\%s>"},\39\\{mult}\E\T{2}\?\.{"2"}:\\{mult}\I\T{1}\?\.{"?"}:%
\.{""});{}$\6
${}\\{print\_slice}(\|q,\39\\{weight},\39\|l);{}$\6
\4${}\}{}$\2\6
${}\\{buf}\K(\\{mult}\LL\T{27})+(\\{weight}\LL\T{8})+\|l{}$;\C{ multiplicity,
birth, death }\6
\&{if} ${}(\R\\{out\_it}(\,)){}$\1\5
\\{panic}(\.{"Bad\ write\ of\ initia}\)\.{l\ transition"});\2\6
${}\\{slices}\MRL{+{\K}}\|l-\\{weight}+\T{1}{}$;\par
\U47.\fi

\M{58}\B\X6:Global variables\X${}\mathrel+\E{}$\6
\&{int} \\{prev\_pat};\C{ the number of patterns encountered so far }\par
\fi

\M{59}The verbose output marks the first appearance of a slice by
printing both birth and death dates as a range of subscripts.
Later it will use a single subscript within that interval. (For
example, if \PB{$\\{nmax}\K\T{30}$} the simple pattern \.1 will be shown
first as `\.{1:1..30}', and the pattern \.{12} will be shown
first as `\.{12:2..29}'. But later when \.1 occurs as a successor
of pattern \.{12}, it will show up as `\.{1:3}', meaning that \lab1:3/
is a successor of \lab12:2/, \lab1:4/ is a successor of \lab12:3/, etc.)

\Y\B\4\X2:Subroutines\X${}\mathrel+\E{}$\6
\&{void} \\{print\_slice}(\&{patt} ${}{*}\|p,\39{}$\&{int} \|m${},\39{}$\&{int}
\\{death})\1\1\2\2\6
${}\{{}$\1\6
\&{register} \&{int} \|j;\7
\&{for} ${}(\|j\K\T{0};{}$ ${}\|j<\\{length}(\|p\MG\\{key});{}$ ${}\|j\PP){}$\1%
\5
${}\\{printf}(\.{"\%x"},\39\|j\E\T{0}\?\T{1}:\|j\AND\T{1}\?\|p\MG\\{key}.%
\\{bytes}[(\|j+\T{1})\GG\T{1}]\GG\T{4}:\|p\MG\\{key}.\\{bytes}[\|j\GG\T{1}]\AND%
\T{\^f});{}$\2\6
\&{if} (\\{death})\1\5
${}\\{printf}(\.{":\%d..\%d\\n"},\39\|m,\39\\{death});{}$\2\6
\&{else}\1\5
${}\\{printf}(\.{":\%d\\n"},\39\|m);{}$\2\6
\4${}\}{}$\2\par
\fi

\M{60}\B\X60:Record \PB{\|p} as the current predecessor\X${}\E{}$\6
${}\{{}$\1\6
\&{if} (\\{verbose})\1\5
${}\\{print\_slice}(\|p,\39\|m,\39\T{0});{}$\2\6
${}\\{buf}\K(\\{new\_pred\_code}\LL\T{27})+\\{patt\_code}(\|p);{}$\6
\&{if} ${}(\R\\{out\_it}(\,)){}$\1\5
\\{panic}(\.{"Bad\ write\ of\ predec}\)\.{essor\ pattern"});\2\6
\4${}\}{}$\2\par
\U50.\fi

\M{61}\B\X61:Record a transition to \PB{\|s}\X${}\E{}$\6
${}\{{}$\1\6
\&{if} (\\{verbose})\5
${}\{{}$\1\6
\&{if} ${}(\|s\MG\\{degree}\E\T{1}){}$\1\5
\\{printf}(\.{"->"});\2\6
\&{else}\1\5
${}\\{printf}(\.{"-\%d>"},\39\|s\MG\\{degree});{}$\2\6
\&{if} ${}(\R\|s\MG\\{pat}){}$\1\5
${}\\{printf}(\.{"0:\%d\\n"},\39\|m);{}$\2\6
\&{else}\1\5
${}\\{print\_slice}(\|s\MG\\{pat},\39\|m+\|s\MG\\{weight},\39\\{patt\_code}(\|s%
\MG\\{pat})>\\{prev\_pat}\?\|s\MG\\{death}:\T{0});{}$\2\6
\4${}\}{}$\2\6
\&{if} ${}(\|s\MG\\{degree}>\T{30}){}$\1\5
\\{panic}(\.{"Surprisingly\ large\ }\)\.{arc\ multiplicity"});\2\6
\&{if} ${}(\R\|s\MG\\{pat}){}$\1\5
${}\\{buf}\K(\T{1}\LL\T{27})+(\T{1}\LL\T{26});{}$\2\6
\&{else} \&{if} ${}(\\{patt\_code}(\|s\MG\\{pat})\Z\\{prev\_pat}){}$\1\5
${}\\{buf}\K(\|s\MG\\{degree}\LL\T{27})+(\T{1}\LL\T{26})+\\{patt\_code}(\|s\MG%
\\{pat});{}$\2\6
\&{else}\5
${}\{{}$\1\6
${}\\{prev\_pat}\PP,\39\\{buf}\K(\|s\MG\\{degree}\LL\T{27})+((\|s\MG\\{weight}+%
\|m)\LL\T{8})+\|s\MG\\{death};{}$\6
${}\\{slices}\MRL{+{\K}}\|s\MG\\{death}-(\|s\MG\\{weight}+\|m)+\T{1};{}$\6
\4${}\}{}$\2\6
\&{if} ${}(\R\\{out\_it}(\,)){}$\1\5
\\{panic}(\.{"Bad\ write\ of\ transi}\)\.{tion"});\2\6
${}\\{arcs}\PP;{}$\6
\4${}\}{}$\2\par
\U50.\fi

\M{62}Hooray, we are done.

\Y\B\4\X62:Print the results\X${}\E{}$\6
\&{if} ${}(\\{patt\_ptr}\I\\{patt\_table}+\\{prev\_pat}){}$\1\5
\\{panic}(\.{"Output\ out\ of\ sync"});\2\6
\\{printf}(\.{"All\ done!\\n"});\6
${}\\{printf}(\.{"\ \%d\ patterns\ genera}\)\.{ted,"},\39\\{prev\_pat});{}$\6
${}\\{printf}(\.{"\ \%d\ slices,"},\39\\{slices});{}$\6
${}\\{printf}(\.{"\ \%d\ arcs.\\n"},\39\\{arcs});{}$\6
\\{close\_it}(\,);\par
\U1.\fi

\M{63}Multiplicity of arcs is not taken into account.

\Y\B\4\X6:Global variables\X${}\mathrel+\E{}$\6
\&{int} \\{slices};\C{ total number of slices }\6
\&{int} \\{arcs};\C{ total number of arcs (double precision) }\par
\fi

\M{64}Note, added two weeks later: This program, though interesting, is
obsolete. See the much better {\mc POLYNUM}, which runs hundreds of
times faster when $n=30$ and faster yet for larger values of $n$.

\fi

\N{1}{65}Index.
\fi

\inx
\fin
\con
