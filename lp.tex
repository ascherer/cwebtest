\input cwebmac
\srcloctrue
\datethis

\N[2 lp.w]{1}{1}Introduction. I'm writing this program in order to gain
personal
experience implementing the simplex algorithm---even though I know
that commercial codes do the job much, much better.
My aim is to learn, to make the logic ``crystal clear'' if not
lightning fast, and perhaps also to watch and
interact with this magical process.

The computational task to be solved has been traditionally called
Linear Programming, and it takes many forms. The particular case
considered here is to maximize $b_1v_1+\cdots+b_nv_n$ subject to the
constraints $v_j\ge0$ for $1\le j\le n$ and
$a_{i1}v_1+\cdots+a_{in}v_n\le c_i$ for $1\le i\le m$,
where $a_{ij}$ is a given $m\times n$ matrix of integers
and the other parameters $b_j$, $c_i$ are integers with $c_i\ge0$.
For example, what is the maximum of $v_1+3v_2-v_3$, if we require that
$$3v_1-v_2+5v_3\le 8,\qquad
-v_1+2v_2-v_3\le 1,\qquad
2v_1+4v_2+v_3\le 5,$$
and $v_1,v_2,v_3\ge0$? [I took this example from Muroga's book
on threshold logic (1971).] The fact that the constants $c_i$ on the
right-hand side are nonnegative means that the trivial values
$v_1=\cdots=v_n=0$ always satisfy the constraints; hence the
maximum is always nonnegative.

The algorithm below also solves a ``dual'' problem as a special bonus.
Namely, it tells us how to minimize the quantity
$c_1u_1+\cdots+c_mu_m$ subject to $u_i\ge 0$ and
$a_{1j}u_1+\cdots+a_{mj}u_m\ge b_j$ for $1\le j\le n$.

There may no values $(u_1,\ldots,u_m)$ that meet those dual constraints.
In such a case, the algorithm will effectively prove the impossibility,
and it will also demonstrate that the maximum in the original problem
is $+\infty$. (For example, suppose $m=n=1$, $b_1=1$, $c_1=0$, and
$a_{11}=-1$. The problem of maximizing $v_1$ subject to $-v_1\le 0$
and $v_1\ge0$ obviously has $+\infty$ as its answer; and $+\infty$ is
also the ``minimum'' of the quantity~0 taken over all $u_1$ such
that $u_1\ge0$ and $-u_1\ge1$, because no such numbers $u_1$ exist.)

The first line of the standard input file should contain
the integers $b_1$ $b_2$ \dots~$b_n$ in decimal notation, separated
by spaces. That first line should be followed by
$m$ further lines that each contain $n+1$ integer values $c_i$ $a_{i1}$
$a_{i2}$ \dots~$a_{in}$, for $1\le i\le m$.

To enhance this learning experience,
I'm solving the problem both with floating-point arithmetic and
with an all-integer method that produces rational numbers as output.
Hopefully the two answers will agree. But the all-integer method might
overflow, and the floating-point method may suffer from rounding errors.

\Y\B\4\D$\\{maxm}$ \5
\T{10}\C{ of course these limits can be raised if desired }\par
\B\4\D$\\{maxn}$ \5
\T{100}\C{ (up to a point) }\par
\B\4\D$\\{buf\_size}$ \5
\.{BUFSIZ}\par
\Y\B\8\#\&{include} \.{<stdio.h>}\6
\8\#\&{include} \.{<ctype.h>}\6
\X2:Include tricky code for zapping\X\7
\&{typedef} \&{long} \&{intword};\C{ will be \PB{\&{long}} \PB{\&{long}} on my
other computer }\6
\&{char} \\{buf}[\\{buf\_size}];\6
\&{intword} ${}\|a[\\{maxm}+\T{1}][\\{maxm}+\\{maxn}+\T{1}]{}$;\C{ integer work
area }\6
\&{intword} ${}\\{denom}[\\{maxm}+\\{maxn}+\T{1}]{}$;\C{ scale factors }\6
\&{double} ${}\\{aa}[\\{maxm}+\T{1}][\\{maxm}+\\{maxn}+\T{1}]{}$;\C{
floating-point work area }\6
\&{double} ${}\\{trial}[\\{maxm}+\\{maxn}+\T{1}]{}$;\C{ pivot testing area }\6
\&{int} \\{verbose}${}\K\T{1}{}$;\C{ can be set positive for extra output }\6
\&{int} \\{count};\C{ the number of steps taken so far }\6
\&{int} ${}\|p[\\{maxm}+\T{1}],{}$ ${}\|q[\\{maxm}+\\{maxn}+\T{1}]{}$;\C{
current basis and inverse }\7
\\{main}(\,)\1\1\2\2\6
${}\{{}$\1\6
\&{register} \&{intword} \|h${},{}$ \|i${},{}$ \|j${},{}$ \|k${},{}$ \|l${},{}$
\|m${},{}$ \|n${},{}$ \|s;\6
\&{register} \&{double} \|z;\7
\X3:Check the zap trick\X;\6
\X4:Read the input matrix\X;\6
\X13:Solve the problem\X;\6
\4${}\}{}$\2\par
\fi

\M[80 lp.w]{2}The algorithm is very sensitive to zeroness or nonzeroness of
numbers.
I'll try to avoid problems with floating-point roundoff by zapping
anything near 0.0 to 0.0. For speed, I do this in ``machine language.''

\Y\B\4\X2:Include tricky code for zapping\X${}\E{}$\6
\8\#\&{define} \\{little\_endian} \5\T{1}\C{ on less crazy machines I would
define `\PB{\\{big\_endian}}' instead }\6
\8\#\&{ifdef} \\{big\_endian}\6
\8\#\&{define} \\{bigend} \5\\{first}\6
\8\#\&{else}\6
\8\#\&{define} \\{bigend} \5\\{second}\6
\8\#\&{endif}\6
\8\#\&{define} \\{zap}\5
\&{if} ${}((\\{zbuf}.\\{uint}.\\{bigend}\AND\T{\^7fffffff})<\T{\^3e000000}){}$%
\1\5
${}\\{zbuf}.\\{dbl}\K\T{0.0};{}$\2\7
\&{union} ${}\{{}$\1\6
\&{struct} ${}\{{}$\5
\1\&{unsigned} \&{int} \\{first}${},{}$ \\{second};\5
\2${}\}{}$ \\{uint};\6
\&{double} \\{dbl};\2\6
${}\}{}$ \\{zbuf};\par
\U1.\fi

\M[99 lp.w]{3}Here's a test that my intentions are being fulfilled by that
trickery.

\Y\B\4\X3:Check the zap trick\X${}\E{}$\6
$\\{zbuf}.\\{dbl}\K\T{.000000001}{}$;\C{ this value should not be zapped }\6
\\{zap};\6
\&{if} ${}(\\{zbuf}.\\{dbl}){}$\5
${}\{{}$\1\6
${}\\{zbuf}.\\{dbl}\K{-}\T{.0000000001}{}$;\C{ but this one should }\6
\\{zap};\6
\&{if} ${}(\R\\{zbuf}.\\{dbl}){}$\1\5
\&{goto} \\{zap\_OK};\2\6
\4${}\}{}$\2\6
${}\\{fprintf}(\\{stderr},\39\.{"Zapping\ doesn't\ wor}\)\.{k!\\n"});{}$\6
${}\\{exit}({-}\T{666});{}$\6
\4\\{zap\_OK}:\par
\U1.\fi

\M[113 lp.w]{4}\B\X4:Read the input matrix\X${}\E{}$\6
\&{for} ${}(\|i\K\|n\K\T{0};{}$  ; ${}\|i\PP){}$\5
${}\{{}$\1\6
\&{if} ${}(\R\\{fgets}(\\{buf},\39\\{buf\_size},\39\\{stdin})){}$\1\5
\&{break};\2\6
\&{if} ${}(\|i>\\{maxm}){}$\5
${}\{{}$\1\6
${}\\{fprintf}(\\{stderr},\39\.{"Sorry,\ I'm\ set\ up\ o}\)\.{nly\ for\ m<=\%d!%
\\n"},\39\\{maxm});{}$\6
${}\\{exit}({-}\T{9});{}$\6
\4${}\}{}$\2\6
\&{for} ${}(\|k\K\T{0},\39\|j\K(\|i\E\T{0});{}$ \\{buf}[\|k]; \,)\5
${}\{{}$\1\6
\&{while} ${}(\\{isspace}(\\{buf}[\|k])\W\\{buf}[\|k]\I\.{'\\n'}){}$\1\5
${}\|k\PP;{}$\2\6
\&{if} ${}(\\{buf}[\|k]\E\.{'\\n'}){}$\1\5
\&{break};\2\6
\&{if} ${}(\\{buf}[\|k]\E\.{'-'}){}$\1\5
${}\|l\K\T{1},\39\|k\PP{}$;\5
\2\&{else}\1\5
${}\|l\K\T{0};{}$\2\6
\&{for} ${}(\|s\K\T{0};{}$ ${}\\{buf}[\|k]\G\.{'0'}\W\\{buf}[\|k]\Z\.{'9'};{}$
${}\|k\PP){}$\1\5
${}\|s\K\T{10}*\|s+\\{buf}[\|k]-\.{'0'};{}$\2\6
${}\|a[\|i][\|j\PP]\K(\|l\?{-}\|s:\|s);{}$\6
\4${}\}{}$\2\6
\&{if} ${}(\R\\{buf}[\|k]){}$\5
${}\{{}$\C{ no end-of-line in the buffer }\1\6
${}\\{fprintf}(\\{stderr},\39\.{"Input\ line\ too\ long}\)\.{!\ (\%s...)\\n"},%
\39\\{buf});{}$\6
${}\\{exit}({-}\T{1});{}$\6
\4${}\}{}$\2\6
\&{if} ${}(\|i\E\T{0}){}$\5
${}\{{}$\1\6
${}\|n\K\|j-\T{1};{}$\6
\&{if} ${}(\|n>\\{maxn}){}$\5
${}\{{}$\1\6
${}\\{fprintf}(\\{stderr},\39\.{"Sorry,\ I'm\ set\ up\ o}\)\.{nly\ for\ n<=\%d,%
\ not\ n}\)\.{=\%d!\\n"},\39\\{maxn},\39\|n);{}$\6
${}\\{exit}({-}\T{2});{}$\6
\4${}\}{}$\2\6
\4${}\}{}$\5
\2\&{else}\5
${}\{{}$\1\6
\&{if} ${}(\|n\I\|j-\T{1}){}$\5
${}\{{}$\1\6
${}\\{fprintf}(\\{stderr},\39\.{"Row\ \%d\ should\ have\ }\)\.{\%d\ numbers!%
\\n>\%s"},\39\|i,\39\|n+\T{1},\39\\{buf});{}$\6
${}\\{exit}({-}\T{3});{}$\6
\4${}\}{}$\2\6
\&{if} ${}(\|a[\|i][\|j]<\T{0}){}$\5
${}\{{}$\1\6
${}\\{fprintf}(\\{stderr},\39\.{"Row\ \%d's\ constant\ t}\)\.{erm\ shouldn't\
be\ neg}\)\.{ative!\\n>\%s"},\39\|i,\39\\{buf});{}$\6
${}\\{exit}({-}\T{4});{}$\6
\4${}\}{}$\2\6
\4${}\}{}$\2\6
\4${}\}{}$\2\6
${}\|m\K\|i-\T{1}{}$;\par
\U1.\fi

\N[152 lp.w]{1}{5}The algorithm: An example. The famous simplex procedure is
subtle
yet not difficult to fathom, even when we are careful to avoid infinite
loops. But I always tend to forget the details a short time after seeing them
explained in a book. Therefore I will try here to present the algorithm
in my own favorite way---which tends to be algebraic and combinatoric rather
than geometric---in hopes that the ideas will then be forever memorable,
at least in my own mind. I'm not going to explain how the algorithm was
invented, but I am going to explain how and why it works.

To clarify the exposition, I'll work through the simple example in the
introduction, watching each step in slow motion. That problem was
to maximize the quantity $v_1+3v_2-v_3$ over all nonnegative real values
$(v_1,v_2,v_3)$ for which
$$3v_1-v_2+5v_3\le 8,\qquad
-v_1+2v_2-v_3\le 1,\qquad
2v_1+4v_2+v_3\le 5.$$

\fi

\M[169 lp.w]{6}The first step in the simplex method is to introduce $m$
additional
nonnegative variables called ``slack variables'' $(s_1,\ldots,s_m)$,
which allow us to deal with equivalities instead of inequalities.
For example, the inequality $3v_1-v_2+5v_3\le 8$ becomes the equality
$s_1+3v_1-v_2+5v_3=8$ when we add in the nonnegative variable~$s_1$.
These slack variables lead to $m$ further columns of the input
matrix $a_{ij}$; we conceptually place these columns between
the constant terms $c_i$ and the other coefficients $a_{i1}$,
\dots,~$a_{in}$.

The state of the computation is conveniently represented in a
working array with $m+1$ rows and $m+n+1$ columns. In our example,
the working array takes the following form:
$$\def\\{\phantom-}
\vcenter{\halign{\hfil$#$\hfil\enspace&&\quad\hfil$#$\hfil\cr
\rm Row&(-1)&(s_1)&(s_2)&(s_3)&(v_1)&(v_2)&(v_3)\cr
\noalign{\vskip2pt}
0& 0& 0& 0& 0&-1&-3&\\1\cr
1& 8& 1& 0& 0&\\3&-1&\\5\cr
2& 1& 0& 1& 0&-1&\\2&-1\cr
3& 5& 0& 0& 1&\\2&\\4&\\1\cr}}$$
Rows 1 through $m$ represent the given equations, with coefficients
to be multiplied by the column labels. For example, the bottom row
represents the equation $s_3+2v_1+4v_2+v_3=5$, namely
$$5(-1)+0(s_1)+0(s_2)+1(s_3)+2(v_1)+4(v_2)+1(v_3)\;=\;0.$$
The top row, row~0, is somewhat special, but it also represents an
equation---in this case
$$v_1+3v_2-v_3+0(-1)+0(s_1)+0(s_2)+0(s_3)-1(v_1)-3(v_2)+1(v_3)\;=\;0.$$

Let $X$ be the set of all nonnegative vectors $(s_1,s_2,s_3,v_1,v_2,v_3)$
that satisfy the equations in rows 1, 2, and~3. The simplex algorithm
will transform the array in such a way that the rows change, but the
new set of rows will still define exactly the same set~$X$. Furthermore
the special equation represented by row~0 will also remain true, as
we will see below.

\fi

\M[206 lp.w]{7}The algorithm also maintains two other invariants. First, there
will always
be $m$ special columns called ``basis columns,'' one for each index~$i$
in the range $1\le i\le m$. All entries of basis column~$i$ are~0
except for a 1 in row~$i$. When we begin, the initial basis columns are those
for slack variables, labeled $(s_1)$ through $(s_m)$.

Second, all rows except row~0 will remain {\it lexicographically
nonnegative}. In other words, the leftmost nonzero entry of row~$i$
will always be positive, for $1\le i\le m$. (That row might begin with
many zeros, but it cannot be
entirely zero, because of the~1 in its basis column.)

\fi

\M[219 lp.w]{8}Elementary row operations on this array do not change the
set~$X$.
Namely, we can multiply each element of row~$i$ by a nonzero constant,
when $i\ne0$; and we can also add any multiple of row~$i\ne0$ to any
other row $j\ne i$, even when $j=0$.

Of course such row operations might mess up the basis columns. But a suitable
combination of row operations, called a ``pivot step,''
does allow us to change the position of a basis column. For example,
suppose we multiply row~2 by $-1$, then add multiples of that row to the other
rows so as to zero out the other entries in column~$(v_1)$; we get
a new basis column for row~2:
$$\def\\{\phantom-}
\vcenter{\halign{\hfil$#$\hfil\enspace&&\quad\hfil$#$\hfil\cr
\rm Row&(-1)&(s_1)&(s_2)&(s_3)&(v_1)&(v_2)&(v_3)\cr
\noalign{\vskip2pt}
0&-1& 0&-1& 0& 0&-5&\\2\cr
1&\\\llap11&1&\\3& 0& 0&\\5&\\2\cr
2&-1& 0&-1& 0& 1&-2&\\1\cr
3&\\7& 0&\\2& 1& 0&\\8&-1\cr}}$$
However, we don't really want to do this, because row~2 has now become
lexicographically negative.

Going back to the former matrix, let's try pivoting in column~$(v_1)$ on row~1
instead of row~2. This time we divide row~1 by~3 and add suitable
multiples to rows 0,~2, and~3, obtaining
$$\def\\{\phantom-}
\vcenter{\halign{\hfil$#$\hfil\enspace&&\quad\hfil$#$\hfil\cr
\rm Row&(-1)&(s_1)&(s_2)&(s_3)&(v_1)&(v_2)&(v_3)\cr
\noalign{\vskip2pt}
0&8/3&1/3&0&0&0&-10/3&\\8/3\cr
1&8/3&1/3&0&0&1&-1/3&\\5/3\cr
2&11/3&1/3&1&0&0&5/3&\\2/3\cr
3&\phantom0\llap{$-$}2/3&0&7&1&0&14/3&-7/3\cr}}$$
Hmm; it's still no good. Now the {\it bottom\/} row is giving trouble.

If we want column $(v_1)$ to
move into the basis, our only hope is to pivot on row~3. That works:
$$\def\\{\phantom-}
\vcenter{\halign{\hfil$#$\hfil\enspace&&\quad\hfil$#$\hfil\cr
\rm Row&(-1)&(s_1)&(s_2)&(s_3)&(v_1)&(v_2)&(v_3)\cr
\noalign{\vskip2pt}
0&5/2&0&0&\\1/2&0&-1&\\3/2\cr
1&1/2&1&0&-3/2&0&-7&\\7/2\cr
2&7/2&0&1&\\1/2&0&\\4&-1/2\cr
3&5/2&0&0&\\1/2&1&\\2&\\1/2\cr}}$$
All of the invariants mentioned above have now been preserved. And we've moved
the basis, thereby obtaining a new way to look at the set~$X$ over
which we wish to take a maximum.

\fi

\M[269 lp.w]{9}But oops, we've now run into fractions instead of nice, simple
integers.
No problem: We can also scale the columns, by changing the labels:
$$\def\\{\phantom-}
\vcenter{\halign{\hfil$#$\hfil\enspace&&\quad\hfil$#$\hfil\cr
\rm Row&(-1/2)&(s_1)&(s_2)&(s_3/2)&(v_1)&(v_2)&(v_3/2)\cr
\noalign{\vskip2pt}
0&5&0&0&\\1&0&-1&\\3\cr
1&1&1&0&-3&0&-7&\\7\cr
2&7&0&1&\\1&0&\\4&-1\cr
3&5&0&0&\\1&1&\\2&\\1\cr}}$$

\fi

\M[280 lp.w]{10}Okay, let's continue.
Another pivot operation, in column $(v_2)$ and row 2, followed by
another rescaling, yields
$$\def\\{\phantom-}
\vcenter{\halign{\hfil$#$\hfil\enspace&&\quad\hfil$#$\hfil\cr
\rm Row&(-1/8)&(s_1)&(s_2/4)&(s_3/8)&(v_1)&(v_2)&(v_3/8)\cr
\noalign{\vskip2pt}
0&27&0&\\1&\\5&0&0&\\\llap11\cr
1&53&1&\\7&-5&0&0&\\\llap21\cr
2&7&0&\\1&\\1&0&1&-1\cr
3&6&0&-2&\\2&1&0&\\6\cr}}$$

\fi

\M[292 lp.w]{11}And now we're done! From this tableau we can read off the
answer to the
maximization problem, namely that the desired maximum of $v_1+3v_2-v_3$ is
$27/8$, and that it is obtained when $v_1=6/8$, $v_2=7/8$, and $v_3=0$.

Buy why are we done, you ask? Good question. Here's why: First, by looking at
the three basis columns, we see that the point $(s_1,s_2,s_3,v_1,v_2,v_3)=
(53/8,0,0,6/8,7/8,0)$ is in~$X$, since rows 1, 2, and~3 represent equations
that hold everywhere in that set.

Second, row 0 also represents a valid equation, namely
$$v_1+3v_2-3v_3+27(-1/8)+0(s_1)+1/(s_2/4)+5(s_3/8)+0(v_1)+0(v_2)+11(v_3/8)
\;=\;0.$$
(Think about it: Pivot steps don't change the validity of the equation
represented by row~0, which always is prefaced by `$v_1+3v_2-v_3$',
since they simply add or subtract multiples of zero.)

Thus, at all points $(s_1,s_2,s_3,v_1,v_2,v_3)$ of $X$,
the value of $v_1+3v_2-3v_3$ is equal to
${27\over8}-{1\over4}s_2-{5\over8}s_3-{11\over8}v_3$; it can't possibly
get any lower than 27/8.

\fi

\M[313 lp.w]{12}There's more good news, too: The solution to the corresponding
{\it minimization\/} problem, namely to minimize $8u_1+u_2+5u_3$ subject to
$u_1,u_2,u_3\ge0$ and
$$3u_1-u_2+2u_3\ge1,\qquad
-u_1+2u_2+4u_3\ge3,\qquad
5u_1-u_2+u_3\ge-1,$$
can {\it also\/} be read from our final tableau,
by looking at the slack-variable
columns of row~0. The minimum value 27/8 is attained when $u_1=0$,
$u_2=1/4$, and $u_3=5/8$.

Again you ask, why? Again it's a good question. In the first place we can
use the $v$'s from the maximization problem to prove that 27/8 is
indeed unbeatable, because the inequalities
$${6\over8}\bigl(3u_1-u_2+2u_3\bigr)\ge{6\over8},\qquad
{7\over8}\bigl(-u_1+2u_2+4u_3\bigr)\ge{21\over8},\qquad
0\bigl(5u_1-u_2+u_3\bigr)\ge0$$
can be added to give ${11\over8}u_1+u_2+5u_3\ge{27\over8}$; increasing
${11\over8}u_1$ to $8u_1$ can't make the value any smaller. In~general,
if we multiply the inequalities that affect $(u_1,u_2,u_3)$ by any
values $(v_1,v_2,v_3)\in X$, we obtain a lower bound
$8u_1+u_2+5u_3\ge t_1u_1+t_2u_2+t_3u_3\ge v_1+3v_2-v_3$, because
$t_1\le 8$, $t_2\le1$, and $t_3\le 5$.

Why, however, do the values $(u_1,u_2,u_3)$ from the slack-variable columns
of row~0 actually attain the best lower bound? To understand the answer,
let's look at the final tableau {\it without\/} suppressing the
scale factors that were used to avoid fractions:
$$\def\\{\phantom-}
\vcenter{\halign{\hfil$#$\hfil\enspace&&\quad\hfil$#$\hfil\cr
\rm Row&(-1)&(s_1)&(s_2)&(s_3)&(v_1)&(v_2)&(v_3)\cr
\noalign{\vskip2pt}
0&27/8&0&\\1/4&\\5/8&0&0&\\\llap11/8\cr
1&53/8&1&\\7/4&-5/8&0&0&\\\llap21/8\cr
2&7/8&0&\\1/4&\\1&0&1&-1/8\cr
3&3/4&0&-1/2&\\1/4&1&0&\\3/4\cr}}$$
Consider especially the $(m+1)\times m$ submatrix in the slack columns;
these entries encapsulate the effects of all the pivot steps that brought us to
the present state. Namely, we replaced row~0 by $r_0+0r_1+{1\over4}r_2+
{5\over8}r_3$, where $r_0$, $r_1$, $r_2$, and $r_3$ are the original
contents of those rows. (We also replaced row~1 by ${7\over4}r_1-{5\over8}r_2$;
we replaced row~2 by ${1\over4}r_1+{1\over8}r_2$; and
we replaced row~3 by $-{1\over2}r_1+{1\over4}r_2+r_3$. These coefficients
should suffice to convince a skeptic that our final tableau does follow
from the original constraints, without forcing him or her to replay
the actual pivot steps.)

In particular, we got the number ${27\over8}$ in the constant column
by adding $0\cdot 5+{1\over4}\cdot1+{5\over8}\cdot5$.

\fi

\N[363 lp.w]{1}{13}The algorithm: An implementation.
We've been looking at a small example, but our reasoning has been perfectly
general. The main point is that we were able to find a sequence of pivot steps
that preserved the desired invariants and that also led to a tableau
in which row~0 had no negative entries. Whenever such a tableau is
found, we have solved the maximization problem for $(v_1,\ldots,v_n)$
as well as the minimization problem for $(u_1,\ldots,u_m)$.

\Y\B\4\X13:Solve the problem\X${}\E{}$\6
\X14:Set up the initial tableaux\X;\6
\4\\{loop}:\5
\&{if} (\\{verbose})\1\5
\X22:Print out the current state\X;\2\6
\&{for} ${}(\|j\K\|m+\|n;{}$ ${}\|j>\T{0};{}$ ${}\|j\MM){}$\1\6
\&{if} ${}(\|a[\T{0}][\|j]<\T{0}){}$\5
${}\{{}$\1\6
\X15:Try to pivot in column $j$\X;\6
${}\\{count}\PP;{}$\6
\&{goto} \\{loop};\6
\4${}\}{}$\2\2\6
\X23:Report the answers\X;\par
\U1.\fi

\M[381 lp.w]{14}Instead of distinguishing slack variables $s_i$ from ordinary
variables
$v_j$, we will henceforth call the variables $w_1$, \dots,~$w_{m+n}$,
with $w_i=s_i$ for $1\le i\le m$ and $w_{m+j}=v_j$ for $1\le j\le n$.

Here we set up an $(m+1)\times(m+n+1)$ working tableau~$a$ of integers, as
well as a table of scale factors. A~column label like $(s_2/4)$ in the
example above will be represented by \PB{$\\{denom}[\T{2}]\K\T{4}$} in this
program.

A floating-point tableau \PB{\\{aa}} is also inaugurated here, since we will
compute everything in two ways.

The current basis is represented by arrays \PB{\|p} and \PB{\|q}. If the basis
column for row~$i$ is column~$j$, we have \PB{$\|p[\|i]\K\|j$} and \PB{$\|q[%
\|j]\K\|i$}.
Other entries of the \PB{\|q} array are zero.

\Y\B\4\X14:Set up the initial tableaux\X${}\E{}$\6
\&{for} ${}(\|i\K\T{0};{}$ ${}\|i\Z\|m;{}$ ${}\|i\PP){}$\5
${}\{{}$\1\6
\&{for} ${}(\|j\K\|n;{}$ ${}\|j>\T{0};{}$ ${}\|j\MM){}$\1\6
\&{if} ${}(\|i\E\T{0}){}$\1\5
${}\|a[\T{0}][\|j+\|m]\K{-}\|a[\T{0}][\|j]{}$;\5
\2\&{else}\1\5
${}\|a[\|i][\|j+\|m]\K\|a[\|i][\|j];{}$\2\2\6
\&{for} ${}(\|j\K\|m;{}$ ${}\|j>\T{0};{}$ ${}\|j\MM){}$\1\5
${}\|a[\|i][\|j]\K(\|i\E\|j);{}$\2\6
${}\|p[\|i]\K\|q[\|i]\K\|i;{}$\6
\4${}\}{}$\2\6
\&{for} ${}(\|j\K\|m+\|n+\T{1};{}$ ${}\|j\G\T{0};{}$ ${}\|j\MM){}$\1\5
${}\\{denom}[\|j]\K\T{1};{}$\2\6
\&{for} ${}(\|i\K\T{0};{}$ ${}\|i\Z\|m;{}$ ${}\|i\PP){}$\1\6
\&{for} ${}(\|j\K\|m+\|n+\T{1};{}$ ${}\|j\G\T{0};{}$ ${}\|j\MM){}$\1\5
${}\\{aa}[\|i][\|j]\K\|a[\|i][\|j]{}$;\2\2\par
\U13.\fi

\M[406 lp.w]{15}At this point we have reached a tableau with a negative entry
in row~0
and column~$(w_j)$. Two cases arise: The corresponding column might
contain at least one positive entry; or it might not.

In the latter case, we can stop. Our tableau proves that the maximization
problem has $+\infty$ as its answer, because arbitrarily large values
of $w_j$ lie in~$X$. These values increase $c_1v_1+\cdots+c_nv_n$ without
limit, because of the negative coefficient in row~0. Moreover, there
cannot be any values $(u_1,\ldots,u_m)$ that satisfy the dual inequalities;
if they did, $b_1u_1+\cdots+b_mu_m$ would be an upper bound on
$c_1v_1+\cdots+c_nv_n$.

\Y\B\4\X15:Try to pivot in column $j$\X${}\E{}$\6
$\|l\K\T{0};{}$\6
\&{for} ${}(\|i\K\T{1};{}$ ${}\|i\Z\|m;{}$ ${}\|i\PP){}$\1\6
\&{if} ${}(\|a[\|i][\|j]>\T{0}){}$\1\5
\X16:Consider pivoting at $(i,j)$\X;\2\2\6
\&{if} ${}(\|l\E\T{0}){}$\5
${}\{{}$\1\6
\\{printf}(\.{"The\ maximum\ is\ infi}\)\.{nite;\ the\ dual\ has\ n}\)\.{o\
solution!\\n"});\6
\X22:Print out the current state\X;\6
\\{exit}(\T{0});\6
\4${}\}{}$\2\6
\X17:Pivot at $(l,j)$\X;\par
\U13.\fi

\M[429 lp.w]{16}When $a_{0j}<0$ and $a_{ij}>0$, a pivot step in row $i$ and
column~$j$
always increases the lexicographic value of row~0, because it adds a positive
multiple of the (lexicographically positive) row~$i$. Therefore Pivoting
is a Good Thing: It leads to continual progress toward larger and larger
top rows.

But which rows can we pivot on, without making another row lexicographically
negative? Our example above showed that random pivoting doesn't always work.
Perhaps we were just lucky to find a good pivot in that problem;
it's conceivable that another example might run into a state from which
no decent pivot is possible.

Fortunately there {\it is\/} always a row on which to pivot, in fact
a {\it unique\/} row, in any
given column $j$ that has at least one positive entry $a_{ij}$.
The reason is that the operation
of pivoting on $a_{ij}$ causes row~$k$ (call it $r_k$) to be
replaced by $r_k-a_{kj}r_i/a_{ij}$ for each $k\ne i$;
and it is easy to see
that $r_k-a_{kj}r_i/a_{ij}$ is lexicographically positive if and only
if $(r_k/a_{kj})-(r_i/a_{ij})$ is lexicographically positive.
Hence we must pivot on the row for which $r_i/a_{ij}$ is {\it lexicographically
smallest}, among all rows~$i$ with $a_{ij}>0$.

We cannot have $r_k/a_{kj}$ exactly equal to $r_i/a_{ij}$ when $k\ne i$,
because those rows differ in basis columns $k$ and~$i$.

Notice that this choice of pivot row does not depend on
the scale factors in \PB{\\{denom}}. We can safely use floating-point
arithmetic
when making the choice, because such rounding errors are tightly controlled.

\Y\B\4\X16:Consider pivoting at $(i,j)$\X${}\E{}$\6
\&{if} ${}(\|l\E\T{0}){}$\1\5
${}\|l\K\|i,\39\|s\K\T{0};{}$\2\6
\&{else}\5
${}\{{}$\1\6
\&{for} ${}(\|h\K\T{0};{}$  ; ${}\|h\PP){}$\5
${}\{{}$\1\6
\&{if} ${}(\|h\E\|s){}$\1\5
${}\\{trial}[\|s\PP]\K{}$(\&{double}) \|a[\|l][\|h]${}/{}$(\&{double}) \|a[%
\|l][\|j];\2\6
${}\|z\K{}$(\&{double}) \|a[\|i][\|h]${}/{}$(\&{double}) \|a[\|i][\|j];\6
\&{if} ${}(\\{trial}[\|h]\I\|z){}$\1\5
\&{break};\2\6
\4${}\}{}$\2\6
\&{if} ${}(\\{trial}[\|h]>\|z){}$\1\5
${}\|l\K\|i,\39\\{trial}[\|h]\K\|z,\39\|s\K\|h+\T{1}{}$;\C{ \PB{\\{trial}[\|h]}
is best so far, for $0\le h<s$ }\2\6
\4${}\}{}$\2\par
\U15.\fi

\M[472 lp.w]{17}\B\X17:Pivot at $(l,j)$\X${}\E{}$\6
\X18:Do floating-point pivoting\X;\6
\X20:Do integer pivoting\X;\6
${}\|q[\|p[\|l]]\K\T{0},\39\|p[\|l]\K\|j,\39\|q[\|j]\K\|l{}$;\par
\U15.\fi

\M[477 lp.w]{18}Before we do any integer pivoting, we'd like to be sure that
an all-floating-point method would make the same decision.
So this step repeats some of the work we've already done, but
it uses the \PB{\\{aa}} tableau instead of \PB{\|a}.

\Y\B\4\X18:Do floating-point pivoting\X${}\E{}$\6
${}\{{}$\1\6
\&{register} \&{int} \\{ii}${},{}$ \\{jj}${},{}$ \\{kk}${},{}$ \\{ll};\7
\&{for} ${}(\\{ll}\K\T{0},\39\\{jj}\K\|m+\|n;{}$ ${}\\{jj}>\T{0};{}$ ${}\\{jj}%
\MM){}$\1\6
\&{if} ${}(\\{aa}[\T{0}][\\{jj}]<\T{0}){}$\5
${}\{{}$\1\6
\&{if} ${}(\\{jj}\I\|j){}$\1\5
\&{goto} \\{mismatch};\2\6
\&{for} ${}(\\{ii}\K\T{1};{}$ ${}\\{ii}\Z\|m;{}$ ${}\\{ii}\PP){}$\1\6
\&{if} ${}(\\{aa}[\\{ii}][\|j]>\T{0}){}$\5
${}\{{}$\1\6
\&{if} ${}(\\{ll}\E\T{0}){}$\1\5
${}\\{ll}\K\\{ii},\39\|s\K\T{0};{}$\2\6
\&{else}\5
${}\{{}$\1\6
\&{for} ${}(\|h\K\T{0};{}$  ; ${}\|h\PP){}$\5
${}\{{}$\1\6
\&{if} ${}(\|h\E\|s){}$\1\5
${}\\{trial}[\|s\PP]\K\\{aa}[\\{ll}][\|h]/\\{aa}[\\{ll}][\|j];{}$\2\6
${}\|z\K\\{aa}[\\{ii}][\|h]/\\{aa}[\\{ii}][\|j];{}$\6
${}\\{zbuf}.\\{dbl}\K\\{trial}[\|h]-\|z{}$;\5
\\{zap};\6
\&{if} ${}(\\{zbuf}.\\{dbl}){}$\1\5
\&{break};\2\6
\4${}\}{}$\2\6
\&{if} ${}(\\{zbuf}.\\{dbl}>\T{0.0}){}$\1\5
${}\\{ll}\K\\{ii},\39\\{trial}[\|h]\K\|z,\39\|s\K\|h+\T{1};{}$\2\6
\4${}\}{}$\2\6
\4${}\}{}$\2\2\6
\&{if} ${}(\\{ll}\I\|l){}$\1\5
\&{goto} \\{mismatch};\2\6
\X19:Really do floating-point pivoting\X;\6
\&{goto} \\{fp\_pivot\_done};\6
\4${}\}{}$\2\2\6
\4\\{mismatch}:\5
\\{printf}(\.{"The\ floating-point\ }\)\.{and\ fixed-point\ impl}\)%
\.{ementations\ disagree}\)\.{!\\n"});\6
${}\\{printf}(\.{"(Floating-point\ piv}\)\.{oting\ at\ (\%d,\%d),\ no}\)\.{t\ (%
\%d,\%d).)\\n"},\39\\{ll},\39\\{jj},\39\|l,\39\|j);{}$\6
\X22:Print out the current state\X;\6
${}\\{exit}({-}\T{99});{}$\6
\4${}\}{}$\2\6
\4\\{fp\_pivot\_done}:\par
\U17.\fi

\N[511 lp.w]{1}{19}Pivoting. We're ready at last to update the tableaux:
Arithmetic happens here, as we pivot on column~$j$ of row~$l$.

\Y\B\4\X19:Really do floating-point pivoting\X${}\E{}$\6
\&{for} ${}(\|k\K\T{0},\39\|z\K\\{aa}[\|l][\|j];{}$ ${}\|k\Z\|m+\|n;{}$ ${}\|k%
\PP){}$\1\6
\&{if} (\\{aa}[\|l][\|k])\1\5
${}\\{aa}[\|l][\|k]\K\\{aa}[\|l][\|k]/\|z{}$;\C{ no zap needed }\2\2\6
\&{for} ${}(\|i\K\T{0};{}$ ${}\|i\Z\|m;{}$ ${}\|i\PP){}$\1\6
\&{if} ${}(\|i\I\|l){}$\5
${}\{{}$\1\6
\&{for} ${}(\|k\K\T{0},\39\|z\K\\{aa}[\|i][\|j];{}$ ${}\|k\Z\|m+\|n;{}$ ${}\|k%
\PP){}$\1\6
\&{if} ${}(\|k\E\|j){}$\1\5
${}\\{aa}[\|i][\|k]\K\T{0.0};{}$\2\6
\&{else}\5
${}\{{}$\1\6
${}\\{zbuf}.\\{dbl}\K\\{aa}[\|i][\|k]-\|z*\\{aa}[\|l][\|k]{}$;\5
\\{zap};\6
${}\\{aa}[\|i][\|k]\K\\{zbuf}.\\{dbl};{}$\6
\4${}\}{}$\2\2\6
\4${}\}{}$\2\2\par
\U18.\fi

\M[526 lp.w]{20}In the all-integer version I'm hoping with fingers crossed that
the
numerators and denominators will stay small, at least in the simple cases that
are greatest interest to me at the moment.

\Y\B\4\X20:Do integer pivoting\X${}\E{}$\6
\&{if} (\\{verbose})\1\5
${}\\{printf}(\.{"Pivoting\ on\ (\%d,\%d)}\)\.{.\\n"},\39\|l,\39\|j);{}$\2\6
\&{for} ${}(\|k\K\T{0};{}$ ${}\|k\Z\|m+\|n;{}$ ${}\|k\PP){}$\1\6
\&{if} ${}(\|a[\|l][\|k]\W\|k\I\|j){}$\5
${}\{{}$\1\6
\&{register} \&{intword} \|t${},{}$ \|u${}\K\|a[\|l][\|k],{}$ \|v${}\K\|a[\|l][%
\|j];{}$\7
\&{if} ${}(\|u<\T{0}){}$\1\5
${}\|u\K{-}\|u;{}$\2\6
\&{if} ${}(\|v<\T{0}){}$\1\5
${}\|v\K{-}\|v;{}$\2\6
\&{while} (\|v)\1\5
${}\|t\K\|u,\39\|u\K\|v,\39\|v\K\|t\MOD\|v{}$;\C{ Euclid's algorithm, sets $u%
\gets\gcd(u,v)$ }\2\6
\&{if} ${}(\|u\E\|a[\|l][\|j]){}$\1\5
${}\|a[\|l][\|k]\K\|a[\|l][\|k]/\|u;{}$\2\6
\&{else}\5
${}\{{}$\1\6
${}\|v\K\|a[\|l][\|j]/\|u,\39\\{denom}[\|k]\MRL{*{\K}}\|v{}$;\C{ scale factor
goes up in column $k$ }\6
\&{for} ${}(\|i\K\T{0};{}$ ${}\|i\Z\|m;{}$ ${}\|i\PP){}$\1\5
${}\|a[\|i][\|k]\K(\|i\E\|l\?\|a[\|l][\|k]/\|u:\|a[\|i][\|k]*\|v);{}$\2\6
\4${}\}{}$\2\6
\4${}\}{}$\2\2\6
\&{for} ${}(\|i\K\T{0};{}$ ${}\|i\Z\|m;{}$ ${}\|i\PP){}$\1\6
\&{if} ${}(\|i\I\|l){}$\5
${}\{{}$\1\6
\&{for} ${}(\|k\K\T{0},\39\|h\K\|a[\|i][\|j];{}$ ${}\|k\Z\|m+\|n;{}$ ${}\|k%
\PP){}$\1\5
${}\|a[\|i][\|k]\K(\|k\E\|j\?\T{0}:\|a[\|i][\|k]-\|h*\|a[\|l][\|k]);{}$\2\6
\4${}\}{}$\2\2\6
${}\|a[\|l][\|j]\K\T{1};{}$\6
\&{if} ${}(\\{denom}[\|j]\I\T{1}){}$\5
${}\{{}$\1\6
\&{for} ${}(\|h\K\\{denom}[\|j],\39\\{denom}[\|j]\K\T{1},\39\|k\K\T{0};{}$ ${}%
\|k\Z\|m+\|n;{}$ ${}\|k\PP){}$\1\6
\&{if} ${}(\|k\I\|j){}$\1\5
${}\|a[\|l][\|k]\MRL{*{\K}}\|h;{}$\2\2\6
\4${}\}{}$\2\par
\U17.\fi

\N[551 lp.w]{1}{21}Final touches. A few last-minute odds and ends remain.

\fi

\M[553 lp.w]{22}\B\X22:Print out the current state\X${}\E{}$\6
${}\{{}$\1\6
${}\\{printf}(\.{"Step\ \%d:\\n"},\39\\{count});{}$\6
\&{for} ${}(\|i\K\T{0};{}$ ${}\|i\Z\|m;{}$ ${}\|i\PP){}$\5
${}\{{}$\1\6
\&{for} ${}(\|j\K\T{0};{}$ ${}\|j\Z\|m+\|n;{}$ ${}\|j\PP){}$\1\5
${}\\{printf}(\.{"\ \%d"},\39\|a[\|i][\|j]);{}$\2\6
\\{printf}(\.{"\\n"});\6
\4${}\}{}$\2\6
\\{printf}(\.{"denom"});\6
\&{for} ${}(\|j\K\T{0};{}$ ${}\|j\Z\|m+\|n;{}$ ${}\|j\PP){}$\1\5
${}\\{printf}(\.{"\ \%d"},\39\\{denom}[\|j]);{}$\2\6
\\{printf}(\.{"\\n"});\6
\&{for} ${}(\|i\K\T{0};{}$ ${}\|i\Z\|m;{}$ ${}\|i\PP){}$\5
${}\{{}$\1\6
\&{for} ${}(\|j\K\T{0};{}$ ${}\|j\Z\|m+\|n;{}$ ${}\|j\PP){}$\1\5
${}\\{printf}(\.{"\ \%.15g"},\39\\{aa}[\|i][\|j]);{}$\2\6
\\{printf}(\.{"\\n"});\6
\4${}\}{}$\2\6
\4${}\}{}$\2\par
\Us13, 15\ETs18.\fi

\M[569 lp.w]{23}\B\X23:Report the answers\X${}\E{}$\6
$\\{printf}(\.{"Optimal\ value\ \%.15g}\)\.{=\%d/\%d\ found\ after\ \%}\)\.{d\
pivots.\\n"},\39\\{aa}[\T{0}][\T{0}],\39\|a[\T{0}][\T{0}],\39\\{denom}[\T{0}],%
\39\\{count});{}$\6
\\{printf}(\.{"\ Optimal\ v's:"});\6
\&{for} ${}(\|j\K\|m+\T{1};{}$ ${}\|j\Z\|m+\|n;{}$ ${}\|j\PP){}$\1\6
\&{if} (\|q[\|j])\1\5
${}\\{printf}(\.{"\ \%.15g=\%d/\%d"},\39\\{aa}[\|q[\|j]][\T{0}],\39\|a[\|q[%
\|j]][\T{0}],\39\\{denom}[\T{0}]);{}$\2\6
\&{else}\1\5
\\{printf}(\.{"\ 0"});\2\2\6
\\{printf}(\.{"\\n\ Optimal\ u's:"});\6
\&{for} ${}(\|j\K\T{1};{}$ ${}\|j\Z\|m;{}$ ${}\|j\PP){}$\1\5
${}\\{printf}(\.{"\ \%.15g=\%d/\%d"},\39\\{aa}[\T{0}][\|j],\39\|a[\T{0}][\|j],%
\39\\{denom}[\|j]);{}$\2\6
\\{printf}(\.{"\\n"});\par
\U13.\fi

\M[580 lp.w]{24}Well, our little program is done. But an attentive reader may
well have
noticed that an important point has not yet been considered: We haven't proved
that the algorithm must terminate.

An elementary knowledge of matrix theory suffices to close this final
gap. We shall prove the following lemma: {\sl Given any ordered choice of
$m$ columns, there is at most one achievable tableau for which those columns
are the basis.}

{\it Proof}.
Let $A_0$ be rows 1 to $m$ of the original tableau. At every stage of
the algorithm, rows 1 to~$m$ of the current tableau are equal to $BA_0$,
for some nonsingular matrix~$B$ determined by the pivot operations.
Furthermore, if we are told which columns are the basis columns, the
matrix~$B$ is fully determined; only one~$B$ can yield the correct
values in those columns. Therefore the choice of basis columns also
tells us the entire contents of rows 1 to~$m$. And row~0 is also known,
because it is the original row~0 plus $\sum_{i=1}^m c'_ir^{\phantom\prime}_i$,
where $c'_i=0$ if basis column~$i$ corresponds to a slack variable,
$c'_i=c_j$ if basis column~$i$ corresponds to $v_j$. QED.

But we have shown that row~0 continually increases, lexicographically.
So the algorithm cannot get to the same basis twice; and there are
only finitely many bases. Termination is inevitable.

\fi

\N[605 lp.w]{1}{25}Index.
\fi

\inx
\fin
\con
