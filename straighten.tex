\input cwebmac
\datethis

\N{1}{1}Intro. This program computes matrix representations of permutations,
based on tableaux of a given shape~$\lambda$. If there are $f$
standard Young tableaux of that shape, it produces $f\times f$ matrices
$B_\pi$ for any given permutation $\pi$, with the property that
$B_\pi B_\sigma=B_{\pi\sigma}$.

I'm trying to learn concrete details of such representations, and my
experience has always been that the best way to learn something is
to try to program it whenever possible. Therefore I'm writing this
code as part of my own education. But I haven't seen any book that
mentions the method used below, so other readers may also find
aspects of interest here.
Of course I can't claim to have read very much of the huge literature
that already exists on this topic; probably I have rediscovered
something that's fairly well known.

Let $\lambda$ be a partition of $n$, namely $\lambda=(a_1,\ldots,a_k)$ where
$a_1\ge\cdots\ge a_k\ge1$ and $a_1+\cdots+a_k=n$. A {\it tableau\/}
of shape~$\lambda$ is a way to place the numbers $\{1,\ldots,n\}$ into
an array with $n$ left-justified rows and $a_j$ entries in row~$j$.
The tableau is {\it standard\/} if the entries in each row are
increasing from left to right and the entries in each column are
increasing from top to bottom.

\begingroup
\setbox0=\hbox{0}
\dimen0=\ht0 \advance\dimen0 by 2pt \ht0=\dimen0 \dp0=2pt
\def\\#1#2#3#4#5#6#7#8#9{\vcenter{\offinterlineskip
\hrule\halign{\vrule\vphantom{\copy0}$\,##\,$\vrule&&$\,##\,$\vrule\cr
#1&#2&#3\cr\noalign{\hrule}
#4&#5&#6\cr\noalign{\hrule}
#7&#8&#9\cr\noalign{\hrule}}}}
The method of this program is based on a straightforward algorithm that
takes a not-necessarily-standard tableau and determines all ways to
permute its columns in such a fashion that a subsequent row-sorting
will produce a standard tableau. For example, if $\lambda=(3,3,3)$ and if
the given tableau is
$$\\314592687\,,$$
there are nine solutions,
$$\\312584697\quad\\312684597\quad\\512384697\quad
\\612384597\quad\\314582697\quad\\314682597\quad
\\317582694\quad\\514682397\quad\\614582397\,;$$
row-sorting converts these respective solutions to the standard tableaux
$$\\123458679\quad\\123468579\quad\\125348679\quad
\\126348579\quad\\134258679\quad\\134268579\quad
\\137258469\quad\\145268379\quad\\146258379\,.$$
Another way to state the given problem is, ``Find all standard tableaux
that can be produced from a given one by permuting columns, then
permuting rows.''
\endgroup

The first line of standard input should contain the partition elements
$a_1$, \dots,~$a_k$, separated by spaces and followed by~0. Subsequent
lines should contain permutations whose representative matrices are desired;
each permutation is given as a sequence $p_1$, \dots,~$p_n$, separated by
spaces.

N.B.: The permutation $p_1\ldots p_n$ takes $1\mapsto p_1$, $2\mapsto p_2$,
etc., and the representation matrices produced by this program multiply
permutations from left to right. Thus, for example, if $A$, $B$, and $C$
are the matrices representing $132$, $213$, and $231$, respectively,
aka the permutations (23), (12), and (123), we have $(23)(12)=(123)$
hence $AB=C$.

\fi

\M{2}\B\D$\\{maxn}$ \5
\T{100}\C{ let's not permute more than a hundred guys }\par
\B\4\D$\\{maxf}$ \5
\T{300}\C{ and let's not find matrices of size more than $300\times300$ }\par
\Y\B\8\#\&{include} \.{<stdio.h>}\6
\X3:Global variables\X\6
\X11:Subroutines\X\7
\\{main}(\,)\1\1\2\2\6
${}\{{}$\1\6
\&{register} \&{int} \|j${},{}$ \\{jj}${},{}$ \|k${},{}$ \|l;\7
\X4:Read the shape $\lambda$\X;\6
\X5:Compute the transposed shape $\lambda^T$\X;\6
\X6:Find all the standard tableaux of the given shape\X;\6
${}\\{printf}(\.{"There\ are\ \%d\ standa}\)\.{rd\ tableaux\ of\ shape}\)\.{"},%
\39\|f);{}$\6
\&{for} ${}(\|j\K\T{1};{}$ ${}\|j\Z\\{kk};{}$ ${}\|j\PP){}$\1\5
${}\\{printf}(\.{"\ \%d"},\39\|a[\|j]);{}$\2\6
\\{printf}(\.{".\\n"});\6
\X16:Compute \PB{\|f} basis vectors for the representation\X;\6
\&{while} (\T{1})\5
${}\{{}$\1\6
\X17:Read a permutation \PB{\|p} (but \PB{\&{break}} if there's no more good
data)\X;\6
\\{printf}(\.{"Representation\ of"});\6
\&{for} ${}(\|j\K\T{1};{}$ ${}\|j\Z\|n;{}$ ${}\|j\PP){}$\1\5
${}\\{printf}(\.{"\ \%d"},\39\|p[\|j]);{}$\2\6
\\{printf}(\.{":\\n"});\6
\&{for} ${}(\\{jj}\K\T{0};{}$ ${}\\{jj}<\|f;{}$ ${}\\{jj}\PP){}$\5
${}\{{}$\1\6
\X18:Compute the representation of the \PB{\\{jj}}th standard tableau, permuted
by \PB{\|p}\X;\6
\X19:Reduce the representation to a linear combination of basis elements\X;\6
\&{for} ${}(\|j\K\T{0};{}$ ${}\|j<\|f;{}$ ${}\|j\PP){}$\1\5
${}\\{printf}(\.{"\%\ 3d"},\39\\{rep}[\|j]);{}$\2\6
\\{printf}(\.{"\\n"});\6
\4${}\}{}$\2\6
\4${}\}{}$\2\6
\4${}\}{}$\2\par
\fi

\M{3}\B\X3:Global variables\X${}\E{}$\6
\&{int} ${}\|a[\\{maxn}+\T{2}]{}$;\C{ the shape }\6
\&{int} ${}\|b[\\{maxn}+\T{1}]{}$;\C{ its transpose }\6
\&{int} \|n;\C{ the number of elements permuted }\6
\&{int} \\{kk};\C{ the number of rows }\6
\&{int} \|f;\C{ the number of standard tableaux }\6
\&{int} ${}\|p[\\{maxn}+\T{1}]{}$;\C{ the permutation to be represented }\6
\&{int} ${}\|q[\\{maxn}+\T{1}]{}$;\C{ the inverse of $p$ }\6
\&{int} \|t[\\{maxn}][\\{maxn}]${},{}$ \\{tt}[\\{maxn}][\\{maxn}];\C{ working
tableaux }\6
\&{int} ${}\\{aa}[\\{maxn}+\T{2}]{}$;\C{ row sizes of \PB{\\{tt}} }\6
\&{int} \\{stand}[\\{maxf}][\\{maxn}];\C{ standard tableaux }\6
\&{int} \\{basis}[\\{maxf}][\\{maxf}];\C{ basis elements }\6
\&{int} \\{rep}[\\{maxf}];\C{ a linear combination of standard tableaux }\par
\A9.
\U2.\fi

\M{4}\B\X4:Read the shape $\lambda$\X${}\E{}$\6
\&{for} ${}(\|j\K\T{0};{}$  ; ${}\|j\PP){}$\5
${}\{{}$\1\6
\&{if} ${}(\|j>\\{maxn}){}$\5
${}\{{}$\1\6
${}\\{fprintf}(\\{stderr},\39\.{"Partition\ too\ long\ }\)\.{(maxn=\%d)!\\n"},%
\39\\{maxn});{}$\6
${}\\{exit}({-}\T{1});{}$\6
\4${}\}{}$\2\6
\&{if} ${}(\\{scanf}(\.{"\%d"},\39{\AND}\|a[\|j+\T{1}])\I\T{1}){}$\5
${}\{{}$\1\6
${}\\{fprintf}(\\{stderr},\39\.{"Partition\ should\ en}\)\.{d\ with\ zero!%
\\n"});{}$\6
${}\\{exit}({-}\T{2});{}$\6
\4${}\}{}$\2\6
\&{if} ${}(\|a[\|j+\T{1}]\E\T{0}){}$\1\5
\&{break};\2\6
\&{if} ${}(\|a[\|j+\T{1}]<\T{0}){}$\5
${}\{{}$\1\6
${}\\{fprintf}(\\{stderr},\39\.{"Partition\ contains\ }\)\.{a\ negative\
element\ (}\)\.{\%d)!\\n"},\39\|a[\|j+\T{1}]);{}$\6
${}\\{exit}({-}\T{3});{}$\6
\4${}\}{}$\2\6
\&{if} ${}(\|a[\|j+\T{1}]>\\{maxn}){}$\5
${}\{{}$\1\6
${}\\{fprintf}(\\{stderr},\39\.{"Partition\ element\ \%}\)\.{d\ is\ too\ big\
(maxn=\%}\)\.{d)!\\n"},\39\|a[\|j+\T{1}],\39\\{maxn});{}$\6
${}\\{exit}({-}\T{4});{}$\6
\4${}\}{}$\2\6
\4${}\}{}$\2\6
${}\\{kk}\K\|j;{}$\6
\&{for} ${}(\|j\K\T{2},\39\|n\K\|a[\T{1}];{}$ ${}\|j\Z\\{kk};{}$ ${}\|j\PP){}$%
\1\5
${}\|n\MRL{+{\K}}\|a[\|j];{}$\2\6
\&{if} ${}(\|n>\\{maxn}){}$\5
${}\{{}$\1\6
${}\\{fprintf}(\\{stderr},\39\.{"Shape\ is\ too\ big\ (n}\)\.{=\%d,\ maxn=\%d)!%
\\n"},\39\|n,\39\\{maxn});{}$\6
${}\\{exit}({-}\T{5});{}$\6
\4${}\}{}$\2\par
\U2.\fi

\M{5}This is exercise 7.2.1.4--6.

\Y\B\4\X5:Compute the transposed shape $\lambda^T$\X${}\E{}$\6
\&{for} ${}(\|k\K\|a[\T{1}],\39\|j\K\T{1};{}$ \|k; ${}\|j\PP){}$\1\6
\&{while} ${}(\|k>\|a[\|j+\T{1}]){}$\1\5
${}\|b[\|k\MM]\K\|j{}$;\2\2\par
\U2.\fi

\N{1}{6}Generating the standard tableaux. Here I use the Varol--Rotem algorithm
to run through all the Young tableaux (Algorithm 7.2.1.2V).

All algorithms in this program are pretty much ``brute force,'' with
little attempt at optimization.

\Y\B\4\X6:Find all the standard tableaux of the given shape\X${}\E{}$\6
\X7:Generate the order relation for the desired tableaux\X;\6
\4\\{v1}:\5
\&{for} ${}(\|j\K\T{0};{}$ ${}\|j\Z\|n;{}$ ${}\|j\PP){}$\1\5
${}\|p[\|j]\K\|q[\|j]\K\|j,\39\\{prec}[\T{0}][\|j]\K\T{1};{}$\2\6
\4\\{v2}:\5
\X8:Record the tableau represented by $p$ and $q$\X;\6
${}\|k\K\|n;{}$\6
\4\\{v3}:\5
${}\|j\K\|q[\|k],\39\|l\K\|p[\|j-\T{1}];{}$\6
\&{if} (\\{prec}[\|l][\|k])\1\5
\&{goto} \\{v5};\2\6
\4\\{v4}:\5
${}\|p[\|j-\T{1}]\K\|k,\39\|p[\|j]\K\|l,\39\|q[\|k]\K\|j-\T{1},\39\|q[\|l]\K%
\|j;{}$\6
\&{goto} \\{v2};\6
\4\\{v5}:\5
\&{while} ${}(\|j<\|k){}$\1\5
${}\|l\K\|p[\|j+\T{1}],\39\|p[\|j]\K\|l,\39\|q[\|l]\K\|j,\39\|j\PP;{}$\2\6
${}\|p[\|k]\K\|q[\|k]\K\|k;{}$\6
${}\|k\MM;{}$\6
\&{if} (\|k)\1\5
\&{goto} \\{v3};\2\6
\X10:Assign index numbers to each tableau found\X;\par
\U2.\fi

\M{7}\B\X7:Generate the order relation for the desired tableaux\X${}\E{}$\6
\&{for} ${}(\|j\K\\{jj}\K\T{0};{}$ ${}\|j<\\{kk};{}$ ${}\|j\PP){}$\1\6
\&{for} ${}(\|k\K\T{0};{}$ ${}\|k<\|a[\|j+\T{1}];{}$ ${}\|k\PP){}$\5
${}\{{}$\1\6
${}\|t[\|j][\|k]\K\PP\\{jj};{}$\6
\&{if} ${}(\|k>\T{0}){}$\1\5
${}\\{prec}[\\{jj}-\T{1}][\\{jj}]\K\T{1};{}$\2\6
\&{if} ${}(\|j>\T{0}){}$\1\5
${}\\{prec}[\|t[\|j-\T{1}][\|k]][\\{jj}]\K\T{1};{}$\2\6
\4${}\}{}$\2\2\par
\U6.\fi

\M{8}At this point we've found a standard tableau, whose entry in
position \PB{\|t[\|j][\|k]} is \PB{\|q[\|t[\|j][\|k]]}.

It is convenient to
record a standard tableau as a permutation $w_1\ldots w_n$ of the multiset
$\{a_1\cdot1,\ldots,a_k\cdot k\}$, where the $l$th element of this
permutation specifies the row occupied by the number~$l$. Then a~trie
is used to keep track of all such permutations we've found.

\Y\B\4\X8:Record the tableau represented by $p$ and $q$\X${}\E{}$\6
\&{if} ${}(\|f\E\\{maxf}){}$\5
${}\{{}$\1\6
${}\\{fprintf}(\\{stderr},\39\.{"Too\ many\ standard\ t}\)\.{ableaux\ exist\
(maxf=}\)\.{\%d)!\\n"},\39\\{maxf});{}$\6
${}\\{exit}({-}\T{6});{}$\6
\4${}\}{}$\2\6
\&{for} ${}(\|j\K\T{0};{}$ ${}\|j<\\{kk};{}$ ${}\|j\PP){}$\1\6
\&{for} ${}(\|k\K\T{0};{}$ ${}\|k<\|a[\|j+\T{1}];{}$ ${}\|k\PP){}$\1\5
${}\|w[\|q[\|t[\|j][\|k]]]\K\|j+\T{1};{}$\2\2\6
\&{for} ${}(\|j\K\T{1},\39\|k\K\T{0};{}$ ${}\|j<\|n;{}$ ${}\|j\PP,\39\|k\K%
\|l){}$\5
${}\{{}$\1\6
${}\|l\K\\{trie}[\|k][\|w[\|j]];{}$\6
\&{if} ${}(\|l\E\T{0}){}$\1\5
${}\|l\K\\{trie}[\|k][\|w[\|j]]\K\PP\\{trienodes};{}$\2\6
\4${}\}{}$\2\6
${}\\{trie}[\|k][\|w[\|n]]\K\T{1}{}$;\C{ mark a unique entry in the leif }\6
${}\|f\PP{}$;\par
\U6.\fi

\M{9}\B\X3:Global variables\X${}\mathrel+\E{}$\6
\&{int} ${}\\{prec}[\\{maxn}+\T{1}][\\{maxn}+\T{1}]{}$;\C{ \PB{\\{prec}[\|j][%
\|k]} is nonzero if $j\prec k$ }\6
\&{int} ${}\|w[\\{maxn}+\T{1}]{}$;\C{ codeword for a standard tableau }\6
\&{int} ${}\\{trie}[\\{maxf}*\\{maxn}][\\{maxn}+\T{1}]{}$;\C{ trie memory, see
Algorithm 6.3T }\6
\&{int} \\{trienodes};\C{ this many trie nodes have been allocated so far }\par
\fi

\M{10}The standard tableaux are now given code numbers from 0 to $f-1$.
We walk through the trie in lexicographic order.
(Yes, I could/should have done it recursively.)

\Y\B\4\X10:Assign index numbers to each tableau found\X${}\E{}$\6
$\|l\K\T{1},\39\|k\K\T{0},\39\|j\K\T{0};{}$\6
\4\\{newlev}:\5
${}\|w[\|l]\K\T{1};{}$\6
\4\\{tryit}:\5
\&{if} (\\{trie}[\|k][\|w[\|l]])\5
${}\{{}$\1\6
\&{if} ${}(\|l\E\|n){}$\5
${}\{{}$\1\6
\&{for} ( ; \|l; ${}\|l\MM){}$\1\5
${}\\{stand}[\|j][\|l]\K\|w[\|l];{}$\2\6
${}\|l\K\|n;{}$\6
${}\\{trie}[\|k][\|w[\|l]]\K\|j\PP;{}$\6
\&{goto} \\{levdone};\6
\4${}\}{}$\2\6
${}\|q[\|l]\K\|k,\39\|k\K\\{trie}[\|k][\|w[\|l]],\39\|l\PP;{}$\6
\&{goto} \\{newlev};\6
\4${}\}{}$\2\6
\4\\{tryagain}:\5
\&{if} ${}(\|w[\|l]\E\\{kk}){}$\1\5
\&{goto} \\{levdone};\2\6
${}\|w[\|l]\PP;{}$\6
\&{goto} \\{tryit};\6
\4\\{levdone}:\5
${}\|l\MM;{}$\6
\&{if} (\|l)\5
${}\{{}$\1\6
${}\|k\K\|q[\|l];{}$\6
\&{goto} \\{tryagain};\6
\4${}\}{}$\2\6
\&{if} ${}(\|j\I\|f){}$\5
${}\{{}$\1\6
${}\\{fprintf}(\\{stderr},\39\.{"Oops,\ I\ goofed!\\n"});{}$\6
${}\\{exit}({-}\T{7});{}$\6
\4${}\}{}$\2\par
\U6.\fi

\N{1}{11}Finding admissible column perms. Now comes the heart of this program,
the routine for solving the problem mentioned in the introduction.

Instead of producing a list of solutions, it sets $\PB{\\{rep}}[j]=\pm1$
for each standard tableau~$j$ achievable by column-then-row permutation,
using the sign of the column permutation.

\Y\B\4\X11:Subroutines\X${}\E{}$\6
\&{void} \\{findrep}(\&{void})\C{ the input tableau is in \PB{\|t} }\6
${}\{{}$\1\6
\&{register} \&{int} \|i${},{}$ \|j${},{}$ \|k${},{}$ \|l${},{}$ \\{inv}${},{}$
\\{sign};\6
\&{int} ${}\\{row}[\\{maxn}+\T{1}],{}$ ${}\\{col}[\\{maxn}+\T{1}]{}$;\C{
positions inside \PB{\|t} }\7
\X12:Sort the columns of \PB{\|t}\X;\6
\&{for} ${}(\|j\K\T{0};{}$ ${}\|j<\|f;{}$ ${}\|j\PP){}$\1\5
${}\\{rep}[\|j]\K\T{0};{}$\2\6
\X13:Figure out where each element is, and set \PB{\\{tt}} zero\X;\6
\X14:Run through all solutions\X;\6
\4${}\}{}$\2\par
\U2.\fi

\M{12}Insertion sort wins here, of course.

\Y\B\4\X12:Sort the columns of \PB{\|t}\X${}\E{}$\6
$\\{inv}\K\T{0};{}$\6
\&{for} ${}(\|k\K\T{0};{}$ ${}\|k<\|a[\T{1}];{}$ ${}\|k\PP){}$\5
${}\{{}$\1\6
\&{for} ${}(\|j\K\T{1};{}$ ${}\|j<\|b[\|k+\T{1}];{}$ ${}\|j\PP){}$\1\6
\&{if} ${}(\|t[\|j][\|k]<\|t[\|j-\T{1}][\|k]){}$\5
${}\{{}$\1\6
\&{for} ${}(\|i\K\|j-\T{1},\39\|l\K\|t[\|j][\|k];{}$  ; ${}\|i\MM){}$\5
${}\{{}$\1\6
${}\|t[\|i+\T{1}][\|k]\K\|t[\|i][\|k];{}$\6
${}\\{inv}\PP{}$;\C{ inversions removed in this column }\6
\&{if} ${}(\|i\E\T{0}\V\|l>\|t[\|i-\T{1}][\|k]){}$\1\5
\&{break};\2\6
\4${}\}{}$\2\6
${}\|t[\|i][\|k]\K\|l;{}$\6
\4${}\}{}$\2\2\6
\4${}\}{}$\2\par
\U11.\fi

\M{13}\B\X13:Figure out where each element is, and set \PB{\\{tt}} zero\X${}%
\E{}$\6
\&{for} ${}(\|j\K\T{0};{}$ ${}\|j<\\{kk};{}$ ${}\|j\PP){}$\1\6
\&{for} ${}(\|k\K\T{0};{}$ ${}\|k<\|a[\|j+\T{1}];{}$ ${}\|k\PP){}$\5
${}\{{}$\1\6
${}\|l\K\|t[\|j][\|k];{}$\6
${}\\{row}[\|l]\K\|j,\39\\{col}[\|l]\K\|k;{}$\6
${}\\{tt}[\|j][\|k]\K\T{0};{}$\6
\4${}\}{}$\2\2\par
\U11.\fi

\M{14}Now we use a simple backtrack method to build a tableau \PB{\\{tt}} from
which
row-sorting will be standard, by first placing~1 in \PB{\\{tt}}, then~2, etc.

There always is at least one solution, because row sorting does not
``mess up'' column sorting. (See exercise 5.3.4--27.)

\Y\B\4\X14:Run through all solutions\X${}\E{}$\6
\&{for} ${}(\|j\K\T{1};{}$ ${}\|j\Z\\{kk};{}$ ${}\|j\PP){}$\1\5
${}\\{aa}[\|j]\K\T{0};{}$\2\6
${}\\{aa}[\T{0}]\K\\{maxn}+\T{1};{}$\6
${}\|l\K\T{1};{}$\6
\4\\{newlev}:\5
${}\|j\K\T{1};{}$\6
\4\\{tryit}:\5
\&{if} ${}(\\{tt}[\|j-\T{1}][\\{col}[\|l]]\E\T{0}\W\\{aa}[\|j-\T{1}]>\\{aa}[%
\|j]){}$\5
${}\{{}$\1\6
${}\|w[\|l]\K\|j,\39\\{tt}[\|j-\T{1}][\\{col}[\|l]]\K\|l,\39\\{aa}[\|j]\PP;{}$\6
\&{if} ${}(\|l\E\|n){}$\1\5
\X15:Use this solution and go to \PB{\\{levdone}}\X;\2\6
${}\|l\PP;{}$\6
\&{goto} \\{newlev};\6
\4${}\}{}$\2\6
\4\\{tryagain}:\6
\&{if} ${}(\PP\|j\Z\|b[\\{col}[\|l]+\T{1}]){}$\1\5
\&{goto} \\{tryit};\2\6
\4\\{levdone}:\5
${}\|l\MM;{}$\6
\&{if} (\|l)\5
${}\{{}$\1\6
${}\|j\K\|w[\|l],\39\\{tt}[\|j-\T{1}][\\{col}[\|l]]\K\T{0},\39\\{aa}[\|j]%
\MM;{}$\6
\&{goto} \\{tryagain};\6
\4${}\}{}$\2\par
\U11.\fi

\M{15}\B\X15:Use this solution and go to \PB{\\{levdone}}\X${}\E{}$\6
${}\{{}$\1\6
${}\\{sign}\K\\{inv};{}$\6
\&{for} ${}(\|j\K\T{1};{}$ ${}\|j<\\{kk};{}$ ${}\|j\PP){}$\1\6
\&{for} ${}(\|k\K\T{0};{}$ ${}\|k<\|a[\|j+\T{1}];{}$ ${}\|k\PP){}$\1\6
\&{for} ${}(\|l\K\T{0};{}$ ${}\|l<\|j;{}$ ${}\|l\PP){}$\1\6
\&{if} ${}(\\{tt}[\|l][\|k]>\\{tt}[\|j][\|k]){}$\1\5
${}\\{sign}\PP;{}$\2\2\2\2\6
\&{for} ${}(\|k\K\T{0},\39\|j\K\T{1};{}$ ${}\|j<\|n;{}$ ${}\|j\PP){}$\1\5
${}\|k\K\\{trie}[\|k][\|w[\|j]];{}$\2\6
${}\\{rep}[\\{trie}[\|k][\|w[\|n]]]\K(\\{sign}\AND\T{1}\?{-}\T{1}:\T{1});{}$\6
${}\|l\K\|n;{}$\6
${}\|j\K\|w[\|l],\39\\{tt}[\|j-\T{1}][\\{col}[\|l]]\K\T{0},\39\\{aa}[\|j]%
\MM;{}$\6
\&{goto} \\{levdone};\6
\4${}\}{}$\2\par
\U14.\fi

\N{1}{16}Finishing up. The theory to justify all these maneuverings can
be found in many places; I wrote this program after reading
Bruce Sagan's book {\sl The Symmetric Group}, and Bruce based much of
his treatment on G.~D. James's monograph on representation theory
[{\sl Lecture Notes in Mathematics\/ \bf682} (1978)].

Those books use a more complicated ``straightening rule'' to compute
representations and to prove important theorems. But once the theorems are
proved, we can use them to justify the more direct approach taken here.

\Y\B\4\X16:Compute \PB{\|f} basis vectors for the representation\X${}\E{}$\6
\&{for} ${}(\|k\K\T{0};{}$ ${}\|k<\|f;{}$ ${}\|k\PP){}$\5
${}\{{}$\1\6
\&{for} ${}(\|j\K\T{1};{}$ ${}\|j\Z\\{kk};{}$ ${}\|j\PP){}$\1\5
${}\\{aa}[\|j]\K\T{0};{}$\2\6
\&{for} ${}(\|j\K\T{1};{}$ ${}\|j\Z\|n;{}$ ${}\|j\PP){}$\1\5
${}\|l\K\\{stand}[\|k][\|j],\39\|t[\|l-\T{1}][\\{aa}[\|l]]\K\|j,\39\\{aa}[\|l]%
\PP;{}$\2\6
\\{findrep}(\,);\6
\&{for} ${}(\|j\K\T{0};{}$ ${}\|j<\|f;{}$ ${}\|j\PP){}$\1\5
${}\\{basis}[\|k][\|j]\K\\{rep}[\|j];{}$\2\6
\4${}\}{}$\2\par
\U2.\fi

\M{17}\B\X17:Read a permutation \PB{\|p} (but \PB{\&{break}} if there's no more
good data)\X${}\E{}$\6
\&{for} ${}(\|j\K\T{1};{}$ ${}\|j\Z\|n;{}$ ${}\|j\PP){}$\1\6
\&{if} ${}(\\{scanf}(\.{"\%d"},\39{\AND}\|p[\|j])\I\T{1}){}$\1\5
\&{break};\2\2\6
\&{if} ${}(\|j\Z\|n){}$\1\5
\&{break};\2\6
\&{for} ${}(\|j\K\T{1};{}$ ${}\|j\Z\|n;{}$ ${}\|j\PP){}$\1\5
${}\|q[\|j]\K\T{0};{}$\2\6
\&{for} ${}(\|j\K\T{1};{}$ ${}\|j\Z\|n;{}$ ${}\|j\PP){}$\1\6
\&{if} ${}(\|p[\|j]\Z\T{0}\V\|p[\|j]>\|n\V\|q[\|p[\|j]]){}$\5
${}\{{}$\1\6
${}\\{fprintf}(\\{stderr},\39\.{"Not\ a\ permutation\ o}\)\.{f\ \{1,...,\%d%
\}:"},\39\|n);{}$\6
\&{for} ${}(\|j\K\T{1};{}$ ${}\|j\Z\|n;{}$ ${}\|j\PP){}$\1\5
${}\\{fprintf}(\\{stderr},\39\.{"\ \%d"},\39\|p[\|j]);{}$\2\6
${}\\{fprintf}(\\{stderr},\39\.{"!\\n"});{}$\6
${}\\{exit}({-}\T{8});{}$\6
\4${}\}{}$\5
\2\&{else}\1\5
${}\|q[\|p[\|j]]\K\|j{}$;\2\2\par
\U2.\fi

\M{18}\B\X18:Compute the representation of the \PB{\\{jj}}th standard tableau,
permuted by \PB{\|p}\X${}\E{}$\6
\&{for} ${}(\|j\K\T{1};{}$ ${}\|j\Z\\{kk};{}$ ${}\|j\PP){}$\1\5
${}\\{aa}[\|j]\K\T{0};{}$\2\6
\&{for} ${}(\|j\K\T{1};{}$ ${}\|j\Z\|n;{}$ ${}\|j\PP){}$\1\5
${}\|l\K\\{stand}[\\{jj}][\|j],\39\|t[\|l-\T{1}][\\{aa}[\|l]]\K\|p[\|j],\39%
\\{aa}[\|l]\PP;{}$\2\6
\\{findrep}(\,);\par
\U2.\fi

\M{19}\B\X19:Reduce the representation to a linear combination of basis
elements\X${}\E{}$\6
\&{for} ${}(\|j\K\T{0};{}$ ${}\|j<\|f;{}$ ${}\|j\PP){}$\5
${}\{{}$\1\6
${}\|l\K\\{rep}[\|j];{}$\6
\&{if} (\|l)\1\6
\&{for} ${}(\|k\K\|j+\T{1};{}$ ${}\|k<\|f;{}$ ${}\|k\PP){}$\1\5
${}\\{rep}[\|k]\MRL{-{\K}}\|l*\\{basis}[\|j][\|k];{}$\2\2\6
\4${}\}{}$\2\par
\U2.\fi

\N{1}{20}Index.
\fi

\inx
\fin
\con
