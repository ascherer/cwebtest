\input cwebmac
\datethis
\input epsf
\let\from=\gets
\def\bit#1{\\{bit}[#1]}
\def\losub#1{^{\vphantom\prime}_{#1}} % for contexts like $x'_n+x\losub n$


\N{1}{1}Introduction. The purpose of this program is to implement a pretty
algorithm that has a very pleasant theory. But I apologize
at the outset that the algorithm seems to be rather subtle, and
I have not been able to think of any way to explain it to dummies.
Readers who like discrete mathematics and computer science are
encouraged to persevere nonetheless.

The companion program called {\mc KODA-RUSKEY} should probably be
read first, because it solves a significantly simpler problem.
After I wrote that program, Frank Ruskey told me that he had developed
a much more general procedure, in yet-unpublished work with his
student Gang (Kenny) Li.
I~wasn't able to refrain from pondering what their method might
be, so I came up with the algorithm below, which probably produces the
same sequence of outputs as theirs.

\Y\B\8\#\&{include} \.{<stdio.h>}\6
\X6:Type definitions\X\6
\X4:Global variables\X\6
\X15:Subroutines\X\7
\&{int} \\{main}(\&{int} \\{argc}${},\39{}$\&{char} ${}{*}\\{argv}[\,]){}$\1\1%
\2\2\6
${}\{{}$\1\6
\X5:Local variables\X;\6
\X2:Parse the command line\X;\6
\X7:Initialize the data structures\X;\6
\X46:Generate the answers\X;\6
\&{return} \T{0};\6
\4${}\}{}$\2\par
\fi

\M{2}Given a digraph that is {\it totally acyclic}, in the sense that it has
no cycles when we ignore the arc directions, we want to find all ways to
label its vertices with 0s and 1s in such a way that
$x\to y$ implies $\bit x\le\bit y$.
Moreover, we want to list all such labelings as a Gray path,
changing only one bit at a time. The algorithm below does this, with
an extra proviso: Given a designated ``root'' vertex~$v$, $\bit v$
begins at 0 and changes exactly once.

The simple three-vertex digraph with $x\to y\from z$ has only five such
labelings, and they form a Gray path in essentially only one way,
namely $(000,010,011,111,110)$. This example shows that we cannot require
the Gray path to end at a convenient prespecified labeling like $11\ldots1$;
and the dual graph, obtained by reversing all the arrows and complementing
all the bits, shows that we can't require the path to start at $00\ldots0$.
[Generalizations of this example, in which the vertices are
$\{x_0,x_1,\ldots,x_n\}$ and each arc is either $x_{k-1}\to x_k$ or
$x_{k-1}\from x_k$, have solutions related to continued fractions.
Interested readers will enjoy working out the details.]

We may assume that the given graph is connected.
It is convenient to describe a connected, rooted totally acyclic digraph by
calling the root~0, and by assigning the integer names $\{1,\ldots,n\}$
to the other vertices in such a way that the undirected path connecting
each vertex to~0 runs through vertex names monotonically. In other
words, we specify for each vertex $k>0$ a vertex $j_k<k$ such that
either $j_k\to k$ or $j_k\from k$. Notice that there are
$2^n\,n!$ ways to do this. The program below works on the graph defined
by a sequence of the form ${\pm}j_1{\pm}j_2\ldots{\pm}j_n$, using
a plus sign when $j_k\to k$ and a minus sign when $j_k\from k$.
For example, the digraph $0\to1\from2$ would be defined by \.{+0-1},
while $1\to0\from2$ would be \.{-0-0}.

We may assume further that the vertices of the tree have been numbered
in {\it preorder}, namely that we do not have $j_k<j_l<k<l$ for any
$k$ and~$l$. This assumption limits the number of possibilities to
$2^n{2n\choose n}{1\over n+1}$.

Breaking the arc between $j_k$ and $k$ disconnects the graph into two
pieces, one of which is a totally acyclic digraph rooted at~$k$. Our
algorithm will use the Gray paths obtained from these smaller subgraphs
to construct a Gray path for the whole graph.

\Y\B\4\D$\\{maxn}$ \5
\T{100}\C{ limit in number of vertices }\par
\B\4\D$\\{abort}(\|f,\|d,\|n)$ \6
${}\{{}$\5
\1${}\\{fprintf}(\\{stderr},\39\|f,\39\|d){}$;\5
${}\\{exit}({-}\|n){}$;\5
${}\}{}$\2\par
\Y\B\4\X2:Parse the command line\X${}\E{}$\6
${}\{{}$\1\6
\&{register} \&{char} ${}{*}\|c\K\\{argv}[\T{1}];{}$\7
\&{if} ${}(\\{argc}\I\T{2}){}$\1\5
${}\\{abort}(\.{"Usage:\ \%s\ graphspec}\)\.{\\n"},\39\\{argv}[\T{0}],\39%
\T{1});{}$\2\6
\&{for} ${}(\|k\K\T{1};{}$ ${}{*}\|c;{}$ ${}\|k\PP){}$\5
${}\{{}$\1\6
\&{if} ${}(\|k\G\\{maxn}){}$\1\5
${}\\{abort}(\.{"Sorry,\ I\ can\ only\ h}\)\.{andle\ \%d\ vertices!\\n}\)\.{"},%
\39\\{maxn},\39\T{2});{}$\2\6
\&{if} ${}({*}\|c\E\.{'+'}){}$\1\5
${}\\{js}[\|k]\K\T{0};{}$\2\6
\&{else} \&{if} ${}({*}\|c\E\.{'-'}){}$\1\5
${}\\{js}[\|k]\K\T{1};{}$\2\6
\&{else}\1\5
${}\\{abort}(\.{"Parsing\ error:\ `\%s'}\)\.{\ should\ start\ with\ +}\)\.{\ or%
\ -!\\n"},\39\|c,\39\T{3});{}$\2\6
\&{for} ${}(\|j\K\T{0},\39\|c\PP;{}$ ${}{*}\|c\G\.{'0'}\W{*}\|c\Z\.{'9'};{}$
${}\|c\PP){}$\1\5
${}\|j\K\T{10}*\|j+{*}\|c-\.{'0'};{}$\2\6
\X3:Abort if \PB{\|j} is not a legal value for $j_k$\X;\6
${}\\{jj}[\|k]\K\|j;{}$\6
\4${}\}{}$\2\6
${}\|n\K\|k-\T{1};{}$\6
\4${}\}{}$\2\par
\U1.\fi

\M{3}Here we test the preorder condition, using the fact that \PB{$\\{stack}[%
\T{0}]\K\T{0}$}
and that \PB{\|l} is initially 0.

\Y\B\4\X3:Abort if \PB{\|j} is not a legal value for $j_k$\X${}\E{}$\6
\&{if} ${}(\|j\G\|k){}$\5
${}\{{}$\1\6
${}\\{fprintf}(\\{stderr},\39\.{"Parsing\ error:\ `\%d\%}\)\.{s'\ should\ start%
\ "},\39\|j,\39\|c);{}$\6
${}\\{abort}(\.{"with\ a\ number\ less\ }\)\.{than\ \%d!\\n"},\39\|k,\39%
\T{4});{}$\6
\4${}\}{}$\2\6
\&{while} ${}(\|j<\\{stack}[\|l]){}$\1\5
${}\|l\MM;{}$\2\6
\&{if} ${}(\|j\I\\{stack}[\|l]){}$\5
${}\{{}$\1\6
${}\\{fprintf}(\\{stderr},\39\.{"Parsing\ error:\ `\%d\%}\)\.{s'\ shouldn't\
start\ "},\39\|j,\39\|c);{}$\6
${}\\{fprintf}(\\{stderr},\39\.{"with\ a\ number\ betwe}\)\.{en\ \%d\ "},\39%
\\{stack}[\|l]);{}$\6
${}\\{abort}(\.{"and\ \%d!\\n"},\39\\{stack}[\|l+\T{1}],\39\T{5});{}$\6
\4${}\}{}$\2\6
${}\\{stack}[\PP\|l]\K\|k{}$;\par
\U2.\fi

\M{4}\B\X4:Global variables\X${}\E{}$\6
\&{char} \\{jj}[\\{maxn}];\C{ the vertex $j_k$ }\6
\&{char} \\{js}[\\{maxn}];\C{ 0 if $j_k\to k$,\quad 1 if $j_k\from k$ }\6
\&{char} \\{stack}[\\{maxn}];\C{ values that are legitimate for the next $j_k$
}\par
\As8, 14, 24, 30, 32, 37, 47, 57\ETs60.
\U1.\fi

\M{5}\B\X5:Local variables\X${}\E{}$\6
\&{register} \&{int} \|j${},{}$ \|k${},{}$ \|l${}\K\T{0}{}$;\C{ heavily-used
miscellaneous indices }\6
\&{int} \|n;\C{ size of the input graph }\par
\As9\ET45.
\U1.\fi

\M{6}Consider the example
$$\epsfbox{li-ruskey.1}$$
in which all arcs are directed upward; it would be written
\.{+0+1-2+1+0-5-0+7} in the notation above. A nonroot vertex~$k$ is
called {\it positive\/} if $j_k\to k$ and {\it negative\/} if $j_k\from k$;
thus $\{1,2,4,5,8\}$ are positive in this example, and $\{3,6,7\}$ are
negative.

We write $x\preceq y$ if there is a directed path from $x$ to $y$.
Removing all vertices $x$ such that $x\preceq 0$ disconnects the graph
into a number of pieces having positive roots; in our example, removing
$\{0,7\}$ leaves three components rooted at $\{1,5,8\}$. We call these roots
the set of {\it positive vertices near\/}~0, and denote that set by~$A$.
Similarly, the {\it negative vertices near\/}~0 are obtained when we remove
all $y$ such that $0\preceq y$; the set of resulting roots, denoted by~$B$, is
$\{3,6,7\}$ in our example.

Why are the sets $A$ and $B$ so important? Because
the labelings for which $\bit0=0$ are
precisely those that we obtain by setting $\bit x=0$ for all $x\preceq0$
and then labeling the subgraphs rooted at $a$ for $a\in A$.
Similarly, all labelings for which $\bit0=1$ are obtained by
setting $\bit y=1$ for all $0\preceq y$ and labeling the subgraphs
rooted at $b$ for $b\in B$.

Thus if $n_k$ is the number of labelings of the subgraph rooted at~$k$,
the total number of labelings of the whole graph is
$\prod_{a\in A}n_a\,+\,\prod_{b\in B}n_b$.

\smallskip
For each subgraph rooted at $k$, we define the positive and negative
vertices near~$k$ in the same way as we did for the case $k=0$,
calling those sets $A_k$ and $B_k$.

Every positive vertex adjacent to 0 appears in $A$, and every negative
vertex adjacent to~0 appears in $B$. These are called the {\it principal\/}
elements of~$A$ and~$B$. Every nonprincipal member of~$A$ is a member
of $A_b$ for some unique principal vertex~$b\in B$. Similarly, every
nonprincipal member of~$B$ is a member of $B_a$ for some unique
principal vertex $a\in A$. For example, 8~belongs to $A_7$, and
$3\in B_1$.

\Y\B\4\X6:Type definitions\X${}\E{}$\6
\&{typedef} \&{struct} \&{info\_struct} ${}\{{}$\1\6
\&{char} \\{id};\C{ name of vertex in an $A$ or $B$ set }\7
\X12:Other fields of an \&{info} record\X;\7
\&{struct} \&{info\_struct} ${}{*}\\{sib}{}$;\C{ previous element in the set }\6
\&{struct} \&{info\_struct} ${}{*}\\{ref}{}$;\C{ further info about
nonprincipal element }\2\6
${}\}{}$ \&{info};\par
\As29\ET55.
\U1.\fi

\M{7}\B\X7:Initialize the data structures\X${}\E{}$\6
$\\{list}[\T{0}][\T{0}]\K\\{list}[\T{0}][\T{1}]\K\NULL;{}$\6
\&{for} ${}(\|k\K\T{1},\39\|i\K{\AND}\\{infonode}[\T{0}];{}$ ${}\|k\Z\|n;{}$
${}\|k\PP,\39\|i\PP){}$\1\6
\&{for} ${}(\|j\K\\{jj}[\|k],\39\\{ii}\K\NULL;{}$  ; ${}\\{ii}\K\|i\PP,\39\|j\K%
\\{jj}[\|j]){}$\5
${}\{{}$\1\6
${}\|i\MG\\{id}\K\|k,\39\|i\MG\\{ref}\K\\{ii};{}$\6
${}\|i\MG\\{sib}\K\\{list}[\|j][\\{js}[\|k]],\39\\{list}[\|j][\\{js}[\|k]]\K%
\|i;{}$\6
\&{if} ${}(\|j\E\T{0}\V\\{js}[\|j]\E\\{js}[\|k]){}$\1\5
\&{break};\2\6
\4${}\}{}$\2\2\par
\As13, 16, 23, 31\ETs61.
\U1.\fi

\M{8}\B\X4:Global variables\X${}\mathrel+\E{}$\6
\&{info} ${}\\{infonode}[((\\{maxn}*\\{maxn})\GG\T{2})+(\\{maxn}\GG\T{1})]{}$;%
\C{ elements of $A_k$ and $B_k$ }\6
\&{info} ${}{*}\\{list}[\\{maxn}][\T{2}]{}$;\C{ heads of those sets }\par
\fi

\M{9}\B\X5:Local variables\X${}\mathrel+\E{}$\6
\&{info} ${}{*}\|i,{}$ ${}{*}\\{ii}{}$;\C{ pointers to \PB{\&{info}} nodes }\par
\fi

\M{10}Aha! We can begin to see how to get the desired Gray path. The well-known
reflected Gray code for mixed-radix number systems tells us how to obtain a
path $P_0$ of length $\prod_{a\in A}n_a$ for the labelings with $\bit0=0$ as
well as a path $P_1$ of length $\prod_{b\in B}n_b$ for the labelings with
$\bit0=1$. All we have to do is figure out a way to end $P_0$ with
a labeling that differs only in $\bit0$ from the starting point of $P_1$.

Suppose $P_0$ begins with $\bit a=\alpha_a$ for each $a\in A$,
and $P_1$ ends with $\bit b=\beta_b$ for each $b\in B$. Then $P_0$
ends with $\bit a=\alpha_a'=(\alpha_a+\delta_a)\bmod 2$ and $P_1$ begins
with $\bit b=\beta_b'=(\beta_b+\epsilon_b)\bmod 2$, where
$$\delta_a=\prod_{\scriptstyle a'<a\atop\scriptstyle a'\in A} n_{a'},\qquad
\epsilon_b=\prod_{\scriptstyle b'<b\atop\scriptstyle b'\in B} n_{b'}.$$

These equations have a unique solution such that
the final labeling of $P_0$ is adjacent to the first labeling of~$P_1$.
Namely, we set $\alpha_a'=1$ for all principal elements $a\in A$,
and $\beta_b'=0$ for all principal elements $b\in B$. And if $a$ is
nonprincipal but $a\in A_b$, where $b$ is principal, we let
$\alpha_a'=\alpha_{ba}$, the value of $\bit a$ at the beginning
of the Gray path for the subgraph rooted at~$b$. Similarly, if $b$ is
nonprincipal but $b\in B_a$, where $a$ is principal, we let
$\beta_b'=\beta_{ab}$. These formulas and those of the
previous paragraph determine the values of $\alpha_a$ and $\beta_b$
for all $a\in A$ and $b\in B$.

For example, the calculations yield the following numbers when
we apply them to the smaller subgraphs, from the bottom up:
$$\vbox{\halign{$#\hfill$&&\quad$#\hfill$&\qquad$#\hfill$\cr
A_8=\emptyset&&B_8=\emptyset&&n_8=1+1=2\cr
A_7=\{8\}&\alpha_{78}'=1;\ \alpha\losub{78}=0&B_7=\emptyset&&n_7=2+1=3\cr
A_6=\emptyset&&B_6=\emptyset&&n_6=1+1=2\cr
A_5=\emptyset&&B_5=\{6\}&\beta_{56}'=0;\ \beta\losub{56}=1&n_5=1+2=3\cr
A_4=\emptyset&&B_4=\emptyset&&n_4=1+1=2\cr
A_3=\emptyset&&B_3=\emptyset&&n_3=1+1=2\cr
A_2=\emptyset&&B_2=\{3\}&\beta_{23}'=0;\ \beta\losub{23}=1&n_2=1+2=3\cr
A_1=\{2,4\}&\alpha_{12}'=1,\alpha_{14}'=1;\ \alpha\losub{12}=0,
\alpha\losub{14}=0&
B_1=\{3\}&\beta_{13}'=1;\ \beta\losub{13}=0&n_1=6+2=8\cr}}$$
For the graph as a whole, therefore, which has
$A_0=\{1,5,8\}$ and $B_0=\{3,6,7\}$, we have
$\alpha_1'=1$, $\alpha_5'=1$, $\alpha_8'=\alpha_{78}=0$;
$\alpha\losub1=0$, $\alpha\losub5=1$, $\alpha\losub8=0$;
$\beta_3'=\beta_{13}=0$, $\beta_6'=\beta_{56}=1$, $\beta_7'=0$;
$\beta\losub3=1$, $\beta\losub6=1$, $\beta\losub7=0$. There are
$n_0=48+12=60$ possible labelings altogether.

The Gray path for a trivial subgraph like $G_8$, $G_6$, $G_4$, or $G_3$
is simply `0,~1'. For $G_7$ on bits 7 and 8, the path is
`00, 01, 11'. For $G_5$ on bits 5 and 6, or $G_2$ on bits 2 and~3,
the path is `00, 10, 11'. And for $G_1$ on bits 1234, the path of length~8 is
$$0000,\ 0001,\ 0101,\ 0100,\ 0110,\ 0111,\ 1111,\ 1101.$$
(Notice that it ends with $\bit3=\beta_{13}=0$.)

\fi

\M{11}The overall Gray path for our example starts out with
$\bit1=\alpha_1=0$, $\bit5=\alpha_5=1$,
and $\bit8=\allowbreak\alpha_8=0$,
so the first $n_1n_5n_8=48$ labelings $\bit0\ldots\bit8$ are
$$\vbox{\halign{$#$\hfill\cr
0\,0000\,11\,0\,0\cr
0\,0000\,11\,0\,1\cr
0\,0000\,10\,0\,1\cr
0\,0000\,10\,0\,0\cr
0\,0000\,00\,0\,0\cr
0\,0000\,00\,0\,1\cr
0\,0001\,00\,0\,1\cr
0\,0001\,00\,0\,0\cr
\qquad\vdots\cr
0\,1101\,11\,0\,0\,.\cr}}$$
Then comes the change to $\bit0$, and we finish up with $n_3n_6n_7=12$ more:
$$\vbox{\halign{$#$\hfill\cr
1\,11\,0\,11\,1\,00\cr
1\,11\,0\,11\,1\,01\cr
1\,11\,0\,11\,1\,11\cr
1\,11\,0\,11\,0\,11\cr
1\,11\,0\,11\,0\,01\cr
1\,11\,0\,11\,0\,00\cr
1\,11\,1\,11\,0\,00\cr
\qquad\vdots\cr
1\,11\,1\,11\,1\,00\,.\cr}}$$

\fi

\M{12}\B\X12:Other fields of an \&{info} record\X${}\E{}$\6
\&{char} \\{alfprime};\C{ $\alpha'_{ka}$ or $\beta'_{kb}$ }\6
\&{char} \\{del};\C{ $\delta_{ka}\bmod2$ or $\epsilon_{kb}\bmod2$ }\par
\U6.\fi

\M{13}We need not actually calculate the values $n_k$ exactly; we only need
to know if $n_k$ is even or odd. However, this program does compute the
exact values (unless they exceed our computer's word size).

\Y\B\4\X7:Initialize the data structures\X${}\mathrel+\E{}$\6
\&{for} ${}(\|k\K\|n;{}$ ${}\|k\G\T{0};{}$ ${}\|k\MM){}$\5
${}\{{}$\1\6
\&{register} \&{int} \|s${},{}$ \|t;\7
\&{for} ${}(\|s\K\T{1},\39\|i\K\\{list}[\|k][\T{0}],\39\\{ii}\K\NULL;{}$ \|i;
${}\|s\MRL{*{\K}}\\{nn}[\|i\MG\\{id}],\39\|i\K\|i\MG\\{sib}){}$\5
${}\{{}$\1\6
${}\|i\MG\\{del}\K\T{1};{}$\6
\&{if} ${}((\\{nn}[\|i\MG\\{id}]\AND\T{1})\E\T{0}){}$\1\5
${}\\{ii}\K\|i;{}$\2\6
\&{if} ${}(\|i\MG\\{ref}){}$\1\5
${}\|i\MG\\{alfprime}\K\|i\MG\\{ref}\MG\\{alfprime}\XOR\|i\MG\\{ref}\MG%
\\{del};{}$\2\6
\&{else}\1\5
${}\|i\MG\\{alfprime}\K\T{1};{}$\2\6
\4${}\}{}$\2\6
\&{for} ${}(\|i\K\\{list}[\|k][\T{0}];{}$ \\{ii}; ${}\|i\K\|i\MG\\{sib}){}$\1\6
\&{if} ${}(\|i\E\\{ii}){}$\1\5
${}\\{ii}\K\NULL{}$;\5
\2\&{else}\1\5
${}\|i\MG\\{del}\K\T{0};{}$\2\2\6
\&{for} ${}(\|t\K\T{1},\39\|i\K\\{list}[\|k][\T{1}],\39\\{ii}\K\NULL;{}$ \|i;
${}\|t\MRL{*{\K}}\\{nn}[\|i\MG\\{id}],\39\|i\K\|i\MG\\{sib}){}$\5
${}\{{}$\1\6
${}\|i\MG\\{del}\K\T{1};{}$\6
\&{if} ${}((\\{nn}[\|i\MG\\{id}]\AND\T{1})\E\T{0}){}$\1\5
${}\\{ii}\K\|i;{}$\2\6
\&{if} ${}(\|i\MG\\{ref}){}$\1\5
${}\|i\MG\\{alfprime}\K\|i\MG\\{ref}\MG\\{alfprime}\XOR\|i\MG\\{ref}\MG%
\\{del};{}$\2\6
\&{else}\1\5
${}\|i\MG\\{alfprime}\K\T{0};{}$\2\6
\4${}\}{}$\2\6
\&{for} ${}(\|i\K\\{list}[\|k][\T{1}];{}$ \\{ii}; ${}\|i\K\|i\MG\\{sib}){}$\1\6
\&{if} ${}(\|i\E\\{ii}){}$\1\5
${}\\{ii}\K\NULL{}$;\5
\2\&{else}\1\5
${}\|i\MG\\{del}\K\T{0};{}$\2\2\6
${}\\{nn}[\|k]\K\|s+\|t;{}$\6
\4${}\}{}$\2\par
\fi

\M{14}\B\X4:Global variables\X${}\mathrel+\E{}$\6
\&{int} \\{nn}[\\{maxn}];\C{ $n_k$, the number of labelings of $G_k$ }\par
\fi

\M{15}Here are two subroutines that I used when debugging.

\Y\B\4\X15:Subroutines\X${}\E{}$\6
\&{void} \\{print\_info}(\&{int} \|k)\1\1\2\2\6
${}\{{}$\1\6
\&{register} \&{info} ${}{*}\|i;{}$\7
${}\\{printf}(\.{"Info\ for\ vertex\ \%d:}\)\.{\ A\ ="},\39\|k);{}$\6
\&{for} ${}(\|i\K\\{list}[\|k][\T{0}];{}$ \|i; ${}\|i\K\|i\MG\\{sib}){}$\1\5
${}\\{printf}(\.{"\ \%d"},\39\|i\MG\\{id});{}$\2\6
\&{if} (\\{list}[\|k][\T{0}])\1\5
\\{printf}(\.{",\ B\ ="});\5
\2\&{else}\1\5
\\{printf}(\.{"\ (),\ B\ ="});\2\6
\&{for} ${}(\|i\K\\{list}[\|k][\T{1}];{}$ \|i; ${}\|i\K\|i\MG\\{sib}){}$\1\5
${}\\{printf}(\.{"\ \%d"},\39\|i\MG\\{id});{}$\2\6
\&{if} (\\{list}[\|k][\T{1}])\1\5
\\{printf}(\.{"\\n"});\5
\2\&{else}\1\5
\\{printf}(\.{"\ ()\\n"});\2\6
\&{for} ${}(\|i\K\\{list}[\|k][\T{0}];{}$ \|i; ${}\|i\K\|i\MG\\{sib}){}$\1\5
${}\\{printf}(\.{"\ alf\%d=\%d,\ alf\%d'=\%}\)\.{d\\n"},\39\|i\MG\\{id},\39\|i%
\MG\\{alfprime}\XOR\|i\MG\\{del},\39\|i\MG\\{id},\39\|i\MG\\{alfprime});{}$\2\6
\&{for} ${}(\|i\K\\{list}[\|k][\T{1}];{}$ \|i; ${}\|i\K\|i\MG\\{sib}){}$\1\5
${}\\{printf}(\.{"\ bet\%d=\%d,\ bet\%d'=\%}\)\.{d\\n"},\39\|i\MG\\{id},\39\|i%
\MG\\{alfprime}\XOR\|i\MG\\{del},\39\|i\MG\\{id},\39\|i\MG\\{alfprime});{}$\2\6
\4${}\}{}$\2\7
\&{void} \\{print\_all\_info}(\&{int} \|n)\1\1\2\2\6
${}\{{}$\1\6
\&{register} \&{int} \|k;\7
\&{for} ${}(\|k\K\T{0};{}$ ${}\|k\Z\|n;{}$ ${}\|k\PP){}$\1\5
\\{print\_info}(\|k);\2\6
\4${}\}{}$\2\par
\As58\ET59.
\U1.\fi

\M{16}\B\X7:Initialize the data structures\X${}\mathrel+\E{}$\6
\&{if} (\\{verbose})\5
${}\{{}$\1\6
\\{print\_all\_info}(\|n);\6
${}\\{printf}(\.{"(Altogether\ \%d\ solu}\)\.{tions.)\\n"},\39\\{nn}[%
\T{0}]);{}$\6
\4${}\}{}$\2\par
\fi

\N{1}{17}Coroutines. As in the program {\mc KODA-RUSKEY}, we can represent
the path-generation process by considering a system of cooperating programs,
each of which has the same basic structure. There is one coroutine
for each pair of nodes $(k,l)$ such that $k\in A_l$ or $k\in B_l$,
and it looks essentially like this:
$$\vbox{\halign{#\hfil\cr
\&{coroutine} \PB{\|p(\,)}\cr
\quad\PB{$\{$}\cr
\quad\quad\PB{\&{while} (\T{1}) $\{$}\cr
\quad\quad\quad\PB{ \&{while} ${}(\|p\MG\\{child\_A}(\,)){}$ \&{return} %
\\{true};\2}\cr
\quad\quad\quad\PB{$\|p\MG\\{bit}\K\T{1}{}$; \&{return} \\{true};}\cr
\quad\quad\quad\PB{ \&{while} ${}(\|p\MG\\{child\_B}(\,)){}$ \&{return} %
\\{true};\2}\cr
\quad\quad\quad\PB{\&{return} \|p${}\MG\\{sib}(\,);$}\cr
\quad\quad\quad\PB{ \&{while} ${}(\|p\MG\\{child\_B}(\,)){}$ \&{return} %
\\{true};\2}\cr
\quad\quad\quad\PB{$\|p\MG\\{bit}\K\T{0}{}$; \&{return} \\{true};}\cr
\quad\quad\quad\PB{ \&{while} ${}(\|p\MG\\{child\_A}(\,)){}$ \&{return} %
\\{true};\2}\cr
\quad\quad\quad\PB{\&{return} \|p${}\MG\\{sib}(\,);$}\cr
\quad\quad\PB{$\}$}\cr
\quad\PB{$\}$}\cr
}}$$
Here \PB{$\|p\MG\\{child\_A}$} is a pointer to the rightmost element of $A_k$,
and
\PB{$\|p\MG\\{child\_B}$} points to the rightmost element of $B_k$; \PB{$\|p\MG%
\\{sib}$} points
to the nearest sibling of \PB{\|p} to the left, in $A_l$ or $B_l$.
If any of these pointers is \PB{$\NULL$}, the corresponding coroutine
\PB{$\NULL(\,)$} is assumed to simply return \PB{\\{false}}.

The reader is urged to compare this coroutine with the analogous one
in {\mc KODA-RUSKEY}, which is the special case where \PB{$\|p\MG\\{child\_A}\K%
\NULL$}
for all~\PB{\|p}.

Suppose \PB{$\|p\MG\\{child\_A}(\,)$} first returns \PB{\\{false}} after it has
been called
$\alpha$ times; thus \PB{$\|p\MG\\{child\_A}(\,)$} generates $\alpha$ different
labelings,
including the initial one. Similarly, suppose that \PB{$\|p\MG\\{child\_B}(%
\,)$}
and \PB{$\|p\MG\\{sib}(\,)$} generate $\beta$ and $\sigma$
different labelings, respectively, before first returning \PB{\\{false}}.
Then the coroutine \PB{\|p(\,)}
itself will generate $\sigma(\alpha+\beta)$ labelings between the times
when it returns \PB{\\{false}}.
The final labeling for \PB{\|p} will be the final labeling
for \PB{$\|p\MG\\{sib}$}, together with either the initial or final labeling of
the subtree
rooted at~\PB{\|k}, depending on whether $\sigma$ is even or odd, respectively.

After the coroutine has first returned \PB{\\{false}}, invoking it again will
cause it to generate the labelings in reverse order before
returning \PB{\\{false}} a second time. Then the process repeats.

\fi

\M{18}How many coroutines can there be? Our example graph defines a total of
13 coroutines (including a coroutine~0, which corresponds to the special
case where $k=0$ and there are no siblings).

It isn't difficult to prove that the worst case, about $n^2\!/4$, occurs in
graphs that have a specification like \.{+0+1+2+3+4-5-5-5-5-5}.
Such graphs have roughly $n$ times $2^{n/2}$ labelings, so we do not have to be
embarrassed about the nonlinear initialization time needed to set up
data structures and to compute the values $\alpha\losub{ka}$, $\alpha'_{ka}$,
$\beta\losub{kb}$, and $\beta'_{kb}$.

Indeed, if $k$ is a positive vertex, it appears in $r$ coroutines
$(k,l_1)$, \dots, $(k,l_r)$ if and only if the path from $k$ to~0
has the form $k\from l_1\to\cdots\to l_r=0$ or begins with
$k\from l_1\to\cdots\to l_r\from l_{r+1}$.
If $k$ is a negative vertex, it appears in $r$ coroutines
$(k,l_1)$, \dots, $(k,l_r)$ if and only if the path from $k$ to~0
has the form $k\to l_1\from\cdots\from l_r=0$ or begins with
$k\to l_1\from\cdots\from l_r\to l_{r+1}$.
Only the coroutine $(k,l_1)$ is principal. Notice that if $r>1$,
nodes $(l_1,\ldots,l_{r-1})$ appear only in their principal coroutine,
because a nonprincipal coroutine arises only at points where the path to~0
switches directions.

If there are $s$ direction-switching vertices $k$ such that the path to 0
begins $k\to l_1\from l_2$, the $2^s$ independent settings of their bits all
appear in at least one labeling.
And if there are $t$ direction-switching vertices $k$ such that the path to 0
begins $k\from l_1\to l_2$, the same remark applies to the $2^t$ independent
settings of {\it their\/} individual bits.
Therefore if the number of coroutines $(k,l)$ exceeds $(c+1)n$, the
number of labelings must exceed $2^{\,\max(s,t)}\ge2^{\,c/2}$.

\fi

\M{19}In Section 1.4.2 of {\sl Fundamental Algorithms}, I wrote,
``Initialization of coroutines tends
to be a little tricky, although not really difficult.''
Perhaps I should reconsider that statement in the light of the present
program, because the initialization of all the coroutines considered
here is more than a little tricky.

In fact, I've decided not to implement the coroutines directly,
although a simulation along the lines of {\mc KODA-RUSKEY} would
make a nice exercise. My real goal is to come up with a loopless
implementation, because I was told that no such implementation
(nor even an implementation with constant {\it amortized\/} time)
was presently known.

[Readers who do take the time to work out the exercise suggested in the
previous paragraph will notice a interesting difference with respect to
the previous case: The \PB{\\{parent}} pointers in
the coroutine implementation of {\mc KODA-RUSKEY} are essentially
static, but now they must in general be dynamic. For example,
in a graph like $0\from1\to2\from3$, coroutine $(3,2)$ is called
by both $(2,0)$ and $(2,1)$. This is related to the key difficulty
we will face in coming up with a loopless way to solve our
more general problem.]

\fi

\N{1}{20}A generalized fringe. The loopless implementation in {\mc KODA-RUSKEY}
is based on a notion called the {\it fringe}, which is a linear list
containing nodes that are either {\it active\/} or {\it passive}.
We are dealing here with a generalization of the problem solved there,
so we naturally seek an extension of the fringe concept in hopes that
fringes will help us again.

Each loopless {\mc KODA-RUSKEY} step consists of four basic operations:
\smallskip
\indent\indent 1) Find the rightmost active node, \PB{\|p};\par
\indent\indent 2) Complement $\bit p$;\par
\indent\indent 3) Insert or delete appropriate
fringe nodes at the right of \PB{\|p};\par
\indent\indent 4) Make \PB{\|p} passive and activate all nodes to its right.
\smallskip\noindent
Suitable data structures make all these operations efficient.
Fortunately, the same scheme does in fact handle our more general problem,
except that operation (3) must now involve both insertion and deletion.

The root node 0 is always present in the fringe.
And if $k$ is any node in the fringe, it is immediately followed
by its current {\it descendants}, which are defined as follows:
If $\bit k=0$, the descendants of $k$ are the nodes $a$ for $a\in A_k$
and their descendants. If $\bit k=1$, the descendants of $k$ are the nodes
$b$ for $b\in B_k$ and their descendants.

\fi

\M{21}An example will make this clear; for simplicity, we will consider only
the subgraph $G_1$ on the vertices $\{1,2,3,4\}$ of our larger example.
The initial labeling 0000 means that the fringe begins with three nodes
$$1_0\quad 2_0\quad 4_0\,,$$
where the subscript on $k$ indicates the value of $\bit k$. Nodes
$2_0$ and $4_0$ have no descendants, because $A_2=A_4=\emptyset$.
All nodes are initially active.

\def\p#1{\overline#1}
Since $4_0$ is the rightmost active node, we complement $\bit4$, and
the fringe becomes
$$1_0\quad 2_0\quad \p4_1\,.$$
The bar over 4 indicates that this node is now passive. Again, $\p4_1$
has no descendants, this time because $B_4=\emptyset$.

The next step complements $\bit2$, and $B_2=\{3\}$ now enters the fringe:
$$1_0\quad \p2_1\quad 3_0\quad 4_1\,.$$
The initial value of $\bit3$ is $\beta'_{23}=0$.

\fi

\M{22}Proceeding in this way,
the first eight steps can be summarized as follows:
$$\vbox{\halign{$#\hfil$&
\quad\smash{\lower.5\baselineskip\hbox{$\cdots$ complement $\bit#$\hfil}}\cr
1_0\quad 2_0\quad 4_0&4\cr
1_0\quad 2_0\quad \p4_1&2\cr
1_0\quad \p2_1\quad 3_0\quad 4_1&4\cr
1_0\quad \p2_1\quad 3_0\quad \p4_0&3\cr
1_0\quad \p2_1\quad \p3_1\quad 4_0&4\cr
1_0\quad \p2_1\quad \p3_1\quad \p4_1&1\cr
\p1_1\quad 3_1&3\cr
\p1_1\quad \p3_0\cr}}$$
and at this point all of the labelings have been generated.
Restarting the process causes
them to be regenerated in reverse order, stopping again when the original
starting point is reached:
$$\vbox{\halign{$#\hfil$&
\quad\smash{\lower.5\baselineskip\hbox{$\cdots$ complement $\bit#$\hfil}}\cr
1_1\quad 3_0&3\cr
1_1\quad \p3_1&1\cr
\p1_0\quad 2_1\quad 3_1\quad 4_1&4\cr
\p1_0\quad 2_1\quad 3_1\quad \p4_0&3\cr
\p1_0\quad 2_1\quad \p3_0\quad 4_0&4\cr
\p1_0\quad 2_1\quad \p3_0\quad \p4_1&2\cr
\p1_0\quad \p2_0\quad 4_1&4\cr
\p1_0\quad \p2_0\quad \p4_0\cr}}$$

\fi

\M{23}Let's look more closely at what happens when $\bit k$ changes its state.
Suppose the rightmost active node is~$k_0$. Then $k_0$ is immediately followed
in the fringe by its descendants; those descendants were all passive, but
we can reactivate them in anticipation of step~(4). The current descendants
of $k_0$ are the elements of $A_k$ and {\it their\/} descendants, and each
element $a\in A_k$ has $\bit a=\alpha'_{ka}$.

We will say that the {\it entourage\/} of $k_0$ is the contents of the
fringe at the beginning of the process that would generate the Gray path
for the subgraph rooted at~$k$; and the entourage of $k_1$ is the
contents of the fringe at the end of that process. Thus, the small
example we've just seen shows us that the entourage of $1_0$ is
$1_0\,2_0\,4_0$, and the entourage of $1_1$ is $1_1\,3_0$.

In general, the entourage of $k_0$ consists of $k_0$ itself followed by
the consecutive entourages of $a_t$ for each $a\in A_k$, with $t=\alpha_{ka}$,
in order of increasing~$a$. We find therefore that the entourage of $0_0$
in our main example is
$$0_0\quad 1_0\quad 2_0\quad 4_0\quad 5_1\quad 6_1\quad 8_0\,;$$
this is the initial contents of the fringe. The corresponding entourage of
$0_1$---the final fringe---is
$$0_1\quad 3_1\quad 6_1\quad 7_0\quad 8_0\,.$$

The final element of an entourage always has the form $j_s$ where
either $s=0$ and $A_j=\emptyset$ or $s=1$ and $B_j=\emptyset$.
Here is a short program that locates the final~$j$ value at the
end of the entourage for $k_t$:

\Y\B\4\X7:Initialize the data structures\X${}\mathrel+\E{}$\6
\&{for} ${}(\|k\K\|n;{}$ ${}\|k\G\T{0};{}$ ${}\|k\MM){}$\1\6
\&{for} ${}(\|j\K\T{0};{}$ ${}\|j\Z\T{1};{}$ ${}\|j\PP){}$\5
${}\{{}$\1\6
${}\|i\K\\{list}[\|k][\|j];{}$\6
\&{if} (\|i)\1\5
${}\\{fj}[\|k][\|j]\K\\{fj}[\|i\MG\\{id}][\|i\MG\\{alfprime}\XOR\|i\MG%
\\{del}];{}$\2\6
\&{else}\1\5
${}\\{fj}[\|k][\|j]\K\|k;{}$\2\6
\4${}\}{}$\2\2\par
\fi

\M{24}\B\X4:Global variables\X${}\mathrel+\E{}$\6
\&{char} \\{fj}[\\{maxn}][\T{2}];\C{ final vertices in the entourages of $k_0$,
$k_1$ }\par
\fi

\M{25}At the point where $\bit k$ is about to change from 0 to 1, the fringe
following $k_0$ contains $k$'s {\it transition string\/} $\tau_{k0}$, which
consists again of the consecutive entourages $a_t$ of all $a\in A_k$, but
this time with $t=\alpha'_{ka}$ instead of $t=\alpha\losub{ka}$.
The fringe is then modified from `$\ldots\,k_0\,\tau_{k0}\,\ldots$' to
`$\ldots\,k_1\,\tau_{k1}\,\ldots$',
where the transition string $\tau_{k1}$ consists of the consecutive
entourages $b_t$ of all $b\in B_k$, with $t=\beta'_{kb}$.

Thus the two transition strings for vertex 0 in our main example are
$\tau_{00}=1_1\,3_0\,5_1\,6_1\,8_0$ and $\tau_{01}=3_0\,6_1\,7_0\,8_0$.
At the moment $\bit0$ changes from 0 to~1, the fringe changes from
$0_0\,\p1_1\,\p3_0\,\p5_1\,\p6_1\,\p8_0$ to $\p0_1\,3_0\,6_1\,7_0\,8_0$;
and if we run the sequence backwards, it will go from
$0_1\,\p3_0\,\p6_1\,\p7_0\,\p8_0$ to $\p0_0\,1_1\,3_0\,5_1\,6_1\,8_0$
when $\bit0$ becomes~0.

\fi

\M{26}Now comes a key observation: Recall that some elements of $A_k$ and $B_k$
are called ``principal,'' namely the vertices adjacent to~$k$ (the
``children'' of $k$ in tree terminology). {\sl The nonprincipal vertices of
$A_k$ and $B_k$ appear in both transition strings $\tau_{k0}$ and $\tau_{k1}$,
with their entourages in the same states.} Indeed, the entourage
of each principal vertex $a$ of $A_k$ is $a_1$ followed by the
entourages of all $b\in B_a$; and all such $b$ are nonprincipal,
because $\alpha'_{ka}=1$ and $B_a\subseteq B_k$ when $a$ is principal.
Similarly, the entourage
of each principal vertex $b$ of $B_k$ is $b_0$ followed by the
entourages of all $a\in A_b$; and all such $a$ are nonprincipal,
because $\beta'_{kb}=0$ and $A_b\subseteq A_k$ when $b$ is principal.

Therefore, if we maintain the fringe as a doubly linked list---which we
definitely want to do---the task of replacing one of $k$'s transition strings
by the other is fairly easy to describe at link level: When $\bit k$ changes
from 0 to~1, we remove the existing substring $\tau_{k0}$ from the fringe
and replace it by the entourages of each $b\in B$, in increasing order
of the $b$'s. If $b$ is principal, the entourage of $b$ consists of
$b_0$ followed by the entourages of all $a\in A_b$, and the latter vertices
are nonprincipal; therefore the entourages of every such $a$ are already
properly linked within themselves, because they appear as substrings of
$\tau_{k0}$. Moreover, our assumption that the tree vertices were
labeled in preorder guarantees that the entourages of all $a\in A_b$
appear consecutively in\/ $\tau_{k0}$; thus they are properly linked
to each other, and all we need to do when forming the entourage of a
principal vertex $b\in B$ is link it to the leftmost element of $A_b$.
If on the other hand $b$ is nonprincipal, none of its internal links
need to be changed at all, because its entourage was already present
as a substring of~$\tau_{k0}$.

All these entourages appear in preorder, in both $\tau_{k0}$ and $\tau_{k1}$.
Therefore, to change $\tau_{k0}$ to $\tau_{k1}$ we simply need to
remove all of the principal members of $A_k$ and insert all of the
principal members of $B_k$, in their appropriate preorder position.
For example, we saw a moment ago that the task of
going from $\tau_{00}$ to $\tau_{01}$ consists of
removing $1_1$ and $5_1$, then inserting $7_0$.
The total number of link-pairs we need to change is at most
twice the number of elements of $A_k$ plus twice the number of
elements of $B_k$.

\fi

\M{27}The preorder assumption makes still further simplification possible.
Indeed, we know that the special case in {\mc KODA-RUSKEY} requires
at most two splices per transition, so we can expect to discover
a generalization of the principle that led to such an improvement.

Suppose all of the elements of $A_k\cup B_k$ have been merged into
preorder, thus giving us the ``union'' of $\tau_{k0}$ and $\tau_{k1}$.
Suppose further that two vertices $b\,b'$ of $B_k$ are adjacent in
this union, where $b$ is principal. It follows that $b'$ is also
principal; otherwise some element of $A_k$ would intervene.
When $\tau_{k0}$ is being replaced by $\tau_{k1}$ in the fringe,
we therefore need to insert both $b$ and~$b'$. Fortunately, they were
properly linked to each other when they last left the fringe,
and neither one has entered the fringe since then; so they still are
properly linked. (We also need to be sure that they were linked properly
to each other during the initialization, before processing starts.)
The same argument applies when two vertices $a\,a'$ of $A_k$ are
adjacent and $a$ is principal.

One can often, but not always,
avoid some of the link-updating with respect to $\tau_{k0}$ and
$\tau_{k1}$ by cleverly reordering the subtrees of~$k$.

\fi

\M{28}An algorithm that changes transition strings as just described will run
in constant amortized time per output, because the cost of changing
links in the fringe can be charged to the processing time when the
corresponding fringe element next becomes passive.

But it looks like bad news for our goal of loopless computation,
because we might need to do $\Omega(n)$ operations when $\bit k$ changes.

No, all is not lost: We can leave ``stale'' links in the fringe,
fixing them one at a time, just before the algorithm needs to look at them!

Namely, suppose the transition string $\tau_{k1}$ consists of substrings
$\sigma_1\sigma_2\ldots\sigma_s$ that are properly linked within themselves
but not necessarily to each other. We can prepare a list of ``fixup
instructions'' $(x_1,\ldots,x_s)$, where $x_j$ says that the first element
of $\sigma_j$ should be joined to the last element of $\sigma_{j-1}$, or
to element $k_1$ if $j=1$. And we can plant a flag that will cause $x_j$ to
be performed just as the first element of $\sigma_j$ becomes passive.
The execution of $x_j$ will then move the flag in preparation for $x_{j-1}$,
or it will remove the flag if $j=1$.

\fi

\N{1}{29}Data structures. Now that we know basically what needs to be done,
we are ready to design the records that will be linked together
to form the fringe. These records, which we will call fnodes, naturally
contain \PB{\\{left}} and \PB{\\{right}} fields, because the fringe is doubly
linked (modulo stale pointers). An fnode also contains a \PB{\\{fixup}} field,
which (if non-\PB{$\NULL$}) points to information that will correct a
stale \PB{\\{left}} pointer. And of course there's also a \PB{\\{bit}} field,
giving the status of $\bit k$ in fnode number~$k$.

Each fnode also has a \PB{\\{focus}} pointer, which implicitly classifies
fringe elements as active or passive, just as it did in the
{\mc KODA-RUSKEY} program: Usually \PB{$\|p\MG\\{focus}\K\|p$}, except when
\PB{\|p} is a passive node to the immediate left of an active node.
In the latter case, \PB{$\|p\MG\\{focus}$} points to the first active node
to the left of~\PB{\|p}.

Besides these dynamically changing fields, each fnode also contains
two static fields that are precomputed during the initialization,
namely \PB{\\{tau}[\T{0}]} and \PB{\\{tau}[\T{1}]}. These fields refer to
sequential
lists that specify the transition strings $\tau_{k0}$ and $\tau_{k1}$
for fnode number~$k$.

\Y\B\4\X6:Type definitions\X${}\mathrel+\E{}$\6
\&{typedef} \&{struct} \&{fnode\_struct} ${}\{{}$\1\6
\&{char} \\{bit};\C{ either 0 or 1; always 0 when an A-child's bit is~0,
                       always 1 when a B-child's bit is~1 }\6
\&{struct} \&{fnode\_struct} ${}{*}\\{left},{}$ ${}{*}\\{right}{}$;\C{
neighbors in the fringe }\6
\&{struct} \&{fnode\_struct} ${}{*}{*}\\{fixup}{}$;\C{ remedy for a stale \PB{%
\\{left}} link }\6
\&{struct} \&{fnode\_struct} ${}{*}\\{focus}{}$;\C{ red-tape cutter for
efficiency }\6
\&{struct} \&{fnode\_struct} ${}{*}{*}\\{tau}[\T{2}]{}$;\C{ list of transition
string link points }\2\6
${}\}{}$ \&{fnode};\par
\fi

\M{30}The fringe is extended to contain a special fnode called \PB{\\{head}},
which makes the list circular. In other words, \PB{$\\{head}\MG\\{right}$} and %
\PB{$\\{head}\MG\\{left}$}
are the leftmost and rightmost elements of the fringe. Also,
\PB{$\\{head}\MG\\{left}\MG\\{focus}$} is the rightmost active element.

\Y\B\4\D$\\{head}$ \5
$(\\{vertex}+\T{1}+\|n{}$)\par
\Y\B\4\X4:Global variables\X${}\mathrel+\E{}$\6
\&{fnode} ${}\\{vertex}[\\{maxn}+\T{1}]{}$;\C{ the collection of all potential
fringe nodes }\par
\fi

\M{31}And how should we store the \PB{\\{tau}} information? Let's ``compile''
the
set of necessary link changes into a sequential list $(r_1,l_1,
\ldots, r_t, l_t,\,$\PB{$\NULL$}) with the meaning that
we want to set \PB{$\hbox{$l_j$}\MG\\{right}\K\hbox{$r_j$}$}
and \PB{$\hbox{$r_j$}\MG\\{left}\K\hbox{$l_j$}$}.
The case $j=1$ is, however, an exception, because it refers to a
potentially unknown part of the fringe (lying to the right of
$\tau_{k0}$ and $\tau_{k1}$); in that case we want to set
we want to set \PB{$\hbox{$l_j$}\MG\\{right}\K\hbox{$r_j$}\MG\\{right}$}
and \PB{$\hbox{$r_j$}\MG\\{right}\MG\\{left}\K\hbox{$l_j$}$}.

\Y\B\4\X7:Initialize the data structures\X${}\mathrel+\E{}$\6
$\\{tau\_ptr}\K\\{tau\_table};{}$\6
\&{for} ${}(\|k\K\T{0};{}$ ${}\|k\Z\|n;{}$ ${}\|k\PP){}$\5
${}\{{}$\1\6
\X33:Merge the sets $A_k$ and $B_k$, retaining preorder\X;\6
\X34:Compile the link-change strings \PB{$\\{vertex}[\|k].\\{tau}[\,]$}\X;\6
\4${}\}{}$\2\par
\fi

\M{32}\B\X4:Global variables\X${}\mathrel+\E{}$\6
\&{fnode} ${}{*}\\{tau\_table}[((\\{maxn}*\\{maxn})\GG\T{1})+\\{maxn}+%
\\{maxn}]{}$;\C{ elements of \PB{\\{tau}} fields }\6
\&{fnode} ${}{*}{*}\\{tau\_ptr}{}$;\C{ the first unused place in \PB{\\{tau%
\_table}} }\6
\&{char} \\{ct}[\\{maxn}];\C{ \PB{$\\{ct}[\|l]\K\T{0}$} if $\\{ab}[l]\in A_k$,
otherwise \PB{$\\{ct}[\|l]\K\T{1}$} }\6
\&{char} \\{verbose}${}\K\T{1}{}$;\C{ should details of tau compilation be
printed? }\6
\&{info} ${}{*}\\{ab}[\\{maxn}]{}$;\C{ an element of $A_k$ or $B_k$ }\par
\fi

\M{33}It is most convenient to sort in {\it decreasing\/} order here,
so that the rightmost link-updates are listed first, because we have
stored them in that order and we will need them in that order.

This part of the program is the most subtle, so I've included a
provision for verbose printing during debugging sessions.
A simple case in which $A_k=\{5\}$ and $B_k=\emptyset$ will be
printed as \.{(a5)}. A slightly more complicated case in which
$A_k=\{2\}$ and $B_k=\{4\}$ would be \.{(a2)(b4)} if both are
principal, or \.{(a2b4)} if only 2~is principal. A more complicated
case, used as an example below, has $A_k=\{1,2,5,8,11\}$ and
$B_k=\{3,4,6,7,9,10\}$, with elements $\{1,2,5,7,9,10\}$ principal;
this would be printed as \.{(a1)(a2b3b4)(a5b6)(b7a8)(b9)(b10a11)}.

Recall that an element \PB{\|j} of $A_k\cup B_k$ is principal if and only if
it is a child of~$k$. The parentheses in the verbose printout
are used to group $k$'s children with their nonprincipal descendants.

\Y\B\4\D$\\{principal}(\|l)$ \5
$(\\{jj}[\\{ab}[\|l]\MG\\{id}]\E\|k{}$)\par
\Y\B\4\X33:Merge the sets $A_k$ and $B_k$, retaining preorder\X${}\E{}$\6
\&{for} ${}(\|i\K\\{list}[\|k][\T{0}],\39\\{ii}\K\\{list}[\|k][\T{1}],\39\|l\K%
\T{0};{}$ ${}\|i\V\\{ii};{}$ ${}\|l\PP){}$\5
${}\{{}$\1\6
\&{if} ${}(\R\\{ii}\V(\|i\W\|i\MG\\{id}>\\{ii}\MG\\{id})){}$\1\5
${}\\{ab}[\|l]\K\|i,\39\\{ct}[\|l]\K\T{0},\39\|i\K\|i\MG\\{sib};{}$\2\6
\&{else}\1\5
${}\\{ab}[\|l]\K\\{ii},\39\\{ct}[\|l]\K\T{1},\39\\{ii}\K\\{ii}\MG\\{sib};{}$\2\6
\4${}\}{}$\2\6
\&{if} ${}(\|l\W\\{verbose}){}$\5
${}\{{}$\1\6
${}\\{printf}(\.{"Union\%d:\ ("},\39\|k);{}$\6
\&{for} ${}(\|j\K\|l-\T{1};{}$  ; \,)\5
${}\{{}$\1\6
${}\\{printf}(\.{"\%c\%d"},\39\.{'a'}+\\{ct}[\|j],\39\\{ab}[\|j]\MG\\{id});{}$\6
\&{if} ${}(\|j\E\T{0}){}$\1\5
\&{break};\2\6
${}\|j\MM;{}$\6
\&{if} (\\{principal}(\|j))\1\5
\\{printf}(\.{")("});\2\6
\4${}\}{}$\2\6
\\{printf}(\.{")\\n"});\6
\4${}\}{}$\2\par
\U31.\fi

\M{34}\B\X34:Compile the link-change strings \PB{$\\{vertex}[\|k].\\{tau}[\,]$}%
\X${}\E{}$\6
\&{if} ${}(\R\|l){}$\1\5
${}\\{vertex}[\|k].\\{tau}[\T{0}]\K\\{vertex}[\|k].\\{tau}[\T{1}]\K\NULL;{}$\2\6
\&{else}\5
${}\{{}$\1\6
${}\\{ab}[\|l]\K\NULL{}$;\C{ sentinel at end of list }\6
${}\\{vertex}[\|k].\\{tau}[\T{1}]\K\\{tau\_ptr};{}$\6
\X35:Compile \PB{$\\{vertex}[\|k].\\{tau}[\T{1}]$}\X;\6
${}\\{vertex}[\|k].\\{tau}[\T{0}]\K\\{tau\_ptr};{}$\6
\X40:Compile \PB{$\\{vertex}[\|k].\\{tau}[\T{0}]$}\X;\6
\X44:Link siblings together\X;\6
\4${}\}{}$\2\par
\U31.\fi

\M{35}Our job here is to specify the links that must change when the
principal elements of~$A_k$ are deleted and the principal elements of~$B_k$
are inserted. In the simple example \.{(a5)} mentioned above,
we simply set $r_1=\;$\PB{${\AND}\\{vertex}[\T{5}]$} and $l_1=\;$\PB{%
\\{vertex}[\|k]}. The
verbose printout will say `\.{k:5+}', meaning that \PB{\\{vertex}[\|k]} is to
be followed in the fringe by the vertex that currently follows \PB{\\{vertex}[%
\T{5}]}.

In the next simplest example, \.{(a2)(b4)}, we will set
$r_1=\;$\PB{${\AND}\\{vertex}[\T{2}]$},
$l_1=\;$\PB{${\AND}\\{vertex}[\T{4}]$},
$r_2=\;$\PB{${\AND}\\{vertex}[\T{4}]$},
$l_2=\;$\PB{${\AND}\\{vertex}[\|k]$};
the verbose printout will symbolize this by `\.{4:2+ k:4}'.
The result of \.{(a2b4)} would, on the other hand, be `\.{k:2+}',
because only \PB{\\{vertex}[\T{2}]} needs to be removed from the fringe when
\PB{\\{vertex}[\T{4}]} is nonprincipal.

Finally, the instructions compiled from the complex example
\.{(a1)(a2b3b4)(a5b6)(b7a8)(b9)(b10a11)} would be
$$\.{10:8r+}\quad\.{8r:9}\quad\.{7:8}\quad\.{6r:7}\quad
\.{4r:6}\quad\.{k:3}.$$
The `\.r' here denotes the rightmost element of a nonprincipal
vertex's entourage.
(As explained earlier, the instruction `\.{9:10}' need
not be compiled.)

The reader might be able to understand from these examples why the
author took time out to think before writing this part of the code.

\Y\B\4\X35:Compile \PB{$\\{vertex}[\|k].\\{tau}[\T{1}]$}\X${}\E{}$\6
\X36:Find $r_1$ for \PB{\\{tau}[\T{1}]}\X;\6
\X38:Find $l_1$ for \PB{\\{tau}[\T{1}]}\X;\6
\&{while} (\\{ab}[\|l])\1\5
\X39:Find the next $r_j$ and $l_j$ for \PB{\\{tau}[\T{1}]}\X;\2\6
${}\\{tau\_ptr}\PP{}$;\C{ the \PB{\\{tau}} list terminates with \PB{$\NULL$} }%
\par
\U34.\fi

\M{36}\B\X36:Find $r_1$ for \PB{\\{tau}[\T{1}]}\X${}\E{}$\6
\&{for} ${}(\|l\K\T{0};{}$ ${}\R\\{principal}(\|l);{}$ ${}\|l\PP){}$\1\5
;\C{ nonprincipals at the end can be ignored }\2\6
\&{if} (\\{ct}[\|l])\5
${}\{{}$\C{ principal $b$ }\1\6
\&{for} ${}(\|j\K\|l+\T{1};{}$ ${}\\{ab}[\|j]\W\\{ct}[\|j];{}$ ${}\|j\PP){}$\1\5
;\2\6
\&{if} (\\{ab}[\|j])\5
${}\{{}$\1\6
\&{if} (\\{verbose})\1\5
${}\\{sprintf}(\\{rbuf},\39\.{"\%dr+"},\39\\{ab}[\|j]\MG\\{id});{}$\2\6
${}{*}\\{tau\_ptr}\PP\K\\{vertex}+\\{fj}[\\{ab}[\|j]\MG\\{id}][\\{ab}[\|j]\MG%
\\{alfprime}];{}$\6
\4${}\}{}$\5
\2\&{else}\5
${}\{{}$\1\6
\&{if} (\\{verbose})\1\5
${}\\{sprintf}(\\{rbuf},\39\.{"\%d+"},\39\|k);{}$\2\6
${}{*}\\{tau\_ptr}\PP\K\\{vertex}+\|k;{}$\6
\4${}\}{}$\2\6
\4${}\}{}$\5
\2\&{else}\5
${}\{{}$\C{ principal $a$ }\1\6
\&{if} (\\{verbose})\1\5
${}\\{sprintf}(\\{rbuf},\39\.{"\%d+"},\39\\{ab}[\|l]\MG\\{id});{}$\2\6
${}{*}\\{tau\_ptr}\PP\K\\{vertex}+\\{ab}[\|l]\MG\\{id};{}$\6
\4${}\}{}$\2\par
\U35.\fi

\M{37}\B\X4:Global variables\X${}\mathrel+\E{}$\6
\&{char} \\{rbuf}[\T{8}];\C{ symbolic form of $r_1$ for verbose printing }\par
\fi

\M{38}Once we've found $r_1$, principal $a$'s at the end can be ignored.

\Y\B\4\X38:Find $l_1$ for \PB{\\{tau}[\T{1}]}\X${}\E{}$\6
\&{while} ${}(\\{ab}[\|l]\W\R\\{ct}[\|l]\W\\{principal}(\|l)){}$\1\5
${}\|l\PP;{}$\2\6
\&{if} ${}(\R\\{ab}[\|l]){}$\5
${}\{{}$\1\6
\&{if} (\\{verbose})\1\5
${}\\{printf}(\.{"\ \%d:\%s\\n"},\39\|k,\39\\{rbuf});{}$\2\6
${}{*}\\{tau\_ptr}\PP\K\\{vertex}+\|k;{}$\6
\4${}\}{}$\5
\2\&{else} \&{if} (\\{principal}(\|l))\5
${}\{{}$\1\6
\&{if} (\\{verbose})\1\5
${}\\{printf}(\.{"\ \%d:\%s"},\39\\{ab}[\|l]\MG\\{id},\39\\{rbuf});{}$\2\6
${}{*}\\{tau\_ptr}\PP\K\\{vertex}+\\{ab}[\|l]\MG\\{id};{}$\6
\4${}\}{}$\5
\2\&{else}\5
${}\{{}$\1\6
\&{if} (\\{verbose})\1\5
${}\\{printf}(\.{"\ \%dr:\%s"},\39\\{ab}[\|l]\MG\\{id},\39\\{rbuf});{}$\2\6
${}{*}\\{tau\_ptr}\PP\K\\{vertex}+\\{fj}[\\{ab}[\|l]\MG\\{id}][\\{ab}[\|l]\MG%
\\{alfprime}];{}$\6
\4${}\}{}$\2\par
\U35.\fi

\M{39}At this point \PB{\\{ab}[\|l]} is the \PB{\&{info}} node from which we
produced $l_{j-1}$.
If it refers to a principal element, it's an element $b$ that is absent from
$\tau_{k0}$ but present in $\tau_{k1}$. Otherwise it's a nonprincipal
element, which appears in both $\tau_{k0}$ and $\tau_{k1}$.

\Y\B\4\X39:Find the next $r_j$ and $l_j$ for \PB{\\{tau}[\T{1}]}\X${}\E{}$\6
${}\{{}$\1\6
\&{if} (\\{principal}(\|l))\1\6
\&{for} ${}(\|l\PP;{}$ ${}\\{ab}[\|l]\W\\{ct}[\|l]\W\\{principal}(\|l);{}$ ${}%
\|l\PP){}$\1\5
;\2\2\6
\&{else}\5
\1\&{for} ${}(\|l\PP;{}$ ${}\R\\{principal}(\|l);{}$ ${}\|l\PP){}$\1\5
;\2\2\6
${}{*}\\{tau\_ptr}\PP\K\\{vertex}+\\{ab}[\|l-\T{1}]\MG\\{id};{}$\6
\&{if} ${}(\\{ab}[\|l]\W\R\\{ct}[\|l]\W\\{principal}(\|l)){}$\1\6
\&{for} ${}(\|l\PP;{}$ ${}\\{ab}[\|l]\W\R\\{ct}[\|l]\W\\{principal}(\|l);{}$
${}\|l\PP){}$\1\5
;\2\2\6
\&{if} ${}(\R\\{ab}[\|l]){}$\5
${}\{{}$\1\6
\&{if} (\\{verbose})\1\5
${}\\{printf}(\.{"\ \%d:\%d\\n"},\39\|k,\39{*}(\\{tau\_ptr}-\T{1})-%
\\{vertex});{}$\2\6
${}{*}\\{tau\_ptr}\PP\K\\{vertex}+\|k;{}$\6
\4${}\}{}$\5
\2\&{else} \&{if} (\\{principal}(\|l))\5
${}\{{}$\1\6
\&{if} (\\{verbose})\1\5
${}\\{printf}(\.{"\ \%d:\%d"},\39\\{ab}[\|l]\MG\\{id},\39{*}(\\{tau\_ptr}-%
\T{1})-\\{vertex});{}$\2\6
${}{*}\\{tau\_ptr}\PP\K\\{vertex}+\\{ab}[\|l]\MG\\{id};{}$\6
\4${}\}{}$\5
\2\&{else}\5
${}\{{}$\1\6
\&{if} (\\{verbose})\1\5
${}\\{printf}(\.{"\ \%dr:\%d"},\39\\{ab}[\|l]\MG\\{id},\39{*}(\\{tau\_ptr}-%
\T{1})-\\{vertex});{}$\2\6
${}{*}\\{tau\_ptr}\PP\K\\{vertex}+\\{fj}[\\{ab}[\|l]\MG\\{id}][\\{ab}[\|l]\MG%
\\{alfprime}];{}$\6
\4${}\}{}$\2\6
\4${}\}{}$\2\par
\U35.\fi

\M{40}The next few sections are identical to the previous ones, with
$a$'s and $b$'s swapped. (Unless I have erred.) The compiled instructions
for \PB{\\{tau}[\T{0}]} in the complex example are
$$\.{8r:10+}\quad \.{6r:8}\quad \.{5:6}\quad \.{4r:5}\quad \.{2:3}\quad \.{k:1}
\,.$$

\Y\B\4\X40:Compile \PB{$\\{vertex}[\|k].\\{tau}[\T{0}]$}\X${}\E{}$\6
\X41:Find $r_1$ for \PB{\\{tau}[\T{0}]}\X;\6
\X42:Find $l_1$ for \PB{\\{tau}[\T{0}]}\X;\6
\&{while} (\\{ab}[\|l])\1\5
\X43:Find the next $r_j$ and $l_j$ for \PB{\\{tau}[\T{0}]}\X;\2\6
${}\\{tau\_ptr}\PP{}$;\C{ the \PB{\\{tau}} list terminates with \PB{$\NULL$} }%
\par
\U34.\fi

\M{41}\B\X41:Find $r_1$ for \PB{\\{tau}[\T{0}]}\X${}\E{}$\6
\&{for} ${}(\|l\K\T{0};{}$ ${}\R\\{principal}(\|l);{}$ ${}\|l\PP){}$\1\5
;\C{ nonprincipals at the end can be ignored }\2\6
\&{if} ${}(\R\\{ct}[\|l]){}$\5
${}\{{}$\C{ principal $a$ }\1\6
\&{for} ${}(\|j\K\|l+\T{1};{}$ ${}\\{ab}[\|j]\W\R\\{ct}[\|j];{}$ ${}\|j\PP){}$%
\1\5
;\2\6
\&{if} (\\{ab}[\|j])\5
${}\{{}$\1\6
\&{if} (\\{verbose})\1\5
${}\\{sprintf}(\\{rbuf},\39\.{"\%dr+"},\39\\{ab}[\|j]\MG\\{id});{}$\2\6
${}{*}\\{tau\_ptr}\PP\K\\{vertex}+\\{fj}[\\{ab}[\|j]\MG\\{id}][\\{ab}[\|j]\MG%
\\{alfprime}];{}$\6
\4${}\}{}$\5
\2\&{else}\5
${}\{{}$\1\6
\&{if} (\\{verbose})\1\5
${}\\{sprintf}(\\{rbuf},\39\.{"\%d+"},\39\|k);{}$\2\6
${}{*}\\{tau\_ptr}\PP\K\\{vertex}+\|k;{}$\6
\4${}\}{}$\2\6
\4${}\}{}$\5
\2\&{else}\5
${}\{{}$\C{ principal $b$ }\1\6
\&{if} (\\{verbose})\1\5
${}\\{sprintf}(\\{rbuf},\39\.{"\%d+"},\39\\{ab}[\|l]\MG\\{id});{}$\2\6
${}{*}\\{tau\_ptr}\PP\K\\{vertex}+\\{ab}[\|l]\MG\\{id};{}$\6
\4${}\}{}$\2\par
\U40.\fi

\M{42}\B\X42:Find $l_1$ for \PB{\\{tau}[\T{0}]}\X${}\E{}$\6
\&{while} ${}(\\{ab}[\|l]\W\\{ct}[\|l]\W\\{principal}(\|l)){}$\1\5
${}\|l\PP;{}$\2\6
\&{if} ${}(\R\\{ab}[\|l]){}$\5
${}\{{}$\1\6
\&{if} (\\{verbose})\1\5
${}\\{printf}(\.{"\ \%d:\%s\\n"},\39\|k,\39\\{rbuf});{}$\2\6
${}{*}\\{tau\_ptr}\PP\K\\{vertex}+\|k;{}$\6
\4${}\}{}$\5
\2\&{else} \&{if} (\\{principal}(\|l))\5
${}\{{}$\1\6
\&{if} (\\{verbose})\1\5
${}\\{printf}(\.{"\ \%d:\%s"},\39\\{ab}[\|l]\MG\\{id},\39\\{rbuf});{}$\2\6
${}{*}\\{tau\_ptr}\PP\K\\{vertex}+\\{ab}[\|l]\MG\\{id};{}$\6
\4${}\}{}$\5
\2\&{else}\5
${}\{{}$\1\6
\&{if} (\\{verbose})\1\5
${}\\{printf}(\.{"\ \%dr:\%s"},\39\\{ab}[\|l]\MG\\{id},\39\\{rbuf});{}$\2\6
${}{*}\\{tau\_ptr}\PP\K\\{vertex}+\\{fj}[\\{ab}[\|l]\MG\\{id}][\\{ab}[\|l]\MG%
\\{alfprime}];{}$\6
\4${}\}{}$\2\par
\U40.\fi

\M{43}At this point \PB{\\{ab}[\|l]} is the \PB{\&{info}} node from which we
produced $l_{j-1}$.
If it refers to a principal element, it's an element $a$ that is absent from
$\tau_{k1}$ but present in $\tau_{k0}$. Otherwise it's a nonprincipal
element, which appears in both $\tau_{k0}$ and $\tau_{k1}$.

\Y\B\4\X43:Find the next $r_j$ and $l_j$ for \PB{\\{tau}[\T{0}]}\X${}\E{}$\6
${}\{{}$\1\6
\&{if} (\\{principal}(\|l))\1\6
\&{for} ${}(\|l\PP;{}$ ${}\\{ab}[\|l]\W\R\\{ct}[\|l]\W\\{principal}(\|l);{}$
${}\|l\PP){}$\1\5
;\2\2\6
\&{else}\5
\1\&{for} ${}(\|l\PP;{}$ ${}\R\\{principal}(\|l);{}$ ${}\|l\PP){}$\1\5
;\2\2\6
${}{*}\\{tau\_ptr}\PP\K\\{vertex}+\\{ab}[\|l-\T{1}]\MG\\{id};{}$\6
\&{if} ${}(\\{ab}[\|l]\W\\{ct}[\|l]\W\\{principal}(\|l)){}$\1\6
\&{for} ${}(\|l\PP;{}$ ${}\\{ab}[\|l]\W\\{ct}[\|l]\W\\{principal}(\|l);{}$ ${}%
\|l\PP){}$\1\5
;\2\2\6
\&{if} ${}(\R\\{ab}[\|l]){}$\5
${}\{{}$\1\6
\&{if} (\\{verbose})\1\5
${}\\{printf}(\.{"\ \%d:\%d\\n"},\39\|k,\39{*}(\\{tau\_ptr}-\T{1})-%
\\{vertex});{}$\2\6
${}{*}\\{tau\_ptr}\PP\K\\{vertex}+\|k;{}$\6
\4${}\}{}$\5
\2\&{else} \&{if} (\\{principal}(\|l))\5
${}\{{}$\1\6
\&{if} (\\{verbose})\1\5
${}\\{printf}(\.{"\ \%d:\%d"},\39\\{ab}[\|l]\MG\\{id},\39{*}(\\{tau\_ptr}-%
\T{1})-\\{vertex});{}$\2\6
${}{*}\\{tau\_ptr}\PP\K\\{vertex}+\\{ab}[\|l]\MG\\{id};{}$\6
\4${}\}{}$\5
\2\&{else}\5
${}\{{}$\1\6
\&{if} (\\{verbose})\1\5
${}\\{printf}(\.{"\ \%dr:\%d"},\39\\{ab}[\|l]\MG\\{id},\39{*}(\\{tau\_ptr}-%
\T{1})-\\{vertex});{}$\2\6
${}{*}\\{tau\_ptr}\PP\K\\{vertex}+\\{fj}[\\{ab}[\|l]\MG\\{id}][\\{ab}[\|l]\MG%
\\{alfprime}];{}$\6
\4${}\}{}$\2\6
\4${}\}{}$\2\par
\U40.\fi

\M{44}We need to link the siblings of a family together if they are adjacent
and
of the same type, because of one of the optimization we're doing.
So we might as well link {\it all\/} siblings together.

\Y\B\4\X44:Link siblings together\X${}\E{}$\6
\&{for} ${}(\|l\K\T{0},\39\|p\K\NULL;{}$ \\{ab}[\|l]; ${}\|l\PP){}$\1\6
\&{if} (\\{principal}(\|l))\5
${}\{{}$\1\6
${}\|q\K\\{vertex}+\\{ab}[\|l]\MG\\{id};{}$\6
\&{if} (\|p)\1\5
${}\|q\MG\\{right}\K\|p,\39\|p\MG\\{left}\K\|q;{}$\2\6
${}\|p\K\|q;{}$\6
\4${}\}{}$\2\2\par
\U34.\fi

\M{45}\B\X5:Local variables\X${}\mathrel+\E{}$\6
\&{fnode} ${}{*}\|p,{}$ ${}{*}\|q,{}$ ${}{*}\|r{}$;\par
\fi

\N{1}{46}Doing it. The time has come to construct the loopless implementation
in practice, as we have been doing so far in theory.

\Y\B\4\X46:Generate the answers\X${}\E{}$\6
\&{while} (\T{1})\5
${}\{{}$\1\6
\X53:Print out all the current bits\X;\6
\X48:Set \PB{\|p} to the rightmost active node of the fringe, and activate
everything to its right\X;\6
\&{if} ${}(\|p\I\\{head}){}$\5
${}\{{}$\1\6
\&{if} ${}(\|p\MG\\{fixup}){}$\1\5
\X50:Repair the stale pointer \PB{$\|p\MG\\{left}$}\X;\2\6
\&{if} ${}(\|p\MG\\{bit}\E\T{0}){}$\1\5
\X51:Move forward, setting \PB{$\|p\MG\\{bit}\K\T{1}$}\X\2\6
\&{else}\1\5
\X52:Move backward, setting \PB{$\|p\MG\\{bit}\K\T{0}$}\X\2\6
\4${}\}{}$\5
\2\&{else} \&{if} (\\{been\_there\_and\_done\_that})\1\5
\&{break};\2\6
\&{else}\5
${}\{{}$\1\6
\\{printf}(\.{"...\ and\ now\ we\ gene}\)\.{rate\ in\ reverse:\\n"});\6
${}\\{been\_there\_and\_done\_that}\K\T{1}{}$;\5
\&{continue};\6
\4${}\}{}$\2\6
\X49:Make node \PB{\|p} passive\X;\6
\4${}\}{}$\2\par
\U1.\fi

\M{47}\B\X4:Global variables\X${}\mathrel+\E{}$\6
\&{char} \\{been\_there\_and\_done\_that};\C{ have we completed a cycle
already? }\par
\fi

\M{48}The nice thing is that the algorithm not only is loopless, its
operations are simple.

\Y\B\4\X48:Set \PB{\|p} to the rightmost active node of the fringe, and
activate everything to its right\X${}\E{}$\6
$\|q\K\\{head}\MG\\{left};{}$\6
${}\|p\K\|q\MG\\{focus};{}$\6
${}\|q\MG\\{focus}\K\|q{}$;\par
\U46.\fi

\M{49}At this point we know that \PB{$\|p\MG\\{right}$} (more precisely, the
node that would be \PB{$\|p\MG\\{right}$} if links weren't stale) is active.
And we also know that \PB{$\|p\MG\\{left}$} is not stale.

\Y\B\4\X49:Make node \PB{\|p} passive\X${}\E{}$\6
$\|q\K\|p\MG\\{left};{}$\6
${}\|p\MG\\{focus}\K\|q\MG\\{focus};{}$\6
${}\|q\MG\\{focus}\K\|q{}$;\par
\U46.\fi

\M{50}\B\X50:Repair the stale pointer \PB{$\|p\MG\\{left}$}\X${}\E{}$\6
${}\{{}$\1\6
${}\|q\K{*}(\|p\MG\\{fixup}),\39\|r\K{*}(\|p\MG\\{fixup}+\T{1});{}$\6
${}\|p\MG\\{left}\K\|q,\39\|q\MG\\{right}\K\|p;{}$\6
\&{if} (\|r)\1\5
${}\|r\MG\\{fixup}\K\|p\MG\\{fixup}+\T{2};{}$\2\6
${}\|p\MG\\{fixup}\K\NULL;{}$\6
\4${}\}{}$\2\par
\U46.\fi

\M{51}\B\X51:Move forward, setting \PB{$\|p\MG\\{bit}\K\T{1}$}\X${}\E{}$\6
${}\{{}$\1\6
${}\|p\MG\\{bit}\K\T{1};{}$\6
\&{if} ${}(\|p\MG\\{tau}[\T{1}]){}$\5
${}\{{}$\1\6
${}\|q\K({*}(\|p\MG\\{tau}[\T{1}]))\MG\\{right};{}$\6
${}\|r\K{*}(\|p\MG\\{tau}[\T{1}]+\T{1});{}$\6
${}\|q\MG\\{left}\K\|r,\39\|r\MG\\{right}\K\|q;{}$\6
${}\|r\K{*}(\|p\MG\\{tau}[\T{1}]+\T{2});{}$\6
\&{if} (\|r)\1\5
${}\|r\MG\\{fixup}\K\|p\MG\\{tau}[\T{1}]+\T{3};{}$\2\6
\4${}\}{}$\2\6
\4${}\}{}$\2\par
\U46.\fi

\M{52}\B\X52:Move backward, setting \PB{$\|p\MG\\{bit}\K\T{0}$}\X${}\E{}$\6
${}\{{}$\1\6
${}\|p\MG\\{bit}\K\T{0};{}$\6
\&{if} ${}(\|p\MG\\{tau}[\T{0}]){}$\5
${}\{{}$\1\6
${}\|q\K({*}(\|p\MG\\{tau}[\T{0}]))\MG\\{right};{}$\6
${}\|r\K{*}(\|p\MG\\{tau}[\T{0}]+\T{1});{}$\6
${}\|q\MG\\{left}\K\|r,\39\|r\MG\\{right}\K\|q;{}$\6
${}\|r\K{*}(\|p\MG\\{tau}[\T{0}]+\T{2});{}$\6
\&{if} (\|r)\1\5
${}\|r\MG\\{fixup}\K\|p\MG\\{tau}[\T{0}]+\T{3};{}$\2\6
\4${}\}{}$\2\6
\4${}\}{}$\2\par
\U46.\fi

\M{53}\B\X53:Print out all the current bits\X${}\E{}$\6
\&{for} ${}(\|k\K\T{0};{}$ ${}\|k\Z\|n;{}$ ${}\|k\PP){}$\1\5
${}\\{putchar}(\.{'0'}+\\{vertex}[\|k].\\{bit});{}$\2\6
\&{if} (\\{verbose})\1\5
\X54:Print the fringe in symbolic form\X;\2\6
\\{putchar}(\.{'\\n'});\par
\U46.\fi

\M{54}Once again, I'm writing optional code that should help me gain confidence
(and pinpoint errors) while debugging. Such code also ought to help an observer
understand what the program thinks it is doing, or at least what I thought
it should be doing when I wrote it.

The fringe is printed right-to-left, because that's the way the implicit
parts of the data need to be interpreted (namely, the active/passive bits
and the stale-pointer corrections). I could, of course, take the trouble to
reverse the order and make the printout more intuitive; but hey, this is for
educated eyes only.

Passive vertices are shown in parentheses.

\Y\B\4\X54:Print the fringe in symbolic form\X${}\E{}$\6
\&{for} ${}(\|l\K\T{0},\39\|q\K\\{head}\MG\\{left},\39\|p\K\|q\MG\\{focus};{}$
${}\|q\I\\{head};{}$ ${}\|q\K\|r){}$\5
${}\{{}$\1\6
${}\\{printf}(\.{"\ \%s\%d\_\%d\%s"},\39\|q\I\|p\?\.{"("}:\.{""},\39\|q-%
\\{vertex},\39\|q\MG\\{bit},\39\|q\I\|p\?\.{")"}:\.{""});{}$\6
\X56:Set \PB{\|r} to the non-stale form of \PB{$\|q\MG\\{left}$}\X;\6
\&{if} ${}(\|q\E\|p){}$\1\5
${}\|p\K\|r\MG\\{focus};{}$\2\6
\4${}\}{}$\2\par
\U53.\fi

\M{55}Here we need to maintain a stack of currently active fixups.

\Y\B\4\X6:Type definitions\X${}\mathrel+\E{}$\6
\&{typedef} \&{struct} ${}\{{}$\1\6
\&{fnode} ${}{*}\\{vert}{}$;\C{ vertex with a stale \PB{\\{left}} pointer }\6
\&{fnode} ${}{*}{*}\\{ref}{}$;\C{ \PB{\\{tau}} string with info about future
vertices }\2\6
${}\}{}$ \&{fstack\_node};\par
\fi

\M{56}\B\X56:Set \PB{\|r} to the non-stale form of \PB{$\|q\MG\\{left}$}\X${}%
\E{}$\6
\&{if} ${}(\|q\MG\\{fixup}){}$\1\5
${}\\{fstack}[\PP\|l].\\{vert}\K\|q,\39\\{fstack}[\|l].\\{ref}\K\|q\MG%
\\{fixup};{}$\2\6
\&{if} ${}(\|q\E\\{fstack}[\|l].\\{vert}){}$\5
${}\{{}$\1\6
\\{putchar}(\.{'!'});\C{ indicate that a stale link is being fixed }\6
${}\|r\K{*}(\\{fstack}[\|l].\\{ref});{}$\6
\&{if} ${}({*}(\\{fstack}[\|l].\\{ref}+\T{1})){}$\1\5
${}\\{fstack}[\|l].\\{vert}\K{*}(\\{fstack}[\|l].\\{ref}+\T{1}),\39\\{fstack}[%
\|l].\\{ref}\MRL{+{\K}}\T{2};{}$\2\6
\&{else}\1\5
${}\|l\MM;{}$\2\6
\4${}\}{}$\5
\2\&{else}\1\5
${}\|r\K\|q\MG\\{left}{}$;\2\par
\U54.\fi

\M{57}\B\X4:Global variables\X${}\mathrel+\E{}$\6
\&{fstack\_node} \\{fstack}[\\{maxn}];\C{ stack used in verbose printout }\par
\fi

\N{1}{58}Priming the pump. The program is complete except for one niggling
detail: We need to set up the initial contents of the fringe.
Everything will maintain itself, once the whole structure is up and running,
but how do we get chickens before we have eggs?

Here are two recursive subroutines that come to the rescue. The
first one sets \PB{$\\{vertex}[\|j].\\{bit}\K\T{1}$} in all vertices \PB{\|j}
such that $k\preceq j$.
The second one contributes the entourage of given vertex in a given state to
the current fringe.

\Y\B\4\X15:Subroutines\X${}\mathrel+\E{}$\6
\&{void} \\{setbits}(\&{char} \|k)\1\1\2\2\6
${}\{{}$\1\6
\&{register} \&{info} ${}{*}\|i;{}$\7
${}\\{vertex}[\|k].\\{bit}\K\T{1};{}$\6
\&{for} ${}(\|i\K\\{list}[\|k][\T{0}];{}$ \|i; ${}\|i\K\|i\MG\\{sib}){}$\1\6
\&{if} ${}(\\{jj}[\|i\MG\\{id}]\E\|k){}$\1\5
${}\\{setbits}(\|i\MG\\{id});{}$\2\2\6
\4${}\}{}$\2\par
\fi

\M{59}A global variable called \PB{\\{cur\_vert}} points to the left end of the
fringe-so-far. We construct the entourages from right to left, because
that's the way the \PB{\\{list}} info is set up.

All bits are assumed to be zero initially; therefore we don't need a
routine to set them zero, only the \PB{\\{setbits}} routine to make
some of them nonzero.

\Y\B\4\X15:Subroutines\X${}\mathrel+\E{}$\6
\&{void} \\{entourage}(\&{char} \|k${},\39{}$\&{char} \|t)\1\1\2\2\6
${}\{{}$\1\6
\&{register} \&{fnode} ${}{*}\|p\K\\{vertex}+\|k;{}$\6
\&{register} \&{info} ${}{*}\|i;{}$\7
\&{if} (\|t)\1\5
\\{setbits}(\|k);\2\6
\&{for} ${}(\|i\K\\{list}[\|k][\|t];{}$ \|i; ${}\|i\K\|i\MG\\{sib}){}$\1\5
${}\\{entourage}(\|i\MG\\{id},\39\|i\MG\\{alfprime}\XOR\|i\MG\\{del});{}$\2\6
${}\\{cur\_vert}\MG\\{left}\K\|p,\39\|p\MG\\{right}\K\\{cur\_vert},\39\\{cur%
\_vert}\K\|p;{}$\6
\4${}\}{}$\2\par
\fi

\M{60}\B\X4:Global variables\X${}\mathrel+\E{}$\6
\&{fnode} ${}{*}\\{cur\_vert}{}$;\C{ front of a doubly linked list under
construction }\par
\fi

\M{61}And now we're done!

\Y\B\4\X7:Initialize the data structures\X${}\mathrel+\E{}$\6
\&{for} ${}(\|k\K\T{0};{}$ ${}\|k\Z\|n+\T{1};{}$ ${}\|k\PP){}$\1\5
${}\\{vertex}[\|k].\\{focus}\K{\AND}\\{vertex}[\|k];{}$\2\6
${}\\{cur\_vert}\K\\{head};{}$\6
${}\\{entourage}(\T{0},\39\T{0});{}$\6
${}\\{cur\_vert}\MG\\{left}\K\\{head},\39\\{head}\MG\\{right}\K\\{cur%
\_vert}{}$;\par
\fi

\N{1}{62}Comment about looplessness. In practice, the algorithm
would run a bit faster if the \PB{\\{fixup}} links were dispensed with
and all splicing were done directly at transition time. Constant
{\it amortized\/} time---or, rather, total execution time in general---is the
main criterion in most applications; there usually is no
reason to care whether some generation steps are slow and others are
fast, as long as the entire job is done quickly.

However, the construction of loopless algorithms carries an academic cachet:
``Look at how clever I am, I can even do it looplessly!''
That's why this program has gone the
extra mile to assure constant time per generation step, even though the
added complications
may well have been a step backwards from a practical standpoint.
[Excuse me, I'm thinking only of sequential computation here; loopless
algorithms can have definite advantages with respect to parallel processing.]

On the other hand, many academic purists may reasonably claim that I have not
actually reached my stated goal.
When Gideon Ehrlich introduced the notion of loopless implementations
in {\sl JACM\/ \bf20} (1973), 500--513, he insisted that the
initialization time be $O(n)$; the program above can only
guarantee an initialization time that is $O\bigl(\sum_k(\Vert A_k\Vert+
\Vert B_k\Vert)\bigr)$, and that sum can be $\sim n^2\!/4$.
Clearly the initialization time
ought to be substantially smaller than the number of outputs,
or a loopless algorithm would be trivial.
But I think it's fair to allow initialization to take as long
as, say, $O\bigl(\min(m,n^2)\bigr)$ when there are $m$ outputs,
especially when $m$ is usually (but not always) exponential in~$n$.

\fi

\N{1}{63}Index.
\fi

\inx
\fin
\con
