\input cwebmac
\datethis

\N{1}{1}Introduction.
I'm writing this program to experiment with recursive algorithms on trees
that I think are educational and fun. It's an interactive system that reads
online commands in a primitive language and displays the results.

The algorithms are based on a recursive way to represent nonnegative integers
by binary trees: The empty tree is $r(0)$, the representation of zero.
And the tree that represents the number $2^a+b$, where $0\le b<2^a$, is
the binary tree whose left and right subtrees are $r(a)$ and $r(b)$.

Stating this another way, let $\star$ be the binary operation on nonnegative
integers defined by the rule $a\star b=2^a+b$. This operator is not
associative, so we need to insert parentheses to indicate the meaning;
however, right associativity is implied whenever there's any doubt,
so that $a\star b\star c$ means $a\star(b\star c)$. Notice that $a\star b
\star c=2^a+a^b+c$, so the partial commutative law $a\star b\star c=b\star a
\star c$ is valid. We can use $\star$ to assign a number $v(T)$ to each binary
tree~$T$, by saying that $v(\Lambda)=0$ and $v(T)=v(T_l)\star v(T_r)$ when
$T$ is nonempty.

A binary tree is a {\it normal form\/} if it is the representation of
some integer as described above. It isn't hard to prove that this condition
holds if and only if each node $x$ that has a right child $x_r$ satisfies
the condition $v(xl)>v(x_{rl})$.

The main algorithms in this program compute the sum and product of
binary trees, in the sense that $v(T+T')=v(T)+v(T')$ and $v(T\times T')
=v(T)v(T')$. If the tree operands are normal, the results are too. Otherwise
the sum and product operations are a bit peculiar, but they are still
well defined (they don't blow up), and they might even turn out to define
interesting groupoids.

Lots of interesting research problems arise immediately in this study, and
I haven't time to answer them now. So I'll just state a few of the more
obvious ones. For example: How many $n$-node binary trees are in normal form?
Call this number $b_{n+1}$. It can be shown that the generating function
$B(z)$ is defined by the formula $B(z)=z\exp\bigl(B(z)-{1\over2}B(z^2)
{1\over3}B(z^3)-\cdots\,\bigr)$.
I'm virtually certain that a little analysis will establish
the formula $b_n\sim c\alpha^n/n^{3/2}$, where $c\approx0.36$ and
$\alpha\approx2.52$, using methods that P\'olya applied to the similar
equation $A(z)=z\exp\bigl(A(z)+{1\over2}A(z^2)+{1\over3}A(z^3)+\cdots\,\bigr)$;
see {\sl Fundamental Algorithms}, exercise 2.3.4.4--4. The latter equation,
incidentally, enumerates binary trees that are in normal form under the
weaker condition $v(x_l)\ge v(x_{rl})$. The operator corresponding to this
weaker condition is $a\star b=\omega^a+b$; it gives all the ``small''
{\it ordinal\/} numbers. The free groupoid on one letter satisfying the
axiom $a\star b\star c=b\star a\star c$ is isomorphic to the binary trees
that have this weaker normal form.

Another tantalizing problem: Estimate the size of the binary tree that
represents $n$, when $n$ is large. This is the solution to the recurrence
$$f(0)=0\,;\qquad f(2^a+b)=1+f(a)+f(b)\,,\quad\hbox{when $0\le b<2^a$}.$$
Let $L(1)=1$ and $L(n)=\lfloor\lg n\rfloor L\bigl(\lfloor\lg n\rfloor\bigr)$
for $n>1$. (Thus, $L(n)=\lg n(\lg\lg n)(\lg\lg\lg n)\ldots\,$, rounding
each factor down to an integer and continuing until that integer reaches~1).
Then it can be shown that $f(n)=\lfloor cL(n)/2^{\lg*n}\rfloor-2$ when
$n$ has the special form $2\uparrow\uparrow m-1$ (namely a stack of 2s
minus one): 1, 3, 7, 65535, $2^{65536}-1$, etc. Here $\lg*1=0$ and
$\lg*n=1+\lg*\lfloor\lg n\rfloor$ when $n>1$. I conjecture that
$f(n)\le\lfloor cL(n)/2^{\lg*n}\rfloor-2$ for all $n>0$. It is quite easy
to prove the weaker bound $f(n)\le4L(n)-1$ by induction.

How many binary trees give the value $n$? If this number is $c_n$, the
generating function $C(z)$ satisfies
$$C(z)={1\over1-c_0z-c_1z^2-c_2z^4-c_3z^8-\cdots}\,,$$
so we find that $C(z)=1+z+2z^2+3z^3+7z^4+12z^5+23z^6+41z^7+81z^8+149z^9
+282z^{10}+\cdots\,$; what is the asymptotic growth?

How many distinct values can you get by inserting parentheses into
the expression $2\uparrow2\uparrow\cdots\uparrow2$, when there are $n$~2s?
Since $2^{2^b}\uparrow2^{2^a}=2^{2^{a\star b}}$, this is the same as the
number of distinct values you can get by inserting parentheses into
the expression $0\star0\star\cdots\star0$ when there are $n$~0s. So it's
the number of distinct values obtainable from $n-1$-node binary trees.
This sequence begins $1,1,1,2,4,8,17,36,78,171,379$, according to
Guy and Selfridge [{\sl AMM\/ \bf80} (1973), 868--876], but its general
characteristics are unknown.

Other problems concern time bounds for the algorithms below.

\Y\B\8\#\&{include} \.{<stdio.h>}\7
\X11:Type definitions\X\6
\X2:Global variables\X\6
\X13:Basic subroutines\X\6
\X20:Subroutines\X\7
\\{main}(\,)\1\1\2\2\6
${}\{{}$\1\6
\&{register} \&{int} \|k;\6
\&{register} \&{node} ${}{*}\|p;{}$\7
\X5:Initialize the data structures\X;\6
\&{while} (\T{1})\1\5
\X3:Prompt the user for a command and execute it\X\2\6
\4${}\}{}$\2\par
\fi

\N{1}{2}Input conventions. The user types short commands in a simple postfix
language. For example, `\.{f2} \.{f3} \.+ \.s' means, ``fetch a copy of
tree"2 (the output of a previous step), also fetch tree~3,
add them, and compute the successor.''
This tree will be displayed, and it might be fetched in later commands.

Spaces in the input are ignored. Numbers are treated as decimal parameters
for the operator that they follow; the default is zero if no explicit
parameter is given. Each operator is a single character with mnemonic
significance.

The ``user manual'' is distributed throughout this program, since I keep
adding features as I go. The operator `\.h' gives online help---a brief
summary of all operators.

\Y\B\4\D$\\{buf\_size}$ \5
\T{200}\par
\Y\B\4\X2:Global variables\X${}\E{}$\6
\&{char} ${}{*}\\{helps}[\T{128}]{}$;\C{ strings describing each operator }\6
\&{char} \\{buf}[\\{buf\_size}];\C{ the user's input goes here }\6
\&{char} ${}{*}\\{loc}{}$;\C{ where we are looking in the buffer }\6
\&{char} \\{op};\C{ the current operator }\6
\&{int} \\{param};\C{ the parameter to the current operator }\par
\As12, 24, 28, 45, 50, 53, 58\ETs63.
\U1.\fi

\M{3}\B\X3:Prompt the user for a command and execute it\X${}\E{}$\6
${}\{{}$\1\6
\X9:Fill \PB{\\{buf}} with the user's next sequence of commands\X;\6
\X29:Clear the current stack\X;\6
\&{while} (\T{1})\5
${}\{{}$\1\6
\X10:Set \PB{\\{op}} and \PB{\\{param}} for the next operator\X;\6
\&{switch} (\\{op})\5
${}\{{}$\1\6
\4\&{case} \.{'\\n'}:\5
\&{goto} \\{dump\_stack};\6
\X6:Cases for one-character operators\X\6
\4\&{default}:\5
${}\\{printf}(\.{"Unknown\ operator\ `\%}\)\.{c'!\\n"},\39\\{op});{}$\6
\4${}\}{}$\2\6
\4${}\}{}$\2\6
\4\\{dump\_stack}:\5
\X30:Display and save all trees currently in the stack\X;\6
\X61:Check that the saved trees account for all the \PB{\\{used}} nodes\X;\6
\4${}\}{}$\2\par
\U1.\fi

\M{4}Here's an example of how each operator is introduced; we begin with
the `help' feature.

\Y\B\4\X4:Define the help strings\X${}\E{}$\6
$\\{helps}[\.{'h'}]\K\.{".helpful\ summary\ of}\)\.{\ all\ known\ operators}\)%
\.{"}{}$;\C{ `\..' means that this operator ignores its parameter }\par
\As7, 31, 33, 36, 38, 48\ETs51.
\U5.\fi

\M{5}\B\X5:Initialize the data structures\X${}\E{}$\6
\X4:Define the help strings\X;\par
\A46.
\U1.\fi

\M{6}I must remember to say \PB{\&{break}} at the end of the program for each
case.

\Y\B\4\X6:Cases for one-character operators\X${}\E{}$\6
\4\&{case} \.{'h'}:\5
\\{printf}(\.{"The\ following\ opera}\)\.{tors\ are\ currently\ i}\)%
\.{mplemented:\\n"});\6
\&{for} ${}(\|k\K\T{0};{}$ ${}\|k<\T{128};{}$ ${}\|k\PP){}$\1\6
\&{if} (\\{helps}[\|k])\1\5
${}\\{printf}(\.{"\ \ \%c\%s\ \%s\\n"},\39\|k,\39({*}\\{helps}[\|k]\E\.{'.'}\?%
\.{":\ \ \ "}:\.{"<n>:"}),\39\\{helps}[\|k]+\T{1});{}$\2\2\6
\&{break};\par
\As8, 32, 34, 35, 37, 39, 40, 41, 42, 43, 49\ETs52.
\U3.\fi

\M{7}Here's another easy case: The normal way to stop, instead of resorting
to control-C, is to give the quit command.

\Y\B\4\X4:Define the help strings\X${}\mathrel+\E{}$\6
$\\{helps}[\.{'q'}]\K\.{".quit\ the\ program"}{}$;\par
\fi

\M{8}\B\X6:Cases for one-character operators\X${}\mathrel+\E{}$\6
\4\&{case} \.{'q'}:\5
\\{printf}(\.{"Type\ <return>\ to\ co}\)\.{nfirm\ quitting:"});\6
\&{if} ${}(\\{getchar}(\,)\E\.{'\\n'}){}$\1\5
\&{return} \T{0};\2\6
${}\\{fgets}(\\{buf},\39\\{buf\_size},\39\\{stdin}){}$;\C{ flush the rest of
that line }\6
\&{goto} \\{dump\_stack};\par
\fi

\M{9}\B\X9:Fill \PB{\\{buf}} with the user's next sequence of commands\X${}%
\E{}$\6
\\{printf}(\.{"?\ "});\C{ this is the prompt }\6
\&{if} ${}(\\{fgets}(\\{buf},\39\\{buf\_size},\39\\{stdin})\E\T{0}){}$\1\5
\&{return} \T{0};\C{ we quit at end of file }\2\6
${}\\{loc}\K\\{buf}{}$;\C{ get ready to scan the buffer }\par
\U3.\fi

\M{10} The scanning routine is intentionally simple.

\Y\B\4\D$\\{large}$ \5
\T{1000000000}\C{ parameter numbers aren't allowed to get this big }\par
\B\4\D$\\{larg}$ \5
\T{100000000}\C{ \PB{$\\{large}/\T{10}$} }\par
\Y\B\4\X10:Set \PB{\\{op}} and \PB{\\{param}} for the next operator\X${}\E{}$\6
\&{while} ${}(\\{loc}<{\AND}\\{buf}[\\{buf\_size}]\W({*}\\{loc}\E\.{'\ '}\V{*}%
\\{loc}<\T{0}\V{*}\\{loc}\G\T{128})){}$\1\5
${}\\{loc}\PP{}$;\C{ bypass blanks and exotic characters }\2\6
${}\\{param}\K\T{0}{}$;\C{ assign the default value }\6
\&{if} ${}(\\{loc}\E{\AND}\\{buf}[\\{buf\_size}]){}$\1\5
${}\\{op}\K\.{'\\n'};{}$\2\6
\&{else}\5
${}\{{}$\1\6
${}\\{op}\K{*}\\{loc}\PP;{}$\6
\&{if} ${}(\\{op}\I\.{'\\n'}){}$\1\6
\&{while} ${}(\\{loc}<{\AND}\\{buf}[\\{buf\_size}]\W({*}\\{loc}\E\.{'\ '}\V({*}%
\\{loc}\Z\.{'9'}\W{*}\\{loc}\G\.{'0'}))){}$\5
${}\{{}$\1\6
\&{if} ${}({*}\\{loc}\I\.{'\ '}){}$\5
${}\{{}$\1\6
\&{if} ${}(\\{param}\G\\{larg}){}$\1\5
${}\\{printf}(\.{"(I'm\ reducing\ your\ }\)\.{large\ parameter\ mod\ }\)\.{\%d)%
\\n"},\39\\{large});{}$\2\6
${}\\{param}\K((\\{param}\MOD\\{larg})*\T{10})+{*}\\{loc}-\.{'0'};{}$\6
\4${}\}{}$\2\6
${}\\{loc}\PP;{}$\6
\4${}\}{}$\2\2\6
\4${}\}{}$\2\par
\U3.\fi

\N{1}{11}Data structures. I'm representing trees in the obvious way: Each node
consists of two pointers and one integer field for multipurpose use.

\Y\B\4\X11:Type definitions\X${}\E{}$\6
\&{typedef} \&{struct} \&{node\_struct} ${}\{{}$\1\6
\&{int} \\{val};\C{ a value temporarily stored with this node }\6
\&{struct} \&{node\_struct} ${}{*}\|l,{}$ ${}{*}\|r{}$;\C{ left and right
subtree pointers }\2\6
${}\}{}$ \&{node};\par
\U1.\fi

\M{12}Storage allocation is dynamic. We will explicitly free nodes when they
are no longer active.

\Y\B\4\X2:Global variables\X${}\mathrel+\E{}$\6
\&{int} \\{used};\C{ this many nodes are active }\6
\&{node} ${}{*}\\{cur\_node}{}$;\C{ the next node to be allocated when we run
out }\6
\&{node} ${}{*}\\{bad\_node}{}$;\C{ if \PB{\\{cur\_node}} equals \PB{\\{bad%
\_node}}, we need another block }\6
\&{node} ${}{*}\\{avail}{}$;\C{ head of list of recycled nodes }\6
\&{int} \\{mems};\C{ the number of memory references to node pointers }\par
\fi

\M{13}\B\D$\\{nodes\_per\_block}$ \5
\T{1000}\par
\Y\B\4\X13:Basic subroutines\X${}\E{}$\6
\&{node} ${}{*}{}$\\{get\_avail}(\,)\C{ allocate a node }\6
${}\{{}$\1\6
\&{register} \&{node} ${}{*}\|p;{}$\7
\&{if} (\\{avail})\5
${}\{{}$\1\6
${}\|p\K\\{avail};{}$\6
${}\\{avail}\K\|p\MG\|r;{}$\6
\4${}\}{}$\2\6
\&{else}\5
${}\{{}$\1\6
\&{if} ${}(\\{cur\_node}\E\\{bad\_node}){}$\5
${}\{{}$\1\6
${}\\{cur\_node}\K{}$(\&{node} ${}{*}){}$ \\{calloc}${}(\\{nodes\_per\_block},%
\39\&{sizeof}(\&{node}));{}$\6
\&{if} ${}(\R\\{cur\_node}){}$\5
${}\{{}$\1\6
\\{printf}(\.{"Omigosh,\ the\ memory}\)\.{\ is\ all\ gone!\\n"});\6
${}\\{exit}({-}\T{1});{}$\6
\4${}\}{}$\2\6
${}\\{bad\_node}\K\\{cur\_node}+\\{nodes\_per\_block};{}$\6
\4${}\}{}$\2\6
${}\|p\K\\{cur\_node}\PP;{}$\6
\4${}\}{}$\2\6
${}\|p\MG\|l\K\|p\MG\|r\K\NULL;{}$\6
${}\\{mems}\PP;{}$\6
${}\\{used}\PP;{}$\6
\&{return} \|p;\6
\4${}\}{}$\2\par
\As14, 15, 16, 17, 18, 19, 55, 60\ETs62.
\U1.\fi

\M{14}\B\X13:Basic subroutines\X${}\mathrel+\E{}$\6
\&{void} \\{free\_node}(\|p)\C{ deallocate a node }\1\1\6
\&{node} ${}{*}\|p;\2\2{}$\6
${}\{{}$\1\6
${}\|p\MG\|r\K\\{avail};{}$\6
${}\\{avail}\K\|p;{}$\6
${}\\{used}\MM;{}$\6
${}\\{mems}\PP;{}$\6
\4${}\}{}$\2\par
\fi

\M{15}We often want to free all nodes of a tree that has served its purpose.

\Y\B\4\X13:Basic subroutines\X${}\mathrel+\E{}$\6
\&{void} \\{recycle}(\|p)\C{ deallocate an entire tree }\1\1\6
\&{node} ${}{*}\|p;\2\2{}$\6
${}\{{}$\1\6
\&{if} ${}(\R\|p){}$\1\5
\&{return};\2\6
${}\\{recycle}(\|p\MG\|l);{}$\6
${}\\{recycle}(\|p\MG\|r);{}$\6
\\{free\_node}(\|p);\6
\4${}\}{}$\2\par
\fi

\M{16}The algorithms for arithmetic are careful (I hope) to access node
pointers only via the subroutines \PB{\\{left}}, \PB{\\{right}}, and \PB{%
\\{change}}. This
makes the program longer and slightly less readable, but it also
ensures that \PB{\\{mems}} will be properly counted.

In my first draft of this code I implemented a reference counter scheme,
but I soon found that explicit deallocation was much better.;

\Y\B\4\X13:Basic subroutines\X${}\mathrel+\E{}$\6
\&{node} ${}{*}{}$\\{left}(\|p)\C{ get the left subtree of a nonempty binary
tree }\1\1\6
\&{node} ${}{*}\|p;\2\2{}$\6
${}\{{}$\1\6
${}\\{mems}\PP;{}$\6
\&{return} \|p${}\MG\|l;{}$\6
\4${}\}{}$\2\7
\&{node} ${}{*}{}$\\{right}(\|p)\C{ get the right subtree of a nonempty binary
tree }\1\1\6
\&{node} ${}{*}\|p;\2\2{}$\6
${}\{{}$\1\6
${}\\{mems}\PP;{}$\6
\&{return} \|p${}\MG\|r;{}$\6
\4${}\}{}$\2\7
\&{void} ${}\\{change}(\|p,\39\|q{}$)\C{ change pointer field \PB{\|p} to \PB{%
\|q} }\1\1\6
\&{node} ${}{*}{*}\|p;{}$\6
\&{node} ${}{*}\|q;\2\2{}$\6
${}\{{}$\1\6
${}{*}\|p\K\|q;{}$\6
${}\\{mems}\PP;{}$\6
\4${}\}{}$\2\par
\fi

\N{1}{17}Simple tree operations. The multiplication routine needs to make
several
copies of subtrees, and the copying algorithm is one of the simplest we
will need. So let's start with it. We can do the harder stuff once we
get into the groove.

\Y\B\4\X13:Basic subroutines\X${}\mathrel+\E{}$\6
\&{node} ${}{*}{}$\\{copy}(\|p)\C{ make a fresh copy of a binary tree }\1\1\6
\&{node} ${}{*}\|p;\2\2{}$\6
${}\{{}$\1\6
\&{register} \&{node} ${}{*}\|q;{}$\7
\&{if} ${}(\R\|p){}$\1\5
\&{return} ${}\NULL;{}$\2\6
${}\|q\K\\{get\_avail}(\,);{}$\6
${}\\{change}({\AND}\|q\MG\|l,\39\\{copy}(\\{left}(\|p)));{}$\6
${}\\{change}({\AND}\|q\MG\|r,\39\\{copy}(\\{right}(\|p)));{}$\6
\&{return} \|q;\6
\4${}\}{}$\2\par
\fi

\M{18}Sometimes I want to copy tree behind the scenes; then I don't want to
count mems.

\Y\B\4\X13:Basic subroutines\X${}\mathrel+\E{}$\6
\&{node} ${}{*}{}$\\{cheap\_copy}(\|p)\C{ make a copy with no mem cost }\1\1\6
\&{node} ${}{*}\|p;\2\2{}$\6
${}\{{}$\1\6
\&{register} \&{node} ${}{*}\|q;{}$\6
\&{register} \&{int} \|m${}\K\\{mems};{}$\7
${}\|q\K\\{copy}(\|p);{}$\6
${}\\{mems}\K\|m;{}$\6
\&{return} \|q;\6
\4${}\}{}$\2\par
\fi

\M{19}Another easy case, frequently needed for arithmetic, is the
lexicographic comparison of binary trees. The \PB{\\{compare}} subroutine
returns $-1$, $0$, or~$+1$ according as $p<q$, $p=q$, or $p>q$.

\Y\B\4\X13:Basic subroutines\X${}\mathrel+\E{}$\6
\&{int} ${}\\{compare}(\|p,\39\|q{}$)\C{ determine whether \PB{\|p} is less
than, equal to, or                                    greater than~\PB{\|q} }\1%
\1\6
\&{node} ${}{*}\|p,{}$ ${}{*}\|q;\2\2{}$\6
${}\{{}$\1\6
\&{register} \&{int} \|k;\7
\&{if} ${}(\R\|p){}$\5
${}\{{}$\1\6
\&{if} ${}(\R\|q){}$\1\5
\&{return} \T{0};\C{ they were both empty }\2\6
\&{return} ${}{-}\T{1}{}$;\C{ only \PB{\|p} was empty, so it's less }\6
\4${}\}{}$\2\6
\&{if} ${}(\R\|q){}$\1\5
\&{return} \T{1};\C{ only \PB{\|q} was empty, so \PB{\|p} was greater }\2\6
${}\|k\K\\{compare}(\\{left}(\|p),\39\\{left}(\|q));{}$\6
\&{if} ${}(\|k\I\T{0}){}$\1\5
\&{return} \|k;\2\6
\&{return} \\{compare}${}(\\{right}(\|p),\39\\{right}(\|q));{}$\6
\4${}\}{}$\2\par
\fi

\M{20}Having laid the groundwork, we come now to the first interesting case:
The \PB{\\{succ}} function adds one to the value represented by a binary tree.

This function would have been more efficient if we had represented
numbers from right to left instead of from left to right. (A dual
representation considers $2^a+b$ where $b$ is a multiple of $2^{a+1}$
rather than a number less than $2^a$. The same number of nodes appears
in both representations; the difference is sort of a reflection
of the tree about a diagonal line, with a few additional alterations.)
But the dual representation makes the comparison operation slower, and
comparisons is more important than successions.

The \PB{\\{succ}} routine is given a pointer to a nonempty binary tree.
It changes that tree $T$ so that the new tree $T'$ has $v(T')=v(T)+1$;
furthermore, $T'$ has the same root node as $T$. Thus this operation
is sort of like `\PB{$\|T\PP$}'.

\Y\B\4\X20:Subroutines\X${}\E{}$\6
\&{void} \\{succ}(\|p)\C{ add one to tree \PB{\|p} }\1\1\6
\&{node} ${}{*}\|p;\2\2{}$\6
${}\{{}$\1\6
\&{register} \&{node} ${}{*}\\{pr},{}$ ${}{*}\\{pl},{}$ ${}{*}\\{prr},{}$
${}{*}\\{prl};{}$\7
${}\\{pr}\K\\{right}(\|p);{}$\6
\&{if} ${}(\R\\{pr}){}$\5
${}\{{}$\1\6
${}\\{pr}\K\\{get\_avail}(\,);{}$\6
${}\\{change}({\AND}\|p\MG\|r,\39\\{pr});{}$\6
\4${}\}{}$\2\6
\&{else}\1\5
\\{succ}(\\{pr});\2\6
${}\\{prr}\K\\{right}(\\{pr});{}$\6
\&{if} ${}(\R\\{prr}){}$\5
${}\{{}$\C{ the successor of \PB{\\{pr}} was a power of two;
should we propagate a carry? }\1\6
${}\\{pl}\K\\{left}(\|p);{}$\6
${}\\{prl}\K\\{left}(\\{pr});{}$\6
\&{if} ${}(\\{compare}(\\{pl},\39\\{prl})\E\T{0}){}$\5
${}\{{}$\C{ yes, we should }\1\6
\\{recycle}(\\{pr});\6
${}\\{change}({\AND}\|p\MG\|r,\39\NULL);{}$\6
\&{if} (\\{pl})\1\5
\\{succ}(\\{pl});\2\6
\&{else}\1\5
${}\\{change}({\AND}\|p\MG\|l,\39\\{get\_avail}(\,));{}$\2\6
\4${}\}{}$\2\6
\4${}\}{}$\2\6
\4${}\}{}$\2\par
\As21, 25, 26, 27, 44, 47, 54\ETs57.
\U1.\fi

\N{1}{21}Addition. Arithmetic is now within our grasp. Again
we need to think a bit about how to do it from left to right (i.e., from
most significant bit to least significant). Here is a way that keeps the
number of subtree comparisons to essentially the same amount as would
be needed if we went from right to left.

Trees \PB{\|p} and \PB{\|q} are destroyed by the action of the \PB{\\{sum}}
procedure,
which returns a tree that represents $v(T)+v(T')$. It also sets the
global variable \PB{\\{easy}} nonzero if no carry propagated from the most
significant bits.

Incidentally, this addition operation is not associative on abnormal
trees. The associative law fails, for example, on the first case I
tried when I got the program working:
$$\eqalign{
b_{123}+(b_{456}+b_{789})&=2^{1+1+4}+2^{1+1}+4+2+2^{1+2^{1+1}}+1\,;\cr
(b_{123}+b_{456})+b_{789}&=2^{1+1+4}+2^{1+1}+4+1+2+2^{1+2^{1+1}}\,.\cr}$$

\Y\B\4\X20:Subroutines\X${}\mathrel+\E{}$\6
\&{node} ${}{*}\\{sum}(\|p,\39\|q{}$)\C{ compute the sum of two binary trees }%
\1\1\6
\&{node} ${}{*}\|p,{}$ ${}{*}\|q;\2\2{}$\6
${}\{{}$\1\6
\&{register} \&{node} ${}{*}\\{pl},{}$ ${}{*}\\{ql};{}$\6
\&{register} \&{int} \|s;\7
${}\\{easy}\K\T{1};{}$\6
\&{if} ${}(\R\|p){}$\1\5
\&{return} \|q;\2\6
\&{if} ${}(\R\|q){}$\1\5
\&{return} \|p;\2\6
${}\\{pl}\K\\{left}(\|p);{}$\6
${}\\{ql}\K\\{left}(\|q);{}$\6
${}\|s\K\\{compare}(\\{pl},\39\\{ql});{}$\6
\&{if} ${}(\|s\E\T{0}){}$\5
${}\{{}$\1\6
\X23:Add \PB{\\{right}(\|p)} to \PB{\\{right}(\|q)} and append this to \PB{%
\\{succ}(\\{pl})}\X\6
${}\\{easy}\K\T{0}{}$;\5
\&{return} \|p;\6
\4${}\}{}$\2\6
\&{else}\5
${}\{{}$\1\6
\&{if} ${}(\|s<\T{0}){}$\1\5
\X22:Swap \PB{\|p} and \PB{\|q} so that \PB{$\|p>\|q$}\X;\2\6
${}\|q\K\\{sum}(\\{right}(\|p),\39\|q);{}$\6
\&{if} (\\{easy})\1\5
\&{goto} \\{no\_sweat};\2\6
\&{else}\5
${}\{{}$\1\6
${}\\{ql}\K\\{left}(\|q);{}$\6
${}\|s\K\\{compare}(\\{pl},\39\\{ql}){}$;\C{ does a carry need to be
propagated? }\6
\&{if} ${}(\|s\E\T{0}){}$\5
${}\{{}$\C{ yup }\1\6
${}\\{change}({\AND}\|p\MG\|r,\39\\{right}(\|q));{}$\6
\\{recycle}(\\{ql});\5
\\{free\_node}(\|q);\6
\&{if} (\\{pl})\1\5
\\{succ}(\\{pl});\2\6
\&{else}\1\5
${}\\{change}({\AND}\|p\MG\|l,\39\\{get\_avail}(\,));{}$\2\6
\&{return} \|p;\6
\4${}\}{}$\2\6
\&{else}\1\5
${}\\{easy}\K\T{1}{}$;\C{ nope }\2\6
\4${}\}{}$\2\6
\4\\{no\_sweat}:\5
${}\\{change}({\AND}\|p\MG\|r,\39\|q);{}$\6
\&{return} \|p;\6
\4${}\}{}$\2\6
\4${}\}{}$\2\par
\fi

\M{22}\B\X22:Swap \PB{\|p} and \PB{\|q} so that \PB{$\|p>\|q$}\X${}\E{}$\6
${}\{{}$\1\6
${}\\{pl}\K\\{ql};{}$\6
${}\\{ql}\K\|p;{}$\6
${}\|p\K\|q;{}$\6
${}\|q\K\\{ql};{}$\6
\4${}\}{}$\2\par
\U21.\fi

\M{23}\B\X23:Add \PB{\\{right}(\|p)} to \PB{\\{right}(\|q)} and append this to %
\PB{\\{succ}(\\{pl})}\X${}\E{}$\6
\\{recycle}(\\{ql});\6
\&{if} (\\{pl})\1\5
\\{succ}(\\{pl});\2\6
\&{else}\1\5
${}\\{change}({\AND}\|p\MG\|l,\39\\{get\_avail}(\,));{}$\2\6
${}\\{change}({\AND}\|p\MG\|r,\39\\{sum}(\\{right}(\|p),\39\\{right}(\|q)));{}$%
\6
\\{free\_node}(\|q);\par
\U21.\fi

\M{24}\B\X2:Global variables\X${}\mathrel+\E{}$\6
\&{int} \\{easy};\C{ communication parameter for the \PB{\\{sum}} routine }\par
\fi

\M{25}One nice spinoff of the addition routine is the following procedure
for normalization:

\Y\B\4\X20:Subroutines\X${}\mathrel+\E{}$\6
\&{node} ${}{*}{}$\\{normalize}(\|p)\C{ change \PB{\|p} to normal form without
changing the value }\1\1\6
\&{node} ${}{*}\|p;\2\2{}$\6
${}\{{}$\1\6
\&{register} \&{node} ${}{*}\|q,{}$ ${}{*}\\{qq};{}$\7
\&{if} ${}(\R\|p){}$\1\5
\&{return};\2\6
${}\|q\K\\{qq}\K\\{left}(\|p);{}$\6
${}\|q\K\\{normalize}(\|q);{}$\6
\&{if} ${}(\|q\I\\{qq}){}$\1\5
${}\\{change}({\AND}\|p\MG\|l,\39\|q);{}$\2\6
${}\|q\K\\{qq}\K\\{right}(\|p);{}$\6
${}\|q\K\\{normalize}(\|q);{}$\6
${}\\{change}({\AND}\|p\MG\|r,\39\NULL);{}$\6
\&{return} \\{sum}${}(\|p,\39\|q);{}$\6
\4${}\}{}$\2\par
\fi

\N{1}{26}Multiplication. A moment's thought reveals the somewhat surprising
fact that
it's easier to multiply by $2^a$ than by $a$. (Because if $b=2^{b_1}+\cdots
+2^{b_k}$, we have $2^ab=2^{a+b_1}-\cdots+2^{a+b_k}$.)

\Y\B\4\X20:Subroutines\X${}\mathrel+\E{}$\6
\&{node} ${}{*}\\{ez\_prod}(\|p,\39\|q{}$)\C{ add \PB{\|p} to exponents of \PB{%
\|q} }\1\1\6
\&{node} ${}{*}\|p,{}$ ${}{*}\|q;\2\2{}$\6
${}\{{}$\1\6
\&{register} \&{node} ${}{*}\\{qq},{}$ ${}{*}\\{qqr};{}$\7
\&{if} ${}(\R\|q){}$\5
${}\{{}$\1\6
\\{recycle}(\|p);\6
\&{return} ${}\NULL;{}$\6
\4${}\}{}$\2\6
\&{for} ${}(\\{qq}\K\|q;{}$ \\{qq}; ${}\\{qq}\K\\{qqr}){}$\5
${}\{{}$\1\6
${}\\{qqr}\K\\{right}(\\{qq});{}$\6
\&{if} (\\{qqr})\1\5
${}\\{change}({\AND}\\{qq}\MG\|l,\39\\{sum}(\\{left}(\\{qq}),\39\\{copy}(%
\|p)));{}$\2\6
\&{else}\1\5
${}\\{change}({\AND}\\{qq}\MG\|l,\39\\{sum}(\\{left}(\\{qq}),\39\|p));{}$\2\6
\4${}\}{}$\2\6
\&{return} \|q;\6
\4${}\}{}$\2\par
\fi

\M{27}Full multiplication is, of course, a sum of such partial multiplications.
I am not implementing it in the cleverest way, since I compute the
final sum as $(\,\cdots((2^{a_1}b +2^{a_2}b)+2^{a_3}b)+\cdots\,)+2^{a_k}b$,
thereby passing $k$ times over many of the nodes. It's obvious how to
reduce this to $\log k$ times per node, but is there a better way?
I leave that as an open problem for now.

A bit of experimentation shows that the product of abnormal trees
might not even be commutative, much less associative.

\Y\B\4\X20:Subroutines\X${}\mathrel+\E{}$\6
\&{node} ${}{*}\\{prod}(\|p,\39\|q{}$)\C{ form the product of \PB{\|p} and \PB{%
\|q} }\1\1\6
\&{node} ${}{*}\|p,{}$ ${}{*}\|q;\2\2{}$\6
${}\{{}$\1\6
\&{register} \&{node} ${}{*}\\{pp},{}$ ${}{*}\\{ppr},{}$ ${}{*}\\{ss};{}$\7
\&{if} ${}(\R\|p\V\R\|q){}$\5
${}\{{}$\1\6
\\{recycle}(\|p);\6
\\{recycle}(\|q);\6
\&{return} ${}\NULL;{}$\6
\4${}\}{}$\2\6
\&{for} ${}(\\{pp}\K\|p,\39\\{ss}\K\NULL;{}$ \\{pp}; ${}\\{pp}\K\\{ppr}){}$\5
${}\{{}$\1\6
${}\\{ppr}\K\\{right}(\\{pp});{}$\6
\&{if} (\\{ppr})\1\5
${}\\{ss}\K\\{sum}(\\{ss},\39\\{ez\_prod}(\\{left}(\\{pp}),\39\\{copy}(%
\|q)));{}$\2\6
\&{else}\1\5
${}\\{ss}\K\\{sum}(\\{ss},\39\\{ez\_prod}(\\{left}(\\{pp}),\39\|q));{}$\2\6
\\{free\_node}(\\{pp});\6
\4${}\}{}$\2\6
\&{return} \\{ss};\6
\4${}\}{}$\2\par
\fi

\N{1}{28}Stack discipline. Some commands put trees on the stack; others operate
on those trees. Everything left on the stack at the end of a command line
is displayed, and assigned an identification number for later use.

\Y\B\4\D$\\{stack\_size}$ \5
\T{20}\C{ this many trees can be on the stack at once }\par
\B\4\D$\\{save\_size}$ \5
\T{1000}\C{ this many trees can be recalled }\par
\Y\B\4\X2:Global variables\X${}\mathrel+\E{}$\6
\&{node} ${}{*}\\{saved}[\\{save\_size}]{}$;\C{ trees that the user might
recall }\6
\&{int} \\{save\_ptr};\C{ the number of saved trees }\6
\&{node} ${}{*}\\{stack}[\\{stack\_size}+\T{1}]{}$;\C{ there's one extra slot
for breathing space }\6
\&{int} \\{stack\_ptr};\C{ the number of items on the stack }\6
\&{int} \\{showing\_mems};\C{ should we tell the user how many mems were used?
}\6
\&{int} \\{showing\_size};\C{ should we tell the user how big each tree is? }\6
\&{int} \\{showing\_usage};\C{ should we tell the user how many nodes are
active? }\6
\&{int} \\{old\_mems};\C{ holding place for \PB{\\{mems}} until we're ready to
report it }\par
\fi

\M{29}\B\X29:Clear the current stack\X${}\E{}$\6
$\\{stack\_ptr}\K\T{0};{}$\6
${}\\{mems}\K\T{0}{}$;\par
\U3.\fi

\M{30}The tree most recently in the stack is kept in \PB{\\{saved}[\T{0}]}.

\Y\B\4\D$\\{operand}(\|n)$ \5
$\\{stack}[\\{stack\_ptr}-(\|n){}$]\par
\Y\B\4\X30:Display and save all trees currently in the stack\X${}\E{}$\6
$\\{old\_mems}\K\\{mems};{}$\6
\&{while} (\\{stack\_ptr})\5
${}\{{}$\1\6
${}\\{stack\_ptr}\MM;{}$\6
\&{if} ${}(\PP\\{save\_ptr}<\\{save\_size}){}$\1\5
${}\|k\K\\{save\_ptr};{}$\2\6
\&{else}\5
${}\{{}$\1\6
${}\|k\K\T{0};{}$\6
\\{recycle}(\\{saved}[\T{0}]);\6
${}\\{save\_ptr}\K\\{save\_size}-\T{1};{}$\6
\4${}\}{}$\2\6
${}\\{saved}[\|k]\K\\{operand}(\T{0});{}$\6
\&{if} ${}(\\{stack\_ptr}\E\T{0}\W\|k>\T{0}){}$\5
${}\{{}$\1\6
\\{recycle}(\\{saved}[\T{0}]);\6
${}\\{saved}[\T{0}]\K\\{copy}(\\{saved}[\|k]);{}$\6
\4${}\}{}$\2\6
\X59:Display tree \PB{\\{saved}[\|k]}\X;\6
\4${}\}{}$\2\6
\&{if} ${}(\\{showing\_mems}\W\\{old\_mems}){}$\1\5
${}\\{printf}(\.{"Operations\ cost\ \%d\ }\)\.{mems\\n"},\39\\{old\_mems});{}$%
\2\6
\&{if} (\\{showing\_usage})\1\5
${}\\{printf}(\.{"(\%d\ nodes\ are\ now\ i}\)\.{n\ use)\\n"},\39\\{used}){}$;\2%
\par
\U3.\fi

\M{31}\B\X4:Define the help strings\X${}\mathrel+\E{}$\6
$\\{helps}[\.{'S'}]\K\.{":show\ tree\ sizes,\ i}\)\.{f\ <n>\ is\ nonzero"};{}$\6
${}\\{helps}[\.{'T'}]\K\.{":show\ computation\ t}\)\.{ime\ in\ mems,\ if\ <n>\
}\)\.{is\ nonzero"};{}$\6
${}\\{helps}[\.{'U'}]\K\.{":show\ node\ usage,\ i}\)\.{f\ <n>\ is\
nonzero"};{}$\6
${}\\{helps}[\.{'k'}]\K\.{":kill\ \%<n>\ to\ conse}\)\.{rve\ memory"}{}$;\par
\fi

\M{32}\B\X6:Cases for one-character operators\X${}\mathrel+\E{}$\6
\4\&{case} \.{'S'}:\5
${}\\{showing\_size}\K\\{param}{}$;\5
\&{break};\6
\4\&{case} \.{'T'}:\5
${}\\{showing\_mems}\K\\{param}{}$;\5
\&{break};\6
\4\&{case} \.{'U'}:\5
${}\\{showing\_usage}\K\\{param}{}$;\5
\&{break};\6
\4\&{case} \.{'k'}:\6
\&{if} ${}(\\{param}>\\{save\_ptr}){}$\1\5
${}\\{printf}(\.{"You\ can't\ do\ k\%d,\ b}\)\.{ecause\ \%\%\%d\ doesn't\ }\)%
\.{exist!\\n"},\39\\{param},\39\\{param});{}$\2\6
\&{else}\5
${}\{{}$\1\6
\\{recycle}(\\{saved}[\\{param}]);\6
${}\\{saved}[\\{param}]\K\NULL;{}$\6
\4${}\}{}$\2\6
\&{break};\par
\fi

\M{33}One way to put a new item on the stack is to copy an old item.

\Y\B\4\X4:Define the help strings\X${}\mathrel+\E{}$\6
$\\{helps}[\.{'\%'}]\K\.{":recall\ a\ previousl}\)\.{y\ computed\ tree"};{}$\6
${}\\{helps}[\.{'d'}]\K\.{":duplicate\ a\ tree\ t}\)\.{hat's\ already\ on\ the}%
\)\.{\ stack"}{}$;\par
\fi

\M{34}\B\X6:Cases for one-character operators\X${}\mathrel+\E{}$\6
\4\&{case} \.{'\%'}:\6
\&{if} ${}(\\{param}>\\{save\_ptr}){}$\5
${}\{{}$\1\6
${}\\{printf}(\.{"(\%\%\%d\ is\ unknown;\ I}\)\.{'m\ using\ \%\%0\ instead}\)%
\.{)\\n"},\39\\{param});{}$\6
${}\\{param}\K\T{0};{}$\6
\4${}\}{}$\2\6
${}\\{operand}(\T{0})\K\\{cheap\_copy}(\\{saved}[\\{param}]);{}$\6
\4\\{inc\_stack}:\6
\&{if} ${}(\\{stack\_ptr}<\\{stack\_size}){}$\5
${}\{{}$\1\6
${}\\{stack\_ptr}\PP;{}$\6
\&{break};\6
\4${}\}{}$\2\6
\\{printf}(\.{"Oops---the\ stack\ ov}\)\.{erflowed!\\n"});\6
\\{recycle}(\\{operand}(\T{0}));\6
\&{goto} \\{dump\_stack};\par
\fi

\M{35}The command `\.d' duplicates the top tree on the stack. Similarly,
`\.{d3}' duplicates the item three down from the top.

\Y\B\4\D$\\{check\_stack}(\|k)$ \6
\&{if} ${}(\\{stack\_ptr}<\|k){}$\5
${}\{{}$\1\6
${}\\{printf}(\.{"Not\ enough\ items\ on}\)\.{\ the\ stack\ for\ opera}\)\.{tor%
\ \%c!\\n"},\39\\{op});{}$\6
\&{goto} \\{dump\_stack};\6
\4${}\}{}$\2\par
\Y\B\4\X6:Cases for one-character operators\X${}\mathrel+\E{}$\6
\4\&{case} \.{'d'}:\5
${}\\{check\_stack}(\\{param}+\T{1});{}$\6
${}\\{operand}(\T{0})\K\\{cheap\_copy}(\\{operand}(\\{param}+\T{1}));{}$\6
\&{goto} \\{inc\_stack};\par
\fi

\M{36}Here are two trivial operations that seem pointless, because I haven't
allowed the user to define macros. But in fact, users do have macros,
because they can run {\sc TCALC} from an emacs shell.

\Y\B\4\X4:Define the help strings\X${}\mathrel+\E{}$\6
$\\{helps}[\.{'p'}]\K\.{".pop\ the\ top\ tree\ o}\)\.{ff\ the\ stack"};{}$\6
${}\\{helps}[\.{'x'}]\K\.{".exchange\ the\ top\ t}\)\.{wo\ trees"}{}$;\par
\fi

\M{37}Of course I could generalize these commands so that \PB{\\{param}} is
relevant.

\Y\B\4\X6:Cases for one-character operators\X${}\mathrel+\E{}$\6
\4\&{case} \.{'p'}:\5
\\{check\_stack}(\T{1});\6
${}\\{stack\_ptr}\MM;{}$\6
\\{recycle}(\\{operand}(\T{0}));\6
\&{break};\6
\4\&{case} \.{'x'}:\5
\\{check\_stack}(\T{2});\6
${}\|p\K\\{operand}(\T{2}){}$;\5
${}\\{operand}(\T{2})\K\\{operand}(\T{1}){}$;\5
${}\\{operand}(\T{1})\K\|p;{}$\6
\&{break};\par
\fi

\M{38}Now we implement the arithmetic operators. Later we'll define an
operator \.t such that `\.{tj}' replaces tree $a$ by $2^a$.

\Y\B\4\X4:Define the help strings\X${}\mathrel+\E{}$\6
$\\{helps}[\.{'l'}]\K\.{".replace\ tree\ by\ it}\)\.{s\ log\ (the\ left\ subt}%
\)\.{ree)"};{}$\6
${}\\{helps}[\.{'r'}]\K\.{".replace\ tree\ by\ it}\)\.{s\ remainder\ (the\ rig}%
\)\.{ht\ subtree)"};{}$\6
${}\\{helps}[\.{'s'}]\K\.{".replace\ tree\ by\ it}\)\.{s\ successor"};{}$\6
${}\\{helps}[\.{'n'}]\K\.{".normalize\ a\ tree"};{}$\6
${}\\{helps}[\.{'+'}]\K\.{".replace\ a,b\ by\ a+b}\)\.{"};{}$\6
${}\\{helps}[\.{'*'}]\K\.{".replace\ a,b\ by\ ab"};{}$\6
${}\\{helps}[\.{'\^'}]\K\.{".replace\ a,b\ by\ a\^b}\)\.{,\ assuming\ that\ a\
is}\)\.{\ a\ power\ of\ 2"};{}$\6
${}\\{helps}[\.{'j'}]\K\.{".replace\ a,b\ by\ 2\^a}\)\.{+b"}{}$;\C{ \.j is for
``join'' }\6
${}\\{helps}[\.{'m'}]\K\.{".replace\ a,b\ by\ 2\^a}\)\.{\ b"}{}$;\par
\fi

\M{39}Here's a typical unary operator.

\Y\B\4\X6:Cases for one-character operators\X${}\mathrel+\E{}$\6
\4\&{case} \.{'n'}:\5
\\{check\_stack}(\T{1});\C{ normalization }\6
${}\\{operand}(\T{1})\K\\{normalize}(\\{operand}(\T{1}));{}$\6
\&{break};\par
\fi

\M{40}And another, only slightly more tricky.

\Y\B\4\X6:Cases for one-character operators\X${}\mathrel+\E{}$\6
\4\&{case} \.{'s'}:\5
\\{check\_stack}(\T{1});\C{ the succ operation }\6
\&{if} (\\{operand}(\T{1}))\1\5
\\{succ}(\\{operand}(\T{1}));\2\6
\&{else}\1\5
${}\\{operand}(\T{1})\K\\{get\_avail}(\,);{}$\2\6
\&{break};\par
\fi

\M{41}The \.l and \.r operators are charged as many mems as it takes to
recycle the discarded nodes of the tree.

\Y\B\4\X6:Cases for one-character operators\X${}\mathrel+\E{}$\6
\4\&{case} \.{'l'}:\5
\\{check\_stack}(\T{1});\C{ the log operation }\6
${}\|p\K\\{operand}(\T{1});{}$\6
\&{if} ${}(\R\|p){}$\1\5
\\{printf}(\.{"(log\ 0\ is\ undefined}\)\.{;\ I'm\ using\ 0)\\n"});\2\6
\&{else}\5
${}\{{}$\1\6
${}\\{operand}(\T{1})\K\\{left}(\|p);{}$\6
\\{recycle}(\\{right}(\|p));\6
\\{free\_node}(\|p);\6
\4${}\}{}$\2\6
\&{break};\6
\4\&{case} \.{'r'}:\5
\\{check\_stack}(\T{1});\C{ the rem operation }\6
${}\|p\K\\{operand}(\T{1});{}$\6
\&{if} ${}(\R\|p){}$\1\5
\\{printf}(\.{"(rem\ 0\ is\ undefined}\)\.{;\ I'm\ using\ 0)\\n"});\2\6
\&{else}\5
${}\{{}$\1\6
${}\\{operand}(\T{1})\K\\{right}(\|p);{}$\6
\\{recycle}(\\{left}(\|p));\6
\\{free\_node}(\|p);\6
\4${}\}{}$\2\6
\&{break};\par
\fi

\M{42}Binary operations are equally simple.

\Y\B\4\X6:Cases for one-character operators\X${}\mathrel+\E{}$\6
\4\&{case} \.{'j'}:\5
\\{check\_stack}(\T{2});\C{ prepare for joining }\6
${}\\{stack\_ptr}\MM;{}$\6
${}\|p\K\\{get\_avail}(\,);{}$\6
${}\|p\MG\|l\K\\{operand}(\T{1});{}$\6
${}\|p\MG\|r\K\\{operand}(\T{0});{}$\6
\4\\{return\_p}:\5
${}\\{operand}(\T{1})\K\|p;{}$\6
\&{break};\6
\4\&{case} \.{'+'}:\5
\\{check\_stack}(\T{2});\C{ prepare for addition }\6
${}\\{stack\_ptr}\MM;{}$\6
${}\\{operand}(\T{1})\K\\{sum}(\\{operand}(\T{1}),\39\\{operand}(\T{0}));{}$\6
\&{break};\6
\4\&{case} \.{'*'}:\5
\\{check\_stack}(\T{2});\C{ prepare for multiplication }\6
${}\\{stack\_ptr}\MM;{}$\6
${}\\{operand}(\T{1})\K\\{prod}(\\{operand}(\T{1}),\39\\{operand}(\T{0}));{}$\6
\&{break};\6
\4\&{case} \.{'m'}:\5
\\{check\_stack}(\T{2});\C{ prepare for power-of-2 multiplication }\6
${}\\{stack\_ptr}\MM;{}$\6
${}\\{operand}(\T{1})\K\\{ez\_prod}(\\{operand}(\T{1}),\39\\{operand}(%
\T{0}));{}$\6
\&{break};\par
\fi

\M{43}Here's the only one that's not quite trivial. Strictly speaking, I
should disallow $0^x$; but the implementation is so easy, I went ahead
and did it.

\Y\B\4\X6:Cases for one-character operators\X${}\mathrel+\E{}$\6
\4\&{case} \.{'\^'}:\5
\\{check\_stack}(\T{2});\C{ prepare for exponentiation }\6
${}\\{stack\_ptr}\MM;{}$\6
${}\|p\K\\{operand}(\T{1});{}$\6
\&{if} ${}(\R\|p){}$\5
${}\{{}$\1\6
\&{if} (\\{operand}(\T{0}))\1\5
\\{recycle}(\\{operand}(\T{0}));\2\6
\&{else}\1\5
${}\|p\K\\{get\_avail}(\,){}$;\C{ $0^0=1$ }\2\6
\4${}\}{}$\2\6
\&{else} \&{if} (\\{right}(\|p))\5
${}\{{}$\1\6
\\{printf}(\.{"Sorry,\ I\ don't\ do\ a}\)\.{\^b\ unless\ a\ is\ a\ pow}\)\.{er\
of\ 2!\\n"});\6
${}\\{stack\_ptr}\PP;{}$\6
\&{goto} \\{dump\_stack};\6
\4${}\}{}$\2\6
\&{else}\1\5
${}\\{change}({\AND}\|p\MG\|l,\39\\{prod}(\\{left}(\|p),\39\\{operand}(%
\T{0})));{}$\2\6
\&{goto} \\{return\_p};\par
\fi

\N{1}{44}Generating binary trees. But how do the trees get built in the first
place?
One useful way to get a fairly big tree is to ask for `\.t$n$', the tree
that canonically represents~$n$. Then we can get bigger by multiplication
and exponentiation, etc.

If this program is working properly, and if $n$ does not exceed the
\PB{\\{threshold}} for compression to be described below, the binary tree
created here will be displayed simply as the integer~$n$.

\Y\B\4\X20:Subroutines\X${}\mathrel+\E{}$\6
\&{node} ${}{*}{}$\\{normal\_tree}(\|n)\C{ generate the standard tree
representation of \PB{\|n} }\1\1\6
\&{int} \|n;\2\2\6
${}\{{}$\1\6
\&{register} \&{int} \|k;\6
\&{register} \&{node} ${}{*}\|p;{}$\7
\&{if} ${}(\R\|n){}$\1\5
\&{return} ${}\NULL;{}$\2\6
\&{for} ${}(\|k\K\T{0};{}$ ${}(\T{1}\LL\|k)\Z\|n;{}$ ${}\|k\PP){}$\1\5
;\C{ compute $k=1+\lfloor\lg n\rfloor$ }\2\6
${}\|p\K\\{get\_avail}(\,){}$;\5
${}\\{mems}\MM;{}$\6
${}\|p\MG\|l\K\\{normal\_tree}(\|k-\T{1});{}$\6
${}\|p\MG\|r\K\\{normal\_tree}(\|n-(\T{1}\LL(\|k-\T{1})));{}$\6
\&{return} \|p;\6
\4${}\}{}$\2\par
\fi

\M{45}There's also a convenient way to build random binary trees, so that
we can experiment with abnormal structures.

For these, it's handy to have a table of the Catalan numbers, which
enumerate the binary trees that have $n$~nodes.

\Y\B\4\X2:Global variables\X${}\mathrel+\E{}$\6
\&{int} \\{cat}[\T{20}];\C{ the first twenty Catalan numbers; \PB{$\\{cat}[%
\T{19}]\K\T{1767263190}$} }\par
\fi

\M{46}We have to be careful when evaluating $\PB{\\{cat}}[n]=(4n-2)\PB{%
\\{cat}}[n-1]/(n+1)$,
because the intermediate result might overflow even though the answer
is a single-precision integer.

\Y\B\4\X5:Initialize the data structures\X${}\mathrel+\E{}$\6
$\\{cat}[\T{0}]\K\T{1};{}$\6
\&{for} ${}(\|k\K\T{1};{}$ ${}\|k<\T{20};{}$ ${}\|k\PP){}$\5
${}\{{}$\1\6
\&{register} \&{int} \\{quot}${}\K\\{cat}[\|k-\T{1}]/(\|k+\T{1}),{}$ \\{rem}${}%
\K\\{cat}[\|k-\T{1}]\MOD(\|k+\T{1});{}$\7
${}\\{cat}[\|k]\K(\T{4}*\|k-\T{2})*\\{quot}+(\&{int})(((\T{4}*\|k-\T{2})*%
\\{rem})/(\|k+\T{1}));{}$\6
\4${}\}{}$\2\par
\fi

\M{47}The \PB{\\{btree}} subroutine is called only when $0\le m<\PB{%
\\{cat}}[n]$.

\Y\B\4\X20:Subroutines\X${}\mathrel+\E{}$\6
\&{node} ${}{*}\\{btree}(\|n,\39\|m{}$)\C{ generate the \PB{\|m}th binary tree
that has \PB{\|n} nodes }\1\1\6
\&{int} \|n${},{}$ \|m;\2\2\6
${}\{{}$\1\6
\&{register} \&{node} ${}{*}\|p;{}$\6
\&{register} \&{int} \|k;\7
\&{if} ${}(\R\|n){}$\1\5
\&{return} ${}\NULL;{}$\2\6
\&{for} ${}(\|k\K\T{0};{}$ ${}\\{cat}[\|k]*\\{cat}[\|n-\T{1}-\|k]\Z\|m;{}$ ${}%
\|k\PP){}$\1\5
${}\|m\MRL{-{\K}}\\{cat}[\|k]*\\{cat}[\|n-\T{1}-\|k];{}$\2\6
${}\|p\K\\{get\_avail}(\,){}$;\5
${}\\{mems}\MM;{}$\6
${}\|p\MG\|l\K\\{btree}(\|k,\39(\&{int})(\|m/\\{cat}[\|n-\T{1}-\|k]));{}$\6
${}\|p\MG\|r\K\\{btree}(\|n-\T{1}-\|k,\39\|m\MOD\\{cat}[\|n-\T{1}-\|k]);{}$\6
\&{return} \|p;\6
\4${}\}{}$\2\par
\fi

\M{48}\B\X4:Define the help strings\X${}\mathrel+\E{}$\6
$\\{helps}[\.{'t'}]\K\.{":the\ standard\ tree\ }\)\.{that\ represents\
<n>"};{}$\6
${}\\{helps}[\.{'b'}]\K\.{":the\ binary\ tree\ of}\)\.{\ rank\ <n>\ in\
lexicog}\)\.{raphic\ order"}{}$;\par
\fi

\M{49}Lexicographic order of binary trees is taken to mean that we order
them first by number of nodes, then recursively by the order of
the \PB{\\{compare}} function.

\Y\B\4\X6:Cases for one-character operators\X${}\mathrel+\E{}$\6
\4\&{case} \.{'t'}:\5
${}\\{operand}(\T{0})\K\\{normal\_tree}(\\{param});{}$\6
\&{goto} \\{inc\_stack};\6
\4\&{case} \.{'b'}:\6
\&{for} ${}(\|k\K\T{0};{}$ ${}\\{cat}[\|k]\Z\\{param};{}$ ${}\|k\PP){}$\1\5
${}\\{param}\MRL{-{\K}}\\{cat}[\|k];{}$\2\6
${}\\{operand}(\T{0})\K\\{btree}(\|k,\39\\{param});{}$\6
\&{goto} \\{inc\_stack};\par
\fi

\N{1}{50}Displaying the results. And finally, the grand climax---the most
interesting
algorithm in this whole program.

A special form of display is appropriate for the binary trees we're
considering. If a tree is in normal form, we can describe it by simply
stating its value. However, a small binary tree can have a
super-astronomical value; there is in fact a tree with six nodes whose
numerical value involves more decimal digits than there are molecules in the
universe! So we use power-of-two notation whenever the value of a
subtree exceeds a given \PB{\\{thrsehold}}.

If \PB{$\\{threshold}\K\T{0}$}, for example, the printed representation of~19,
a tree of seven nodes, takes five lines:
$$\catcode`?=\active \let?=\space
\vbox{\halign{\tt#\hfil\cr
????0\cr
???2\cr
??2?????0\cr
?2?????2???0\cr
2????+2??+2\cr}}$$
(There's one `\.2' for each node, and one `\.+' for each node with a
nonnull right subtree.) But if \PB{$\\{threshold}\K\T{1}$}, the displayed
output will be
$$\catcode`?=\active \let?=\space
\vbox{\halign{\tt#\hfil\cr
???1\cr
??2\cr
?2????1\cr
2???+2?+1\cr}}$$
And with \PB{$\\{threshold}\K\T{2}$} it becomes simpler yet:
$$\catcode`?=\active \let?=\space
\vbox{\halign{\tt#\hfil\cr
??2\cr
?2\cr
2??+2+1\cr}}$$
With \PB{$\\{threshold}\K\T{3}$} the `\.{2+1}' becomes `\.3', and with \PB{$%
\\{threshold}\G\T{19}$}
the whole tree is displayed simply as `\.{19}'.

If a binary tree is not in normal form, its normal-form subtrees are displayed
as usual but its abnormal subtrees are displayed as if the threshold were
exceeded. For example, a two-node tree that has no left subtree will be
displayed as `\.{1+1}' for all values of \PB{$\\{threshold}>\T{0}$}. This
convention
ensures that the tree structure is uniquely characterized by the display.

Some binary trees are so huge, we don't want to see them displayed in full.
The user can suppress detailed output of any tree with
\PB{\\{max\_display\_size}} or more nodes. The value of
\PB{\\{max\_display\_size}} must exceed the default value of 1000.

\Y\B\4\D$\\{max\_tree}$ \5
\T{1000}\C{ we don't display trees having this many nodes }\par
\Y\B\4\X2:Global variables\X${}\mathrel+\E{}$\6
\&{int} \\{threshold};\C{ trees are compressed if their value is at most this }%
\6
\&{int} \\{max\_display\_size}${}\K\\{max\_tree}{}$;\C{ trees are shown if
their size is less than this }\par
\fi

\M{51}\B\X4:Define the help strings\X${}\mathrel+\E{}$\6
$\\{helps}[\.{'M'}]\K\.{".use\ maximum\ possib}\)\.{le\ compression\ thres}\)%
\.{hold\ for\ tree\ displa}\)\.{y"};{}$\6
${}\\{helps}[\.{'N'}]\K\.{":compress\ tree\ disp}\)\.{lays\ only\ for\ t0..t<}%
\)\.{n>"};{}$\6
${}\\{helps}[\.{'O'}]\K\.{":omit\ display\ of\ tr}\)\.{ees\ having\ <n>\ or\
mo}\)\.{re\ nodes"}{}$;\par
\fi

\M{52}\B\X6:Cases for one-character operators\X${}\mathrel+\E{}$\6
\4\&{case} \.{'M'}:\5
${}\\{param}\K\\{large}-\T{1};{}$\6
\4\&{case} \.{'N'}:\5
${}\\{threshold}\K\\{param};{}$\6
\&{break};\6
\4\&{case} \.{'O'}:\6
\&{if} ${}(\\{param}>\\{max\_tree}){}$\5
${}\{{}$\1\6
${}\\{printf}(\.{"(I've\ changed\ O\%d\ t}\)\.{o\ the\ maximum\ permit}\)\.{ted%
\ value,\ O\%d)\\n"},\39\\{param},\39\\{max\_tree});{}$\6
${}\\{param}\K\\{max\_tree};{}$\6
\4${}\}{}$\2\6
${}\\{max\_display\_size}\K\\{param};{}$\6
\&{break};\par
\fi

\M{53}The idea we'll use to display a tree is to tackle the job in two phases.
First, we compute statistics about the tree nodes, so that the root of
the tree ``knows'' about its subtrees. Then we recursively print each
line of the display.

The statistics-gathering phase is handled by a routine called \PB{\\{get%
\_state}}.
It first stamps each node with a serial number \PB{\|j}, which turns out to be
the index of that node in postorder. Then it computes several important
facts about that node's subtree: \PB{\\{width}[\|j]}, the number of columns
needed to display this subtree; \PB{\\{height}[\|j]}, the number
of rows needed to display this subtree, not counting the base row;
\PB{\\{code}[\|j]}, the numerical value of this subtree; and
\PB{\\{lcode}[\|j]}, which is zero if no \.+ sign will be printed for
this subtree, otherwise it's the code for the part that precedes
the \.+. An abnormal subtree is always considered \PB{\\{large}}.

Initialization constants here apply to the empty binary tree, whose
width is~1 because it's always displayed as `\.0'.

\Y\B\4\X2:Global variables\X${}\mathrel+\E{}$\6
\&{int} \\{width}[\\{max\_tree}]${}\K\{\T{1}\}{}$;\C{ columns needed to display
a subtree }\6
\&{int} \\{height}[\\{max\_tree}];\C{ extra rows needed to display a subtree }\6
\&{int} \\{code}[\\{max\_tree}];\C{ compressed numerical value, or \PB{%
\\{large}} }\6
\&{int} \\{lcode}[\\{max\_tree}];\C{ extra info when this subtree needs a \.+
sign }\6
\&{int} \\{count};\C{ this will be set to the number of nodes in the tree }\par
\fi

\M{54}\B\X20:Subroutines\X${}\mathrel+\E{}$\6
\&{void} \\{get\_stats}(\|p)\C{ walk the tree and determine widths, lengths,
etc. }\1\1\6
\&{node} ${}{*}\|p;\2\2{}$\6
${}\{{}$\1\6
\&{register} \&{int} \|j${},{}$ \\{jl}${},{}$ \\{jr};\7
\&{if} ${}(\R\|p){}$\1\5
\&{return};\2\6
${}\\{get\_stats}(\|p\MG\|l){}$;\5
${}\\{get\_stats}(\|p\MG\|r){}$;\C{ postorder traversal }\6
${}\\{jl}\K(\|p\MG\|l\?\|p\MG\|l\MG\\{val}:\T{0});{}$\6
${}\\{jr}\K(\|p\MG\|r\?\|p\MG\|r\MG\\{val}:\T{0});{}$\6
${}\|p\MG\\{val}\K\|j\K\PP\\{count};{}$\6
\&{if} ${}(\\{count}<\\{max\_display\_size}){}$\1\5
\X56:Compute stats for \PB{\|j} from the stats of \PB{\\{jl}}, \PB{\\{jr}}\X;\2%
\6
\4${}\}{}$\2\par
\fi

\M{55}We need a subroutine to compute the width of a decimal number.

\Y\B\4\X13:Basic subroutines\X${}\mathrel+\E{}$\6
\&{int} \\{dwidth}(\|n)\C{ how many digits do we need to print \PB{\|n}? }\1\1\6
\&{int} \|n;\2\2\6
${}\{{}$\1\6
\&{register} \&{int} \|j${},{}$ \|k;\7
\&{for} ${}(\|j\K\T{1},\39\|k\K\T{10};{}$ ${}\|n\G\|k;{}$ ${}\|j\PP,\39\|k%
\MRL{*{\K}}\T{10}){}$\1\5
;\C{ $k=10^j$ }\2\6
\&{return} \|j;\6
\4${}\}{}$\2\par
\fi

\M{56}Here we assume that the constant called \PB{\\{large}} is 1000000000.
We use the facts that \PB{$\\{threshold}\Z\\{large}$} and $2^{29}<\PB{%
\\{large}}<2^{30}$.
Also the fact that \PB{$\\{large}+\\{large}<\\{maxint}$}.

\Y\B\4\D$\\{lg\_large}$ \5
\T{29}\C{ $\lfloor \log_2 \PB{\\{large}}\rfloor$ }\par
\Y\B\4\X56:Compute stats for \PB{\|j} from the stats of \PB{\\{jl}}, \PB{%
\\{jr}}\X${}\E{}$\6
${}\{{}$\1\6
\&{register} \&{int} \\{tjl};\C{ $2^{\hbox{\PB{\\{jl}}}}$, or \PB{\\{large}} }\7
${}\\{tjl}\K(\\{code}[\\{jl}]\Z\\{lg\_large}\?\T{1}\LL\\{code}[\\{jl}]:%
\\{large});{}$\6
\&{if} ${}(\\{tjl}\Z\\{threshold}){}$\5
${}\{{}$\1\6
\&{if} ${}(\\{code}[\\{jr}]<\\{tjl}\W\\{tjl}+\\{code}[\\{jr}]\Z%
\\{threshold}){}$\5
${}\{{}$\1\6
${}\\{code}[\|j]\K\\{tjl}+\\{code}[\\{jr}];{}$\6
${}\\{lcode}[\|j]\K\T{0};{}$\6
${}\\{width}[\|j]\K\\{dwidth}(\\{code}[\|j]);{}$\6
${}\\{height}[\|j]\K\T{0};{}$\6
\4${}\}{}$\2\6
\&{else}\5
${}\{{}$\1\6
${}\\{code}[\|j]\K\\{large};{}$\6
${}\\{lcode}[\|j]\K\\{tjl};{}$\6
${}\\{width}[\|j]\K\\{dwidth}(\\{tjl})+\\{width}[\\{jr}]+\T{1};{}$\6
${}\\{height}[\|j]\K\\{height}[\\{jr}];{}$\6
\4${}\}{}$\2\6
\4${}\}{}$\2\6
\&{else}\5
${}\{{}$\1\6
${}\\{code}[\|j]\K\\{large};{}$\6
${}\\{width}[\|j]\K\\{width}[\\{jl}]+\\{width}[\\{jr}];{}$\6
\&{if} ${}(\|p\MG\|r\E\T{0}){}$\1\5
${}\\{lcode}[\|j]\K\T{0};{}$\2\6
\&{else}\1\5
${}\\{lcode}[\|j]\K\\{large},\39\\{width}[\|j]\MRL{+{\K}}\T{2};{}$\2\6
${}\\{height}[\|j]\K\T{1}+\\{height}[\\{jl}];{}$\6
\&{if} ${}(\\{height}[\\{jr}]>\\{height}[\|j]){}$\1\5
${}\\{height}[\|j]\K\\{height}[\\{jr}];{}$\2\6
\4${}\}{}$\2\6
\4${}\}{}$\2\par
\U54.\fi

\M{57}The second phase is governed by another recursive procedure. This
one, called \PB{\\{print\_rep}}, has three parameters representing a subtree
to be displayed and its starting line and column numbers. Lines are
numbered 0 and up from bottom to top.

Global variable \PB{\|h} contains the line actually being printed. If \PB{$\|l%
\I\|h$},
we keep track of our position but don't emit any characters. The
subroutine is called only when \PB{$\|l\Z\|h\Z\|l+\\{height}[\|j]$}, where \PB{%
\|j} is the
postorder index of the subtree being printed.

Another global variable, \PB{\\{col}}, represents the number of columns output
so far on line~\PB{\|h}.

\Y\B\4\D$\\{align\_to}(\|c)$ \6
\&{while} ${}(\\{col}<\|c){}$\5
${}\{{}$\5
\1${}\\{col}\PP{}$;\5
\\{putchar}(\.{'\ '});\5
${}\}{}$\2\par
\B\4\D$\\{print\_digs}(\|n)$ \6
${}\{{}$\5
\1\\{align\_to}(\|c);\5
${}\\{printf}(\.{"\%d"},\39\|n){}$;\5
${}\\{col}\MRL{+{\K}}\\{dwidth}(\|n){}$;\5
${}\}{}$\2\par
\B\4\D$\\{print\_char}(\|n)$ \6
${}\{{}$\5
\1\\{align\_to}(\|c);\5
\\{putchar}(\|n);\5
${}\\{col}\PP{}$;\5
${}\}{}$\2\par
\Y\B\4\X20:Subroutines\X${}\mathrel+\E{}$\6
\&{void} ${}\\{print\_rep}(\|p,\39\|l,\39\|c{}$)\C{ print the representation of
\PB{\|p} }\1\1\6
\&{node} ${}{*}\|p{}$;\C{ the subtree in question }\6
\&{int} \|l${},{}$ \|c;\C{ the starting line and column positions }\2\2\6
${}\{{}$\1\6
\&{register} \&{int} \|j${}\K(\|p\?\|p\MG\\{val}:\T{0});{}$\7
\&{if} ${}(\\{code}[\|j]<\\{large}){}$\5
${}\{{}$\1\6
\&{if} ${}(\|l\E\|h){}$\1\5
\\{print\_digs}(\\{code}[\|j]);\2\6
\4${}\}{}$\2\6
\&{else} \&{if} ${}(\\{lcode}[\|j]\W\\{lcode}[\|j]<\\{large}){}$\5
${}\{{}$\1\6
\&{if} ${}(\|l\E\|h){}$\1\5
\\{print\_digs}(\\{lcode}[\|j]);\2\6
\4${}\}{}$\2\6
\&{else}\5
${}\{{}$\1\6
\&{register} \&{int} \\{jl}${}\K(\|p\MG\|l\?\|p\MG\|l\MG\\{val}:\T{0});{}$\7
\&{if} ${}(\|l\E\|h){}$\1\5
\\{print\_char}(\.{'2'});\2\6
\&{if} ${}(\|l<\|h\W\|l+\T{1}+\\{height}[\\{jl}]\G\|h){}$\1\5
${}\\{print\_rep}(\|p\MG\|l,\39\|l+\T{1},\39\|c+\T{1});{}$\2\6
\4${}\}{}$\2\6
\&{if} (\\{lcode}[\|j])\5
${}\{{}$\1\6
\&{register} \\{jr}${}\K\|p\MG\|r\MG\\{val}{}$;\C{ we know that \PB{$\|p\MG\|r%
\I\NULL$} }\7
\&{if} ${}(\|l+\\{height}[\\{jr}]\G\|h){}$\5
${}\{{}$\1\6
${}\|c\MRL{+{\K}}\\{width}[\|j]-\\{width}[\\{jr}]-\T{1};{}$\6
\&{if} ${}(\|l\E\|h){}$\1\5
\\{print\_char}(\.{'+'});\2\6
${}\\{print\_rep}(\|p\MG\|r,\39\|l,\39\|c+\T{1});{}$\6
\4${}\}{}$\2\6
\4${}\}{}$\2\6
\4${}\}{}$\2\par
\fi

\M{58}\B\X2:Global variables\X${}\mathrel+\E{}$\6
\&{int} \|h;\C{ the row currently being printed }\6
\&{int} \\{col};\C{ the col currently being printed }\par
\fi

\M{59}OK, we've built the necessary recursive mechanisms; now we just have
to supply the driver program.

\Y\B\4\X59:Display tree \PB{\\{saved}[\|k]}\X${}\E{}$\6
$\\{count}\K\T{0};{}$\6
${}\|p\K\\{saved}[\|k];{}$\6
\\{get\_stats}(\|p);\6
\&{if} ${}(\\{count}\G\\{max\_display\_size}){}$\1\5
${}\\{printf}(\.{"\%\%\%d=large"},\39\|k);{}$\2\6
\&{else}\1\6
\&{for} ${}(\|h\K(\|p\?\\{height}[\\{count}]:\T{0});{}$ ${}\|h\G\T{0};{}$ ${}%
\|h\MM){}$\5
${}\{{}$\1\6
\&{if} ${}(\|h\E\T{0}){}$\1\5
${}\\{printf}(\.{"\%\%\%d="},\39\|k);{}$\2\6
${}\\{col}\K(\|h\E\T{0}\?\\{dwidth}(\|k)+\T{2}:\T{0});{}$\6
${}\\{print\_rep}(\|p,\39\T{0},\39\\{dwidth}(\|k)+\T{2});{}$\6
\&{if} (\|h)\1\5
\\{printf}(\.{"\\n"});\2\6
\&{else} \&{if} (\\{showing\_size})\5
${}\{{}$\1\6
\&{int} \|c${}\K\\{dwidth}(\|k)+\T{2}+\\{width}[\\{count}];{}$\7
\\{align\_to}(\|c);\6
\4${}\}{}$\2\6
\4${}\}{}$\2\2\6
\&{if} (\\{showing\_size})\1\5
${}\\{printf}(\.{"\ (\%d\ nodes)\\n"},\39\\{count});{}$\2\6
\&{else}\1\5
\\{printf}(\.{"\\n"});\2\par
\U30.\fi

\N{1}{60}Debugging. Finally, here are some quick-and-dirty routines
that might be useful while I'm debugging.

The \PB{\\{eval}} routine, which is invoked only by the debugger, computes
$x_l\star x_r$ at every node of a possibly abnormal tree, and leaves
these values in the \PB{\\{val}} fields. It also returns the value of the
whole tree.

\Y\B\4\X13:Basic subroutines\X${}\mathrel+\E{}$\6
\&{int} \\{eval}(\|p)\C{ fills the \PB{\\{val}} fields of nodes }\1\1\6
\&{node} ${}{*}\|p;\2\2{}$\6
${}\{{}$\1\6
\&{register} \&{int} \\{lv}${},{}$ \\{rv};\7
\&{if} ${}(\R\|p){}$\1\5
\&{return} \T{0};\2\6
${}\\{lv}\K\\{eval}(\|p\MG\|l);{}$\6
${}\\{rv}\K\\{eval}(\|p\MG\|r);{}$\6
${}\|p\MG\\{val}\K(\\{lv}\Z\\{lg\_large}\?\T{1}\LL\\{lv}:\\{large})+\\{rv};{}$\6
\&{if} ${}(\|p\MG\\{val}>\\{large}){}$\1\5
${}\|p\MG\\{val}\K\\{large};{}$\2\6
\&{return} \|p${}\MG\\{val};{}$\6
\4${}\}{}$\2\par
\fi

\M{61}The next routine is used to check that I've recycled all the nodes.
I could take it out, now that the program appears to work; but what the
heck, this isn't a production program.

\Y\B\4\X61:Check that the saved trees account for all the \PB{\\{used}} nodes%
\X${}\E{}$\6
$\PP\\{time\_stamp};{}$\6
${}\\{count}\K\T{0};{}$\6
\&{for} ${}(\|k\K\T{0};{}$ ${}\|k\Z\\{save\_ptr};{}$ ${}\|k\PP){}$\1\5
\\{stamp}(\\{saved}[\|k]);\2\6
\&{if} ${}(\\{count}\I\\{used}){}$\1\5
${}\\{printf}(\.{"We\ lost\ track\ of\ \%d}\)\.{\ nodes!\\n"},\39\\{used}-%
\\{count}){}$;\2\par
\U3.\fi

\M{62}\B\X13:Basic subroutines\X${}\mathrel+\E{}$\6
\&{void} \\{stamp}(\|p)\C{ stamp all nodes of \PB{\|p} with \PB{\\{time%
\_stamp}}, and count them }\1\1\6
\&{node} ${}{*}\|p;\2\2{}$\6
${}\{{}$\1\6
\&{if} ${}(\R\|p){}$\1\5
\&{return};\2\6
${}\\{stamp}(\|p\MG\|l);{}$\6
${}\\{stamp}(\|p\MG\|r);{}$\6
\&{if} ${}(\|p\MG\\{val}\E\\{time\_stamp}){}$\1\5
\\{printf}(\.{"***Node\ overlap!!\\n}\)\.{"});\2\6
${}\|p\MG\\{val}\K\\{time\_stamp};{}$\6
${}\\{count}\PP;{}$\6
\4${}\}{}$\2\par
\fi

\M{63}\B\X2:Global variables\X${}\mathrel+\E{}$\6
\&{int} \\{time\_stamp}${}\K\\{large}{}$;\C{ unique number }\par
\fi

\N{1}{64}Index.
\fi

\inx
\fin
\con
