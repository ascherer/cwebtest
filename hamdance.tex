\input cwebmac
\datethis



\N{1}{1}Introduction. This little program finds all the Hamiltonian circuits
of a given graph, using an interesting algorithm that illustrates
the technique of ``dancing links'' [see my paper in {\sl Millennial
Perspectives in Computer Science}, edited by Jim Davies, Bill Roscoe,
and Jim Woodcock (Houndmills, Basingstoke, Hampshire:\ Palgrave, 2000),
187--214]. The idea is to allow long paths to
grow in segments that gradually merge together, instead of to build such
paths strictly in order from beginning to end. At each stage in the decision
process, certain edges have been chosen to be in the final circuit, with no
three touching any vertex; we repeatedly choose further edges, preserving this
condition while not completing any cycles that are too short.

\Y\B\8\#\&{include} \.{"gb\_graph.h"}\C{ use the Stanford GraphBase conventions
}\6
\8\#\&{include} \.{"gb\_save.h"}\C{ and its routine for inputting graphs }\6
\ATH\6
\X3:Global variables\X\7
\&{Graph} ${}{*}\|g{}$;\C{ the given graph }\7
\X5:Subroutines\X\7
\&{int} \\{main}(\&{int} \\{argc}${},\39{}$\&{char} ${}{*}\\{argv}[\,]){}$\1\1%
\2\2\6
${}\{{}$\1\6
\&{register} \&{Vertex} ${}{*}\|u,{}$ ${}{*}\|v,{}$ ${}{*}\|w;{}$\6
\&{register} \&{Arc} ${}{*}\|a;{}$\6
\&{int} \|k${},{}$ \|d;\7
\X2:Process the command line, inputting the graph\X;\6
\X9:Prepare the graph for backtracking\X;\6
\X14:Backtrack through all solutions\X;\6
\X13:Print the results\X;\6
\\{exit}(\T{0});\6
\4${}\}{}$\2\par
\fi

\M{2}The given graph should be in Stanford GraphBase format, in a file
like \PB{\.{"foo.gb"}} named on the command line. This file name can optionally
be followed by a modulus \PB{\|m}, which causes every $\vert m\vert$th solution
to be printed.
If a third command line argument appears, the output will be extremely verbose.

The modulus \PB{\|m} might be negative; this indicates that solutions should be
printed showing edges in the order they were discovered, rather than in the
natural circuit order.

\Y\B\4\D$\\{max\_n}$ \5
\T{100}\C{ our arrays will accommodate this many vertices at most }\par
\B\4\D$\\{infty}$ \5
\T{1000000000}\C{ infinity (approximately) }\par
\Y\B\4\X2:Process the command line, inputting the graph\X${}\E{}$\6
\&{if} ${}(\\{argc}>\T{1}){}$\1\5
${}\|g\K\\{restore\_graph}(\\{argv}[\T{1}]){}$;\5
\2\&{else}\1\5
${}\|g\K\NULL;{}$\2\6
\&{if} ${}(\\{argc}<\T{3}\V\\{sscanf}(\\{argv}[\T{2}],\39\.{"\%d"},\39{\AND}%
\\{modulus})\I\T{1}){}$\1\5
${}\\{modulus}\K\\{infty};{}$\2\6
\&{if} ${}(\R\|g\V\\{modulus}\E\T{0}){}$\5
${}\{{}$\1\6
${}\\{fprintf}(\\{stderr},\39\.{"Usage:\ \%s\ foo.gb\ [[}\)\.{-]modulus]\
[verbose]}\)\.{\\n"},\39\\{argv}[\T{0}]);{}$\6
${}\\{exit}({-}\T{1});{}$\6
\4${}\}{}$\2\6
\&{if} ${}(\|g\MG\|n>\\{max\_n}){}$\5
${}\{{}$\1\6
${}\\{fprintf}(\\{stderr},\39\.{"Sorry,\ I'm\ set\ up\ t}\)\.{o\ handle\ at\
most\ \%d\ }\)\.{vertices!\\n"},\39\\{max\_n});{}$\6
${}\\{exit}({-}\T{2});{}$\6
\4${}\}{}$\2\6
\&{if} ${}(\\{argc}>\T{3}){}$\1\5
${}\\{verbose}\K\T{1}{}$;\2\par
\U1.\fi

\M{3}The \PB{\\{verbose}} variable is declared in \.{gb\_graph.h}.

\Y\B\4\X3:Global variables\X${}\E{}$\6
\&{int} \\{modulus};\C{ how often we should show solutions }\par
\As4, 8, 10, 11\ETs35.
\U1.\fi

\N{1}{4}Data structures. Each vertex is either \PB{\\{bare}} (touching none of
the chosen
edges) or \PB{\\{outer}} (touching just one) or \PB{\\{inner}} (touching two).
An \PB{\\{outer}}
vertex has a \PB{\\{mate}}, which is the vertex at the other end of the path of
chosen vertices that it belongs to.  All nonchosen edges that touch inner
vertices have effectively been removed from the graph. Any edge that runs from
a vertex to its mate has also effectively been removed.

The degree \PB{\\{deg}} of a \PB{\\{bare}} or \PB{\\{outer}} vertex is the
number of edges that
currently touch it. All vertices begin \PB{\\{bare}} and end up \PB{\\{inner}}.
A bare
vertex of degree~2 is converted to an inner vertex, since its two edges
must be in the final circuit; this mechanism causes \PB{\\{outer}} vertices to
spring up more or less spontaneously, and it helps in the decision-making.
At moments when all bare vertices have degree~3 or more, we choose an
end vertex of minimum degree, and make it inner in all possible ways.

The main data structure is a doubly linked list of all the \PB{\\{outer}}
vertices.
Links in this list are called \PB{\\{llink}} and \PB{\\{rlink}}. When a vertex
is
removed from the list, its \PB{\\{llink}} and \PB{\\{rlink}} retain important
information
about how to undo this operation when backtracking; this idea makes
the links ``dance.'' Similarly, when an \PB{\\{outer}} vertex becomes \PB{%
\\{inner}},
its \PB{\\{mate}} field retains the name of its former mate, so that we needn't
recompute mates when undoing previous changes to the data structures.

The \PB{\\{mate}} field of a vertex that was promoted directly from \PB{%
\\{bare}} to
\PB{\\{inner}} is one of its two neighbors. The other neighbor is stored
in another field called \PB{\\{comate}}.

Utility fields \PB{\|u}, \PB{\|v}, \PB{\|w}, \PB{\|x}, \PB{\|y}, and \PB{\|z}
of a \PB{\&{Vertex}}
are used to hold the \PB{\\{type}}, \PB{\\{deg}}, \PB{\\{llink}}, \PB{%
\\{rlink}}, \PB{\\{mate}}, and \PB{\\{comate}}.

\Y\B\4\D$\\{bare}$ \5
\T{2}\par
\B\4\D$\\{outer}$ \5
\T{1}\par
\B\4\D$\\{inner}$ \5
\T{0}\par
\B\4\D$\\{type}$ \5
$\|u.{}$\|I\C{ either \PB{\\{bare}}, \PB{\\{outer}}, or \PB{\\{inner}} }\par
\B\4\D$\\{deg}$ \5
$\|v.{}$\|I\C{ current degree, for non-\PB{\\{inner}} vertices }\par
\B\4\D$\\{llink}$ \5
$\|w.{}$\|V\C{ link to the left in the basic list }\par
\B\4\D$\\{rlink}$ \5
$\|x.{}$\|V\C{ link to the right in the basic list }\par
\B\4\D$\\{mate}$ \5
$\|y.{}$\|V\C{ the mate of an \PB{\\{outer}} vertex }\par
\B\4\D$\\{comate}$ \5
$\|z.{}$\|V\C{ neighbor of fast-promoted \PB{\\{inner}} vertex }\par
\B\4\D$\\{head}$ \5
$({\AND}\\{list\_head}{}$)\par
\Y\B\4\X3:Global variables\X${}\mathrel+\E{}$\6
\&{Vertex} \\{list\_head};\C{ the doubly linked list starts here }\6
\&{char} ${}{*}\\{decode}[\T{3}]\K\{\.{"inner"},\39\.{"outer"},\39\.{"bare"}%
\}{}$;\par
\fi

\M{5}Here's a routine that should be useful for debugging: It displays
the fields of a given vertex symbolically.

\Y\B\4\X5:Subroutines\X${}\E{}$\6
\&{void} \\{print\_vert}(\&{Vertex} ${}{*}\|v){}$\1\1\2\2\6
${}\{{}$\1\6
${}\\{printf}(\.{"\%s:\ \%s,\ deg=\%d"},\39\|v\MG\\{name},\39\\{decode}[\|v\MG%
\\{type}],\39\|v\MG\\{deg});{}$\6
\&{if} ${}(\|v\MG\\{llink}){}$\1\5
${}\\{printf}(\.{",\ llink=\%s"},\39\|v\MG\\{llink}\MG\\{name});{}$\2\6
\&{if} ${}(\|v\MG\\{rlink}){}$\1\5
${}\\{printf}(\.{",\ rlink=\%s"},\39\|v\MG\\{rlink}\MG\\{name});{}$\2\6
\&{if} ${}(\|v\MG\\{mate}){}$\1\5
${}\\{printf}(\.{",\ mate=\%s"},\39\|v\MG\\{mate}\MG\\{name});{}$\2\6
\&{if} ${}(\|v\MG\\{comate}){}$\1\5
${}\\{printf}(\.{",\ comate=\%s"},\39\|v\MG\\{comate}\MG\\{name});{}$\2\6
\\{printf}(\.{"\\n"});\6
\4${}\}{}$\2\par
\As6, 7\ETs12.
\U1.\fi

\M{6}And if we want to see them all:

\Y\B\4\X5:Subroutines\X${}\mathrel+\E{}$\6
\&{void} \\{print\_verts}(\,)\1\1\2\2\6
${}\{{}$\1\6
\&{register} \&{Vertex} ${}{*}\|v;{}$\7
\&{for} ${}(\|v\K\|g\MG\\{vertices};{}$ ${}\|v<\|g\MG\\{vertices}+\|g\MG\|n;{}$
${}\|v\PP){}$\1\5
\\{print\_vert}(\|v);\2\6
\4${}\}{}$\2\par
\fi

\M{7}Even more important for debugging is the \PB{\\{sanity\_check}} routine,
which painstakingly makes sure that I haven't let the data structure get
out of sync with itself. (Vertex \PB{\\{vv}} is either \PB{$\NULL$} or an \PB{%
\\{inner}}
vertex whose mate is currently \PB{\\{outer}}. In the latter case, some of
the sanity checks are not made.)

\Y\B\4\X5:Subroutines\X${}\mathrel+\E{}$\6
\&{void} \\{sanity\_check}(\&{Vertex} ${}{*}\\{vv}){}$\1\1\2\2\6
${}\{{}$\1\6
\&{register} \&{Vertex} ${}{*}\|u,{}$ ${}{*}\|v,{}$ ${}{*}\|w;{}$\6
\&{register} \&{Arc} ${}{*}\|a;{}$\6
\&{register} \&{int} \|c${},{}$ \|d;\7
\&{for} ${}(\|v\K\|g\MG\\{vertices},\39\|c\K\T{0};{}$ ${}\|v<\|g\MG%
\\{vertices}+\|g\MG\|n;{}$ ${}\|v\PP){}$\5
${}\{{}$\1\6
${}\|w\K\|v\MG\\{mate};{}$\6
\&{if} ${}(\|v\MG\\{type}\E\\{bare}\W\|w\I\NULL){}$\1\5
${}\\{printf}(\.{"Bare\ vertex\ \%s\ shou}\)\.{ldn't\ have\ mate\ \%s!\\}\)%
\.{n"},\39\|v\MG\\{name},\39\|w\MG\\{name});{}$\2\6
\&{if} ${}(\|v\MG\\{type}\E\\{outer}){}$\1\5
${}\|c\PP;{}$\2\6
\&{if} ${}(\|v\MG\\{type}\E\\{outer}\W(\|w\MG\\{mate}\I\|v\V\|w\MG\\{type}\I%
\\{outer})){}$\1\6
\&{if} ${}(\|w\I\\{vv}\V\|w\MG\\{type}\I\\{inner}){}$\1\5
${}\\{printf}(\.{"Outer\ vertex\ \%s\ has}\)\.{\ mate\ problem\ vis-a-}\)\.{vis%
\ \%s!\\n"},\39\|v\MG\\{name},\39\|w\MG\\{name});{}$\2\2\6
\&{for} ${}(\|a\K\|v\MG\\{arcs},\39\|d\K\T{0};{}$ \|a; ${}\|a\K\|a\MG%
\\{next}){}$\5
${}\{{}$\1\6
${}\|u\K\|a\MG\\{tip};{}$\6
\&{if} ${}(\|u\MG\\{type}\I\\{inner}\W\|u\I\|w){}$\1\5
${}\|d\PP;{}$\2\6
\4${}\}{}$\2\6
\&{if} ${}(\|v\MG\\{type}\I\\{inner}\W\|v\MG\\{deg}\I\|d\W\\{ocount}\I\|g\MG%
\|n-\T{1}){}$\1\5
${}\\{printf}(\.{"Vertex\ \%s\ should\ ha}\)\.{ve\ degree\ \%d,\ not\ \%d}\)%
\.{!\\n"},\39\|v\MG\\{name},\39\|d,\39\|v\MG\\{deg});{}$\2\6
\&{if} ${}(\|v\MG\\{type}\E\\{bare}\W\|d<\T{3}\W\\{vv}\E\NULL){}$\1\5
${}\\{printf}(\.{"Vertex\ \%s\ (degree\ \%}\)\.{d)\ should\ not\ be\ bar}\)%
\.{e!\\n"},\39\|v\MG\\{name},\39\|d);{}$\2\6
\4${}\}{}$\2\6
\&{for} ${}(\|v\K\\{head}\MG\\{rlink};{}$ ${}\|c>\T{0};{}$ ${}\|c\MM,\39\|v\K%
\|v\MG\\{rlink}){}$\5
${}\{{}$\1\6
\&{if} ${}(\|v\MG\\{type}\I\\{outer}){}$\1\5
${}\\{printf}(\.{"Vertex\ \%s\ (\%s)\ shou}\)\.{ldn't\ be\ in\ the\ list}\)\.{!%
\\n"},\39\|v\MG\\{name},\39\\{decode}[\|v\MG\\{type}]);{}$\2\6
\&{if} ${}(\|v\MG\\{llink}\MG\\{rlink}\I\|v\V\|v\MG\\{rlink}\MG\\{llink}\I%
\|v){}$\1\5
${}\\{printf}(\.{"Double-link\ failure}\)\.{\ at\ vertex\ \%s!\\n"},\39\|v\MG%
\\{name});{}$\2\6
\4${}\}{}$\2\6
\&{if} ${}(\|v\I\\{head}){}$\1\5
\\{printf}(\.{"The\ list\ doesn't\ co}\)\.{ntain\ all\ the\ outer\ }\)%
\.{vertices!\\n"});\2\6
\4${}\}{}$\2\par
\fi

\M{8}The next most interesting data structure is the \PB{\\{barelist}},
which receives the names of \PB{\\{bare}} vertices at the moment their
degree drops to~2. Such vertices must be clothed before we advance
to a new level of backtracking.

\Y\B\4\X3:Global variables\X${}\mathrel+\E{}$\6
\&{Vertex} ${}{*}\\{barelist}[\\{max\_n}];{}$\6
\&{int} \\{bcount};\C{ the current number of entries in \PB{\\{barelist}} }\6
\&{int} \\{curb}[\\{max\_n}];\C{ value of \PB{\\{bcount}} at the beginning of
each level }\6
\&{int} \\{curbb}[\\{max\_n}];\C{ value of \PB{\\{bcount}} in mid-level }\6
\&{Vertex} ${}{*}\\{bareback}[\\{max\_n}]{}$;\C{ used for undoing \PB{%
\\{barelist}} manipulations }\par
\fi

\M{9}\B\X9:Prepare the graph for backtracking\X${}\E{}$\6
$\|d\K\\{infty};{}$\6
${}\\{bcount}\K\\{ocount}\K\T{0};{}$\6
\&{for} ${}(\|v\K\|g\MG\\{vertices};{}$ ${}\|v<\|g\MG\\{vertices}+\|g\MG\|n;{}$
${}\|v\PP){}$\5
${}\{{}$\1\6
${}\|v\MG\\{type}\K\\{bare};{}$\6
\&{for} ${}(\|a\K\|v\MG\\{arcs},\39\|k\K\T{0};{}$ \|a; ${}\|a\K\|a\MG%
\\{next}){}$\1\5
${}\|k\PP;{}$\2\6
${}\|v\MG\\{deg}\K\|k;{}$\6
\&{if} ${}(\|k\E\T{2}){}$\1\5
${}\\{barelist}[\\{bcount}\PP]\K\|v;{}$\2\6
\&{if} ${}(\|k<\|d){}$\1\5
${}\|d\K\|k,\39\\{curv}[\T{0}]\K\|v;{}$\2\6
${}\|v\MG\\{llink}\K\|v\MG\\{rlink}\K\|v\MG\\{mate}\K\|v\MG\\{comate}\K%
\NULL;{}$\6
\4${}\}{}$\2\6
${}\\{head}\MG\\{rlink}\K\\{head}\MG\\{llink}\K\\{head};{}$\6
${}\\{head}\MG\\{name}\K\.{"head"};{}$\6
\&{if} ${}(\|d<\T{2}){}$\5
${}\{{}$\1\6
${}\\{printf}(\.{"There\ are\ no\ Hamilt}\)\.{onian\ circuits,\ beca}\)\.{use\ %
\%s\ has\ degree\ \%d}\)\.{!\\n"},\39\\{curv}[\T{0}]\MG\\{name},\39\|d);{}$\6
\\{exit}(\T{0});\6
\4${}\}{}$\2\par
\U1.\fi

\M{10}The arcs currently chosen appear in lists called \PB{\\{source}} and \PB{%
\\{dest}}.
Some arcs are chosen when a bare vertex is being clothed; others are
chosen at a level of backtracking when an outer vertex becomes inner.

\Y\B\4\X3:Global variables\X${}\mathrel+\E{}$\6
\&{Vertex} ${}{*}\\{source}[\\{max\_n}],{}$ ${}{*}\\{dest}[\\{max\_n}]{}$;\C{
the answers }\6
\&{int} \\{ocount};\C{ the current number of entries in \PB{\\{source}} and %
\PB{\\{dest}} }\6
\&{int} \\{curo}[\\{max\_n}];\C{ value of \PB{\\{ocount}} at the beginning of
each level }\par
\fi

\M{11}Finally, a few other minor structures help us with backtracking or
when we want to assess the progress of a potentially long calculation.

\Y\B\4\X3:Global variables\X${}\mathrel+\E{}$\6
\&{Vertex} ${}{*}\\{curv}[\\{max\_n}]{}$;\C{ outer vertex chosen for branching
}\6
\&{Arc} ${}{*}\\{cura}[\\{max\_n}]{}$;\C{ edge chosen for branching }\6
\&{int} \\{curi}[\\{max\_n}];\C{ index of the choice }\6
\&{int} \\{maxi}[\\{max\_n}];\C{ total number of choices }\6
\&{int} \\{profile}[\\{max\_n}];\C{ number of times we reached this level }\6
\&{int} \|l;\C{ the current level of backtracking }\6
\&{int} \\{maxl};\C{ the largest \PB{\|l} seen so far }\6
\&{unsigned} \&{int} \\{total};\C{ this many solutions so far }\par
\fi

\M{12}Hamiltonian path problems often take a long time. The following
subroutine can be called with an online debugger, to assess how
far the work has progressed.

\Y\B\4\X5:Subroutines\X${}\mathrel+\E{}$\6
\&{void} \\{print\_state}(\,)\1\1\2\2\6
${}\{{}$\1\6
\&{register} \&{int} \|i${},{}$ \|j${},{}$ \|k;\7
\&{for} ${}(\|j\K\|k\K\T{0};{}$ ${}\|k\Z\|l;{}$ ${}\|j\PP,\39\|k\PP){}$\5
${}\{{}$\1\6
\&{while} ${}(\|j<\\{curo}[\|k]){}$\5
${}\{{}$\1\6
${}\\{printf}(\.{"\ \ \ \ \ \ \%s--\%s\\n"},\39\\{source}[\|j]\MG\\{name},\39%
\\{dest}[\|j]\MG\\{name});{}$\6
${}\|j\PP;{}$\6
\4${}\}{}$\2\6
\&{if} (\|k)\5
${}\{{}$\1\6
\&{if} ${}(\|j<\|g\MG\|n){}$\1\5
${}\\{printf}(\.{"\ \%3d:\ \%s--\%s\ (\%d\ of}\)\.{\ \%d)\\n"},\39\|k,\39%
\\{source}[\|j]\MG\\{name},\39\\{dest}[\|j]\MG\\{name},\39\\{curi}[\|k],\39%
\\{maxi}[\|k]);{}$\2\6
\4${}\}{}$\5
\2\&{else}\1\5
\X39:Print the state line for the bottom level\X;\2\6
\4${}\}{}$\2\6
\4${}\}{}$\2\par
\fi

\M{13}\B\X13:Print the results\X${}\E{}$\6
$\\{printf}(\.{"Altogether\ \%u\ solut}\)\.{ions.\\n"},\39\\{total});{}$\6
\&{if} (\\{verbose})\5
${}\{{}$\1\6
\&{for} ${}(\|k\K\T{1};{}$ ${}\|k\Z\\{maxl};{}$ ${}\|k\PP){}$\1\5
${}\\{printf}(\.{"\%3d:\ \%d\\n"},\39\|k,\39\\{profile}[\|k]);{}$\2\6
\4${}\}{}$\2\par
\U1.\fi

\N{1}{14}Marching forward. Here we follow the usual pattern of a backtrack
process
(and I follow my usual practice of \PB{\&{goto}}-ing). In this particular case
it's a bit tricky to get the whole process started, so I'm deferring
that bootstrap calculation until the program for levels \PB{$\|l\G\T{1}$} is in
place and
understood.

\Y\B\4\X14:Backtrack through all solutions\X${}\E{}$\6
\X36:Bootstrap the backtrack process\X;\6
\4\\{advance}:\5
\X18:Clothe everything on the bare list\X;\C{ here I said \PB{$\\{sanity%
\_check}(\NULL)$} when debugging }\6
${}\|l\PP;{}$\6
\&{if} (\\{verbose})\5
${}\{{}$\1\6
\&{if} ${}(\|l>\\{maxl}){}$\1\5
${}\\{maxl}\K\|l;{}$\2\6
${}\\{printf}(\.{"Entering\ level\ \%d:"},\39\|l);{}$\6
${}\\{profile}[\|l]\PP;{}$\6
\4${}\}{}$\2\6
\&{if} ${}(\\{ocount}\G\|g\MG\|n-\T{1}){}$\1\5
\X32:Check for solution and \PB{\&{goto} \\{backup}}\X;\2\6
\X15:Choose an outer vertex \PB{\|v} of minimum degree \PB{\|d}\X;\6
\&{if} (\\{verbose})\1\5
${}\\{printf}(\.{"\ choosing\ \%s(\%d)\\n"},\39\|v\MG\\{name},\39\|d);{}$\2\6
\&{if} ${}(\|d\E\T{0}){}$\1\5
\&{goto} \\{backup};\2\6
${}\\{curv}[\|l]\K\|v,\39\\{curi}[\|l]\K\T{1},\39\\{maxi}[\|l]\K\|d,\39%
\\{curb}[\|l]\K\\{bcount},\39\\{curo}[\|l]\K\\{ocount};{}$\6
${}\\{source}[\\{ocount}]\K\|v;{}$\6
${}\|w\K\|v\MG\\{mate};{}$\6
\X16:Promote \PB{\|v} from \PB{\\{outer}} to \PB{\\{inner}}\X;\6
${}\|a\K\|v\MG\\{arcs};{}$\6
\4\\{try\_move}:\5
\&{for} ( ;  ; ${}\|a\K\|a\MG\\{next}){}$\5
${}\{{}$\1\6
${}\|u\K\|a\MG\\{tip};{}$\6
\&{if} ${}(\|u\MG\\{type}\I\\{inner}\W\|u\I\|w){}$\1\5
\&{break};\2\6
\4${}\}{}$\2\6
${}\\{cura}[\|l]\K\|a;{}$\6
\X17:Update data structures to account for choosing edge \PB{\\{cura}[\|l]}\X;\6
\&{goto} \\{advance};\6
\4\\{backup}:\5
${}\|l\MM;{}$\6
\&{if} (\\{verbose})\1\5
${}\\{printf}(\.{"\ back\ to\ level\ \%d:\\}\)\.{n"},\39\|l);{}$\2\6
\X25:Unclothe everything clothed on level \PB{\|l}\X;\6
\&{if} (\|l)\5
${}\{{}$\1\6
\X30:Downdate data structures to deaccount for choosing edge \PB{\\{cura}[\|l]}%
\X;\C{ here I said \PB{\\{sanity\_check}(\|v)} when debugging }\6
\&{if} ${}(\\{curi}[\|l]<\\{maxi}[\|l]){}$\5
${}\{{}$\1\6
${}\\{curi}[\|l]\PP;{}$\6
${}\|w\K\|v\MG\\{mate}{}$;\5
${}\|a\K\\{cura}[\|l]\MG\\{next};{}$\6
\&{goto} \\{try\_move};\6
\4${}\}{}$\2\6
\X31:Demote \PB{\|v} from \PB{\\{inner}} to \PB{\\{outer}}\X;\6
\&{if} ${}(\|l>\T{1}){}$\1\5
\&{goto} \\{backup};\2\6
\4${}\}{}$\2\6
\X38:Advance at bottom level\X;\par
\U1.\fi

\M{15}All the outer vertices are in the doubly linked list, and it
is not empty.

\Y\B\4\X15:Choose an outer vertex \PB{\|v} of minimum degree \PB{\|d}\X${}\E{}$%
\6
\&{for} ${}(\|u\K\\{head}\MG\\{rlink},\39\|d\K\\{infty};{}$ ${}\|u\I%
\\{head};{}$ ${}\|u\K\|u\MG\\{rlink}){}$\5
${}\{{}$\1\6
\&{if} (\\{verbose})\1\5
${}\\{printf}(\.{"\ \%s(\%d)"},\39\|u\MG\\{name},\39\|u\MG\\{deg});{}$\2\6
\&{if} ${}(\|u\MG\\{deg}<\|d){}$\1\5
${}\|d\K\|u\MG\\{deg},\39\|v\K\|u;{}$\2\6
\4${}\}{}$\2\par
\U14.\fi

\M{16}At the beginning of a level, when we're about to choose a
neighbor for the outer vertex \PB{\|v}, we convert \PB{\|v} to \PB{\\{inner}}
type
because this conversion will be valid regardless of which edge we choose.

\Y\B\4\D$\\{dancing\_delete}(\|u)$ \5
$\|u\MG\\{llink}\MG\\{rlink}\K\|u\MG\\{rlink},\39\|u\MG\\{rlink}\MG\\{llink}\K%
\|u\MG{}$\\{llink}\par
\B\4\D$\\{decrease\_deg}(\|u,\|w)$ \6
\&{if} ${}(\|u\MG\\{type}\E\\{bare}){}$\5
${}\{{}$\1\6
${}\|u\MG\\{deg}\MM;{}$\6
\&{if} ${}(\|u\MG\\{deg}\E\T{2}){}$\1\5
${}\\{barelist}[\\{bcount}\PP]\K\|u;{}$\2\6
\4${}\}{}$\5
\2\&{else} \&{if} ${}(\|u\I\|w)$ $\|u\MG\\{deg}\MM{}$\C{ \PB{\|u} is an \PB{%
\\{outer}} neighbor of \PB{\|v} with \PB{$\|v\MG\\{mate}\K\|w$} }\par
\Y\B\4\X16:Promote \PB{\|v} from \PB{\\{outer}} to \PB{\\{inner}}\X${}\E{}$\6
\&{for} ${}(\|a\K\|v\MG\\{arcs};{}$ \|a; ${}\|a\K\|a\MG\\{next}){}$\5
${}\{{}$\1\6
${}\|u\K\|a\MG\\{tip};{}$\6
\&{if} ${}(\|u\MG\\{type}>\\{inner}){}$\1\5
${}\\{decrease\_deg}(\|u,\39\|w);{}$\2\6
\4${}\}{}$\2\6
${}\|v\MG\\{type}\K\\{inner};{}$\6
\\{dancing\_delete}(\|v);\6
${}\\{curbb}[\|l]\K\\{bcount}{}$;\par
\U14.\fi

\M{17}At this point, \PB{\|v} is a formerly outer vertex that we're joining to
vertex~\PB{\|u}. Also, \PB{$\|w\K\|v\MG\\{mate}$}.

If \PB{\|u} is type \PB{\\{outer}}, we're joining two segments into one, making
\PB{\|u} of type \PB{\\{inner}}.
But if \PB{\|u} is bare, we're lengthening a segment, and \PB{\|u} becomes \PB{%
\\{outer}}.

\Y\B\4\D$\\{make\_outer}(\|u)$ \6
${}\{{}$\1\6
${}\|u\MG\\{rlink}\K\\{head}\MG\\{rlink},\39\\{head}\MG\\{rlink}\MG\\{llink}\K%
\|u;{}$\6
${}\|u\MG\\{llink}\K\\{head},\39\\{head}\MG\\{rlink}\K\|u;{}$\6
${}\|u\MG\\{type}\K\\{outer};{}$\6
\4${}\}{}$\2\par
\B\4\D$\\{vprint}()$ \5
\&{if} (\\{verbose}) $\\{printf}(\.{"\ \%s--\%s\\n"},\39\\{source}[\\{ocount}-%
\T{1}]\MG\\{name},\39\\{dest}[\\{ocount}-\T{1}]\MG\\{name}{}$)\par
\Y\B\4\X17:Update data structures to account for choosing edge \PB{\\{cura}[%
\|l]}\X${}\E{}$\6
$\\{dest}[\\{ocount}\PP]\K\|u{}$;\5
\\{vprint}(\,);\6
\&{if} ${}(\|u\MG\\{type}\E\\{outer}){}$\5
${}\{{}$\1\6
\&{for} ${}(\|a\K\|w\MG\\{arcs};{}$ \|a; ${}\|a\K\|a\MG\\{next}){}$\1\6
\&{if} ${}(\|a\MG\\{tip}\E\|u\MG\\{mate}){}$\5
${}\{{}$\1\6
${}\|u\MG\\{mate}\MG\\{deg}\MM,\39\|w\MG\\{deg}\MM;{}$\6
\&{break};\6
\4${}\}{}$\2\2\6
${}\|w\MG\\{mate}\K\|u\MG\\{mate},\39\|u\MG\\{mate}\MG\\{mate}\K\|w;{}$\6
\\{dancing\_delete}(\|u);\6
${}\|u\MG\\{type}\K\\{inner};{}$\6
\&{for} ${}(\|a\K\|u\MG\\{arcs};{}$ \|a; ${}\|a\K\|a\MG\\{next}){}$\5
${}\{{}$\1\6
${}\|w\K\|a\MG\\{tip};{}$\6
\&{if} ${}(\|w\MG\\{type}>\\{inner}){}$\1\5
${}\\{decrease\_deg}(\|w,\39\|u\MG\\{mate});{}$\2\6
\4${}\}{}$\2\6
\4${}\}{}$\5
\2\&{else}\5
${}\{{}$\C{ \PB{$\|u\MG\\{type}\E\\{bare}$} }\1\6
\&{for} ${}(\|a\K\|w\MG\\{arcs};{}$ \|a; ${}\|a\K\|a\MG\\{next}){}$\1\6
\&{if} ${}(\|a\MG\\{tip}\E\|u){}$\5
${}\{{}$\1\6
${}\|u\MG\\{deg}\MM,\39\|w\MG\\{deg}\MM;{}$\6
\&{break};\6
\4${}\}{}$\2\2\6
${}\|w\MG\\{mate}\K\|u,\39\|u\MG\\{mate}\K\|w;{}$\6
\\{make\_outer}(\|u);\6
\4${}\}{}$\2\par
\U14.\fi

\M{18}The situation might have changed since a vertex entered the bare list,
because its type and/or degree may have been altered.

Also, giving clothes to one bare vertex might have a ripple effect, causing
other vertices to enter the bare list. The value of \PB{\\{bcount}} in the
following
loop might therefore be a moving target.

One case needs to handled with special care: If the two neighbors of \PB{\|v}
are mates of each other, we are forced to complete a cycle. This is
legitimate only if the cycle includes all vertices.

\Y\B\4\X18:Clothe everything on the bare list\X${}\E{}$\6
\&{for} ${}(\|k\K\\{curb}[\|l];{}$ ${}\|k<\\{bcount};{}$ ${}\|k\PP){}$\5
${}\{{}$\1\6
${}\|v\K\\{barelist}[\|k];{}$\6
\&{if} ${}(\|v\MG\\{type}\I\\{bare}){}$\1\5
${}\\{bareback}[\|k]\K\|v,\39\\{barelist}[\|k]\K\NULL;{}$\2\6
\&{else}\5
${}\{{}$\1\6
\&{if} ${}(\|v\MG\\{deg}\I\T{2}){}$\5
${}\{{}$\1\6
\&{if} (\\{verbose})\1\5
\\{printf}(\.{"(oops,\ low\ degree;\ }\)\.{backing\ up)\\n"});\2\6
\&{goto} \\{emergency\_backup};\C{ see below }\6
\4${}\}{}$\2\6
\X19:Find the two neighbors, \PB{\|u} and \PB{\|w}, of vertex \PB{\|v}\X;\6
\&{if} ${}(\|u\MG\\{mate}\E\|w\W\\{ocount}\I\|g\MG\|n-\T{2}){}$\5
${}\{{}$\1\6
\&{if} (\\{verbose})\1\5
\\{printf}(\.{"(oops,\ short\ cycle;}\)\.{\ backing\ up)\\n"});\2\6
\&{goto} \\{emergency\_backup};\6
\4${}\}{}$\2\6
${}\|v\MG\\{mate}\K\|u,\39\|v\MG\\{comate}\K\|w;{}$\6
${}\|v\MG\\{type}\K\\{inner};{}$\6
${}\\{source}[\\{ocount}]\K\|u,\39\\{dest}[\\{ocount}\PP]\K\|v{}$;\5
\\{vprint}(\,);\6
${}\\{source}[\\{ocount}]\K\|v,\39\\{dest}[\\{ocount}\PP]\K\|w{}$;\5
\\{vprint}(\,);\6
\&{if} ${}(\|u\MG\\{type}\E\\{bare}){}$\1\6
\&{if} ${}(\|w\MG\\{type}\E\\{bare}){}$\1\5
\X20:Promote BBB to OIO\X\2\6
\&{else}\1\5
\X21:Promote BBO to OII\X\2\2\6
\&{else} \&{if} ${}(\|w\MG\\{type}\E\\{bare}){}$\1\5
\X22:Promote OBB to IIO\X\2\6
\&{else}\1\5
\X23:Promote OBO to III\X;\2\6
\4${}\}{}$\2\6
\4${}\}{}$\2\par
\U14.\fi

\M{19}\B\X19:Find the two neighbors, \PB{\|u} and \PB{\|w}, of vertex \PB{\|v}%
\X${}\E{}$\6
\&{for} ${}(\|a\K\|v\MG\\{arcs};{}$  ; ${}\|a\K\|a\MG\\{next}){}$\5
${}\{{}$\1\6
${}\|u\K\|a\MG\\{tip};{}$\6
\&{if} ${}(\|u\MG\\{type}\I\\{inner}){}$\1\5
\&{break};\2\6
\4${}\}{}$\2\6
\&{for} ${}(\|a\K\|a\MG\\{next};{}$  ; ${}\|a\K\|a\MG\\{next}){}$\5
${}\{{}$\1\6
${}\|w\K\|a\MG\\{tip};{}$\6
\&{if} ${}(\|w\MG\\{type}\I\\{inner}){}$\1\5
\&{break};\2\6
\4${}\}{}$\2\par
\U18.\fi

\M{20}The clothing process involves four similar subcases (which, I admit,
are slightly boring). We will see, however, that all of these manipulations
are easily undone; and that fact, to me, is interesting indeed, almost
climactic.

\Y\B\4\X20:Promote BBB to OIO\X${}\E{}$\6
${}\{{}$\1\6
${}\|u\MG\\{deg}\MM,\39\|w\MG\\{deg}\MM;{}$\6
\\{make\_outer}(\|u);\6
\\{make\_outer}(\|w);\6
${}\|u\MG\\{mate}\K\|w,\39\|w\MG\\{mate}\K\|u;{}$\6
\&{for} ${}(\|a\K\|u\MG\\{arcs};{}$ \|a; ${}\|a\K\|a\MG\\{next}){}$\1\6
\&{if} ${}(\|a\MG\\{tip}\E\|w){}$\5
${}\{{}$\1\6
${}\|u\MG\\{deg}\MM,\39\|w\MG\\{deg}\MM;{}$\6
\&{break};\6
\4${}\}{}$\2\2\6
\4${}\}{}$\2\par
\U18.\fi

\M{21}\B\X21:Promote BBO to OII\X${}\E{}$\6
${}\{{}$\1\6
${}\|u\MG\\{deg}\MM;{}$\6
\\{make\_outer}(\|u);\6
${}\|u\MG\\{mate}\K\|w\MG\\{mate},\39\|w\MG\\{mate}\MG\\{mate}\K\|u;{}$\6
\&{for} ${}(\|a\K\|u\MG\\{arcs};{}$ \|a; ${}\|a\K\|a\MG\\{next}){}$\1\6
\&{if} ${}(\|a\MG\\{tip}\E\|w\MG\\{mate}){}$\5
${}\{{}$\1\6
${}\|u\MG\\{deg}\MM,\39\|w\MG\\{mate}\MG\\{deg}\MM;{}$\6
\&{break};\6
\4${}\}{}$\2\2\6
\&{for} ${}(\|a\K\|w\MG\\{arcs};{}$ \|a; ${}\|a\K\|a\MG\\{next}){}$\5
${}\{{}$\1\6
${}\|v\K\|a\MG\\{tip};{}$\6
\&{if} ${}(\|v\MG\\{type}\I\\{inner}){}$\1\5
${}\\{decrease\_deg}(\|v,\39\|w\MG\\{mate});{}$\2\6
\4${}\}{}$\2\6
${}\|w\MG\\{type}\K\\{inner};{}$\6
\\{dancing\_delete}(\|w);\6
\4${}\}{}$\2\par
\U18.\fi

\M{22}\B\X22:Promote OBB to IIO\X${}\E{}$\6
${}\{{}$\1\6
${}\|w\MG\\{deg}\MM;{}$\6
\\{make\_outer}(\|w);\6
${}\|w\MG\\{mate}\K\|u\MG\\{mate},\39\|u\MG\\{mate}\MG\\{mate}\K\|w;{}$\6
\&{for} ${}(\|a\K\|w\MG\\{arcs};{}$ \|a; ${}\|a\K\|a\MG\\{next}){}$\1\6
\&{if} ${}(\|a\MG\\{tip}\E\|u\MG\\{mate}){}$\5
${}\{{}$\1\6
${}\|w\MG\\{deg}\MM,\39\|u\MG\\{mate}\MG\\{deg}\MM;{}$\6
\&{break};\6
\4${}\}{}$\2\2\6
\&{for} ${}(\|a\K\|u\MG\\{arcs};{}$ \|a; ${}\|a\K\|a\MG\\{next}){}$\5
${}\{{}$\1\6
${}\|v\K\|a\MG\\{tip};{}$\6
\&{if} ${}(\|v\MG\\{type}\I\\{inner}){}$\1\5
${}\\{decrease\_deg}(\|v,\39\|u\MG\\{mate});{}$\2\6
\4${}\}{}$\2\6
${}\|u\MG\\{type}\K\\{inner};{}$\6
\\{dancing\_delete}(\|u);\6
\4${}\}{}$\2\par
\U18.\fi

\M{23}\B\X23:Promote OBO to III\X${}\E{}$\6
${}\{{}$\1\6
\&{for} ${}(\|a\K\|u\MG\\{arcs};{}$ \|a; ${}\|a\K\|a\MG\\{next}){}$\5
${}\{{}$\1\6
${}\|v\K\|a\MG\\{tip};{}$\6
\&{if} ${}(\|v\MG\\{type}\I\\{inner}){}$\1\5
${}\\{decrease\_deg}(\|v,\39\|u\MG\\{mate});{}$\2\6
\4${}\}{}$\2\6
${}\|u\MG\\{type}\K\\{inner};{}$\6
\\{dancing\_delete}(\|u);\6
\&{for} ${}(\|a\K\|w\MG\\{arcs};{}$ \|a; ${}\|a\K\|a\MG\\{next}){}$\5
${}\{{}$\1\6
${}\|v\K\|a\MG\\{tip};{}$\6
\&{if} ${}(\|v\MG\\{type}\I\\{inner}){}$\1\5
${}\\{decrease\_deg}(\|v,\39\|w\MG\\{mate});{}$\2\6
\4${}\}{}$\2\6
${}\|w\MG\\{type}\K\\{inner};{}$\6
\\{dancing\_delete}(\|w);\6
\&{if} ${}(\|u\MG\\{mate}\I\|w){}$\5
${}\{{}$\1\6
${}\|u\MG\\{mate}\MG\\{mate}\K\|w\MG\\{mate},\39\|w\MG\\{mate}\MG\\{mate}\K\|u%
\MG\\{mate};{}$\6
\&{for} ${}(\|a\K\|u\MG\\{mate}\MG\\{arcs};{}$ \|a; ${}\|a\K\|a\MG\\{next}){}$%
\1\6
\&{if} ${}(\|a\MG\\{tip}\E\|w\MG\\{mate}){}$\5
${}\{{}$\1\6
${}\|u\MG\\{mate}\MG\\{deg}\MM,\39\|w\MG\\{mate}\MG\\{deg}\MM;{}$\6
\&{break};\6
\4${}\}{}$\2\2\6
\4${}\}{}$\2\6
\4${}\}{}$\2\par
\U18.\fi

\N{1}{24}Backtracking.
The fascinating thing about dancing links is the almost magical way in which
the linked data structures snap back into place when we
run the updating algorithm backwards. We do need constant vigilance,
though, because the validity of the algorithms hangs by a slender thread.

\Y\B\4\D$\\{dancing\_undelete}(\|v)$ \5
$\|v\MG\\{llink}\MG\\{rlink}\K\|v\MG\\{rlink}\MG\\{llink}\K{}$\|v\par
\B\4\D$\\{make\_bare}(\|v)$ \5
$\\{dancing\_delete}(\|v),\39\|v\MG\\{type}\K\\{bare},\39\|v\MG\\{mate}\K%
\NULL{}$\par
\fi

\M{25}The \PB{\\{emergency\_backup}} label in this section provides
an interesting example of a case where it is right and proper to
\PB{\&{goto}} a statement in the middle of one loop from the middle of another.
[See the discussion in Examples 6c and 7a of my paper ``Structured programming
with {\bf go to} statements, {\sl Computing Surveys\/~\bf6} (December 1974),
261--301.] The program jumps to \PB{\\{emergency\_backup}} when it is running
through the bare list and finds a situation that cannot be completed
to a Hamiltonian circuit; it will then undo whatever actions it had
completed so far in the clothing loop, because the unclothing loop
operates in reverse order.

\Y\B\4\X25:Unclothe everything clothed on level \PB{\|l}\X${}\E{}$\6
\&{for} ${}(\|k\K\\{bcount}-\T{1};{}$ ${}\|k\G\\{curb}[\|l];{}$ ${}\|k\MM){}$\5
${}\{{}$\1\6
${}\|v\K\\{barelist}[\|k];{}$\6
\&{if} ${}(\R\|v){}$\1\5
${}\\{barelist}[\|k]\K\\{bareback}[\|k];{}$\2\6
\&{else}\5
${}\{{}$\1\6
${}\|u\K\|v\MG\\{mate},\39\|w\K\|v\MG\\{comate};{}$\6
${}\|v\MG\\{type}\K\\{bare},\39\|v\MG\\{mate}\K\NULL;{}$\6
${}\|v\MG\\{comate}\K\NULL{}$;\C{ this isn't necessary, but I'm feeling tidy
today }\6
\&{if} ${}(\|u\MG\\{type}\E\\{outer}){}$\1\6
\&{if} ${}(\|w\MG\\{type}\E\\{outer}){}$\1\5
\X26:Demote OIO to BBB\X\2\6
\&{else}\1\5
\X27:Demote OII to BBO\X\2\2\6
\&{else} \&{if} ${}(\|w\MG\\{type}\E\\{outer}){}$\1\5
\X28:Demote IIO to OBB\X\2\6
\&{else}\1\5
\X29:Demote III to OBO\X;\2\6
\4${}\}{}$\2\6
\4\\{emergency\_backup}:\5
;\6
\4${}\}{}$\2\par
\U14.\fi

\M{26}\B\X26:Demote OIO to BBB\X${}\E{}$\6
${}\{{}$\1\6
${}\|u\MG\\{deg}\PP,\39\|w\MG\\{deg}\PP;{}$\6
\\{make\_bare}(\|u);\6
\\{make\_bare}(\|w);\6
\&{for} ${}(\|a\K\|u\MG\\{arcs};{}$ \|a; ${}\|a\K\|a\MG\\{next}){}$\1\6
\&{if} ${}(\|a\MG\\{tip}\E\|w){}$\5
${}\{{}$\1\6
${}\|u\MG\\{deg}\PP,\39\|w\MG\\{deg}\PP;{}$\6
\&{break};\6
\4${}\}{}$\2\2\6
\4${}\}{}$\2\par
\U25.\fi

\M{27}The first statement here, `\PB{$\|v\MG\\{deg}\MM$}', compensates for the
spurious
increases that will occur because \PB{\|v} is a neighbor of \PB{\|w} and \PB{$%
\|v\MG\\{type}$}
is no longer \PB{\\{inner}}.

\Y\B\4\X27:Demote OII to BBO\X${}\E{}$\6
${}\{{}$\1\6
${}\|v\MG\\{deg}\MM;{}$\6
${}\|w\MG\\{mate}\MG\\{mate}\K\|w;{}$\6
\\{dancing\_undelete}(\|w);\6
${}\|w\MG\\{type}\K\\{outer};{}$\6
\&{for} ${}(\|a\K\|u\MG\\{arcs};{}$ \|a; ${}\|a\K\|a\MG\\{next}){}$\1\6
\&{if} ${}(\|a\MG\\{tip}\E\|w\MG\\{mate}){}$\5
${}\{{}$\1\6
${}\|u\MG\\{deg}\PP,\39\|w\MG\\{mate}\MG\\{deg}\PP;{}$\6
\&{break};\6
\4${}\}{}$\2\2\6
\&{for} ${}(\|a\K\|w\MG\\{arcs};{}$ \|a; ${}\|a\K\|a\MG\\{next}){}$\5
${}\{{}$\1\6
${}\|v\K\|a\MG\\{tip};{}$\6
\&{if} ${}(\|v\MG\\{type}\I\\{inner}\W\|v\I\|w\MG\\{mate}){}$\1\5
${}\|v\MG\\{deg}\PP;{}$\2\6
\4${}\}{}$\2\6
${}\|u\MG\\{deg}\PP;{}$\6
\\{make\_bare}(\|u);\6
\4${}\}{}$\2\par
\U25.\fi

\M{28}\B\X28:Demote IIO to OBB\X${}\E{}$\6
${}\{{}$\1\6
${}\|v\MG\\{deg}\MM;{}$\6
${}\|u\MG\\{mate}\MG\\{mate}\K\|u;{}$\6
\\{dancing\_undelete}(\|u);\6
${}\|u\MG\\{type}\K\\{outer};{}$\6
\&{for} ${}(\|a\K\|w\MG\\{arcs};{}$ \|a; ${}\|a\K\|a\MG\\{next}){}$\1\6
\&{if} ${}(\|a\MG\\{tip}\E\|u\MG\\{mate}){}$\5
${}\{{}$\1\6
${}\|w\MG\\{deg}\PP,\39\|u\MG\\{mate}\MG\\{deg}\PP;{}$\6
\&{break};\6
\4${}\}{}$\2\2\6
\&{for} ${}(\|a\K\|u\MG\\{arcs};{}$ \|a; ${}\|a\K\|a\MG\\{next}){}$\5
${}\{{}$\1\6
${}\|v\K\|a\MG\\{tip};{}$\6
\&{if} ${}(\|v\MG\\{type}\I\\{inner}\W\|v\I\|u\MG\\{mate}){}$\1\5
${}\|v\MG\\{deg}\PP;{}$\2\6
\4${}\}{}$\2\6
${}\|w\MG\\{deg}\PP;{}$\6
\\{make\_bare}(\|w);\6
\4${}\}{}$\2\par
\U25.\fi

\M{29}\B\X29:Demote III to OBO\X${}\E{}$\6
${}\{{}$\1\6
${}\|v\MG\\{deg}\MRL{-{\K}}\T{2}{}$;\C{ compensate for two spurious increases
below }\6
\&{if} ${}(\|u\MG\\{mate}\I\|w){}$\5
${}\{{}$\1\6
${}\|u\MG\\{mate}\MG\\{mate}\K\|u,\39\|w\MG\\{mate}\MG\\{mate}\K\|w;{}$\6
\&{for} ${}(\|a\K\|u\MG\\{mate}\MG\\{arcs};{}$ \|a; ${}\|a\K\|a\MG\\{next}){}$%
\1\6
\&{if} ${}(\|a\MG\\{tip}\E\|w\MG\\{mate}){}$\5
${}\{{}$\1\6
${}\|u\MG\\{mate}\MG\\{deg}\PP,\39\|w\MG\\{mate}\MG\\{deg}\PP;{}$\6
\&{break};\6
\4${}\}{}$\2\2\6
\4${}\}{}$\2\6
\\{dancing\_undelete}(\|w);\6
${}\|w\MG\\{type}\K\\{outer};{}$\6
\&{for} ${}(\|a\K\|w\MG\\{arcs};{}$ \|a; ${}\|a\K\|a\MG\\{next}){}$\5
${}\{{}$\1\6
${}\|v\K\|a\MG\\{tip};{}$\6
\&{if} ${}(\|v\MG\\{type}\I\\{inner}\W\|v\I\|w\MG\\{mate}){}$\1\5
${}\|v\MG\\{deg}\PP;{}$\2\6
\4${}\}{}$\2\6
\\{dancing\_undelete}(\|u);\6
${}\|u\MG\\{type}\K\\{outer};{}$\6
\&{for} ${}(\|a\K\|u\MG\\{arcs};{}$ \|a; ${}\|a\K\|a\MG\\{next}){}$\5
${}\{{}$\1\6
${}\|v\K\|a\MG\\{tip};{}$\6
\&{if} ${}(\|v\MG\\{type}\I\\{inner}\W\|v\I\|u\MG\\{mate}){}$\1\5
${}\|v\MG\\{deg}\PP;{}$\2\6
\4${}\}{}$\2\6
\4${}\}{}$\2\par
\U25.\fi

\M{30}A somewhat subtle point deserve special mention here:
We want to reset \PB{\\{bcount}} to \PB{\\{curbb}[\|l]}, not to \PB{\\{curb}[%
\|l]},
because entries that were put onto the \PB{\\{barelist}} while \PB{\|v} was
becoming \PB{\\{inner}} should remain there.

\Y\B\4\X30:Downdate data structures to deaccount for choosing edge \PB{%
\\{cura}[\|l]}\X${}\E{}$\6
$\|v\K\\{curv}[\|l];{}$\6
${}\\{ocount}\K\\{curo}[\|l];{}$\6
${}\|u\K\\{dest}[\\{ocount}]{}$;\C{ \PB{$\\{cura}[\|l]\MG\\{tip}$} }\6
\&{if} ${}(\|u\MG\\{type}\E\\{inner}){}$\5
${}\{{}$\1\6
\&{for} ${}(\|a\K\|u\MG\\{arcs};{}$ \|a; ${}\|a\K\|a\MG\\{next}){}$\5
${}\{{}$\1\6
${}\|w\K\|a\MG\\{tip};{}$\6
\&{if} ${}(\|w\MG\\{type}\I\\{inner}\W\|w\I\|u\MG\\{mate}){}$\1\5
${}\|w\MG\\{deg}\PP;{}$\2\6
\4${}\}{}$\2\6
${}\|u\MG\\{type}\K\\{outer};{}$\6
\\{dancing\_undelete}(\|u);\6
${}\|w\K\|v\MG\\{mate};{}$\6
${}\|u\MG\\{mate}\MG\\{mate}\K\|u,\39\|w\MG\\{mate}\K\|v;{}$\6
\&{for} ${}(\|a\K\|w\MG\\{arcs};{}$ \|a; ${}\|a\K\|a\MG\\{next}){}$\1\6
\&{if} ${}(\|a\MG\\{tip}\E\|u\MG\\{mate}){}$\5
${}\{{}$\1\6
${}\|u\MG\\{mate}\MG\\{deg}\PP,\39\|w\MG\\{deg}\PP;{}$\6
\&{break};\6
\4${}\}{}$\2\2\6
\4${}\}{}$\5
\2\&{else}\5
${}\{{}$\C{ \PB{$\|u\MG\\{type}\E\\{outer}$} }\1\6
\\{make\_bare}(\|u);\6
${}\|w\K\|v\MG\\{mate};{}$\6
${}\|w\MG\\{mate}\K\|v;{}$\6
\&{for} ${}(\|a\K\|w\MG\\{arcs};{}$ \|a; ${}\|a\K\|a\MG\\{next}){}$\1\6
\&{if} ${}(\|a\MG\\{tip}\E\|u){}$\5
${}\{{}$\1\6
${}\|u\MG\\{deg}\PP,\39\|w\MG\\{deg}\PP;{}$\6
\&{break};\6
\4${}\}{}$\2\2\6
\4${}\}{}$\2\6
${}\\{bcount}\K\\{curbb}[\|l]{}$;\par
\U14.\fi

\M{31}\B\X31:Demote \PB{\|v} from \PB{\\{inner}} to \PB{\\{outer}}\X${}\E{}$\6
$\\{bcount}\K\\{curb}[\|l];{}$\6
\\{dancing\_undelete}(\|v);\6
${}\|v\MG\\{type}\K\\{outer};{}$\6
\&{for} ${}(\|a\K\|v\MG\\{arcs};{}$ \|a; ${}\|a\K\|a\MG\\{next}){}$\5
${}\{{}$\1\6
${}\|u\K\|a\MG\\{tip};{}$\6
\&{if} ${}(\|u\MG\\{type}\I\\{inner}\W\|u\I\|w){}$\1\5
${}\|u\MG\\{deg}\PP;{}$\2\6
\4${}\}{}$\2\par
\U14.\fi

\N{1}{32}Reaping the rewards. Once all vertices have been connected up,
no more decisions need to be made. In most such cases, we'll have found a
valid Hamiltonian circuit, although its last link usually still needs
to be filled in.

\Y\B\4\X32:Check for solution and \PB{\&{goto} \\{backup}}\X${}\E{}$\6
${}\{{}$\1\6
\&{if} ${}(\\{ocount}<\|g\MG\|n){}$\1\5
\X33:If the two \PB{\\{outer}} vertices aren't adjacent, \PB{\&{goto} %
\\{backup}}\X;\2\6
${}\\{total}\PP;{}$\6
\&{if} ${}(\\{total}\MOD\\{abs}(\\{modulus})\E\T{0}\V\\{verbose}){}$\5
${}\{{}$\1\6
${}\\{curo}[\|l]\K\\{ocount};{}$\6
${}\\{source}[\\{ocount}]\K\\{head}\MG\\{rlink},\39\\{dest}[\\{ocount}]\K%
\\{head}\MG\\{llink};{}$\6
${}\\{curi}[\|l]\K\\{maxi}[\|l]\K\T{1};{}$\6
\&{if} ${}(\\{modulus}<\T{0}){}$\5
${}\{{}$\1\6
${}\\{printf}(\.{"\\n\%d:\\n"},\39\\{total}){}$;\5
\\{print\_state}(\,);\6
\4${}\}{}$\5
\2\&{else}\1\5
\X34:Unscramble and print the current solution\X;\2\6
\4${}\}{}$\2\6
\&{goto} \\{backup};\6
\4${}\}{}$\2\par
\U14.\fi

\M{33}At this point we've formed a Hamiltonian path, which will be a
Hamiltonian circuit if and only if its two \PB{\\{outer}} vertices are
neighbors.

\Y\B\4\X33:If the two \PB{\\{outer}} vertices aren't adjacent, \PB{\&{goto} %
\\{backup}}\X${}\E{}$\6
${}\{{}$\1\6
${}\|u\K\\{head}\MG\\{llink},\39\|v\K\\{head}\MG\\{rlink};{}$\6
\&{for} ${}(\|a\K\|u\MG\\{arcs};{}$ \|a; ${}\|a\K\|a\MG\\{next}){}$\1\6
\&{if} ${}(\|a\MG\\{tip}\E\|v){}$\1\5
\&{goto} \\{yes};\2\2\6
\&{goto} \\{backup};\6
\4\\{yes}:\5
;\6
\4${}\}{}$\2\par
\U32.\fi

\M{34}\B\D$\\{index}(\|v)$ \5
$((\|v)-\|g\MG\\{vertices}{}$)\par
\Y\B\4\X34:Unscramble and print the current solution\X${}\E{}$\6
${}\{{}$\1\6
\&{register} \&{int} \|i${},{}$ \|j${},{}$ \|k;\7
\&{for} ${}(\|k\K\T{0};{}$ ${}\|k<\|g\MG\|n;{}$ ${}\|k\PP){}$\1\5
${}\\{v1}[\|k]\K{-}\T{1};{}$\2\6
\&{for} ${}(\|k\K\T{0};{}$ ${}\|k<\|g\MG\|n;{}$ ${}\|k\PP){}$\5
${}\{{}$\1\6
${}\|i\K\\{index}(\\{source}[\|k]);{}$\6
${}\|j\K\\{index}(\\{dest}[\|k]);{}$\6
\&{if} ${}(\\{v1}[\|i]<\T{0}){}$\1\5
${}\\{v1}[\|i]\K\|j;{}$\2\6
\&{else}\1\5
${}\\{v2}[\|i]\K\|j;{}$\2\6
\&{if} ${}(\\{v1}[\|j]<\T{0}){}$\1\5
${}\\{v1}[\|j]\K\|i;{}$\2\6
\&{else}\1\5
${}\\{v2}[\|j]\K\|i;{}$\2\6
\4${}\}{}$\2\6
${}\\{path}[\T{0}]\K\T{0},\39\\{path}[\T{1}]\K\\{v1}[\T{0}];{}$\6
\&{for} ${}(\|k\K\T{2};{}$  ; ${}\|k\PP){}$\5
${}\{{}$\1\6
\&{if} ${}(\\{v1}[\\{path}[\|k-\T{1}]]\E\\{path}[\|k-\T{2}]){}$\1\5
${}\\{path}[\|k]\K\\{v2}[\\{path}[\|k-\T{1}]];{}$\2\6
\&{else}\1\5
${}\\{path}[\|k]\K\\{v1}[\\{path}[\|k-\T{1}]];{}$\2\6
\&{if} ${}(\\{path}[\|k]\E\T{0}){}$\1\5
\&{break};\2\6
\4${}\}{}$\2\6
\&{if} (\\{verbose})\1\5
\\{printf}(\.{"\\n"});\2\6
${}\\{printf}(\.{"\%d:"},\39\\{total});{}$\6
\&{for} ${}(\|k\K\T{0};{}$ ${}\|k\Z\|g\MG\|n;{}$ ${}\|k\PP){}$\1\5
${}\\{printf}(\.{"\ \%s"},\39(\|g\MG\\{vertices}+\\{path}[\|k])\MG\\{name});{}$%
\2\6
\\{printf}(\.{"\\n"});\6
\4${}\}{}$\2\par
\U32.\fi

\M{35}\B\X3:Global variables\X${}\mathrel+\E{}$\6
\&{int} \\{v1}[\\{max\_n}]${},{}$ \\{v2}[\\{max\_n}];\C{ the neighbors of a
given vertex }\6
\&{int} ${}\\{path}[\\{max\_n}+\T{1}]{}$;\C{ the Hamiltonian circuit, in order
}\par
\fi

\N{1}{36}Getting started. Our program is almost complete, but we still need to
figure out how to get the ball rolling by setting things up properly
at backtrack level~0.

There's no problem if the graph has at least one vertex of degree 2,
because the \PB{\\{barelist}} will provide us with at least two \PB{\\{outer}}
vertices
in such a case. But if all vertices have degree 3 or more, we've got to
have some \PB{\\{outer}} vertices as seeds for the rest of the computation.

In the former (easy) case, we set \PB{$\\{maxi}[\T{0}]\K\T{0}$}. In the latter
case,
we take a vertex \PB{\|v} of minimum degree \PB{\|d}; we set \PB{$\\{maxi}[%
\T{0}]\K\|d-\T{1}$},
and try each neighbor of \PB{\|v} in turn. (More precisely, after we've found
all Hamiltonian cycles that contain an edge from \PB{\|v} to some other vertex,
\PB{\|u}, we'll remove that edge physically from the graph, and repeat
the process until \PB{\|v} or some other vertex has only two neighbors left.)

\Y\B\4\X36:Bootstrap the backtrack process\X${}\E{}$\6
$\|l\K\T{0};{}$\6
\&{if} ${}(\|d>\T{2}){}$\5
${}\{{}$\1\6
${}\\{maxi}[\T{0}]\K\|d-\T{1};{}$\6
${}\\{source}[\T{0}]\K\|v\K\\{curv}[\T{0}];{}$\6
\\{make\_outer}(\|v);\6
\4\\{force}:\5
${}\\{cura}[\T{0}]\K\|a\K\|v\MG\\{arcs};{}$\6
${}\|v\MG\\{arcs}\K\|a\MG\\{next};{}$\6
${}\\{curi}[\T{0}]\PP;{}$\6
${}\\{dest}[\T{0}]\K\|u\K\|a\MG\\{tip};{}$\6
${}\\{ocount}\K\T{1}{}$;\5
\\{vprint}(\,);\6
\\{make\_outer}(\|u);\6
${}\|v\MG\\{deg}\MM;{}$\6
${}\|u\MG\\{deg}\MM;{}$\6
\X37:Remove the arc from \PB{\|u} to \PB{\|v}\X;\6
${}\|v\MG\\{mate}\K\|u,\39\|u\MG\\{mate}\K\|v;{}$\6
\4${}\}{}$\2\par
\U14.\fi

\M{37}\B\X37:Remove the arc from \PB{\|u} to \PB{\|v}\X${}\E{}$\6
\&{if} ${}(\|u\MG\\{arcs}\MG\\{tip}\E\|v){}$\1\5
${}\|u\MG\\{arcs}\K\|u\MG\\{arcs}\MG\\{next};{}$\2\6
\&{else}\5
${}\{{}$\1\6
\&{for} ${}(\|a\K\|u\MG\\{arcs};{}$ ${}\|a\MG\\{next}\MG\\{tip}\I\|v;{}$ ${}\|a%
\K\|a\MG\\{next}){}$\1\5
;\2\6
${}\|a\MG\\{next}\K\|a\MG\\{next}\MG\\{next};{}$\6
\4${}\}{}$\2\par
\U36.\fi

\M{38}When the edge between \PB{\|u} and \PB{\|v} is removed, and \PB{\|u}
reverts to a
\PB{\\{bare}} vertex, it might now have degree~2. In such cases we don't
need \PB{\|v} as a seed vertex, so we revert to the simpler algorithm.

\Y\B\4\X38:Advance at bottom level\X${}\E{}$\6
\&{if} ${}(\\{curi}[\T{0}]<\\{maxi}[\T{0}]){}$\5
${}\{{}$\1\6
\&{if} (\\{verbose})\1\5
\\{printf}(\.{"\ back\ to\ level\ 0:\\n}\)\.{"});\2\6
${}\|l\K\T{0};{}$\6
${}\\{ocount}\K\T{0};{}$\6
${}\|u\K\\{dest}[\T{0}];{}$\6
\\{dancing\_delete}(\|u);\6
${}\|u\MG\\{type}\K\\{bare};{}$\6
\&{if} ${}(\|u\MG\\{deg}\E\T{2}){}$\1\5
${}\\{barelist}[\T{0}]\K\|u,\39\\{bcount}\K\T{1};{}$\2\6
\&{else}\1\5
${}\\{bcount}\K\T{0}{}$;\C{ we never undo \PB{\\{barelist}} conversions at
level zero }\2\6
${}\|v\K\\{source}[\T{0}];{}$\6
\&{if} ${}(\|v\MG\\{deg}\E\T{2}){}$\5
${}\{{}$\1\6
${}\|v\MG\\{type}\K\\{bare};{}$\6
\\{dancing\_delete}(\|v);\6
${}\\{barelist}[\\{bcount}\PP]\K\|v;{}$\6
\4${}\}{}$\2\6
\&{if} ${}(\\{bcount}\E\T{0}){}$\1\5
\&{goto} \\{force};\2\6
${}\\{maxi}[\T{0}]\K\\{curi}[\T{0}]\K\\{curi}[\T{0}]+\T{1}{}$;\C{ cut to the
chase }\6
${}\\{cura}[\T{0}]\K\NULL;{}$\6
\&{goto} \\{advance};\6
\4${}\}{}$\2\par
\U14.\fi

\M{39}\B\X39:Print the state line for the bottom level\X${}\E{}$\6
\&{if} (\\{cura}[\T{0}])\1\5
${}\\{printf}(\.{"\ \%3d:\ \%s--\%s\ (\%d\ of}\)\.{\ \%d)\\n"},\39\T{0},\39%
\\{source}[\T{0}]\MG\\{name},\39\\{dest}[\T{0}]\MG\\{name},\39\\{curi}[\T{0}],%
\39\\{maxi}[\T{0}]);{}$\2\6
\&{else}\5
${}\{{}$\1\6
${}\|j\K{-}\T{1}{}$;\C{ this trick will make \PB{\\{source}[\T{0}]} and \PB{%
\\{dest}[\T{0}]} appear }\6
\&{if} (\\{maxi}[\T{0}])\1\5
${}\\{printf}(\.{"\ \%3d:\ (\%d\ of\ \%d)\\n"},\39\T{0},\39\\{curi}[\T{0}],\39%
\\{maxi}[\T{0}]);{}$\2\6
\4${}\}{}$\2\par
\U12.\fi

\N{1}{40}Index.
\fi

\inx
\fin
\con
