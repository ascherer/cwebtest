\input cwebmac
\srcloctrue
% This file is part of the Stanford GraphBase (c) Stanford University 1990
\nocon


\N[4 hull.w]{1}{1}Introduction. This is a hastily written implementation of the
daghull
algorithm.

\Y\B\F\\{Graph} \5
\\{int}\C{ \PB{\\{gb\_graph}} defines the \PB{\&{Graph}} type and a few others
}\par
\B\F\\{Vertex} \5
\\{int}\par
\B\F\\{Arc} \5
\\{int}\par
\B\F\\{Area} \5
\\{int}\par
\Y\B\8\#\&{include} \.{"gb\_graph.h"}\6
\8\#\&{include} \.{"gb\_miles.h"}\6
\&{int} \|n${}\K\T{128};{}$\7
\X2:Global variables\X\6
\X13:Procedures\X\7
\\{main}(\,)\1\1\2\2\6
${}\{{}$\1\6
\X7:Local variables\X\7
\&{Graph} ${}{*}\|g\K\\{miles}(\T{128},\39\T{0},\39\T{0},\39\T{0},\39\T{0},\39%
\T{0},\39\T{0}){}$;\7
${}\\{mems}\K\\{ccs}\K\T{0};{}$\6
\X8:Find convex hull of \PB{\|g}\X;\6
${}\\{printf}(\.{"Total\ of\ \%d\ mems\ an}\)\.{d\ \%d\ calls\ on\ ccw.\\n}\)%
\.{"},\39\\{mems},\39\\{ccs});{}$\6
\4${}\}{}$\2\par
\fi

\M[29 hull.w]{2}I'm instrumenting this in a simple way.

\Y\B\4\D$\|o$ \5
$\\{mems}\PP{}$\par
\B\4\D$\\{oo}$ \5
$\\{mems}\MRL{+{\K}}{}$\T{2}\par
\Y\B\4\X2:Global variables\X${}\E{}$\6
\&{int} \\{mems};\C{ memory accesses }\6
\&{int} \\{ccs};\C{ calls on \PB{\\{ccw}} }\par
\A5.
\U1.\fi

\N[38 hull.w]{1}{3}Data structures.
For now, each vertex is represented by two coordinates stored in the
utility fields \PB{$\|x.\|i$} and \PB{$\|y.\|i$}. I'm also putting a serial
number into
\PB{$\|z.\|i$}, so that I can check whether different algorithms generate
identical hulls.

A vertex \PB{\|v} in the convex hull also has a successor \PB{$\|v\MG\\{succ}$}
and
and predecessor \PB{$\|v\MG\\{pred}$}, stored in utility fields \PB{\|u} and %
\PB{\|v}.
There's also a pointer to an ``instruction,'' \PB{$\|v\MG\\{inst}$}, for which
I'm
using an arc record although I need only two fields; that one goes in
utility field \PB{\|w}.

An instruction has two parts. Its \PB{\\{tip}} is a vertex to be used in
counterclockwise testing. Its \PB{\\{next}} is another instruction to be
followed;
or, if \PB{$\NULL$}, there's no next instruction; or, if a pointer to the
smallest possible instruction, we will do something special to update
the current convex hull.

\Y\B\4\D$\\{succ}$ \5
$\|u.{}$\|v\par
\B\4\D$\\{pred}$ \5
$\|v.{}$\|v\par
\B\4\D$\\{inst}$ \5
$\|w.{}$\|a\par
\fi

\M[60 hull.w]{4}\B\X4:Initialize the array of instructions\X${}\E{}$\6
$\\{first\_inst}\K{}$(\&{Arc} ${}{*}){}$ \\{gb\_alloc}${}((\T{4}*\|g\MG\|n-%
\T{2})*\&{sizeof}(\&{Arc}),\39\\{working\_storage});{}$\6
\&{if} ${}(\\{first\_inst}\E\NULL){}$\1\5
\&{return} (\T{1});\C{ fixthis }\2\6
${}\\{next\_inst}\K\\{first\_inst}{}$;\par
\U6.\fi

\M[65 hull.w]{5}\B\X2:Global variables\X${}\mathrel+\E{}$\6
\&{Arc} ${}{*}\\{first\_inst}{}$;\C{ beginning of the array of instructions }\6
\&{Arc} ${}{*}\\{next\_inst}{}$;\C{ first unused slot in that array }\6
\&{Vertex} ${}{*}\\{rover}{}$;\C{ one of the vertices in the convex hull }\6
\&{Area} \\{working\_storage};\6
\&{int} \\{serial\_no}${}\K\T{1}{}$;\C{ used to disambiguate entries with equal
coordinates }\par
\fi

\M[72 hull.w]{6}We assume that the vertices have been given to us in a
GraphBase-type
graph. The algorithm begins with a trivial hull that contains
only the first two vertices.

\Y\B\4\X6:Initialize the data structures\X${}\E{}$\6
\\{init\_area}(\\{working\_storage});\6
\X4:Initialize the array of instructions\X;\6
${}\|o,\39\|u\K\|g\MG\\{vertices};{}$\6
${}\|v\K\|u+\T{1};{}$\6
${}\|u\MG\|z.\|i\K\T{0};{}$\6
${}\|v\MG\|z.\|i\K\T{1};{}$\6
${}\\{oo},\39\|u\MG\\{succ}\K\|u\MG\\{pred}\K\|v;{}$\6
${}\\{oo},\39\|v\MG\\{succ}\K\|v\MG\\{pred}\K\|u;{}$\6
${}\\{oo},\39\\{first\_inst}\MG\\{tip}\K\|u{}$;\5
${}\\{first\_inst}\MG\\{next}\K\\{first\_inst};{}$\6
${}\\{oo},\39(\PP\\{next\_inst})\MG\\{tip}\K\|v{}$;\5
${}\\{next\_inst}\MG\\{next}\K\\{first\_inst};{}$\6
${}\|o,\39\|u\MG\\{inst}\K\\{first\_inst};{}$\6
${}\|o,\39\|v\MG\\{inst}\K\\{next\_inst}\PP;{}$\6
${}\\{rover}\K\|u;{}$\6
\&{if} ${}(\|n<\T{150}){}$\1\5
${}\\{printf}(\.{"Beginning\ with\ (\%s;}\)\.{\ \%s)\\n"},\39\|u\MG\\{name},\39%
\|v\MG\\{name}){}$;\2\par
\U8.\fi

\M[91 hull.w]{7}We'll probably need a bunch of local variables to do elementary
operations on
data structures.

\Y\B\4\X7:Local variables\X${}\E{}$\6
\&{Vertex} ${}{*}\|u,{}$ ${}{*}\|v,{}$ ${}{*}\\{vv},{}$ ${}{*}\|w;{}$\6
\&{Arc} ${}{*}\|p,{}$ ${}{*}\|q,{}$ ${}{*}\|r,{}$ ${}{*}\|s{}$;\par
\U1.\fi

\N[98 hull.w]{1}{8}Hull updating.
The main loop of the algorithm updates the data structure incrementally
by adding one new vertex at a time. If the new vertex lies outside the
current convex hull, we put it into the cycle and possibly delete some
vertices that were previously part of the hull.

\Y\B\4\X8:Find convex hull of \PB{\|g}\X${}\E{}$\6
\X6:Initialize the data structures\X;\6
\&{for} ${}(\\{oo},\39\\{vv}\K\|g\MG\\{vertices}+\T{2};{}$ ${}\\{vv}<\|g\MG%
\\{vertices}+\|g\MG\|n;{}$ ${}\\{vv}\PP){}$\5
${}\{{}$\1\6
${}\\{vv}\MG\|z.\|i\K\PP\\{serial\_no};{}$\6
\X10:Follow the instructions; \PB{\&{continue}} if \PB{\\{vv}} is inside the
current hull\X;\6
\X11:Update the convex hull, knowing that \PB{\\{vv}} lies outside the
consecutive hull vertices \PB{\|u} and \PB{\|v}\X;\6
\4${}\}{}$\2\6
\X9:Print the convex hull\X;\par
\U1.\fi

\M[114 hull.w]{9}Let me do the easy part first, since it's bedtime and I can
worry about
the rest tomorrow.

\Y\B\4\X9:Print the convex hull\X${}\E{}$\6
$\|u\K\\{rover};{}$\6
\\{printf}(\.{"The\ convex\ hull\ is:}\)\.{\\n"});\6
\&{do}\5
${}\{{}$\1\6
${}\\{printf}(\.{"\ \ \%s\\n"},\39\|u\MG\\{name});{}$\6
${}\|u\K\|u\MG\\{succ};{}$\6
\4${}\}{}$\2\5
\&{while} ${}(\|u\I\\{rover}){}$;\par
\U8.\fi

\M[125 hull.w]{10}\B\X10:Follow the instructions; \PB{\&{continue}} if \PB{%
\\{vv}} is inside the current hull\X${}\E{}$\6
$\|p\K\\{first\_inst};{}$\6
\&{do}\5
${}\{{}$\1\6
\&{if} ${}(\\{oo},\39\\{ccw}(\|p\MG\\{tip},\39\\{vv},\39(\|p+\T{1})\MG%
\\{tip})){}$\1\5
${}\|p\PP;{}$\2\6
${}\|q\K\|p;{}$\6
${}\|o,\39\|p\K\|p\MG\\{next};{}$\6
\4${}\}{}$\2\5
\&{while} ${}(\|p>\\{first\_inst});{}$\6
\&{if} ${}(\|p\E\NULL){}$\1\5
\&{continue};\2\6
${}\|o,\39\|v\K\|q\MG\\{tip};{}$\6
${}\|o,\39\|u\K\|v\MG\\{pred}{}$;\par
\U8.\fi

\M[136 hull.w]{11}\B\X11:Update the convex hull, knowing that \PB{\\{vv}} lies
outside the consecutive hull vertices \PB{\|u} and \PB{\|v}\X${}\E{}$\6
$\|o,\39\|q\MG\\{next}\K\\{next\_inst};{}$\6
\&{while} (\T{1})\5
${}\{{}$\1\6
${}\|o,\39\|w\K\|u\MG\\{pred};{}$\6
\&{if} ${}(\|w\E\|v){}$\1\5
\&{break};\2\6
\&{if} ${}(\\{ccw}(\\{vv},\39\|w,\39\|u)){}$\1\5
\&{break};\2\6
${}\|o,\39\|u\MG\\{inst}\MG\\{next}\K\\{next\_inst};{}$\6
\&{if} ${}(\\{rover}\E\|u){}$\1\5
${}\\{rover}\K\|w;{}$\2\6
${}\|u\K\|w;{}$\6
\4${}\}{}$\2\6
\&{while} (\T{1})\5
${}\{{}$\1\6
${}\|o,\39\|w\K\|v\MG\\{succ};{}$\6
\&{if} ${}(\|w\E\|u){}$\1\5
\&{break};\2\6
\&{if} ${}(\\{ccw}(\|w,\39\\{vv},\39\|v)){}$\1\5
\&{break};\2\6
\&{if} ${}(\\{rover}\E\|v){}$\1\5
${}\\{rover}\K\|w;{}$\2\6
${}\|v\K\|w;{}$\6
${}\|o,\39\|v\MG\\{inst}\MG\\{next}\K\\{next\_inst};{}$\6
\4${}\}{}$\2\6
${}\\{oo},\39\|u\MG\\{succ}\K\|v\MG\\{pred}\K\\{vv};{}$\6
${}\\{oo},\39\\{vv}\MG\\{pred}\K\|u{}$;\5
${}\\{vv}\MG\\{succ}\K\|v;{}$\6
\X12:Compile two new instructions, for \PB{$(\|u,\\{vv})$} and \PB{$(\\{vv},%
\|v)$}\X;\par
\U8.\fi

\M[159 hull.w]{12}\B\X12:Compile two new instructions, for \PB{$(\|u,\\{vv})$}
and \PB{$(\\{vv},\|v)$}\X${}\E{}$\6
$\|o,\39\\{next\_inst}\MG\\{tip}\K\\{vv};{}$\6
${}\|o,\39\\{next\_inst}\MG\\{next}\K\\{first\_inst};{}$\6
${}\|o,\39\\{vv}\MG\\{inst}\K\\{next\_inst};{}$\6
${}\\{next\_inst}\PP;{}$\6
${}\|o,\39\\{next\_inst}\MG\\{tip}\K\|u;{}$\6
${}\|o,\39\\{next\_inst}\MG\\{next}\K\\{next\_inst}+\T{1};{}$\6
${}\\{next\_inst}\PP;{}$\6
${}\|o,\39\\{next\_inst}\MG\\{tip}\K\|v;{}$\6
${}\|o,\39\\{next\_inst}\MG\\{next}\K\\{first\_inst};{}$\6
${}\|o,\39\|v\MG\\{inst}\K\\{next\_inst};{}$\6
${}\\{next\_inst}\PP;{}$\6
${}\|o,\39\\{next\_inst}\MG\\{tip}\K\\{vv};{}$\6
${}\|o,\39\\{next\_inst}\MG\\{next}\K\NULL;{}$\6
${}\\{next\_inst}\PP;{}$\6
\&{if} ${}(\|n<\T{150}){}$\1\5
${}\\{printf}(\.{"New\ hull\ sequence\ (}\)\.{\%s;\ \%s;\ \%s)\\n"},\39\|u\MG%
\\{name},\39\\{vv}\MG\\{name},\39\|v\MG\\{name}){}$;\2\par
\U11.\fi

\N[176 hull.w]{1}{13}Determinants. I need code for the primitive function \PB{%
\\{ccw}}.
Floating-point arithmetic suffices for my purposes.

We want to evaluate the determinant
$$ccw(u,v,w)=\left\vert\matrix{u(x)&u(y)&1\cr v(x)&v(y)&1\cr w(x)&w(y)&1\cr}
\right\vert=\left\vert\matrix{u(x)-w(x)&u(y)-w(y)\cr v(x)-w(x)&v(y)-w(y)\cr}
\right\vert\,.$$

\Y\B\4\X13:Procedures\X${}\E{}$\6
\&{int} ${}\\{ccw}(\|u,\39\|v,\39\|w){}$\1\1\6
\&{Vertex} ${}{*}\|u,{}$ ${}{*}\|v,{}$ ${}{*}\|w;\2\2{}$\6
${}\{{}$\5
\1\&{register} \&{double} \\{wx}${}\K{}$(\&{double}) \|w${}\MG\|x.\|i,{}$ %
\\{wy}${}\K{}$(\&{double}) \|w${}\MG\|y.\|i;{}$\6
\&{register} \&{double} \\{det}${}\K{}$((\&{double}) \|u${}\MG\|x.\|i-%
\\{wx})*{}$((\&{double}) \|v${}\MG\|y.\|i-\\{wy})-{}$((\&{double}) \|u${}\MG%
\|y.\|i-\\{wy})*{}$((\&{double}) \|v${}\MG\|x.\|i-\\{wx});{}$\6
\&{Vertex} ${}{*}\\{uu}\K\|u,{}$ ${}{*}\\{vv}\K\|v,{}$ ${}{*}\\{ww}\K\|w,{}$
${}{*}\|t;{}$\7
\&{if} ${}(\\{det}\E\T{0}){}$\5
${}\{{}$\1\6
${}\\{det}\K\T{1};{}$\6
\&{if} ${}(\|u\MG\|x.\|i>\|v\MG\|x.\|i\V(\|u\MG\|x.\|i\E\|v\MG\|x.\|i\W(\|u\MG%
\|y.\|i>\|v\MG\|y.\|i\V(\|u\MG\|y.\|i\E\|v\MG\|y.\|i\W\|u\MG\|z.\|i>\|v\MG\|z.%
\|i)))){}$\5
${}\{{}$\1\6
${}\|t\K\|u{}$;\5
${}\|u\K\|v{}$;\5
${}\|v\K\|t{}$;\5
${}\\{det}\K{-}\\{det};{}$\6
\4${}\}{}$\2\6
\&{if} ${}(\|v\MG\|x.\|i>\|w\MG\|x.\|i\V(\|v\MG\|x.\|i\E\|w\MG\|x.\|i\W(\|v\MG%
\|y.\|i>\|w\MG\|y.\|i\V(\|v\MG\|y.\|i\E\|w\MG\|y.\|i\W\|v\MG\|z.\|i>\|w\MG\|z.%
\|i)))){}$\5
${}\{{}$\1\6
${}\|t\K\|v{}$;\5
${}\|v\K\|w{}$;\5
${}\|w\K\|t{}$;\5
${}\\{det}\K{-}\\{det};{}$\6
\4${}\}{}$\2\6
\&{if} ${}(\|u\MG\|x.\|i>\|v\MG\|x.\|i\V(\|u\MG\|x.\|i\E\|v\MG\|x.\|i\W(\|u\MG%
\|y.\|i>\|v\MG\|y.\|i\V(\|u\MG\|y.\|i\E\|v\MG\|y.\|i\W\|u\MG\|z.\|i<\|v\MG\|z.%
\|i)))){}$\5
${}\{{}$\1\6
${}\\{det}\K{-}\\{det};{}$\6
\4${}\}{}$\2\6
\4${}\}{}$\2\6
\&{if} ${}(\|n<\T{150}){}$\1\5
${}\\{printf}(\.{"cc(\%s;\ \%s;\ \%s)\ is\ \%}\)\.{s\\n"},\39\\{uu}\MG\\{name},%
\39\\{vv}\MG\\{name},\39\\{ww}\MG\\{name},\39\\{det}>\T{0}\?\.{"true"}:%
\.{"false"});{}$\2\6
${}\\{ccs}\PP;{}$\6
\&{return} ${}(\\{det}>\T{0});{}$\6
\4${}\}{}$\2\par
\U1.\fi

\inx
\fin
\con
