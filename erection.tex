\input cwebmac
\datethis

\N{1}{1}Intro. This is a transcription of my ``random matroid'' program in %
\#P72.

Standard input contains a sequence of integers. The first of these is
the universe size, $n$, which should be at most 16.
Then comes, for $r=1$, 2, \dots, a list of sets that are stipulated to have
rank $\le r$. Sets are specified in hexadecimal notation, and each
list is terminated by 0. Thus, the $\pi$-based example in my paper
corresponds to the standard input
$$\.{10 1a 222 64 128 288 10c}$$
because $\PB{\T{\^1a}}=2^4+2^3+2^1$ represents the set $\{1,3,4\}$, and
\PB{\T{\^222}} represents $\{1,5,9\}$, etc. The program appends zeros
to the data on standard input if necessary, so trailing zeros can
be omitted. Similarly, the standard input
$$\.{5 7 0 1e}$$
specifies a five-point matroid in
which $\{0,1,2\}$ has rank $\le2$ and $\{1,2,3,4\}$ has rank~$\le3$.

\Y\B\4\D$\\{nmax}$ \5
\T{16}\C{ to go higher, extend \PB{\\{print\_set}} to larger-than-hex digits }%
\par
\B\4\D$\\{lmax}$ \5
\T{25742}\C{ $2({16\choose8}+1)$, a safe upper bound on list size }\par
\Y\B\8\#\&{include} \.{<stdio.h>}\6
\&{int} \|n;\C{ number of elements in the universe }\6
\&{int} \\{mask};\C{ $2^n-1$ }\6
\&{int} ${}\|S[\\{lmax}+\T{1}],{}$ ${}\|L[\\{lmax}+\T{1}]{}$;\C{ list memory }\6
\&{int} \|r;\C{ the current rank }\6
\&{int} \|h;\C{ head of circular list of closed sets for rank \PB{\|r} }\6
\&{int} \\{nh};\C{ head of circular list being formed for rank \PB{$\|r+\T{1}$}
}\6
\&{int} \\{avail};\C{ beginning the list of available space }\6
\&{int} \\{unused};\C{ the first unused slot in \PB{\|S} and \PB{\|L} arrays }\6
\&{int} \|x;\C{ a set used to communicate with the \PB{\\{insert}} routine }\6
\&{int} ${}\\{rank}[\T{1}\LL\\{nmax}]{}$;\C{ $\rm 100+cardinality$, or assigned
rank }\7
\X8:Subroutines\X\7
\\{main}(\,)\1\1\2\2\6
${}\{{}$\1\6
\&{register} \&{int} \|i${},{}$ \|j${},{}$ \|k;\7
\&{if} ${}(\\{scanf}(\.{"\%d"},\39{\AND}\|n)\I\T{1}\V\|n>\T{16}\V\|n<\T{0}){}$\5
${}\{{}$\1\6
${}\\{fprintf}(\\{stderr},\39\.{"Sorry,\ I\ can't\ deal}\)\.{\ with\ a\
universe\ of\ }\)\.{size\ \%d.\\n"},\39\|n);{}$\6
${}\\{exit}({-}\T{1});{}$\6
\4${}\}{}$\2\6
${}\\{mask}\K(\T{1}\LL\|n)-\T{1};{}$\6
\X2:Set initial contents of \PB{\\{rank}} table\X;\6
\X3:Initialize list memory to available\X;\6
${}\\{rank}[\T{0}]\K\T{0},\39\|r\K\T{0};{}$\6
\&{while} ${}(\\{rank}[\\{mask}]>\|r){}$\1\5
\X4:Pass from rank $r$ to $r+1$\X;\2\6
\\{print\_circuits}(\,);\6
\4${}\}{}$\2\par
\fi

\M{2}\B\X2:Set initial contents of \PB{\\{rank}} table\X${}\E{}$\6
$\|k\K\T{1};{}$\6
${}\\{rank}[\T{0}]\K\T{100};{}$\6
\&{while} ${}(\|k\Z\\{mask}){}$\5
${}\{{}$\1\6
\&{for} ${}(\|i\K\T{0};{}$ ${}\|i<\|k;{}$ ${}\|i\PP){}$\1\5
${}\\{rank}[\|k+\|i]\K\\{rank}[\|i]+\T{1};{}$\2\6
${}\|k\K\|k+\|k;{}$\6
\4${}\}{}$\2\par
\U1.\fi

\M{3}The published paper had a comparatively inefficient algorithm here;
it initialized thousands of links that usually remained unused.

\Y\B\4\X3:Initialize list memory to available\X${}\E{}$\6
$\|L[\T{1}]\K\T{2};{}$\6
${}\|L[\T{2}]\K\T{1};{}$\6
${}\|S[\T{2}]\K\T{0};{}$\6
${}\|h\K\T{1}{}$;\C{ list containing the empty set }\6
${}\\{unused}\K\T{3}{}$;\par
\U1.\fi

\M{4}\B\X4:Pass from rank $r$ to $r+1$\X${}\E{}$\6
${}\{{}$\1\6
\X5:Create empty list\X;\6
\\{generate}(\,);\6
\&{if} (\|r)\1\5
\\{enlarge}(\,);\2\6
\X6:Return list \PB{\|h} to available storage\X;\6
${}\|r\PP;{}$\6
${}\|h\K\\{nh};{}$\6
\\{sort}(\,);\C{ optional }\6
\\{print\_list}(\|h);\6
\X7:Assign rank to sets and print independent ones\X;\6
\4${}\}{}$\2\par
\U1.\fi

\M{5}\B\X5:Create empty list\X${}\E{}$\6
$\\{nh}\K\\{avail};{}$\6
\&{if} (\\{nh})\1\5
${}\\{avail}\K\|L[\\{nh}];{}$\2\6
\&{else}\1\5
${}\\{nh}\K\\{unused}\PP;{}$\2\6
${}\|L[\\{nh}]\K\\{nh}{}$;\par
\U4.\fi

\M{6}\B\X6:Return list \PB{\|h} to available storage\X${}\E{}$\6
\&{for} ${}(\|j\K\|h;{}$ ${}\|L[\|j]\I\|h;{}$ ${}\|j\K\|L[\|j]){}$\1\5
;\2\6
${}\|L[\|j]\K\\{avail};{}$\6
${}\\{avail}\K\|h{}$;\par
\U4.\fi

\M{7}\B\X7:Assign rank to sets and print independent ones\X${}\E{}$\6
$\\{printf}(\.{"Independent\ sets\ fo}\)\.{r\ rank\ \%d:"},\39\|r);{}$\6
\&{for} ${}(\|j\K\|L[\|h];{}$ ${}\|j\I\|h;{}$ ${}\|j\K\|L[\|j]){}$\1\5
\\{mark}(\|S[\|j]);\2\6
\\{printf}(\.{"\\n"});\par
\U4.\fi

\M{8}The \PB{\\{generate}} procedure inserts minimal closed sets for rank \PB{$%
\|r+\T{1}$}
into a circular list headed by \PB{\\{nh}}. (It corresponds to ``Step 2'' in
the published algorithm.)

\Y\B\4\X8:Subroutines\X${}\E{}$\6
\&{void} \\{insert}(\&{void});\C{ details coming soon }\7
\&{void} \\{generate}(\&{void})\1\1\2\2\6
${}\{{}$\1\6
\&{register} \&{int} \|t${},{}$ \|v${},{}$ \|y${},{}$ \|j${},{}$ \|k;\7
\&{for} ${}(\|j\K\|L[\|h];{}$ ${}\|j\I\|h;{}$ ${}\|j\K\|L[\|j]){}$\5
${}\{{}$\1\6
${}\|y\K\|S[\|j]{}$;\C{ a closed set of rank \PB{\|r} }\6
${}\|t\K\\{mask}-\|y;{}$\6
\X9:Find all sets in list \PB{\\{nh}} that already contain \PB{\|y} and remove
excess elements from \PB{\|t}\X;\6
\X10:Insert $y\cup a$ for each $a\in t$\X;\6
\4${}\}{}$\2\6
\4${}\}{}$\2\par
\As11, 12, 13, 14, 15, 16, 17\ETs18.
\U1.\fi

\M{9}\B\X9:Find all sets in list \PB{\\{nh}} that already contain \PB{\|y} and
remove excess elements from \PB{\|t}\X${}\E{}$\6
\&{for} ${}(\|k\K\|L[\\{nh}];{}$ ${}\|k\I\\{nh};{}$ ${}\|k\K\|L[\|k]){}$\1\6
\&{if} ${}((\|S[\|k]\AND\|y)\E\|y){}$\1\5
${}\|t\MRL{\AND{\K}}\CM\|S[\|k]{}$;\2\2\par
\U8.\fi

\M{10}\B\X10:Insert $y\cup a$ for each $a\in t$\X${}\E{}$\6
\&{while} (\|t)\5
${}\{{}$\1\6
${}\|x\K\|y\OR(\|t\AND{-}\|t);{}$\6
\\{insert}(\,);\C{ insert \PB{\|x} into \PB{\\{nh}}, possibly enlarging \PB{%
\|x} }\6
${}\|t\MRL{\AND{\K}}\CM\|x;{}$\6
\4${}\}{}$\2\par
\U8.\fi

\M{11}The following key procedure basically inserts the set \PB{\|x} into list %
\PB{\\{nh}}.
But it augments \PB{\|x} if necessary (and deletes existing entries of the
list)
so that no two entries have an intersection of rank greater than~\PB{\|r}.
Thus it incorporates the idea of ``Step 4,'' but it is more efficient than a
brute force implementation of that step.

\Y\B\4\X8:Subroutines\X${}\mathrel+\E{}$\6
\&{void} \\{insert}(\&{void})\1\1\2\2\6
${}\{{}$\1\6
\&{register} \&{int} \|j${},{}$ \|k;\7
${}\|j\K\\{nh};{}$\6
\4\\{store}:\5
${}\|S[\\{nh}]\K\|x;{}$\6
\4\\{loop}:\5
${}\|k\K\|j;{}$\6
\4\\{continu}:\5
${}\|j\K\|L[\|k];{}$\6
\&{if} ${}(\\{rank}[\|S[\|j]\AND\|x]\Z\|r){}$\1\5
\&{goto} \\{loop};\2\6
\&{if} ${}(\|j\I\\{nh}){}$\5
${}\{{}$\1\6
\&{if} ${}(\|x\E(\|x\OR\|S[\|j])){}$\5
${}\{{}$\C{ remove from list and continue }\1\6
${}\|L[\|k]\K\|L[\|j],\39\|L[\|j]\K\\{avail},\39\\{avail}\K\|j;{}$\6
\&{goto} \\{continu};\6
\4${}\}{}$\5
\2\&{else}\5
${}\{{}$\C{ augment \PB{\|x} and go around again }\1\6
${}\|x\MRL{{\OR}{\K}}\|S[\|j],\39\\{nh}\K\|j;{}$\6
\&{goto} \\{store};\6
\4${}\}{}$\2\6
\4${}\}{}$\2\6
${}\|j\K\\{avail};{}$\6
\&{if} (\|j)\1\5
${}\\{avail}\K\|L[\|j];{}$\2\6
\&{else}\1\5
${}\|j\K\\{unused}\PP;{}$\2\6
${}\|L[\|j]\K\|L[\\{nh}];{}$\6
${}\|L[\\{nh}]\K\|j;{}$\6
${}\|S[\|j]\K\|x;{}$\6
\4${}\}{}$\2\par
\fi

\M{12}The \PB{\\{enlarge}} procedure inserts sets that are read from standard
input
until encountering an empty set.
(It corresponds to ``Step~3.'')

\Y\B\4\X8:Subroutines\X${}\mathrel+\E{}$\6
\&{void} \\{enlarge}(\&{void})\1\1\2\2\6
${}\{{}$\1\6
\&{while} (\T{1})\5
${}\{{}$\1\6
${}\|x\K\T{0};{}$\6
${}\\{scanf}(\.{"\%x"},\39{\AND}\|x);{}$\6
\&{if} ${}(\R\|x){}$\1\5
\&{return};\2\6
\&{if} ${}(\\{rank}[\|x]>\|r){}$\1\5
\\{insert}(\,);\2\6
\4${}\}{}$\2\6
\4${}\}{}$\2\par
\fi

\M{13}We don't output a set as a hexadecimal number according to the
convention used on standard input; instead, we print an increasing sequence
of hexadecimal digits that name the actual set elements.
For example, the set that was input as \.{1a} would be output as \.{134}.

\Y\B\4\X8:Subroutines\X${}\mathrel+\E{}$\6
\&{void} \\{print\_set}(\&{int} \|t)\1\1\2\2\6
${}\{{}$\1\6
\&{register} \&{int} \|j${},{}$ \|k;\7
\\{printf}(\.{"\ "});\6
\&{for} ${}(\|j\K\T{1},\39\|k\K\T{0};{}$ ${}\|j\Z\|t;{}$ ${}\|j\MRL{{\LL}{\K}}%
\T{1},\39\|k\PP){}$\1\6
\&{if} ${}(\|t\AND\|j){}$\1\5
${}\\{printf}(\.{"\%x"},\39\|k);{}$\2\2\6
\4${}\}{}$\2\par
\fi

\M{14}\B\X8:Subroutines\X${}\mathrel+\E{}$\6
\&{void} \\{print\_list}(\&{int} \|h)\1\1\2\2\6
${}\{{}$\1\6
\&{register} \&{int} \|j;\7
${}\\{printf}(\.{"Closed\ sets\ for\ ran}\)\.{k\ \%d:"},\39\|r);{}$\6
\&{for} ${}(\|j\K\|L[\|h];{}$ ${}\|j\I\|h;{}$ ${}\|j\K\|L[\|j]){}$\1\5
\\{print\_set}(\|S[\|j]);\2\6
\\{printf}(\.{"\\n"});\6
\4${}\}{}$\2\par
\fi

\M{15}The subroutine \PB{\\{mark}(\|m)} sets $\PB{\\{rank}}[m']=r$ for all
subsets $m'\subseteq m$
whose rank is not already $\le r$, and outputs $m'$ if it is independent
(that is, if its rank equals its cardinality).

\Y\B\4\X8:Subroutines\X${}\mathrel+\E{}$\6
\&{void} \\{mark}(\&{int} \|m)\1\1\2\2\6
${}\{{}$\1\6
\&{register} \&{int} \|t${},{}$ \|v;\7
\&{if} ${}(\\{rank}[\|m]>\|r){}$\5
${}\{{}$\1\6
\&{if} ${}(\\{rank}[\|m]\E\T{100}+\|r){}$\1\5
\\{print\_set}(\|m);\2\6
${}\\{rank}[\|m]\K\|r;{}$\6
\&{for} ${}(\|t\K\|m;{}$ \|t; ${}\|t\K\|v){}$\5
${}\{{}$\1\6
${}\|v\K\|t\AND(\|t-\T{1});{}$\6
${}\\{mark}(\|m-\|t+\|v);{}$\6
\4${}\}{}$\2\6
\4${}\}{}$\2\6
\4${}\}{}$\2\par
\fi

\M{16}I've added a \PB{\\{tl}} array to the data structure, to speed up and
shorten
this routine.

\Y\B\4\X8:Subroutines\X${}\mathrel+\E{}$\6
\&{void} \\{sort}(\,)\1\1\2\2\6
${}\{{}$\1\6
\&{register} \&{int} \|i${},{}$ \|j${},{}$ \|k;\6
\&{int} ${}\\{hd}[\T{101}+\\{nmax}],{}$ ${}\\{tl}[\T{101}+\\{nmax}];{}$\7
\&{for} ${}(\|i\K\T{100};{}$ ${}\|i\Z\T{100}+\|n;{}$ ${}\|i\PP){}$\1\5
${}\\{hd}[\|i]\K{-}\T{1};{}$\2\6
${}\|j\K\|L[\|h];{}$\6
${}\|L[\|h]\K\|h;{}$\6
\&{while} ${}(\|j\I\|h){}$\5
${}\{{}$\1\6
${}\|i\K\\{rank}[\|S[\|j]];{}$\6
${}\|k\K\|L[\|j];{}$\6
${}\|L[\|j]\K\\{hd}[\|i];{}$\6
\&{if} ${}(\|L[\|j]<\T{0}){}$\1\5
${}\\{tl}[\|i]\K\|j;{}$\2\6
${}\\{hd}[\|i]\K\|j;{}$\6
${}\|j\K\|k;{}$\6
\4${}\}{}$\2\6
\&{for} ${}(\|i\K\T{100};{}$ ${}\|i\Z\T{100}+\|n;{}$ ${}\|i\PP){}$\1\6
\&{if} ${}(\\{hd}[\|i]\G\T{0}){}$\1\5
${}\|L[\\{tl}[\|i]]\K\|L[\|h],\39\|L[\|h]\K\\{hd}[\|i];{}$\2\2\6
\4${}\}{}$\2\par
\fi

\M{17}The parameter \PB{\\{card}} is 100 plus the cardinality of \PB{\|m}
in the following subroutine.

\Y\B\4\X8:Subroutines\X${}\mathrel+\E{}$\6
\&{void} \\{unmark}(\&{int} \|m${},\39{}$\&{int} \\{card})\1\1\2\2\6
${}\{{}$\1\6
\&{register} \|t${},{}$ \|v;\7
\&{if} ${}(\\{rank}[\|m]<\T{100}){}$\5
${}\{{}$\1\6
${}\\{rank}[\|m]\K\\{card};{}$\6
\&{for} ${}(\|t\K\\{mask}-\|m;{}$ \|t; ${}\|t\K\|v){}$\5
${}\{{}$\1\6
${}\|v\K\|t\AND(\|t-\T{1});{}$\6
${}\\{unmark}(\|m+\|t-\|v,\39\\{card}+\T{1});{}$\6
\4${}\}{}$\2\6
\4${}\}{}$\2\6
\4${}\}{}$\2\par
\fi

\M{18}\B\X8:Subroutines\X${}\mathrel+\E{}$\6
\&{void} \\{print\_circuits}(\&{void})\1\1\2\2\6
${}\{{}$\1\6
\&{register} \&{int} \|i${},{}$ \|k;\7
\\{printf}(\.{"The\ circuits\ are:"});\6
\&{for} ${}(\|k\K\T{1};{}$ ${}\|k\Z\\{mask};{}$ ${}\|k\MRL{+{\K}}\|k){}$\1\6
\&{for} ${}(\|i\K\T{0};{}$ ${}\|i<\|k;{}$ ${}\|i\PP){}$\1\6
\&{if} ${}(\\{rank}[\|k+\|i]\E\\{rank}[\|i]){}$\5
${}\{{}$\1\6
${}\\{print\_set}(\|k+\|i);{}$\6
${}\\{unmark}(\|k+\|i,\39\\{rank}[\|i]+\T{101});{}$\6
\4${}\}{}$\2\2\2\6
\\{printf}(\.{"\\n"});\6
\4${}\}{}$\2\par
\fi

\N{1}{19}Index.

\fi


\inx
\fin
\con
