\input cwebmac
\srcloctrue
\datethis




\N[4 maxcliques.w]{1}{1}Intro. This is a simple program to find all maximal
cliques of a
graph, when the graph has at most 64 vertices and is given by
its adjacency matrix.

The algorithm, due to Moody and Hollis, is described in Appendix 5
of the book {\sl Mathematical Taxonomy\/} by Jardine and Sibson (1971).
I'm writing this program hastily today because it's a nice example of
bitwise manipulation in graph theory, probably suitable for Section
7.1.3 of {\sl The Art of Computer Programming}.

In brief, we have vectors $\rho_v$ and $\delta_v$ for each vertex~$v$,
where $\rho_v$ is $v$'s row in the adjacency matrix and
$\delta_v$ is all 1s except for 0 in position~$v$. (Position~$v$
in $\rho_v$ is always~1; in other words, we assume that each vertex
is adjacent to itself.) One can easily show that a set of
vertices~$q$ is a clique if and only if $q$ is the bitwise intersection
of $(v{\in}q?\,\rho_v{:}\,\delta_v)$ over all vertices~$v$.
To find all cliques, we could compute all $2^n$ such bitwise
intersections, discarding duplicates. To find all maximal cliques,
we could compute all $2^n$ such bitwise intersections and discard
any clique $q$ that is found to be contained in another. To speed
the process up, we can do those discards much more cleverly.

The input graph is specified by an ASCII file in Stanford GraphBase format.
The name of that file, say `\.{foo.gb}',
is the one-and-only command-line parameter.

I've instrumented this program to see how many memory references
it makes (\PB{\\{mems}}) and how many words of workspace it needs (\PB{%
\\{space}}),
exclusive of input-output.

(Note: I consider a ``clique'' to be any complete subgraph of
a graph. Many of the older books on graph theory define it to
be a {\it maximal\/} complete subgraph; but that terminology
is now dying out. The earlier definition is less desirable, because
for example we'd like a clique of~$G$ to be equivalent to
an independent set of $\overline G$.)

Beware: Memory overflow is not checked. This is not intended to
be a robust program; I wrote it only as a experiment.

\Y\B\4\D$\\{size}$ \5
\T{100000}\C{ size reserved for the workspace }\par
\B\4\D$\|o$ \5
$\\{mems}\PP{}$\par
\B\4\D$\\{oo}$ \5
$\\{mems}\MRL{+{\K}}{}$\T{2}\par
\Y\B\8\#\&{include} \.{<stdio.h>}\6
\8\#\&{include} \.{"gb\_graph.h"}\6
\8\#\&{include} \.{"gb\_save.h"}\6
\&{unsigned} \&{long} \&{long} \\{rho}[\T{64}];\C{ rows of the adjacency matrix
}\6
\&{unsigned} \&{long} \&{long} \\{work}[\\{size}];\6
\&{unsigned} \&{int} \\{mems};\6
\&{int} \\{space};\6
\&{char} \\{table}[\T{64}];\7
\\{main}(\&{int} \\{argc}${},\39{}$\&{char} ${}{*}\\{argv}[\,]){}$\1\1\2\2\6
${}\{{}$\1\6
\&{register} \&{int} \|i${},{}$ \|j${},{}$ \|k${},{}$ \|l${},{}$ \|m${},{}$ %
\|n${},{}$ \|p${},{}$ \|q${},{}$ \|r;\6
\&{register} \&{Graph} ${}{*}\|g;{}$\6
\&{register} \&{Arc} ${}{*}\|a;{}$\6
\&{unsigned} \&{long} \&{long} \|u${},{}$ \|v${},{}$ \|w;\7
\X2:Input the graph\X;\6
\X6:Find the maximal cliques\X;\6
\X4:Output the maximal cliques\X;\6
${}\\{printf}(\.{"(The\ computation\ to}\)\.{ok\ \%u\ mems,\ using\ \%d}\)\.{\
words\ of\ workspace.}\)\.{)\\n"},\39\\{mems},\39\\{space});{}$\6
\4${}\}{}$\2\par
\fi

\M[71 maxcliques.w]{2}\B\X2:Input the graph\X${}\E{}$\6
\&{if} ${}(\\{argc}\I\T{2}){}$\5
${}\{{}$\1\6
${}\\{fprintf}(\\{stderr},\39\.{"Usage:\ \%s\ inputgrap}\)\.{h.gb\\n"},\39%
\\{argv}[\T{0}]);{}$\6
${}\\{exit}({-}\T{1});{}$\6
\4${}\}{}$\2\6
${}\|g\K\\{restore\_graph}(\\{argv}[\T{1}]);{}$\6
\&{if} ${}(\R\|g){}$\5
${}\{{}$\1\6
${}\\{fprintf}(\\{stderr},\39\.{"I\ can't\ input\ the\ g}\)\.{raph\ \%s\ (panic%
\ code\ }\)\.{\%ld)!\\n"},\39\\{argv}[\T{1}],\39\\{panic\_code});{}$\6
${}\\{exit}({-}\T{2});{}$\6
\4${}\}{}$\2\6
${}\|n\K\|g\MG\|n;{}$\6
\&{if} ${}(\|n>\T{64}){}$\5
${}\{{}$\1\6
${}\\{fprintf}(\\{stderr},\39\.{"Sorry,\ that\ graph\ h}\)\.{as\ \%d\ vertices;%
\ "},\39\|n);{}$\6
${}\\{fprintf}(\\{stderr},\39\.{"I\ can't\ handle\ more}\)\.{\ than\ 64!%
\\n"});{}$\6
${}\\{exit}({-}\T{3});{}$\6
\4${}\}{}$\2\6
\X3:Set up the \PB{\\{rho}} table\X;\par
\U1.\fi

\M[90 maxcliques.w]{3}\B\X3:Set up the \PB{\\{rho}} table\X${}\E{}$\6
\&{for} ${}(\|j\K\T{0};{}$ ${}\|j<\|n;{}$ ${}\|j\PP){}$\5
${}\{{}$\1\6
${}\|w\K\T{1\$L\$L}\LL\|j;{}$\6
\&{for} ${}(\|a\K(\|g\MG\\{vertices}+\|j)\MG\\{arcs};{}$ \|a; ${}\|a\K\|a\MG%
\\{next}){}$\1\5
${}\|w\MRL{{\OR}{\K}}\T{1\$L\$L}\LL(\|a\MG\\{tip}-\|g\MG\\{vertices});{}$\2\6
${}\\{rho}[\|j]\K\|w;{}$\6
\4${}\}{}$\2\par
\U2.\fi

\M[97 maxcliques.w]{4}\B\D$\\{deBruijn}$ \5
\T{\^03f79d71b4ca8b09}\C{ the least de Bruijn cycle of length 64 }\par
\Y\B\4\X4:Output the maximal cliques\X${}\E{}$\6
$\\{printf}(\.{"Graph\ \%s\ has\ \%d\ max}\)\.{imal\ cliques:\\n"},\39\|g\MG%
\\{id},\39\|m);{}$\6
\X5:Set up the de Bruijn table\X;\6
\&{for} ${}(\|k\K\T{1};{}$ ${}\|k\Z\|m;{}$ ${}\|k\PP){}$\5
${}\{{}$\1\6
\&{for} ${}(\|w\K\\{work}[\|k];{}$ \|w; ${}\|w\K\|w\XOR\|v){}$\5
${}\{{}$\1\6
${}\|v\K\|w\AND{-}\|w;{}$\6
${}\|u\K\|v*\\{deBruijn};{}$\6
${}\|j\K\\{table}[\|u\GG\T{58}];{}$\6
${}\\{printf}(\.{"\ \%s"},\39(\|g\MG\\{vertices}+\|j)\MG\\{name});{}$\6
\4${}\}{}$\2\6
\\{printf}(\.{"\\n"});\6
\4${}\}{}$\2\par
\U1.\fi

\M[112 maxcliques.w]{5}The ruler-function calculation in the previous section
isn't part
of the inner loop; so I could use a slower, brute-force scheme
that simply examines each bit, one bit at a time.
But I'm using the de Bruijn method anyway, in order to get
experience with it.

\Y\B\4\X5:Set up the de Bruijn table\X${}\E{}$\6
\&{for} ${}(\|j\K\T{0},\39\|v\K\T{1};{}$ \|v; ${}\|j\PP,\39\|v\MRL{{\LL}{\K}}%
\T{1}){}$\5
${}\{{}$\1\6
${}\|u\K\|v*\\{deBruijn};{}$\6
${}\\{table}[\|u\GG\T{58}]\K\|j;{}$\6
\4${}\}{}$\2\par
\U4.\fi

\N[124 maxcliques.w]{1}{6}The algorithm. The \PB{\\{work}} area holds all
maximal elements of the
$2^i$ bitwise-ands of either $\rho_v$ or $\delta_v$ for the first
$i$ vertices. There are $m$ of them.

A vertex that's adjacent to all other vertices is just carried
along in all entries, so we needn't bother with it.

\Y\B\4\X6:Find the maximal cliques\X${}\E{}$\6
$\|w\K\T{1\$L\$L}\LL(\|n-\T{1});{}$\6
${}\\{oo},\39\\{work}[\T{1}]\K\\{work}[\\{size}-\T{1}]\K(\|w\LL\T{1})-\T{1}{}$;%
\C{ $n$ 1s }\6
${}\|m\K\T{1},\39\\{space}\K\T{4};{}$\6
\&{for} ${}(\|i\K\T{0};{}$ ${}\|i<\|n;{}$ ${}\|i\PP){}$\1\6
\&{if} ${}(\\{oo},\39\\{rho}[\|i]\I\\{work}[\\{size}-\T{1}]){}$\5
${}\{{}$\1\6
${}\|v\K\T{1\$L\$L}\LL\|i{}$;\C{ this is the new vertex we're considering }\6
\X7:Partition the \PB{\\{work}} area, putting entries that contain~$v$ last\X;\6
\X8:Convert the $v$-containing entries $u$ into $u\AND\rho_v$, $u\AND\delta_v$%
\X;\6
\4${}\}{}$\2\2\par
\U1.\fi

\M[142 maxcliques.w]{7}To visualize the current situation, suppose bit~$v$ is
the leftmost.
Then we want to rearrange \PB{\\{work}[\T{1}]} thru \PB{\\{work}[\|m]} so that
they have the form
$$\vcenter{\halign{#\thinspace&$#$\hfil\cr
0&\alpha_1\cr
&\vdots\cr
0&\alpha_k\cr
1&\beta_1\cr
&\vdots\cr
1&\beta_{m-k}\cr}}$$

This loop is like the partitioning step of radix-exchange.
There always is at least one element $u$ with $u\AND v\ne0$.
For convenience (and speed) we keep \PB{$\\{work}[\T{0}]\K\T{0}$}.

\Y\B\4\X7:Partition the \PB{\\{work}} area, putting entries that contain~$v$
last\X${}\E{}$\6
$\|j\K\T{1},\39\|k\K\|m;{}$\6
\&{while} (\T{1})\5
${}\{{}$\1\6
\&{while} ${}((\|o,\39\\{work}[\|j]\AND\|v)\E\T{0}){}$\1\5
${}\|j\PP;{}$\2\6
\&{while} ${}((\|o,\39\\{work}[\|k]\AND\|v)\I\T{0}){}$\1\5
${}\|k\MM;{}$\2\6
\&{if} ${}(\|j>\|k){}$\1\5
\&{break};\2\6
${}\\{oo},\39\|u\K\\{work}[\|j],\39\\{work}[\|j]\K\\{work}[\|k],\39\\{work}[%
\|k]\K\|u;{}$\6
${}\|j\PP,\39\|k\MM;{}$\6
\4${}\}{}$\2\par
\U6.\fi

\M[166 maxcliques.w]{8}Now comes the interesting part. At this point $j=k+1$.

Entries 1 thru $k$ of the workspace should simply be carried
over to the next round. For if $u=0\,\alpha_i$ in the notation above,
we have $u\AND\rho_i\subseteq u\AND\delta_i=u$; and $u$~(which is
currently maximal) won't be contained in any other entries.

Thus we focus our attention on the remaining entries in the current workspace,
namely \PB{\\{work}[\|j]} thru \PB{\\{work}[\|m]}, which need to be ``split.''

Let $u$ be the current entry; we want to split it into
$u'=u\AND\rho_v$ and $u''=u\AND\delta_v$. The first one,
$u'=(1\beta)\AND\rho_v$, must be checked
against all other first entries before it is accepted for the
new round. (It can't be contained in a second entry, because it has 1~in
position~$v$.)
If it is contained in a first entry already generated, we drop it.
But if it contains one of those entries, we might need to drop
more than one of them.

The second entry, $u''=(1\beta)\AND\delta_v=0\beta$,
is fairly easy to deal with. It can be
contained in some first entry $(1\beta')\AND\rho_v$ only if
$\beta\subseteq\beta'$; hence $\beta=\beta'$ and $1\beta\subseteq\rho_v$.
If $1\beta\not\subseteq\rho_v$, we accept $0\beta$ if and only if it
isn't contained in any held-over entry $0\alpha$.

In the following steps, positions $p$ thru \PB{$\\{size}-\T{2}$} of the
workspace
hold the first entries $u'$ that have been tentatively accepted.
Positions $k+1$ thru $l$ hold the second entries $u'$ that have
definitely been accepted.

Notice that the resulting algorithm avoids linked memory, so it
ought to be cache-friendly.

\Y\B\4\X8:Convert the $v$-containing entries $u$ into $u\AND\rho_v$, $u\AND%
\delta_v$\X${}\E{}$\6
\&{for} ${}(\|l\K\|k,\39\|p\K\\{size}-\T{1};{}$ ${}\|j\Z\|m;{}$ ${}\|j\PP){}$\5
${}\{{}$\1\6
${}\|o,\39\|u\K\\{work}[\|j],\39\|q\K\\{size}-\T{2};{}$\6
${}\|w\K\|u\AND\\{rho}[\|i]{}$;\C{ $w=u'$; we've already fetched \PB{\\{rho}[%
\|i]} from memory }\6
\&{if} ${}(\|u\I\|w){}$\5
${}\{{}$\1\6
\&{for} ( ; ${}\|q\G\|p;{}$ ${}\|q\MM){}$\5
${}\{{}$\1\6
\&{if} ${}((\|o,\39\|w\AND\\{work}[\|q])\E\|w){}$\1\5
\&{goto} \\{second\_entry};\2\6
\&{if} ${}((\|w\AND\\{work}[\|q])\E\\{work}[\|q]){}$\1\5
\&{goto} \\{absorb};\2\6
\4${}\}{}$\2\6
${}\|o,\39\\{work}[\MM\|p]\K\|w{}$;\C{ accept $u'$, tentatively }\6
\&{if} ${}(\\{space}<\|m+\T{2}+\\{size}-\|p){}$\1\5
${}\\{space}\K\|m+\T{2}+\\{size}-\|p;{}$\2\6
\&{goto} \\{second\_entry};\6
\4${}\}{}$\2\6
\4\\{absorb}:\5
\X9:Handle the cases where $w$ may absorb previous first entries\X;\6
\&{if} ${}(\|u\E\|w){}$\1\5
\&{continue};\2\6
\4\\{second\_entry}:\5
${}\|w\K\|u\AND\CM\|v{}$;\C{ $w=u''$ }\6
\&{for} ${}(\|q\K\T{1};{}$ ${}\|q\Z\|k;{}$ ${}\|q\PP){}$\1\6
\&{if} ${}((\|o,\39\|w\AND\\{work}[\|q])\E\|w){}$\1\5
\&{goto} \\{done\_with\_u};\2\2\6
${}\|o,\39\\{work}[\PP\|l]\K\|w{}$;\C{ accept $u''$ }\6
\4\\{done\_with\_u}:\5
\&{continue};\6
\4${}\}{}$\2\6
\&{for} ${}(\|m\K\|l;{}$ ${}\|p<\\{size}-\T{1};{}$ ${}\|p\PP){}$\1\5
${}\\{oo},\39\\{work}[\PP\|m]\K\\{work}[\|p]{}$;\2\par
\U6.\fi

\M[223 maxcliques.w]{9}Finally, we need another loop analogous to
radix-exchange partitioning.
But this time we will move all entries contained in~\PB{\|w} down, and
all entries not contained in~\PB{\|w} up. (In fact we don't really
move anything down, because those entries will be discarded.)

When we come here with $u\ne w$, the first test made is
redundant. (I mean, we know that $w\AND\PB{\\{work}}[q]=\PB{\\{work}}[q]$,
hence
$w\OR\PB{\\{work}}[q]=w$.) I could avoid that by reordering the loop, and
recopying part of it to be down only when $u=w$;
but I decided not to bother with such a tricky optimization.

\Y\B\4\X9:Handle the cases where $w$ may absorb previous first entries\X${}%
\E{}$\6
$\|o,\39\|r\K\|p,\39\\{work}[\|p-\T{1}]\K\T{0};{}$\6
\&{while} (\T{1})\5
${}\{{}$\1\6
\&{while} ${}(\|o,\39(\|w\OR\\{work}[\|q])\I\|w){}$\1\5
${}\|q\MM;{}$\2\6
\&{while} ${}(\|o,\39(\|w\OR\\{work}[\|r])\E\|w){}$\1\5
${}\|r\PP;{}$\2\6
\&{if} ${}(\|q<\|r){}$\1\5
\&{break};\2\6
${}\\{oo},\39\\{work}[\|q]\K\\{work}[\|r],\39\\{work}[\|r]\K\T{0},\39\|q\MM,\39%
\|r\PP;{}$\6
\4${}\}{}$\2\6
${}\|o,\39\\{work}[\|q]\K\|w,\39\|p\K\|q{}$;\par
\U8.\fi

\N[244 maxcliques.w]{1}{10}Index.
\fi

\inx
\fin
\con
