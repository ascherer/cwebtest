\input cwebmac
\datethis


\N{1}{1}Intro. This program is designed to compose multiplication-skeleton
puzzles
of a type pioneered by Junya Take. For example, consider his puzzle for
the lettter \.O, in {\sl Journal of Recreational Mathematics\/ \bf38} (2014),
132:
$$\vcenter{\halign{\hfil\tt#\cr
.......\cr
$\times$\hfil......\cr
\noalign{\medskip\hrule\medskip}
........\cr
OO.....\ \cr
..O..O..\ \ \ \cr
...O..O.\ \ \ \ \cr
...O..O\ \ \ \ \ \cr
\noalign{\medskip\hrule\medskip}
....OO......\cr}}$$
Each occurrence of `\.O' should be replaced by some digit~$d$, and
each `\..' should be replaced by a digit $\ne d$. (And no zero should be
in a most significant position.) The solution is unique:
$$\vcenter{\halign{\hfil\tt#\cr
2208068\cr
$\times$\hfil357029\cr
\noalign{\medskip\hrule\medskip}
19872612\cr
4416136\ \cr
15456476\ \ \ \cr
11040340\ \ \ \ \cr
6624204\ \ \ \ \ \cr
\noalign{\medskip\hrule\medskip}
788344309972\cr}}$$
But the purpose of this program is not to {\it solve\/} such a puzzle!
The purpose of this program is to {\it invent\/} such a puzzle, namely to
find integers $x$ and $y$ whose partial
products and final product have digits that match a given binary pattern.

The pattern is given in \PB{\\{stdin}} as a set of lines, with asterisks
marking
the position of the special digit. For example, the `\.O' shape in the
puzzle above would be specified thus:
$$\vcenter{\halign{\hfill\tt#\cr
.**.\cr
*..*\cr
*..*\cr
*..*\cr
.**.\cr}}$$

\fi

\M{2}The examples above show that zeros in the multiplier will ``offset''
the shape in different ways. We try all possible offsets, for a
given number $m$ of nonzero multiplier digits.

A second parameter, $z$, specifies the maximum number of zeros
in the multiplier. Both $m$ and $z$ are specified on the command line.

\Y\B\4\D$\\{maxdigs}$ \5
\T{22}\C{ size of the longest numbers considered, plus 2 }\par
\B\4\D$\\{maxdim}$ \5
\T{8}\C{ maximum size of pattern }\par
\B\4\D$\\{bufsize}$ \5
$\\{maxdim}+{}$\T{5}\par
\B\4\D$\\{maxm}$ \5
\T{8}\C{ \PB{\|m} must be less than this }\par
\B\4\D$\|o$ \5
$\\{mems}\PP{}$\par
\B\4\D$\\{oo}$ \5
$\\{mems}\MRL{+{\K}}{}$\T{2}\par
\Y\B\8\#\&{include} \.{<stdio.h>}\6
\8\#\&{include} \.{<stdlib.h>}\6
\X6:Typedefs\X;\7
\&{int} \|m;\C{ the number of nonzero digits in the multiplier }\6
\&{int} \|z;\C{ the maximum number of zero digits in the multiplier }\6
\&{int} \\{vbose};\C{ level of verbosity }\6
\&{char} \\{buf}[\\{bufsize}];\C{ buffer used when inputting the shape }\6
\&{char} \\{rawpat}[\\{maxdim}][\\{maxdim}];\C{ pixels of the raw pattern }\6
\&{char} \\{last}[\\{maxdim}];\C{ positions of the rightmost asterisks }\6
\&{int} \\{count};\C{ this many solutions found }\6
\&{unsigned} \&{long} \&{long} \\{nodes};\C{ size of the backtrack trees, times
10 }\6
\&{int} \\{unresolved};\C{ this many cases left unresolved }\6
\&{unsigned} \&{long} \&{long} \\{mems};\C{ memory accesses }\7
\X11:Global variables\X;\6
\X7:Subroutines\X;\7
\\{main}(\&{int} \\{argc}${},\39{}$\&{char} ${}{*}\\{argv}[\,]){}$\1\1\2\2\6
${}\{{}$\1\6
\&{register} \&{int} \|d${},{}$ \|i${},{}$ \\{ii}${},{}$ \\{imax}${},{}$ %
\|j${},{}$ \\{jj}${},{}$ \|k${},{}$ \\{kk}${},{}$ \|l${},{}$ \\{lc}${},{}$ %
\\{lj}${},{}$ \|n${},{}$ \|t${},{}$ \\{tt}${},{}$ \|x${},{}$ \\{pos}${},{}$ %
\\{maxl}${},{}$ \\{printed};\7
\X3:Process the command line\X;\6
\X4:Input the pattern\X;\6
\X10:Build the table of constants\X;\6
\X13:Establish the minimum offsets\X;\6
\&{while} (\T{1})\5
${}\{{}$\1\6
\X18:Create detailed specifications from the pattern\X;\6
\&{for} ${}(\|d\K\T{0};{}$ ${}\|d<\T{10};{}$ ${}\|d\PP){}$\5
${}\{{}$\1\6
\&{if} (\\{vbose})\1\5
${}\\{fprintf}(\\{stderr},\39\.{"\ *=\%d:\\n"},\39\|d);{}$\2\6
\X20:Find all solutions for the current offsets and special digit $d$\X;\6
\4${}\}{}$\2\6
\X14:Advance to the next offset, or \PB{\&{break}} if it needs too many zeros%
\X;\6
\4${}\}{}$\2\6
${}\\{fprintf}(\\{stderr},\39\.{"Altogether\ \%d\ solut}\)\.{ions,\ \%lld\
nodes,\ \%l}\)\.{ld\ mems.\\n"},\39\\{count},\39\\{nodes}/\T{10},\39%
\\{mems});{}$\6
\&{if} (\\{unresolved})\1\5
${}\\{fprintf}(\\{stderr},\39\.{"...\ \%d\ cases\ were\ u}\)\.{nresolved!\\n"},%
\39\\{unresolved});{}$\2\6
\4${}\}{}$\2\par
\fi

\M{3}\B\X3:Process the command line\X${}\E{}$\6
\&{if} ${}(\\{argc}<\T{3}\V\\{sscanf}(\\{argv}[\T{1}],\39\.{"\%d"},\39{\AND}%
\|m)\I\T{1}\V\\{sscanf}(\\{argv}[\T{2}],\39\.{"\%d"},\39{\AND}\|z)\I\T{1}){}$\5
${}\{{}$\1\6
${}\\{fprintf}(\\{stderr},\39\.{"Usage:\ \%s\ m\ z\ [verb}\)\.{ose]\
[extraverbose]\ }\)\.{<\ foo.dots\\n"},\39\\{argv}[\T{0}]);{}$\6
${}\\{exit}({-}\T{1});{}$\6
\4${}\}{}$\2\6
\&{if} ${}(\|m<\T{2}\V\|m\G\\{maxm}){}$\5
${}\{{}$\1\6
${}\\{fprintf}(\\{stderr},\39\.{"m\ should\ be\ between}\)\.{\ 2\ and\ \%d,\
not\ \%d!\\n}\)\.{"},\39\\{maxm}-\T{1},\39\|m);{}$\6
${}\\{exit}({-}\T{2});{}$\6
\4${}\}{}$\2\6
\&{if} ${}(\|m+\|z>\\{maxdigs}-\T{2}){}$\5
${}\{{}$\1\6
${}\\{fprintf}(\\{stderr},\39\.{"m+z\ should\ be\ at\ mo}\)\.{st\ \%d,\ not\ %
\%d!\\n"},\39\\{maxdigs}-\T{2},\39\|m+\|z);{}$\6
${}\\{exit}({-}\T{3});{}$\6
\4${}\}{}$\2\6
${}\\{vbose}\K\\{argc}-\T{3}{}$;\par
\U2.\fi

\M{4}\B\X4:Input the pattern\X${}\E{}$\6
\&{for} ${}(\|n\K\|k\K\T{0};{}$  ; ${}\|n\PP){}$\5
${}\{{}$\1\6
\&{if} ${}(\R\\{fgets}(\\{buf},\39\\{bufsize},\39\\{stdin})){}$\1\5
\&{break};\2\6
\&{if} ${}(\|n\G\\{maxdim}){}$\5
${}\{{}$\1\6
${}\\{fprintf}(\\{stderr},\39\.{"Recompile\ me:\ I\ all}\)\.{ow\ at\ most\ \%d\
lines\ }\)\.{of\ input!\\n"},\39\\{maxdim});{}$\6
${}\\{exit}({-}\T{3});{}$\6
\4${}\}{}$\2\6
\X5:Input row \PB{\|n} of the shape\X;\6
\4${}\}{}$\2\6
${}\\{fprintf}(\\{stderr},\39\.{"OK,\ I've\ got\ a\ patt}\)\.{ern\ with\ \%d\
rows\ and}\)\.{\ \%d\ asterisks.\\n"},\39\|n,\39\|k);{}$\6
\&{if} ${}(\|m<\|n-\T{1}){}$\5
${}\{{}$\1\6
${}\\{fprintf}(\\{stderr},\39\.{"So\ there\ must\ be\ at}\)\.{\ least\ \%d\
multiplier}\)\.{\ digits,\ not\ \%d!\\n"},\39\|n-\T{1},\39\|m);{}$\6
${}\\{exit}({-}\T{2});{}$\6
\4${}\}{}$\2\par
\U2.\fi

\M{5}\B\X5:Input row \PB{\|n} of the shape\X${}\E{}$\6
\&{for} ${}(\|j\K\T{0};{}$ ${}\\{buf}[\|j]\W\\{buf}[\|j]\I\.{'\\n'};{}$ ${}\|j%
\PP){}$\5
${}\{{}$\1\6
\&{if} ${}(\\{buf}[\|j]\E\.{'*'}){}$\5
${}\{{}$\1\6
\&{if} ${}(\|j\G\\{maxdim}){}$\5
${}\{{}$\1\6
${}\\{fprintf}(\\{stderr},\39\.{"Recompile\ me:\ I\ all}\)\.{ow\ at\ most\ \%d\
column}\)\.{s\ per\ row!\\n"},\39\\{maxdim});{}$\6
${}\\{exit}({-}\T{5});{}$\6
\4${}\}{}$\2\6
${}\\{oo},\39\\{rawpat}[\|n][\|j]\K\T{1},\39\|k\PP,\39\\{last}[\|n]\K\|j+%
\T{1};{}$\6
\4${}\}{}$\2\6
\4${}\}{}$\2\par
\U4.\fi

\N{1}{6}Bignums. We implement elementary decimal addition on nonnegative
integers.
Each integer is represented by an array of bytes, in which the first
byte specifies the number of significant digits, and the remaining bytes
specify the digits themselves (right to left).

\Y\B\4\X6:Typedefs\X${}\E{}$\6
\&{typedef} \&{char} \&{bignum}[\\{maxdigs}];\par
\U2.\fi

\M{7}For example, it's easy to test equality of two such bignums,
or to copy one to another.

\Y\B\4\X7:Subroutines\X${}\E{}$\6
\&{int} \\{isequal}(\&{bignum} \|a${},\39{}$\&{bignum} \|b)\1\1\2\2\6
${}\{{}$\1\6
\&{register} \&{int} \\{la}${}\K\|a[\T{0}],{}$ \|i;\7
\&{if} ${}(\\{oo},\39\\{la}\I\|b[\T{0}]){}$\1\5
\&{return} \T{0};\2\6
\&{for} ${}(\|i\K\T{1};{}$ ${}\|i\Z\\{la};{}$ ${}\|i\PP){}$\1\6
\&{if} ${}(\\{oo},\39\|a[\|i]\I\|b[\|i]){}$\1\5
\&{return} \T{0};\2\2\6
\&{return} \T{1};\6
\4${}\}{}$\2\7
\&{void} \\{copy}(\&{bignum} \|a${},\39{}$\&{bignum} \|b)\1\1\2\2\6
${}\{{}$\1\6
\&{register} \&{int} \\{lb}${}\K\|b[\T{0}],{}$ \|i;\7
\&{for} ${}(\|o,\39\|i\K\T{0};{}$ ${}\|i\Z\\{lb};{}$ ${}\|i\PP){}$\1\5
${}\\{oo},\39\|a[\|i]\K\|b[\|i];{}$\2\6
\4${}\}{}$\2\par
\As8\ET9.
\U2.\fi

\M{8}Here's the basic routine. It's OK to have \PB{$\|a\K\|b$} or \PB{$\|b\K%
\|c$}.
(But beware of \PB{$\|a\K\|c$}.)

\Y\B\4\X7:Subroutines\X${}\mathrel+\E{}$\6
\&{void} \\{add}(\&{bignum} \|a${},\39{}$\&{bignum} \|b${},\39{}$\&{bignum} %
\|c${},\39{}$\&{int} \|p)\1\1\2\2\6
${}\{{}$\C{ set $a=b+10^p c$ }\1\6
\&{register} \&{int} \\{lb}${}\K\|b[\T{0}],{}$ \\{lc}${}\K\|c[\T{0}],{}$ %
\|i${},{}$ \|k${},{}$ \|d;\7
\&{if} ${}(\\{oo},\39\\{lc}\E\T{0}){}$\5
${}\{{}$\1\6
${}\\{copy}(\|a,\39\|b);{}$\6
\&{return};\6
\4${}\}{}$\2\6
\&{for} ${}(\|i\K\T{1};{}$ ${}\|i\Z\|p\W\|i\Z\\{lb};{}$ ${}\|i\PP){}$\1\5
${}\\{oo},\39\|a[\|i]\K\|b[\|i];{}$\2\6
\&{for} ${}(\|k\K\T{0};{}$ ${}\|i\Z\\{lb}\V\|i\Z\\{lc}+\|p\V\|k;{}$ ${}\|i%
\PP){}$\5
${}\{{}$\1\6
${}\|d\K\|k+(\|i\Z\\{lb}\?\|o,\39\|b[\|i]:\T{0})+(\|i\Z\\{lc}+\|p\W\|i>\|p\?%
\|o,\39\|c[\|i-\|p]:\T{0});{}$\6
\&{if} ${}(\|d\G\T{10}){}$\1\5
${}\|k\K\T{1},\39\|d\MRL{-{\K}}\T{10}{}$;\5
\2\&{else}\1\5
${}\|k\K\T{0};{}$\2\6
${}\|o,\39\|a[\|i]\K\|d;{}$\6
\4${}\}{}$\2\6
${}\|o,\39\|a[\T{0}]\K\|i-\T{1};{}$\6
\&{if} ${}(\|i\G\\{maxdigs}){}$\5
${}\{{}$\1\6
${}\\{fprintf}(\\{stderr},\39\.{"Integer\ overflow,\ m}\)\.{ore\ than\ \%d\
digits!\\}\)\.{n"},\39\\{maxdigs}-\T{1});{}$\6
${}\\{exit}({-}\T{666});{}$\6
\4${}\}{}$\2\6
\&{if} ${}(\|a[\|a[\T{0}]]\E\T{0}){}$\1\5
${}\\{fprintf}(\\{stderr},\39\.{"why?\\n"});{}$\2\6
\4${}\}{}$\2\par
\fi

\M{9}\B\X7:Subroutines\X${}\mathrel+\E{}$\6
\&{void} \\{print\_bignum}(\&{bignum} \|a)\1\1\2\2\6
${}\{{}$\1\6
\&{register} \&{int} \|i${},{}$ \\{la}${}\K\|a[\T{0}];{}$\7
\&{if} ${}(\R\\{la}){}$\1\5
${}\\{fprintf}(\\{stderr},\39\.{"0"});{}$\2\6
\&{else}\1\6
\&{for} ${}(\|i\K\\{la};{}$ \|i; ${}\|i\MM){}$\1\5
${}\\{fprintf}(\\{stderr},\39\.{"\%d"},\39\|a[\|i]);{}$\2\2\6
\4${}\}{}$\2\par
\fi

\M{10}We might as well have a primitive multiplication table.

\Y\B\4\X10:Build the table of constants\X${}\E{}$\6
$\|o,\39\\{cnst}[\T{0}][\T{0}]\K\T{0};{}$\6
\&{for} ${}(\|k\K\T{1};{}$ ${}\|k<\T{10};{}$ ${}\|k\PP){}$\1\5
${}\\{oo},\39\\{cnst}[\|k][\T{0}]\K\T{1},\39\\{cnst}[\|k][\T{1}]\K\|k;{}$\2\6
\&{for} ( ; ${}\|k\Z\T{81};{}$ ${}\|k\PP){}$\1\5
${}\\{oo},\39\|o,\39\\{cnst}[\|k][\T{0}]\K\T{2},\39\\{cnst}[\|k][\T{2}]\K\|k/%
\T{10},\39\\{cnst}[\|k][\T{1}]\K\|k\MOD\T{10}{}$;\2\par
\U2.\fi

\M{11}\B\X11:Global variables\X${}\E{}$\6
\&{bignum} \\{cnst}[\T{82}];\par
\As17, 21\ETs40.
\U2.\fi

\N{1}{12}Offsets and constraints.
The $k$th partial product, for $0\le k\le m$, will be shifted left
by \PB{\\{off}[\|k]}. (When $k=m$ this is the entire product, the sum of the
shifted partials.) It inherits the constraints of row $k-(m+1-n)$ of
the $n$-row pattern in \PB{\\{rawpat}}.

The data in \PB{\\{rawpat}} appears ``left to right,'' but the constraints
on digits are ``right to left.'' I mean, column~0 in \PB{\\{rawpat}} refers
to the most significant digit that is constrained.

The constraints on a partial product $({}\ldots p_2p_1p_0)_{10}$ say that
$p_i=d$ for certain~$i$, while $p_i\ne d$ for the others. We represent
them as a bignum, with 1 in the ``\PB{\|d}'' positions and 0~elsewhere.

For example, the opening problem in the introduction has $m=5$, $z=1$,
offsets (0, 1, 3, 4, 5), and constraints
(0, 1100000, 100100, 10010, 1001, 11000000).

We do not constrain the length of the multiplicand or the partial products;
we simply require that any digits to the left of explicitly
constrained positions must differ from~\PB{\|d}. This produces multiple
potential puzzles, some of which won't have unique solutions.

\fi

\M{13}\B\X13:Establish the minimum offsets\X${}\E{}$\6
\&{for} ${}(\|i\K\T{0};{}$ ${}\|i<\|m;{}$ ${}\|i\PP){}$\1\5
${}\|o,\39\\{off}[\|i]\K\|i{}$;\2\par
\U2.\fi

\M{14}The offset table runs through all combinations $s_0<s_1<\cdots<s_{m-1}$
with $s_0=0$ and $s_{m-1}<m+z$, in lexicographic order.

\Y\B\4\X14:Advance to the next offset, or \PB{\&{break}} if it needs too many
zeros\X${}\E{}$\6
\&{for} ${}(\|i\K\|m-\T{1};{}$ ${}\|i>\T{0};{}$ ${}\|i\MM){}$\1\6
\&{if} ${}(\|o,\39\\{off}[\|i]<\|i+\|z){}$\1\5
\&{break};\2\2\6
\&{if} ${}(\|i\E\T{0}){}$\1\5
\&{break};\2\6
${}\|o,\39\\{off}[\|i]\PP;{}$\6
\&{for} ${}(\|i\PP;{}$ ${}\|i<\|m;{}$ ${}\|i\PP){}$\1\5
${}\\{oo},\39\\{off}[\|i]\K\\{off}[\|i-\T{1}]+\T{1}{}$;\2\par
\U2.\fi

\M{15}We must choose the position \PB{\\{pos}} where column 0 of the raw
pattern
will appear in the final product. Then column~\PB{\|j} of the \PB{\|k}th
partial
product will be in position \PB{$\\{pos}-\\{off}[\|k]-\|j$}.

In the rightmost (smallest) setting of \PB{\\{pos}}, at least one of the
constraints will end with~1. A~harder puzzle is obtained if \PB{\\{pos}}
exceeds this minimum. This program sets \PB{\\{pos}} to the minimum possible,
plus a compile-time parameter called \PB{\\{slack}}. Junja Take has published
several examples with \PB{$\\{slack}\K\T{1}$}, and I~want to explore such
cases;
however, the default version of this program sets \PB{$\\{slack}\K\T{0}$}.

\Y\B\4\D$\\{slack}$ \5
\T{0}\C{ amount to shift the pattern left in harder problems }\par
\Y\B\4\X15:Choose \PB{\\{pos}}\X${}\E{}$\6
\&{for} ${}(\|i\K\\{pos}\K\T{0};{}$ ${}\|i\Z\|m;{}$ ${}\|i\PP){}$\1\6
\&{if} ${}(\\{oo},\39\\{off}[\|m+\T{1}-\|n+\|i]+\\{last}[\|i]>\\{pos}){}$\1\5
${}\\{pos}\K\\{off}[\|m+\T{1}-\|n+\|i]+\\{last}[\|i];{}$\2\2\6
${}\\{pos}\MRL{+{\K}}\\{slack}-\T{1}{}$;\par
\U18.\fi

\M{16}Sometimes two constraints are identical, and we'll want to know that
fact.
So we set up a table called \PB{\\{id}}, where \PB{$\\{id}[\|j]\K\\{id}[\|k]$}
if and only if
$c_j=c_k$.

\Y\B\4\X16:Set up the constraints\X${}\E{}$\6
\&{for} ${}(\|k\K\\{ids}\K\T{0};{}$ ${}\|k\Z\|m;{}$ ${}\|k\PP){}$\5
${}\{{}$\1\6
${}\|o,\39\|i\K\|k-(\|m+\T{1}-\|n),\39\\{constr}[\|k][\T{0}]\K\T{0};{}$\6
\&{if} ${}(\|i\G\T{0}){}$\5
${}\{{}$\1\6
\&{for} ${}(\\{oo},\39\|j\K\\{pos}-\\{off}[\|k]-\\{last}[\|i]+\T{1};{}$ ${}\|j%
\G\T{0};{}$ ${}\|j\MM){}$\1\5
${}\|o,\39\\{constr}[\|k][\|j]\K\T{0};{}$\2\6
\&{for} ${}(\|o,\39\|j\K\\{last}[\|i]-\T{1};{}$ ${}\|j\G\T{0};{}$ ${}\|j\MM){}$%
\5
${}\{{}$\1\6
\&{if} ${}(\|o,\39\\{rawpat}[\|i][\|j]){}$\1\5
${}\\{oo},\39\|o,\39\\{constr}[\|k][\\{pos}-\\{off}[\|k]-\|j+\T{1}]\K\T{1},\39%
\\{constr}[\|k][\T{0}]\K\\{pos}-\\{off}[\|k]-\|j+\T{1};{}$\2\6
\&{else}\1\5
${}\\{oo},\39\\{constr}[\|k][\\{pos}-\\{off}[\|k]-\|j+\T{1}]\K\T{0};{}$\2\6
\4${}\}{}$\2\6
\4${}\}{}$\2\6
\&{for} ${}(\|j\K\|k-\T{1};{}$ ${}\|j\G\T{0};{}$ ${}\|j\MM){}$\1\6
\&{if} ${}(\\{oo},\39\\{isequal}(\\{constr}[\|j],\39\\{constr}[\|k])){}$\1\5
\&{break};\2\2\6
\&{if} ${}(\|j\G\T{0}){}$\1\5
${}\\{oo},\39\\{id}[\|k]\K\\{id}[\|j]{}$;\5
\2\&{else}\1\5
${}\|o,\39\\{id}[\|k]\K\\{ids}\PP;{}$\2\6
\4${}\}{}$\2\par
\U18.\fi

\M{17}\B\X11:Global variables\X${}\mathrel+\E{}$\6
\&{char} \\{off}[\\{maxm}];\C{ blanks at right of partial products }\6
\&{bignum} \\{constr}[\\{maxm}];\C{ the constraint patterns, decimalized }\6
\&{char} \\{id}[\\{maxm}];\C{ equivalence class number for a given constraint }%
\6
\&{char} \\{ids};\C{ how many classes are there? }\par
\fi

\M{18}\B\X18:Create detailed specifications from the pattern\X${}\E{}$\6
${}\{{}$\1\6
\X15:Choose \PB{\\{pos}}\X;\6
\X16:Set up the constraints\X;\6
\&{if} (\\{vbose})\5
${}\{{}$\1\6
${}\\{fprintf}(\\{stderr},\39\.{"Constraints\ for\ off}\)\.{sets"});{}$\6
\&{for} ${}(\|k\K\T{0};{}$ ${}\|k\Z\|m;{}$ ${}\|k\PP){}$\1\5
${}\\{fprintf}(\\{stderr},\39\.{"\ \%d"},\39\\{off}[\|k]);{}$\2\6
${}\\{fprintf}(\\{stderr},\39\.{":"});{}$\6
\&{for} ${}(\|k\K\T{0};{}$ ${}\|k\Z\|m;{}$ ${}\|k\PP){}$\5
${}\{{}$\1\6
${}\\{fprintf}(\\{stderr},\39\.{"\ "});{}$\6
\\{print\_bignum}(\\{constr}[\|k]);\6
\4${}\}{}$\2\6
${}\\{fprintf}(\\{stderr},\39\.{".\\n"});{}$\6
\4${}\}{}$\2\6
\4${}\}{}$\2\par
\U2.\fi

\N{1}{19}Backtracking.
Let the multiplicand be $(a_l\ldots a_2a_1a_0)_{10}$. We proceed by
trying all possibilities $\ne d$ for $a_0$, then all possibilities
consistent with $a_0$ for $a_1$, and so on. The upper limit on~$l$
is $\PB{\\{maxdigs}}-2-s_{m-1}$, because of our limit on the size of bignums;
but I~doubt if we'll often get really big solutions.

(If \PB{$\\{slack}>\T{0}$}, we forbid $a_0=0$, because those solutions would
have
been obtained with lesser \PB{\\{slack}}.)

The basic ideas will become clear if we look more closely at
the constraints and offsets of our running example, supposing for convenience
that $d=1$. The multiplier is $(b_5b_4b_30b_1b_0)_{10}$, because of
the given offsets. The partial products $(p_0,p_1,p_2,p_3,p_4,p_5)$
apply respectively to $b_0$, $b_1$, $b_3$, $b_4$, $b_5$, and the
grand total. They are supposed to satisfy the constraints (0, 1100000, 100100,
10010, 1001, 11000000), as stated earlier.

Suppose $a_0=3$. Then we must have $b_5=7$; that's the only way to
have $p_4$ end with~1.

And $b_5=7$ implies that $b_0$, $b_1$, $b_3$, $b_4$ can't be 7: All five
constraints are different in this problem, hence no two $b$'s can be equal.

Moving on, if $a_0=3$ we cannot have $a_1=3$. The reason is that
the candidates for multiplier digits are 2 thru~9, and the
values of $33k\bmod100$ for $2\le k\le 9$ are respectively
(66, 99, 32, 65, 98, 31, 64, 97); none of those is suitable for
the constraint 10010.

If $a_0=3$ and $a_1=4$, we must have $b_5=7$ and $b_4=5$.
Furthermore, $a_2=4$ will mess up the constraint 1001, because
$443\times7=3101$. The values $a_2\in\{3,8,9\}$
are also impossible, because they yield no multiplier digits for the
constraint 100100. Thus $a_2$ must be 0, 2, or~6.

Proceeding in this way, we're able to rule out most of the potential trailing
digits of the multiplicand before exploring very far. When we're choosing
suitable values of $a_l$, we check the least significant $l$
digits of each constraint $c_k$ for $0\le k<m$; at least one of the eight
possible nonzero multiplier digits $\ne d$ must satisfy it.
Furthermore, if {\it exactly\/} one multiplier digit is valid, we've
forced one of the multiplier digits $b_i$ to a particular value.

When sufficiently many multiplier digits are forced, we can begin to enforce
the final constraint~$c_m$ (i.e., the constraint on the total product).
This program does that only if the current number of ways to satisfy
the other $m$ constraints individually is less than a certain threshold.
Suppose, for example, that $m=5$ and the current ``status'' is 33121,
meaning that constraints $(c_0,c_1,c_2,c_3,c_4)$ can be individually satisfied
in $(3,3,1,2,1)$ ways. Then we test $c_m$ only if the threshold is 18 or more.

A constraint that is satisfied to infinite precision, not just
with respect to the $l$ trailing digits, is said to be {\it totally\/}
satisfied. Whenever all constraints are totally satisfied, we have a
solution.

After a solution is found, we can sometimes extend it by prepending
nonzero digits to the multiplicand.
For example, we know that $a=2208068$, $b=357029$,
$d=4$ leads to a valid puzzle for the \.O~pattern; so does
$a=302208068$, $b=357029$, $d=4$. The extra prefix `30' doesn't
introduce any unwanted 4's into the partial products or the total product.
This version of the program does {\it not\/} look for such extensions.

\fi

\M{20}Such considerations lead us to a standard backtracking scheme
that takes the following overall form, if we follow the recipe of
Algorithm 7.2.2B:

\Y\B\4\X20:Find all solutions for the current offsets and special digit $d$%
\X${}\E{}$\6
\4\\{b1}:\5
${}\|o,\39\\{maxl}\K\\{maxdigs}-\T{2}-\\{off}[\|m-\T{1}];{}$\6
${}\|l\K\T{0};{}$\6
\X22:Initialize the data structures\X;\6
\4\\{b2}:\5
${}\\{nodes}\MRL{+{\K}}\T{10};{}$\6
\&{if} ${}(\\{vbose}>\T{1}){}$\5
${}\{{}$\1\6
${}\\{fprintf}(\\{stderr},\39\.{"Level\ \%d,"},\39\|l);{}$\6
\X23:Print the \PB{\\{csize}} status information\X;\6
\4${}\}{}$\2\6
\&{if} ${}(\|l\G\\{maxl}){}$\1\5
\X34:Check for unusual solutions and \PB{\&{goto} \\{b5}}\X;\2\6
\X30:If all constraints are totally satisfied, print a solution\X;\6
${}\|x\K\T{0};{}$\6
\4\\{b3}:\5
\&{if} ${}(\\{slack}\W\|l\E\T{0}\W\|x\E\T{0}){}$\1\5
\&{goto} \\{b4};\2\6
\&{if} ${}(\|x\E\|d){}$\1\5
\&{goto} \\{b4};\2\6
\&{if} ${}(\\{vbose}>\T{2}){}$\1\5
${}\\{fprintf}(\\{stderr},\39\.{"\ testing\ \%d\\n"},\39\|x);{}$\2\6
\X24:If some constraint can't be satisfied when $a_l=x$, \PB{\&{goto} \\{b4}}%
\X;\6
${}\|o,\39\|a[\|l]\K\|x;{}$\6
\&{if} ${}(\\{vbose}>\T{1}){}$\1\5
${}\\{fprintf}(\\{stderr},\39\.{"Trying\ a[\%d]=\%d\\n"},\39\|l,\39\|x);{}$\2\6
\X28:Update the data structures\X;\6
${}\|l\K\|l+\T{1}{}$;\5
\&{goto} \\{b2};\6
\4\\{b4}:\5
\&{if} ${}(\|x\E\T{9}){}$\1\5
\&{goto} \\{b5};\2\6
${}\|x\K\|x+\T{1}{}$;\5
\&{goto} \\{b3};\6
\4\\{b5}:\5
${}\|l\K\|l-\T{1};{}$\6
\&{if} ${}(\|l\G\T{0}){}$\5
${}\{{}$\1\6
\&{if} ${}(\\{vbose}>\T{1}){}$\1\5
${}\\{fprintf}(\\{stderr},\39\.{"Back\ to\ level\ \%d\\n"},\39\|l);{}$\2\6
${}\|o,\39\|x\K\|a[\|l];{}$\6
\X29:Downdate the data structures\X;\6
\&{goto} \\{b4};\6
\4${}\}{}$\2\par
\U2.\fi

\M{21}What data structures will support this computation nicely?
First, there's an array
of bignums: \PB{\\{ja}[\|l][\|j]} contains $j$ times the partial multiplier
$(a_l\ldots a_0)_{10}$ at a given level. Clearly
\PB{\\{ja}[\|l][\|j]} is \PB{$\\{ja}[\|l-\T{1}][\|j]$} plus $j\cdot10^{%
\mkern1mul}a_l$. These entries
are computed only for values of~\PB{\|j} that are necessary;
\PB{\\{stamp}[\|l][\|j]} contains the node number at which they were
most recently computed (actually it contains \PB{$\\{nodes}+\|x$}).

We also maintain arrays called \PB{\\{choice}[\|k]}, which list the all nonzero
multiplier digits that haven't been ruled out for constraint~\PB{\|k}.
Their sizes at level~\PB{\|l} are \PB{\\{csize}[\|l][\|k]}.
Actually \PB{\\{choice}[\|k]} is a permutation of $\{0,1,\ldots,9\}$, and
\PB{\\{where}[\|k]} is the inverse permutation; the viable elements at
level~\PB{\|l} are those $j$ with \PB{$\\{where}[\|k][\|j]<\\{csize}[\|l][%
\|k]$}.
This setup permits easy deletion from the lists while backtracking.

\Y\B\4\X11:Global variables\X${}\mathrel+\E{}$\6
\&{bignum} \\{ja}[\\{maxdigs}][\T{10}];\C{ multiples of the multiplicand }\6
\&{unsigned} \&{long} \&{long} \\{stamp}[\\{maxdigs}][\T{10}];\C{ when they
were computed }\6
\&{char} \\{choice}[\\{maxm}][\T{10}]${},{}$ \\{where}[\\{maxm}][\T{10}];\C{
available multipliers, ranked }\6
\&{char} \\{csize}[\\{maxdigs}][\\{maxm}];\C{ current degree of viability }\6
\&{char} \\{stack}[\\{maxm}];\C{ constraints that have become uniquely
satisfied }\6
\&{char} \\{stackptr};\C{ current size of \PB{\\{stack}} }\6
\&{char} \|a[\\{maxdigs}];\C{ the multiplicand }\6
\&{bignum} \\{total};\C{ grand total when checking for a solution }\par
\fi

\M{22}\B\X22:Initialize the data structures\X${}\E{}$\6
\&{if} ${}(\|d\E\T{0}\W\\{off}[\|m-\T{1}]\G\|m){}$\1\5
\&{goto} \\{b5};\C{ forbid zeros in multiplier if \PB{$\|d\K\T{0}$} }\2\6
\&{for} ${}(\|i\K\T{0},\39\|j\K\T{1};{}$ ${}\|j<\T{10};{}$ ${}\|j\PP){}$\1\6
\&{if} ${}(\|j\I\|d){}$\5
${}\{{}$\1\6
\&{for} ${}(\|k\K\T{0};{}$ ${}\|k<\|m;{}$ ${}\|k\PP){}$\1\5
${}\\{oo},\39\\{choice}[\|k][\|i]\K\|j,\39\\{where}[\|k][\|j]\K\|i;{}$\2\6
${}\|i\PP;{}$\6
\4${}\}{}$\2\2\6
\&{for} ${}(\|k\K\T{0};{}$ ${}\|k<\|m;{}$ ${}\|k\PP){}$\1\5
${}\\{oo},\39\\{oo},\39\\{csize}[\T{0}][\|k]\K\|i,\39\\{choice}[\|k][\|i]\K\|d,%
\39\\{where}[\|k][\|d]\K\|i,\39\\{where}[\|k][\T{0}]\K\T{9}{}$;\C{ note that %
\PB{$\|i\K\T{9}$} if \PB{$\|d\K\T{0}$}, otherwise 8 }\2\par
\U20.\fi

\M{23}\B\X23:Print the \PB{\\{csize}} status information\X${}\E{}$\6
\&{for} ${}(\|k\K\T{0};{}$ ${}\|k<\|m;{}$ ${}\|k\PP){}$\1\5
${}\\{fprintf}(\\{stderr},\39\.{"\%d"},\39\\{csize}[\|l][\|k]);{}$\2\6
${}\\{fprintf}(\\{stderr},\39\.{"\\n"}){}$;\par
\U20.\fi

\M{24}\B\D$\\{thresh}$ \5
\T{25}\par
\Y\B\4\X24:If some constraint can't be satisfied when $a_l=x$, \PB{\&{goto} %
\\{b4}}\X${}\E{}$\6
\&{for} ${}(\\{stackptr}\K\T{0},\39\|k\K\|m-\T{1};{}$ ${}\|k\G\T{0};{}$ ${}\|k%
\MM){}$\1\5
\X25:If constraint \PB{\|k} can't be satisfied when $a_l=x$, \PB{\&{goto} %
\\{b4}}\X;\2\6
\&{while} (\\{stackptr})\5
${}\{{}$\1\6
${}\|o,\39\|k\K\\{stack}[\MM\\{stackptr}];{}$\6
\&{if} ${}(\\{vbose}>\T{2}){}$\1\5
${}\\{fprintf}(\\{stderr},\39\.{"\ b\%d\ must\ be\ \%d\\n"},\39\\{off}[\|k],\39%
\\{choice}[\|k][\T{0}]);{}$\2\6
\X27:Delete \PB{\\{choice}[\|k][\T{0}]} from all constraints $\ne c_k$\X;\6
\4${}\}{}$\2\6
\&{for} ${}(\|o,\39\|t\K\\{csize}[\|l+\T{1}][\T{0}],\39\|k\K\T{1};{}$ ${}\|k<%
\|m\W\|t\Z\\{thresh};{}$ ${}\|k\PP){}$\1\5
${}\|o,\39\|t\MRL{*{\K}}\\{csize}[\|l+\T{1}][\|k];{}$\2\6
\&{if} ${}(\|t\Z\\{thresh}){}$\5
${}\{{}$\1\6
\X35:Test the overall product constraint $c_m$\X;\6
\&{while} (\\{stackptr})\5
${}\{{}$\1\6
${}\|o,\39\|k\K\\{stack}[\MM\\{stackptr}];{}$\6
\&{if} ${}(\\{vbose}>\T{2}){}$\1\5
${}\\{fprintf}(\\{stderr},\39\.{"\ b\%d\ has\ to\ be\ \%d\\n}\)\.{"},\39%
\\{off}[\|k],\39\\{choice}[\|k][\T{0}]);{}$\2\6
\X27:Delete \PB{\\{choice}[\|k][\T{0}]} from all constraints $\ne c_k$\X;\6
\4${}\}{}$\2\6
\4${}\}{}$\2\par
\U20.\fi

\M{25}Now we've come to the heart and soul of the program. As we test
each constraint, we also store some data that will be needed on
level~\PB{$\|l+\T{1}$} if we get there.

\Y\B\4\X25:If constraint \PB{\|k} can't be satisfied when $a_l=x$, \PB{\&{goto}
\\{b4}}\X${}\E{}$\6
${}\{{}$\1\6
${}\|o,\39\\{imax}\K\\{csize}[\|l][\|k]{}$;\C{ how many multipliers worked in
the previous level? }\6
\&{for} ${}(\|i\K\T{0};{}$ ${}\|i<\\{imax};{}$ ${}\|i\PP){}$\5
${}\{{}$\1\6
${}\|o,\39\|j\K\\{choice}[\|k][\|i];{}$\6
\X26:If \PB{\|j} remains satisfactory when $a_l=x$, \PB{\&{goto} \\{jok}}\X;\6
\&{if} ${}(\\{vbose}>\T{2}){}$\1\5
${}\\{fprintf}(\\{stderr},\39\.{"\ c\%d\ loses\ option\ \%}\)\.{d\\n"},\39\|k,%
\39\|j);{}$\2\6
\&{if} ${}(\MM\\{imax}\E\T{0}){}$\1\5
\&{goto} \\{b4};\C{ we've lost the last option }\2\6
\&{if} ${}(\|i\I\\{imax}){}$\1\5
${}\\{oo},\39\\{oo},\39\\{oo},\39\\{choice}[\|k][\|i]\K\\{choice}[\|k][%
\\{imax}],\39\\{where}[\|k][\\{choice}[\|k][\\{imax}]]\K\|i\MM,\39\\{choice}[%
\|k][\\{imax}]\K\|j,\39\\{where}[\|k][\|j]\K\\{imax}{}$;\C{ swap \PB{\|j} into
last position (for easy backtracking) }\2\6
\4\\{jok}:\5
\&{continue};\6
\4${}\}{}$\2\6
${}\|o,\39\\{csize}[\|l+\T{1}][\|k]\K\\{imax};{}$\6
\&{if} ${}(\\{imax}\E\T{1}\W(\|o,\39\\{csize}[\|l][\|k]\I\T{1})){}$\1\5
${}\|o,\39\\{stack}[\\{stackptr}\PP]\K\|k;{}$\2\6
\4${}\}{}$\2\par
\U24.\fi

\M{26}We've previously verified constraint \PB{\|k} in the least significant
\PB{\|l} digits, and those digits don't depend on $a_l$. Thus it suffices
to do an ``incremental'' test, looking only at digit~\PB{\|l} of the
constraint.

\Y\B\4\X26:If \PB{\|j} remains satisfactory when $a_l=x$, \PB{\&{goto} \\{jok}}%
\X${}\E{}$\6
\&{if} ${}(\|o,\39\\{stamp}[\|l][\|j]\I\\{nodes}+\|x){}$\5
${}\{{}$\C{ have we already updated \PB{\\{ja}[\|l]}? }\1\6
${}\|o,\39\\{stamp}[\|l][\|j]\K\\{nodes}+\|x;{}$\6
\&{if} ${}(\|l\E\T{0}){}$\1\5
${}\\{oo},\39\\{copy}(\\{ja}[\T{0}][\|j],\39\\{cnst}[\|x*\|j]);{}$\2\6
\&{else}\1\5
${}\\{oo},\39\\{add}(\\{ja}[\|l][\|j],\39\\{ja}[\|l-\T{1}][\|j],\39\\{cnst}[%
\|x*\|j],\39\|l);{}$\2\6
\4${}\}{}$\2\6
${}\\{oo},\39\|t\K(\\{ja}[\|l][\|j][\T{0}]\Z\|l\?\T{0}:\\{ja}[\|l][\|j][\|l+%
\T{1}]);{}$\6
${}\|o,\39\\{tt}\K(\\{constr}[\|k][\T{0}]\Z\|l\?\T{0}:\|o,\39\\{constr}[\|k][%
\|l+\T{1}]);{}$\6
\&{if} ${}((\\{tt}\E\T{1}\W\|t\E\|d)\V(\\{tt}\I\T{1}\W\|t\I\|d)){}$\1\5
\&{goto} \\{jok};\2\par
\U25.\fi

\M{27}\B\X27:Delete \PB{\\{choice}[\|k][\T{0}]} from all constraints $\ne c_k$%
\X${}\E{}$\6
\&{for} ${}(\|o,\39\\{kk}\K\T{0},\39\|j\K\\{choice}[\|k][\T{0}];{}$ ${}\\{kk}<%
\|m;{}$ ${}\\{kk}\PP){}$\1\6
\&{if} ${}(\\{oo},\39\\{id}[\\{kk}]\I\\{id}[\|k]){}$\5
${}\{{}$\1\6
${}\\{oo},\39\|i\K\\{csize}[\|l+\T{1}][\\{kk}]-\T{1},\39\\{ii}\K\\{where}[%
\\{kk}][\|j];{}$\6
\&{if} ${}(\\{ii}\Z\|i){}$\5
${}\{{}$\1\6
\&{if} ${}(\|i\E\T{0}){}$\1\5
\&{goto} \\{b4};\2\6
${}\|o,\39\\{csize}[\|l+\T{1}][\\{kk}]\K\|i;{}$\6
\&{if} ${}(\|i\E\T{1}){}$\1\5
${}\|o,\39\\{stack}[\\{stackptr}\PP]\K\\{kk};{}$\2\6
\&{if} ${}(\\{ii}\I\|i){}$\1\5
${}\\{oo},\39\\{oo},\39\\{oo},\39\\{choice}[\\{kk}][\\{ii}]\K\\{choice}[%
\\{kk}][\|i],\39\\{where}[\\{kk}][\\{choice}[\\{kk}][\|i]]\K\\{ii},\39%
\\{choice}[\\{kk}][\|i]\K\|j,\39\\{where}[\\{kk}][\|j]\K\|i;{}$\2\6
\4${}\}{}$\2\6
\4${}\}{}$\2\2\par
\U24.\fi

\M{28}The data structures that I've got don't seem to need any updating
(other than what has already been done during the tests),
except in one respect: When a zero digit is prepended to the multiplicand,
we may have already printed the current solution. Otherwise we haven't.

\Y\B\4\X28:Update the data structures\X${}\E{}$\6
\&{if} (\|x)\1\5
${}\\{printed}\K\T{0}{}$;\2\par
\U20.\fi

\M{29}Downdating seems to be completely unnecessary, thanks largely to
the \PB{\\{choice}} and \PB{\\{csize}} mechanism, and the fact that other data
is recomputed at each level.

\Y\B\4\X29:Downdate the data structures\X${}\E{}$\par
\U20.\fi

\M{30}\B\X30:If all constraints are totally satisfied, print a solution\X${}%
\E{}$\6
\&{if} (\\{printed})\1\5
\&{goto} \\{nope};\C{ we've already printed this guy }\2\6
\&{for} ${}(\|k\K\T{0};{}$ ${}\|k<\|m;{}$ ${}\|k\PP){}$\1\6
\&{if} ${}(\|o,\39\\{csize}[\|l][\|k]>\T{1}){}$\1\5
\&{goto} \\{nope};\2\2\6
\&{for} ${}(\|k\K\|m-\T{1};{}$ ${}\|k\G\T{0};{}$ ${}\|k\MM){}$\1\5
\X31:If constraint $c_k$ isn't totally satisfied, \PB{\&{goto} \\{nope}}\X;\2\6
\X32:If constraint $c_m$ isn't totally satisfied, \PB{\&{goto} \\{nope}}\X;\6
\X33:Print a solution\X;\6
\&{while} ${}(\|o,\39\|a[\|l-\T{1}]\E\T{0}){}$\1\5
${}\|l\MM;{}$\2\6
\&{goto} \\{b5}; \\{nope}:\par
\U20.\fi

\M{31}\B\X31:If constraint $c_k$ isn't totally satisfied, \PB{\&{goto} %
\\{nope}}\X${}\E{}$\6
${}\{{}$\1\6
${}\\{oo},\39\|o,\39\|j\K\\{choice}[\|k][\T{0}],\39\\{lj}\K\\{ja}[\|l-\T{1}][%
\|j][\T{0}],\39\\{lc}\K\\{constr}[\|k][\T{0}];{}$\6
\&{if} ${}(\\{lc}>\\{lj}){}$\1\5
\&{goto} \\{nope};\C{ this is correct even if \PB{$\|d\K\T{0}$} }\2\6
\&{for} ${}(\|i\K\T{1};{}$ ${}\|i\Z\\{lj};{}$ ${}\|i\PP){}$\5
${}\{{}$\1\6
${}\|o,\39\|t\K\\{ja}[\|l-\T{1}][\|j][\|i],\39\\{tt}\K(\|i\Z\\{lc}\?\|o,\39%
\\{constr}[\|k][\|i]:\T{0});{}$\6
\&{if} ${}((\|t\E\|d\W\\{tt}\E\T{0})\V(\|t\I\|d\W\\{tt}\I\T{0})){}$\1\5
\&{goto} \\{nope};\2\6
\4${}\}{}$\2\6
\4${}\}{}$\2\par
\U30.\fi

\M{32}\B\X32:If constraint $c_m$ isn't totally satisfied, \PB{\&{goto} %
\\{nope}}\X${}\E{}$\6
$\\{oo},\39\\{oo},\39\\{add}(\\{total},\39\\{ja}[\|l-\T{1}][\\{choice}[\T{0}][%
\T{0}]],\39\\{ja}[\|l-\T{1}][\\{choice}[\T{1}][\T{0}]],\39\\{off}[\T{1}]);{}$\6
\&{for} ${}(\|k\K\T{2};{}$ ${}\|k<\|m;{}$ ${}\|k\PP){}$\1\5
${}\\{oo},\39\|o,\39\\{add}(\\{total},\39\\{total},\39\\{ja}[\|l-\T{1}][%
\\{choice}[\|k][\T{0}]],\39\\{off}[\|k]);{}$\2\6
${}\|o,\39\\{lj}\K\\{total}[\T{0}],\39\\{lc}\K\\{constr}[\|m][\T{0}];{}$\6
\&{if} ${}(\\{lc}>\\{lj}){}$\1\5
\&{goto} \\{nope};\C{ this is correct even if \PB{$\|d\K\T{0}$} }\2\6
\&{for} ${}(\|i\K\T{1};{}$ ${}\|i\Z\\{lj};{}$ ${}\|i\PP){}$\5
${}\{{}$\1\6
${}\|o,\39\|t\K\\{total}[\|i],\39\\{tt}\K(\|i\Z\\{lc}\?\|o,\39\\{constr}[\|m][%
\|i]:\T{0});{}$\6
\&{if} ${}((\|t\E\|d\W\\{tt}\E\T{0})\V(\|t\I\|d\W\\{tt}\I\T{0})){}$\1\5
\&{goto} \\{nope};\2\6
\4${}\}{}$\2\par
\U30.\fi

\M{33}When a solution is found, I first print out the lengths of
the multiplicand, multiplier, partial products, and total product.
(By sorting these lines later, I can distinguish unique solutions.)
Then I print the multiplicand, multiplier, \PB{\|d}, and the solution number.

\Y\B\4\X33:Print a solution\X${}\E{}$\6
$\\{count}\PP;{}$\6
\&{for} ${}(\|i\K\|l-\T{1};{}$ ${}\|a[\|i]\E\T{0};{}$ ${}\|i\MM){}$\1\5
;\C{ bypass leading zeros of multiplicand }\2\6
${}\\{printf}(\.{"\%d,\%d;"},\39\|i+\T{1},\39\\{off}[\|m-\T{1}]+\T{1});{}$\6
\&{for} ${}(\|k\K\T{0};{}$ ${}\|k<\|m;{}$ ${}\|k\PP){}$\1\5
${}\\{printf}(\.{"\%d|\%d,"},\39\\{ja}[\|l-\T{1}][\\{choice}[\|k][\T{0}]][%
\T{0}],\39\\{off}[\|k]);{}$\2\6
${}\\{printf}(\.{"\%d,\ "},\39\\{total}[\T{0}]);{}$\6
\&{for} ( ; ${}\|i\G\T{0};{}$ ${}\|i\MM){}$\1\5
${}\\{printf}(\.{"\%d"},\39\|a[\|i]);{}$\2\6
\\{printf}(\.{"\ x\ "});\6
\&{for} ${}(\|k\K\|m-\T{1},\39\|i\K\\{off}[\|k];{}$ ${}\|k\G\T{0};{}$ ${}\|k%
\MM,\39\|i\MM){}$\5
${}\{{}$\1\6
\&{while} ${}(\|i>\\{off}[\|k]){}$\1\5
${}\\{printf}(\.{"0"}),\39\|i\MM;{}$\2\6
${}\\{printf}(\.{"\%d"},\39\\{choice}[\|k][\T{0}]);{}$\6
\4${}\}{}$\2\6
${}\\{printf}(\.{",d=\%d\ (\#\%d)\\n"},\39\|d,\39\\{count});{}$\6
${}\\{printed}\K\T{1}{}$;\par
\U30.\fi

\M{34}It's conceivable that we've constructed a max-length multiplicand
without finding enough obstructions to force all digits of the
multiplier. In such cases constraint~\PB{\|m} (the constraint on the
entire product) has probably not yet been fully tested. We should
therefore backtrack over all choices of multipliers, in order to
be sure that no solutions have been overlooked.

Pathological patterns can make this happen, but I don't think it
will occur in the cases that interest me. So I am simply
reporting the unusual case here. Then I can follow up later if additional
investigations are called for.

(If $a_{l-1}!=0$, there might exist very long solutions that cannot
be tested without exceeding our \PB{\\{maxdigits}} precision.)

\Y\B\4\D$\\{show\_unresolved}$ \5
\T{0}\par
\Y\B\4\X34:Check for unusual solutions and \PB{\&{goto} \\{b5}}\X${}\E{}$\6
${}\{{}$\1\6
\&{for} ${}(\|k\K\T{0};{}$ ${}\|k<\|m;{}$ ${}\|k\PP){}$\1\6
\&{if} ${}(\|o,\39\\{csize}[\|l][\|k]>\T{1}){}$\1\5
\&{break};\2\2\6
\&{if} ${}(\|k<\|m){}$\5
${}\{{}$\1\6
${}\\{unresolved}\PP;{}$\6
\&{if} ${}(\|o,\39\|a[\|l-\T{1}]\E\T{0}\V\\{show\_unresolved}){}$\5
${}\{{}$\1\6
${}\\{fprintf}(\\{stderr},\39\.{"Unresolved\ case\ wit}\)\.{h\ d=\%d\ and\
offsets"},\39\|d);{}$\6
\&{for} ${}(\|k\K\T{0};{}$ ${}\|k<\|m;{}$ ${}\|k\PP){}$\1\5
${}\\{fprintf}(\\{stderr},\39\.{"\ \%d"},\39\\{off}[\|k]);{}$\2\6
${}\\{fprintf}(\\{stderr},\39\.{":\\n\ a=..."});{}$\6
\&{for} ${}(\|k\K\|l-\T{1};{}$ ${}\|k\G\T{0};{}$ ${}\|k\MM){}$\1\5
${}\\{fprintf}(\\{stderr},\39\.{"\%d"},\39\|a[\|k]);{}$\2\6
${}\\{fprintf}(\\{stderr},\39\.{",\ status\ "});{}$\6
\&{for} ${}(\|k\K\T{0};{}$ ${}\|k<\|m;{}$ ${}\|k\PP){}$\1\5
${}\\{fprintf}(\\{stderr},\39\.{"\%d"},\39\\{csize}[\|l][\|k]);{}$\2\6
${}\\{fprintf}(\\{stderr},\39\.{"!\\n"});{}$\6
\4${}\}{}$\2\6
\4${}\}{}$\2\6
\&{goto} \\{b5};\6
\4${}\}{}$\2\par
\U20.\fi

\N{1}{35}An inner loop. When we're testing the ``bottom line''
constraint~$c_m$,
we might need to vary several of the multiplier digits independently.
The process is a bit tedious, but straightforward: It's just
a loop over all $m$-tuples that haven't yet been filtered out,
and we know that the total number of such $m$-tuples is \PB{\\{thresh}} or
less.

The multiplier digit that is subject to constraint $c_k$ is one of the
\PB{$\\{csize}[\|l+\T{1}][\|k]$} possibilities that appear at the beginning of
the
list \PB{\\{choice}[\|k]}.
So we represent it by an index \PB{\|g[\|k]}, meaning that the digit we're
trying is \PB{\\{choice}[\|k][\|g[\|k]]}.

For every such $m$-tuple $g_0g_1\ldots g_{m-1}$, we check if
constraint~$c_m$ holds in its rightmost $l+1$ digits. If so,
we set bit $g_k$ to~1 in \PB{\\{shadow}[\|k]}, for $0\le k<m$, thereby
indicating that $g_k$ is valid in at least one solution.

After running through all the $m$-tuples, we can backtrack if
no solutions were found. Otherwise the shadows will tell us
whether any of the \PB{\\{csize}} entries can be lowered.

I could do this step in a fancier way, by working only
``incrementally'' after having gotten $l$-digit compliance
instead of always working to higher and higher precision.
(In such a case I'd have to save the sum of carries from
the lower \PB{\|l} digits, for use in testing the (\PB{$\|l+\T{1}$})st digit
incrementally.)

I could also avoid many of the $m$-tuples by backtracking
during this process, because $c_m$ can be tested digit-by-digit
as those digits become known.

But I don't think this step will be a bottleneck, so I've
opted for simplicity.

\Y\B\4\X35:Test the overall product constraint $c_m$\X${}\E{}$\6
${}\{{}$\1\6
\&{for} ${}(\|k\K\T{0};{}$ ${}\|k<\|m;{}$ ${}\|k\PP){}$\1\5
${}\|o,\39\\{shadow}[\|k]\K\T{0};{}$\2\6
\X36:Run through all $m$-tuples $g_0\ldots g_{m-1}$\X;\6
\&{if} ${}(\|o,\39\\{shadow}[\T{0}]\E\T{0}){}$\1\5
\&{goto} \\{b4};\C{ there were no solutions }\2\6
\&{for} ${}(\|k\K\T{0};{}$ ${}\|k<\|m;{}$ ${}\|k\PP){}$\5
${}\{{}$\1\6
\&{if} ${}(\\{oo},\39\\{shadow}[\|k]+\T{1}\I\T{1}\LL\\{csize}[\|l+\T{1}][%
\|k]){}$\1\5
\X39:Remove items from \PB{\\{choice}[\|k]}\X;\2\6
\4${}\}{}$\2\6
\4${}\}{}$\2\par
\U24.\fi

\M{36}\B\X36:Run through all $m$-tuples $g_0\ldots g_{m-1}$\X${}\E{}$\6
\4\\{bb1}:\5
${}\|k\K\T{0};{}$\6
\4\\{bb2}:\5
\&{if} ${}(\|k\E\|m){}$\1\5
\X38:Test compliance with $c_m$ and \PB{\&{goto} \\{bb5}}\X;\2\6
${}\|g[\|k]\K\T{0};{}$\6
\4\\{bb3}:\5
\X37:Set \PB{\\{acc}[\|k]} to the least significant digits of the $k$th partial
sum\X;\6
${}\|k\PP;{}$\6
\&{goto} \\{bb2};\6
\4\\{bb4}:\5
${}\\{oo},\39\|g[\|k]\PP;{}$\6
\&{if} ${}(\|o,\39\|g[\|k]<\\{csize}[\|l+\T{1}][\|k]){}$\1\5
\&{goto} \\{bb3};\2\6
\4\\{bb5}:\5
${}\|k\MM;{}$\6
\&{if} ${}(\|k\G\T{0}){}$\1\5
\&{goto} \\{bb4};\2\par
\U35.\fi

\M{37}\B\X37:Set \PB{\\{acc}[\|k]} to the least significant digits of the $k$th
partial sum\X${}\E{}$\6
$\\{oo},\39\|o,\39\|j\K\\{choice}[\|k][\|g[\|k]],\39\\{lj}\K\\{ja}[\|l][\|j][%
\T{0}];{}$\6
\&{for} ${}(\|i\K\T{0};{}$ \|o${},\39\|i<\\{off}[\|k];{}$ ${}\|i\PP){}$\1\5
${}\\{oo},\39\\{acc}[\|k][\|i]\K\\{acc}[\|k-\T{1}][\|i];{}$\2\6
\&{for} ${}(\\{ii}\K\T{1},\39\\{kk}\K\T{0};{}$ ${}\|i\Z\|l;{}$ ${}\|i\PP,\39%
\\{ii}\PP){}$\5
${}\{{}$\1\6
${}\|t\K(\|k>\T{0}\?\|o,\39\\{acc}[\|k-\T{1}][\|i]+\\{kk}:\\{kk});{}$\6
\&{if} ${}(\\{ii}\Z\\{lj}){}$\1\5
${}\|o,\39\|t\MRL{+{\K}}\\{ja}[\|l][\|j][\\{ii}];{}$\2\6
\&{if} ${}(\|t\G\T{10}){}$\1\5
${}\|o,\39\\{acc}[\|k][\|i]\K\|t-\T{10},\39\\{kk}\K\T{1}{}$;\5
\2\&{else}\1\5
${}\|o,\39\\{acc}[\|k][\|i]\K\|t,\39\\{kk}\K\T{0};{}$\2\6
\4${}\}{}$\2\par
\U36.\fi

\M{38}\B\X38:Test compliance with $c_m$ and \PB{\&{goto} \\{bb5}}\X${}\E{}$\6
${}\{{}$\1\6
\&{for} ${}(\|o,\39\|i\K\T{0},\39\\{lc}\K\\{constr}[\|m][\T{0}];{}$ ${}\|i\Z%
\|l;{}$ ${}\|i\PP){}$\5
${}\{{}$\1\6
${}\|o,\39\|t\K\\{acc}[\|m-\T{1}][\|i];{}$\6
\&{if} ${}(\|i<\\{lc}){}$\1\5
${}\|o,\39\\{tt}\K\\{constr}[\|m][\|i+\T{1}]{}$;\5
\2\&{else}\1\5
${}\\{tt}\K\T{0};{}$\2\6
\&{if} ${}((\|t\E\|d\W\\{tt}\E\T{0})\V(\|t\I\|d\W\\{tt}\I\T{0})){}$\1\5
\&{goto} \\{noncomp};\2\6
\4${}\}{}$\2\6
\&{if} ${}(\\{vbose}>\T{2}){}$\5
${}\{{}$\1\6
${}\\{fprintf}(\\{stderr},\39\.{"\ ok\ "});{}$\6
\&{for} ${}(\|k\K\|m-\T{1};{}$ ${}\|k\G\T{0};{}$ ${}\|k\MM){}$\1\5
${}\\{fprintf}(\\{stderr},\39\.{"\%d"},\39\\{choice}[\|k][\|g[\|k]]);{}$\2\6
${}\\{fprintf}(\\{stderr},\39\.{"\\n"});{}$\6
\4${}\}{}$\2\6
\&{for} ${}(\|k\K\T{0};{}$ ${}\|k<\|m;{}$ ${}\|k\PP){}$\1\5
${}\\{oo},\39\\{shadow}[\|k]\MRL{{\OR}{\K}}\T{1}\LL\|g[\|k];{}$\2\6
\4\\{noncomp}:\5
\&{goto} \\{bb5};\6
\4${}\}{}$\2\par
\U36.\fi

\M{39}\B\X39:Remove items from \PB{\\{choice}[\|k]}\X${}\E{}$\6
${}\{{}$\1\6
${}\|o,\39\\{imax}\K\\{csize}[\|l+\T{1}][\|k];{}$\6
\&{for} ${}(\|i\K\\{imax}-\T{1};{}$ ${}\|i\G\T{0};{}$ ${}\|i\MM){}$\1\6
\&{if} ${}(\|o,\39(\\{shadow}[\|k]\AND(\T{1}\LL\|i))\E\T{0}){}$\5
${}\{{}$\1\6
${}\|o,\39\|j\K\\{choice}[\|k][\|i];{}$\6
\&{if} ${}(\\{vbose}>\T{2}){}$\1\5
${}\\{fprintf}(\\{stderr},\39\.{"\ b\%d\ ain't\ \%d\\n"},\39\|k,\39\|j);{}$\2\6
${}\\{imax}\MM;{}$\6
\&{if} ${}(\|i\I\\{imax}){}$\1\5
${}\\{oo},\39\\{oo},\39\\{oo},\39\\{choice}[\|k][\|i]\K\\{choice}[\|k][%
\\{imax}],\39\\{where}[\|k][\\{choice}[\|k][\\{imax}]]\K\|i,\39\\{choice}[\|k][%
\\{imax}]\K\|j,\39\\{where}[\|k][\|j]\K\\{imax};{}$\2\6
\4${}\}{}$\2\2\6
${}\|o,\39\\{csize}[\|l+\T{1}][\|k]\K\\{imax};{}$\6
\&{if} ${}(\\{imax}\E\T{1}){}$\1\5
${}\|o,\39\\{stack}[\\{stackptr}\PP]\K\|k;{}$\2\6
\4${}\}{}$\2\par
\U35.\fi

\M{40}\B\X11:Global variables\X${}\mathrel+\E{}$\6
\&{char} \\{acc}[\\{maxm}][\\{maxdigs}];\C{ partial sums }\6
\&{char} \|g[\\{maxm}];\C{ indices for inner loop }\6
\&{int} \\{shadow}[\\{maxm}];\C{ bits where solutions were found }\par
\fi

\N{1}{41}Index.
\fi

\inx
\fin
\con
