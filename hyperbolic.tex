\input cwebmac
\datethis
\input epsf

\N{1}{1}Introduction. This program computes numerical coordinates so that I can
experiment with some of the fascinating patterns that arise when the
hyperbolic plane is tiled with $36^\circ$-$45^\circ$-$90^\circ$ triangles.
Such a tiling is unique, but it can be viewed in many different ways.

Maurice Margenstern discovered a beautiful way to assign numbers to the
vertices, based on Fibonacci representations; his method is discussed in
\S6 of the paper ``A universal cellular automaton in the hyperbolic
plane,'' by Francine Herrmann and Maurice Margenstern, {\sl Theoretical
Computer Science\/ \bf296} (2003), 327--364. I~want to play with those
ideas further, and for this purpose I need to make special kinds
of graph paper.

I'm writing this program for fun and experience. So I'm using basic brute
force, together with data structures that ought to help me gain both a local
and global understanding of the tiling.

\Y\B\4\D$\\{maxn}$ \5
\T{300}\C{ this many triangles will be computed }\C{ in this implementation I
assume that \PB{$\T{3}*\\{maxn}<\T{1024}$} }\par
\B\4\D$\\{hprime}$ \5
\T{1009}\C{ a prime number, should be at least \PB{$\T{2}*\\{maxn}$} }\par
\Y\B\8\#\&{include} \.{<stdio.h>}\6
\8\#\&{include} \.{<math.h>}\6
\X2:Type definitions\X\6
\X6:Global variables\X\6
\X3:Subroutines\X\7
\\{main}(\,)\1\1\2\2\6
${}\{{}$\1\6
\&{register} \&{int} \|a${},{}$ \|b${},{}$ \|c${},{}$ \|j${},{}$ \|k${},{}$ %
\|t;\6
\&{register} \&{double} \\{phi};\7
\X11:Set up triangle 0\X;\6
\&{for} ${}(\|k\K\T{0};{}$ ${}\\{tptr}<\\{maxn};{}$ ${}\|k\PP){}$\1\5
\X13:Compute the neighbors of triangle \PB{\|k}\X;\2\6
\4${}\}{}$\2\par
\fi

\N{1}{2}Hyperbolic data structures. I think the most convenient way to
deal with the hyperbolic plane is to consider its points to be those
of the Euclidean upper half plane, namely the points $(x,y)$ with $y>0$.
Its ``lines'' are half-circles centered on the $x$-axis, namely the
sets of points $(c+r\cos\theta,r\sin\theta)$ for some center~$c$
and some radius~$r>0$, as $\theta$ runs from 0 to~$\pi$. Given a
point $(x,y)$ and an angle~$\theta$ between 0 and~$\pi$, we can therefore
can construct a corresponding hyperbolic line with center and radius
$$c=x-y\cot\theta,\qquad r=y\csc\theta.$$

\Y\B\4\X2:Type definitions\X${}\E{}$\6
\&{typedef} \&{struct} ${}\{{}$\5
\1\&{double} \|x${},{}$ \|y;\5
\2${}\}{}$\5
\&{point};\6
\&{typedef} \&{struct} ${}\{{}$\5
\1\&{double} \|c${},{}$ \|r;\5
\2${}\}{}$\5
\&{circle};\par
\A7.
\U1.\fi

\M{3}The unique hyperbolic line that passes through two given points
$(x,y)$ and $(x',y')$ is centered at
$$c={x^2+y^2-{x'}^2-{y'}^2\over 2(x-x')}
={x+x'\over2}+{y^2-{y'}^2\over2(x-x')}.$$
(If $x=x'$, we have $c=\infty$ and the ``circle'' is actually a
straight vertical line. But I don't have to worry about that case,
because it won't arise in this program.)

\Y\B\4\X3:Subroutines\X${}\E{}$\6
\&{circle} \\{common}(\&{point} \|z${},\39{}$\&{point} \\{zp})\1\1\2\2\6
${}\{{}$\1\6
\&{circle} \|t;\7
${}\|t.\|c\K(\|z.\|x+\\{zp}.\|x)/\T{2.0}+((\|z.\|y+\\{zp}.\|y)/\T{2.0})*((\|z.%
\|y-\\{zp}.\|y)/(\|z.\|x-\\{zp}.\|x));{}$\6
\&{if} ${}(\\{fabs}(\|t.\|c)<\T{0.00001}){}$\1\5
${}\|t.\|c\K\T{0.0};{}$\2\6
${}\|t.\|r\K\\{sqrt}((\|z.\|x-\|t.\|c)*(\|z.\|x-\|t.\|c)+\|z.\|y*\|z.\|y);{}$\6
\&{return} \|t;\6
\4${}\}{}$\2\par
\As4, 5\ETs8.
\U1.\fi

\M{4}The main technical operation in this program is to
``reflect'' a point with respect to a given hyperbolic line.
If the line has center~$c$ and radius~$r$, the reflection
of $(c+s\cos\theta,s\sin\theta)$ is defined to be
$(c+t\cos\theta,t\sin\theta)$, where $st=r^2$. One can show
that this mapping is an automorphism of the hyperbolic plane.

We're interested in reflection because every triangle in the
tiling being computed has three neighbors, each of which is
obtained by reflecting one of the vertices about the
opposite edge. Consider, for example, the triangle $ABC$
shown here:
$$\vcenter{\epsfbox{hyperbolic.1}}$$
Its neighbors $A'BC$, $AB'C$, and $ABC'$ are found by
reflecting $A$ about~$BC$, $B$ about $C\!A$, and $C$ about $AB$.

Repleated reflections will generate the whole tiling. Notice that,
in this example, four triangles of the complete tiling will surround
point~$A$, eight triangles will surround point~$B$, and
ten triangles will surround point~$C$. (The angles around any vertex
of a tiling must sum to $360^\circ$, even though the
angles of a hyperbolic triangle always sum to {\it less\/} than
$180^\circ$.)

Incidentally, I could have stored $r^2$ instead of $r$ in the
\PB{\&{circle}} nodes, because this program computes reflections from
$r^2$. But what the heck, I prefer to work with~$r$ instead of
$r^2$ when I'm looking at this stuff.

\Y\B\4\X3:Subroutines\X${}\mathrel+\E{}$\6
\&{point} \\{reflect}(\&{point} \|z${},\39{}$\&{circle} \|l)\1\1\2\2\6
${}\{{}$\1\6
\&{point} \|t;\6
\&{register} \&{double} \\{alpha};\7
${}\\{alpha}\K\|l.\|r*\|l.\|r/((\|z.\|x-\|l.\|c)*(\|z.\|x-\|l.\|c)+\|z.\|y*\|z.%
\|y);{}$\6
${}\|t.\|x\K\|l.\|c+\\{alpha}*(\|z.\|x-\|l.\|c);{}$\6
${}\|t.\|y\K\\{alpha}*\|z.\|y;{}$\6
\&{return} \|t;\6
\4${}\}{}$\2\par
\fi

\M{5}As our algorithm proceeds, it will repeatedly compute points
and/or circles that have already been seen. Therefore we maintain
a dictionary of what we know.

At first I tried using a hash table. But that was unsatisfactory,
because near-but-unequal values should be considered equivalent.
Therefore binary search trees are used in the present code.

In practice, I found that most of the equivalent values agreed to
within $10^{-16}$ or so, although exact agreement was rather rare.
Only two cases had an absolute error greater than $10^{-11}$, and
in those cases the error was $\approx 1.1\times10^{-10}$.

The following routines return an index to the saved copy of
a given point or circle.

\Y\B\4\D$\\{eps}$ \5
\T{0.000001}\C{ fuzziness for comparisons }\par
\Y\B\4\X3:Subroutines\X${}\mathrel+\E{}$\6
\&{int} \\{savepoint}(\&{point} \|z)\1\1\2\2\6
${}\{{}$\1\6
\&{register} \&{int} \|p${},{}$ ${}{*}\|q\K{\AND}\\{pleft}[\T{0}];{}$\7
\&{for} ${}(\|p\K{*}\|q;{}$ \|p; ${}\|p\K{*}\|q){}$\5
${}\{{}$\1\6
\&{if} ${}(\\{fabs}(\\{hpoint}[\|p].\|x-\|z.\|x)<\\{eps}){}$\5
${}\{{}$\1\6
\&{if} ${}(\\{fabs}(\\{hpoint}[\|p].\|y-\|z.\|y)<\\{eps}){}$\1\5
\&{goto} \\{found};\2\6
\&{if} ${}(\\{hpoint}[\|p].\|y<\|z.\|y){}$\1\5
${}\|q\K{\AND}\\{pleft}[\|p];{}$\2\6
\&{else}\1\5
${}\|q\K{\AND}\\{pright}[\|p];{}$\2\6
\4${}\}{}$\5
\2\&{else} \&{if} ${}(\\{hpoint}[\|p].\|x<\|z.\|x){}$\1\5
${}\|q\K{\AND}\\{pleft}[\|p];{}$\2\6
\&{else}\1\5
${}\|q\K{\AND}\\{pright}[\|p];{}$\2\6
\4${}\}{}$\2\6
${}\|p\K\PP\\{pptr};{}$\6
${}{*}\|q\K\|p;{}$\6
${}\\{printf}(\.{"z\%d=(\%.15g,\%.15g)\\n}\)\.{"},\39\|p,\39\|z.\|x,\39\|z.%
\|y);{}$\6
${}\\{hpoint}[\|p]\K\|z;{}$\6
\4\\{found}:\5
\&{return} \|p;\6
\4${}\}{}$\2\7
\&{int} \\{savecircle}(\&{circle} \|l)\1\1\2\2\6
${}\{{}$\1\6
\&{register} \&{int} \|p${},{}$ ${}{*}\|q\K{\AND}\\{cleft}[\T{0}];{}$\7
\&{for} ${}(\|p\K{*}\|q;{}$ \|p; ${}\|p\K{*}\|q){}$\5
${}\{{}$\1\6
\&{if} ${}(\\{fabs}(\\{hcircle}[\|p].\|c-\|l.\|c)<\\{eps}){}$\5
${}\{{}$\1\6
\&{if} ${}(\\{fabs}(\\{hcircle}[\|p].\|r-\|l.\|r)<\\{eps}){}$\1\5
\&{goto} \\{found};\2\6
\&{if} ${}(\\{hcircle}[\|p].\|r<\|l.\|r){}$\1\5
${}\|q\K{\AND}\\{cleft}[\|p];{}$\2\6
\&{else}\1\5
${}\|q\K{\AND}\\{cright}[\|p];{}$\2\6
\4${}\}{}$\5
\2\&{else} \&{if} ${}(\\{hcircle}[\|p].\|c<\|l.\|c){}$\1\5
${}\|q\K{\AND}\\{cleft}[\|p];{}$\2\6
\&{else}\1\5
${}\|q\K{\AND}\\{cright}[\|p];{}$\2\6
\4${}\}{}$\2\6
${}\|p\K\PP\\{cptr};{}$\6
${}{*}\|q\K\|p;{}$\6
${}\\{printf}(\.{"l\%d=(\%.15g,\%.15g)\\n}\)\.{"},\39\|p,\39\|l.\|c,\39\|l.%
\|r);{}$\6
${}\\{hcircle}[\|p]\K\|l;{}$\6
\4\\{found}:\5
\&{return} \|p;\6
\4${}\}{}$\2\par
\fi

\M{6}\B\X6:Global variables\X${}\E{}$\6
\&{point} ${}\\{hpoint}[\T{3}*\\{maxn}]{}$;\C{ dictionary of known points }\6
\&{int} \\{pptr};\C{ the number of known points }\6
\&{int} ${}\\{pleft}[\T{3}*\\{maxn}],{}$ ${}\\{pright}[\T{3}*\\{maxn}]{}$;\C{
links for binary tree search }\6
\&{circle} ${}\\{hcircle}[\T{3}*\\{maxn}]{}$;\C{ dictionary of known hyperbolic
lines }\6
\&{int} \\{cptr};\C{ the number of known lines }\6
\&{int} ${}\\{cleft}[\T{3}*\\{maxn}],{}$ ${}\\{cright}[\T{3}*\\{maxn}]{}$;\C{
links for binary tree search }\par
\As9\ET10.
\U1.\fi

\M{7}The main component of our data structure is the table of all
triangles that we have identified so far. Each triangle is
represented by indices that point to its three vertices, its
three edges, and its three neighbors.

The vertex indices are called \PB{\\{v36}}, \PB{\\{v45}}, and \PB{\\{v90}},
because each
triangle has a vertex with each of the angles $(36^\circ,45^\circ,90^\circ)$.
Edges \PB{\\{e36}}, \PB{\\{e45}}, and \PB{\\{e90}} are {\it opposite\/} those
vertices;
triangles \PB{\\{t36}}, \PB{\\{t45}}, and \PB{\\{t90}} are the neighbors on the
other side
of those edges.

\Y\B\4\X2:Type definitions\X${}\mathrel+\E{}$\6
\&{typedef} \&{struct} ${}\{{}$\1\6
\&{int} \\{v36}${},{}$ \\{v45}${},{}$ \\{v90};\C{ where the vertices occur in %
\PB{\\{hpoint}} }\6
\&{int} \\{e36}${},{}$ \\{e45}${},{}$ \\{e90};\C{ where the edges occur in \PB{%
\\{hcircle}} }\6
\&{int} \\{t36}${},{}$ \\{t45}${},{}$ \\{t90};\C{ where the neighbors occur in %
\PB{\\{triang}} }\2\6
${}\}{}$ \&{triangle};\par
\fi

\M{8}An auxiliary hash table keeps track of the triangles we've seen.

(I've imposed the restriction \PB{$\T{3}*\\{maxn}<\T{1024}$} simply because I
want to
pack the values \PB{$(\\{v36},\\{v45},\\{v90})$} into a single word on my
old 32-bit computer.)

\Y\B\4\X3:Subroutines\X${}\mathrel+\E{}$\6
\&{int} \\{savetriangle}(\&{int} \\{v36}${},\39{}$\&{int} \\{v45}${},\39{}$%
\&{int} \\{v90})\1\1\2\2\6
${}\{{}$\1\6
\&{register} \&{unsigned} \&{int} \|w${}\K(((\\{v36}\LL\T{10})+\\{v45})\LL%
\T{10})+\\{v90};{}$\6
\&{register} \&{int} \|h${}\K\|w\MOD\\{hprime};{}$\7
\&{while} (\\{triple}[\|h])\5
${}\{{}$\1\6
\&{if} ${}(\\{triple}[\|h]\E\|w){}$\1\5
\&{goto} \\{found};\2\6
${}\|h\K(\|h\?\|h:\\{hprime})-\T{1};{}$\6
\4${}\}{}$\2\6
${}\\{triple}[\|h]\K\|w,\39\\{tripnum}[\|h]\K\\{tptr};{}$\6
${}\\{triang}[\\{tptr}].\\{v36}\K\\{v36},\39\\{triang}[\\{tptr}].\\{v45}\K%
\\{v45},\39\\{triang}[\\{tptr}].\\{v90}\K\\{v90};{}$\6
${}\\{tptr}\PP;{}$\6
\4\\{found}:\5
\&{return} \\{tripnum}[\|h];\6
\4${}\}{}$\2\par
\fi

\M{9}\B\X6:Global variables\X${}\mathrel+\E{}$\6
\&{int} \\{triple}[\\{hprime}];\C{ the vertex triples that we've seen }\6
\&{int} \\{tripnum}[\\{hprime}];\C{ their serial numbers }\6
\&{int} \\{tptr};\C{ the number of triangles we've seen }\6
\&{triangle} ${}\\{triang}[\\{maxn}+\T{3}]{}$;\C{ their details }\par
\fi

\N{1}{10}Getting started. To prime the pump, I need one
$36^\circ$-$45^\circ$-$90^\circ$ triangle to begin the process.
The simplest one that I could think of
has vertices $e^{i\theta}$, $i/r$, and $i$ in the complex plane,
where
$$r=\sqrt{\phi+\sqrt{\phi}},\qquad
\cos\theta=1/\sqrt{\mskip1mu2\phi},$$
and $\phi=(1+\sqrt5\,)/2$ is the golden ratio.

\Y\B\4\D$\\{makepoint}(\|v,\\{xx},\\{yy})$ \5
$\|z.\|x\K\\{xx},\39\|z.\|y\K\\{yy},\39\|v\K\\{savepoint}{}$(\|z)\par
\B\4\D$\\{makecircle}(\|v,\\{cc},\\{rr})$ \5
$\|l.\|c\K\\{cc},\39\|l.\|r\K\\{rr},\39\|v\K\\{savecircle}{}$(\|l)\par
\Y\B\4\X6:Global variables\X${}\mathrel+\E{}$\6
\&{point} \|z;\C{ staging area for \PB{\\{makepoint}} }\6
\&{circle} \|l;\C{ staging area for \PB{\\{makecircle}} }\par
\fi

\M{11}\B\X11:Set up triangle 0\X${}\E{}$\6
$\\{phi}\K(\T{1.0}+\\{sqrt}(\T{5.0}))/\T{2.0};{}$\6
${}\\{makepoint}(\|a,\39\\{sqrt}(\T{0.5}/\\{phi}),\39\\{sqrt}(\T{1.0}-\T{0.5}/%
\\{phi}));{}$\6
${}\\{makepoint}(\|b,\39\T{0.0},\39\T{1.0}/\\{sqrt}(\\{phi}+\\{sqrt}(%
\\{phi})));{}$\6
${}\\{makepoint}(\|c,\39\T{0.0},\39\T{1.0});{}$\6
\X12:Compute the edges of triangle 0\X;\6
${}\\{savetriangle}(\|a,\39\|b,\39\|c){}$;\C{ now \PB{\\{tptr}} will equal 1 }\6
${}\\{printf}(\.{"triangle\ 0\ =\ (\%d,\%d}\)\.{,\%d),"},\39\|a,\39\|b,\39%
\|c);{}$\6
${}\\{printf}(\.{"\ edges\ (*,\%d,\%d)\\n"},\39\\{triang}[\T{0}].\\{e45},\39%
\\{triang}[\T{0}].\\{e90}){}$;\par
\U1.\fi

\M{12}Edge \PB{\\{e36}} of the starting triangle is a vertical line from
$i/r$ to $i$. This is the only vertical line that we will need in
the present program. (In fact, I'll explain shortly that
the computations will be limited to the subtiling that appears
in an annulus. Then one can prove
a strict upper bound on the size of the $c$ values that occur,
even if the computation proceeds indefinitely.)

It turns out easiest to handle the exceptional vertical case by
setting \PB{$\\{e36}\K\\{e45}$}. I~do admit however that this is a sneaky
trick,
hard to justify on moral grounds.

\Y\B\4\X12:Compute the edges of triangle 0\X${}\E{}$\6
$\\{makecircle}(\\{triang}[\T{0}].\\{e45},\39\T{0.0},\39\T{1.0});{}$\6
${}\\{triang}[\T{0}].\\{e36}\K\\{triang}[\T{0}].\\{e45};{}$\6
${}\\{makecircle}(\\{triang}[\T{0}].\\{e90},\39\\{hpoint}[\|b].\|y,\39\\{sqrt}(%
\T{2.0})*\\{hpoint}[\|b].\|y){}$;\par
\U11.\fi

\N{1}{13}The algorithm. One more important thing needs to be mentioned,
before we put all the pieces together: I don't actually
want to compute the entire tiling. I only want to compute it
in the quarter-annulus that consists of the points
between circles $\vert z\vert=1$ and $\vert z\vert=1/r$, in
the upper right quadrant of the complex plane.

The reason is that the tiling between this annulus and the
next-smaller one, $\vert z\vert=1/r^2$, is just the reflection
of the first tiling about the line $\vert z\vert=1/r$. Then in
the next annulus, we shrink the outer-annulus tiling by a factor
of $1/r^2$, and so on.

Indeed, this restriction of the tiling accounts for my claims that
triangle~0 is the only triangle with a vertical edge.

To implement the restriction, we simply refrain from computing
the neighbor at any edge whose center~$c$ is zero. (And
that is why the sneaky trick mentioned on the previous page
actually works.)

\Y\B\4\X13:Compute the neighbors of triangle \PB{\|k}\X${}\E{}$\6
${}\{{}$\1\6
\&{if} ${}(\\{hcircle}[\\{triang}[\|k].\\{e36}].\|c){}$\1\5
\X14:Compute \PB{\\{t36}}\X;\2\6
\&{if} ${}(\\{hcircle}[\\{triang}[\|k].\\{e45}].\|c){}$\1\5
\X15:Compute \PB{\\{t45}}\X;\2\6
\&{if} ${}(\\{hcircle}[\\{triang}[\|k].\\{e90}].\|c){}$\1\5
\X16:Compute \PB{\\{t90}}\X;\2\6
${}\\{printf}(\.{"Triangle\ \%d\ neighbo}\)\.{rs:"},\39\|k);{}$\6
\&{if} ${}(\\{hcircle}[\\{triang}[\|k].\\{e36}].\|c){}$\1\5
${}\\{printf}(\.{"\ t36=\%d"},\39\\{triang}[\|k].\\{t36});{}$\2\6
\&{if} ${}(\\{hcircle}[\\{triang}[\|k].\\{e45}].\|c){}$\1\5
${}\\{printf}(\.{"\ t45=\%d"},\39\\{triang}[\|k].\\{t45});{}$\2\6
\&{if} ${}(\\{hcircle}[\\{triang}[\|k].\\{e90}].\|c){}$\1\5
${}\\{printf}(\.{"\ t90=\%d"},\39\\{triang}[\|k].\\{t90});{}$\2\6
\\{printf}(\.{"\\n"});\6
\4${}\}{}$\2\par
\U1.\fi

\M{14}\B\X14:Compute \PB{\\{t36}}\X${}\E{}$\6
${}\{{}$\1\6
${}\|j\K\\{savepoint}(\\{reflect}(\\{hpoint}[\\{triang}[\|k].\\{v36}],\39%
\\{hcircle}[\\{triang}[\|k].\\{e36}]));{}$\6
${}\|t\K\\{tptr};{}$\6
${}\\{triang}[\|k].\\{t36}\K\\{savetriangle}(\|j,\39\\{triang}[\|k].\\{v45},\39%
\\{triang}[\|k].\\{v90});{}$\6
\&{if} ${}(\\{tptr}>\|t){}$\5
${}\{{}$\C{ that triangle is new }\1\6
${}\\{triang}[\|t].\\{e36}\K\\{triang}[\|k].\\{e36};{}$\6
${}\\{triang}[\|t].\\{e45}\K\\{savecircle}(\\{common}(\\{hpoint}[\\{triang}[%
\|t].\\{v36}],\39\\{hpoint}[\\{triang}[\|t].\\{v90}]));{}$\6
${}\\{triang}[\|t].\\{e90}\K\\{savecircle}(\\{common}(\\{hpoint}[\\{triang}[%
\|t].\\{v36}],\39\\{hpoint}[\\{triang}[\|t].\\{v45}]));{}$\6
${}\\{printf}(\.{"triangle\ \%d\ =\ (z\%d,}\)\.{z\%d,z\%d),"},\39\|t,\39%
\\{triang}[\|t].\\{v36},\39\\{triang}[\|t].\\{v45},\39\\{triang}[\|t].%
\\{v90});{}$\6
${}\\{printf}(\.{"\ edges\ (\%d,\%d,\%d)\\n}\)\.{"},\39\\{triang}[\|t].\\{e36},%
\39\\{triang}[\|t].\\{e45},\39\\{triang}[\|t].\\{e90});{}$\6
\4${}\}{}$\2\6
\4${}\}{}$\2\par
\U13.\fi

\M{15}\B\X15:Compute \PB{\\{t45}}\X${}\E{}$\6
${}\{{}$\1\6
${}\|j\K\\{savepoint}(\\{reflect}(\\{hpoint}[\\{triang}[\|k].\\{v45}],\39%
\\{hcircle}[\\{triang}[\|k].\\{e45}]));{}$\6
${}\|t\K\\{tptr};{}$\6
${}\\{triang}[\|k].\\{t45}\K\\{savetriangle}(\\{triang}[\|k].\\{v36},\39\|j,\39%
\\{triang}[\|k].\\{v90});{}$\6
\&{if} ${}(\\{tptr}>\|t){}$\5
${}\{{}$\C{ that triangle is new }\1\6
${}\\{triang}[\|t].\\{e45}\K\\{triang}[\|k].\\{e45};{}$\6
${}\\{triang}[\|t].\\{e36}\K\\{savecircle}(\\{common}(\\{hpoint}[\\{triang}[%
\|t].\\{v45}],\39\\{hpoint}[\\{triang}[\|t].\\{v90}]));{}$\6
${}\\{triang}[\|t].\\{e90}\K\\{savecircle}(\\{common}(\\{hpoint}[\\{triang}[%
\|t].\\{v36}],\39\\{hpoint}[\\{triang}[\|t].\\{v45}]));{}$\6
${}\\{printf}(\.{"triangle\ \%d\ =\ (z\%d,}\)\.{z\%d,z\%d),"},\39\|t,\39%
\\{triang}[\|t].\\{v36},\39\\{triang}[\|t].\\{v45},\39\\{triang}[\|t].%
\\{v90});{}$\6
${}\\{printf}(\.{"\ edges\ (\%d,\%d,\%d)\\n}\)\.{"},\39\\{triang}[\|t].\\{e36},%
\39\\{triang}[\|t].\\{e45},\39\\{triang}[\|t].\\{e90});{}$\6
\4${}\}{}$\2\6
\4${}\}{}$\2\par
\U13.\fi

\M{16}\B\X16:Compute \PB{\\{t90}}\X${}\E{}$\6
${}\{{}$\1\6
${}\|j\K\\{savepoint}(\\{reflect}(\\{hpoint}[\\{triang}[\|k].\\{v90}],\39%
\\{hcircle}[\\{triang}[\|k].\\{e90}]));{}$\6
${}\|t\K\\{tptr};{}$\6
${}\\{triang}[\|k].\\{t90}\K\\{savetriangle}(\\{triang}[\|k].\\{v36},\39%
\\{triang}[\|k].\\{v45},\39\|j);{}$\6
\&{if} ${}(\\{tptr}>\|t){}$\5
${}\{{}$\C{ that triangle is new }\1\6
${}\\{triang}[\|t].\\{e90}\K\\{triang}[\|k].\\{e90};{}$\6
${}\\{triang}[\|t].\\{e36}\K\\{savecircle}(\\{common}(\\{hpoint}[\\{triang}[%
\|t].\\{v45}],\39\\{hpoint}[\\{triang}[\|t].\\{v90}]));{}$\6
${}\\{triang}[\|t].\\{e45}\K\\{savecircle}(\\{common}(\\{hpoint}[\\{triang}[%
\|t].\\{v36}],\39\\{hpoint}[\\{triang}[\|t].\\{v90}]));{}$\6
${}\\{printf}(\.{"triangle\ \%d\ =\ (z\%d,}\)\.{z\%d,z\%d),"},\39\|t,\39%
\\{triang}[\|t].\\{v36},\39\\{triang}[\|t].\\{v45},\39\\{triang}[\|t].%
\\{v90});{}$\6
${}\\{printf}(\.{"\ edges\ (\%d,\%d,\%d)\\n}\)\.{"},\39\\{triang}[\|t].\\{e36},%
\39\\{triang}[\|t].\\{e45},\39\\{triang}[\|t].\\{e90});{}$\6
\4${}\}{}$\2\6
\4${}\}{}$\2\par
\U13.\fi

\N{1}{17}Output. Here's what we get when the circles \.{l1}, \.{l2}, \dots~are
plotted:
\bigskip
\centerline{\epsfbox{hyperbolic.2}}

\fi

\N{1}{18}Dual output. And here's what happens when those circles are reflected
with
respect to $\vert z\vert=1/r$:
\bigskip
\centerline{\epsfbox{hyperbolic.3}}

\fi

\N{1}{19}The overall tiling. Finally, the right half of the
whole thing, omitting circles of radius $<0.007$:
\bigskip
\centerline{\epsfbox{hyperbolic.4}}

\fi

\N{1}{20}Index.

\fi


\inx
\fin
\con
