\input cwebmac
\hypertextrue\srcloctrue

\N[1 co-debruijn.w]{1}{1}Introduction. This program implements the coroutines
of Algorithms
7.2.1.1R and 7.2.1.1D, in the important case $m=2$.

\Y\B\4\D$\\{nn}$ \5
\T{10}\C{ we will test this value of \PB{\|n} }\par
\Y\B\8\#\&{include} \.{<stdio.h>}\6
\&{int} \|p[\\{nn}];\C{ program locations }\6
\&{int} \|x[\\{nn}]${},{}$ \|y[\\{nn}]${},{}$ \|t[\\{nn}]${},{}$ \\{xp}[%
\\{nn}]${},{}$ \\{yp}[\\{nn}]${},{}$ \\{tp}[\\{nn}];\C{ local variables }\6
\&{int} \|n[\\{nn}];\C{ the value of `\PB{\|n}' in each coroutine }\7
\X2:Subroutines\X;\7
\\{main}(\,)\1\1\2\2\6
${}\{{}$\1\6
\&{register} \&{int} \|k${},{}$ \\{kp};\7
\X3:Initialize the coroutines\X;\6
\&{for} ${}(\|k\K\T{0};{}$ ${}\|k<(\T{1}\LL\\{nn});{}$ ${}\|k\PP){}$\1\5
${}\\{printf}(\.{"\%d"},\39\\{co}(\\{nn}-\T{1}));{}$\2\6
\\{printf}(\.{"\\n"});\6
\4${}\}{}$\2\par
\fi

\M[20 co-debruijn.w]{2}We simulate the behavior of recursive coroutines, in
such a way that
repeated calls on \PB{$\\{co}(\|n-\T{1})$} will yield a
cyclic sequence of period~$2^{nn}$ in which each $nn$-tuple
occurs exactly once.

The coroutines are of types S (simple), R (recursive), and
D (doubly recursive), as explained in the book. There are $nn-1$
coroutines altogether (see exercise 96); the main one will
be number \PB{$\\{nn}-\T{1}$}.

Each coroutine \PB{\|q}, for $1\le q<nn$, has a current position \PB{\|p[\|q]},
as well as local variables \PB{\|x[\|q]}, \PB{\|y[\|q]}, and so on; and it
generates a de Bruijn sequence of length $2^{n[q]}$.

If $n=2$, the coroutine for order~$n$ simply outputs the sequence 0, 0, 1, 1.
Otherwise, if $n$ is odd, coroutine \PB{$\|q\K\|n-\T{1}$} invokes coroutine
\PB{$\|q-\T{1}\K\|n-\T{2}$} and doubles its length.
Otherwise coroutine $q=2n'-1$ invokes coroutines
$q-1=2n'-2$ and $(q-1)/2=n'-1$, where coroutines $2n'-1$ through
$n'$ are ``clones'' of coroutines $n'-1$ through~1; the effect
is to square the length of the cycles output by those coroutines.

\Y\B\4\D$\|S$ \5
\T{0}\C{ base for positions of an S coroutine }\par
\B\4\D$\|R$ \5
\T{10}\C{ base for positions of an R coroutine }\par
\B\4\D$\|D$ \5
\T{20}\C{ base for positions of a D coroutine }\par
\Y\B\4\X2:Subroutines\X${}\E{}$\6
\&{void} \\{init}(\&{int} \|r)\1\1\2\2\6
${}\{{}$\1\6
\&{register} \|q${}\K\|r-\T{1};{}$\7
${}\|n[\|q]\K\|r;{}$\6
\&{if} ${}(\|r\E\T{2}){}$\1\5
${}\|p[\|q]\K\|S+\T{1};{}$\2\6
\&{else} \&{if} ${}(\|r\AND\T{1}){}$\5
${}\{{}$\1\6
${}\|p[\|q]\K\|R;{}$\6
${}\|x[\|q]\K\T{0};{}$\6
\\{init}(\|q);\6
\4${}\}{}$\5
\2\&{else}\5
${}\{{}$\1\6
\&{register} \&{int} \|k${},{}$ \\{qq};\7
${}\\{qq}\K\|q\GG\T{1};{}$\6
${}\|p[\|q]\K\|D+\T{1};{}$\6
${}\|x[\|q]\K\\{xp}[\|q]\K\T{2};{}$\6
${}\\{init}(\\{qq}+\T{1});{}$\6
\&{for} ${}(\|k\K\|q-\T{1};{}$ ${}\|k>\\{qq};{}$ ${}\|k\MM){}$\1\5
${}\|p[\|k]\K\|p[\|k-\\{qq}],\39\|x[\|k]\K\|x[\|k-\\{qq}],\39\\{xp}[\|k]\K%
\\{xp}[\|k-\\{qq}],\39\|n[\|k]\K\|n[\|k-\\{qq}];{}$\2\6
\4${}\}{}$\2\6
\4${}\}{}$\2\par
\A4.
\U1.\fi

\M[66 co-debruijn.w]{3}\B\X3:Initialize the coroutines\X${}\E{}$\6
\\{init}(\\{nn});\par
\U1.\fi

\M[69 co-debruijn.w]{4}Now here's how we invoke a coroutine and obtain its next
value.

\Y\B\4\X2:Subroutines\X${}\mathrel+\E{}$\6
\&{int} \\{co}(\&{int} \|q)\1\1\2\2\6
${}\{{}$\1\6
\&{switch} (\|p[\|q])\5
${}\{{}$\1\6
\X5:Cases for individual coroutines\X\6
\4${}\}{}$\2\6
\4${}\}{}$\2\par
\fi

\M[78 co-debruijn.w]{5}Each coroutine resets its \PB{\|p} before returning a
value.
For example, type S is the simplest.

\Y\B\4\X5:Cases for individual coroutines\X${}\E{}$\6
\4\&{case} \|S${}+\T{1}{}$:\5
${}\|p[\|q]\K\|S+\T{2}{}$;\5
\&{return} \T{0};\6
\4\&{case} \|S${}+\T{2}{}$:\5
${}\|p[\|q]\K\|S+\T{3}{}$;\5
\&{return} \T{0};\6
\4\&{case} \|S${}+\T{3}{}$:\5
${}\|p[\|q]\K\|S+\T{4}{}$;\5
\&{return} \T{1};\6
\4\&{case} \|S${}+\T{4}{}$:\5
${}\|p[\|q]\K\|S+\T{1}{}$;\5
\&{return} \T{1};\par
\As6\ET7.
\U4.\fi

\M[87 co-debruijn.w]{6}Type R is next in difficulty. I change the numbering
slightly here,
so that case \PB{\|R} does the first part of the text's step R1.
The text's \PB{\|n} is \PB{$\|n[\|q-\T{1}]$} in this code, because of the
initialization we've done.

\Y\B\4\X5:Cases for individual coroutines\X${}\mathrel+\E{}$\6
\4\.{R1}:\5
\&{case} \|R:\5
${}\|p[\|q]\K\|R+\T{1}{}$;\5
\&{return} \|x[\|q];\6
\4\&{case} \|R${}+\T{1}{}$:\5
\&{if} ${}(\|x[\|q]\I\T{0}\W\|t[\|q]\G\|n[\|q-\T{1}]){}$\1\5
\&{goto} \.{R3};\2\6
\4\.{R2}:\5
${}\|y[\|q]\K\\{co}(\|q-\T{1});{}$\6
\4\.{R3}:\5
${}\|t[\|q]\K(\|y[\|q]\E\T{1}\?\|t[\|q]+\T{1}:\T{0});{}$\6
\4\.{R4}:\5
\&{if} ${}(\|t[\|q]\E\|n[\|q-\T{1}]\W\|x[\|q]\I\T{0}){}$\1\5
\&{goto} \.{R2};\2\6
\4\.{R5}:\5
${}\|x[\|q]\K(\|x[\|q]+\|y[\|q])\AND\T{1}{}$;\5
\&{goto} \.{R1};\par
\fi

\M[100 co-debruijn.w]{7}And finally there's the coroutine of type D. Again the
text's parameter~\PB{\|n}
is our variable~\PB{$\|n[\|q-\T{1}]$}.

\Y\B\4\X5:Cases for individual coroutines\X${}\mathrel+\E{}$\6
\4\.{D1}:\5
\&{case} \|D${}+\T{1}{}$:\5
\&{if} ${}(\|t[\|q]\I\|n[\|q-\T{1}]\V\|x[\|q]\G\T{2}){}$\1\5
${}\|y[\|q]\K\\{co}(\|q-(\|n[\|q]\GG\T{1}));{}$\2\6
\4\.{D2}:\5
\&{if} ${}(\|x[\|q]\I\|y[\|q]){}$\1\5
${}\|x[\|q]\K\|y[\|q],\39\|t[\|q]\K\T{1}{}$;\5
\2\&{else}\1\5
${}\|t[\|q]\PP;{}$\2\6
\4\.{D3}:\5
${}\|p[\|q]\K\|D+\T{4}{}$;\5
\&{return} \|x[\|q];\6
\4\.{D4}:\5
\&{case} \|D${}+\T{4}{}$:\5
${}\\{yp}[\|q]\K\\{co}(\|q-\T{1});{}$\6
\4\.{D5}:\5
\&{if} ${}(\\{xp}[\|q]\I\\{yp}[\|q]){}$\1\5
${}\\{xp}[\|q]\K\\{yp}[\|q],\39\\{tp}[\|q]\K\T{1}{}$;\5
\2\&{else}\1\5
${}\\{tp}[\|q]\PP;{}$\2\6
\4\.{D6}:\5
\&{if} ${}(\\{tp}[\|q]\E\|n[\|q-\T{1}]\W\\{xp}[\|q]<\T{2}){}$\5
${}\{{}$\1\6
\&{if} ${}(\|t[\|q]<\|n[\|q-\T{1}]\V\\{xp}[\|q]<\|x[\|q]){}$\1\5
\&{goto} \.{D4};\2\6
\&{if} ${}(\\{xp}[\|q]\E\|x[\|q]){}$\1\5
\&{goto} \.{D3};\2\6
\4${}\}{}$\2\6
\4\.{D7}:\5
${}\|p[\|q]\K\|D+\T{8}{}$;\5
\&{return} (\\{xp}[\|q]);\6
\4\&{case} \|D${}+\T{8}{}$:\5
\&{if} ${}(\\{tp}[\|q]\E\|n[\|q-\T{1}]\W\\{xp}[\|q]<\T{2}){}$\1\5
\&{goto} \.{D3};\2\6
\&{goto} \.{D1};\par
\fi

\N[117 co-debruijn.w]{1}{8}Index.
\fi

\inx
\fin
\con
