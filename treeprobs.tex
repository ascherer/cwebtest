\input cwebmac
\datethis
\def\[#1]{[\hbox{$\mkern1mu\thickmuskip=\thinmuskip#1\mkern1mu$}]} % Iverson
\input epsf


\N{1}{1}Intro. I'm experimenting with what may be a new twist(?) on dynamic
programming. It's motivated by ``Bayesian networks'' that form a binary tree.
With this method we can answer queries that are much different from
the usual ``marginal'' distributions. For example, with binary states,
we can determine the probability that exactly half of the nodes are~1,
in $O(n^3)$ steps. In general we can determine the probability that
a Boolean function $f(x_1,\ldots,x_n)$ is true, as long as the BDD for
that function is small when the nodes appear in arbitrary order.
(More precisely, I have a particular order in mind, for each binary tree;
the function should have a small BDD when the variables are inspected
in that order.)

Here's the problem: Given a binary tree of $n$ nodes, with $n-1$
weight functions $w_k(x_j,x_k)$ on the edge from node~$j$ to a
child node~$k$.
Assign binary values $(x_1,\ldots,x_n)$ to the nodes. Every such state
occurs with relative weight $W(x_1,\ldots,x_n)=\prod w_k(x_j,x_k)$,
where the product is over all edges.

For example, the binary tree
$$\vcenter{\epsfbox{treeprobs.1}}$$
has the weight function
$$\displaylines{\qquad\qquad
W(x_1,\ldots,x_{10})=w_2(x_1,x_2)\,w_3(x_1,x_3)\,w_4(x_3,x_4)\,
w_5(x_4,x_5)
\hfill\cr\hfill
w_6(x_4,x_6)\,w_7(x_3,x_7)\,w_8(x_7,x_8)\,w_9(x_8,x_9)\,
w_{10}(x_8,x_{10}).
\qquad\qquad\cr}$$

Without loss of generality we can assume that the left subtree of each node
has at most as many nodes as the corresponding right subtree, and that
the nodes have been numbered in preorder. Both of these assumptions
are in fact fulfilled in the example above. (Surprise!)

If the QDD for a Boolean function $f(x_1,\ldots,x_n)$
has $N$ nodes, a variant of the algorithm below computes
$\sum\{W(x_1,\ldots,x_n) \mid f(x_1,\ldots,x_n)=1\}$ in $O(nN)$ steps.
Here I demonstrate the idea when $f$ is the symmetric function
$S_m(x_1,\ldots,x_n)=\[x_1+\cdots+x_n=m]$.

\fi

\M{2}For $1\le i\le n$, let $W_i(x_i,\ldots,x_n)=\prod w_k(x_j,x_k)$,
where the product includes only edges $jk$ with $j\ge i$. Thus, for instance,
$W_7(x_7,x_8,x_9,x_{10})$ in the example above is
$\,w_8(x_7,x_8)\,w_9(x_8,x_9)\,w_{10}(x_8,x_{10})$.
In general we have $W_n(x_n)=1$ and $W_1(x_1,\ldots,x_n)=W(x_1,\ldots,x_n)$.

If node $j$ has no children, $W_j(x_j,\ldots,x_n)=W_{j+1}(x_{j+1},\ldots,x_n)$.
If node $j$ has one child, it's the right child, and it's node $j+1$;
hence $W_j(x_j,\dots,x_n)=w_{j+1}(x_j,x_{j+1})\,W_{j+1}(x_{j+1},\ldots,x_n)$
in that case. Otherwise node~$j$ has two children, $j+1$ and $k$; then
we have
$$W_j(x_j,\dots,x_n)=w_{j+1}(x_j,x_{j+1})\,w_k(x_j,x_k)\,
W_{j+1}(x_{j+1},\ldots,x_n).$$

Let $S_j$ be the set of all $x_k$ such that $k>j$ and $k$ is the right child
of~$i$ for some $i<j$. For example, the $S$'s corresponding to the tree above
are
$$\eqalign{
S_1&=\emptyset,\cr
S_2&=\{x_3\},\cr
S_3&=\emptyset,\cr
S_4&=\{x_7\},\cr
S_5&=\{x_6,x_7\},\cr
}\hskip4em\eqalign{
S_6&=\{x_7\},\cr
S_7&=\emptyset,\cr
S_8&=\emptyset,\cr
S_9&=\{x_{10}\},\cr
S_{10}&=\emptyset.\cr
}$$
These sets are easy to compute, for increasing values of $j$:
$$\openup1\jot
S_{j+1}=\cases{
S_j\setminus\{x_{j+1}\},&if node $j$ is childless;\cr
S_j,                    &if it has just the right child $j+1$;\cr
S_j\cup\{x_k\},         &if its children are $j+1$ and $k$.\cr}$$

\goodbreak


\fi

\M{3}What's the point? Well, the $S$'s allow us to compute the functions
$$T_j(s,x_j,S_j)=\sum\,\{W_j(x_j,\ldots,x_n)\mid x_j+\cdots+x_n=s\},$$
where the sum is over all variables $x_k$ with $k>j$ that are not
in $S_j$. For example, in the tree above,
$$T_6(2,0,1)=\sum_{p=0}^1\sum_{q=0}^1\sum_{r=0}^1
W_6(0,1,p,q,r)\[p+q+r=1].$$
The overall answer that we're trying to compute is $T_1(m,0)+T_1(m,1)$.

And the $T$'s satisfy a simple bottom-up recursion, starting with
$$T_n(s,x_n)\;=\;\[x_n=s].$$
Namely, if node $j$ is childless, for $j<n$, we have
$T_j(s,x_j,S_j)=T_{j+1}(s-x_j,x_{j+1},S_{j+1})$; notice that this formula
makes sense, because $x_{j+1}\in S_j$ by the definition of preorder.
On the other hand if node~$j$ has the unique child~$j+1$, we have
$T_j(s,x_j,S_j)=\sum_{p=0}^1 w_{j+1}(x_j,p)\,T_{j+1}(s-x_j,p,S_j)$.
And finally if
node~$j$ has both $j+1$ and $k$ as children, the formula is
$$T_j(s,x_j,S_j)=
\sum_{p=0}^1 w_{j+1}(x_j,p)
\sum_{q=0}^1 w_k(x_j,q)\,T_{j+1}(s-x_j,p,q,S_j).$$
In this case $x_k$ is the ``leftmost'' element of~$S_{j+1}$, because the $S$'s
grow in a last-in-first-out manner.

It suffices to restrict the value of $s$ to the range
$\max(0,m+1-j)\le s\le\min(m,n+1-j)$, because no other values
of $s$ at step $j$ contribute to the final $T_1(m,0)$ and $T_1(m,1)$.

\fi

\M{4}Still we might ask, what's the point? We've computed each
function value for $T_j$ with only a few multiplications and additions,
but the number of such function values is potentially huge.
If $S_j$ has $r$ elements, we need to keep $2^{r+1}$ values
of $T_j(s,x_j,S_j)$ for each relevant value of~$s$.

Fortunately, $r$ cannot become very large; and that, in fact, is the
real point of this whole method. The value of $r+1$ cannot exceed
$\lfloor\lg(n+1)\rfloor$ (and it is often much smaller).

{\it Proof.}\enspace
Let $M_n$ be the largest value of $\vert S_j\vert+1$, over all
binary trees with $n$ vertices; we shall show that $M_n=\lfloor\lg(n+1)
\rfloor$, by induction. Clearly $M_0=0$, if we understand that case
properly. When $n\ge0$, it's not difficult to see that
$M_{n+1}=\max_{0\le k\le n-k}(\max(M_k+1,M_{n-k}))$; and this will
exceed $M_n$ only if $M_k=M_{n-k}$, because $M_k\le M_{n-k}$ whenever
$k\le n-k$. QED.

(Even more is true, in fact: The total of $2^{r+1}$ over all levels~$j$
is always $O(n^{\lg 3})=O(n^{1.585})$. Thus the total running time for
the algorithm is $O(n^{2.585})$, not merely $O(n^3)$; in general,
for functions with at most $M$ nodes per level in their QDD, the
running time is $O(n^{1.585}M)$.
I have some notes on this,
and have submitted it as OEIS sequence A193494.)

\fi

\N{1}{5}Implementation. Instead of allocating storage and computing the
results myself, I'm just testing the formulas today. So this program
simply outputs a Mathematica program that does the actual computation.

The input on \PB{\\{stdin}} is supposed to be a list of edge pairs ``$j\ k$'',
one per line, in lexicographic order. The program doesn't check
carefully for bad input, but it does panic if something unexpected
happens.

The command line should contain the parameter $m$.

\Y\B\4\D$\\{maxn}$ \5
\T{100}\par
\B\4\D$\\{bufsize}$ \5
\T{50}\par
\Y\B\8\#\&{include} \.{<stdio.h>}\6
\8\#\&{include} \.{<stdlib.h>}\6
\&{char} \\{buf}[\\{bufsize}];\6
\&{int} \\{edgej}[\\{maxn}]${},{}$ \\{edgek}[\\{maxn}];\6
\&{int} \|S[\\{maxn}][\\{maxn}];\C{ overkill, but we accept left-heavy trees }\6
\&{int} \\{kids}[\\{maxn}];\6
\&{int} \\{where}[\\{maxn}];\6
\&{int} \|x[\\{maxn}];\7
\\{main}(\&{int} \\{argc}${},\39{}$\&{char} ${}{*}\\{argv}[\,]){}$\1\1\2\2\6
${}\{{}$\1\6
\&{register} \&{int} \|j${},{}$ \|k${},{}$ \|n${},{}$ \|p${},{}$ \|q${},{}$ %
\|r${},{}$ \|s;\6
\&{int} \|m;\7
\X6:Parse the command line\X;\6
\X7:Input the tree\X;\6
\X8:Compute the $S$'s\X;\6
\X9:Output the necessary computations\X;\6
\4${}\}{}$\2\par
\fi

\M{6}\B\X6:Parse the command line\X${}\E{}$\6
\&{if} ${}(\\{argc}\I\T{2}\V\\{sscanf}(\\{argv}[\T{1}],\39\.{"\%d"},\39{\AND}%
\|m)\I\T{1}){}$\5
${}\{{}$\1\6
${}\\{fprintf}(\\{stderr},\39\.{"Usage:\ \%s\ m\\n"},\39\\{argv}[\T{0}]);{}$\6
${}\\{exit}({-}\T{99});{}$\6
\4${}\}{}$\2\par
\U5.\fi

\M{7}\B\D$\\{panic}(\\{mess})$ \6
${}\{{}$\5
\1${}\\{fprintf}(\\{stderr},\39\.{"\%s!\\n"},\39\\{mess}){}$;\5
${}\\{exit}({-}\T{1}){}$;\5
${}\}{}$\2\par
\Y\B\4\X7:Input the tree\X${}\E{}$\6
\&{for} ${}(\|n\K\T{1};{}$  ; ${}\|n\PP){}$\5
${}\{{}$\1\6
\&{if} ${}(\R\\{fgets}(\\{buf},\39\\{bufsize},\39\\{stdin})){}$\1\5
\&{break};\2\6
\&{if} ${}(\|n\E\\{maxn}){}$\1\5
\\{panic}(\.{"too\ many\ edges"});\2\6
\&{if} ${}(\\{sscanf}(\\{buf},\39\.{"\%d\ \%d"},\39{\AND}\\{edgej}[\|n],\39{%
\AND}\\{edgek}[\|n])\I\T{2}){}$\1\5
\\{panic}(\.{"bad\ input\ line"});\2\6
${}\\{kids}[\\{edgej}[\|n]]\PP;{}$\6
${}\\{where}[\\{edgej}[\|n]]\K\|n;{}$\6
\4${}\}{}$\2\6
\&{if} ${}(\\{edgek}[\|n-\T{1}]\I\|n){}$\1\5
\\{panic}(\.{"inconsistent\ input"});\2\par
\U5.\fi

\M{8}\B\X8:Compute the $S$'s\X${}\E{}$\6
\&{for} ${}(\|j\K\T{1};{}$ ${}\|j<\|n;{}$ ${}\|j\PP){}$\5
${}\{{}$\1\6
\&{switch} (\\{kids}[\|j])\5
${}\{{}$\1\6
\4\&{case} \T{2}:\5
${}\|S[\|j+\T{1}][\T{0}]\K\\{edgek}[\\{where}[\|j]];{}$\6
\&{for} ${}(\|k\K\T{0};{}$ \|S[\|j][\|k]; ${}\|k\PP){}$\1\5
${}\|S[\|j+\T{1}][\|k+\T{1}]\K\|S[\|j][\|k];{}$\2\6
\&{if} ${}(\\{edgek}[\\{where}[\|j]-\T{1}]\I\|j+\T{1}){}$\1\5
\\{panic}(\.{"bad\ edge\ for\ two-ki}\)\.{d\ node"});\2\6
\&{break};\6
\4\&{case} \T{1}:\5
\&{for} ${}(\|k\K\T{0};{}$ \|S[\|j][\|k]; ${}\|k\PP){}$\1\5
${}\|S[\|j+\T{1}][\|k]\K\|S[\|j][\|k];{}$\2\6
\&{if} ${}(\\{edgek}[\\{where}[\|j]]\I\|j+\T{1}){}$\1\5
\\{panic}(\.{"bad\ edge\ for\ one-ki}\)\.{d\ node"});\2\6
\&{break};\6
\4\&{case} \T{0}:\5
\&{if} ${}(\|S[\|j][\T{0}]\I\|j+\T{1}){}$\1\5
\\{panic}(\.{"bad\ preorder\ for\ no}\)\.{-kid\ node"});\2\6
\&{for} ${}(\|k\K\T{1};{}$ \|S[\|j][\|k]; ${}\|k\PP){}$\1\5
${}\|S[\|j+\T{1}][\|k-\T{1}]\K\|S[\|j][\|k];{}$\2\6
\&{break};\6
\4\&{default}:\5
\\{panic}(\.{"too\ many\ kids"});\6
\4${}\}{}$\2\6
\4${}\}{}$\2\6
\&{if} (\|S[\|n][\T{0}])\1\5
\\{panic}(\.{"S[n]\ not\ empty"});\2\par
\U5.\fi

\M{9}\B\X9:Output the necessary computations\X${}\E{}$\6
\&{for} ${}(\|s\K\T{0};{}$ ${}\|s<\T{2};{}$ ${}\|s\PP){}$\1\6
\&{for} ${}(\|k\K\T{0};{}$ ${}\|k<\T{2};{}$ ${}\|k\PP){}$\1\5
${}\\{printf}(\.{"T[\%d,\%d,\%d]=\%d\\n"},\39\|n,\39\|s,\39\|k,\39\|s\E\|k);{}$%
\2\2\6
\&{for} ${}(\|j\K\|n-\T{1};{}$ \|j; ${}\|j\MM){}$\5
${}\{{}$\1\6
\&{for} ${}(\|s\K\T{0};{}$ ${}\|s\Z\|m;{}$ ${}\|s\PP){}$\5
${}\{{}$\1\6
\&{if} ${}(\|s<\|m+\T{1}-\|j){}$\1\5
${}\|s\K\|m+\T{1}-\|j;{}$\2\6
\&{if} ${}(\|s>\|n+\T{1}-\|j){}$\1\5
\&{break};\2\6
\&{for} ${}(\|k\K\T{0};{}$ \|S[\|j][\|k]; ${}\|k\PP){}$\1\5
;\2\6
${}\|r\K\|k;{}$\6
\&{while} (\T{1})\5
${}\{{}$\1\6
\X10:Output $T[j,s,x[0],\ldots,x[r]]$\X;\6
\&{for} ${}(\|k\K\T{0};{}$ \|x[\|k]; ${}\|k\PP){}$\5
${}\{{}$\1\6
${}\|x[\|k]\K\T{0};{}$\6
\4${}\}{}$\2\6
\&{if} ${}(\|k>\|r){}$\1\5
\&{break};\2\6
${}\|x[\|k]\K\T{1};{}$\6
\4${}\}{}$\2\6
\4${}\}{}$\2\6
\4${}\}{}$\2\6
${}\\{printf}(\.{"ans=T[1,\%d,0]+T[1,\%}\)\.{d,1]\\n"},\39\|m,\39\|m){}$;\par
\U5.\fi

\M{10}\B\X10:Output $T[j,s,x[0],\ldots,x[r]]$\X${}\E{}$\6
$\\{printf}(\.{"T[\%d,\%d"},\39\|j,\39\|s);{}$\6
\&{for} ${}(\|k\K\T{0};{}$ ${}\|k\Z\|r;{}$ ${}\|k\PP){}$\1\5
${}\\{printf}(\.{",\%d"},\39\|x[\|k]);{}$\2\6
\\{printf}(\.{"]="});\6
\&{if} ${}(\|s-\|x[\T{0}]<\T{0}\V\|s-\|x[\T{0}]>\|n-\|j){}$\1\5
\\{printf}(\.{"0"});\2\6
\&{else}\1\6
\&{switch} (\\{kids}[\|j])\5
${}\{{}$\1\6
\4\&{case} \T{0}:\5
\X11:Output the no-kid case\X;\5
\&{break};\6
\4\&{case} \T{1}:\5
\X12:Output the one-kid case\X;\5
\&{break};\6
\4\&{case} \T{2}:\5
\X13:Output the two-kid case\X;\5
\&{break};\6
\4${}\}{}$\2\2\6
\\{printf}(\.{"\\n"});\par
\U9.\fi

\M{11}\B\X11:Output the no-kid case\X${}\E{}$\6
$\\{printf}(\.{"T[\%d,\%d"},\39\|j+\T{1},\39\|s-\|x[\T{0}]);{}$\6
\&{for} ${}(\|k\K\T{1};{}$ ${}\|k\Z\|r;{}$ ${}\|k\PP){}$\1\5
${}\\{printf}(\.{",\%d"},\39\|x[\|k]);{}$\2\6
\\{printf}(\.{"]"});\par
\U10.\fi

\M{12}\B\X12:Output the one-kid case\X${}\E{}$\6
\&{for} ${}(\|p\K\T{0};{}$ ${}\|p<\T{2};{}$ ${}\|p\PP){}$\5
${}\{{}$\1\6
\&{if} (\|p)\1\5
\\{printf}(\.{"+"});\2\6
${}\\{printf}(\.{"w[\%d,\%d,\%d]"},\39\|j+\T{1},\39\|x[\T{0}],\39\|p);{}$\6
${}\\{printf}(\.{"T[\%d,\%d,\%d"},\39\|j+\T{1},\39\|s-\|x[\T{0}],\39\|p);{}$\6
\&{for} ${}(\|k\K\T{1};{}$ ${}\|k\Z\|r;{}$ ${}\|k\PP){}$\1\5
${}\\{printf}(\.{",\%d"},\39\|x[\|k]);{}$\2\6
\\{printf}(\.{"]"});\6
\4${}\}{}$\2\par
\U10.\fi

\M{13}\B\X13:Output the two-kid case\X${}\E{}$\6
\&{for} ${}(\|p\K\T{0};{}$ ${}\|p<\T{2};{}$ ${}\|p\PP){}$\5
${}\{{}$\1\6
\&{if} (\|p)\1\5
\\{printf}(\.{"+"});\2\6
${}\\{printf}(\.{"w[\%d,\%d,\%d]("},\39\|j+\T{1},\39\|x[\T{0}],\39\|p);{}$\6
\&{for} ${}(\|q\K\T{0};{}$ ${}\|q<\T{2};{}$ ${}\|q\PP){}$\5
${}\{{}$\1\6
\&{if} (\|q)\1\5
\\{printf}(\.{"+"});\2\6
${}\\{printf}(\.{"w[\%d,\%d,\%d]"},\39\\{edgek}[\\{where}[\|j]],\39\|x[\T{0}],%
\39\|q);{}$\6
${}\\{printf}(\.{"T[\%d,\%d,\%d,\%d"},\39\|j+\T{1},\39\|s-\|x[\T{0}],\39\|p,\39%
\|q);{}$\6
\&{for} ${}(\|k\K\T{1};{}$ ${}\|k\Z\|r;{}$ ${}\|k\PP){}$\1\5
${}\\{printf}(\.{",\%d"},\39\|x[\|k]);{}$\2\6
\\{printf}(\.{"]"});\6
\4${}\}{}$\2\6
\\{printf}(\.{")"});\6
\4${}\}{}$\2\par
\U10.\fi

\N{1}{14}Index.
\fi

\inx
\fin
\con
