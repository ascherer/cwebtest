\input cwebmac
\datethis

\N{1}{1}Intro. This program is part of a series of ``SAT-solvers'' that I'm
putting
together for my own education as I prepare to write Section 7.2.2.2 of
{\sl The Art of Computer Programming}. My intent is to have a variety of
compatible programs on which I can run experiments to learn how different
approaches work in practice.

This time I'm implementing {\mc WALKSAT}, a notable development of the
{\mc WALK} algorithm that was featured in {\mc SAT7}. Instead of
using completely random choices when a variable is flipped, {\mc WALKSAT}
makes a more informed decision. The {\mc WALKSAT} method was introduced
by B.~Selman, H.~A. Kautz, and B.~Cohen in {\sl National Conference
on Artificial Intelligence\/ \bf12} (1994), 337--343.

\fi

\M{2}If you have already read {\mc SAT7}, or any other program of this
series, you might as well skip now past the rest of this introduction,
and past the code for the
``I/O wrapper'' that is presented in the next dozen or so
sections, because you've seen it before. (Except that there are some
new command-line options.)

The input appears on \PB{\\{stdin}} as a series of lines, with one clause per
line.
Each clause is a sequence of literals separated by spaces. Each literal is
a sequence of one to eight ASCII characters between \.{!} and \.{\}},
inclusive, not beginning with \.{\~},
optionally preceded by \.{\~} (which makes the literal ``negative'').
For example, Rivest's famous clauses on four variables,
found in 6.5--(13) and 7.1.1--(32) of {\sl TAOCP}, can be represented by the
following eight lines of input:
$$\chardef~=`\~
\vcenter{\halign{\tt#\cr
x2 x3 ~x4\cr
x1 x3 x4\cr
~x1 x2 x4\cr
~x1 ~x2 x3\cr
~x2 ~x3 x4\cr
~x1 ~x3 ~x4\cr
x1 ~x2 ~x4\cr
x1 x2 ~x3\cr}}$$
Input lines that begin with \.{\~\ } are ignored (treated as comments).
The output will be `\.{\~?}' if the algorithm could not
find a way to satisfy the input clauses.
Otherwise it will be a list of noncontradictory literals that cover each
clause, separated by spaces. (``Noncontradictory'' means that we don't
have both a literal and its negation.) The input above would, for example,
yield `\.{\~?}'; but if the final clause were omitted, the output would
be `\.{\~x1} \.{\~x2} \.{x3}', together
with either \.{x4} or \.{\~x4} (but not both). No attempt is made to
find all solutions; at most one solution is given.

The running time in ``mems'' is also reported, together with the approximate
number of bytes needed for data storage. One ``mem'' essentially means a
memory access to a 64-bit word.
(These totals don't include the time or space needed to parse the
input or to format the output.)

\fi

\M{3}So here's the structure of the program. (Skip ahead if you are
impatient to see the interesting stuff.)

\Y\B\4\D$\|o$ \5
$\\{mems}\PP{}$\C{ count one mem }\par
\B\4\D$\\{oo}$ \5
$\\{mems}\MRL{+{\K}}{}$\T{2}\C{ count two mems }\par
\B\4\D$\\{ooo}$ \5
$\\{mems}\MRL{+{\K}}{}$\T{3}\C{ count three mems }\par
\Y\B\8\#\&{include} \.{<stdio.h>}\6
\8\#\&{include} \.{<stdlib.h>}\6
\8\#\&{include} \.{<string.h>}\6
\8\#\&{include} \.{"gb\_flip.h"}\6
\&{typedef} \&{unsigned} \&{int} \&{uint};\C{ a convenient abbreviation }\6
\&{typedef} \&{unsigned} \&{long} \&{long} \&{ullng};\C{ ditto }\7
\X6:Type definitions\X;\6
\X4:Global variables\X;\6
\X26:Subroutines\X;\7
\\{main}(\&{int} \\{argc}${},\39{}$\&{char} ${}{*}\\{argv}[\,]){}$\1\1\2\2\6
${}\{{}$\1\6
\&{register} \&{uint} \|c${},{}$ \|g${},{}$ \|h${},{}$ \|i${},{}$ \|j${},{}$ %
\|k${},{}$ \|l${},{}$ \|p${},{}$ \|q${},{}$ \|r${},{}$ \\{ii}${},{}$ %
\\{kk}${},{}$ \\{ll}${},{}$ \\{fcount};\7
\X5:Process the command line\X;\6
\X9:Initialize everything\X;\6
\X10:Input the clauses\X;\6
\&{if} ${}(\\{verbose}\AND\\{show\_basics}){}$\1\5
\X22:Report the successful completion of the input phase\X;\2\6
\X29:Set up the main data structures\X;\6
${}\\{imems}\K\\{mems},\39\\{mems}\K\T{0};{}$\6
\X37:Solve the problem\X;\6
\&{if} ${}(\\{verbose}\AND\\{show\_basics}){}$\1\5
${}\\{fprintf}(\\{stderr},\39\.{"Altogether\ \%llu+\%ll}\)\.{u\ mems,\ \%llu\
bytes,\ }\)\.{\%d\ trial\%s,\ \%llu\ ste}\)\.{ps.\\n"},\39\\{imems},\39%
\\{mems},\39\\{bytes},\39\\{trial}+\T{1},\39\\{trial}\?\.{"s"}:\.{""},\39%
\\{step});{}$\2\6
\4${}\}{}$\2\par
\fi

\M{4}\B\D$\\{show\_basics}$ \5
\T{1}\C{ \PB{\\{verbose}} code for basic stats }\par
\B\4\D$\\{show\_choices}$ \5
\T{2}\C{ \PB{\\{verbose}} code for backtrack logging }\par
\B\4\D$\\{show\_details}$ \5
\T{4}\C{ \PB{\\{verbose}} code for further commentary }\par
\B\4\D$\\{show\_gory\_details}$ \5
\T{8}\C{ \PB{\\{verbose}} code turned on when debugging }\par
\Y\B\4\X4:Global variables\X${}\E{}$\6
\&{int} \\{random\_seed}${}\K\T{0}{}$;\C{ seed for the random words of \PB{%
\\{gb\_rand}} }\6
\&{int} \\{verbose}${}\K\\{show\_basics}{}$;\C{ level of verbosity }\6
\&{int} \\{hbits}${}\K\T{8}{}$;\C{ logarithm of the number of the hash lists }\6
\&{int} \\{buf\_size}${}\K\T{1024}{}$;\C{ must exceed the length of the longest
input line }\6
\&{ullng} \\{maxsteps};\C{ maximum steps per walk (\PB{$\\{maxthresh}*\|n$} by
default) }\6
\&{unsigned} \&{int} \\{maxthresh}${}\K\T{50};{}$\6
\&{int} \\{maxtrials}${}\K\T{1000000}{}$;\C{ maximum walks to try }\6
\&{double} \\{nongreedprob}${}\K\T{0.4}{}$;\C{ the probability bias for
nongreedy choices }\6
\&{unsigned} \&{long} \\{nongreedthresh};\C{ coerced since \PB{\\{gb\_next%
\_rand}} is \PB{\&{long}} }\6
\&{ullng} \\{imems}${},{}$ \\{mems};\C{ mem counts }\6
\&{ullng} \\{thresh}${}\K\T{0}{}$;\C{ report when \PB{\\{mems}} exceeds this,
if \PB{$\\{delta}\I\T{0}$} }\6
\&{ullng} \\{delta}${}\K\T{0}{}$;\C{ report every \PB{\\{delta}} or so mems }\6
\&{ullng} \\{timeout}${}\K\T{\^1fffffffffffffff}{}$;\C{ give up after this many
mems }\6
\&{ullng} \\{bytes};\C{ memory used by main data structures }\par
\As8\ET25.
\U3.\fi

\M{5}On the command line one can specify any or all of the following options:
\smallskip
\item{$\bullet$}
`\.v$\langle\,$integer$\,\rangle$' to enable various levels of verbose
output on \PB{\\{stderr}}.
\item{$\bullet$}
`\.h$\langle\,$positive integer$\,\rangle$' to adjust the hash table size.
\item{$\bullet$}
`\.b$\langle\,$positive integer$\,\rangle$' to adjust the size of the input
buffer.
\item{$\bullet$}
`\.s$\langle\,$integer$\,\rangle$' to define the seed for any random numbers
that are used.
\item{$\bullet$}
`\.d$\langle\,$integer$\,\rangle$' to set \PB{\\{delta}} for periodic state
reports.
\item{$\bullet$}
`\.t$\langle\,$integer$\,\rangle$' to define the maximum number of steps per
random walk.
\item{$\bullet$}
`\.c$\langle\,$integer$\,\rangle$' to define the maximum number of steps per
variable, per random walk, if the \.t parameter hasn't been given.
(The default is 50.)
\item{$\bullet$}
`\.w$\langle\,$integer$\,\rangle$' to define the maximum number of walks
attempted.
\item{$\bullet$}
`\.p$\langle\,$float$\,\rangle$' to define the probability \PB{%
\\{nongreedprob}} of
nongreedy choices.
\item{$\bullet$}
`\.T$\langle\,$integer$\,\rangle$' to set \PB{\\{timeout}}: This program will
abruptly terminate, when it discovers that \PB{$\\{mems}>\\{timeout}$}.

\Y\B\4\X5:Process the command line\X${}\E{}$\6
\&{for} ${}(\|j\K\\{argc}-\T{1},\39\|k\K\T{0};{}$ \|j; ${}\|j\MM){}$\1\6
\&{switch} (\\{argv}[\|j][\T{0}])\5
${}\{{}$\1\6
\4\&{case} \.{'v'}:\5
${}\|k\MRL{{\OR}{\K}}(\\{sscanf}(\\{argv}[\|j]+\T{1},\39\.{"\%d"},\39{\AND}%
\\{verbose})-\T{1}){}$;\5
\&{break};\6
\4\&{case} \.{'h'}:\5
${}\|k\MRL{{\OR}{\K}}(\\{sscanf}(\\{argv}[\|j]+\T{1},\39\.{"\%d"},\39{\AND}%
\\{hbits})-\T{1}){}$;\5
\&{break};\6
\4\&{case} \.{'b'}:\5
${}\|k\MRL{{\OR}{\K}}(\\{sscanf}(\\{argv}[\|j]+\T{1},\39\.{"\%d"},\39{\AND}%
\\{buf\_size})-\T{1}){}$;\5
\&{break};\6
\4\&{case} \.{'s'}:\5
${}\|k\MRL{{\OR}{\K}}(\\{sscanf}(\\{argv}[\|j]+\T{1},\39\.{"\%d"},\39{\AND}%
\\{random\_seed})-\T{1}){}$;\5
\&{break};\6
\4\&{case} \.{'d'}:\5
${}\|k\MRL{{\OR}{\K}}(\\{sscanf}(\\{argv}[\|j]+\T{1},\39\.{"\%lld"},\39{\AND}%
\\{delta})-\T{1}){}$;\5
${}\\{thresh}\K\\{delta}{}$;\5
\&{break};\6
\4\&{case} \.{'t'}:\5
${}\|k\MRL{{\OR}{\K}}(\\{sscanf}(\\{argv}[\|j]+\T{1},\39\.{"\%llu"},\39{\AND}%
\\{maxsteps})-\T{1}){}$;\5
\&{break};\6
\4\&{case} \.{'c'}:\5
${}\|k\MRL{{\OR}{\K}}(\\{sscanf}(\\{argv}[\|j]+\T{1},\39\.{"\%u"},\39{\AND}%
\\{maxthresh})-\T{1}){}$;\5
\&{break};\6
\4\&{case} \.{'w'}:\5
${}\|k\MRL{{\OR}{\K}}(\\{sscanf}(\\{argv}[\|j]+\T{1},\39\.{"\%d"},\39{\AND}%
\\{maxtrials})-\T{1}){}$;\5
\&{break};\6
\4\&{case} \.{'p'}:\5
${}\|k\MRL{{\OR}{\K}}(\\{sscanf}(\\{argv}[\|j]+\T{1},\39\.{"\%lf"},\39{\AND}%
\\{nongreedprob})-\T{1}){}$;\5
\&{break};\6
\4\&{case} \.{'T'}:\5
${}\|k\MRL{{\OR}{\K}}(\\{sscanf}(\\{argv}[\|j]+\T{1},\39\.{"\%lld"},\39{\AND}%
\\{timeout})-\T{1}){}$;\5
\&{break};\6
\4\&{default}:\5
${}\|k\K\T{1}{}$;\C{ unrecognized command-line option }\6
\4${}\}{}$\2\2\6
\&{if} ${}(\|k\V\\{hbits}<\T{0}\V\\{hbits}>\T{30}\V\\{buf\_size}\Z\T{0}){}$\5
${}\{{}$\1\6
${}\\{fprintf}(\\{stderr},\39\.{"Usage:\ \%s\ [v<n>]\ [h}\)\.{<n>]\ [b<n>]\
[s<n>]\ [}\)\.{d<n>]"},\39\\{argv}[\T{0}]);{}$\6
${}\\{fprintf}(\\{stderr},\39\.{"\ [t<n>]\ [c<n>]\ [w<n}\)\.{>]\ [p<f>]\ [T<n>]%
\ <\ f}\)\.{oo.sat\\n"});{}$\6
${}\\{exit}({-}\T{1});{}$\6
\4${}\}{}$\2\6
\&{if} ${}(\\{nongreedprob}<\T{0.0}\V\\{nongreedprob}>\T{1.0}){}$\5
${}\{{}$\1\6
${}\\{fprintf}(\\{stderr},\39\.{"Parameter\ p\ should\ }\)\.{be\ between\ 0.0\
and\ 1}\)\.{.0!\\n"});{}$\6
${}\\{exit}({-}\T{666});{}$\6
\4${}\}{}$\2\par
\U3.\fi

\N{1}{6}The I/O wrapper. The following routines read the input and absorb it
into
temporary data areas from which all of the ``real'' data structures
can readily be initialized. My intent is to incorporate these routines in all
of the SAT-solvers in this series. Therefore I've tried to make the code
short and simple, yet versatile enough so that almost no restrictions are
placed on the sizes of problems that can be handled. These routines are
supposed to work properly unless there are more than
$2^{32}-1=4$,294,967,295 occurrences of literals in clauses,
or more than $2^{31}-1=2$,147,483,647 variables or clauses.

In these temporary tables, each variable is represented by four things:
its unique name; its serial number; the clause number (if any) in which it has
most recently appeared; and a pointer to the previous variable (if any)
with the same hash address. Several variables at a time
are represented sequentially in small chunks of memory called ``vchunks,''
which are allocated as needed (and freed later).

\Y\B\4\D$\\{vars\_per\_vchunk}$ \5
\T{341}\C{ preferably $(2^k-1)/3$ for some $k$ }\par
\Y\B\4\X6:Type definitions\X${}\E{}$\6
\&{typedef} \&{union} ${}\{{}$\1\6
\&{char} \\{ch8}[\T{8}];\6
\&{uint} \\{u2}[\T{2}];\6
\&{long} \&{long} \\{lng};\2\6
${}\}{}$ \&{octa};\6
\&{typedef} \&{struct} \&{tmp\_var\_struct} ${}\{{}$\1\6
\&{octa} \\{name};\C{ the name (one to eight ASCII characters) }\6
\&{uint} \\{serial};\C{ 0 for the first variable, 1 for the second, etc. }\6
\&{int} \\{stamp};\C{ \PB{\|m} if positively in clause \PB{\|m}; \PB{${-}\|m$}
if negatively there }\6
\&{struct} \&{tmp\_var\_struct} ${}{*}\\{next}{}$;\C{ pointer for hash list }\2%
\6
${}\}{}$ \&{tmp\_var};\7
\&{typedef} \&{struct} \&{vchunk\_struct} ${}\{{}$\1\6
\&{struct} \&{vchunk\_struct} ${}{*}\\{prev}{}$;\C{ previous chunk allocated
(if any) }\6
\&{tmp\_var} \\{var}[\\{vars\_per\_vchunk}];\2\6
${}\}{}$ \&{vchunk};\par
\As7\ET24.
\U3.\fi

\M{7}Each clause in the temporary tables is represented by a sequence of
one or more pointers to the \PB{\&{tmp\_var}} nodes of the literals involved.
A negated literal is indicated by adding~1 to such a pointer.
The first literal of a clause is indicated by adding~2.
Several of these pointers are represented sequentially in chunks
of memory, which are allocated as needed and freed later.

\Y\B\4\D$\\{cells\_per\_chunk}$ \5
\T{511}\C{ preferably $2^k-1$ for some $k$ }\par
\Y\B\4\X6:Type definitions\X${}\mathrel+\E{}$\6
\&{typedef} \&{struct} \&{chunk\_struct} ${}\{{}$\1\6
\&{struct} \&{chunk\_struct} ${}{*}\\{prev}{}$;\C{ previous chunk allocated (if
any) }\6
\&{tmp\_var} ${}{*}\\{cell}[\\{cells\_per\_chunk}];{}$\2\6
${}\}{}$ \&{chunk};\par
\fi

\M{8}\B\X4:Global variables\X${}\mathrel+\E{}$\6
\&{char} ${}{*}\\{buf}{}$;\C{ buffer for reading the lines (clauses) of \PB{%
\\{stdin}} }\6
\&{tmp\_var} ${}{*}{*}\\{hash}{}$;\C{ heads of the hash lists }\6
\&{uint} \\{hash\_bits}[\T{93}][\T{8}];\C{ random bits for universal hash
function }\6
\&{vchunk} ${}{*}\\{cur\_vchunk}{}$;\C{ the vchunk currently being filled }\6
\&{tmp\_var} ${}{*}\\{cur\_tmp\_var}{}$;\C{ current place to create new \PB{%
\&{tmp\_var}} entries }\6
\&{tmp\_var} ${}{*}\\{bad\_tmp\_var}{}$;\C{ the \PB{\\{cur\_tmp\_var}} when we
need a new \PB{\&{vchunk}} }\6
\&{chunk} ${}{*}\\{cur\_chunk}{}$;\C{ the chunk currently being filled }\6
\&{tmp\_var} ${}{*}{*}\\{cur\_cell}{}$;\C{ current place to create new elements
of a clause }\6
\&{tmp\_var} ${}{*}{*}\\{bad\_cell}{}$;\C{ the \PB{\\{cur\_cell}} when we need
a new \PB{\&{chunk}} }\6
\&{ullng} \\{vars};\C{ how many distinct variables have we seen? }\6
\&{ullng} \\{clauses};\C{ how many clauses have we seen? }\6
\&{ullng} \\{nullclauses};\C{ how many of them were null? }\6
\&{ullng} \\{cells};\C{ how many occurrences of literals in clauses? }\par
\fi

\M{9}\B\X9:Initialize everything\X${}\E{}$\6
\\{gb\_init\_rand}(\\{random\_seed});\6
${}\\{buf}\K{}$(\&{char} ${}{*}){}$ \\{malloc}${}(\\{buf\_size}*\&{sizeof}(%
\&{char}));{}$\6
\&{if} ${}(\R\\{buf}){}$\5
${}\{{}$\1\6
${}\\{fprintf}(\\{stderr},\39\.{"Couldn't\ allocate\ t}\)\.{he\ input\ buffer\
(buf}\)\.{\_size=\%d)!\\n"},\39\\{buf\_size});{}$\6
${}\\{exit}({-}\T{2});{}$\6
\4${}\}{}$\2\6
${}\\{hash}\K{}$(\&{tmp\_var} ${}{*}{*}){}$ \\{malloc}${}(\&{sizeof}(\&{tmp%
\_var})\LL\\{hbits});{}$\6
\&{if} ${}(\R\\{hash}){}$\5
${}\{{}$\1\6
${}\\{fprintf}(\\{stderr},\39\.{"Couldn't\ allocate\ \%}\)\.{d\ hash\ list\
heads\ (h}\)\.{bits=\%d)!\\n"},\39\T{1}\LL\\{hbits},\39\\{hbits});{}$\6
${}\\{exit}({-}\T{3});{}$\6
\4${}\}{}$\2\6
\&{for} ${}(\|h\K\T{0};{}$ ${}\|h<\T{1}\LL\\{hbits};{}$ ${}\|h\PP){}$\1\5
${}\\{hash}[\|h]\K\NULL{}$;\2\par
\A15.
\U3.\fi

\M{10}The hash address of each variable name has $h$ bits, where $h$ is the
value of the adjustable parameter \PB{\\{hbits}}.
Thus the average number of variables per hash list is $n/2^h$ when there
are $n$ different variables. A warning is printed if this average number
exceeds 10. (For example, if $h$ has its default value, 8, the program will
suggest that you might want to increase $h$ if your input has 2560
different variables or more.)

All the hashing takes place at the very beginning,
and the hash tables are actually recycled before any SAT-solving takes place;
therefore the setting of this parameter is by no means crucial. But I didn't
want to bother with fancy coding that would determine $h$ automatically.

\Y\B\4\X10:Input the clauses\X${}\E{}$\6
\&{while} (\T{1})\5
${}\{{}$\1\6
\&{if} ${}(\R\\{fgets}(\\{buf},\39\\{buf\_size},\39\\{stdin})){}$\1\5
\&{break};\2\6
${}\\{clauses}\PP;{}$\6
\&{if} ${}(\\{buf}[\\{strlen}(\\{buf})-\T{1}]\I\.{'\\n'}){}$\5
${}\{{}$\1\6
${}\\{fprintf}(\\{stderr},\39\.{"The\ clause\ on\ line\ }\)\.{\%lld\ (%
\%.20s...)\ is\ t}\)\.{oo\ long\ for\ me;\\n"},\39\\{clauses},\39\\{buf});{}$\6
${}\\{fprintf}(\\{stderr},\39\.{"\ my\ buf\_size\ is\ onl}\)\.{y\ \%d!\\n"},\39%
\\{buf\_size});{}$\6
${}\\{fprintf}(\\{stderr},\39\.{"Please\ use\ the\ comm}\)\.{and-line\ option\
b<ne}\)\.{wsize>.\\n"});{}$\6
${}\\{exit}({-}\T{4});{}$\6
\4${}\}{}$\2\6
\X11:Input the clause in \PB{\\{buf}}\X;\6
\4${}\}{}$\2\6
\&{if} ${}((\\{vars}\GG\\{hbits})\G\T{10}){}$\5
${}\{{}$\1\6
${}\\{fprintf}(\\{stderr},\39\.{"There\ are\ \%llu\ vari}\)\.{ables\ but\ only\
\%d\ ha}\)\.{sh\ tables;\\n"},\39\\{vars},\39\T{1}\LL\\{hbits});{}$\6
\&{while} ${}((\\{vars}\GG\\{hbits})\G\T{10}){}$\1\5
${}\\{hbits}\PP;{}$\2\6
${}\\{fprintf}(\\{stderr},\39\.{"\ maybe\ you\ should\ u}\)\.{se\ command-line\
opti}\)\.{on\ h\%d?\\n"},\39\\{hbits});{}$\6
\4${}\}{}$\2\6
${}\\{clauses}\MRL{-{\K}}\\{nullclauses};{}$\6
\&{if} ${}(\\{clauses}\E\T{0}){}$\5
${}\{{}$\1\6
${}\\{fprintf}(\\{stderr},\39\.{"No\ clauses\ were\ inp}\)\.{ut!\\n"});{}$\6
${}\\{exit}({-}\T{77});{}$\6
\4${}\}{}$\2\6
\&{if} ${}(\\{vars}\G\T{\^80000000}){}$\5
${}\{{}$\1\6
${}\\{fprintf}(\\{stderr},\39\.{"Whoa,\ the\ input\ had}\)\.{\ \%llu\
variables!\\n"},\39\\{cells});{}$\6
${}\\{exit}({-}\T{664});{}$\6
\4${}\}{}$\2\6
\&{if} ${}(\\{clauses}\G\T{\^80000000}){}$\5
${}\{{}$\1\6
${}\\{fprintf}(\\{stderr},\39\.{"Whoa,\ the\ input\ had}\)\.{\ \%llu\ clauses!%
\\n"},\39\\{cells});{}$\6
${}\\{exit}({-}\T{665});{}$\6
\4${}\}{}$\2\6
\&{if} ${}(\\{cells}\G\T{\^100000000}){}$\5
${}\{{}$\1\6
${}\\{fprintf}(\\{stderr},\39\.{"Whoa,\ the\ input\ had}\)\.{\ \%llu\
occurrences\ of}\)\.{\ literals!\\n"},\39\\{cells});{}$\6
${}\\{exit}({-}\T{666});{}$\6
\4${}\}{}$\2\par
\U3.\fi

\M{11}\B\X11:Input the clause in \PB{\\{buf}}\X${}\E{}$\6
\&{for} ${}(\|j\K\|k\K\T{0};{}$  ; \,)\5
${}\{{}$\1\6
\&{while} ${}(\\{buf}[\|j]\E\.{'\ '}){}$\1\5
${}\|j\PP{}$;\C{ scan to nonblank }\2\6
\&{if} ${}(\\{buf}[\|j]\E\.{'\\n'}){}$\1\5
\&{break};\2\6
\&{if} ${}(\\{buf}[\|j]<\.{'\ '}\V\\{buf}[\|j]>\.{'\~'}){}$\5
${}\{{}$\1\6
${}\\{fprintf}(\\{stderr},\39\.{"Illegal\ character\ (}\)\.{code\ \#\%x)\ in\
the\ cla}\)\.{use\ on\ line\ \%lld!\\n"},\39\\{buf}[\|j],\39\\{clauses});{}$\6
${}\\{exit}({-}\T{5});{}$\6
\4${}\}{}$\2\6
\&{if} ${}(\\{buf}[\|j]\E\.{'\~'}){}$\1\5
${}\|i\K\T{1},\39\|j\PP;{}$\2\6
\&{else}\1\5
${}\|i\K\T{0};{}$\2\6
\X12:Scan and record a variable; negate it if \PB{$\|i\E\T{1}$}\X;\6
\4${}\}{}$\2\6
\&{if} ${}(\|k\E\T{0}){}$\5
${}\{{}$\1\6
${}\\{fprintf}(\\{stderr},\39\.{"(Empty\ line\ \%lld\ is}\)\.{\ being\ ignored)%
\\n"},\39\\{clauses});{}$\6
${}\\{nullclauses}\PP{}$;\C{ strictly speaking it would be unsatisfiable }\6
\4${}\}{}$\2\6
\&{goto} \\{clause\_done};\6
\4\\{empty\_clause}:\5
\X19:Remove all variables of the current clause\X;\6
\4\\{clause\_done}:\5
${}\\{cells}\MRL{+{\K}}\|k{}$;\par
\U10.\fi

\M{12}We need a hack to insert the bit codes 1 and/or 2 into a pointer value.

\Y\B\4\D$\\{hack\_in}(\|q,\|t)$ \5
(\&{tmp\_var} ${}{*})(\|t\OR{}$(\&{ullng}) \|q)\par
\Y\B\4\X12:Scan and record a variable; negate it if \PB{$\|i\E\T{1}$}\X${}\E{}$%
\6
${}\{{}$\1\6
\&{register} \&{tmp\_var} ${}{*}\|p;{}$\7
\&{if} ${}(\\{cur\_tmp\_var}\E\\{bad\_tmp\_var}){}$\1\5
\X13:Install a new \PB{\&{vchunk}}\X;\2\6
\X16:Put the variable name beginning at \PB{\\{buf}[\|j]} in \PB{$\\{cur\_tmp%
\_var}\MG\\{name}$} and compute its hash code \PB{\|h}\X;\6
\X17:Find \PB{$\\{cur\_tmp\_var}\MG\\{name}$} in the hash table at \PB{\|p}\X;\6
\&{if} ${}(\|p\MG\\{stamp}\E\\{clauses}\V\|p\MG\\{stamp}\E{-}\\{clauses}){}$\1\5
\X18:Handle a duplicate literal\X\2\6
\&{else}\5
${}\{{}$\1\6
${}\|p\MG\\{stamp}\K(\|i\?{-}\\{clauses}:\\{clauses});{}$\6
\&{if} ${}(\\{cur\_cell}\E\\{bad\_cell}){}$\1\5
\X14:Install a new \PB{\&{chunk}}\X;\2\6
${}{*}\\{cur\_cell}\K\|p;{}$\6
\&{if} ${}(\|i\E\T{1}){}$\1\5
${}{*}\\{cur\_cell}\K\\{hack\_in}({*}\\{cur\_cell},\39\T{1});{}$\2\6
\&{if} ${}(\|k\E\T{0}){}$\1\5
${}{*}\\{cur\_cell}\K\\{hack\_in}({*}\\{cur\_cell},\39\T{2});{}$\2\6
${}\\{cur\_cell}\PP,\39\|k\PP;{}$\6
\4${}\}{}$\2\6
\4${}\}{}$\2\par
\U11.\fi

\M{13}\B\X13:Install a new \PB{\&{vchunk}}\X${}\E{}$\6
${}\{{}$\1\6
\&{register} \&{vchunk} ${}{*}\\{new\_vchunk};{}$\7
${}\\{new\_vchunk}\K{}$(\&{vchunk} ${}{*}){}$ \\{malloc}(\&{sizeof}(%
\&{vchunk}));\6
\&{if} ${}(\R\\{new\_vchunk}){}$\5
${}\{{}$\1\6
${}\\{fprintf}(\\{stderr},\39\.{"Can't\ allocate\ a\ ne}\)\.{w\ vchunk!%
\\n"});{}$\6
${}\\{exit}({-}\T{6});{}$\6
\4${}\}{}$\2\6
${}\\{new\_vchunk}\MG\\{prev}\K\\{cur\_vchunk},\39\\{cur\_vchunk}\K\\{new%
\_vchunk};{}$\6
${}\\{cur\_tmp\_var}\K{\AND}\\{new\_vchunk}\MG\\{var}[\T{0}];{}$\6
${}\\{bad\_tmp\_var}\K{\AND}\\{new\_vchunk}\MG\\{var}[\\{vars\_per%
\_vchunk}];{}$\6
\4${}\}{}$\2\par
\U12.\fi

\M{14}\B\X14:Install a new \PB{\&{chunk}}\X${}\E{}$\6
${}\{{}$\1\6
\&{register} \&{chunk} ${}{*}\\{new\_chunk};{}$\7
${}\\{new\_chunk}\K{}$(\&{chunk} ${}{*}){}$ \\{malloc}(\&{sizeof}(\&{chunk}));\6
\&{if} ${}(\R\\{new\_chunk}){}$\5
${}\{{}$\1\6
${}\\{fprintf}(\\{stderr},\39\.{"Can't\ allocate\ a\ ne}\)\.{w\ chunk!%
\\n"});{}$\6
${}\\{exit}({-}\T{7});{}$\6
\4${}\}{}$\2\6
${}\\{new\_chunk}\MG\\{prev}\K\\{cur\_chunk},\39\\{cur\_chunk}\K\\{new%
\_chunk};{}$\6
${}\\{cur\_cell}\K{\AND}\\{new\_chunk}\MG\\{cell}[\T{0}];{}$\6
${}\\{bad\_cell}\K{\AND}\\{new\_chunk}\MG\\{cell}[\\{cells\_per\_chunk}];{}$\6
\4${}\}{}$\2\par
\U12.\fi

\M{15}The hash code is computed via ``universal hashing,'' using the following
precomputed tables of random bits.

\Y\B\4\X9:Initialize everything\X${}\mathrel+\E{}$\6
\&{for} ${}(\|j\K\T{92};{}$ \|j; ${}\|j\MM){}$\1\6
\&{for} ${}(\|k\K\T{0};{}$ ${}\|k<\T{8};{}$ ${}\|k\PP){}$\1\5
${}\\{hash\_bits}[\|j][\|k]\K\\{gb\_next\_rand}(\,){}$;\2\2\par
\fi

\M{16}\B\X16:Put the variable name beginning at \PB{\\{buf}[\|j]} in \PB{$%
\\{cur\_tmp\_var}\MG\\{name}$} and compute its hash code \PB{\|h}\X${}\E{}$\6
$\\{cur\_tmp\_var}\MG\\{name}.\\{lng}\K\T{0};{}$\6
\&{for} ${}(\|h\K\|l\K\T{0};{}$ ${}\\{buf}[\|j+\|l]>\.{'\ '}\W\\{buf}[\|j+\|l]%
\Z\.{'\~'};{}$ ${}\|l\PP){}$\5
${}\{{}$\1\6
\&{if} ${}(\|l>\T{7}){}$\5
${}\{{}$\1\6
${}\\{fprintf}(\\{stderr},\39\.{"Variable\ name\ \%.9s.}\)\.{..\ in\ the\
clause\ on\ }\)\.{line\ \%lld\ is\ too\ lon}\)\.{g!\\n"},\39\\{buf}+\|j,\39%
\\{clauses});{}$\6
${}\\{exit}({-}\T{8});{}$\6
\4${}\}{}$\2\6
${}\|h\MRL{{\XOR}{\K}}\\{hash\_bits}[\\{buf}[\|j+\|l]-\.{'!'}][\|l];{}$\6
${}\\{cur\_tmp\_var}\MG\\{name}.\\{ch8}[\|l]\K\\{buf}[\|j+\|l];{}$\6
\4${}\}{}$\2\6
\&{if} ${}(\|l\E\T{0}){}$\1\5
\&{goto} \\{empty\_clause};\C{ `\.\~' by itself is like `true' }\2\6
${}\|j\MRL{+{\K}}\|l;{}$\6
${}\|h\MRL{\AND{\K}}(\T{1}\LL\\{hbits})-\T{1}{}$;\par
\U12.\fi

\M{17}\B\X17:Find \PB{$\\{cur\_tmp\_var}\MG\\{name}$} in the hash table at \PB{%
\|p}\X${}\E{}$\6
\&{for} ${}(\|p\K\\{hash}[\|h];{}$ \|p; ${}\|p\K\|p\MG\\{next}){}$\1\6
\&{if} ${}(\|p\MG\\{name}.\\{lng}\E\\{cur\_tmp\_var}\MG\\{name}.\\{lng}){}$\1\5
\&{break};\2\2\6
\&{if} ${}(\R\|p){}$\5
${}\{{}$\C{ new variable found }\1\6
${}\|p\K\\{cur\_tmp\_var}\PP;{}$\6
${}\|p\MG\\{next}\K\\{hash}[\|h],\39\\{hash}[\|h]\K\|p;{}$\6
${}\|p\MG\\{serial}\K\\{vars}\PP;{}$\6
${}\|p\MG\\{stamp}\K\T{0};{}$\6
\4${}\}{}$\2\par
\U12.\fi

\M{18}The most interesting aspect of the input phase is probably the
``unwinding''
that we might need to do when encountering a literal more than once
in the same clause.

\Y\B\4\X18:Handle a duplicate literal\X${}\E{}$\6
${}\{{}$\1\6
\&{if} ${}((\|p\MG\\{stamp}>\T{0})\E(\|i>\T{0})){}$\1\5
\&{goto} \\{empty\_clause};\2\6
\4${}\}{}$\2\par
\U12.\fi

\M{19}An input line that begins with `\.{\~\ }' is silently treated as a
comment.
Otherwise redundant clauses are logged, in case they were unintentional.
(One can, however, intentionally
use redundant clauses to force the order of the variables.)

\Y\B\4\X19:Remove all variables of the current clause\X${}\E{}$\6
\&{while} (\|k)\5
${}\{{}$\1\6
\X20:Move \PB{\\{cur\_cell}} backward to the previous cell\X;\6
${}\|k\MM;{}$\6
\4${}\}{}$\2\6
\&{if} ${}((\\{buf}[\T{0}]\I\.{'\~'})\V(\\{buf}[\T{1}]\I\.{'\ '})){}$\1\5
${}\\{fprintf}(\\{stderr},\39\.{"(The\ clause\ on\ line}\)\.{\ \%lld\ is\
always\ sati}\)\.{sfied)\\n"},\39\\{clauses});{}$\2\6
${}\\{nullclauses}\PP{}$;\par
\U11.\fi

\M{20}\B\X20:Move \PB{\\{cur\_cell}} backward to the previous cell\X${}\E{}$\6
\&{if} ${}(\\{cur\_cell}>{\AND}\\{cur\_chunk}\MG\\{cell}[\T{0}]){}$\1\5
${}\\{cur\_cell}\MM;{}$\2\6
\&{else}\5
${}\{{}$\1\6
\&{register} \&{chunk} ${}{*}\\{old\_chunk}\K\\{cur\_chunk};{}$\7
${}\\{cur\_chunk}\K\\{old\_chunk}\MG\\{prev}{}$;\5
\\{free}(\\{old\_chunk});\6
${}\\{bad\_cell}\K{\AND}\\{cur\_chunk}\MG\\{cell}[\\{cells\_per\_chunk}];{}$\6
${}\\{cur\_cell}\K\\{bad\_cell}-\T{1};{}$\6
\4${}\}{}$\2\par
\Us19\ET33.\fi

\M{21}Notice that the old ``temporary variable'' data goes away here.
(A bug bit me in the first version of the code because of this.)

\Y\B\4\X21:Move \PB{\\{cur\_tmp\_var}} backward to the previous temporary
variable\X${}\E{}$\6
\&{if} ${}(\\{cur\_tmp\_var}>{\AND}\\{cur\_vchunk}\MG\\{var}[\T{0}]){}$\1\5
${}\\{cur\_tmp\_var}\MM;{}$\2\6
\&{else}\5
${}\{{}$\1\6
\&{register} \&{vchunk} ${}{*}\\{old\_vchunk}\K\\{cur\_vchunk};{}$\7
${}\\{cur\_vchunk}\K\\{old\_vchunk}\MG\\{prev}{}$;\5
\\{free}(\\{old\_vchunk});\6
${}\\{bad\_tmp\_var}\K{\AND}\\{cur\_vchunk}\MG\\{var}[\\{vars\_per%
\_vchunk}];{}$\6
${}\\{cur\_tmp\_var}\K\\{bad\_tmp\_var}-\T{1};{}$\6
\4${}\}{}$\2\par
\U35.\fi

\M{22}\B\X22:Report the successful completion of the input phase\X${}\E{}$\6
$\\{fprintf}(\\{stderr},\39\.{"(\%llu\ variables,\ \%l}\)\.{lu\ clauses,\ \%llu%
\ lit}\)\.{erals\ successfully\ r}\)\.{ead)\\n"},\39\\{vars},\39\\{clauses},\39%
\\{cells}){}$;\par
\U3.\fi

\N{1}{23}SAT solving, version 8. The {\mc WALKSAT} algorithm is only a little
bit
more complicated than the {\mc WALK} method, but the differences mean that
we cannot simulate simultaneous runs with bitwise operations.

Let $x=x_1\ldots x_n$ be a binary vector that represents all $n$ variables,
and let $T$ be a given tolerance (representing the amount of patience that
we have). We start by setting $x$ to a completely random vector;
then we repeat the following steps, at most $T$ times:
{\smallskip\narrower\noindent
Check to see if $x$ satisfies all the clauses. If so, output~$x$; we're done!
If not, select a clause $c$ that isn't true, uniformly at random from
all such clauses; say $c$ is the union of $k$ literals,
$l_1\vee\cdots\vee l_k$. Sort those literals according to their
``break count,'' which is the number of clauses that will become false
when that literal is flipped. Choose a literal to flip by the following
method: If no literal has a break count of zero, and if a biased coin turns
up heads, choose $l_j$ at random from among all $k$ literals.
Otherwise, choose $l_j$ at random from among those with smallest break count.
Then change the bit of~$x$ that will make $l_j$ true.
\par}
\smallskip\noindent If that random walk doesn't succeed, we can
try again with another starting value of~$x$, until we've seen
enough failures to be convinced that we're probably doomed to defeat.

\fi

\M{24}The data structures are somewhat interesting, but not tricky: There are
four main arrays, \PB{\\{cmem}}, \PB{\\{vmem}}, \PB{\\{mem}}, and \PB{%
\\{tmem}}. Structured
\PB{\&{clause}} nodes appear in \PB{\\{cmem}}, and structured \PB{\&{variable}}
nodes
appear in \PB{\\{vmem}}. Each clause points to a sequential list of literals
in~\PB{\\{mem}}; each literal points to a sequential list of clauses in~\PB{%
\\{tmem}},
which is essentially the ``transpose'' of the information in~\PB{\\{mem}}.
If \PB{\\{fcount}} clauses are currently false, the first \PB{\\{fcount}}
entries
of~\PB{\\{cmem}} also contain the indices of those clauses.

As in most previous programs of this series, the literals $x$ and $\bar x$
are represented internally by $2k$ and $2k+1$ when $x$ is the $k$th variable.

The symbolic names of variables are kept separately in \PB{\\{nmem}}, not in %
\PB{\\{vmem}},
for reasons of efficiency. (Otherwise a \PB{\&{variable}} struct would
take up five octabytes, and addressing would be slower.)

\Y\B\4\D$\\{value}(\|l)$ \5
$(\\{vmem}[(\|l)\GG\T{1}].\\{val}\XOR((\|l)\AND\T{1}){}$)\par
\Y\B\4\X6:Type definitions\X${}\mathrel+\E{}$\6
\&{typedef} \&{struct} ${}\{{}$\1\6
\&{uint} \\{val};\C{ the variable's current value }\6
\&{uint} \\{breakcount};\C{ how many clauses are false except for this variable
}\6
\&{uint} \\{pos\_start}${},{}$ \\{neg\_start};\C{ where the clause lists start
in \PB{\\{tmem}} }\2\6
${}\}{}$ \&{variable};\6
\&{typedef} \&{struct} ${}\{{}$\1\6
\&{uint} \\{start};\C{ where the literal list starts in \PB{\\{mem}} }\6
\&{uint} \\{tcount};\C{ how many of those literals are currently true? }\6
\&{uint} \\{fplace};\C{ if \PB{$\\{tcount}\K\T{0}$}, which \PB{\\{fslot}} holds
this clause? }\6
\&{uint} \\{fslot};\C{ the number of a false clause, if needed }\2\6
${}\}{}$ \&{clause};\par
\fi

\M{25}\B\X4:Global variables\X${}\mathrel+\E{}$\6
\&{clause} ${}{*}\\{cmem}{}$;\C{ the master array of clauses }\6
\&{variable} ${}{*}\\{vmem}{}$;\C{ the master array of variables }\6
\&{uint} ${}{*}\\{mem}{}$;\C{ the master array of literals in clauses }\6
\&{uint} ${}{*}\\{cur\_mcell}{}$;\C{ the current cell of interest in \PB{%
\\{mem}} }\6
\&{uint} ${}{*}\\{tmem}{}$;\C{ the master array of clauses containing literals
}\6
\&{octa} ${}{*}\\{nmem}{}$;\C{ the master array of symbolic variable names }\6
\&{int} \\{trial};\C{ which trial are we on? }\6
\&{ullng} \\{step};\C{ which step are we on? }\6
\&{uint} ${}{*}\\{best}{}$;\C{ temporary array to hold literal names for a
clause }\par
\fi

\M{26}Here is a subroutine that prints a clause symbolically. It illustrates
some of the conventions of the data structures that have been explained above.
I use it only for debugging.

\Y\B\4\X26:Subroutines\X${}\E{}$\6
\&{void} \\{print\_clause}(\&{uint} \|c)\1\1\2\2\6
${}\{{}$\C{ the first clause is called clause 1, not 0 }\1\6
\&{register} \&{uint} \|l${},{}$ \\{ll};\7
${}\\{fprintf}(\\{stderr},\39\.{"\%d:"},\39\|c){}$;\C{ show the clause number }%
\6
\&{for} ${}(\|l\K\\{cmem}[\|c-\T{1}].\\{start};{}$ ${}\|l<\\{cmem}[\|c].%
\\{start};{}$ ${}\|l\PP){}$\5
${}\{{}$\1\6
${}\\{ll}\K\\{mem}[\|l];{}$\6
${}\\{fprintf}(\\{stderr},\39\.{"\ \%s\%.8s(\%d)"},\39\\{ll}\AND\T{1}\?\.{"%
\~"}:\.{""},\39\\{nmem}[\\{ll}\GG\T{1}].\\{ch8},\39\\{value}(\\{ll}));{}$\6
\4${}\}{}$\2\6
${}\\{fprintf}(\\{stderr},\39\.{"\\n"});{}$\6
\4${}\}{}$\2\par
\As27\ET28.
\U3.\fi

\M{27}Another version of that routine, used to display unsatisfied clauses
in verbose mode, shows the current breakcounts of each literal.

\Y\B\4\X26:Subroutines\X${}\mathrel+\E{}$\6
\&{void} \\{print\_unsat\_clause}(\&{uint} \|c)\1\1\2\2\6
${}\{{}$\1\6
\&{register} \&{uint} \|l${},{}$ \\{ll};\7
${}\\{fprintf}(\\{stderr},\39\.{"\%d:"},\39\|c){}$;\C{ show the clause number }%
\6
\&{for} ${}(\|l\K\\{cmem}[\|c-\T{1}].\\{start};{}$ ${}\|l<\\{cmem}[\|c].%
\\{start};{}$ ${}\|l\PP){}$\5
${}\{{}$\1\6
${}\\{ll}\K\\{mem}[\|l];{}$\6
${}\\{fprintf}(\\{stderr},\39\.{"\ \%s\%.8s(\%d)"},\39\\{ll}\AND\T{1}\?\.{"%
\~"}:\.{""},\39\\{nmem}[\\{ll}\GG\T{1}].\\{ch8},\39\\{vmem}[\\{ll}\GG\T{1}].%
\\{breakcount});{}$\6
\4${}\}{}$\2\6
${}\\{fprintf}(\\{stderr},\39\.{"\\n"});{}$\6
\4${}\}{}$\2\par
\fi

\M{28}Similarly, we can list the clause numbers that contain a given literal.
(Notice the limits on~\PB{\|c} in the loop here.)

\Y\B\4\X26:Subroutines\X${}\mathrel+\E{}$\6
\&{void} \\{print\_literal\_uses}(\&{uint} \|l)\1\1\2\2\6
${}\{{}$\1\6
\&{register} \&{uint} \\{ll}${},{}$ \|c;\7
${}\\{ll}\K\|l\GG\T{1};{}$\6
${}\\{fprintf}(\\{stderr},\39\.{"\%s\%.8s(\%d)\ is\ in"},\39\|l\AND\T{1}\?\.{"%
\~"}:\.{""},\39\\{nmem}[\\{ll}].\\{ch8},\39\\{value}(\|l));{}$\6
\&{for} ${}(\|c\K(\|l\AND\T{1}\?\\{vmem}[\\{ll}].\\{neg\_start}:\\{vmem}[%
\\{ll}].\\{pos\_start});{}$ ${}\|c<(\|l\AND\T{1}\?\\{vmem}[\\{ll}+\T{1}].\\{pos%
\_start}:\\{vmem}[\\{ll}].\\{neg\_start});{}$ ${}\|c\PP){}$\1\5
${}\\{fprintf}(\\{stderr},\39\.{"\ \%d"},\39\\{tmem}[\|c]);{}$\2\6
${}\\{fprintf}(\\{stderr},\39\.{"\\n"});{}$\6
\4${}\}{}$\2\par
\fi

\N{1}{29}Initializing the real data structures.
We're ready now to convert the temporary chunks of data into the
form we want, and to recycle those chunks.

\Y\B\4\X29:Set up the main data structures\X${}\E{}$\6
\X30:Allocate the main arrays\X;\6
\X31:Initialize the \PB{\\{pos\_start}} and \PB{\\{neg\_start}} fields\X;\6
\X32:Copy all the temporary cells to the \PB{\\{mem}} and \PB{\\{cmem}} arrays
in proper format\X;\6
\X35:Copy all the temporary variable nodes to the \PB{\\{nmem}} array in proper
format\X;\6
\X34:Set up the \PB{\\{tmem}} array\X;\6
\X36:Check consistency\X;\par
\U3.\fi

\M{30}\B\X30:Allocate the main arrays\X${}\E{}$\6
\\{free}(\\{buf});\5
\\{free}(\\{hash});\C{ a tiny gesture to make a little room }\6
${}\\{vmem}\K{}$(\&{variable} ${}{*}){}$ \\{malloc}${}((\\{vars}+\T{1})*%
\&{sizeof}(\&{variable}));{}$\6
\&{if} ${}(\R\\{vmem}){}$\5
${}\{{}$\1\6
${}\\{fprintf}(\\{stderr},\39\.{"Oops,\ I\ can't\ alloc}\)\.{ate\ the\ vmem\
array!\\}\)\.{n"});{}$\6
${}\\{exit}({-}\T{12});{}$\6
\4${}\}{}$\2\6
${}\\{bytes}\K(\\{vars}+\T{1})*\&{sizeof}(\&{variable});{}$\6
${}\\{nmem}\K{}$(\&{octa} ${}{*}){}$ \\{malloc}${}(\\{vars}*\&{sizeof}(%
\&{octa}));{}$\6
\&{if} ${}(\R\\{nmem}){}$\5
${}\{{}$\1\6
${}\\{fprintf}(\\{stderr},\39\.{"Oops,\ I\ can't\ alloc}\)\.{ate\ the\ nmem\
array!\\}\)\.{n"});{}$\6
${}\\{exit}({-}\T{13});{}$\6
\4${}\}{}$\2\6
${}\\{bytes}\MRL{+{\K}}\\{vars}*\&{sizeof}(\&{octa});{}$\6
${}\\{mem}\K{}$(\&{uint} ${}{*}){}$ \\{malloc}${}(\\{cells}*\&{sizeof}(%
\&{uint}));{}$\6
\&{if} ${}(\R\\{mem}){}$\5
${}\{{}$\1\6
${}\\{fprintf}(\\{stderr},\39\.{"Oops,\ I\ can't\ alloc}\)\.{ate\ the\ big\ mem%
\ arra}\)\.{y!\\n"});{}$\6
${}\\{exit}({-}\T{10});{}$\6
\4${}\}{}$\2\6
${}\\{bytes}\MRL{+{\K}}\\{cells}*\&{sizeof}(\&{uint});{}$\6
${}\\{tmem}\K{}$(\&{uint} ${}{*}){}$ \\{malloc}${}(\\{cells}*\&{sizeof}(%
\&{uint}));{}$\6
\&{if} ${}(\R\\{tmem}){}$\5
${}\{{}$\1\6
${}\\{fprintf}(\\{stderr},\39\.{"Oops,\ I\ can't\ alloc}\)\.{ate\ the\ big\
tmem\ arr}\)\.{ay!\\n"});{}$\6
${}\\{exit}({-}\T{14});{}$\6
\4${}\}{}$\2\6
${}\\{bytes}\MRL{+{\K}}\\{cells}*\&{sizeof}(\&{uint});{}$\6
${}\\{cmem}\K{}$(\&{clause} ${}{*}){}$ \\{malloc}${}((\\{clauses}+\T{1})*%
\&{sizeof}(\&{clause}));{}$\6
\&{if} ${}(\R\\{cmem}){}$\5
${}\{{}$\1\6
${}\\{fprintf}(\\{stderr},\39\.{"Oops,\ I\ can't\ alloc}\)\.{ate\ the\ cmem\
array!\\}\)\.{n"});{}$\6
${}\\{exit}({-}\T{11});{}$\6
\4${}\}{}$\2\6
${}\\{bytes}\MRL{+{\K}}(\\{clauses}+\T{1})*\&{sizeof}(\&{clause}){}$;\par
\U29.\fi

\M{31}\B\X31:Initialize the \PB{\\{pos\_start}} and \PB{\\{neg\_start}} fields%
\X${}\E{}$\6
\&{for} ${}(\|c\K\\{vars};{}$ \|c; ${}\|c\MM){}$\1\5
${}\|o,\39\\{vmem}[\|c-\T{1}].\\{pos\_start}\K\\{vmem}[\|c-\T{1}].\\{neg%
\_start}\K\T{0}{}$;\2\par
\U29.\fi

\M{32}\B\X32:Copy all the temporary cells to the \PB{\\{mem}} and \PB{\\{cmem}}
arrays in proper format\X${}\E{}$\6
\&{for} ${}(\|c\K\\{clauses},\39\\{cur\_mcell}\K\\{mem}+\\{cells},\39\\{kk}\K%
\T{0};{}$ \|c; ${}\|c\MM){}$\5
${}\{{}$\1\6
${}\|o,\39\\{cmem}[\|c].\\{start}\K\\{cur\_mcell}-\\{mem},\39\|k\K\T{0};{}$\6
\X33:Insert the cells for the literals of clause \PB{\|c}\X;\6
\&{if} ${}(\|k>\\{kk}){}$\1\5
${}\\{kk}\K\|k{}$;\C{ maximum clause size seen so far }\2\6
\4${}\}{}$\2\6
\&{if} ${}(\\{cur\_mcell}\I\\{mem}){}$\5
${}\{{}$\1\6
${}\\{fprintf}(\\{stderr},\39\.{"Confusion\ about\ the}\)\.{\ number\ of\
cells!\\n"});{}$\6
${}\\{exit}({-}\T{99});{}$\6
\4${}\}{}$\2\6
${}\\{cmem}[\T{0}].\\{start}\K\T{0};{}$\6
${}\\{best}\K{}$(\&{uint} ${}{*}){}$ \\{malloc}${}(\\{kk}*\&{sizeof}(%
\&{uint}));{}$\6
\&{if} ${}(\R\\{best}){}$\5
${}\{{}$\1\6
${}\\{fprintf}(\\{stderr},\39\.{"Oops,\ I\ can't\ alloc}\)\.{ate\ the\ best\
array!\\}\)\.{n"});{}$\6
${}\\{exit}({-}\T{16});{}$\6
\4${}\}{}$\2\6
${}\\{bytes}\MRL{+{\K}}\\{kk}*\&{sizeof}(\&{uint}){}$;\par
\U29.\fi

\M{33}The basic idea is to ``unwind'' the steps that we went through while
building up the chunks.

\Y\B\4\D$\\{hack\_out}(\|q)$ \5
(((\&{ullng}) \|q)${}\AND\T{\^3}{}$)\par
\B\4\D$\\{hack\_clean}(\|q)$ \5
((\&{tmp\_var} ${}{*})({}$(\&{ullng}) \|q${}\AND{-}\T{4}){}$)\par
\Y\B\4\X33:Insert the cells for the literals of clause \PB{\|c}\X${}\E{}$\6
\&{for} ${}(\|i\K\T{0};{}$ ${}\|i<\T{2};{}$ ${}\|k\PP){}$\5
${}\{{}$\1\6
\X20:Move \PB{\\{cur\_cell}} backward to the previous cell\X;\6
${}\|i\K\\{hack\_out}({*}\\{cur\_cell});{}$\6
${}\|p\K\\{hack\_clean}({*}\\{cur\_cell})\MG\\{serial};{}$\6
${}\\{cur\_mcell}\MM;{}$\6
${}\|o,\39{*}\\{cur\_mcell}\K\|l\K\|p+\|p+(\|i\AND\T{1});{}$\6
\&{if} ${}(\|l\AND\T{1}){}$\1\5
${}\\{oo},\39\\{vmem}[\|l\GG\T{1}].\\{neg\_start}\PP;{}$\2\6
\&{else}\1\5
${}\\{oo},\39\\{vmem}[\|l\GG\T{1}].\\{pos\_start}\PP;{}$\2\6
\4${}\}{}$\2\par
\U32.\fi

\M{34}\B\X34:Set up the \PB{\\{tmem}} array\X${}\E{}$\6
\&{for} ${}(\|j\K\|k\K\T{0};{}$ ${}\|k<\\{vars};{}$ ${}\|k\PP){}$\5
${}\{{}$\1\6
${}\|o,\39\|i\K\\{vmem}[\|k].\\{pos\_start},\39\\{ii}\K\\{vmem}[\|k].\\{neg%
\_start};{}$\6
${}\|o,\39\\{vmem}[\|k].\\{pos\_start}\K\|j+\|i,\39\\{vmem}[\|k].\\{neg\_start}%
\K\|j+\|i+\\{ii};{}$\6
${}\|j\K\|j+\|i+\\{ii};{}$\6
\4${}\}{}$\2\6
${}\|o,\39\\{vmem}[\|k].\\{pos\_start}\K\|j{}$;\C{ \PB{$\|j\K\\{cells}$} at
this point }\6
\&{for} ${}(\|c\K\|k\K\T{0},\39\|o,\39\\{kk}\K\\{cmem}[\T{1}].\\{start};{}$ ${}%
\|k<\\{cells};{}$ ${}\|k\PP){}$\5
${}\{{}$\1\6
\&{if} ${}(\|k\E\\{kk}){}$\1\5
${}\|o,\39\|c\PP,\39\\{kk}\K\\{cmem}[\|c+\T{1}].\\{start};{}$\2\6
${}\|l\K\\{mem}[\|k];{}$\6
\&{if} ${}(\|l\AND\T{1}){}$\1\5
${}\\{ooo},\39\|i\K\\{vmem}[\|l\GG\T{1}].\\{neg\_start}-\T{1},\39\\{tmem}[\|i]%
\K\|c,\39\\{vmem}[\|l\GG\T{1}].\\{neg\_start}\K\|i;{}$\2\6
\&{else}\1\5
${}\\{ooo},\39\|i\K\\{vmem}[\|l\GG\T{1}].\\{pos\_start}-\T{1},\39\\{tmem}[\|i]%
\K\|c,\39\\{vmem}[\|l\GG\T{1}].\\{pos\_start}\K\|i;{}$\2\6
\4${}\}{}$\2\par
\U29.\fi

\M{35}\B\X35:Copy all the temporary variable nodes to the \PB{\\{nmem}} array
in proper format\X${}\E{}$\6
\&{for} ${}(\|c\K\\{vars};{}$ \|c; ${}\|c\MM){}$\5
${}\{{}$\1\6
\X21:Move \PB{\\{cur\_tmp\_var}} backward to the previous temporary variable\X;%
\6
${}\|o,\39\\{nmem}[\|c-\T{1}].\\{lng}\K\\{cur\_tmp\_var}\MG\\{name}.\\{lng};{}$%
\6
\4${}\}{}$\2\par
\U29.\fi

\M{36}We should now have unwound all the temporary data chunks back to their
beginnings.

\Y\B\4\X36:Check consistency\X${}\E{}$\6
\&{if} ${}(\\{cur\_cell}\I{\AND}\\{cur\_chunk}\MG\\{cell}[\T{0}]\V\\{cur%
\_chunk}\MG\\{prev}\I\NULL\V\\{cur\_tmp\_var}\I{\AND}\\{cur\_vchunk}\MG\\{var}[%
\T{0}]\V\\{cur\_vchunk}\MG\\{prev}\I\NULL){}$\5
${}\{{}$\1\6
${}\\{fprintf}(\\{stderr},\39\.{"This\ can't\ happen\ (}\)\.{consistency\ check%
\ fa}\)\.{ilure)!\\n"});{}$\6
${}\\{exit}({-}\T{14});{}$\6
\4${}\}{}$\2\6
\\{free}(\\{cur\_chunk});\5
\\{free}(\\{cur\_vchunk});\par
\U29.\fi

\N{1}{37}Doing it. So we take random walks.

\Y\B\4\X37:Solve the problem\X${}\E{}$\6
\&{if} ${}(\\{maxsteps}\E\T{0}){}$\1\5
${}\\{maxsteps}\K\\{maxthresh}*\\{vars};{}$\2\6
${}\\{nongreedthresh}\K\\{nongreedprob}*{}$(\&{unsigned} \&{long}) \T{%
\^80000000};\6
\&{for} ${}(\\{trial}\K\T{0};{}$ ${}\\{trial}<\\{maxtrials};{}$ ${}\\{trial}%
\PP){}$\5
${}\{{}$\1\6
\&{if} ${}(\\{delta}\W(\\{mems}\G\\{thresh})){}$\5
${}\{{}$\1\6
${}\\{thresh}\MRL{+{\K}}\\{delta};{}$\6
${}\\{fprintf}(\\{stderr},\39\.{"\ after\ \%lld\ mems,\ b}\)\.{eginning\ trial\
\%d\\n"},\39\\{mems},\39\\{trial}+\T{1});{}$\6
\4${}\}{}$\5
\2\&{else} \&{if} ${}(\\{verbose}\AND\\{show\_choices}){}$\1\5
${}\\{fprintf}(\\{stderr},\39\.{"beginning\ trial\ \%d\\}\)\.{n"},\39\\{trial}+%
\T{1});{}$\2\6
\X38:Initialize all values\X;\6
\&{if} ${}(\\{verbose}\AND\\{show\_details}){}$\1\5
\X47:Print the initial guess\X;\2\6
\X39:Initialize the clause data structures\X;\6
\&{for} ${}(\\{step}\K\T{0};{}$  ; ${}\\{step}\PP){}$\5
${}\{{}$\1\6
\&{if} ${}(\\{fcount}\E\T{0}){}$\1\5
\X48:Print a solution and \PB{\&{goto} \\{done}}\X;\2\6
\&{if} ${}(\\{mems}>\\{timeout}){}$\5
${}\{{}$\1\6
${}\\{fprintf}(\\{stderr},\39\.{"TIMEOUT!\\n"});{}$\6
\&{goto} \\{done};\6
\4${}\}{}$\2\6
\&{if} ${}(\\{step}\E\\{maxsteps}){}$\1\5
\&{break};\2\6
\X40:Choose a random unsatisfied clause, \PB{\|c}\X;\6
\X41:Choose a literal \PB{\|l} in \PB{\|c}\X;\6
\X42:Flip the value of \PB{\|l}\X;\6
\4${}\}{}$\2\6
\4${}\}{}$\2\6
\\{printf}(\.{"\~?\\n"});\C{ we weren't able to satisfy all the clauses }\6
\&{if} ${}(\\{verbose}\AND\\{show\_basics}){}$\1\5
${}\\{fprintf}(\\{stderr},\39\.{"DUNNO\\n"});{}$\2\6
${}\\{trial}\MM{}$;\C{ restore the actual number of trials made }\6
\\{done}:\par
\U3.\fi

\M{38}The macro \PB{\\{gb\_next\_rand}(\,)} delivers a 31-bit random integer,
and my convention is to charge four mems whenever it is called.

\Y\B\4\X38:Initialize all values\X${}\E{}$\6
\&{for} ${}(\|k\K\T{0},\39\|r\K\T{1};{}$ ${}\|k<\\{vars};{}$ ${}\|k\PP){}$\5
${}\{{}$\1\6
\&{if} ${}(\|r\E\T{1}){}$\1\5
${}\\{mems}\MRL{+{\K}}\T{4},\39\|r\K\\{gb\_next\_rand}(\,)+(\T{1\$U}\LL%
\T{31});{}$\2\6
${}\|o,\39\\{vmem}[\|k].\\{val}\K\|r\AND\T{1},\39\|r\MRL{{\GG}{\K}}\T{1};{}$\6
${}\\{vmem}[\|k].\\{breakcount}\K\T{0};{}$\6
\4${}\}{}$\2\par
\U37.\fi

\M{39}\B\X39:Initialize the clause data structures\X${}\E{}$\6
$\\{fcount}\K\T{0};{}$\6
\&{for} ${}(\|c\K\|k\K\T{0};{}$ ${}\|c<\\{clauses};{}$ ${}\|c\PP){}$\5
${}\{{}$\1\6
${}\|o,\39\\{kk}\K\\{cmem}[\|c+\T{1}].\\{start};{}$\6
${}\|p\K\T{0}{}$;\C{ \PB{\|p} true literals seen so far in clause \PB{\|c} }\6
\&{for} ( ; ${}\|k<\\{kk};{}$ ${}\|k\PP){}$\5
${}\{{}$\1\6
${}\|o,\39\|l\K\\{mem}[\|k];{}$\6
\&{if} ${}(\|o,\39\\{value}(\|l)){}$\1\5
${}\|p\PP,\39\\{ll}\K\|l;{}$\2\6
\4${}\}{}$\2\6
${}\|o,\39\\{cmem}[\|c].\\{tcount}\K\|p;{}$\6
\&{if} ${}(\|p\Z\T{1}){}$\5
${}\{{}$\1\6
\&{if} (\|p)\1\5
${}\\{oo},\39\\{vmem}[\\{ll}\GG\T{1}].\\{breakcount}\PP;{}$\2\6
\&{else}\1\5
${}\\{oo},\39\\{cmem}[\|c].\\{fplace}\K\\{fcount},\39\\{cmem}[\\{fcount}\PP].%
\\{fslot}\K\|c;{}$\2\6
\4${}\}{}$\2\6
\4${}\}{}$\2\par
\U37.\fi

\M{40}\B\X40:Choose a random unsatisfied clause, \PB{\|c}\X${}\E{}$\6
\&{if} ${}(\\{verbose}\AND\\{show\_gory\_details}){}$\5
${}\{{}$\1\6
${}\\{fprintf}(\\{stderr},\39\.{"currently\ false:\\n"});{}$\6
\&{for} ${}(\|k\K\T{0};{}$ ${}\|k<\\{fcount};{}$ ${}\|k\PP){}$\1\5
${}\\{print\_unsat\_clause}(\\{cmem}[\|k].\\{fslot}+\T{1});{}$\2\6
\4${}\}{}$\2\6
${}\\{mems}\MRL{+{\K}}\T{5},\39\|c\K\\{cmem}[\\{gb\_unif\_rand}(\\{fcount})].%
\\{fslot};{}$\6
\&{if} ${}(\\{verbose}\AND\\{show\_choices}){}$\1\5
${}\\{fprintf}(\\{stderr},\39\.{"in\ \%u(\%d)"},\39\|c+\T{1},\39\\{fcount}){}$;%
\2\par
\U37.\fi

\M{41}\B\X41:Choose a literal \PB{\|l} in \PB{\|c}\X${}\E{}$\6
$\\{oo},\39\|k\K\\{cmem}[\|c].\\{start},\39\\{kk}\K\\{cmem}[\|c+\T{1}].%
\\{start},\39\|h\K\\{kk}-\|k;{}$\6
${}\\{ooo},\39\|p\K\\{mem}[\|k],\39\|r\K\\{vmem}[\|p\GG\T{1}].\\{breakcount},%
\39\\{best}[\T{0}]\K\|p,\39\|j\K\T{1};{}$\6
\&{for} ${}(\|k\PP;{}$ ${}\|k<\\{kk};{}$ ${}\|k\PP){}$\5
${}\{{}$\1\6
${}\\{oo},\39\|p\K\\{mem}[\|k],\39\|q\K\\{vmem}[\|p\GG\T{1}].\\{breakcount};{}$%
\6
\&{if} ${}(\|q\Z\|r){}$\5
${}\{{}$\1\6
\&{if} ${}(\|q<\|r){}$\1\5
${}\|o,\39\|r\K\|q,\39\\{best}[\T{0}]\K\|p,\39\|j\K\T{1};{}$\2\6
\&{else}\1\5
${}\|o,\39\\{best}[\|j\PP]\K\|p;{}$\2\6
\4${}\}{}$\2\6
\4${}\}{}$\2\6
\&{if} ${}(\|r\E\T{0}){}$\1\5
\&{goto} \\{greedy};\2\6
\&{if} ${}(\\{mems}\MRL{+{\K}}\T{4},\39(\\{gb\_next\_rand}(\,)<%
\\{nongreedthresh})){}$\5
${}\{{}$\1\6
${}\\{mems}\MRL{+{\K}}\T{5},\39\|l\K\\{mem}[\\{kk}-\T{1}-\\{gb\_unif\_rand}(%
\|h)],\39\|g\K\T{0};{}$\6
\&{goto} \\{got\_l};\6
\4${}\}{}$\2\6
\4\\{greedy}:\5
${}\|g\K\T{1};{}$\6
\&{if} ${}(\|j\E\T{1}){}$\1\5
${}\|l\K\\{best}[\T{0}];{}$\2\6
\&{else}\1\5
${}\\{mems}\MRL{+{\K}}\T{5},\39\|l\K\\{best}[\\{gb\_unif\_rand}(\|j)];{}$\2\6
\4\\{got\_l}:\5
${}\|p\K\|l\GG\T{1};{}$\6
\&{if} ${}(\\{verbose}\AND\\{show\_choices}){}$\5
${}\{{}$\1\6
\&{if} ${}(\\{verbose}\AND\\{show\_details}){}$\1\5
${}\\{fprintf}(\\{stderr},\39\.{",\ \%d*\%d\ of\ \%d\%s,"},\39\|r,\39\|j,\39%
\|h,\39\|g\?\.{""}:\.{"\ nongreedy"});{}$\2\6
${}\\{fprintf}(\\{stderr},\39\.{"\ flip\ \%s\%.8s\ (cost\ }\)\.{\%d)\\n"},\39%
\\{vmem}[\|p].\\{val}\?\.{""}:\.{"\~"},\39\\{nmem}[\|p].\\{ch8},\39\\{vmem}[%
\|p].\\{breakcount});{}$\6
\4${}\}{}$\2\par
\U37.\fi

\M{42}At this point \PB{$\|p\K\|l\GG\T{1}$}.

\Y\B\4\X42:Flip the value of \PB{\|l}\X${}\E{}$\6
\&{if} ${}(\|l\AND\T{1}){}$\5
${}\{{}$\1\6
${}\\{oo},\39\|k\K\\{vmem}[\|p].\\{neg\_start},\39\\{kk}\K\\{vmem}[\|p+\T{1}].%
\\{pos\_start};{}$\6
\X43:Make clauses \PB{\\{tmem}[\|k]}, \PB{$\\{tmem}[\|k+\T{1}]$}, \dots\
happier\X;\6
${}\|o,\39\\{vmem}[\|p].\\{breakcount}\K\|h,\39\\{vmem}[\|p].\\{val}\K\T{0};{}$%
\6
${}\|k\K\\{vmem}[\|p].\\{pos\_start},\39\\{kk}\K\\{vmem}[\|p].\\{neg%
\_start};{}$\6
\X44:Make clauses \PB{\\{tmem}[\|k]}, \PB{$\\{tmem}[\|k+\T{1}]$}, \dots\ sadder%
\X;\6
\4${}\}{}$\5
\2\&{else}\5
${}\{{}$\1\6
${}\|o,\39\|k\K\\{vmem}[\|p].\\{pos\_start},\39\\{kk}\K\\{vmem}[\|p].\\{neg%
\_start};{}$\6
\X43:Make clauses \PB{\\{tmem}[\|k]}, \PB{$\\{tmem}[\|k+\T{1}]$}, \dots\
happier\X;\6
${}\|o,\39\\{vmem}[\|p].\\{breakcount}\K\|h,\39\\{vmem}[\|p].\\{val}\K\T{1};{}$%
\6
${}\|o,\39\|k\K\\{kk},\39\\{kk}\K\\{vmem}[\|p+\T{1}].\\{pos\_start};{}$\6
\X44:Make clauses \PB{\\{tmem}[\|k]}, \PB{$\\{tmem}[\|k+\T{1}]$}, \dots\ sadder%
\X;\6
\4${}\}{}$\2\par
\U37.\fi

\M{43}\B\X43:Make clauses \PB{\\{tmem}[\|k]}, \PB{$\\{tmem}[\|k+\T{1}]$}, \dots%
\ happier\X${}\E{}$\6
\&{for} ${}(\|h\K\T{0};{}$ ${}\|k<\\{kk};{}$ ${}\|k\PP){}$\5
${}\{{}$\1\6
${}\\{ooo},\39\|c\K\\{tmem}[\|k],\39\|j\K\\{cmem}[\|c].\\{tcount},\39\\{cmem}[%
\|c].\\{tcount}\K\|j+\T{1};{}$\6
\&{if} ${}(\|j\Z\T{1}){}$\5
${}\{{}$\1\6
\&{if} (\|j)\1\5
\X45:Decrease the breakcount of \PB{\|c}'s critical variable\X\2\6
\&{else}\5
${}\{{}$\C{ delete \PB{\|c} from false list }\1\6
${}\\{oo},\39\|i\K\\{cmem}[\|c].\\{fplace},\39\|q\K\\{cmem}[\MM\\{fcount}].%
\\{fslot};{}$\6
${}\\{oo},\39\\{cmem}[\|i].\\{fslot}\K\|q,\39\\{cmem}[\|q].\\{fplace}\K\|i;{}$\6
${}\|h\PP{}$;\C{ the flipped literal is now critical }\6
\4${}\}{}$\2\6
\4${}\}{}$\2\6
\4${}\}{}$\2\par
\U42.\fi

\M{44}\B\X44:Make clauses \PB{\\{tmem}[\|k]}, \PB{$\\{tmem}[\|k+\T{1}]$}, \dots%
\ sadder\X${}\E{}$\6
\&{for} ( ; ${}\|k<\\{kk};{}$ ${}\|k\PP){}$\5
${}\{{}$\1\6
${}\\{ooo},\39\|c\K\\{tmem}[\|k],\39\|j\K\\{cmem}[\|c].\\{tcount}-\T{1},\39%
\\{cmem}[\|c].\\{tcount}\K\|j;{}$\6
\&{if} ${}(\|j\Z\T{1}){}$\5
${}\{{}$\1\6
\&{if} (\|j)\1\5
\X46:Increase the breakcount of \PB{\|c}'s critical variable\X\2\6
\&{else}\5
${}\{{}$\C{ insert \PB{\|c} into false list }\1\6
${}\\{oo},\39\\{cmem}[\\{fcount}].\\{fslot}\K\|c,\39\\{cmem}[\|c].\\{fplace}\K%
\\{fcount}\PP;{}$\6
\4${}\}{}$\2\6
\4${}\}{}$\2\6
\4${}\}{}$\2\par
\U42.\fi

\M{45}We know that \PB{\|c} has exactly one true literal at this moment.

\Y\B\4\X45:Decrease the breakcount of \PB{\|c}'s critical variable\X${}\E{}$\6
${}\{{}$\1\6
\&{for} ${}(\|o,\39\|i\K\\{cmem}[\|c].\\{start};{}$  ; ${}\|i\PP){}$\5
${}\{{}$\1\6
${}\|o,\39\|q\K\\{mem}[\|i];{}$\6
\&{if} ${}(\|o,\39\\{value}(\|q)){}$\1\5
\&{break};\2\6
\4${}\}{}$\2\6
${}\|o,\39\\{vmem}[\|q\GG\T{1}].\\{breakcount}\MM;{}$\6
\4${}\}{}$\2\par
\U43.\fi

\M{46}As an experiment, I'm swapping the first true literal into the first
position of its clause, hoping that subsequent ``decrease'' loops will
thereby be shortened.

\Y\B\4\X46:Increase the breakcount of \PB{\|c}'s critical variable\X${}\E{}$\6
${}\{{}$\1\6
\&{for} ${}(\|o,\39\\{ii}\K\|i\K\\{cmem}[\|c].\\{start};{}$  ; ${}\|i\PP){}$\5
${}\{{}$\1\6
${}\|o,\39\|q\K\\{mem}[\|i];{}$\6
\&{if} ${}(\|o,\39\\{value}(\|q)){}$\1\5
\&{break};\2\6
\4${}\}{}$\2\6
${}\|o,\39\\{vmem}[\|q\GG\T{1}].\\{breakcount}\PP;{}$\6
\&{if} ${}(\|i\I\\{ii}){}$\1\5
${}\\{oo},\39\\{mem}[\|i]\K\\{mem}[\\{ii}],\39\\{mem}[\\{ii}]\K\|q;{}$\2\6
\4${}\}{}$\2\par
\U44.\fi

\M{47}\B\X47:Print the initial guess\X${}\E{}$\6
${}\{{}$\1\6
${}\\{fprintf}(\\{stderr},\39\.{"\ initial\ guess"});{}$\6
\&{for} ${}(\|k\K\T{0};{}$ ${}\|k<\\{vars};{}$ ${}\|k\PP){}$\1\5
${}\\{fprintf}(\\{stderr},\39\.{"\ \%s\%.8s"},\39\\{vmem}[\|k].\\{val}\?\.{""}:%
\.{"\~"},\39\\{nmem}[\|k].\\{ch8});{}$\2\6
${}\\{fprintf}(\\{stderr},\39\.{"\\n"});{}$\6
\4${}\}{}$\2\par
\U37.\fi

\M{48}\B\X48:Print a solution and \PB{\&{goto} \\{done}}\X${}\E{}$\6
${}\{{}$\1\6
\&{for} ${}(\|k\K\T{0};{}$ ${}\|k<\\{vars};{}$ ${}\|k\PP){}$\1\5
${}\\{printf}(\.{"\ \%s\%.8s"},\39\\{vmem}[\|k].\\{val}\?\.{""}:\.{"\~"},\39%
\\{nmem}[\|k].\\{ch8});{}$\2\6
\\{printf}(\.{"\\n"});\6
\&{if} ${}(\\{verbose}\AND\\{show\_basics}){}$\1\5
${}\\{fprintf}(\\{stderr},\39\.{"!SAT!\\n"});{}$\2\6
\&{goto} \\{done};\6
\4${}\}{}$\2\par
\U37.\fi

\N{1}{49}Index.
\fi

\inx
\fin
\con
