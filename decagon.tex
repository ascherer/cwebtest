\input cwebmac
\datethis


\N{1}{1}Introduction. Nob sent me another problem today, and as usual I
couldn't
put it down. The challenge is to count how many ways we can pack isosceles
triangles into a regular decagon. There are two kinds of triangles, one
with base angles $72^\circ$ and the other with base angles $36^\circ$.
We are given 25 of the former and 5 of the latter.

These triangles have nice properties, which makes the problem appealing.
If we represent angles as multiples of $36^\circ$, the large triangles
have angles 2, 2, 1 and sides 1, 1, $\phi^{-1}$; the small triangles
have angles 1, 1, 3 and sides $\phi^{-2}$, $\phi^{-2}$, $\phi^{-1}$.
The area of the 10-gon, which has unit sides, is $10\phi^2A$, where $A$
is the area of a large triangle. The small triangle has $\phi^{-3}A$.
And it turns out that $25+5\phi^{-3}=10\phi^2$, because $\phi^{-3}=
\sqrt5-2$ and $2\phi^2=\sqrt5+3$.

My backtrack program works by maintaining a residual polygon-to-be-filled,
since I don't think I can use cells as I do with polyominoes or polyhexes.
Each polygon is represented as a cyclic list of the form $a_0$, $x_0$, $a_1$,
$x_1$, $a_2$, \dots, $a_n$, $x_n$, $a_0$, where the $a$'s are angles and
the $x$'s are lengths. When $a_k<5$, angle $a_k$ is a convex corner; at
each stage we choose a convex corner and replace $(a_k,x_k)$ by
$(a_k-\theta_1$, $s_1$, $10-\theta_2$, $s_2$, $5-\theta_3$, $x_k-s_3)$,
where $(\theta_1,s_1,\theta_2,s_2,\theta_3,s_3)$ is one of the six ways to
place a triangle at that corner. Adjustments are then made so that all
the $a$'s and $x$'s are positive.

Here are transformations that convert to positive, when possible:
If $x_k=0$ we replace $(a_{k-1},0,a_k)$ by $(a_{k-1}+a_k-5)$. If $x_k<0$
we replace $(a_{k-1},x_k,a_k)$ by $(a_{k-1}+5,-x_k,a_k-5)$. If $a_k=0$
we replace $(a_{k-1},x_{k-1},0,x_k)$ by $(a_{k-1}-5,x_k-x_{k-1})$.
When all $x$'s are positive and no $a$'s are zero, the polygon is erroneous
if any $a_k$ is negative or $\ge10$.

I thought about replacing $(x_{k-1},5,x_k)$ by $(x_{k-1}+x_k)$, but
decided against it.  Later, after watching the method in action, I
decided to do that replacement after all.

The first draft of this program seemed to work fine on special cases,
but when I ran it to completion on the full decagon problem it missed
some of the solutions. (It found only 5463628, while I knew from
symmetry considerations that the correct total would have the form
$20x+32$. The correct total is 5464292.) A serious bug in my original
reasoning, explained below, had to be corrected, hence the program is
now considerably more complicated than I thought it would be.

\Y\B\4\D$\\{big}$ \5
\T{25}\C{ this many big triangles must be placed }\par
\B\4\D$\\{small}$ \5
\T{5}\C{ and this many small ones }\par
\B\4\D$\\{total\_req}$ \5
$(\\{big}+\\{small}{}$)\par
\B\4\D$\\{eps}$ \5
$(\\{argc}>\T{2}{}$)\C{ causes PostScript output, one file per solution }\par
\B\4\D$\\{debug}$ \5
$(\\{argc}>\T{3}{}$)\C{ enables regular consistency checks }\par
\B\4\D$\\{verbose}$ \5
$(\\{argc}>\T{4}{}$)\C{ causes extra printout }\par
\Y\B\8\#\&{include} \.{<stdio.h>}\6
\X2:Type definitions\X\6
\X4:Global variables\X\6
\X3:Subroutines\X\7
${}\\{main}(\\{argc},\39\\{argv}){}$\1\1\6
\&{int} \\{argc};\6
\&{char} ${}{*}\\{argv}[\,];\2\2{}$\6
${}\{{}$\1\6
\&{int} \|i${},{}$ \|j${},{}$ \|k;\6
\&{register} \&{int} \|l;\C{ level of backtracking }\6
\&{int} \\{vert}${}\K\T{0}{}$;\C{ number of vertices known }\6
\&{int} \\{count}${}\K\T{0},{}$ \\{interval}${}\K\T{1},{}$ \\{eps\_interval}${}%
\K\T{1},{}$ \\{big\_need}${}\K\\{big},{}$ \\{small\_need}${}\K\\{small};{}$\6
\&{register} \&{node} ${}{*}\|p,{}$ ${}{*}\|q,{}$ ${}{*}\\{pp},{}$ ${}{*}%
\\{qq},{}$ ${}{*}\|r,{}$ ${}{*}\\{rr};{}$\7
\&{if} ${}(\\{argc}>\T{1}){}$\5
${}\{{}$\1\6
${}\\{sscanf}(\\{argv}[\T{1}],\39\.{"\%d"},\39{\AND}\\{interval});{}$\6
\&{if} (\\{eps})\1\5
${}\\{sscanf}(\\{argv}[\T{2}],\39\.{"\%d"},\39{\AND}\\{eps\_interval});{}$\2\6
\4${}\}{}$\2\6
\X7:Initialize the tables\X;\6
\X12:Backtrack through all solutions\X;\6
${}\\{printf}(\.{"Altogether\ \%d\ solut}\)\.{ions.\\n"},\39\\{count});{}$\6
\4${}\}{}$\2\par
\fi

\N{1}{2}Polygons. Circular lists that represent polygons
are doubly linked in a straightforward way.
The only slightly tricky thing is that we represent lengths in the
form $s\phi^{-1}+t\phi^{-2}$, where $s$ and $t$ are integers.

Each angle contains a vertex number. The first few vertices come from
the initial input. Each level of backtracking adds two more, some of
which will be identified with earlier vertices.

\Y\B\4\X2:Type definitions\X${}\E{}$\6
\&{typedef} \&{struct} \&{node\_struct} ${}\{{}$\1\6
\&{struct} \&{node\_struct} ${}{*}\\{next},{}$ ${}{*}\\{prev};{}$\6
\&{int} \|s;\C{ angle, or first component of length }\6
\&{int} \|t;\C{ vertex number, or second component of length }\6
\&{int} \\{dir};\C{ direction to the next vertex (used only in angle nodes }\2\6
${}\}{}$ \&{node};\par
\U1.\fi

\M{3}The nodes are allocated with a normal sort of available space list.

\Y\B\4\X3:Subroutines\X${}\E{}$\6
\&{node} ${}{*}\\{get\_avail}(\,){}$\1\1\2\2\6
${}\{{}$\1\6
\&{register} \&{node} ${}{*}\|p;{}$\7
\&{if} (\\{avail})\5
${}\{{}$\1\6
${}\|p\K\\{avail};{}$\6
${}\\{avail}\K\|p\MG\\{next};{}$\6
\4${}\}{}$\2\6
\&{else} \&{if} ${}(\\{next\_node}\E\\{bad\_node}){}$\5
${}\{{}$\1\6
\\{printf}(\.{"ALLOCATING...\\n"});\C{ temporary }\6
${}\|p\K{}$(\&{node} ${}{*}){}$ \\{calloc}${}(\T{1000},\39\&{sizeof}(%
\&{node}));{}$\6
\&{if} ${}(\|p\E\NULL){}$\5
${}\{{}$\1\6
\\{printf}(\.{"Out\ of\ memory!\\n"});\6
${}\\{exit}({-}\T{1});{}$\6
\4${}\}{}$\2\6
${}\\{next\_node}\K\|p+\T{1};{}$\6
${}\\{bad\_node}\K\|p+\T{1000};{}$\6
\4${}\}{}$\2\6
\&{else}\1\5
${}\|p\K\\{next\_node}\PP;{}$\2\6
\&{return} \|p;\6
\4${}\}{}$\2\par
\As35\ET38.
\U1.\fi

\M{4}\B\X4:Global variables\X${}\E{}$\6
\&{node} ${}{*}\\{avail}{}$;\C{ a node that was recycled }\6
\&{node} ${}{*}\\{next\_node}{}$;\C{ the next node not yet used }\6
\&{node} ${}{*}\\{bad\_node}{}$;\C{ end of currently allocated block of nodes }%
\par
\As5, 6, 8, 11, 13, 33\ETs36.
\U1.\fi

\M{5}Here are the six ways to place triangles---three ways each.

\Y\B\4\X4:Global variables\X${}\mathrel+\E{}$\6
\&{int} \\{triang}[\T{6}][\T{9}]${}\K\{\{\T{2},\39\T{1},\39\T{1},\39\T{1},\39%
\T{1},\39\T{1},\39\T{2},\39\T{1},\39\T{0}\},\39\{\T{1},\39\T{1},\39\T{1},\39%
\T{2},\39\T{1},\39\T{0},\39\T{2},\39\T{1},\39\T{1}\},\39\{\T{2},\39\T{1},\39%
\T{0},\39\T{2},\39\T{1},\39\T{1},\39\T{1},\39\T{1},\39\T{1}\},\3{-1}\39\{\T{1},%
\39\T{0},\39\T{1},\39\T{3},\39\T{0},\39\T{1},\39\T{1},\39\T{1},\39\T{0}\},\39\{%
\T{3},\39\T{0},\39\T{1},\39\T{1},\39\T{1},\39\T{0},\39\T{1},\39\T{0},\39\T{1}%
\},\39\{\T{1},\39\T{1},\39\T{0},\39\T{1},\39\T{0},\39\T{1},\39\T{3},\39\T{0},%
\39\T{1}\}\}{}$;\par
\fi

\M{6}A complication mentioned later will make it necessary to work with more
than
one polygon in certain cases. So in general we assume that there is a
stack of polygons to be filled, pointed to by \PB{\\{poly}[\T{0}]}, \dots, \PB{%
\\{poly}[\\{top}]};
we will currently be working on \PB{\\{poly}[\\{top}]}.

\Y\B\4\X4:Global variables\X${}\mathrel+\E{}$\6
\&{node} ${}{*}\\{poly}[\\{total\_req}]{}$;\C{ polygons to be filled }\6
\&{int} \\{top};\C{ index to the topmost one }\par
\fi

\M{7}The \PB{\\{dir}} fields of the polygon will always satisfy the invariant
relation \PB{$\|p\MG\\{dir}\K\|p\MG\\{prev}\MG\\{prev}\MG\\{dir}+\T{5}-\|p\MG%
\|s$}, modulo~10,
when \PB{\|p} is an angle node. Moreover, the sum of $5-s$ over all angle
nodes of a polygon will equal 10.

\Y\B\4\D$\\{init\_pts}$ \5
\T{10}\par
\Y\B\4\X7:Initialize the tables\X${}\E{}$\6
$\|p\K\\{get\_avail}(\,);{}$\6
${}\\{poly}[\T{0}]\K\|p;{}$\6
\&{for} ${}(\|j\K\T{0};{}$ ${}\|j<\\{init\_pts};{}$ ${}\|j\PP){}$\5
${}\{{}$\1\6
${}\|q\K\\{get\_avail}(\,);{}$\6
${}\|p\MG\|s\K\T{4};{}$\6
${}\|p\MG\|t\K\\{vert}\PP;{}$\6
${}\|p\MG\\{dir}\K\|j;{}$\6
${}\|p\MG\\{next}\K\|q;{}$\6
${}\|q\MG\\{prev}\K\|p;{}$\6
${}\|p\K(\|j<\\{init\_pts}-\T{1}\?\\{get\_avail}(\,):\\{poly}[\T{0}]);{}$\6
${}\|q\MG\|s\K\T{1};{}$\6
${}\|q\MG\|t\K\T{1};{}$\6
${}\|q\MG\\{next}\K\|p;{}$\6
${}\|p\MG\\{prev}\K\|q;{}$\6
\4${}\}{}$\2\par
\As9, 10\ETs37.
\U1.\fi

\N{1}{8}Coordinates. The method I sketched in the introduction sounded good
to me at first, but it has a fatal flaw. The problem occurs when we
try to branch at a convex corner that has already been covered by
another part of the polygon. (Consider, for example, the case where
the polygon consists of two nonadjacent triangles, separated by a
crooked path traced in both directions so that it contributes nothing
to the total area.) To avoid this bug, it is necessary to know more
than the sequence of lengths and angles; we need to be able to tell
when two vertices are identical as points in the plane.

Floating-point arithmetic could be used for this purpose, with care, but
I prefer to use exact integer arithmetic. We can regard each vertex
as a point in the complex plane, represented in the form $\sum_{k=0}^9
x_k\zeta^k$, where the $x$'s are integers and $\zeta=e^{\pi i/5}$ is a
10th root of unity. This is possible because the location of each point
is the location of a previous point plus a number of the form $(s
+t\phi^{-1})\zeta^k$, and because $\phi^{-1}=\zeta^2-\zeta^3$. (We scale
all dimensions up by $\phi$ for convenience.) Such a representation is
highly redundant, because $\zeta$ satisfies the equation $\zeta^4-\zeta^3
+\zeta^2-\zeta+1=0$; but it is unique if $x_4=\cdots=x_9=0$, because that
equation is irreducible over the rationals. (See {\sl Seminumerical
Algorithms}, exercise 4.6.2-32.)

The absolute values of $x_0$, $x_1$, $x_2$, and $x_3$ will be small in any
covering, because they are obtained by adding small numbers of the
form $(s+t\phi^{-1})\zeta^k$ for at most 30 values of $(s,t,k)$.  So
we will represent each point internally as a single 32-bit number,
$$(x_3+128)\cdot 2^{24}+(x_2+128)\cdot 2^{16}+(x_1+128)\cdot 2^8+x_0+128\,.$$
To compute the coordinates of each point it suffices to have short tables
for the amounts to add to the representation when we want to add
$\zeta^k$ or $\phi^{-1}\zeta^k$.

\Y\B\4\D$\\{pack}(\|a,\|b,\|c,\|d)$ \5
$(\|a\LL\T{24})+(\|b\LL\T{16})+(\|c\LL\T{8})+{}$\|d\par
\Y\B\4\X4:Global variables\X${}\mathrel+\E{}$\6
\&{unsigned} \&{int} ${}\|x[\\{init\_pts}+\T{2}*\\{total\_req}]{}$;\C{ the
coordinates }\6
\&{unsigned} \&{int} \\{delta\_s}[\T{10}]${}\K\{\\{pack}(\T{0},\39\T{0},\39%
\T{0},\39\T{1}),\39\\{pack}(\T{0},\39\T{0},\39\T{1},\39\T{0}),\39\\{pack}(%
\T{0},\39\T{1},\39\T{0},\39\T{0}),\39\\{pack}(\T{1},\39\T{0},\39\T{0},\39%
\T{0}),\39\\{pack}(\T{1},\39{-}\T{1},\39\T{1},\39{-}\T{1}),\3{-1}\39\\{pack}(%
\T{0},\39\T{0},\39\T{0},\39{-}\T{1}),\39\\{pack}(\T{0},\39\T{0},\39{-}\T{1},\39%
\T{0}),\39\\{pack}(\T{0},\39{-}\T{1},\39\T{0},\39\T{0}),\39\\{pack}({-}\T{1},%
\39\T{0},\39\T{0},\39\T{0}),\39\\{pack}({-}\T{1},\39\T{1},\39{-}\T{1},\39\T{1})%
\};{}$\6
\&{unsigned} \&{int} \\{delta\_t}[\T{10}]${}\K\{\\{pack}({-}\T{1},\39\T{1},\39%
\T{0},\39\T{0}),\39\\{pack}(\T{0},\39\T{1},\39{-}\T{1},\39\T{1}),\39\\{pack}(%
\T{1},\39{-}\T{1},\39\T{1},\39\T{0}),\39\\{pack}(\T{0},\39\T{0},\39\T{1},\39{-}%
\T{1}),\39\\{pack}(\T{0},\39\T{1},\39{-}\T{1},\39\T{0}),\3{-1}\39\\{pack}(%
\T{1},\39{-}\T{1},\39\T{0},\39\T{0}),\39\\{pack}(\T{0},\39{-}\T{1},\39\T{1},%
\39{-}\T{1}),\39\\{pack}({-}\T{1},\39\T{1},\39{-}\T{1},\39\T{0}),\39\\{pack}(%
\T{0},\39\T{0},\39{-}\T{1},\39\T{1}),\39\\{pack}(\T{0},\39{-}\T{1},\39\T{1},\39%
\T{0})\}{}$;\par
\fi

\M{9}\B\X7:Initialize the tables\X${}\mathrel+\E{}$\6
$\|x[\T{0}]\K\\{pack}(\T{128},\39\T{128},\39\T{128},\39\T{128});{}$\6
\&{for} ${}(\|j\K\T{1},\39\|p\K\\{poly}[\T{0}];{}$ ${}\|j<\\{init\_pts};{}$ ${}%
\|j\PP,\39\|p\K\|p\MG\\{next}\MG\\{next}){}$\1\5
${}\|x[\|j]\K\|x[\|j-\T{1}]+\|p\MG\\{next}\MG\|s*\\{delta\_s}[\|p\MG\\{dir}]+%
\|p\MG\\{next}\MG\|t*\\{delta\_t}[\|p\MG\\{dir}]{}$;\2\par
\fi

\M{10}We will occasionally need to decide whether a number of the form
$s+t\phi^{-1}$ is positive, negative, or zero. There is an interesting
recursive way to make this test: The answer is obvious unless $st<0$;
and in the latter case $s+t\phi^{-1}$ has the same sign as $s\phi+t=
s+t+s\phi^{-1}$, so we can replace $(s,t)$ by the pair $(s+t,s)$.

But for our purposes it is sufficient simply to test the sign of $13s+8t$,
since $s$ and $t$ will not get large enough to make this trick fail.

\Y\B\4\X7:Initialize the tables\X${}\mathrel+\E{}$\6
\&{for} ${}(\|j\K\T{0};{}$ ${}\|j<\T{6};{}$ ${}\|j\PP){}$\5
${}\{{}$\1\6
${}\\{thresh1}[\|j]\K\T{13}*\\{triang}[\|j][\T{1}]+\T{8}*\\{triang}[\|j][%
\T{2}];{}$\6
${}\\{thresh3}[\|j]\K\T{13}*\\{triang}[\|j][\T{7}]+\T{8}*\\{triang}[\|j][%
\T{8}];{}$\6
\4${}\}{}$\2\par
\fi

\M{11}\B\X4:Global variables\X${}\mathrel+\E{}$\6
\&{int} \\{thresh1}[\T{6}];\C{ encoded version of the first length }\6
\&{int} \\{thresh3}[\T{6}];\C{ encoded version of the third length }\par
\fi

\N{1}{12}Backtracking.

Two heuristics allow a quick decision: A triangle position is impossible
if the existing angle is too small, or if the existing
side is too small and between two convex angles (i.e., between angles
that each have $s<5$).

\Y\B\4\X12:Backtrack through all solutions\X${}\E{}$\6
$\|l\K\T{0};{}$\6
\4\\{newlev}:\6
\&{if} ${}(\|l\E\\{total\_req}){}$\5
${}\{{}$\1\6
\&{if} ${}(\\{top}<\T{0}){}$\1\5
\X30:Record a solution\X;\2\6
\&{goto} \\{backup};\6
\4${}\}{}$\2\6
${}\\{ht}[\|l]\K\\{top};{}$\6
${}\\{lb}[\|l]\K(\\{big\_need}\E\T{0}\?\T{3}:\T{0});{}$\6
${}\\{ub}[\|l]\K(\\{small\_need}\E\T{0}\?\T{3}:\T{6});{}$\6
\X14:Find corner to branch on\X;\6
${}\\{way}[\|l]\K\\{lb}[\|l];{}$\6
\4\\{tryit}:\5
${}\|j\K\\{way}[\|l];{}$\6
${}\|p\K\\{choice}[\|l];{}$\6
\&{if} ${}(\|p\MG\|s<\\{triang}[\|j][\T{0}]){}$\1\5
\&{goto} \\{nogood};\C{ angle is too small }\2\6
${}\|q\K\|p\MG\\{next};{}$\6
${}\|r\K\|q\MG\\{next};{}$\6
\&{if} ${}(\|r\MG\|s<\T{5}){}$\5
${}\{{}$\1\6
\&{if} ${}(\T{13}*\|q\MG\|s+\T{8}*\|q\MG\|t<\\{thresh3}[\|j]){}$\1\5
\&{goto} \\{nogood};\C{ side after \PB{\|p} is too small }\2\6
\&{if} ${}(\T{13}*\|q\MG\|s+\T{8}*\|q\MG\|t\E\\{thresh3}[\|j]\W\|r\MG\|s<%
\\{triang}[\|j][\T{6}]){}$\1\5
\&{goto} \\{nogood};\2\6
\4${}\}{}$\2\6
\&{if} ${}(\|p\MG\|s\E\\{triang}[\|j][\T{0}]\W\|p\MG\\{prev}\MG\\{prev}\MG\|s<%
\T{5}){}$\5
${}\{{}$\1\6
\&{if} ${}(\T{13}*\|p\MG\\{prev}\MG\|s+\T{8}*\|p\MG\\{prev}\MG\|t<\\{thresh1}[%
\|j]){}$\1\5
\&{goto} \\{nogood};\C{ side preceding \PB{\|p} is too small }\2\6
\&{if} ${}(\T{13}*\|p\MG\\{prev}\MG\|s+\T{8}*\|p\MG\\{prev}\MG\|t\E\\{thresh1}[%
\|j]\W\|p\MG\\{prev}\MG\\{prev}\MG\|s<\\{triang}[\|j][\T{3}]){}$\1\5
\&{goto} \\{nogood};\2\6
\4${}\}{}$\2\6
\X15:Install triangle \PB{\|j} at position \PB{\\{choice}[\|l]}\X;\6
\&{if} (\\{debug})\1\5
\X27:Examine the current choice and its ramifications\X;\2\6
\&{if} ${}(\\{way}[\|l]<\T{3}){}$\1\5
${}\\{big\_need}\MM{}$;\5
\2\&{else}\1\5
${}\\{small\_need}\MM;{}$\2\6
${}\|l\PP;{}$\6
${}\\{vert}\MRL{+{\K}}\T{2};{}$\6
\&{goto} \\{newlev};\6
\4\\{nogood}:\6
\&{if} ${}(\PP\\{way}[\|l]<\\{ub}[\|l]){}$\1\5
\&{goto} \\{tryit};\2\6
\4\\{backup}:\6
\&{if} ${}(\|l\E\T{0}){}$\1\5
\&{goto} \\{done};\2\6
${}\|l\MM;{}$\6
${}\\{vert}\MRL{-{\K}}\T{2};{}$\6
\&{if} ${}(\\{way}[\|l]<\T{3}){}$\1\5
${}\\{big\_need}\PP{}$;\5
\2\&{else}\1\5
${}\\{small\_need}\PP;{}$\2\6
\X28:Undo the changes made in level \PB{\|l}\X;\6
\&{goto} \\{nogood};\6
\4\\{done}:\par
\U1.\fi

\M{13}\B\X4:Global variables\X${}\mathrel+\E{}$\6
\&{node} ${}{*}\\{bhoice}[\\{total\_req}]{}$;\C{ convex corner where branching
occurs }\6
\&{int} \\{way}[\\{total\_req}];\C{ which way we tried to place a triangle }\6
\&{node} ${}{*}\\{choice}[\\{total\_req}]{}$;\C{ where we tried to place it }\6
\&{node} ${}{*}\\{save}[\\{total\_req}]{}$;\C{ polygons to restore when
backtracing }\6
\&{int} \\{lb}[\\{total\_req}]${},{}$ \\{ub}[\\{total\_req}];\C{ bounds on \PB{%
\\{way}} }\6
\&{int} \\{ht}[\\{total\_req}];\C{ size of stack when the choice was made }\par
\fi

\M{14}After experimenting with simpler rules here, I decided it was best to
choose a corner that results in the fewest possibilities with respect to
the two heuristics just mentioned.

We also must restrict the branch point to a convex corner. Otherwise
we might miss important solutions.

\Y\B\4\X14:Find corner to branch on\X${}\E{}$\6
\&{for} ${}(\|p\K\\{poly}[\\{top}],\39\\{rr}\K\|p\MG\\{prev}\MG\\{prev},\39\|i%
\K\T{10000000};{}$  ; ${}\\{rr}\K\|p,\39\|p\K\|r){}$\5
${}\{{}$\1\6
${}\|q\K\|p\MG\\{next};{}$\6
${}\|r\K\|q\MG\\{next};{}$\6
\&{if} ${}(\|p\MG\|s<\T{5}){}$\5
${}\{{}$\1\6
\&{for} ${}(\|j\K\\{lb}[\|l],\39\|k\K\T{0};{}$ ${}\|j<\\{ub}[\|l];{}$ ${}\|j%
\PP){}$\1\6
\&{if} ${}(\|p\MG\|s\G\\{triang}[\|j][\T{0}]\W(\|r\MG\|s>\T{5}\V(\T{13}*(\|q\MG%
\|s)+\T{8}*(\|q\MG\|t))\G\\{thresh3}[\|j])\W(\|p\MG\|s>\\{triang}[\|j][\T{0}]\V%
\\{rr}\MG\|s>\T{5}\V\T{13}*\|p\MG\\{prev}\MG\|s+\T{8}*\|p\MG\\{prev}\MG\|t\G%
\\{thresh1}[\|j])){}$\1\5
${}\|k\PP;{}$\2\2\6
\&{if} ${}(\|k<\|i){}$\1\5
${}\|i\K\|k,\39\\{pp}\K\|p;{}$\2\6
\4${}\}{}$\2\6
\&{if} ${}(\|r\E\\{poly}[\\{top}]){}$\1\5
\&{break};\2\6
\4${}\}{}$\2\6
${}\\{choice}[\|l]\K\\{pp}{}$;\par
\U12.\fi

\M{15}\B\X15:Install triangle \PB{\|j} at position \PB{\\{choice}[\|l]}\X${}%
\E{}$\6
\X16:Copy the current polygon and save the old version\X;\6
\X17:Create new vertices \PB{\\{pp}}, \PB{\\{qq}}, and the line \PB{\|r}
between them\X;\6
\X18:Insert \PB{\\{qq}} at the choice point; split into two polygons if
necessary\X;\6
\X22:Insert \PB{\\{pp}} at the choice point; split into two polygons if
necessary\X;\par
\U12.\fi

\M{16}My first draft program avoided full copying by copying only the nodes
that changed. It was rather elegant, but alas---it implemented a bad algorithm.
The correct algorithm manipulates the lists in more complex ways, hence
partial copying is no longer feasible; it would be too complicated.

\Y\B\4\X16:Copy the current polygon and save the old version\X${}\E{}$\6
$\\{save}[\|l]\K\\{poly}[\\{top}];{}$\6
${}\\{rr}\K\\{get\_avail}(\,);{}$\6
\&{for} ${}(\\{pp}\K\\{rr},\39\|p\K\\{choice}[\|l];{}$  ; ${}\|p\K\|p\MG%
\\{next}){}$\5
${}\{{}$\1\6
${}\\{pp}\MG\|s\K\|p\MG\|s;{}$\6
${}\\{pp}\MG\|t\K\|p\MG\|t;{}$\6
${}\\{pp}\MG\\{dir}\K\|p\MG\\{dir};{}$\6
${}\\{qq}\K\\{get\_avail}(\,);{}$\6
${}\\{pp}\MG\\{next}\K\\{qq};{}$\6
${}\\{qq}\MG\\{prev}\K\\{pp};{}$\6
${}\|p\K\|p\MG\\{next};{}$\6
${}\\{qq}\MG\|s\K\|p\MG\|s;{}$\6
${}\\{qq}\MG\|t\K\|p\MG\|t;{}$\6
\&{if} ${}(\|p\MG\\{next}\E\\{choice}[\|l]){}$\1\5
\&{break};\2\6
${}\\{pp}\K\\{get\_avail}(\,);{}$\6
${}\\{qq}\MG\\{next}\K\\{pp};{}$\6
${}\\{pp}\MG\\{prev}\K\\{qq};{}$\6
\4${}\}{}$\2\6
${}\\{qq}\MG\\{next}\K\\{rr};{}$\6
${}\\{rr}\MG\\{prev}\K\\{qq}{}$;\C{ \PB{\\{poly}[\\{top}]} has not been updated
}\par
\U15.\fi

\M{17}\B\X17:Create new vertices \PB{\\{pp}}, \PB{\\{qq}}, and the line \PB{%
\|r} between them\X${}\E{}$\6
$\\{pp}\K\\{get\_avail}(\,);{}$\6
${}\\{pp}\MG\|t\K\\{vert};{}$\6
${}\\{qq}\K\\{get\_avail}(\,);{}$\6
${}\\{qq}\MG\|t\K\\{vert}+\T{1};{}$\6
${}\|r\K\\{get\_avail}(\,);{}$\6
${}\|r\MG\|s\K\\{triang}[\|j][\T{4}];{}$\6
${}\|r\MG\|t\K\\{triang}[\|j][\T{5}];{}$\6
${}\\{pp}\MG\\{next}\K\|r;{}$\6
${}\|r\MG\\{prev}\K\\{pp};{}$\6
${}\|r\MG\\{next}\K\\{qq};{}$\6
${}\\{qq}\MG\\{prev}\K\|r;{}$\6
${}\|k\K(\\{rr}\MG\\{dir}+\\{triang}[\|j][\T{0}]+\T{100})\MOD\T{10}{}$;\C{
direction from the choice node to \PB{\\{pp}} }\6
${}\|x[\\{vert}]\K\|x[\\{rr}\MG\|t]+\\{triang}[\|j][\T{1}]*\\{delta\_s}[\|k]+%
\\{triang}[\|j][\T{2}]*\\{delta\_t}[\|k];{}$\6
${}\|k\K(\|k+\\{triang}[\|j][\T{3}]+\T{5})\MOD\T{10};{}$\6
${}\\{pp}\MG\\{dir}\K\|k;{}$\6
${}\\{pp}\MG\|s\K\T{10}-\\{triang}[\|j][\T{3}];{}$\6
${}\|x[\\{vert}+\T{1}]\K\|x[\\{vert}]+\\{triang}[\|j][\T{4}]*\\{delta\_s}[\|k]+%
\\{triang}[\|j][\T{5}]*\\{delta\_t}[\|k]{}$;\par
\U15.\fi

\M{18}We maintain the following conditions in the polygons: (1) All angles
$s$ are in the range $s\le 9$, $s\ne0$, $s\ne5$. (2)~All vertices are at
distinct points in the plane.

We don't bother to check that the new polygon doesn't intersect itself,
except when the point of intersection is at a vertex. Self-intersecting
polygons of other types  will not lead to solutions,
since they will doubly cover some
points and will therefore be incompletely filled when we have used up
our quota of triangles. If we checked for self-intersection, the search
tree would be smaller, but I think the total search time would be
longer, because of the extra time spent in checking.

Previous steps have created nodes \PB{\\{rr}}, \PB{\\{pp}}, \PB{\\{qq}} for the
new
triangle. Node \PB{\\{rr}} is the choice point in the current polygon; we have
not
yet linked \PB{\\{pp}} and \PB{\\{qq}} into that polygon, nor have we recorded
anything about it in \PB{\\{poly}[\\{top}]}. The triangle will be inserted in
such
a way that the line from \PB{\\{rr}} to \PB{\\{qq}} runs along the existing
line
from \PB{\\{rr}} to its successor.

\Y\B\4\X18:Insert \PB{\\{qq}} at the choice point; split into two polygons if
necessary\X${}\E{}$\6
$\|q\K\\{rr}\MG\\{next};{}$\6
${}\|p\K\|q\MG\\{next}{}$;\C{ \PB{\|q} is the line between \PB{\\{rr}} and \PB{%
\|p} }\6
${}\|k\K\T{13}*\|q\MG\|s+\T{8}*\|q\MG\|t;{}$\6
\&{if} ${}(\|k\E\\{thresh3}[\|j]){}$\1\5
\X19:Connect \PB{\\{pp}} directly to existing vertex \PB{$\|p\E\\{qq}$}\X\2\6
\&{else}\5
${}\{{}$\1\6
\&{if} ${}(\|k>\\{thresh3}[\|j]){}$\5
${}\{{}$\C{ the line from \PB{\\{rr}} to \PB{\|p} is longer than needed }\1\6
${}\|q\MG\|s\MRL{-{\K}}\\{triang}[\|j][\T{7}];{}$\6
${}\|q\MG\|t\MRL{-{\K}}\\{triang}[\|j][\T{8}];{}$\6
${}\\{qq}\MG\|s\K\T{5}-\\{triang}[\|j][\T{6}];{}$\6
\4${}\}{}$\2\6
\&{else}\5
${}\{{}$\C{ the line from \PB{\\{rr}} to \PB{\|p} is shorter than from \PB{%
\\{rr}} to \PB{\\{qq}} }\1\6
${}\|p\MG\|s\MRL{-{\K}}\T{5}{}$;\C{ we know this is $>0$ }\6
${}\|q\MG\|s\K\\{triang}[\|j][\T{7}]-\|q\MG\|s;{}$\6
${}\|q\MG\|t\K\\{triang}[\|j][\T{8}]-\|q\MG\|t;{}$\6
${}\\{qq}\MG\|s\K\T{10}-\\{triang}[\|j][\T{6}];{}$\6
\4${}\}{}$\2\6
${}\\{qq}\MG\\{next}\K\|q;{}$\6
${}\|q\MG\\{prev}\K\\{qq};{}$\6
${}\\{qq}\MG\\{dir}\K\\{pp}\MG\\{dir}+\T{5}-\\{qq}\MG\|s;{}$\6
\&{for} ${}(\|p\K\|p\MG\\{next}\MG\\{next};{}$ ${}\|p\I\\{rr};{}$ ${}\|p\K\|p%
\MG\\{next}\MG\\{next}){}$\1\6
\&{if} ${}(\|x[\|p\MG\|t]\E\|x[\\{vert}+\T{1}]){}$\5
${}\{{}$\C{ \PB{\\{qq}} coincides with a previous point }\1\6
\X20:Split off a polygon at position \PB{$\\{qq}\E\|p$}\X;\6
\&{break};\6
\4${}\}{}$\2\2\6
\4${}\}{}$\2\6
\X21:Remove angle 0 or 5 at \PB{\|p}, if present after the \PB{\\{qq}}
insertion stage\X;\par
\U15.\fi

\M{19}Node \PB{\|r} is the line between \PB{\\{pp}} and \PB{\\{qq}}; node \PB{%
\|q} is the line between
\PB{\\{rr}} and \PB{\|p}. We've discovered that these lines are identical; so
we discard
\PB{\|q} and \PB{\\{qq}}. If the new angle at \PB{\|p} is negative,
backtracking will
occur at level \PB{$\|l+\T{1}$}, so we don't bother to check for that unlikely
event.

\Y\B\4\X19:Connect \PB{\\{pp}} directly to existing vertex \PB{$\|p\E\\{qq}$}%
\X${}\E{}$\6
${}\{{}$\1\6
${}\|r\MG\\{next}\K\|p;{}$\6
${}\|p\MG\\{prev}\K\|r;{}$\6
${}\|q\MG\\{next}\K\\{qq};{}$\6
${}\\{qq}\MG\\{next}\K\\{avail};{}$\6
${}\\{avail}\K\|q;{}$\6
${}\|p\MG\|s\MRL{-{\K}}\\{triang}[\|j][\T{6}];{}$\6
\4${}\}{}$\2\par
\U18.\fi

\M{20}This part of the program is the price I had to pay to fix my original
ill-understood algorithm. We separate out the subpolygon (\PB{\\{qq}}, \dots,
\PB{$\|p-\\{pred}$}, \PB{\\{qq}}) and connect \PB{\\{pp}} to \PB{\|p} and its
successors rather than to~\PB{\\{qq}}.

The split-off polygon might not have a winding number of 1 (I mean, the
sum of its exterior angles \PB{$\T{5}-\|s$} might not be 10). Then it might
have
no convex corners. But in such a case, the remaining polygon would have
a negative angle, so we would never have to look at the split-off polygon
(which is lower in the stack). This reasoning is somewhat subtle, and the
case may never arise, but I do want to record it here because I think it
is correct and because I don't want to imply that I ignored a
potential problem.

\Y\B\4\X20:Split off a polygon at position \PB{$\\{qq}\E\|p$}\X${}\E{}$\6
$\\{qq}\MG\\{prev}\K\|p\MG\\{prev};{}$\6
${}\|p\MG\\{prev}\MG\\{next}\K\\{qq};{}$\6
${}\|k\K\\{qq}\MG\|s+\|p\MG\|s-\T{10};{}$\6
${}\\{qq}\MG\|s\K(\|p\MG\\{prev}\MG\\{prev}\MG\\{dir}+\T{105}-\\{qq}\MG\\{dir})%
\MOD\T{10}{}$;\C{ \PB{$\\{qq}\MG\\{dir}$} stays the same }\6
${}\|p\MG\\{prev}\K\|r;{}$\6
${}\|r\MG\\{next}\K\|p;{}$\6
${}\|p\MG\|s\K\|k-\\{qq}\MG\|s{}$;\C{ if negative, we'll discover a problem
soon }\6
${}\|k\K\\{qq}\MG\|s;{}$\6
\&{if} ${}(\|k\E\T{0}\V\|k\E\T{5}){}$\5
${}\{{}$\C{ recall that \PB{$\|q\K\\{qq}\MG\\{next}$} }\1\6
${}\\{qq}\K\\{qq}\MG\\{prev};{}$\6
\&{if} ${}(\|k\E\T{5}){}$\1\5
${}\\{qq}\MG\|s\MRL{+{\K}}\|q\MG\|s,\39\\{qq}\MG\|t\MRL{+{\K}}\|q\MG\|t;{}$\2\6
\&{else} \&{if} ${}(\T{13}*(\\{qq}\MG\|s-\|q\MG\|s)+\T{8}*(\\{qq}\MG\|t-\|q\MG%
\|t)<\T{0}){}$\1\5
${}\\{qq}\MG\|s\K\|q\MG\|s-\\{qq}\MG\|s,\39\\{qq}\MG\|t\K\|q\MG\|t-\\{qq}\MG%
\|t,\39\\{qq}\MG\\{prev}\MG\|s\MRL{-{\K}}\T{5},\39\\{qq}\MG\\{prev}\MG\\{dir}%
\MRL{+{\K}}\T{5};{}$\2\6
\&{else}\1\5
${}\\{qq}\MG\|s\MRL{-{\K}}\|q\MG\|s,\39\\{qq}\MG\|t\MRL{-{\K}}\|q\MG\|t,\39\|q%
\MG\\{next}\MG\|s\MRL{-{\K}}\T{5};{}$\2\6
${}\\{qq}\MG\\{next}\K\|q\MG\\{next};{}$\6
${}\|q\MG\\{next}\MG\\{prev}\K\\{qq};{}$\6
${}\\{qq}\K\|q\MG\\{next};{}$\6
${}\|q\MG\\{next}\K\\{avail};{}$\6
${}\\{avail}\K\|q\MG\\{prev};{}$\6
\4${}\}{}$\2\6
${}\\{poly}[\\{top}\PP]\K\\{qq}{}$;\par
\U18.\fi

\M{21}If the new angle at \PB{\|p} is zero, there's a possibility that point %
\PB{\\{pp}}
is coincident with the successor of \PB{\|p}. In that case we temporarily
retain both points, with a line of length 0 between them.

\Y\B\4\X21:Remove angle 0 or 5 at \PB{\|p}, if present after the \PB{\\{qq}}
insertion stage\X${}\E{}$\6
\&{if} ${}(\|p\MG\|s\E\T{0}\V\|p\MG\|s\E\T{5}){}$\5
${}\{{}$\1\6
${}\|q\K\|p\MG\\{next}{}$;\C{ at this point \PB{$\|r\K\|p\MG\\{prev}$} }\6
\&{if} ${}(\|p\MG\|s\E\T{5}){}$\1\5
${}\|r\MG\|s\MRL{+{\K}}\|q\MG\|s,\39\|r\MG\|t\MRL{+{\K}}\|q\MG\|t;{}$\2\6
\&{else} \&{if} ${}(\T{13}*(\|r\MG\|s-\|q\MG\|s)+\T{8}*(\|r\MG\|t-\|q\MG\|t)\Z%
\T{0}){}$\1\5
${}\|r\MG\|s\K\|q\MG\|s-\|r\MG\|s,\39\|r\MG\|t\K\|q\MG\|t-\|r\MG\|t,\39\\{pp}%
\MG\|s\MRL{-{\K}}\T{5},\39\\{pp}\MG\\{dir}\MRL{+{\K}}\T{5}{}$;\C{ \PB{$\\{pp}\K%
\|r\MG\\{prev}$} }\2\6
\&{else}\1\5
${}\|r\MG\|s\MRL{-{\K}}\|q\MG\|s,\39\|r\MG\|t\MRL{-{\K}}\|q\MG\|t,\39\|q\MG%
\\{next}\MG\|s\MRL{-{\K}}\T{5};{}$\2\6
${}\|r\MG\\{next}\K\|q\MG\\{next};{}$\6
${}\|q\MG\\{next}\MG\\{prev}\K\|r;{}$\6
${}\|q\MG\\{next}\K\\{avail};{}$\6
${}\\{avail}\K\|p;{}$\6
\4${}\}{}$\2\par
\U18.\fi

\M{22}How do things stand now? We have a path from \PB{\\{pp}} to \PB{\\{rr}},
and \PB{\|r} is
the line out of \PB{\\{pp}}; the length of \PB{\|r} might be zero. Variables %
\PB{\|p}, \PB{\|q},
and \PB{\\{qq}} are currently unused. The remaining task is to insert the line
from \PB{\\{rr}} to \PB{\\{pp}}.

The happiest situation occurs when we find that the former angle at \PB{\\{rr}}
is
just the angle of the new triangle, and the vertex preceding \PB{\\{rr}}
coincides
with \PB{\\{pp}}, and the length of \PB{\|r} is zero. This means the current
polygon has been completely filled, and we've made progress!

\Y\B\4\X22:Insert \PB{\\{pp}} at the choice point; split into two polygons if
necessary\X${}\E{}$\6
\&{if} ${}(\\{rr}\MG\|s\E\\{triang}[\|j][\T{0}]){}$\5
${}\{{}$\1\6
${}\|q\K\\{rr}\MG\\{prev};{}$\6
${}\|p\K\|q\MG\\{prev}{}$;\C{ \PB{\|q} is the line between \PB{\|p} and \PB{%
\\{rr}} }\6
${}\|k\K\T{13}*\|q\MG\|s+\T{8}*\|q\MG\|t;{}$\6
\&{if} ${}(\|k\E\\{thresh1}[\|j]){}$\5
${}\{{}$\1\6
\&{if} ${}(\|p\E\\{pp}\MG\\{next}\MG\\{next}){}$\5
${}\{{}$\C{ hurray }\1\6
${}\\{rr}\MG\\{next}\K\\{avail};{}$\6
${}\\{avail}\K\\{pp};{}$\6
${}\\{top}\MM;{}$\6
\&{goto} \\{insert\_done};\6
\4${}\}{}$\2\6
\X23:Connect existing vertex \PB{$\|p\E\\{pp}$} directly to the path following %
\PB{\\{pp}}\X;\6
\&{goto} \\{insert\_almost\_done};\6
\4${}\}{}$\2\6
\&{else}\1\5
\X24:Connect vertex \PB{\|p} to \PB{\\{pp}}, removing node \PB{\\{rr}}\X;\2\6
\4${}\}{}$\2\6
\&{else}\5
${}\{{}$\1\6
${}\\{rr}\MG\|s\MRL{-{\K}}\\{triang}[\|j][\T{0}];{}$\6
${}\\{rr}\MG\\{dir}\MRL{+{\K}}\\{triang}[\|j][\T{0}];{}$\6
${}\|q\K\\{get\_avail}(\,);{}$\6
${}\|q\MG\\{prev}\K\\{rr};{}$\6
${}\\{rr}\MG\\{next}\K\|q;{}$\6
${}\|q\MG\|s\K\\{triang}[\|j][\T{1}];{}$\6
${}\|q\MG\|t\K\\{triang}[\|j][\T{2}];{}$\6
${}\|q\MG\\{next}\K\\{pp};{}$\6
${}\\{pp}\MG\\{prev}\K\|q;{}$\6
${}\|p\K\\{rr};{}$\6
\4${}\}{}$\2\6
\&{for} ${}(\|p\K\|p\MG\\{prev}\MG\\{prev};{}$ ${}\|p\I\\{pp};{}$ ${}\|p\K\|p%
\MG\\{prev}\MG\\{prev}){}$\1\6
\&{if} ${}(\|x[\|p\MG\|t]\E\|x[\\{vert}]){}$\5
${}\{{}$\C{ \PB{\\{pp}} coincides with a previous point }\1\6
\X25:Split off a polygon at position \PB{$\\{pp}\E\|p$}\X;\6
\&{break};\6
\4${}\}{}$\2\2\6
\4\\{insert\_almost\_done}:\5
\X26:Remove angle 0 or 5 at \PB{\|p}, if present after the \PB{\\{pp}}
insertion stage\X;\6
${}\\{poly}[\\{top}]\K\|p;{}$\6
\4\\{insert\_done}:\par
\U15.\fi

\M{23}At this point \PB{$\|q\K\\{rr}\MG\\{prev}$}. We will recycle nodes \PB{%
\|q}, \PB{\\{rr}}, and \PB{\\{pp}}.

\Y\B\4\X23:Connect existing vertex \PB{$\|p\E\\{pp}$} directly to the path
following \PB{\\{pp}}\X${}\E{}$\6
$\|p\MG\\{next}\K\\{pp}\MG\\{next};{}$\6
${}\\{pp}\MG\\{next}\MG\\{prev}\K\|p;{}$\6
${}\|p\MG\|s\MRL{-{\K}}\T{10}-\\{pp}\MG\|s;{}$\6
${}\|p\MG\\{dir}\K\\{pp}\MG\\{dir};{}$\6
${}\\{pp}\MG\\{next}\K\\{avail};{}$\6
${}\\{rr}\MG\\{next}\K\\{pp};{}$\6
${}\\{avail}\K\|q{}$;\par
\U22.\fi

\M{24}\B\X24:Connect vertex \PB{\|p} to \PB{\\{pp}}, removing node \PB{\\{rr}}%
\X${}\E{}$\6
${}\{{}$\1\6
\&{if} ${}(\|k>\\{thresh1}[\|j]){}$\5
${}\{{}$\C{ the line from \PB{\|p} to \PB{\\{rr}} is longer than needed }\1\6
${}\|q\MG\|s\MRL{-{\K}}\\{triang}[\|j][\T{1}];{}$\6
${}\|q\MG\|t\MRL{-{\K}}\\{triang}[\|j][\T{2}];{}$\6
${}\\{pp}\MG\|s\MRL{-{\K}}\T{5};{}$\6
\4${}\}{}$\2\6
\&{else}\5
${}\{{}$\C{ it's shorter than from \PB{\\{rr}} to \PB{\\{pp}} }\1\6
${}\|p\MG\|s\MRL{-{\K}}\T{5}{}$;\C{ we know this is $>0$ }\6
${}\|p\MG\\{dir}\MRL{+{\K}}\T{5};{}$\6
${}\|q\MG\|s\K\\{triang}[\|j][\T{1}]-\|q\MG\|s;{}$\6
${}\|q\MG\|t\K\\{triang}[\|j][\T{2}]-\|q\MG\|t;{}$\6
\4${}\}{}$\2\6
${}\|q\MG\\{next}\K\\{pp};{}$\6
${}\\{pp}\MG\\{prev}\K\|q;{}$\6
${}\\{rr}\MG\\{next}\K\\{avail};{}$\6
${}\\{avail}\K\\{rr};{}$\6
\4${}\}{}$\2\par
\U22.\fi

\M{25}\B\X25:Split off a polygon at position \PB{$\\{pp}\E\|p$}\X${}\E{}$\6
$\\{qq}\K\\{pp}\MG\\{next};{}$\6
\&{if} ${}(\\{qq}\MG\\{next}\E\|p){}$\5
${}\{{}$\C{ remove trivial length-0 leg }\1\6
${}\|p\MG\|s\MRL{+{\K}}\\{pp}\MG\|s-\T{5};{}$\6
${}\|p\MG\\{prev}\K\\{pp}\MG\\{prev};{}$\6
${}\\{pp}\MG\\{prev}\MG\\{next}\K\|p;{}$\6
${}\\{qq}\MG\\{next}\K\\{avail};{}$\6
${}\\{avail}\K\\{pp};{}$\6
\4${}\}{}$\2\6
\&{else}\5
${}\{{}$\1\6
${}\|q\K\|p\MG\\{next};{}$\6
${}\|p\MG\\{next}\K\\{qq};{}$\6
${}\\{qq}\MG\\{prev}\K\|p;{}$\6
${}\\{pp}\MG\\{next}\K\|q;{}$\6
${}\|q\MG\\{prev}\K\\{pp};{}$\6
${}\|k\K\\{pp}\MG\\{dir};{}$\6
${}\\{pp}\MG\\{dir}\K\|p\MG\\{dir};{}$\6
${}\|p\MG\\{dir}\K\|k;{}$\6
${}\|k\K\|p\MG\|s+\\{pp}\MG\|s-\T{10};{}$\6
${}\\{pp}\MG\|s\K(\\{pp}\MG\\{prev}\MG\\{prev}\MG\\{dir}+\T{105}-\\{pp}\MG%
\\{dir})\MOD\T{10};{}$\6
${}\|p\MG\|s\K\|k-\\{pp}\MG\|s{}$;\C{ if negative, we'll catch it }\6
${}\|k\K\\{pp}\MG\|s;{}$\6
\&{if} ${}(\|k\E\T{0}\V\|k\E\T{5}){}$\5
${}\{{}$\1\6
${}\\{pp}\K\\{pp}\MG\\{prev};{}$\6
\&{if} ${}(\|k\E\T{5}){}$\1\5
${}\\{pp}\MG\|s\MRL{+{\K}}\|q\MG\|s,\39\\{pp}\MG\|t\MRL{+{\K}}\|q\MG\|t;{}$\2\6
\&{else} \&{if} ${}(\T{13}*(\\{pp}\MG\|s-\|q\MG\|s)+\T{8}*(\\{pp}\MG\|t-\|q\MG%
\|t)<\T{0}){}$\1\5
${}\\{pp}\MG\|s\K\|q\MG\|s-\\{pp}\MG\|s,\39\\{pp}\MG\|t\K\|q\MG\|t-\\{pp}\MG%
\|t,\39\\{pp}\MG\\{prev}\MG\|s\MRL{-{\K}}\T{5},\39\\{pp}\MG\\{prev}\MG\\{dir}%
\MRL{+{\K}}\T{5};{}$\2\6
\&{else}\1\5
${}\\{pp}\MG\|s\MRL{-{\K}}\|q\MG\|s,\39\\{pp}\MG\|t\MRL{-{\K}}\|q\MG\|t,\39\|q%
\MG\\{next}\MG\|s\MRL{-{\K}}\T{5};{}$\2\6
${}\\{pp}\MG\\{next}\K\|q\MG\\{next};{}$\6
${}\|q\MG\\{next}\MG\\{prev}\K\\{pp};{}$\6
${}\\{pp}\K\|q\MG\\{next};{}$\6
${}\|q\MG\\{next}\K\\{avail};{}$\6
${}\\{avail}\K\|q\MG\\{prev};{}$\6
\4${}\}{}$\2\6
${}\\{poly}[\\{top}\PP]\K\\{pp};{}$\6
\4${}\}{}$\2\par
\U22.\fi

\M{26}I sure wish I had been able to figure out an elegant way to get rid of
so many special cases. Sigh. This, at least, is the last.

The polygon we're left with consists entirely of old vertices, so they
are distinct.

\Y\B\4\X26:Remove angle 0 or 5 at \PB{\|p}, if present after the \PB{\\{pp}}
insertion stage\X${}\E{}$\6
\&{if} ${}(\|p\MG\|s\E\T{0}\V\|p\MG\|s\E\T{5}){}$\5
${}\{{}$\1\6
${}\|q\K\|p\MG\\{next};{}$\6
${}\|r\K\|p\MG\\{prev};{}$\6
\&{if} ${}(\|p\MG\|s\E\T{5}){}$\1\5
${}\|r\MG\|s\MRL{+{\K}}\|q\MG\|s,\39\|r\MG\|t\MRL{+{\K}}\|q\MG\|t;{}$\2\6
\&{else} \&{if} ${}(\T{13}*(\|r\MG\|s-\|q\MG\|s)+\T{8}*(\|r\MG\|t-\|q\MG\|t)\Z%
\T{0}){}$\1\5
${}\|r\MG\|s\K\|q\MG\|s-\|r\MG\|s,\39\|r\MG\|t\K\|q\MG\|t-\|r\MG\|t,\39\|r\MG%
\\{prev}\MG\|s\MRL{-{\K}}\T{5},\39\|r\MG\\{prev}\MG\\{dir}\MRL{+{\K}}\T{5};{}$%
\2\6
\&{else}\1\5
${}\|r\MG\|s\MRL{-{\K}}\|q\MG\|s,\39\|r\MG\|t\MRL{-{\K}}\|q\MG\|t,\39\|q\MG%
\\{next}\MG\|s\MRL{-{\K}}\T{5};{}$\2\6
${}\|r\MG\\{next}\K\|q\MG\\{next};{}$\6
${}\|q\MG\\{next}\MG\\{prev}\K\|r;{}$\6
${}\|q\MG\\{next}\K\\{avail};{}$\6
${}\\{avail}\K\|p;{}$\6
${}\|p\K\|r\MG\\{next};{}$\6
\4${}\}{}$\2\par
\U22.\fi

\M{27}\B\X27:Examine the current choice and its ramifications\X${}\E{}$\6
${}\{{}$\1\6
\&{int} \\{badsums}${}\K\T{0},{}$ \\{negangle}${}\K\T{0};{}$\7
\&{if} (\\{verbose})\1\5
${}\\{printf}(\.{"Level\ \%d:\ vertex\ \%d}\)\.{\ with\ triangle\ \%d\\n"},\39%
\|l,\39\\{choice}[\|l]\MG\|t,\39\\{way}[\|l]);{}$\2\6
\&{for} ${}(\|j\K\\{ht}[\|l];{}$ ${}\|j\Z\\{top};{}$ ${}\|j\PP){}$\5
${}\{{}$\1\6
\&{for} ${}(\|p\K\\{poly}[\|j],\39\|k\K\|p\MG\\{prev}\MG\\{prev}\MG\\{dir},\39%
\|i\K\T{0};{}$  ; \,)\5
${}\{{}$\1\6
${}\|q\K\|p\MG\\{next};{}$\6
\&{if} ${}(\|q\MG\\{prev}\I\|p){}$\1\5
\\{printf}(\.{"\ badlink!"});\2\6
\&{if} (\\{verbose})\1\5
${}\\{printf}(\.{"\ \%d(\%d)"},\39\|p\MG\|t,\39\|p\MG\|s);{}$\2\6
\&{if} ${}(\|p\MG\|s\E\T{0}\V\|p\MG\|s\E\T{5}){}$\1\5
\\{printf}(\.{"\ badangle!"});\2\6
\&{if} ${}(\|p\MG\|s<\T{0}\W\|j\E\\{top}){}$\1\5
${}\\{negangle}\K\T{1};{}$\2\6
\&{if} ${}((\|k+\T{105}-\|p\MG\|s-\|p\MG\\{dir})\MOD\T{10}\I\T{0}){}$\1\5
\\{printf}(\.{"baddir!"});\2\6
${}\|i\MRL{+{\K}}\T{5}-\|p\MG\|s;{}$\6
${}\|k\K\|p\MG\\{dir};{}$\6
${}\|p\K\|q\MG\\{next};{}$\6
\&{if} ${}(\|p\MG\\{prev}\I\|q){}$\1\5
\\{printf}(\.{"\ badlink!"});\2\6
\&{if} (\\{verbose})\1\5
${}\\{printf}(\.{"\ \%d,\%d"},\39\|q\MG\|s,\39\|q\MG\|t);{}$\2\6
\&{if} ${}(\|p\E\\{poly}[\|j]){}$\1\5
\&{break};\2\6
\4${}\}{}$\2\6
\&{if} ${}(\|i\I\T{10}){}$\1\5
${}\\{badsums}\PP;{}$\2\6
\&{if} (\\{verbose})\1\5
\\{printf}(\.{"\\n"});\2\6
\4${}\}{}$\2\6
\&{if} ${}(\\{badsums}\W\R\\{negangle}){}$\1\5
\\{printf}(\.{"\ badsum!\\n"});\2\6
\4${}\}{}$\2\par
\U12.\fi

\M{28}\B\X28:Undo the changes made in level \PB{\|l}\X${}\E{}$\6
\&{for} ${}(\|j\K\\{top};{}$ ${}\|j\G\\{ht}[\|l];{}$ ${}\|j\MM){}$\5
${}\{{}$\1\6
${}\\{poly}[\|j]\MG\\{prev}\MG\\{next}\K\\{avail};{}$\6
${}\\{avail}\K\\{poly}[\|j];{}$\6
\4${}\}{}$\2\6
${}\\{top}\K\\{ht}[\|l];{}$\6
${}\\{poly}[\\{top}]\K\\{save}[\|l]{}$;\par
\U12.\fi

\N{1}{29}Solutions. The terminal gets only a minimum of information
from which a tiling can be constructed.

\fi

\M{30}\B\X30:Record a solution\X${}\E{}$\6
${}\{{}$\1\6
${}\\{count}\PP;{}$\6
\&{if} ${}(\\{count}\MOD\\{interval}\E\T{0}){}$\5
${}\{{}$\1\6
${}\\{printf}(\.{"\%d:"},\39\\{count});{}$\6
\&{for} ${}(\|j\K\T{0};{}$ ${}\|j<\|l;{}$ ${}\|j\PP){}$\1\5
${}\\{printf}(\.{"\ \%d-\%d"},\39\\{choice}[\|j]\MG\|t,\39\\{way}[\|j]);{}$\2\6
\\{printf}(\.{"\\n"});\6
\4${}\}{}$\2\6
\&{if} ${}(\\{eps}\W\\{count}\MOD\\{eps\_interval}\E\T{0}){}$\1\5
\X31:Output a PostScript version\X;\2\6
\4${}\}{}$\2\par
\U12.\fi

\M{31}Here's how we get encapsulated PostScript output for a solution.

\Y\B\4\X31:Output a PostScript version\X${}\E{}$\6
${}\{{}$\1\6
\X32:Open \PB{\\{eps\_file}} for output, and define a triangle subroutine\X;\6
\&{for} ${}(\|j\K\T{0};{}$ ${}\|j<\|l;{}$ ${}\|j\PP){}$\1\5
\X34:Output the triangle for level \PB{\|j}\X;\2\6
\\{fclose}(\\{eps\_file});\6
\4${}\}{}$\2\par
\U30.\fi

\M{32}The PostScript `\.t' subroutine simply draws a triangle between three
given points.

\Y\B\4\X32:Open \PB{\\{eps\_file}} for output, and define a triangle subroutine%
\X${}\E{}$\6
$\\{sprintf}(\\{buffer},\39\.{"\%s.\%d"},\39\\{argv}[\T{0}],\39\\{count});{}$\6
${}\\{eps\_file}\K\\{fopen}(\\{buffer},\39\.{"w"});{}$\6
\&{if} ${}(\R\\{eps\_file}){}$\5
${}\{{}$\1\6
${}\\{printf}(\.{"Can't\ open\ file\ \%s!}\)\.{\\n"},\39\\{buffer});{}$\6
${}\\{exit}({-}\T{4});{}$\6
\4${}\}{}$\2\6
${}\\{fprintf}(\\{eps\_file},\39\.{"\%\%!\\n\%\%\%\%BoundingBo}\)\.{x:\ \%d\ %
\%d\ \%d\ \%d\\n"},\39\\{bbxlo}-\T{1},\39\\{bbylo}-\T{1},\39\\{bbxhi}+\T{1},\39%
\\{bbyhi}+\T{1});{}$\6
${}\\{fprintf}(\\{eps\_file},\39\.{"/t\ \{\ moveto\ lineto\ }\)\.{lineto\
closepath\ str}\)\.{oke\ \}\ bind\ def\\n"}){}$;\par
\U31.\fi

\M{33}\B\X4:Global variables\X${}\mathrel+\E{}$\6
\&{char} \\{buffer}[\T{100}];\C{ output file name (e.g. `\.{decagon.1}') }\6
\&{FILE} ${}{*}\\{eps\_file};{}$\6
\&{int} \\{bbxlo}${},{}$ \\{bbylo}${},{}$ \\{bbxhi}${},{}$ \\{bbyhi};\C{
PostScript bounding box coordinates }\par
\fi

\M{34}\B\X34:Output the triangle for level \PB{\|j}\X${}\E{}$\6
${}\{{}$\1\6
${}\\{print\_coord}(\\{choice}[\|j]\MG\|t);{}$\6
${}\\{print\_coord}(\\{init\_pts}+\|j+\|j);{}$\6
${}\\{print\_coord}(\\{init\_pts}+\T{1}+\|j+\|j);{}$\6
${}\\{fprintf}(\\{eps\_file},\39\.{"\ t\\n"});{}$\6
\4${}\}{}$\2\par
\U31.\fi

\M{35}\B\X3:Subroutines\X${}\mathrel+\E{}$\6
\\{print\_coord}(\|j)\1\1\6
\&{int} \|j;\2\2\6
${}\{{}$\1\6
\&{register} \&{float} \\{xx}${},{}$ \\{yy};\6
\&{register} \&{int} \|k;\6
\&{register} \&{unsigned} \|b;\7
\&{for} ${}(\\{xx}\K\\{yy}\K\T{0.0},\39\|k\K\T{0},\39\|b\K\|x[\|j];{}$ ${}\|k<%
\T{4};{}$ ${}\|k\PP,\39\|b\MRL{{\GG}{\K}}\T{8}){}$\5
${}\{{}$\1\6
${}\\{xx}\MRL{+{\K}}((\&{int})(\|b\AND\T{\^ff})-\T{128})*\\{cos}[\|k];{}$\6
${}\\{yy}\MRL{+{\K}}((\&{int})(\|b\AND\T{\^ff})-\T{128})*\\{sin}[\|k];{}$\6
\4${}\}{}$\2\6
${}\\{fprintf}(\\{eps\_file},\39\.{"\ \%d\ \%d"},\39{}$(\&{int}) \\{xx}${},%
\39{}$(\&{int}) \\{yy});\6
\4${}\}{}$\2\par
\fi

\M{36}\B\D$\\{cos36}$ \5
\T{80.9017}\C{ 100 times $\cos 36^\circ$ }\par
\B\4\D$\\{cos72}$ \5
\T{30.9017}\C{ 100 times $\cos 72^\circ$ }\par
\B\4\D$\\{sin36}$ \5
\T{58.7785}\C{ 100 times $\sin 36^\circ$ }\par
\B\4\D$\\{sin72}$ \5
\T{95.1057}\C{ 100 times $\sin 72^\circ$ }\par
\Y\B\4\X4:Global variables\X${}\mathrel+\E{}$\6
\&{float} \\{cos}[\,]${}\K\{\T{100.0},\39\\{cos36},\39\\{cos72},\39{-}\\{cos72}%
\};{}$\6
\&{float} \\{sin}[\,]${}\K\{\T{0.0},\39\\{sin36},\39\\{sin72},\39\\{sin72}%
\}{}$;\par
\fi

\M{37}\B\X7:Initialize the tables\X${}\mathrel+\E{}$\6
${}\{{}$\1\6
\&{float} \\{xx}${},{}$ \\{yy};\6
\&{unsigned} \|b;\7
${}\\{bbxlo}\K\\{bbylo}\K\T{100000};{}$\6
${}\\{bbxhi}\K\\{bbyhi}\K{-}\T{100000};{}$\6
\&{for} ${}(\|j\K\T{0};{}$ ${}\|j<\\{init\_pts};{}$ ${}\|j\PP){}$\5
${}\{{}$\1\6
\&{for} ${}(\\{xx}\K\\{yy}\K\T{0.0},\39\|k\K\T{0},\39\|b\K\|x[\|j];{}$ ${}\|k<%
\T{4};{}$ ${}\|k\PP,\39\|b\MRL{{\GG}{\K}}\T{8}){}$\5
${}\{{}$\1\6
${}\\{xx}\MRL{+{\K}}((\&{int})(\|b\AND\T{\^ff})-\T{128})*\\{cos}[\|k];{}$\6
${}\\{yy}\MRL{+{\K}}((\&{int})(\|b\AND\T{\^ff})-\T{128})*\\{sin}[\|k];{}$\6
\4${}\}{}$\2\6
\&{if} ((\&{int}) \\{xx}${}<\\{bbxlo}){}$\1\5
${}\\{bbxlo}\K{}$(\&{int}) \\{xx};\2\6
\&{if} ((\&{int}) \\{yy}${}<\\{bbylo}){}$\1\5
${}\\{bbylo}\K{}$(\&{int}) \\{yy};\2\6
\&{if} ((\&{int}) \\{xx}${}>\\{bbxhi}){}$\1\5
${}\\{bbxhi}\K{}$(\&{int}) \\{xx};\2\6
\&{if} ((\&{int}) \\{yy}${}>\\{bbyhi}){}$\1\5
${}\\{bbyhi}\K{}$(\&{int}) \\{yy};\2\6
\4${}\}{}$\2\6
\4${}\}{}$\2\par
\fi

\N{1}{38}Index.
\Y\B\4\X3:Subroutines\X${}\mathrel+\E{}$\6
\\{temp1}(\,)\1\1\2\2\6
${}\{{}$\1\6
\\{printf}(\.{""});\6
\4${}\}{}$\2\7
\\{temp2}(\,)\1\1\2\2\6
${}\{{}$\1\6
\\{printf}(\.{""});\6
\4${}\}{}$\2\par
\fi

\inx
\fin
\con
