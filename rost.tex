\input cwebmac
\hypertextrue\srcloctrue
\datethis


\N[3 rost.w]{1}{1}One-dimensional particle physics. This program is a
quick-and-dirty
implementation of the random process analyzed by Hermann Rost in 1981
(see exercise 5.1.4--40). Start with infinitely many 1s followed by
infinitely many 0s; then randomly interchange adjacent elements that
are out of order.

\Y\B\8\#\&{include} \.{<stdio.h>}\6
\8\#\&{include} \.{<math.h>}\6
\8\#\&{include} \.{"gb\_flip.h"}\6
\&{char} ${}{*}\\{bit};{}$\6
\&{int} ${}{*}\\{list};{}$\6
\&{int} \\{seed};\C{ random number seed }\6
\&{int} \|n;\C{ this many interchanges }\7
${}\\{main}(\\{argc},\39\\{argv}){}$\1\1\6
\&{int} \\{argc};\6
\&{char} ${}{*}\\{argv}[\,];\2\2{}$\6
${}\{{}$\1\6
\&{register} \&{int} \|i${},{}$ \|j${},{}$ \|k${},{}$ \|l${},{}$ \|t${},{}$ %
\|u${},{}$ \|r;\7
\X2:Scan the command line\X;\6
\X3:Initialize everything\X;\6
\&{for} ${}(\|r\K\T{0};{}$ ${}\|r<\|n;{}$ ${}\|r\PP){}$\1\5
\X4:Move\X;\2\6
\X5:Print the results\X;\6
\4${}\}{}$\2\par
\fi

\M[29 rost.w]{2}\B\X2:Scan the command line\X${}\E{}$\6
\&{if} ${}(\\{argc}\I\T{3}\V\\{sscanf}(\\{argv}[\T{1}],\39\.{"\%d"},\39{\AND}%
\|n)\I\T{1}\V\\{sscanf}(\\{argv}[\T{2}],\39\.{"\%d"},\39{\AND}\\{seed})\I%
\T{1}){}$\5
${}\{{}$\1\6
${}\\{fprintf}(\\{stderr},\39\.{"Usage:\ \%s\ n\ seed\ >!}\)\.{\ output.ps%
\\n"},\39\\{argv}[\T{0}]);{}$\6
${}\\{exit}({-}\T{1});{}$\6
\4${}\}{}$\2\par
\U1.\fi

\M[34 rost.w]{3}We maintain the following invariants: \PB{$\\{bit}[\|k]\K%
\T{1}$} for \PB{$\|k\Z\|l$};
\PB{$\\{bit}[\|k]\K\T{0}$} for \PB{$\|k\K\|u$}; the indices \PB{\|i} where %
\PB{$\\{bit}[\|i]>\\{bit}[\|i+\T{1}]$}
are \PB{\\{list}[\|j]} for $0\le j<t$.

\Y\B\4\X3:Initialize everything\X${}\E{}$\6
\\{gb\_init\_rand}(\\{seed});\6
${}\\{bit}\K{}$(\&{char} ${}{*}){}$ \\{malloc}${}(\T{2}*\|n+\T{2});{}$\6
${}\\{list}\K{}$(\&{int} ${}{*}){}$ \\{malloc}${}(\T{4}*\|n+\T{4});{}$\6
\&{for} ${}(\|k\K\T{0};{}$ ${}\|k\Z\|n;{}$ ${}\|k\PP){}$\1\5
${}\\{bit}[\|k]\K\T{1};{}$\2\6
\&{for} ( ; ${}\|k\Z\|n+\|n+\T{1};{}$ ${}\|k\PP){}$\1\5
${}\\{bit}[\|k]\K\T{0};{}$\2\6
${}\|l\K\|u\K\|n;{}$\6
${}\\{list}[\T{0}]\K\|n;{}$\6
${}\|t\K\T{1}{}$;\par
\U1.\fi

\M[48 rost.w]{4}\B\X4:Move\X${}\E{}$\6
${}\{{}$\1\6
${}\|j\K\\{gb\_unif\_rand}(\|t);{}$\6
${}\|i\K\\{list}[\|j];{}$\6
${}\|t\MM;{}$\6
${}\\{list}[\|j]\K\\{list}[\|t];{}$\6
${}\\{bit}[\|i]\K\T{0}{}$;\5
${}\\{bit}[\|i+\T{1}]\K\T{1};{}$\6
\&{if} ${}(\|i\E\|l){}$\1\5
${}\|l\MM;{}$\2\6
\&{if} ${}(\|i\E\|u){}$\1\5
${}\|u\PP;{}$\2\6
\&{if} ${}(\\{bit}[\|i-\T{1}]){}$\1\5
${}\\{list}[\|t\PP]\K\|i-\T{1};{}$\2\6
\&{if} ${}(\R\\{bit}[\|i+\T{2}]){}$\1\5
${}\\{list}[\|t\PP]\K\|i+\T{1};{}$\2\6
\4${}\}{}$\2\par
\U1.\fi

\M[61 rost.w]{5}\B\X5:Print the results\X${}\E{}$\6
\X6:Print the PostScript header info\X;\6
\X8:Print the empirical curve\X;\6
\X9:Print the theoretical curve\X;\6
\X7:Print the PostScript trailer info\X;\par
\U1.\fi

\M[67 rost.w]{6}\B\X6:Print the PostScript header info\X${}\E{}$\6
\\{printf}(\.{"\%\%!PS\\n"});\6
\\{printf}(\.{"\%\%\%\%BoundingBox:\ -1}\)\.{\ -1\ 361\ 361\\n"});\6
${}\\{printf}(\.{"\%\%\%\%Creator:\ \%s\ \%s\ }\)\.{\%s\\n"},\39\\{argv}[%
\T{0}],\39\\{argv}[\T{1}],\39\\{argv}[\T{2}]);{}$\6
\\{printf}(\.{"/d\ \{0\ s\ neg\ rlineto}\)\.{\}\ bind\ def\\n"});\C{ move down
}\6
\\{printf}(\.{"/r\ \{s\ 0\ rlineto\}\ bi}\)\.{nd\ def\\n"});\C{ move right }\par
\U5.\fi

\M[74 rost.w]{7}\B\X7:Print the PostScript trailer info\X${}\E{}$\6
\\{printf}(\.{"showpage\\n"});\par
\U5.\fi

\M[77 rost.w]{8}The empirical curve is scaled so that $\sqrt{6n}$ units is 5
inches.

\Y\B\4\X8:Print the empirical curve\X${}\E{}$\6
$\\{printf}(\.{"/s\ \%g\ def\\n"},\39\T{360.0}/\\{sqrt}(\T{6.0}*\|n));{}$\6
${}\\{printf}(\.{"newpath\ \%d\ \%d\ s\ mul}\)\.{\ moveto\\n"},\39\T{0},\39\|n-%
\|l);{}$\6
\&{for} ${}(\|k\K\|l+\T{1};{}$ ${}\|k\Z\|u;{}$ ${}\|k\PP){}$\5
${}\{{}$\1\6
\&{if} (\\{bit}[\|k])\1\5
\\{printf}(\.{"\ d"});\5
\2\&{else}\1\5
\\{printf}(\.{"\ r"});\2\6
\&{if} ${}((\|k-\|l)\MOD\T{40}\E\T{0}){}$\1\5
\\{printf}(\.{"\\n"});\2\6
\4${}\}{}$\2\6
\\{printf}(\.{"\\n0\ 0\ lineto\ closep}\)\.{ath\\n"});\6
\\{printf}(\.{"1\ setlinewidth\ stro}\)\.{ke\\n"});\par
\U5.\fi

\M[89 rost.w]{9}The theoretical curve $\sqrt{\mathstrut x}+\sqrt{\mathstrut
y}=1$ is
scaled so that 1 unit is 5 inches. We use the fact that this curve
is {\it exactly\/} drawn by PostScript's Bezier curve routines,
from the control points $(0,1)$, $(0,1/3)$, $(1/3,0)$, $(1,0)$.

\Y\B\4\X9:Print the theoretical curve\X${}\E{}$\6
\\{printf}(\.{"newpath\ 0\ 360\ movet}\)\.{o\ 0\ 120\ 120\ 0\ 360\ 0\ }\)%
\.{curveto\\n"});\6
\\{printf}(\.{"\ 0\ 0\ lineto\ closepa}\)\.{th\\n"});\6
\\{printf}(\.{".3\ setlinewidth\ str}\)\.{oke\\n"});\par
\U5.\fi

\N[99 rost.w]{1}{10}Index.



\fi


\inx
\fin
\con
