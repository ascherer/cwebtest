\input cwebmac
\datethis

\N{1}{1}Intro. This program is a transcription of the code that I wrote in
1969 to determine the length $l(n)$ of a shortest addition chain for~$n$.
My original program was written in {\mc IMP}, an idiosyncratic language
that was basically an assembly program for the Control Data 6600.
It computed $l(n)$ for $n\le18269$, and it consumed an unknown
but probably nontrivial amount of background time on the computer
over a period of several weeks.
I decided to see how efficient that program really was, by recoding
it for a modern machine.

Better techniques for that problem are known by now, of course.
I think I can also make the original method go quite a bit faster,
by changing the data structures.
But I'll never know how much speedup is achieved by any of the
newer approaches until I get the old algorithm running again.

The command line should have two parameters, which name an input file
and an output file. Both files contain values of $l(1)$, $l(2)$, \dots,
with one byte per value, using visible {\mc ASCII} characters by
adding \PB{\.{'\ '}} to each integer value. The numbers in the input file
need not be exact, but they must be valid lower bounds; if the
input file contains fewer than $n$ bytes, this program uses
the simple lower bound of Theorem 4.6.3C. The
output file gets answers one byte at a time, and I expect to
``kill'' the program manually before it finishes.

By the way, it's fun to look at the output file with a text editor.

\Y\B\4\D$\\{nmax}$ \5
\T{10000000}\par
\Y\B\8\#\&{include} \.{<stdio.h>}\6
\8\#\&{include} \.{<stdlib.h>}\6
\8\#\&{include} \.{<time.h>}\6
\&{char} \|l[\\{nmax}];\6
\&{int} \|a[\T{129}];\6
\&{struct} ${}\{{}$\1\6
\&{int} \\{lbp}${},{}$ \\{ubp}${},{}$ \\{lbq}${},{}$ \\{ubq}${},{}$ \\{savep};%
\2\6
${}\}{}$ \\{stack}[\T{128}];\6
\&{FILE} ${}{*}\\{infile},{}$ ${}{*}\\{outfile};{}$\6
\&{int} \\{prime}[\T{1000}];\C{ 1000 primes will take us past 60 million }\6
\&{int} \\{pr};\C{ the number of primes known so far }\6
\&{char} \|x[\T{64}];\C{ exponents of the binary representation of $n$ }\7
\&{int} \\{main}(\&{int} \\{argc}${},\39{}$\&{char} ${}{*}\\{argv}[\,]){}$\1\1%
\2\2\6
${}\{{}$\1\6
\&{register} \&{int} \|i${},{}$ \|j${},{}$ \|p${},{}$ \|q${},{}$ \|n${},{}$ %
\|s${},{}$ \\{ubp}${},{}$ \\{ubq}${},{}$ \\{lbp}${},{}$ \\{lbq};\6
\&{int} \\{lb}${},{}$ \\{ub}${},{}$ \\{timer}${}\K\T{0};{}$\7
\X2:Process the command line\X;\6
${}\\{prime}[\T{0}]\K\T{2},\39\\{pr}\K\T{1};{}$\6
${}\|a[\T{0}]\K\T{1},\39\|a[\T{1}]\K\T{2}{}$;\C{ an addition chain always
begins like this }\6
\&{for} ${}(\|n\K\T{1};{}$ ${}\|n<\\{nmax};{}$ ${}\|n\PP){}$\5
${}\{{}$\1\6
\X4:Input the next lower bound, \PB{\\{lb}}\X;\6
\X5:Find an upper bound; or in simple cases, set $l(n)$ and \PB{\&{goto} %
\\{done}}\X;\6
\X7:Backtrack until $l(n)$ is known\X;\6
\4\\{done}:\5
\X3:Output the value of $l(n)$\X;\6
\&{if} ${}(\|n\MOD\T{1000}\E\T{0}){}$\5
${}\{{}$\1\6
${}\|j\K\\{clock}(\,);{}$\6
${}\\{printf}(\.{"\%d..\%d\ done\ in\ \%.5g}\)\.{\ minutes\\n"},\39\|n-\T{999},%
\39\|n,\39(\&{double})(\|j-\\{timer})/(\T{60}*\.{CLOCKS\_PER\_SEC}));{}$\6
${}\\{timer}\K\|j;{}$\6
\4${}\}{}$\2\6
\4${}\}{}$\2\6
\4${}\}{}$\2\par
\fi

\M{2}\B\X2:Process the command line\X${}\E{}$\6
\&{if} ${}(\\{argc}\I\T{3}){}$\5
${}\{{}$\1\6
${}\\{fprintf}(\\{stderr},\39\.{"Usage:\ \%s\ infile\ ou}\)\.{tfile\\n"},\39%
\\{argv}[\T{0}]);{}$\6
${}\\{exit}({-}\T{1});{}$\6
\4${}\}{}$\2\6
${}\\{infile}\K\\{fopen}(\\{argv}[\T{1}],\39\.{"r"});{}$\6
\&{if} ${}(\R\\{infile}){}$\5
${}\{{}$\1\6
${}\\{fprintf}(\\{stderr},\39\.{"I\ couldn't\ open\ `\%s}\)\.{'\ for\ reading!%
\\n"},\39\\{argv}[\T{1}]);{}$\6
${}\\{exit}({-}\T{2});{}$\6
\4${}\}{}$\2\6
${}\\{outfile}\K\\{fopen}(\\{argv}[\T{2}],\39\.{"w"});{}$\6
\&{if} ${}(\R\\{outfile}){}$\5
${}\{{}$\1\6
${}\\{fprintf}(\\{stderr},\39\.{"I\ couldn't\ open\ `\%s}\)\.{'\ for\ writing!%
\\n"},\39\\{argv}[\T{2}]);{}$\6
${}\\{exit}({-}\T{3});{}$\6
\4${}\}{}$\2\par
\U1.\fi

\M{3}\B\X3:Output the value of $l(n)$\X${}\E{}$\6
$\\{fprintf}(\\{outfile},\39\.{"\%c"},\39\|l[\|n]+\.{'\ '});{}$\6
\\{fflush}(\\{outfile});\C{ make sure the result is viewable immediately }\par
\U1.\fi

\M{4}At this point I compute the ``lower bound'' $\lfloor\lg n\rfloor+3$,
which is valid if $\nu n>4$. Simple cases where $\nu n\le 4$ will be
handled separately below.

\Y\B\4\X4:Input the next lower bound, \PB{\\{lb}}\X${}\E{}$\6
\&{for} ${}(\|q\K\|n,\39\|i\K{-}\T{1},\39\|j\K\T{0};{}$ \|q; ${}\|q\MRL{{\GG}{%
\K}}\T{1},\39\|i\PP){}$\1\6
\&{if} ${}(\|q\AND\T{1}){}$\1\5
${}\|x[\|j\PP]\K\|i{}$;\C{ now $i=\lfloor\lg n\rfloor$ and $j=\nu n$ }\2\2\6
${}\\{lb}\K\\{fgetc}(\\{infile})-\.{'\ '}{}$;\C{ \PB{\\{fgetc}} will return a
negative value after EOF }\6
\&{if} ${}(\\{lb}<\|i+\T{3}){}$\1\5
${}\\{lb}\K\|i+\T{3}{}$;\2\par
\U1.\fi

\M{5}Three elementary and well-known upper bounds are considered:
(i)~$l(n)\le\lfloor\lg n\rfloor+\nu n-1$;
(ii)~$l(n)\le l(n-1)+1$;
(iii)~$l(n)\le l(p)+l(q)$ if $n=pq$.

Furthermore, there are four special cases
when Theorem 4.6.3C tells us we can save a step.
In this regard, I had to learn (the hard way) to avoid a
curious bug: Three of the four cases in Theorem 4.6.3C arise
when we factor $n$, so I thought I needed to test only the
other case here. But I got a surprise when $n=165$: Then
$n=3\cdot55$, so the factor method gave the upper bound
$l(3)+l(55)=10$; but another factorization, $n=5\cdot33$,
gives the better bound $l(5)+l(33)=9$.

\Y\B\4\X5:Find an upper bound; or in simple cases, set $l(n)$ and \PB{\&{goto} %
\\{done}}\X${}\E{}$\6
$\\{ub}\K\|i+\|j-\T{1};{}$\6
\&{if} ${}(\\{ub}>\|l[\|n-\T{1}]+\T{1}){}$\1\5
${}\\{ub}\K\|l[\|n-\T{1}]+\T{1};{}$\2\6
\X6:Try reducing \PB{\\{ub}} with the factor method\X;\6
${}\|l[\|n]\K\\{ub};{}$\6
\&{if} ${}(\|j\Z\T{3}){}$\1\5
\&{goto} \\{done};\2\6
\&{if} ${}(\|j\E\T{4}){}$\5
${}\{{}$\1\6
${}\|p\K\|x[\T{3}]-\|x[\T{2}],\39\|q\K\|x[\T{1}]-\|x[\T{0}];{}$\6
\&{if} ${}(\|p\E\|q\V\|p\E\|q+\T{1}\V(\|q\E\T{1}\W(\|p\E\T{3}\V(\|p\E\T{5}\W%
\|x[\T{2}]\E\|x[\T{1}]+\T{1})))){}$\1\5
${}\|l[\|n]\K\|i+\T{2};{}$\2\6
\&{goto} \\{done};\6
\4${}\}{}$\2\par
\U1.\fi

\M{6}It's important to try the factor method even when \PB{$\|j\Z\T{4}$},
because
of the way prime numbers are recognized here: We would miss the
prime~3, for example.

On the other hand, we don't need to remember large primes that will
never arise as factors of any future~$n$.

\Y\B\4\X6:Try reducing \PB{\\{ub}} with the factor method\X${}\E{}$\6
\&{if} ${}(\|n>\T{2}){}$\1\6
\&{for} ${}(\|s\K\T{0};{}$  ; ${}\|s\PP){}$\5
${}\{{}$\1\6
${}\|p\K\\{prime}[\|s];{}$\6
${}\|q\K\|n/\|p;{}$\6
\&{if} ${}(\|n\E\|p*\|q){}$\5
${}\{{}$\1\6
\&{if} ${}(\|l[\|p]+\|l[\|q]<\\{ub}){}$\1\5
${}\\{ub}\K\|l[\|p]+\|l[\|q];{}$\2\6
\&{break};\6
\4${}\}{}$\2\6
\&{if} ${}(\|q\Z\|p){}$\5
${}\{{}$\C{ $n$ is prime }\1\6
\&{if} ${}(\\{pr}<\T{1000}){}$\1\5
${}\\{prime}[\\{pr}\PP]\K\|n;{}$\2\6
\&{break};\6
\4${}\}{}$\2\6
\4${}\}{}$\2\2\par
\U5.\fi

\N{1}{7}The interesting part. All the above was necessary just to get
warmed up and to set the groundwork for nontrivial cases.
The method adopted is simple, but it has some subtleties that
I discovered one by one in the 60s.

If \PB{$\\{lb}<\\{ub}$}, we will try to build an addition chain of length \PB{$%
\\{ub}-\T{1}$}.
If that succeeds, we decrease \PB{\\{ub}} and try again. Finally we will
have established the fact that $l(n)=\PB{\\{ub}}$.

\Y\B\4\X7:Backtrack until $l(n)$ is known\X${}\E{}$\6
\&{while} ${}(\\{lb}<\\{ub}){}$\5
${}\{{}$\1\6
${}\|a[\\{ub}-\T{1}]\K\|n;{}$\6
\&{for} ${}(\|i\K\T{2};{}$ ${}\|i<\\{ub}-\T{1};{}$ ${}\|i\PP){}$\1\5
${}\|a[\|i]\K\T{0};{}$\2\6
\X8:Try to fill in the rest of the chain; \PB{\&{goto} \\{done}} if it's
impossible\X;\6
${}\|l[\|n]\K\MM\\{ub};{}$\6
\4${}\}{}$\2\par
\U1.\fi

\M{8}We maintain a stack of subproblems, as usual when backtracking.
Suppose \PB{\|a[\|t]} is the sum of two items already present, for all
$t>s$; we want to make sure that \PB{\|a[\|s]} is legitimate too.
For this purpose we try all combinations \PB{$\|a[\|s]\K\|p+\|q$} where
$p\ge a[s]/2$, trying to make both $p$ and $q$ present.

Two key methods are used to prune down the number of possibilities
explored. First, the number $p$ can't be inserted into $a[t]$ when $t<l(p)$.
Second, two consecutive nonzero entries of an addition chain must
satisfy $a_t<a_{t+1}\le 2a_t$.

Suppose, for example, that we have a partial solution with $a_{10}=100$
and $a_{13}=500$, but $a_{11}$ and $a_{12}$ still are zero (meaning that
they haven't been filled in). Then we can't set $a_{11}=124$, because
that would force $a_{12}$ to be at most $248$, and $a_{13}$ would be
unsupportable. It follows that $a_{11}$ must lie between 125 and 200.

Now suppose that $a_{10}=100$ and $a_{13}=200$, while $a_{11}$ and
$a_{12}$ are still vacant. In this case $a_{11}$ can be any value
from 101 to 199, {\it including\/} 199 (because we cannot be
sure that the chain will require $a_{12}$). That's what I meant
by a subtlety.

In general, for each value of~$s$, we have three nested loops: one on
the value of~$p$ described above, one on the different places
where $p$ might go into the chain, and one on the different places
where $q$ might go into the chain.

The following program is organized outside-in. I could also have
written it inside-out; but either way (as I tried to explain in my old paper
on \PB{\&{goto}} statements) there seems to be a need for jumping
into a loop.

My program from 1969 didn't have the statement
`\PB{ \&{if} ${}(\\{ubp}\E\|s-\T{1}\W\\{lbp}<\\{ubp}){}$ ${}\\{lbp}\K%
\\{ubp};{}$}'
but that improvement is easily justified. For if there's no solution
with \PB{\|p} in position \PB{$\|a[\|s-\T{1}]$}, there won't be one with \PB{%
\|p} placed
even earlier.

[Hmm: My ``subtle'' argument above doesn't {\it really\/} need to
allow $a_{11}=199$.]

\Y\B\4\X8:Try to fill in the rest of the chain; \PB{\&{goto} \\{done}} if it's
impossible\X${}\E{}$\6
\&{for} ${}(\|s\K\\{ub}-\T{1};{}$ ${}\|s>\T{1};{}$ ${}\|s\MM){}$\1\6
\&{if} (\|a[\|s])\5
${}\{{}$\1\6
\&{for} ${}(\|q\K\|a[\|s]\GG\T{1},\39\|p\K\|a[\|s]-\|q;{}$ \|q; ${}\|p\PP,\39%
\|q\MM){}$\5
${}\{{}$\1\6
\X9:Find bounds \PB{\\{lbp}} and \PB{\\{ubp}} on where \PB{\|p} can be
inserted; \PB{\&{goto} \\{tryq}} if \PB{\|p} is already present\X;\6
\&{if} ${}(\\{ubp}\E\|s-\T{1}\W\\{lbp}<\\{ubp}){}$\1\5
${}\\{lbp}\K\\{ubp};{}$\2\6
\&{for} ( ; ${}\\{ubp}\G\\{lbp};{}$ ${}\\{ubp}\MM){}$\5
${}\{{}$\1\6
${}\|a[\\{ubp}]\K\|p;{}$\6
\4\\{tryq}:\5
\X10:Find bounds \PB{\\{lbq}} and \PB{\\{ubq}} on where \PB{\|q} can be
inserted; \PB{\&{goto} \\{happiness}} if \PB{\|q} is already present\X;\6
\&{for} ( ; ${}\\{ubq}\G\\{lbq};{}$ ${}\\{ubq}\MM){}$\5
${}\{{}$\1\6
${}\|a[\\{ubq}]\K\|q;{}$\6
\4\\{happiness}:\5
${}\\{stack}[\|s].\\{savep}\K\|p;{}$\6
${}\\{stack}[\|s].\\{lbp}\K\\{lbp},\39\\{stack}[\|s].\\{ubp}\K\\{ubp};{}$\6
${}\\{stack}[\|s].\\{lbq}\K\\{lbq},\39\\{stack}[\|s].\\{ubq}\K\\{ubq};{}$\6
\&{goto} \\{onward};\C{ now \PB{\|a[\|s]} is covered; try to fill in \PB{$\|a[%
\|s-\T{1}]$} }\6
\4\\{backup}:\5
${}\|s\PP;{}$\6
\&{if} ${}(\|s\E\\{ub}){}$\1\5
\&{goto} \\{done};\2\6
\&{if} ${}(\|a[\|s]\E\T{0}){}$\1\5
\&{goto} \\{backup};\2\6
${}\\{lbq}\K\\{stack}[\|s].\\{lbq},\39\\{ubq}\K\\{stack}[\|s].\\{ubq};{}$\6
${}\\{lbp}\K\\{stack}[\|s].\\{lbp},\39\\{ubp}\K\\{stack}[\|s].\\{ubp};{}$\6
${}\|p\K\\{stack}[\|s].\\{savep},\39\|q\K\|a[\|s]-\|p;{}$\6
${}\|a[\\{ubq}]\K\T{0};{}$\6
\4${}\}{}$\2\6
${}\|a[\\{ubp}]\K\T{0};{}$\6
\4${}\}{}$\2\6
\4${}\}{}$\2\6
\&{goto} \\{backup};\6
\4\\{onward}:\5
\&{continue};\6
\4${}\}{}$\2\2\par
\U7.\fi

\M{9}The heart of the computation is the following routine, which decides
where insertion is possible. A tedious case analysis seems necessary.
We set \PB{\\{ubp}} to a harmless value so that the subsequent statement
\PB{$\|a[\\{ubp}]\K\T{0}$} doesn't remove \PB{\|p} if \PB{\|p} was already
present.

\Y\B\4\D$\\{harmless}$ \5
\T{128}\par
\Y\B\4\X9:Find bounds \PB{\\{lbp}} and \PB{\\{ubp}} on where \PB{\|p} can be
inserted; \PB{\&{goto} \\{tryq}} if \PB{\|p} is already present\X${}\E{}$\6
$\\{lbp}\K\|l[\|p];{}$\6
\&{if} ${}(\\{lbp}\Z\T{1}){}$\1\5
\&{goto} \\{p\_ready};\C{ if \PB{\|p} is 1 or 2, it's already there }\2\6
\&{if} ${}(\\{lbp}\G\\{ub}){}$\1\5
\&{goto} \\{p\_hopeless};\2\6
\4\\{p\_search}:\5
\&{while} ${}(\|a[\\{lbp}]<\|p){}$\5
${}\{{}$\1\6
\&{if} ${}(\|a[\\{lbp}]\E\T{0}){}$\1\5
\&{goto} \\{p\_empty\_slot};\2\6
${}\\{lbp}\PP;{}$\6
\4${}\}{}$\2\6
\&{if} ${}(\|a[\\{lbp}]\E\|p){}$\1\5
\&{goto} \\{p\_ready};\2\6
\4\\{p\_hopeless}:\5
${}\\{ubp}\K\\{lbp}-\T{1}{}$;\5
\&{goto} \\{p\_done};\C{ no way }\6
\4\\{p\_empty\_slot}:\5
\&{for} ${}(\|j\K\\{lbp}-\T{1};{}$ ${}\|a[\|j]\E\T{0};{}$ ${}\|j\MM){}$\1\5
;\2\6
${}\|i\K\|a[\|j];{}$\6
\&{if} ${}(\|p<\|i){}$\1\5
\&{goto} \\{p\_hopeless};\2\6
\&{for} ${}(\|i\MRL{+{\K}}\|i,\39\|j\PP;{}$ ${}\|j<\\{lbp};{}$ ${}\|j\PP){}$\1\5
${}\|i\MRL{+{\K}}\|i;{}$\2\6
\&{while} ${}(\|p>\|i){}$\5
${}\{{}$\1\6
${}\\{lbp}\PP;{}$\6
\&{if} (\|a[\\{lbp}])\1\5
\&{goto} \\{p\_search};\2\6
${}\|i\MRL{+{\K}}\|i;{}$\6
\4${}\}{}$\2\6
\&{for} ${}(\|j\K\\{lbp}+\T{1};{}$ ${}\|a[\|j]\E\T{0};{}$ ${}\|j\PP){}$\1\5
;\2\6
${}\|i\K\|a[\|j];{}$\6
\&{if} ${}(\|i\Z\|p){}$\5
${}\{{}$\1\6
\&{if} ${}(\|i<\|p){}$\5
${}\{{}$\1\6
${}\\{lbp}\K\|j+\T{1}{}$;\5
\&{goto} \\{p\_search};\6
\4${}\}{}$\2\6
\4\\{p\_ready}:\5
${}\\{ubp}\K\\{lbp}\K\\{harmless};{}$\6
\&{goto} \\{tryq};\C{ we found \PB{\|p} }\6
\4${}\}{}$\2\6
\&{for} ${}(\\{ubp}\K\|j-\T{1},\39\|i\K(\|i+\T{1})\GG\T{1};{}$ ${}\|p<\|i;{}$
${}\\{ubp}\MM){}$\1\5
${}\|i\K(\|i+\T{1})\GG\T{1};{}$\2\6
\4\\{p\_done}:\par
\U8.\fi

\M{10}The other case is essentially the same. So if I have a bug in
one routine, it probably is present in the other one too.

\Y\B\4\X10:Find bounds \PB{\\{lbq}} and \PB{\\{ubq}} on where \PB{\|q} can be
inserted; \PB{\&{goto} \\{happiness}} if \PB{\|q} is already present\X${}\E{}$\6
$\\{lbq}\K\|l[\|q];{}$\6
\&{if} ${}(\\{lbq}\G\\{ub}){}$\1\5
\&{goto} \\{q\_hopeless};\2\6
\&{if} ${}(\\{lbq}\Z\T{1}){}$\1\5
\&{goto} \\{q\_ready};\C{ if \PB{\|q} is 1 or 2, it's already there }\2\6
\4\\{q\_search}:\5
\&{while} ${}(\|a[\\{lbq}]<\|q){}$\5
${}\{{}$\1\6
\&{if} ${}(\|a[\\{lbq}]\E\T{0}){}$\1\5
\&{goto} \\{q\_empty\_slot};\2\6
${}\\{lbq}\PP;{}$\6
\4${}\}{}$\2\6
\&{if} ${}(\|a[\\{lbq}]\E\|q){}$\1\5
\&{goto} \\{q\_ready};\2\6
\4\\{q\_hopeless}:\5
${}\\{ubq}\K\\{lbq}-\T{1}{}$;\5
\&{goto} \\{q\_done};\C{ no way }\6
\4\\{q\_empty\_slot}:\5
\&{for} ${}(\|j\K\\{lbq}-\T{1};{}$ ${}\|a[\|j]\E\T{0};{}$ ${}\|j\MM){}$\1\5
;\2\6
${}\|i\K\|a[\|j];{}$\6
\&{if} ${}(\|q<\|i){}$\1\5
\&{goto} \\{q\_hopeless};\2\6
\&{for} ${}(\|i\MRL{+{\K}}\|i,\39\|j\PP;{}$ ${}\|j<\\{lbq};{}$ ${}\|j\PP){}$\1\5
${}\|i\MRL{+{\K}}\|i;{}$\2\6
\&{while} ${}(\|q>\|i){}$\5
${}\{{}$\1\6
${}\\{lbq}\PP;{}$\6
\&{if} (\|a[\\{lbq}])\1\5
\&{goto} \\{q\_search};\2\6
${}\|i\MRL{+{\K}}\|i;{}$\6
\4${}\}{}$\2\6
\&{for} ${}(\|j\K\\{lbq}+\T{1};{}$ ${}\|a[\|j]\E\T{0};{}$ ${}\|j\PP){}$\1\5
;\2\6
${}\|i\K\|a[\|j];{}$\6
\&{if} ${}(\|i\Z\|q){}$\5
${}\{{}$\1\6
\&{if} ${}(\|i<\|q){}$\5
${}\{{}$\1\6
${}\\{lbq}\K\|j+\T{1}{}$;\5
\&{goto} \\{q\_search};\6
\4${}\}{}$\2\6
\4\\{q\_ready}:\5
${}\\{ubq}\K\\{lbq}\K\\{harmless};{}$\6
\&{goto} \\{happiness};\C{ we found \PB{\|p} }\6
\4${}\}{}$\2\6
\&{for} ${}(\\{ubq}\K\|j-\T{1},\39\|i\K(\|i+\T{1})\GG\T{1};{}$ ${}\|q<\|i;{}$
${}\\{ubq}\MM){}$\1\5
${}\|i\K(\|i+\T{1})\GG\T{1};{}$\2\6
\4\\{q\_done}:\par
\U8.\fi

\N{1}{11}Index.


\fi


\inx
\fin
\con
