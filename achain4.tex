\input cwebmac
\datethis

\N{1}{1}Intro. This program is a revision of {\mc ACHAIN2} ({\it not\/} {\mc
ACHAIN3}), which you should read first. After I found an unpublished preprint
by D. Bleichenbacher and A.~Flammenkamp (1997) on the web, I realized that
several changes would speed that program up significantly.

The main changes here are:
(i)~Links are maintained so that it's easy to skip past cases
with large~\PB{\|l[\|p]}.
(ii)~When an odd number $p$ is inserted into the chain at position $j$,
we make sure that $p\le\min(b[j-2]+b[j-1],b[j])$. Previously the weaker test
$p\le b[j]$ was used.
(iii)~Whenever we find a good chain for~$n$, we update the upper bounds for
larger numbers, in case this chain implies a better way to compute them
than was previously known. (The factor method, used previously, is
just a special case of this technique.)

One change I intentionally did {\it not\/} make: When trying
to make \PB{\|a[\|s]} be equal to \PB{$\|p+\|q$} for some previous values \PB{%
\|p} and~\PB{\|q},
Flammenkamp and Bleichenbacher check to see whether appropriate
\PB{\|p} and~\PB{\|q} are already present; if so, they accept \PB{\|a[\|s]} and
move on. Very plausible. But they don't implement a strong equivalence
test with canonical chains, as I do; and I have not been able to
verify that their ``move on'' heuristic is justifiable together with
the strong cutoffs in my canonical approach, because of subtle
ambiguities that arise in special cases.

\Y\B\4\D$\\{nmax}$ \5
\T{10000000}\C{ should be less than $2^{24}$ on a 32-bit machine }\par
\Y\B\8\#\&{include} \.{<stdio.h>}\6
\8\#\&{include} \.{<stdlib.h>}\6
\8\#\&{include} \.{<time.h>}\6
\&{unsigned} \&{char} ${}\|l[\\{nmax}+\T{2}];{}$\6
\&{int} \|a[\T{128}]${},{}$ \|b[\T{128}];\6
\&{unsigned} \&{int} ${}\\{undo}[\T{128}*\T{128}];{}$\6
\&{int} \\{ptr};\C{ this many items of the \PB{\\{undo}} stack are in use }\6
\&{struct} ${}\{{}$\1\6
\&{int} \\{lbp}${},{}$ \\{ubp}${},{}$ \\{lbq}${},{}$ \\{ubq}${},{}$ \|r${},{}$ %
\\{ptrp}${},{}$ \\{ptrq};\2\6
${}\}{}$ \\{stack}[\T{128}];\6
\&{int} \\{tail}[\T{128}]${},{}$ \\{outdeg}[\T{128}]${},{}$ \\{outsum}[%
\T{128}]${},{}$ \\{limit}[\T{128}];\6
\&{FILE} ${}{*}\\{infile},{}$ ${}{*}\\{outfile};{}$\6
\&{int} \\{down}[\\{nmax}];\C{ a navigation aid discussed below }\6
\&{int} \\{link}[\\{nmax}];\C{ stack links when propagating upper bounds }\6
\&{int} \\{top};\C{ top element of the stack (or 1 when empty) }\7
\&{int} \\{main}(\&{int} \\{argc}${},\39{}$\&{char} ${}{*}\\{argv}[\,]){}$\1\1%
\2\2\6
${}\{{}$\1\6
\&{register} \&{int} \|i${},{}$ \|j${},{}$ \|n${},{}$ \|p${},{}$ \|q${},{}$ %
\|r${},{}$ \|s${},{}$ \\{ubp}${},{}$ \\{ubq}${}\K\T{0},{}$ \\{lbp}${},{}$ %
\\{lbq}${},{}$ \\{ptrp}${},{}$ \\{ptrq};\6
\&{int} \\{lb}${},{}$ \\{ub}${},{}$ \\{timer}${}\K\T{0};{}$\7
\X2:Process the command line\X;\6
${}\|a[\T{0}]\K\|b[\T{0}]\K\T{1},\39\|a[\T{1}]\K\|b[\T{1}]\K\T{2}{}$;\C{ an
addition chain always begins like this }\6
\X5:Initialize the \PB{\|l} and \PB{\\{down}} tables\X;\6
\&{for} ${}(\|n\K\T{1};{}$ ${}\|n<\\{nmax};{}$ ${}\|n\PP){}$\5
${}\{{}$\1\6
\X4:Input the next lower bound, \PB{\\{lb}}\X;\6
\X8:Backtrack until $l(n)$ is known\X;\6
\&{if} ${}(\|n\MOD\T{1000}\E\T{0}){}$\5
${}\{{}$\1\6
${}\|j\K\\{clock}(\,);{}$\6
${}\\{printf}(\.{"\%d..\%d\ done\ in\ \%.5g}\)\.{\ minutes\\n"},\39\|n-\T{999},%
\39\|n,\39(\&{double})(\|j-\\{timer})/(\T{60}*\.{CLOCKS\_PER\_SEC}));{}$\6
${}\\{timer}\K\|j;{}$\6
\4${}\}{}$\2\6
\4\\{done}:\5
\X11:Update the \PB{\\{down}} links\X;\6
\X3:Output the value of $l(n)$\X;\6
\4${}\}{}$\2\6
\4${}\}{}$\2\par
\fi

\M{2}\B\X2:Process the command line\X${}\E{}$\6
\&{if} ${}(\\{argc}\I\T{3}){}$\5
${}\{{}$\1\6
${}\\{fprintf}(\\{stderr},\39\.{"Usage:\ \%s\ infile\ ou}\)\.{tfile\\n"},\39%
\\{argv}[\T{0}]);{}$\6
${}\\{exit}({-}\T{1});{}$\6
\4${}\}{}$\2\6
${}\\{infile}\K\\{fopen}(\\{argv}[\T{1}],\39\.{"r"});{}$\6
\&{if} ${}(\R\\{infile}){}$\5
${}\{{}$\1\6
${}\\{fprintf}(\\{stderr},\39\.{"I\ couldn't\ open\ `\%s}\)\.{'\ for\ reading!%
\\n"},\39\\{argv}[\T{1}]);{}$\6
${}\\{exit}({-}\T{2});{}$\6
\4${}\}{}$\2\6
${}\\{outfile}\K\\{fopen}(\\{argv}[\T{2}],\39\.{"w"});{}$\6
\&{if} ${}(\R\\{outfile}){}$\5
${}\{{}$\1\6
${}\\{fprintf}(\\{stderr},\39\.{"I\ couldn't\ open\ `\%s}\)\.{'\ for\ writing!%
\\n"},\39\\{argv}[\T{2}]);{}$\6
${}\\{exit}({-}\T{3});{}$\6
\4${}\}{}$\2\par
\U1.\fi

\M{3}\B\X3:Output the value of $l(n)$\X${}\E{}$\6
$\\{fprintf}(\\{outfile},\39\.{"\%c"},\39\|l[\|n]+\.{'\ '});{}$\6
\\{fflush}(\\{outfile});\C{ make sure the result is viewable immediately }\par
\U1.\fi

\M{4}At this point I compute the known lower bound
$\lfloor\lg n\rfloor+\lceil\lg\min(\nu n,16)\rceil$.

(With a change file, I could make the program set \PB{$\|l[\|n]\K\\{lb}$} and %
\PB{\&{goto} \\{done}} if
the input value is \PB{\.{'\ '}} or more. This change would amount to
believing that the input file has true values, thereby essentially
restarting a computation that was only partly finished.
Hmm, wait: Actually I should also use the factor method to
update upper bounds, before going to \PB{\\{done}}, if \PB{\|l[\|n]} has
changed.)

\Y\B\4\X4:Input the next lower bound, \PB{\\{lb}}\X${}\E{}$\6
\&{for} ${}(\|q\K\|n,\39\|i\K{-}\T{1},\39\|j\K\T{0};{}$ \|q; ${}\|q\MRL{{\GG}{%
\K}}\T{1},\39\|i\PP){}$\1\5
${}\|j\MRL{+{\K}}\|q\AND\T{1}{}$;\C{ now $i=\lfloor\lg n\rfloor$ and $j=\nu n$
}\2\6
\&{if} ${}(\|j>\T{16}){}$\1\5
${}\|j\K\T{16};{}$\2\6
\&{for} ${}(\|j\MM;{}$ \|j; ${}\|j\MRL{{\GG}{\K}}\T{1}){}$\1\5
${}\|i\PP;{}$\2\6
${}\\{lb}\K\\{fgetc}(\\{infile})-\.{'\ '}{}$;\C{ \PB{\\{fgetc}} will return a
negative value after EOF }\6
\&{if} ${}(\\{lb}<\|i){}$\1\5
${}\\{lb}\K\|i{}$;\2\par
\U1.\fi

\M{5}Initially we want to fill the \PB{\|l} table with fairly decent upper
bounds. Here I use Theorem 4.6.3F, which includes the binary method
as a special case; namely, $l(2^A+xy)\le A+\nu x+\nu y-1$,
when $A\ge\lfloor\lg x\rfloor+\lfloor\lg y\rfloor$.
Then I also apply the factor method.

The \PB{\\{down}} table temporarily holds $\nu n$ values here. But then,
as explained later, we want \PB{$\\{down}[\|n]\K\|n-\T{1}$} initially.

\Y\B\4\X5:Initialize the \PB{\|l} and \PB{\\{down}} tables\X${}\E{}$\6
\&{for} ${}(\|n\K\T{2},\39\\{down}[\T{1}]\K\T{1};{}$ ${}\|n<\\{nmax};{}$ ${}\|n%
\PP){}$\1\5
${}\\{down}[\|n]\K\\{down}[\|n\GG\T{1}]+(\|n\AND\T{1}),\39\|l[\|n]\K\T{127};{}$%
\2\6
\&{for} ${}(\|i\K\T{1},\39\|p\K\T{0};{}$ ${}\|i*\|i<\\{nmax};{}$ ${}\|i\PP,\39%
\|p\K(\|i\E\T{1}\LL(\|p+\T{1})\?\|p+\T{1}:\|p)){}$\5
${}\{{}$\C{ $p=\lfloor i\rfloor$ }\1\6
\&{for} ${}(\|j\K\|i,\39\|q\K\|p;{}$ ${}\|i*\|j<\\{nmax};{}$ ${}\|j\PP,\39\|q%
\K(\|j\E\T{1}\LL(\|q+\T{1})\?\|q+\T{1}:\|q)){}$\5
${}\{{}$\C{ $q=\lfloor j\rfloor$ }\1\6
\&{for} ${}(\|r\K\|p+\|q;{}$ ${}(\T{1}\LL\|r)+\|i*\|j<\\{nmax};{}$ ${}\|r%
\PP){}$\1\6
\&{if} ${}(\|l[(\T{1}\LL\|r)+\|i*\|j]\G\|r+\\{down}[\|i]+\\{down}[\|j]){}$\1\5
${}\|l[(\T{1}\LL\|r)+\|i*\|j]\K\|r+\\{down}[\|i]+\\{down}[\|j]-\T{1}{}$;\C{
Hansen's method }\2\2\6
\4${}\}{}$\2\6
\4${}\}{}$\2\6
\&{for} ${}(\|n\K\T{1};{}$ ${}\|n<\\{nmax};{}$ ${}\|n\PP){}$\1\5
${}\\{down}[\|n]\K\|n-\T{1};{}$\2\6
\X6:Apply the factor method\X;\par
\U1.\fi

\M{6}Whenever we learn a better upper bound, we might as well broadcast all
of its consequences, using the factor method. (The total number of
times this happens for a particular number \PB{\|n} is at most $\lg n$,
and the time to propagate is proportional to \PB{$\\{nmax}/\|n$}, so the
total time for all these updates is at most $O(\log n)^2$ times \PB{\\{nmax}}.)

A linked stack is used to handle these updates. An element \PB{\|p} is on the
stack if and only if \PB{$\\{link}[\|p]\I\T{0}$}, if and only \PB{\|l[\|p]} has
decreased
since the last time we checked all of its multiples.

\Y\B\4\D$\\{upbound}(\|p,\|x)$ \6
\&{if} ${}(\|l[\|p]>\|x){}$\5
${}\{{}$\1\6
${}\|l[\|p]\K\|x;{}$\6
\&{if} ${}(\\{link}[\|p]\E\T{0}){}$\1\5
${}\\{link}[\|p]\K\\{top},\39\\{top}\K\|p;{}$\2\6
\4${}\}{}$\2\par
\Y\B\4\X6:Apply the factor method\X${}\E{}$\6
$\\{top}\K\T{1}{}$;\C{ start with empty stack }\6
\&{for} ${}(\|i\K\T{4};{}$ ${}\|i<\\{nmax};{}$ ${}\|i\PP){}$\5
${}\{{}$\1\6
${}\\{upbound}(\|i,\39\|l[\|i-\T{2}]+\T{1});{}$\6
${}\\{upbound}(\|i,\39\|l[\|i-\T{1}]+\T{1});{}$\6
\4${}\}{}$\2\6
\&{for} ${}(\|i\K\T{2};{}$ ${}\|i*\|i<\\{nmax};{}$ ${}\|i\PP){}$\1\6
\&{for} ${}(\|j\K\|i;{}$ ${}\|i*\|j<\\{nmax};{}$ ${}\|j\PP){}$\1\5
${}\\{upbound}(\|i*\|j,\39\|l[\|i]+\|l[\|j]);{}$\2\2\6
\&{while} ${}(\\{top}>\T{1}){}$\5
${}\{{}$\1\6
${}\|p\K\\{top},\39\\{top}\K\\{link}[\|p],\39\\{link}[\|p]\K\T{0};{}$\6
${}\\{upbound}(\|p+\T{1},\39\|l[\|p]+\T{1});{}$\6
${}\\{upbound}(\|p+\T{2},\39\|l[\|p]+\T{2});{}$\6
\&{for} ${}(\|i\K\T{2};{}$ ${}\|i*\|p<\\{nmax};{}$ ${}\|i\PP){}$\1\5
${}\\{upbound}(\|i*\|p,\39\|l[\|i]+\|l[\|p]);{}$\2\6
\4${}\}{}$\2\par
\U5.\fi

\M{7}Later, whenever we've come up with an addition chain
$a[0]$, \dots, $a[\PB{\\{lb}}]$ that beats the known upper bound,
we use the factor method again to send out the good news. And we also
might as well use an extension of the factor method found in
equation (4.3) in the Bleichenbacher/Flammenkamp paper.

At the beginning of this step, \PB{$\\{top}\K\T{1}$}. We've allocated space for
\PB{\|l[\\{nmax}]} and \PB{$\|l[\\{nmax}+\T{1}]$} in order to avoid making a
special test here.

\Y\B\4\X7:Update the upper bounds based on the chain found\X${}\E{}$\6
\&{for} ${}(\|j\K\T{1};{}$ ${}\|j*\|n<\\{nmax};{}$ ${}\|j\PP){}$\1\5
${}\\{upbound}(\|j*\|n,\39\|l[\|j]+\|l[\|n]);{}$\2\6
\&{for} ${}(\|i\K\T{0};{}$ ${}\|i<\\{lb};{}$ ${}\|i\PP){}$\1\6
\&{for} ${}(\|j\K\T{1};{}$ ${}\|j*\|n+\|a[\|i]<\\{nmax};{}$ ${}\|j\PP){}$\1\5
${}\\{upbound}(\|j*\|n+\|a[\|i],\39\|l[\|j]+\|l[\|n]+\T{1}){}$;\C{ B and F's
method }\2\2\6
\&{while} ${}(\\{top}>\T{1}){}$\5
${}\{{}$\1\6
${}\|p\K\\{top},\39\\{top}\K\\{link}[\|p],\39\\{link}[\|p]\K\T{0};{}$\6
${}\\{upbound}(\|p+\T{1},\39\|l[\|p]+\T{1});{}$\6
${}\\{upbound}(\|p+\T{2},\39\|l[\|p]+\T{2});{}$\6
\&{for} ${}(\|i\K\T{2};{}$ ${}\|i*\|p<\\{nmax};{}$ ${}\|i\PP){}$\1\5
${}\\{upbound}(\|i*\|p,\39\|l[\|i]+\|l[\|p]);{}$\2\6
\4${}\}{}$\2\par
\U12.\fi

\N{1}{8}The interesting part.
Nontrivial changes begin to occur when we get into the guts of
the backtracking structure carried over from the previous versions of this
program, but the controlling loop at the outer level remains intact.

\Y\B\4\X8:Backtrack until $l(n)$ is known\X${}\E{}$\6
$\\{ub}\K\|l[\|n],\39\|l[\|n]\K\\{lb};{}$\6
\&{while} ${}(\\{lb}<\\{ub}){}$\5
${}\{{}$\1\6
\&{for} ${}(\|i\K\T{0};{}$ ${}\|i\Z\\{lb};{}$ ${}\|i\PP){}$\1\5
${}\\{outdeg}[\|i]\K\\{outsum}[\|i]\K\T{0};{}$\2\6
${}\|a[\\{lb}]\K\|b[\\{lb}]\K\|n;{}$\6
\&{for} ${}(\|i\K\T{2};{}$ ${}\|i<\\{lb};{}$ ${}\|i\PP){}$\1\5
${}\|a[\|i]\K\|a[\|i-\T{1}]+\T{1},\39\|b[\|i]\K\|b[\|i-\T{1}]\LL\T{1};{}$\2\6
\&{for} ${}(\|i\K\\{lb}-\T{1};{}$ ${}\|i\G\T{2};{}$ ${}\|i\MM){}$\5
${}\{{}$\1\6
\&{if} ${}((\|a[\|i]\LL\T{1})<\|a[\|i+\T{1}]){}$\1\5
${}\|a[\|i]\K(\|a[\|i+\T{1}]+\T{1})\GG\T{1};{}$\2\6
\&{if} ${}(\|b[\|i]\G\|b[\|i+\T{1}]){}$\1\5
${}\|b[\|i]\K\|b[\|i+\T{1}]-\T{1};{}$\2\6
\4${}\}{}$\2\6
\X12:Try to fix the rest of the chain; \PB{\&{goto} \\{backtrackdone}} if it's
possible\X;\6
${}\|l[\|n]\K\PP\\{lb};{}$\6
\4${}\}{}$\2\6
\\{backtrackdone}:\par
\U1.\fi

\M{9}One of the key operations we need is to increase \PB{\|p} to the smallest
element $p'>p$ that has $l[p']<s$, given that $l[p]<s$. Since
$l[p+1]\le l[p]+1$, we can do this quickly by first setting $p\gets p+1$;
then, if $l[p]=s$, we set $p\gets\PB{\\{down}}[p]$, where \PB{\\{down}[\|p]} is
the
smallest $p'>p$ that has $l[p']<l[p]$.

The links \PB{\\{down}[\|p]} can be prepared as we go, starting them off at $%
\infty$
and updating them whenever we learn a new value of \PB{\|l[\|n]}.

Instead of using infinite links, however, we can save space by
temporarily letting $\PB{\\{down}}[p]=p''$ in such cases, where $p''$ is the
largest element {\it less than\/} $p$ whose \PB{\\{down}} link is effectively
infinite. These temporary links tell us exactly what we need to know during
the updating process. And we can distinguish them from ``real'' \PB{\\{down}}
links by pretending that $\PB{\\{down}}[p]=\infty$ whenever \PB{$\\{down}[\|p]%
\Z\|p$}.

\Y\B\4\X9:Given that \PB{$\|l[\|p]<\|s$}, increase \PB{\|p} to the next such
element\X${}\E{}$\6
${}\{{}$\1\6
${}\|p\PP;{}$\6
\&{if} ${}(\|l[\|p]\E\|s){}$\1\5
${}\|p\K(\\{down}[\|p]>\|p\?\\{down}[\|p]:\\{nmax});{}$\2\6
\4${}\}{}$\2\par
\U14.\fi

\M{10}\B\X10:Given that \PB{$\|l[\|p]\G\|s$}, increase \PB{\|p} to the next
element with \PB{$\|l[\|p]<\|s$}\X${}\E{}$\6
\&{do}\5
${}\{{}$\1\6
\&{if} ${}(\\{down}[\|p]>\|p){}$\1\5
${}\|p\K\\{down}[\|p];{}$\2\6
\&{else}\5
${}\{{}$\1\6
${}\|p\K\\{nmax}{}$;\5
\&{break};\6
\4${}\}{}$\2\6
\4${}\}{}$\5
\2\5
\&{while} ${}(\|l[\|p]\G\|s){}$;\par
\Us13\ET14.\fi

\M{11}I can't help exclaiming that this little algorithm is quite pretty.

\Y\B\4\X11:Update the \PB{\\{down}} links\X${}\E{}$\6
\&{if} ${}(\|l[\|n]<\|l[\|n-\T{1}]){}$\5
${}\{{}$\1\6
\&{for} ${}(\|p\K\\{down}[\|n];{}$ ${}\|l[\|p]>\|l[\|n];{}$ ${}\|p\K\|q){}$\1\5
${}\|q\K\\{down}[\|p],\39\\{down}[\|p]\K\|n;{}$\2\6
${}\\{down}[\|n]\K\|p;{}$\6
\4${}\}{}$\2\par
\U1.\fi

\M{12}We maintain a stack of subproblems, and a stack for undoing,
as in {\mc ACHAIN2} and its predecessors.

\Y\B\4\X12:Try to fix the rest of the chain; \PB{\&{goto} \\{backtrackdone}} if
it's possible\X${}\E{}$\6
$\\{ptr}\K\T{0}{}$;\C{ clear the \PB{\\{undo}} stack }\6
\&{for} ${}(\|r\K\|s\K\\{lb};{}$ ${}\|s>\T{2};{}$ ${}\|s\MM){}$\5
${}\{{}$\1\6
\&{if} ${}(\\{outdeg}[\|s]\E\T{1}){}$\1\5
${}\\{limit}[\|s]\K\|a[\|s]-\\{tail}[\\{outsum}[\|s]]{}$;\5
\2\&{else}\1\5
${}\\{limit}[\|s]\K\|a[\|s]-\T{1}{}$;\C{ the max feasible \PB{\|p} }\2\6
\&{if} ${}(\\{limit}[\|s]>\|b[\|s-\T{1}]){}$\1\5
${}\\{limit}[\|s]\K\|b[\|s-\T{1}];{}$\2\6
\X13:Set \PB{\|p} to its smallest feasible value, and \PB{$\|q\K\|a[\|s]-\|p$}%
\X;\6
\&{while} ${}(\|p\Z\\{limit}[\|s]){}$\5
${}\{{}$\1\6
\X17:Find bounds $(\PB{\\{lbp}},\PB{\\{ubp}})$ and $(\PB{\\{lbq}},\PB{%
\\{ubq}})$ on where \PB{\|p} and \PB{\|q} can be inserted; but go to \PB{%
\\{failpq}} if they can't both be accommodated\X;\6
${}\\{ptrp}\K\\{ptr};{}$\6
\&{for} ( ; ${}\\{ubp}\G\\{lbp};{}$ ${}\\{ubp}\MM){}$\5
${}\{{}$\1\6
\X19:Put \PB{\|p} into the chain at location \PB{\\{ubp}}; \PB{\&{goto} %
\\{failp}} if there's a problem\X;\6
\&{if} ${}(\|p\E\|q){}$\1\5
\&{goto} \\{happiness};\2\6
\&{if} ${}(\\{ubq}\G\\{ubp}){}$\1\5
${}\\{ubq}\K\\{ubp}-\T{1};{}$\2\6
${}\\{ptrq}\K\\{ptr};{}$\6
\&{for} ( ; ${}\\{ubq}\G\\{lbq};{}$ ${}\\{ubq}\MM){}$\5
${}\{{}$\1\6
\X21:Put \PB{\|q} into the chain at location \PB{\\{ubq}}; \PB{\&{goto} %
\\{failq}} if there's a problem\X;\6
\4\\{happiness}:\5
\X15:Put local variables on the stack and update outdegrees\X;\6
\&{goto} \\{onward};\C{ now \PB{\|a[\|s]} is covered; try to cover \PB{$\|a[%
\|s-\T{1}]$} }\6
\4\\{backup}:\5
${}\|s\PP;{}$\6
\&{if} ${}(\|s>\\{lb}){}$\1\5
\&{goto} \\{impossible};\2\6
\X16:Restore local variables from the stack and downdate outdegrees\X;\6
\&{if} ${}(\|p\E\|q){}$\1\5
\&{goto} \\{failp};\2\6
\4\\{failq}:\5
\&{while} ${}(\\{ptr}>\\{ptrq}){}$\1\5
\X18:Undo a change\X;\2\6
\4${}\}{}$\C{ end loop on \PB{\\{ubq}} }\2\6
\4\\{failp}:\5
\&{while} ${}(\\{ptr}>\\{ptrp}){}$\1\5
\X18:Undo a change\X;\2\6
\4${}\}{}$\C{ end loop on \PB{\\{ubp}} }\2\6
\4\\{failpq}:\5
\X14:Advance \PB{\|p} to the next smallest feasible value, and set \PB{$\|q\K%
\|a[\|s]-\|p$}\X;\6
\4${}\}{}$\C{ end loop on \PB{\|p} }\2\6
\&{goto} \\{backup};\6
\4\\{onward}:\5
\&{continue};\6
\4${}\}{}$\C{ end loop on \PB{\|s} }\2\6
\X7:Update the upper bounds based on the chain found\X;\6
\&{goto} \\{backtrackdone};\6
\4\\{impossible}:\par
\U8.\fi

\M{13}At this point we have \PB{$\|a[\|k]\K\|b[\|k]$} for all $r\le k\le\PB{%
\\{lb}}$.

\Y\B\4\X13:Set \PB{\|p} to its smallest feasible value, and \PB{$\|q\K\|a[\|s]-%
\|p$}\X${}\E{}$\6
\&{if} ${}(\|a[\|s]\AND\T{1}){}$\5
${}\{{}$\C{ necessarily \PB{$\|p\I\|q$} }\1\6
\4\\{unequal}:\5
\&{if} ${}(\\{outdeg}[\|s-\T{1}]\E\T{0}){}$\1\5
${}\|q\K\|a[\|s]/\T{3}{}$;\5
\2\&{else}\1\5
${}\|q\K\|a[\|s]\GG\T{1};{}$\2\6
\&{if} ${}(\|q>\|b[\|s-\T{2}]){}$\1\5
${}\|q\K\|b[\|s-\T{2}];{}$\2\6
${}\|p\K\|a[\|s]-\|q;{}$\6
\&{if} ${}(\|l[\|p]\G\|s){}$\5
${}\{{}$\1\6
\X10:Given that \PB{$\|l[\|p]\G\|s$}, increase \PB{\|p} to the next element
with \PB{$\|l[\|p]<\|s$}\X;\6
${}\|q\K\|a[\|s]-\|p;{}$\6
\4${}\}{}$\2\6
\4${}\}{}$\5
\2\&{else}\5
${}\{{}$\1\6
${}\|p\K\|q\K\|a[\|s]\GG\T{1};{}$\6
\&{if} ${}(\|l[\|p]\G\|s){}$\1\5
\&{goto} \\{unequal};\C{ a rare case like \PB{$\|l[\T{191}]\K\|l[\T{382}]$} }\2%
\6
\4${}\}{}$\2\6
\&{if} ${}(\|p>\\{limit}[\|s]){}$\1\5
\&{goto} \\{backup};\2\6
\&{for} ( ; ${}\|r>\T{2}\W\|a[\|r-\T{1}]\E\|b[\|r-\T{1}];{}$ ${}\|r\MM){}$\1\5
;\2\6
\&{if} ${}(\|p>\|b[\|r-\T{1}]){}$\5
${}\{{}$\C{ now \PB{$\|r<\|s$}, since \PB{$\|p\Z\|b[\|s-\T{1}]$} }\1\6
\&{while} ${}(\|p>\|a[\|r]){}$\1\5
${}\|r\PP{}$;\C{ this step keeps \PB{$\|r<\|s$}, since \PB{$\|a[\|s-\T{1}]\K%
\|b[\|s-\T{1}]$} }\2\6
${}\|p\K\|a[\|r],\39\|q\K\|a[\|s]-\|p;{}$\6
\4${}\}{}$\5
\2\&{else} \&{if} ${}(\|q<\|p\W\|q>\|b[\|r-\T{2}]){}$\5
${}\{{}$\1\6
\&{if} ${}(\|a[\|r]\Z\|a[\|s]-\|b[\|r-\T{2}]){}$\1\5
${}\|p\K\|a[\|r],\39\|q\K\|b[\|s]-\|p;{}$\2\6
\&{else}\1\5
${}\|q\K\|b[\|r-\T{2}],\39\|p\K\|a[\|s]-\|q;{}$\2\6
\4${}\}{}$\2\par
\U12.\fi

\M{14}\B\X14:Advance \PB{\|p} to the next smallest feasible value, and set %
\PB{$\|q\K\|a[\|s]-\|p$}\X${}\E{}$\6
\&{if} ${}(\|p\E\|q){}$\5
${}\{{}$\1\6
\&{if} ${}(\\{outdeg}[\|s-\T{1}]\E\T{0}){}$\1\5
${}\|q\K(\|a[\|s]/\T{3})+\T{1}{}$;\C{ will be decreased momentarily }\2\6
\&{if} ${}(\|q>\|b[\|s-\T{2}]){}$\1\5
${}\|q\K\|b[\|s-\T{2}]{}$;\5
\2\&{else}\1\5
${}\|q\MM;{}$\2\6
${}\|p\K\|a[\|s]-\|q;{}$\6
\&{if} ${}(\|l[\|p]\G\|s){}$\5
${}\{{}$\1\6
\X10:Given that \PB{$\|l[\|p]\G\|s$}, increase \PB{\|p} to the next element
with \PB{$\|l[\|p]<\|s$}\X;\6
${}\|q\K\|a[\|s]-\|p;{}$\6
\4${}\}{}$\2\6
\4${}\}{}$\5
\2\&{else}\5
${}\{{}$\1\6
\X9:Given that \PB{$\|l[\|p]<\|s$}, increase \PB{\|p} to the next such element%
\X;\6
${}\|q\K\|a[\|s]-\|p;{}$\6
\4${}\}{}$\2\6
\&{if} ${}(\|q>\T{2}){}$\5
${}\{{}$\1\6
\&{if} ${}(\|a[\|s-\T{1}]\E\|b[\|s-\T{1}]){}$\5
${}\{{}$\C{ maybe \PB{\|p} has to be present already }\1\6
\4\\{doublecheck}:\5
\&{while} ${}(\|p<\|a[\|r]\W\|a[\|r-\T{1}]\E\|b[\|r-\T{1}]){}$\1\5
${}\|r\MM;{}$\2\6
\&{if} ${}(\|p>\|b[\|r-\T{1}]){}$\5
${}\{{}$\1\6
\&{while} ${}(\|p>\|a[\|r]){}$\1\5
${}\|r\PP;{}$\2\6
${}\|p\K\|a[\|r],\39\|q\K\|a[\|s]-\|p{}$;\C{ possibly \PB{$\|r\K\|s$} now }\6
\4${}\}{}$\5
\2\&{else} \&{if} ${}(\|q>\|b[\|r-\T{2}]){}$\5
${}\{{}$\1\6
\&{if} ${}(\|a[\|r]\Z\|a[\|s]-\|b[\|r-\T{2}]){}$\1\5
${}\|p\K\|a[\|r],\39\|q\K\|b[\|s]-\|p;{}$\2\6
\&{else}\1\5
${}\|q\K\|b[\|r-\T{2}],\39\|p\K\|a[\|s]-\|q;{}$\2\6
\4${}\}{}$\2\6
\4${}\}{}$\2\6
\&{if} ${}(\\{ubq}\G\|s){}$\1\5
${}\\{ubq}\K\|s-\T{1};{}$\2\6
\&{while} ${}(\|q\G\|a[\\{ubq}+\T{1}]){}$\1\5
${}\\{ubq}\PP;{}$\2\6
\&{while} ${}(\|q<\|a[\\{ubq}]){}$\1\5
${}\\{ubq}\MM;{}$\2\6
\&{if} ${}(\|q>\|b[\\{ubq}]){}$\5
${}\{{}$\1\6
${}\|q\K\|b[\\{ubq}],\39\|p\K\|a[\|s]-\|q;{}$\6
\&{if} ${}(\|a[\|s-\T{1}]\E\|b[\|s-\T{1}]){}$\1\5
\&{goto} \\{doublecheck};\2\6
\4${}\}{}$\2\6
\4${}\}{}$\2\par
\U12.\fi

\M{15}\B\X15:Put local variables on the stack and update outdegrees\X${}\E{}$\6
$\\{tail}[\|s]\K\|q,\39\\{stack}[\|s].\|r\K\|r;{}$\6
${}\\{outdeg}[\\{ubp}]\PP,\39\\{outsum}[\\{ubp}]\MRL{+{\K}}\|s;{}$\6
${}\\{outdeg}[\\{ubq}]\PP,\39\\{outsum}[\\{ubq}]\MRL{+{\K}}\|s;{}$\6
${}\\{stack}[\|s].\\{lbp}\K\\{lbp},\39\\{stack}[\|s].\\{ubp}\K\\{ubp};{}$\6
${}\\{stack}[\|s].\\{lbq}\K\\{lbq},\39\\{stack}[\|s].\\{ubq}\K\\{ubq};{}$\6
${}\\{stack}[\|s].\\{ptrp}\K\\{ptrp},\39\\{stack}[\|s].\\{ptrq}\K\\{ptrq}{}$;%
\par
\U12.\fi

\M{16}\B\X16:Restore local variables from the stack and downdate outdegrees%
\X${}\E{}$\6
$\\{ptrq}\K\\{stack}[\|s].\\{ptrq},\39\\{ptrp}\K\\{stack}[\|s].\\{ptrp};{}$\6
${}\\{lbq}\K\\{stack}[\|s].\\{lbq},\39\\{ubq}\K\\{stack}[\|s].\\{ubq};{}$\6
${}\\{lbp}\K\\{stack}[\|s].\\{lbp},\39\\{ubp}\K\\{stack}[\|s].\\{ubp};{}$\6
${}\\{outdeg}[\\{ubq}]\MM,\39\\{outsum}[\\{ubq}]\MRL{-{\K}}\|s;{}$\6
${}\\{outdeg}[\\{ubp}]\MM,\39\\{outsum}[\\{ubp}]\MRL{-{\K}}\|s;{}$\6
${}\|q\K\\{tail}[\|s],\39\|p\K\|a[\|s]-\|q,\39\|r\K\\{stack}[\|s].\|r{}$;\par
\U12.\fi

\M{17}After the test in this step is passed, we'll have \PB{$\\{ubp}>\\{ubq}$}
and \PB{$\\{lbp}>\\{lbq}$}.

\Y\B\4\X17:Find bounds $(\PB{\\{lbp}},\PB{\\{ubp}})$ and $(\PB{\\{lbq}},\PB{%
\\{ubq}})$ on where \PB{\|p} and \PB{\|q} can be inserted; but go to \PB{%
\\{failpq}} if they can't both be accommodated\X${}\E{}$\6
\&{if} ${}(\|l[\|p]\G\|s){}$\1\5
\&{goto} \\{failpq};\2\6
${}\\{lbp}\K\|l[\|p];{}$\6
\&{while} ${}(\|b[\\{lbp}]<\|p){}$\1\5
${}\\{lbp}\PP;{}$\2\6
\&{if} ${}((\|p\AND\T{1})\W\|p>\|b[\\{lbp}-\T{2}]+\|b[\\{lbp}-\T{1}]){}$\5
${}\{{}$\1\6
\&{if} ${}(\PP\\{lbp}\G\|s){}$\1\5
\&{goto} \\{failpq};\2\6
\4${}\}{}$\2\6
\&{if} ${}(\|a[\\{lbp}]>\|p){}$\1\5
\&{goto} \\{failpq};\2\6
\&{for} ${}(\\{ubp}\K\\{lbp};{}$ ${}\|a[\\{ubp}+\T{1}]\Z\|p;{}$ ${}\\{ubp}%
\PP){}$\1\5
;\2\6
\&{if} ${}(\\{ubp}\E\|s-\T{1}){}$\1\5
${}\\{lbp}\K\\{ubp};{}$\2\6
\&{if} ${}(\|p\E\|q){}$\1\5
${}\\{lbq}\K\\{lbp},\39\\{ubq}\K\\{ubp};{}$\2\6
\&{else}\5
${}\{{}$\1\6
${}\\{lbq}\K\|l[\|q];{}$\6
\&{if} ${}(\\{lbq}\G\\{ubp}){}$\1\5
\&{goto} \\{failpq};\2\6
\&{while} ${}(\|b[\\{lbq}]<\|q){}$\1\5
${}\\{lbq}\PP;{}$\2\6
\&{if} ${}(\|a[\\{lbq}]<\|b[\\{lbq}]){}$\5
${}\{{}$\1\6
\&{if} ${}((\|q\AND\T{1})\W\|q>\|b[\\{lbq}-\T{2}]+\|b[\\{lbq}-\T{1}]){}$\1\5
${}\\{lbq}\PP;{}$\2\6
\&{if} ${}(\\{lbq}\G\\{ubp}){}$\1\5
\&{goto} \\{failpq};\2\6
\&{if} ${}(\|a[\\{lbq}]>\|q){}$\1\5
\&{goto} \\{failpq};\2\6
\&{if} ${}(\\{lbp}\Z\\{lbq}){}$\1\5
${}\\{lbp}\K\\{lbq}+\T{1};{}$\2\6
\&{while} ${}((\|q\LL(\\{lbp}-\\{lbq}))<\|p){}$\1\6
\&{if} ${}(\PP\\{lbp}>\\{ubp}){}$\1\5
\&{goto} \\{failpq};\2\2\6
\4${}\}{}$\2\6
\&{for} ${}(\\{ubq}\K\\{lbq};{}$ ${}\|a[\\{ubq}+\T{1}]\Z\|q\W(\|q\LL(\\{ubp}-%
\\{ubq}-\T{1}))\G\|p;{}$ ${}\\{ubq}\PP){}$\1\5
;\2\6
\4${}\}{}$\2\par
\U12.\fi

\M{18}The undoing mechanism is very simple: When changing \PB{\|a[\|j]}, we
put \PB{$(\|j\LL\T{24})+\|x$} on the \PB{\\{undo}} stack, where \PB{\|x} was
the former value.
Similarly, when changing \PB{\|b[\|j]}, we stack the value \PB{$(\T{1}\LL%
\T{31})+(\|j\LL\T{24})+\|x$}.

\Y\B\4\D$\\{newa}(\|j,\|y)$ \5
$\\{undo}[\\{ptr}\PP]\K(\|j\LL\T{24})+\|a[\|j],\39\|a[\|j]\K{}$\|y\par
\B\4\D$\\{newb}(\|j,\|y)$ \5
$\\{undo}[\\{ptr}\PP]\K(\T{1}\LL\T{31})+(\|j\LL\T{24})+\|b[\|j],\39\|b[\|j]%
\K{}$\|y\par
\Y\B\4\X18:Undo a change\X${}\E{}$\6
${}\{{}$\1\6
${}\|i\K\\{undo}[\MM\\{ptr}];{}$\6
\&{if} ${}(\|i\G\T{0}){}$\1\5
${}\|a[\|i\GG\T{24}]\K\|i\AND\T{\^ffffff};{}$\2\6
\&{else}\1\5
${}\|b[(\|i\AND\T{\^3fffffff})\GG\T{24}]\K\|i\AND\T{\^ffffff};{}$\2\6
\4${}\}{}$\2\par
\U12.\fi

\M{19}At this point we know that $a[ubp]\le p\le b[ubp]$.

\Y\B\4\X19:Put \PB{\|p} into the chain at location \PB{\\{ubp}}; \PB{\&{goto} %
\\{failp}} if there's a problem\X${}\E{}$\6
\&{if} ${}(\|a[\\{ubp}]\I\|p){}$\5
${}\{{}$\1\6
${}\\{newa}(\\{ubp},\39\|p);{}$\6
\&{for} ${}(\|j\K\\{ubp}-\T{1};{}$ ${}(\|a[\|j]\LL\T{1})<\|a[\|j+\T{1}];{}$ ${}%
\|j\MM){}$\5
${}\{{}$\1\6
${}\|i\K(\|a[\|j+\T{1}]+\T{1})\GG\T{1};{}$\6
\&{if} ${}(\|i>\|b[\|j]){}$\1\5
\&{goto} \\{failp};\2\6
${}\\{newa}(\|j,\39\|i);{}$\6
\4${}\}{}$\2\6
\&{for} ${}(\|j\K\\{ubp}+\T{1};{}$ ${}\|a[\|j]\Z\|a[\|j-\T{1}];{}$ ${}\|j%
\PP){}$\5
${}\{{}$\1\6
${}\|i\K\|a[\|j-\T{1}]+\T{1};{}$\6
\&{if} ${}(\|i>\|b[\|j]){}$\1\5
\&{goto} \\{failp};\2\6
${}\\{newa}(\|j,\39\|i);{}$\6
\4${}\}{}$\2\6
\4${}\}{}$\2\6
\&{if} ${}(\|b[\\{ubp}]\I\|p){}$\5
${}\{{}$\1\6
${}\\{newb}(\\{ubp},\39\|p);{}$\6
\&{for} ${}(\|j\K\\{ubp}-\T{1};{}$ ${}\|b[\|j]\G\|b[\|j+\T{1}];{}$ ${}\|j%
\MM){}$\5
${}\{{}$\1\6
${}\|i\K\|b[\|j+\T{1}]-\T{1};{}$\6
\&{if} ${}(\|i<\|a[\|j]){}$\1\5
\&{goto} \\{failp};\2\6
${}\\{newb}(\|j,\39\|i);{}$\6
\4${}\}{}$\2\6
\&{for} ${}(\|j\K\\{ubp}+\T{1};{}$ ${}\|b[\|j]>\|b[\|j-\T{1}]\LL\T{1};{}$ ${}%
\|j\PP){}$\5
${}\{{}$\1\6
${}\|i\K\|b[\|j-\T{1}]\LL\T{1};{}$\6
\&{if} ${}(\|i<\|a[\|j]){}$\1\5
\&{goto} \\{failp};\2\6
${}\\{newb}(\|j,\39\|i);{}$\6
\4${}\}{}$\2\6
\4${}\}{}$\2\6
\X20:Make forced moves if \PB{\|p} has a special form\X;\par
\U12.\fi

\M{20}If, say, we've just set \PB{$\|a[\T{8}]\K\|b[\T{8}]\K\T{132}$}, special
considerations apply,
because the only addition chains of length~8 for 132 are
$$\eqalign{
&1,2,4,8,16,32,64,128,132;\cr
&1,2,4,8,16,32,64,68,132;\cr
&1,2,4,8,16,32,64,66,132;\cr
&1,2,4,8,16,32,34,66,132;\cr
&1,2,4,8,16,32,33,66,132;\cr
&1,2,4,8,16,17,33,66,132.\cr}$$
The values of \PB{\|a[\T{4}]} and \PB{\|b[\T{4}]} must therefore be 16; and
then, of course,
we also must have \PB{$\|a[\T{3}]\K\|b[\T{3}]\K\T{8}$}, etc. Similar reasoning
applies
whenever we set $a[j]=b[j]=2^j+2^k$ for $k\le j-4$.

Such cases may seem extremely special. But my hunch is that they are
important, because efficient chains need such values. When we try
to prove that no efficient chain exists, we want to show that
such values can't be present. Numbers with small \PB{\|l[\|p]} are harder
to rule out, so it should be helpful to penalize them.

\Y\B\4\X20:Make forced moves if \PB{\|p} has a special form\X${}\E{}$\6
$\|i\K\|p-(\T{1}\LL(\\{ubp}-\T{1}));{}$\6
\&{if} ${}(\|i\W((\|i\AND(\|i-\T{1}))\E\T{0})\W(\|i\LL\T{4})<\|p){}$\5
${}\{{}$\1\6
\&{for} ${}(\|j\K\\{ubp}-\T{2};{}$ ${}(\|i\AND\T{1})\E\T{0};{}$ ${}\|i\MRL{{%
\GG}{\K}}\T{1},\39\|j\MM){}$\1\5
;\2\6
\&{if} ${}(\|b[\|j]<(\T{1}\LL\|j)){}$\1\5
\&{goto} \\{failp};\2\6
\&{for} ( ; ${}\|a[\|j]<(\T{1}\LL\|j);{}$ ${}\|j\MM){}$\1\5
${}\\{newa}(\|j,\39\T{1}\LL\|j);{}$\2\6
\4${}\}{}$\2\par
\U19.\fi

\M{21}At this point we had better not assume that $a[ubq]\le q\le b[ubq]$,
because \PB{\|p} has just been inserted. That insertion can mess up the
bounds that we looked at when \PB{\\{lbq}} and \PB{\\{ubq}} were computed.

\Y\B\4\X21:Put \PB{\|q} into the chain at location \PB{\\{ubq}}; \PB{\&{goto} %
\\{failq}} if there's a problem\X${}\E{}$\6
\&{if} ${}(\|a[\\{ubq}]\I\|q){}$\5
${}\{{}$\1\6
\&{if} ${}(\|a[\\{ubq}]>\|q){}$\1\5
\&{goto} \\{failq};\2\6
${}\\{newa}(\\{ubq},\39\|q);{}$\6
\&{for} ${}(\|j\K\\{ubq}-\T{1};{}$ ${}(\|a[\|j]\LL\T{1})<\|a[\|j+\T{1}];{}$ ${}%
\|j\MM){}$\5
${}\{{}$\1\6
${}\|i\K(\|a[\|j+\T{1}]+\T{1})\GG\T{1};{}$\6
\&{if} ${}(\|i>\|b[\|j]){}$\1\5
\&{goto} \\{failq};\2\6
${}\\{newa}(\|j,\39\|i);{}$\6
\4${}\}{}$\2\6
\&{for} ${}(\|j\K\\{ubq}+\T{1};{}$ ${}\|a[\|j]\Z\|a[\|j-\T{1}];{}$ ${}\|j%
\PP){}$\5
${}\{{}$\1\6
${}\|i\K\|a[\|j-\T{1}]+\T{1};{}$\6
\&{if} ${}(\|i>\|b[\|j]){}$\1\5
\&{goto} \\{failq};\2\6
${}\\{newa}(\|j,\39\|i);{}$\6
\4${}\}{}$\2\6
\4${}\}{}$\2\6
\&{if} ${}(\|b[\\{ubq}]\I\|q){}$\5
${}\{{}$\1\6
\&{if} ${}(\|b[\\{ubq}]<\|q){}$\1\5
\&{goto} \\{failq};\2\6
${}\\{newb}(\\{ubq},\39\|q);{}$\6
\&{for} ${}(\|j\K\\{ubq}-\T{1};{}$ ${}\|b[\|j]\G\|b[\|j+\T{1}];{}$ ${}\|j%
\MM){}$\5
${}\{{}$\1\6
${}\|i\K\|b[\|j+\T{1}]-\T{1};{}$\6
\&{if} ${}(\|i<\|a[\|j]){}$\1\5
\&{goto} \\{failq};\2\6
${}\\{newb}(\|j,\39\|i);{}$\6
\4${}\}{}$\2\6
\&{for} ${}(\|j\K\\{ubq}+\T{1};{}$ ${}\|b[\|j]>\|b[\|j-\T{1}]\LL\T{1};{}$ ${}%
\|j\PP){}$\5
${}\{{}$\1\6
${}\|i\K\|b[\|j-\T{1}]\LL\T{1};{}$\6
\&{if} ${}(\|i<\|a[\|j]){}$\1\5
\&{goto} \\{failq};\2\6
${}\\{newb}(\|j,\39\|i);{}$\6
\4${}\}{}$\2\6
\4${}\}{}$\2\6
\X22:Make forced moves if \PB{\|q} has a special form\X;\par
\U12.\fi

\M{22}\B\X22:Make forced moves if \PB{\|q} has a special form\X${}\E{}$\6
$\|i\K\|q-(\T{1}\LL(\\{ubq}-\T{1}));{}$\6
\&{if} ${}(\|i\W((\|i\AND(\|i-\T{1}))\E\T{0})\W(\|i\LL\T{4})<\|q){}$\5
${}\{{}$\1\6
\&{for} ${}(\|j\K\\{ubq}-\T{2};{}$ ${}(\|i\AND\T{1})\E\T{0};{}$ ${}\|i\MRL{{%
\GG}{\K}}\T{1},\39\|j\MM){}$\1\5
;\2\6
\&{if} ${}(\|b[\|j]<(\T{1}\LL\|j)){}$\1\5
\&{goto} \\{failq};\2\6
\&{for} ( ; ${}\|a[\|j]<(\T{1}\LL\|j);{}$ ${}\|j\MM){}$\1\5
${}\\{newa}(\|j,\39\T{1}\LL\|j);{}$\2\6
\4${}\}{}$\2\par
\U21.\fi

\N{1}{23}Index.
\fi

\inx
\fin
\con
