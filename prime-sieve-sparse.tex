\input cwebmac
\hypertextrue\srcloctrue
\datethis

\N[2 prime-sieve-sparse.w]{1}{1}Intro. This program constructs segments of the
``sieve of Erasthosthenes,''
and outputs the largest prime gaps that it finds. More precisely, it
works with sets of prime numbers between $s_i$ and $s_{i+1}=s_i+\delta$,
represented as an array of bits, and it examines these arrays for
$t$ consecutive intervals beginning with $s_i$ for $i=0$, 1, \dots~$t-1$.
Thus it scans all primes between $s_0$ and $s_t$.

Let $p_k$ be the $k$th prime number. The sieve of Eratotosthenes determines
all primes $\le N$ by starting with the set $\{2,3,\ldots,N\}$ and striking
out the nonprimes: After we know $p_1$ through $p_{k-1}$, the next remaining
element is $p_k$, and we strike out the numbers $p_k^2$, $p_k(p_k+1)$,
$p_k(p_k+2)$, etc. The sieve is complete when we've found the
first prime with $p_k^2>N$.

In this program it's convenient to deal with the nonprimes instead of the
primes, and to assume that we already know all of the ``small'' primes~$p_k$
for which $p_k^2\le s_t$.
And of course we might as well restrict consideration to odd numbers.
Thus, we'll represent the integers between $s_i$ and $s_{i+1}$ by
$\delta/2$ bits; these bits will appear in $\delta/128$ 64-bit numbers
\PB{\\{sieve}[\|j]}, where
$$\PB{\\{sieve}[\|j]}=\sum_{n=s_i+128j}^{s_i+128(j+1)} 2^{(n-s_i-128j-1)/2}
\,\hbox{$\bigl[n$ is an odd multiple of some odd prime
$\le\sqrt{\mathstrut s_{i+1}}\,\bigr]$}.$$

We choose the segment size $\delta$ to be a multiple of 128.
We also assume that $s_0$ is even, and $s_0\ge\sqrt\delta$. It follows
that $s_i$ is even for all~$i$, and that $(s_i+1)^2=s_i^2+s_i+s_{i+1}-\delta
\ge s_i+s_{i+1}>s_{i+1}$. Consequently we have
$$\PB{\\{sieve}[\|j]}=\sum_{n=s_i+128j}^{s_i+128(j+1)} 2^{(n-s_i-128j-1)/2}
\,\hbox{$\bigl[n$ is odd and not prime$\bigr]$},$$
because $n$ appears if and only if it is divisible by some prime~$p$
where $p\le\sqrt{\mathstrut s_{i+1}}<s_i+1\le n$.

In this ``sparse'' version I actually consider only integers of the
form $4m+1$, and I require $\delta$ to be a multiple of 256.
I also require $s_0$ to be a multiple of 4.
Thus the sieve now contains $\delta/256$ octabytes.
Reason: A~gap of size~$g$ between ordinary primes implies a
gap of size~$\ge g$ between primes of the form $4m+1$. If $g\ge1000$,
such gaps are sufficiently rare that I think it's faster to check their
true size by brute force, because we save a factor of two with
the sparse sieve.

``Brute force'' in the previous paragraph means actually a pseudoprime test,
using Miller and Rabin's method.
If that test passes, the probability exceeds $1-2^{-64}$ that I've
incorrectly classified a composite number as a prime.

Although I haven't had much time to experiment with this program, limited
experience has shown that the cache size of the host computer has a
significant effect on speed. Therefore --- counterintuitively ---
it proves to be best to work with rather small segments. In fact,
for numbers in the range of current interest to me (say $4\times10^{17}$,
most of the primes may well exceed $50\delta$.

So this program uses an idea that I found on Tom\'as Oliveira e Silva's
web site: There's a cyclic queue of size~$q$, with lists of the primes that
become relevant in each future segment and their starting places.

\fi

\M[62 prime-sieve-sparse.w]{2}The sieve size $\delta$ and queue size~$q$ are
specified at compile time.
They are preferably powers of two, because we'll want to divide
by~$\delta$ and compute remainders modulo~$q$.

The other fundamental parameters
$s_0$ and $t$ are specified on the command line when this program
is run. And there are two additional command-line parameters,
which name the input and output files.

The input file should contain
all prime numbers $p_1$, $p_2$, \dots, up to the first prime such
that $p_k^2>s_t$; it may also contain further primes, which are ignored.
It is a binary file, with each prime given as an \PB{\&{unsigned} \&{int}}.
(There are 203,280,221 primes less than $2^{32}$, the largest of which
is $2^{32}-5$. Thus I'm implicitly assuming that $s_t<(2^{32}-5)^2
\approx 1.8\times10^{19}$.)

The output file is a short text file that reports large gaps.
Whenever the program discovers consecutive primes for which the gap
$p_{k+1}-p_k$ is greater than or equal to all previously seen gaps,
this gap is output (unless it is smaller than 256).
The smallest and largest
primes between $s_0$ and $s_t$ are also output, so that we can keep
track of gaps between primes that are
found by different instances of this program.

The compile-time parameter \PB{\\{lsize}} is somewhat delicate. We need
$8\PB{\\{qsize}}\times\PB{\\{lsize}}$ bytes of {\mc RAM}, so we don't want \PB{%
\\{lsize}}
to be too large. On the other hand \PB{\\{lsize}} has to be large enough to
to accommodate the queue lists as the program runs. A large \PB{\\{lsize}}
might force \PB{\\{qsize}} to be small, and that will slow things down because
primes will be before they're needed.

\Y\B\4\D$\\{del}$ \5
((\&{long} \&{long})(${}\T{1}\LL\T{23}){}$)\C{ the segment size $\delta$, a
multiple of 256 }\par
\B\4\D$\\{qsize}$ \5
$(\T{1}\LL\T{6}{}$)\C{ the queue size $q$ }\par
\B\4\D$\\{kmax}$ \5
\T{35000000}\C{ an index such that $p_{kmax}^2>s_t$ }\par
\B\4\D$\\{ksmall}$ \5
\T{156000}\C{ an index such that $p_{ksmall}>\delta/4$ }\par
\B\4\D$\\{bestgap}$ \5
\T{1000}\C{ lower bound for gap reporting, $\ge512$, a multiple of 4 }\par
\B\4\D$\\{lsize}$ \5
$(\T{1}\LL\T{21}{}$)\C{ size of queue lists, hopefully big enough }\par
\Y\B\8\#\&{include} \.{<stdio.h>}\6
\8\#\&{include} \.{<stdlib.h>}\6
\8\#\&{include} \.{<time.h>}\6
\&{FILE} ${}{*}\\{infile},{}$ ${}{*}\\{outfile};{}$\6
\&{unsigned} \&{int} \\{prime}[\\{kmax}];\C{ $\PB{\\{prime}}[k]=p_{k+1}$ }\6
\&{unsigned} \&{int} \\{start}[\\{ksmall}];\C{ indices for initializing a
segment }\6
\&{struct} ${}\{{}$\1\6
\&{unsigned} \&{int} \|p;\C{ a prime queued for a segment }\6
\&{unsigned} \&{int} \|s;\C{ its relative starting point }\2\6
${}\}{}$ \\{list}[\\{qsize}][\\{lsize}];\6
\&{int} \\{count}[\\{qsize}];\C{ number of entries in queue lists }\6
\&{int} \\{countmax};\C{ the largest count we've needed so far }\6
\&{unsigned} \&{long} \&{long} ${}\\{sieve}[\T{2}+\\{del}/\T{256}];{}$\6
\&{unsigned} \&{long} \&{long} \\{s0};\C{ beginning of the first segment }\6
\&{int} \\{tt};\C{ number of segments }\6
\&{unsigned} \&{long} \&{long} \\{st};\C{ ending of the last segment }\6
\&{unsigned} \&{long} \&{long} \\{lastprime};\C{ largest prime so far, if any }%
\6
\&{unsigned} \&{long} \&{long} \\{sv}[\T{11}];\C{ bit patterns for the smallest
primes }\6
\&{int} \\{rem}[\T{11}];\C{ shift amounts for the smallest primes }\6
\&{char} \\{nu}[\T{\^10000}];\C{ table for counting bits }\6
\&{int} \\{timer}${},{}$ \\{starttime};\7
\X22:Subroutines\X\7
\\{main}(\&{int} \\{argc}${},\39{}$\&{char} ${}{*}\\{argv}[\,]){}$\1\1\2\2\6
${}\{{}$\1\6
\&{register} \|j${},{}$ \\{jj}${},{}$ \|k;\6
\&{unsigned} \&{long} \&{long} \|x${},{}$ \\{xx}${},{}$ \|y${},{}$ \|z${},{}$ %
\|s${},{}$ \\{ss};\6
\&{int} \|d${},{}$ \\{dd}${},{}$ \\{ii}${},{}$ \\{kk}${},{}$ \\{qq};\7
${}\\{starttime}\K\\{timer}\K\\{time}(\T{0});{}$\6
\X18:Initialize the bit-counting table\X;\6
\X24:Initialize the random number generator\X;\6
\X3:Process the command line and input the primes\X;\6
\X7:Get ready for the first segment\X;\6
\&{for} ${}(\\{ii}\K\T{0};{}$ ${}\\{ii}<\\{tt};{}$ ${}\\{ii}\PP){}$\1\5
\X8:Do segment \PB{\\{ii}}\X;\2\6
\X21:Report the final prime\X;\6
${}\\{printf}(\.{"(Finished;\ the\ last}\)\.{\ segment\ took\ \%d\ sec}\)\.{;\
total\ time\ \%.6g\ ho}\)\.{urs.)\\n"},\39\\{time}(\T{0})-\\{timer},\39((%
\&{double})(\\{time}(\T{0})-\\{starttime}))/\T{3600.0});{}$\6
${}\\{printf}(\.{"(The\ maximum\ list\ s}\)\.{ize\ needed\ was\ \%d.)\\}\)%
\.{n"},\39\\{countmax});{}$\6
\4${}\}{}$\2\par
\fi

\M[142 prime-sieve-sparse.w]{3}\B\X3:Process the command line and input the
primes\X${}\E{}$\6
\&{if} ${}(\\{argc}\I\T{5}\V\\{sscanf}(\\{argv}[\T{1}],\39\.{"\%llu"},\39{\AND}%
\\{s0})\I\T{1}\V\\{sscanf}(\\{argv}[\T{2}],\39\.{"\%d"},\39{\AND}\\{tt})\I%
\T{1}){}$\5
${}\{{}$\1\6
${}\\{fprintf}(\\{stderr},\39\.{"Usage:\ \%s\ s[0]\ t\ in}\)\.{putfile\
outputfile\\n}\)\.{"},\39\\{argv}[\T{0}]);{}$\6
${}\\{exit}({-}\T{1});{}$\6
\4${}\}{}$\2\6
${}\\{infile}\K\\{fopen}(\\{argv}[\T{3}],\39\.{"rb"});{}$\6
\&{if} ${}(\R\\{infile}){}$\5
${}\{{}$\1\6
${}\\{fprintf}(\\{stderr},\39\.{"I\ can't\ open\ \%s\ for}\)\.{\ binary\ input!%
\\n"},\39\\{argv}[\T{3}]);{}$\6
${}\\{exit}({-}\T{2});{}$\6
\4${}\}{}$\2\6
${}\\{outfile}\K\\{fopen}(\\{argv}[\T{4}],\39\.{"w"});{}$\6
\&{if} ${}(\R\\{outfile}){}$\5
${}\{{}$\1\6
${}\\{fprintf}(\\{stderr},\39\.{"I\ can't\ open\ \%s\ for}\)\.{\ text\ output!%
\\n"},\39\\{argv}[\T{4}]);{}$\6
${}\\{exit}({-}\T{3});{}$\6
\4${}\}{}$\2\6
${}\\{st}\K\\{s0}+\\{tt}*\\{del};{}$\6
\&{if} ${}(\\{del}\MOD\T{256}){}$\5
${}\{{}$\1\6
${}\\{fprintf}(\\{stderr},\39\.{"Oops:\ The\ sieve\ siz}\)\.{e\ \%d\ isn't\ a\
multipl}\)\.{e\ of\ 256!\\n"},\39\\{del});{}$\6
${}\\{exit}({-}\T{4});{}$\6
\4${}\}{}$\2\6
\&{if} ${}(\\{s0}\AND\T{3}){}$\5
${}\{{}$\1\6
${}\\{fprintf}(\\{stderr},\39\.{"The\ starting\ point\ }\)\.{\%llu\ isn't\ a\
multipl}\)\.{e\ of\ 4!\\n"},\39\\{s0});{}$\6
${}\\{exit}({-}\T{5});{}$\6
\4${}\}{}$\2\6
\&{if} ${}(\\{s0}*\\{s0}<\\{del}){}$\5
${}\{{}$\1\6
${}\\{fprintf}(\\{stderr},\39\.{"The\ starting\ point\ }\)\.{\%llu\ is\ less\
than\ sq}\)\.{rt(\%llu)!\\n"},\39\\{s0},\39\\{del});{}$\6
${}\\{exit}({-}\T{6});{}$\6
\4${}\}{}$\2\6
\X4:Input the primes\X;\6
${}\\{printf}(\.{"Sieving\ between\ s[0}\)\.{]=\%llu\ and\ s[t]=\%llu}\)\.{:%
\\n"},\39\\{s0},\39\\{st}){}$;\par
\U2.\fi

\M[174 prime-sieve-sparse.w]{4}Primes are divided into three classes: small,
medium, and large.
The small primes (actually ``tiny'') are less than 32; they appear
at least twice in every octabyte of the sieve.
The large primes are greater than $\delta/4$; they appear at most once
in every segment of the sieve.

Since our sieve represents integers of the form $4k+1$, every
segment consists of $\delta/256$ octabytes.

\Y\B\4\D$\\{ddel}$ \5
$(\\{del}/\T{4}{}$)\C{ number of bits per segment }\par
\Y\B\4\X4:Input the primes\X${}\E{}$\6
\&{for} ${}(\|k\K\T{0};{}$  ; ${}\|k\PP){}$\5
${}\{{}$\1\6
\&{if} ${}(\|k\G\\{kmax}){}$\5
${}\{{}$\1\6
${}\\{fprintf}(\\{stderr},\39\.{"Oops:\ Please\ recomp}\)\.{ile\ me\ with\
kmax>\%d!}\)\.{\\n"},\39\\{kmax});{}$\6
${}\\{exit}({-}\T{7});{}$\6
\4${}\}{}$\2\6
\&{if} ${}(\\{fread}({\AND}\\{prime}[\|k],\39{}$\&{sizeof}(\&{unsigned} %
\&{int})${},\39\T{1},\39\\{infile})\I\T{1}){}$\5
${}\{{}$\1\6
${}\\{fprintf}(\\{stderr},\39\.{"The\ input\ file\ ende}\)\.{d\ prematurely\ (%
\%d\^2<}\)\.{\%llu)!\\n"},\39\|k\?\\{prime}[\|k-\T{1}]:\T{0},\39\\{st});{}$\6
${}\\{exit}({-}\T{8});{}$\6
\4${}\}{}$\2\6
\&{if} ${}(\|k\E\T{0}\W\\{prime}[\T{0}]\I\T{2}){}$\5
${}\{{}$\1\6
${}\\{fprintf}(\\{stderr},\39\.{"The\ input\ file\ begi}\)\.{ns\ with\ \%d,\
not\ 2!\\n}\)\.{"},\39\\{prime}[\T{0}]);{}$\6
${}\\{exit}({-}\T{9});{}$\6
\4${}\}{}$\2\6
\&{else} \&{if} ${}(\|k>\T{0}\W\\{prime}[\|k]\Z\\{prime}[\|k-\T{1}]){}$\5
${}\{{}$\1\6
${}\\{fprintf}(\\{stderr},\39\.{"The\ input\ file\ has\ }\)\.{consecutive\
entries\ }\)\.{\%d,\%d!\\n"},\39\\{prime}[\|k-\T{1}],\39\\{prime}[\|k]);{}$\6
${}\\{exit}({-}\T{10});{}$\6
\4${}\}{}$\2\6
\&{if} ${}(\\{prime}[\|k]<\\{ddel}){}$\5
${}\{{}$\1\6
\&{if} ${}(\|k\G\\{ksmall}){}$\5
${}\{{}$\1\6
${}\\{fprintf}(\\{stderr},\39\.{"Oops:\ Please\ recomp}\)\.{ile\ me\ with\
ksmall>\%}\)\.{d!\\n"},\39\\{ksmall});{}$\6
${}\\{exit}({-}\T{11});{}$\6
\4${}\}{}$\2\6
${}\\{dd}\K\|k+\T{1}{}$;\C{ \PB{\\{dd}} will be the index of the first large
prime }\6
\4${}\}{}$\2\6
\&{if} (((\&{unsigned} \&{long} \&{long}) \\{prime}[\|k])${}*\\{prime}[\|k]>%
\\{st}){}$\1\5
\&{break};\2\6
\4${}\}{}$\2\6
${}\\{printf}(\.{"\%d\ primes\ successfu}\)\.{lly\ loaded\ from\ \%s\\n}\)%
\.{"},\39\|k,\39\\{argv}[\T{3}]){}$;\par
\U3.\fi

\N[215 prime-sieve-sparse.w]{1}{5}Sieving. Let's say that the prime $p_k$ is
``active'' if $p_k^2<s_{i+1}$.
Variable \PB{\\{kk}} is the index of the first inactive prime.
The main task of sieving is to mark the multiples of all active
primes in the current segment.

For each active prime $p_k$, let $n_k$ be the smallest multiple of~$p_k$
that exceeds $s_i$ and is congruent to~1 modulo~4.
We let \PB{\\{start}[\|k]} be $(n_k-s_i-1)/4$, the bit offset
of the first such multiple that needs to be marked.

At the beginning, we compute \PB{\\{start}[\|k]} by division. But we'll
be able to compute \PB{\\{start}[\|k]} for subsequent segments as a byproduct
of
sieving, without division; that's why we bother to keep \PB{\\{start}[\|k]} in
memory.

(Actually \PB{\\{start}[\|k]} is computed explicitly only for the small and
medium-sized primes. An equivalent starting point for each large active prime
is recorded in its appropriate queue list.)

\Y\B\4\X5:Initialize the active primes\X${}\E{}$\6
\&{for} ${}(\|k\K\T{1};{}$ ((\&{unsigned} \&{long} \&{long}) \\{prime}[%
\|k])${}*\\{prime}[\|k]<\\{s0};{}$ ${}\|k\PP){}$\5
${}\{{}$\1\6
${}\|j\K{}$(((\&{long} \&{long})(${}\\{prime}[\|k]\AND\T{3})*\\{prime}[\|k])\GG%
\T{2})-{}$(\&{long} \&{long})(${}(\\{s0}\GG\T{2})\MOD\\{prime}[\|k]);{}$\6
\&{if} ${}(\|j<\T{0}){}$\1\5
${}\|j\MRL{+{\K}}\\{prime}[\|k];{}$\2\6
\&{if} ${}(\|k<\\{dd}){}$\1\5
${}\\{start}[\|k]\K\|j;{}$\2\6
\&{else}\5
${}\{{}$\1\6
${}\\{jj}\K(\|j/\\{ddel})\MOD\\{qsize};{}$\6
\&{if} ${}(\\{count}[\\{jj}]\E\\{countmax}){}$\5
${}\{{}$\1\6
${}\\{countmax}\PP;{}$\6
\&{if} ${}(\\{countmax}\G\\{lsize}){}$\5
${}\{{}$\1\6
${}\\{fprintf}(\\{stderr},\39\.{"Oops:\ Please\ recomp}\)\.{ile\ me\ with\
lsize>\%d}\)\.{!\\n"},\39\\{lsize});{}$\6
${}\\{exit}({-}\T{12});{}$\6
\4${}\}{}$\2\6
\4${}\}{}$\2\6
${}\\{list}[\\{jj}][\\{count}[\\{jj}]].\|p\K\\{prime}[\|k],\39\\{list}[\\{jj}][%
\\{count}[\\{jj}]].\|s\K\|j;{}$\6
${}\\{count}[\\{jj}]\PP;{}$\6
\4${}\}{}$\2\6
\4${}\}{}$\2\6
${}\\{kk}\K\|k;{}$\6
\X6:Initialize the tiny active primes\X;\par
\U7.\fi

\M[254 prime-sieve-sparse.w]{6}Primes less than 32 will appear at least twice
in every octabyte of
the sieve. So we handle them in a slightly more efficient way, unless
they're initially inactive.

\Y\B\4\X6:Initialize the tiny active primes\X${}\E{}$\6
\&{for} ${}(\|k\K\T{1};{}$ ${}\\{prime}[\|k]<\T{32}\W\|k<\\{kk};{}$ ${}\|k%
\PP){}$\5
${}\{{}$\1\6
\&{for} ${}(\|x\K\T{0},\39\|y\K\T{1\$L\$L}\LL\\{start}[\|k];{}$ ${}\|x\I\|y;{}$
${}\|x\K\|y,\39\|y\MRL{{\OR}{\K}}\|y\LL\\{prime}[\|k]){}$\1\5
;\2\6
${}\\{sv}[\|k]\K\|x,\39\\{rem}[\|k]\K\T{64}\MOD\\{prime}[\|k];{}$\6
\4${}\}{}$\2\6
${}\|d\K\|k{}$;\C{ \PB{\|d} is the smallest nontiny prime }\par
\U5.\fi

\M[265 prime-sieve-sparse.w]{7}\B\X7:Get ready for the first segment\X${}\E{}$\6
\X5:Initialize the active primes\X;\6
${}\\{ss}\K\\{s0}{}$;\C{ base address of the next segment }\6
${}\\{sieve}[\T{1}+\\{del}/\T{256}]\K{-}\T{1}{}$;\C{ store a sentinel }\par
\U2.\fi

\M[270 prime-sieve-sparse.w]{8}\B\X8:Do segment \PB{\\{ii}}\X${}\E{}$\6
${}\{{}$\1\6
${}\|s\K\\{ss},\39\\{ss}\K\|s+\\{del},\39\\{qq}\K\\{ii}\MOD\\{qsize}{}$;\C{
$s=s_i$, $\PB{\\{ss}}=s_{i+1}$ }\6
\&{if} ${}(\\{qq}\E\T{0}){}$\5
${}\{{}$\1\6
${}\|j\K\\{time}(\T{0});{}$\6
${}\\{printf}(\.{"Beginning\ segment\ \%}\)\.{llu\ (after\ \%d\ sec)\\n}\)%
\.{"},\39\|s,\39\|j-\\{timer});{}$\6
\\{fflush}(\\{stdout});\6
${}\\{timer}\K\|j;{}$\6
\4${}\}{}$\2\6
\X9:Initialize the sieve from the tiny primes\X;\6
\X10:Sieve in the previously active primes\X;\6
\X12:Sieve in the newly active primes\X;\6
\X13:Look for large gaps\X;\6
\4${}\}{}$\2\par
\U2.\fi

\M[285 prime-sieve-sparse.w]{9}\B\X9:Initialize the sieve from the tiny primes%
\X${}\E{}$\6
\&{for} ${}(\|j\K\T{0};{}$ ${}\|j<\\{del}/\T{256};{}$ ${}\|j\PP){}$\5
${}\{{}$\1\6
\&{for} ${}(\|z\K\T{0},\39\|k\K\T{1};{}$ ${}\|k<\|d;{}$ ${}\|k\PP){}$\5
${}\{{}$\1\6
${}\|z\MRL{{\OR}{\K}}\\{sv}[\|k];{}$\6
${}\\{sv}[\|k]\K(\\{sv}[\|k]\LL(\\{prime}[\|k]-\\{rem}[\|k]))\OR(\\{sv}[\|k]\GG%
\\{rem}[\|k]);{}$\6
\4${}\}{}$\2\6
${}\\{sieve}[\|j]\K\|z;{}$\6
\4${}\}{}$\2\par
\U8.\fi

\M[294 prime-sieve-sparse.w]{10}Now we want to set 1 bits for every odd
multiple of \PB{\\{prime}[\|k]}
in the current segment, whenever \PB{\\{prime}[\|k]} is active.
The bit for the integer $s_i+4j+1$ is
\PB{$\T{1}\LL(\|j\AND\T{\^3f})$} in \PB{$\\{sieve}[\|j\GG\T{6}]$}, for $0\le j<%
\delta/4$.

\Y\B\4\X10:Sieve in the previously active primes\X${}\E{}$\6
\&{if} ${}(\\{dd}\G\\{kk}){}$\5
${}\{{}$\C{ no large primes are active }\1\6
\&{for} ${}(\|k\K\|d;{}$ ${}\|k<\\{kk};{}$ ${}\|k\PP){}$\5
${}\{{}$\1\6
\&{for} ${}(\|j\K\\{start}[\|k];{}$ ${}\|j<\\{ddel};{}$ ${}\|j\MRL{+{\K}}%
\\{prime}[\|k]){}$\1\5
${}\\{sieve}[\|j\GG\T{6}]\MRL{{\OR}{\K}}\T{1\$L\$L}\LL(\|j\AND\T{\^3f});{}$\2\6
${}\\{start}[\|k]\K\|j-\\{ddel};{}$\6
\4${}\}{}$\2\6
\4${}\}{}$\5
\2\&{else}\5
${}\{{}$\1\6
\&{for} ${}(\|k\K\|d;{}$ ${}\|k<\\{dd};{}$ ${}\|k\PP){}$\5
${}\{{}$\1\6
\&{for} ${}(\|j\K\\{start}[\|k];{}$ ${}\|j<\\{ddel};{}$ ${}\|j\MRL{+{\K}}%
\\{prime}[\|k]){}$\1\5
${}\\{sieve}[\|j\GG\T{6}]\MRL{{\OR}{\K}}\T{1\$L\$L}\LL(\|j\AND\T{\^3f});{}$\2\6
${}\\{start}[\|k]\K\|j-\\{ddel};{}$\6
\4${}\}{}$\2\6
\X11:Sieve in the enqueued large primes\X;\6
\4${}\}{}$\2\par
\U8.\fi

\M[313 prime-sieve-sparse.w]{11}Each \PB{\|s} entry in \PB{\\{list}} is an
offset relative to the beginning of the
previous segment with \PB{$\\{qq}\K\T{0}$}. Thus, for example, \PB{$\\{list}[%
\T{3}][\|k].\|s$} holds
a number of the form \PB{$\\{ddel}*\T{3}+\|x$}, \PB{$\\{ddel}*(\T{3}+%
\\{qsize})+\|x$}, \PB{$\\{ddel}*(\T{3}+\T{2}*\\{qsize})+\|x$}, etc.,
where $0\le x<\PB{\\{ddel}}$.

\Y\B\4\X11:Sieve in the enqueued large primes\X${}\E{}$\6
\&{for} ${}(\|j\K\|k\K\T{0};{}$ ${}\|k<\\{count}[\\{qq}];{}$ ${}\|k\PP){}$\5
${}\{{}$\1\6
\&{if} ${}(\\{list}[\\{qq}][\|k].\|s\G(\\{qq}+\T{1})*\\{ddel}{}$)\C{ big big
prime has ``looped'' the queue }\1\6
${}\\{list}[\\{qq}][\|j].\|p\K\\{list}[\\{qq}][\|k].\|p,\39\\{list}[\\{qq}][%
\|j].\|s\K\\{list}[\\{qq}][\|k].\|s-\\{qsize}*\\{ddel},\39\|j\PP;{}$\2\6
\&{else}\5
${}\{{}$\1\6
\&{register} \&{unsigned} \&{int} \\{nstart};\7
${}\\{jj}\K\\{list}[\\{qq}][\|k].\|s\MOD\\{ddel};{}$\6
${}\\{sieve}[\\{jj}\GG\T{6}]\MRL{{\OR}{\K}}\T{1\$L\$L}\LL(\\{jj}\AND\T{%
\^3f});{}$\6
${}\\{nstart}\K\\{list}[\\{qq}][\|k].\|s+\\{list}[\\{qq}][\|k].\|p;{}$\6
${}\\{jj}\K(\\{nstart}/\\{ddel})\MOD\\{qsize}{}$;\C{ possibly \PB{$\\{jj}\K%
\\{qq}$}; that's no problem }\6
\&{if} ${}(\\{count}[\\{jj}]\E\\{countmax}){}$\5
${}\{{}$\1\6
${}\\{countmax}\PP;{}$\6
\&{if} ${}(\\{countmax}\G\\{lsize}){}$\5
${}\{{}$\1\6
${}\\{fprintf}(\\{stderr},\39\.{"Oops:\ Please\ recomp}\)\.{ile\ me\ with\
lsize>\%d}\)\.{!\\n"},\39\\{lsize});{}$\6
${}\\{exit}({-}\T{13});{}$\6
\4${}\}{}$\2\6
\4${}\}{}$\2\6
${}\\{list}[\\{jj}][\\{count}[\\{jj}]].\|p\K\\{list}[\\{qq}][\|k].\|p;{}$\6
${}\\{list}[\\{jj}][\\{count}[\\{jj}]].\|s\K(\\{jj}\G\\{qq}\?\\{nstart}:%
\\{nstart}-\\{qsize}*\\{ddel});{}$\6
${}\\{count}[\\{jj}]\PP;{}$\6
\4${}\}{}$\2\6
\4${}\}{}$\2\6
${}\\{count}[\\{qq}]\K\|j{}$;\par
\U10.\fi

\M[342 prime-sieve-sparse.w]{12}The test is \PB{$\\{jj}>\\{qq}$} here, but %
\PB{$\\{jj}\G\\{qq}$} in the previous code. Do you see why?

\Y\B\4\X12:Sieve in the newly active primes\X${}\E{}$\6
\&{for} ${}(\|k\K\\{kk};{}$ ((\&{unsigned} \&{long} \&{long}) \\{prime}[%
\|k])${}*\\{prime}[\|k]<\\{ss};{}$ ${}\|k\PP){}$\5
${}\{{}$\1\6
\&{for} ${}(\|j\K{}$(((\&{unsigned} \&{long} \&{long}) \\{prime}[\|k])${}*%
\\{prime}[\|k]-\|s-\T{1})\GG\T{2};{}$ ${}\|j<\\{ddel};{}$ ${}\|j\MRL{+{\K}}%
\\{prime}[\|k]){}$\1\5
${}\\{sieve}[\|j\GG\T{6}]\MRL{{\OR}{\K}}\T{1\$L\$L}\LL(\|j\AND\T{\^3f});{}$\2\6
\&{if} ${}(\|k<\\{dd}){}$\1\5
${}\\{start}[\|k]\K\|j-\\{ddel};{}$\2\6
\&{else}\5
${}\{{}$\1\6
${}\|j\MRL{+{\K}}\\{qq}*\\{ddel};{}$\6
${}\\{jj}\K(\|j/\\{ddel})\MOD\\{qsize}{}$;\C{ possibly \PB{$\\{jj}\K\\{qq}$};
that's no problem }\6
\&{if} ${}(\\{count}[\\{jj}]\E\\{countmax}){}$\5
${}\{{}$\1\6
${}\\{countmax}\PP;{}$\6
\&{if} ${}(\\{countmax}\G\\{lsize}){}$\5
${}\{{}$\1\6
${}\\{fprintf}(\\{stderr},\39\.{"Oops:\ Please\ recomp}\)\.{ile\ me\ with\
lsize>\%d}\)\.{!\\n"},\39\\{lsize});{}$\6
${}\\{exit}({-}\T{14});{}$\6
\4${}\}{}$\2\6
\4${}\}{}$\2\6
${}\\{list}[\\{jj}][\\{count}[\\{jj}]].\|p\K\\{prime}[\|k];{}$\6
${}\\{list}[\\{jj}][\\{count}[\\{jj}]].\|s\K(\\{jj}>\\{qq}\?\|j:\|j-\\{qsize}*%
\\{ddel});{}$\6
${}\\{count}[\\{jj}]\PP;{}$\6
\4${}\}{}$\2\6
\4${}\}{}$\2\6
${}\\{kk}\K\|k{}$;\par
\U8.\fi

\N[366 prime-sieve-sparse.w]{1}{13}Processing gaps.
If $p_{k+1}-p_k\ge512$, we're bound to find an octabyte of all 1s in the
sieve between the 0~for~$p_k$ and the 0~for~$p_{k+1}$. In such cases,
we check for a potential ``kilogap'' (a gap of length 1000 or more).

Complications occur if the gap appears at the very beginning or end of
a segment, or if an entire segment is prime-free. Further complications
arise because our sieve contains only half of the potential primes.
I've tried to get the logic correct, without slowing the program down.
But if any bugs are present in this code, I suppose they are due to a fallacy
in this aspect of my reasoning.

Two sentinels appear at the end of the sieve, in order to speed up
loop termination: \PB{$\\{sieve}[\\{del}/\T{256}]\K\T{0}$} and \PB{$\\{sieve}[%
\T{1}+\\{del}/\T{256}]\K{-}\T{1}$}.

\Y\B\4\X13:Look for large gaps\X${}\E{}$\6
$\|j\K\T{0},\39\|k\K{-}\T{100};{}$\6
\&{while} (\T{1})\5
${}\{{}$\1\6
\&{for} ( ; ${}\\{sieve}[\|j]\E{-}\T{1};{}$ ${}\|j\PP){}$\1\5
;\2\6
\&{if} ${}(\|j\E\\{del}/\T{256}){}$\1\5
${}\|x\K\\{ss};{}$\2\6
\&{else}\1\5
\X15:Set \PB{\|x} to the smallest prime in \PB{\\{sieve}[\|j]}\X;\2\6
\&{if} ${}(\|k\G\T{0}){}$\1\5
\X16:Set \PB{\\{lastprime}} to the largest prime in \PB{\\{sieve}[\|k]}\X\2\6
\&{else} \&{if} ${}(\\{lastprime}\E\T{0}){}$\1\5
\X14:Set \PB{\\{lastprime}} to the smallest prime $\ge s_0$\X;\2\6
\X19:Look for and report any large gaps between \PB{\\{lastprime}} and \PB{\|x}%
\X;\6
\&{if} ${}(\|j\E\\{del}/\T{256}){}$\1\5
\&{break};\2\6
\&{for} ${}(\|j\PP;{}$ ${}\\{sieve}[\|j]\I{-}\T{1};{}$ ${}\|j\PP){}$\1\5
;\2\6
\&{if} ${}(\|j<\\{del}/\T{256}){}$\1\5
${}\|k\K\|j-\T{1};{}$\2\6
\&{else}\5
${}\{{}$\C{ \PB{$\|j\K\T{1}+\\{del}/\T{256}$} and \PB{$\\{sieve}[\\{del}/%
\T{256}-\T{1}]\I{-}\T{1}$} }\1\6
${}\|k\K\\{del}/\T{256}-\T{1};{}$\6
\X16:Set \PB{\\{lastprime}} to the largest prime in \PB{\\{sieve}[\|k]}\X;\6
\&{break};\6
\4${}\}{}$\2\6
\4${}\}{}$\2\6
\&{for} ${}(\|z\K\\{ss}-\T{1};{}$ ${}\|z>\\{lastprime};{}$ ${}\|z\MRL{-{\K}}%
\T{4}){}$\1\6
\&{if} (\\{isprime}(\|z))\5
${}\{{}$\1\6
${}\\{lastprime}\K\|z{}$;\5
\&{break};\6
\4${}\}{}$\2\2\6
\4\\{donewithseg}:\par
\U8.\fi

\M[404 prime-sieve-sparse.w]{14}\B\X14:Set \PB{\\{lastprime}} to the smallest
prime $\ge s_0$\X${}\E{}$\6
${}\{{}$\1\6
\&{for} ${}(\|z\K\|s+\T{3};{}$ ${}\|z<\|x;{}$ ${}\|z\MRL{+{\K}}\T{4}){}$\1\6
\&{if} (\\{isprime}(\|z))\5
${}\{{}$\1\6
${}\\{lastprime}\K\|z{}$;\5
\&{goto} \\{got\_it};\6
\4${}\}{}$\2\2\6
\&{if} ${}(\|x\E\\{ss}){}$\1\5
\&{goto} \\{donewithseg};\C{ no primes at all below \PB{\\{ss}}! }\2\6
${}\\{lastprime}\K\|x;{}$\6
\4\\{got\_it}:\5
${}\\{fprintf}(\\{outfile},\39\.{"The\ first\ prime\ is\ }\)\.{\%llu\ =\ s[0]+%
\%d\\n"},\39\\{lastprime},\39\\{lastprime}-\\{s0});{}$\6
\\{fflush}(\\{outfile});\6
\4${}\}{}$\2\par
\U13.\fi

\M[416 prime-sieve-sparse.w]{15}\B\X15:Set \PB{\|x} to the smallest prime in %
\PB{\\{sieve}[\|j]}\X${}\E{}$\6
${}\{{}$\1\6
${}\|y\K\CM\\{sieve}[\|j];{}$\6
${}\|y\K\|y\AND{-}\|y{}$;\C{ extract the rightmost 1 bit }\6
\X17:Change \PB{\|y} to its binary logarithm\X;\6
${}\|x\K\|s+(\|j\LL\T{8})+(\|y\LL\T{2})+\T{1}{}$;\C{ this upperbounds the first
prime after a gap }\6
\4${}\}{}$\2\par
\U13.\fi

\M[424 prime-sieve-sparse.w]{16}\B\X16:Set \PB{\\{lastprime}} to the largest
prime in \PB{\\{sieve}[\|k]}\X${}\E{}$\6
${}\{{}$\1\6
\&{for} ${}(\|y\K\CM\\{sieve}[\|k],\39\|z\K\|y\AND(\|y-\T{1});{}$ \|z; ${}\|y\K%
\|z,\39\|z\K\|y\AND(\|y-\T{1})){}$\1\5
;\C{ the leftmost 1 bit }\2\6
\X17:Change \PB{\|y} to its binary logarithm\X;\6
${}\\{lastprime}\K\|s+(\|k\LL\T{8})+(\|y\LL\T{2})+\T{1};{}$\6
\4${}\}{}$\2\par
\U13.\fi

\M[431 prime-sieve-sparse.w]{17}As far as I know, the following method is the
fastest way to compute
binary logarithms on an Opteron computer (which is the machine
I'm targeting here).

\Y\B\4\X17:Change \PB{\|y} to its binary logarithm\X${}\E{}$\6
$\|y\MM;{}$\6
${}\|y\K\\{nu}[\|y\AND\T{\^ffff}]+\\{nu}[(\|y\GG\T{16})\AND\T{\^ffff}]+\\{nu}[(%
\|y\GG\T{32})\AND\T{\^ffff}]+\\{nu}[(\|y\GG\T{48})\AND\T{\^ffff}]{}$;\par
\Us15\ET16.\fi

\M[439 prime-sieve-sparse.w]{18}With a more extensive table, I could count the
1s in an arbitrary
binary word. But seventeen table entries are sufficient for present purposes.

\Y\B\4\X18:Initialize the bit-counting table\X${}\E{}$\6
\&{for} ${}(\|j\K\T{0};{}$ ${}\|j\Z\T{16};{}$ ${}\|j\PP){}$\1\5
${}\\{nu}[((\T{1}\LL\|j)-\T{1})]\K\|j{}$;\2\par
\U2.\fi

\M[445 prime-sieve-sparse.w]{19}When \PB{$\\{sieve}[\|k]\I{-}\T{1}$} and \PB{$%
\\{sieve}[\|j]\I{-}\T{1}$} and everything between them
is \PB{${-}\T{1}$} (all ones), there's a gap of size~$g$ where
$256\vert j-k\vert-126\le g\le256\vert j-k\vert+126$.

If \PB{$\|k<\T{0}$} and \PB{$\\{lastprime}\I\T{0}$}, there are no primes
between \PB{\\{lastprime}} and~\PB{\|s}.

Two or more large gaps may actually be present, in a long interval where
the only primes are of the form $4m+3$. (I doubt if this actually
occurs until the numbers get much larger than I can handle, but I'm
trying to make the program correct.)

\Y\B\4\X19:Look for and report any large gaps between \PB{\\{lastprime}} and %
\PB{\|x}\X${}\E{}$\6
\&{if} ${}(\|j\G\|k+\\{bestgap}/\T{256}){}$\5
${}\{{}$\1\6
${}\\{xx}\K\|x;{}$\6
\4\\{zloop}:\5
\&{if} ${}(\|x-\\{lastprime}<\\{bestgap}){}$\1\5
\&{goto} \\{done\_here};\2\6
${}\|y\K(\|k\G\T{0}\?\\{lastprime}:\|s);{}$\6
\&{for} ${}(\|z\K((\\{lastprime}\AND\CM\T{2})+\\{bestgap}-\T{2});{}$ ${}\|z>%
\|y;{}$ ${}\|z\MRL{-{\K}}\T{4}){}$\1\6
\&{if} (\\{isprime}(\|z))\5
${}\{{}$\1\6
${}\\{lastprime}\K\|z,\39\|k\K\T{0}{}$;\5
\&{goto} \\{zloop};\6
\4${}\}{}$\2\2\6
${}\|z\K(\\{lastprime}\AND\CM\T{2})+\\{bestgap}+\T{2};{}$\6
\&{if} ${}(\|z<\|s){}$\1\5
${}\|z\K\|s+\T{3};{}$\2\6
\&{for} ( ; ${}\|z<\|x;{}$ ${}\|z\MRL{+{\K}}\T{4}){}$\1\6
\&{if} (\\{isprime}(\|z))\5
${}\{{}$\1\6
${}\|x\K\|z{}$;\5
\&{break};\6
\4${}\}{}$\2\2\6
\&{if} ${}(\|x\E\\{ss}){}$\1\5
\&{goto} \\{donewithseg};\C{ \PB{\\{lastprime}} is the largest prime less than %
\PB{\|x} }\2\6
\X20:Report a gap, if it's big enough\X;\6
${}\\{lastprime}\K\|x,\39\|x\K\\{xx}{}$;\5
\&{goto} \\{zloop};\6
\4${}\}{}$\2\6
\4\\{done\_here}:\par
\U13.\fi

\M[476 prime-sieve-sparse.w]{20}\B\X20:Report a gap, if it's big enough\X${}%
\E{}$\6
${}\{{}$\1\6
\&{if} ${}(\|x-\\{lastprime}\G\\{bestgap}){}$\5
${}\{{}$\1\6
${}\\{fprintf}(\\{outfile},\39\.{"\%llu\ is\ followed\ by}\)\.{\ a\ gap\ of\
length\ \%d\\}\)\.{n"},\39\\{lastprime},\39\|x-\\{lastprime});{}$\6
\\{fflush}(\\{outfile});\6
\4${}\}{}$\2\6
\4${}\}{}$\2\par
\U19.\fi

\M[485 prime-sieve-sparse.w]{21}\B\X21:Report the final prime\X${}\E{}$\6
\&{if} (\\{lastprime})\5
${}\{{}$\1\6
${}\\{fprintf}(\\{outfile},\39\.{"The\ final\ prime\ is\ }\)\.{\%llu\ =\ s[t]-%
\%d.\\n"},\39\\{lastprime},\39\\{st}-\\{lastprime});{}$\6
\4${}\}{}$\5
\2\&{else}\1\5
${}\\{fprintf}(\\{outfile},\39\.{"No\ prime\ numbers\ ex}\)\.{ist\ between\
s[0]\ and}\)\.{\ s[t].\\n"}){}$;\2\par
\U2.\fi

\N[491 prime-sieve-sparse.w]{1}{22}Random numbers. The following code comes
directly from
\.{rng.c}, the random number generator in Section 3.6.

\Y\B\4\D$\.{KK}$ \5
\T{100}\C{ the long lag }\par
\B\4\D$\.{LL}$ \5
\T{37}\C{ the short lag }\par
\B\4\D$\.{MM}$ \5
$(\T{1\$L}\LL\T{30}{}$)\C{ the modulus }\par
\B\4\D$\\{mod\_diff}(\|x,\|y)$ \5
$(((\|x)-(\|y))\AND(\.{MM}-\T{1}){}$)\C{ subtraction mod MM }\par
\Y\B\4\X22:Subroutines\X${}\E{}$\6
\&{long} \\{ran\_x}[\.{KK}];\C{ the generator state }\7
\&{void} \\{ran\_array}(\&{long} \\{aa}[\,]${},\39{}$\&{int} \|n)\1\1\2\2\6
${}\{{}$\1\6
\&{register} \&{int} \|i${},{}$ \|j;\7
\&{for} ${}(\|j\K\T{0};{}$ ${}\|j<\.{KK};{}$ ${}\|j\PP){}$\1\5
${}\\{aa}[\|j]\K\\{ran\_x}[\|j];{}$\2\6
\&{for} ( ; ${}\|j<\|n;{}$ ${}\|j\PP){}$\1\5
${}\\{aa}[\|j]\K\\{mod\_diff}(\\{aa}[\|j-\.{KK}],\39\\{aa}[\|j-\.{LL}]);{}$\2\6
\&{for} ${}(\|i\K\T{0};{}$ ${}\|i<\.{LL};{}$ ${}\|i\PP,\39\|j\PP){}$\1\5
${}\\{ran\_x}[\|i]\K\\{mod\_diff}(\\{aa}[\|j-\.{KK}],\39\\{aa}[\|j-\.{LL}]);{}$%
\2\6
\&{for} ( ; ${}\|i<\.{KK};{}$ ${}\|i\PP,\39\|j\PP){}$\1\5
${}\\{ran\_x}[\|i]\K\\{mod\_diff}(\\{aa}[\|j-\.{KK}],\39\\{ran\_x}[\|i-%
\.{LL}]);{}$\2\6
\4${}\}{}$\2\par
\As23, 25, 26\ETs27.
\U2.\fi

\M[510 prime-sieve-sparse.w]{23}\B\D$\.{QUALITY}$ \5
\T{1009}\C{ recommended quality level for high-res use }\par
\B\4\D$\.{TT}$ \5
\T{70}\C{ guaranteed separation between streams }\par
\B\4\D$\\{is\_odd}(\|x)$ \5
$((\|x)\AND\T{1}{}$)\C{ units bit of x }\par
\Y\B\4\X22:Subroutines\X${}\mathrel+\E{}$\6
\&{long} \\{ran\_arr\_buf}[\.{QUALITY}];\6
\&{long} \\{ran\_arr\_dummy}${}\K{-}\T{1},{}$ \\{ran\_arr\_started}${}\K{-}%
\T{1};{}$\6
\&{long} ${}{*}\\{ran\_arr\_ptr}\K{\AND}\\{ran\_arr\_dummy}{}$;\C{ the next
random number, or -1 }\7
\&{void} \\{ran\_start}(\&{long} \\{seed})\1\1\2\2\6
${}\{{}$\1\6
\&{register} \&{int} \|t${},{}$ \|j;\6
\&{long} ${}\|x[\.{KK}+\.{KK}-\T{1}]{}$;\C{ the preparation buffer }\6
\&{register} \&{long} \\{ss}${}\K(\\{seed}+\T{2})\AND(\.{MM}-\T{2});{}$\7
\&{for} ${}(\|j\K\T{0};{}$ ${}\|j<\.{KK};{}$ ${}\|j\PP){}$\5
${}\{{}$\1\6
${}\|x[\|j]\K\\{ss}{}$;\C{ bootstrap the buffer }\6
${}\\{ss}\MRL{{\LL}{\K}}\T{1};{}$\6
\&{if} ${}(\\{ss}\G\.{MM}){}$\1\5
${}\\{ss}\MRL{-{\K}}\.{MM}-\T{2}{}$;\C{ cyclic shift 29 bits }\2\6
\4${}\}{}$\2\6
${}\|x[\T{1}]\PP{}$;\C{ make x[1] (and only x[1]) odd }\6
\&{for} ${}(\\{ss}\K\\{seed}\AND(\.{MM}-\T{1}),\39\|t\K\.{TT}-\T{1};{}$ \|t; %
\,)\5
${}\{{}$\1\6
\&{for} ${}(\|j\K\.{KK}-\T{1};{}$ ${}\|j>\T{0};{}$ ${}\|j\MM){}$\1\5
${}\|x[\|j+\|j]\K\|x[\|j],\39\|x[\|j+\|j-\T{1}]\K\T{0}{}$;\C{ "square" }\2\6
\&{for} ${}(\|j\K\.{KK}+\.{KK}-\T{2};{}$ ${}\|j\G\.{KK};{}$ ${}\|j\MM){}$\1\5
${}\|x[\|j-(\.{KK}-\.{LL})]\K\\{mod\_diff}(\|x[\|j-(\.{KK}-\.{LL})],\39\|x[%
\|j]),\39\|x[\|j-\.{KK}]\K\\{mod\_diff}(\|x[\|j-\.{KK}],\39\|x[\|j]);{}$\2\6
\&{if} (\\{is\_odd}(\\{ss}))\5
${}\{{}$\C{ "multiply by z" }\1\6
\&{for} ${}(\|j\K\.{KK};{}$ ${}\|j>\T{0};{}$ ${}\|j\MM){}$\1\5
${}\|x[\|j]\K\|x[\|j-\T{1}];{}$\2\6
${}\|x[\T{0}]\K\|x[\.{KK}]{}$;\C{ shift the buffer cyclically }\6
${}\|x[\.{LL}]\K\\{mod\_diff}(\|x[\.{LL}],\39\|x[\.{KK}]);{}$\6
\4${}\}{}$\2\6
\&{if} (\\{ss})\1\5
${}\\{ss}\MRL{{\GG}{\K}}\T{1};{}$\2\6
\&{else}\1\5
${}\|t\MM;{}$\2\6
\4${}\}{}$\2\6
\&{for} ${}(\|j\K\T{0};{}$ ${}\|j<\.{LL};{}$ ${}\|j\PP){}$\1\5
${}\\{ran\_x}[\|j+\.{KK}-\.{LL}]\K\|x[\|j];{}$\2\6
\&{for} ( ; ${}\|j<\.{KK};{}$ ${}\|j\PP){}$\1\5
${}\\{ran\_x}[\|j-\.{LL}]\K\|x[\|j];{}$\2\6
\&{for} ${}(\|j\K\T{0};{}$ ${}\|j<\T{10};{}$ ${}\|j\PP){}$\1\5
${}\\{ran\_array}(\|x,\39\.{KK}+\.{KK}-\T{1}){}$;\C{ warm things up }\2\6
${}\\{ran\_arr\_ptr}\K{\AND}\\{ran\_arr\_started};{}$\6
\4${}\}{}$\2\par
\fi

\M[546 prime-sieve-sparse.w]{24}\B\X24:Initialize the random number generator%
\X${}\E{}$\6
\\{ran\_start}(\T{314159\$L});\par
\U2.\fi

\M[549 prime-sieve-sparse.w]{25}After calling \PB{\\{ran\_start}}, we get new
randoms by saying
``\PB{$\|x\K\\{ran\_arr\_next}(\,)$}''.

\Y\B\4\D$\\{ran\_arr\_next}()$ \5
$({*}\\{ran\_arr\_ptr}\G\T{0}\?{*}\\{ran\_arr\_ptr}\PP:\\{ran\_arr\_cycle}(%
\,){}$)\par
\Y\B\4\X22:Subroutines\X${}\mathrel+\E{}$\6
\&{long} \\{ran\_arr\_cycle}(\,)\1\1\2\2\6
${}\{{}$\1\6
\&{if} ${}(\\{ran\_arr\_ptr}\E{\AND}\\{ran\_arr\_dummy}){}$\1\5
\\{ran\_start}(\T{314159\$L});\C{ the user forgot to initialize }\2\6
${}\\{ran\_array}(\\{ran\_arr\_buf},\39\.{QUALITY});{}$\6
${}\\{ran\_arr\_buf}[\.{KK}]\K{-}\T{1};{}$\6
${}\\{ran\_arr\_ptr}\K\\{ran\_arr\_buf}+\T{1};{}$\6
\&{return} \\{ran\_arr\_buf}[\T{0}];\6
\4${}\}{}$\2\par
\fi

\N[565 prime-sieve-sparse.w]{1}{26}Double precision multiplication.
We'll need a subroutine that computes the 128-bit
product of two 64-bit integers. The product goes into \PB{\\{acc\_hi}} and
\PB{\\{acc\_lo}}.

\Y\B\4\X22:Subroutines\X${}\mathrel+\E{}$\6
\&{unsigned} \&{long} \&{long} \\{acc\_hi}${},{}$ \\{acc\_lo};\7
\&{void} \\{mult}(\&{unsigned} \&{long} \&{long} \|x${},\39{}$\&{unsigned} %
\&{long} \&{long} \|y)\1\1\2\2\6
${}\{{}$\1\6
\&{register} \&{unsigned} \&{int} \\{xhi}${},{}$ \\{xlo}${},{}$ \\{yhi}${},{}$ %
\\{ylo};\6
\&{unsigned} \&{long} \&{long} \|t;\7
${}\\{xhi}\K\|x\GG\T{32},\39\\{xlo}\K\|x\AND\T{\^ffffffff};{}$\6
${}\\{yhi}\K\|y\GG\T{32},\39\\{ylo}\K\|y\AND\T{\^ffffffff};{}$\6
${}\|t\K{}$((\&{unsigned} \&{long} \&{long}) \\{xlo})${}*\\{ylo},\39\\{acc\_lo}%
\K\|t\AND\T{\^ffffffff};{}$\6
${}\|t\K{}$((\&{unsigned} \&{long} \&{long}) \\{xhi})${}*\\{ylo}+(\|t\GG%
\T{32}),\39\\{acc\_hi}\K\|t\GG\T{32};{}$\6
${}\|t\K{}$((\&{unsigned} \&{long} \&{long}) \\{xlo})${}*\\{yhi}+(\|t\AND\T{%
\^ffffffff});{}$\6
${}\\{acc\_hi}\MRL{+{\K}}{}$((\&{unsigned} \&{long} \&{long}) \\{xhi})${}*%
\\{yhi}+(\|t\GG\T{32});{}$\6
${}\\{acc\_lo}\MRL{+{\K}}(\|t\AND\T{\^ffffffff})\LL\T{32};{}$\6
\4${}\}{}$\2\par
\fi

\N[584 prime-sieve-sparse.w]{1}{27}Prime testing. I've saved the most
interesting part of this program for
last. It's a subroutine that tries to decide whether a given \PB{\&{long} %
\&{long}}
number~\PB{\|z} is prime. In the experiments I'm doing, \PB{\|z} lies
between $2^{58}$ and $2^{59}$ (but the program does not require that
\PB{\|z} be in this range).

If it's easy to determine that \PB{\|z} is definitely
not prime, the subroutine returns~0.

But if \PB{\|z} passes the Miller--Rabin test for 32 different random
witnesses, the subroutine returns~1.

A nonprime number almost never returns~1. In fact, a nonprime number
that passes the test even once is sufficiently interesting that
I'm printing it out.

Here I implement Algorithm 4.5.4P, using the fact that $z\bmod4=3$,
and using ``Montgomery multiplication'' for speed (exercise 4.3.1--41).

\Y\B\4\X22:Subroutines\X${}\mathrel+\E{}$\6
\&{int} \\{isprime}(\&{unsigned} \&{long} \&{long} \|z)\1\1\2\2\6
${}\{{}$\1\6
\&{register} \&{int} \|k${},{}$ \\{lgz}${},{}$ \\{rep};\6
\&{long} \&{long} \|x${},{}$ \|y${},{}$ \|q;\6
\&{unsigned} \&{long} \&{long} \|m${},{}$ \\{zp}${},{}$ \\{goal};\7
\X32:If $z$ is divisible by a prime $\le53$, \PB{\&{return} \T{0}}\X;\6
\X28:Get ready for Montgomery's method\X;\6
\&{for} ${}(\\{rep}\K\T{0};{}$ ${}\\{rep}<\T{32};{}$ ${}\\{rep}\PP){}$\5
${}\{{}$\1\6
\4\.{P1}:\5
${}\|x\K\\{ran\_arr\_next}(\,);{}$\6
\4\.{P2}:\5
${}\|q\K\|z\GG\T{1};{}$\6
\&{for} ${}(\|y\K\|x,\39\|m\K\T{1\$L\$L}\LL(\\{lgz}-\T{2});{}$ \|m; ${}\|m%
\MRL{{\GG}{\K}}\T{1}){}$\5
${}\{{}$\1\6
\X30:Set $y\gets (y^2\!/2^{64})\bmod z$\X;\6
\&{if} ${}(\|m\AND\|q){}$\1\5
\X31:Set $y\gets (xy/2^{64})\bmod z$\X;\2\6
\4${}\}{}$\2\6
\&{if} ${}(\|y\I\\{goal}\W\|y\I\|z-\\{goal}){}$\5
${}\{{}$\1\6
\&{if} (\\{rep})\5
${}\{{}$\1\6
${}\\{fprintf}(\\{outfile},\39\.{"(\%lld\ is\ a\ pseudopr}\)\.{ime\ of\ rank\ %
\%d)\\n"},\39\|z,\39\\{rep});{}$\6
\\{fflush}(\\{outfile});\6
\4${}\}{}$\2\6
\&{return} \T{0};\6
\4${}\}{}$\2\6
\4${}\}{}$\2\6
\&{return} \T{1};\6
\4${}\}{}$\2\par
\fi

\M[629 prime-sieve-sparse.w]{28}Miller and Rabin's algorithm is based on the
fact that
$x^q\equiv\pm1$ (modulo~\PB{\|z}) when \PB{\|z} is prime and $q=(z-1)/2$.
The loop above actually computes $(2^{64}(x/2^{64})^q)\bmod z$,
so the result should be $(\pm2^{64})\bmod z$.

Montgomery's method also needs the constant $z'$ such that $zz'\equiv1$
(modulo~$2^{64}$).

\Y\B\4\X28:Get ready for Montgomery's method\X${}\E{}$\6
\&{for} ${}(\\{lgz}\K\T{63},\39\|m\K\T{\^8000000000000000};{}$ ${}(\|m\AND\|z)%
\E\T{0};{}$ ${}\|m\MRL{{\GG}{\K}}\T{1},\39\\{lgz}\MM){}$\1\5
;\2\6
\&{for} ${}(\|k\K\\{lgz},\39\\{goal}\K\|m;{}$ ${}\|k<\T{64};{}$ ${}\|k\PP){}$\5
${}\{{}$\1\6
${}\\{goal}\MRL{+{\K}}\\{goal};{}$\6
\&{if} ${}(\\{goal}\G\|z){}$\1\5
${}\\{goal}\MRL{-{\K}}\|z;{}$\2\6
\4${}\}{}$\C{ now $\PB{\\{goal}}=2^{64}\bmod z$ }\2\6
\X29:Set \PB{\\{zp}} to the inverse of \PB{\|z} modulo $2^{64}$\X;\par
\U27.\fi

\M[645 prime-sieve-sparse.w]{29}Here I'm using ``Newton's method.'' (If $z%
\bmod4=1$, the first step
should be changed to \PB{$\\{zp}\K(\|z\AND\T{4}\?\|z\XOR\T{8}:\|z)$}.)

\Y\B\4\X29:Set \PB{\\{zp}} to the inverse of \PB{\|z} modulo $2^{64}$\X${}\E{}$%
\6
${}\{{}$\1\6
${}\\{zp}\K(\|z\AND\T{4}\?\|z:\|z\XOR\T{8}){}$;\C{ $zz'\equiv1$ (modulo $2^4$),
because $z\bmod4=3$ }\6
${}\\{zp}\K(\T{2}-\\{zp}*\|z)*\\{zp}{}$;\C{ now $zz'\equiv1$ (modulo $2^8$) }\6
${}\\{zp}\K(\T{2}-\\{zp}*\|z)*\\{zp}{}$;\C{ now $zz'\equiv1$ (modulo $2^{16}$)
}\6
${}\\{zp}\K(\T{2}-\\{zp}*\|z)*\\{zp}{}$;\C{ now $zz'\equiv1$ (modulo $2^{32}$)
}\6
${}\\{zp}\K(\T{2}-\\{zp}*\|z)*\\{zp}{}$;\C{ now $zz'\equiv1$ (modulo $2^{64}$)
}\6
\4${}\}{}$\2\par
\U28.\fi

\M[657 prime-sieve-sparse.w]{30}To compute $xy/2^{64}\bmod z$, we compute the
128-bit product $xy=
2^{64}t_1+t_0$, then subtract $(z't_0\bmod2^{64})z$ and return the
leading 64~bits.

\Y\B\4\X30:Set $y\gets (y^2\!/2^{64})\bmod z$\X${}\E{}$\6
${}\{{}$\1\6
${}\\{mult}(\|y,\39\|y);{}$\6
${}\|y\K\\{acc\_hi};{}$\6
${}\\{mult}(\\{zp}*\\{acc\_lo},\39\|z);{}$\6
\&{if} ${}(\|y<\\{acc\_hi}){}$\1\5
${}\|y\MRL{+{\K}}\|z-\\{acc\_hi};{}$\2\6
\&{else}\1\5
${}\|y\MRL{-{\K}}\\{acc\_hi};{}$\2\6
\4${}\}{}$\2\par
\U27.\fi

\M[670 prime-sieve-sparse.w]{31}\B\X31:Set $y\gets (xy/2^{64})\bmod z$\X${}%
\E{}$\6
${}\{{}$\1\6
${}\\{mult}(\|x,\39\|y);{}$\6
${}\|y\K\\{acc\_hi};{}$\6
${}\\{mult}(\\{zp}*\\{acc\_lo},\39\|z);{}$\6
\&{if} ${}(\|y<\\{acc\_hi}){}$\1\5
${}\|y\MRL{+{\K}}\|z-\\{acc\_hi};{}$\2\6
\&{else}\1\5
${}\|y\MRL{-{\K}}\\{acc\_hi};{}$\2\6
\4${}\}{}$\2\par
\U27.\fi

\M[679 prime-sieve-sparse.w]{32}The following simple test for nonprimality will
rule out most cases before
we need to resort to the Miller--Rabin scheme.
Algorithm 4.5.2B is a nice divisionless method to use here.
(Note that the product $3\cdot5\cdot\ldots\cdot53$ is between $2^{63}$
and $2^{64}$, so it would be considered ``negative'' as a \PB{\&{long} %
\&{long}}.)

\Y\B\4\D$\\{magic}$ \5
$((\T{3\$L\$L}*\T{5\$L\$L}*\T{7\$L\$L}*\T{11\$L\$L}*\T{13\$L\$L}*\T{17\$L\$L}*%
\T{19\$L\$L}*\T{23\$L\$L}*\T{29\$L\$L}*\T{31\$L\$L}*\T{37\$L\$L}*\T{41\$L\$L}*%
\T{43\$L\$L}*\T{47\$L\$L}*{}$(\&{unsigned} \&{long} \&{long}) \T{53})${}\GG%
\T{1}{}$)\par
\Y\B\4\X32:If $z$ is divisible by a prime $\le53$, \PB{\&{return} \T{0}}\X${}%
\E{}$\6
${}\{{}$\1\6
\&{long} \&{long} \|u${},{}$ \|v${},{}$ \|t;\7
${}\|t\K\\{magic}-(\|z\GG\T{1});{}$\6
${}\|v\K\|z;{}$\6
\4\.{B4}:\5
\&{while} ${}((\|t\AND\T{1})\E\T{0}){}$\1\5
${}\|t\MRL{{\GG}{\K}}\T{1};{}$\2\6
\4\.{B5}:\5
\&{if} ${}(\|t>\T{0}){}$\1\5
${}\|u\K\|t{}$;\5
\2\&{else}\1\5
${}\|v\K{-}\|t;{}$\2\6
\4\.{B6}:\5
${}\|t\K(\|u-\|v)/\T{2};{}$\6
\&{if} (\|t)\1\5
\&{goto} \.{B4};\2\6
\&{if} ${}(\|u>\T{1}){}$\1\5
\&{return} \T{0};\2\6
\4${}\}{}$\2\par
\U27.\fi

\N[700 prime-sieve-sparse.w]{1}{33}Index.
\fi

\inx
\fin
\con
