\input cwebmac
\datethis

\N{1}{1}Intro. A quickie to find a longest string that avoids the interesting
set of ``unavoidable'' $m$-ary strings of length $n$ constructed by
Mykkeltveit in 1972.

His construction can be viewed as finding the minimum number of arcs
to remove from the de Bruijn graph of $(n-1)$-tuples so that the
resulting graph has no oriented cycles. (Because each $n$-letter string
corresponds to an arc that must be avoided.)

This program constructs the graph and finds a longest path.

I hacked it from the previous program {\mc UNAVOIDABLE}, which uses
a different set of strings.

\Y\B\4\D$\|m$ \5
\T{2}\C{ this many letters in the alphabet }\par
\B\4\D$\|n$ \5
\T{20}\C{ this many letters in each string, assumed greater than 2 }\par
\B\4\D$\\{space}$ \5
$(\T{1}\LL(\|n-\T{1}){}$)\C{ $m^{n-1}$ }\par
\Y\B\8\#\&{include} \.{<stdio.h>}\6
\8\#\&{include} \.{<math.h>}\6
\&{char} ${}\\{avoid}[\|m*\\{space}]{}$;\C{ nonzero if the arc is removed }\6
\&{int} \\{deg}[\\{space}];\C{ outdegree, also used as pointer to next level }\6
\&{int} \\{link}[\\{space}];\C{ stack of vertices whose degree has dropped to
zero }\6
\&{int} ${}\|a[\|n+\T{1}]{}$;\C{ staging area }\6
\&{double} \\{sine}[\|n];\C{ imaginary parts of the $n$th roots of unity }\6
\&{int} \\{count};\C{ the number of vertices on the current level }\6
\&{int} \\{code};\C{ an $n$-tuple represented in $m$-ary notation }\7
\\{main}(\,)\1\1\2\2\6
${}\{{}$\1\6
\&{register} \&{int} \|d${},{}$ \|j${},{}$ \|k${},{}$ \|l${},{}$ \|q;\6
\&{register} \&{int} \\{top};\C{ top of the linked stack }\6
\&{double} \|u${}\K\T{2}*\T{3.1415926535897932385}/{}$(\&{double}) \|n;\6
\&{register} \&{double} \|s;\7
\&{for} ${}(\|j\K\T{0};{}$ ${}\|j<\|n;{}$ ${}\|j\PP){}$\1\5
${}\\{sine}[\|j]\K\\{sin}(\|j*\|u);{}$\2\6
\X2:Compute the \PB{\\{avoid}} and \PB{\\{deg}} tables\X;\6
\&{for} ${}(\|d\K\T{0};{}$ \\{count}; ${}\|d\PP){}$\5
${}\{{}$\1\6
${}\\{printf}(\.{"Vertices\ at\ distanc}\)\.{e\ \%d:\ \%d\\n"},\39\|d,\39%
\\{count});{}$\6
\&{for} ${}(\|l\K\\{top},\39\\{top}\K{-}\T{1},\39\\{count}\K\T{0};{}$ ${}\|l\G%
\T{0};{}$ ${}\|l\K\\{link}[\|l]){}$\1\5
\X5:Decrease the degree of \PB{\|l}'s predecessors, and stack them if their
degree drops to zero\X\2\6
\4${}\}{}$\2\6
\X6:Print out a longest path\X;\6
\4${}\}{}$\2\par
\fi

\M{2}Algorithm 7.2.1.1F gives us the relevant prime powers here.

\Y\B\4\X2:Compute the \PB{\\{avoid}} and \PB{\\{deg}} tables\X${}\E{}$\6
\&{for} ${}(\|j\K\T{0};{}$ ${}\|j<\\{space};{}$ ${}\|j\PP){}$\1\5
${}\\{deg}[\|j]\K\|m;{}$\2\6
${}\\{count}\K\|d\K\T{0};{}$\6
${}\\{top}\K{-}\T{1};{}$\6
\&{for} ${}(\|j\K\|n;{}$ \|j; ${}\|j\MM){}$\1\5
${}\|a[\|j]\K\T{0};{}$\2\6
${}\|a[\T{0}]\K{-}\T{1},\39\|j\K\T{1};{}$\6
\&{while} (\T{1})\5
${}\{{}$\1\6
\&{if} ${}(\|n\MOD\|j\E\T{0}){}$\1\5
\X3:Generate an $n$-tuple to avoid\X;\2\6
\&{for} ${}(\|j\K\|n;{}$ ${}\|a[\|j]\E\|m-\T{1};{}$ ${}\|j\MM){}$\1\5
;\2\6
\&{if} ${}(\|j\E\T{0}){}$\1\5
\&{break};\2\6
${}\|a[\|j]\PP;{}$\6
\&{for} ${}(\|k\K\|j+\T{1};{}$ ${}\|k\Z\|n;{}$ ${}\|k\PP){}$\1\5
${}\|a[\|k]\K\|a[\|k-\|j];{}$\2\6
\4${}\}{}$\2\6
${}\\{printf}(\.{"m=\%d,\ n=\%d:\ avoidin}\)\.{g\ one\ arc\ in\ each\ of}\)\.{\
\%d\ disjoint\ cycles\\}\)\.{n"},\39\|m,\39\|n,\39\|d){}$;\par
\U1.\fi

\M{3}At this point $\lambda=a_1\ldots a_j$ is a prime string
and $\alpha=a_1\ldots a_n=\lambda^{n/j}$.
The crux of Mykkeltveit's method is to compute an exponential sum
$s(a)=\sum a_j\omega^(j-1)$, where $\omega=e^{2\pi i/n}$, and to avoid the
``first'' cyclic shift of the \PB{\|a} array
for which the imaginary part of $s(a)$ is positive.
(If no such shift exists, an arbitrary shift is chosen.)

\Y\B\4\X3:Generate an $n$-tuple to avoid\X${}\E{}$\6
${}\{{}$\1\6
${}\|d\PP;{}$\6
\&{if} ${}(\|j<\|n){}$\1\5
${}\|q\K\|n;{}$\2\6
\&{else}\5
${}\{{}$\1\6
\&{for} ${}(\|q\K\T{1};{}$  ; ${}\|q\PP){}$\5
${}\{{}$\1\6
\&{for} ${}(\|l\K\T{1},\39\|s\K\T{0.0};{}$ ${}\|l\Z\|n;{}$ ${}\|l\PP){}$\1\5
${}\|s\MRL{+{\K}}\|a[\|l]*\\{sine}[(\|l-\T{1}+\|n-\|q)\MOD\|n];{}$\2\6
\&{if} ${}(\|s<\T{.0001}){}$\1\5
\&{break};\2\6
\4${}\}{}$\2\6
\&{for} ${}(\|q\PP;{}$ ${}\|q<\|n+\|n;{}$ ${}\|q\PP){}$\5
${}\{{}$\1\6
\&{for} ${}(\|l\K\T{1},\39\|s\K\T{0.0};{}$ ${}\|l\Z\|n;{}$ ${}\|l\PP){}$\1\5
${}\|s\MRL{+{\K}}\|a[\|l]*\\{sine}[(\|l-\T{1}+\|n+\|n-\|q)\MOD\|n];{}$\2\6
\&{if} ${}(\|s\G\T{.0001}){}$\1\5
\&{break};\2\6
\4${}\}{}$\2\6
\&{if} ${}(\|q>\|n){}$\1\5
${}\|q\MRL{-{\K}}\|n;{}$\2\6
\4${}\}{}$\2\6
\&{for} ${}(\\{code}\K\T{0},\39\|k\K\|q+\T{1};{}$ ${}\|k\Z\|n;{}$ ${}\|k\PP){}$%
\1\5
${}\\{code}\K\|m*\\{code}+\|a[\|k];{}$\2\6
\&{for} ${}(\|k\K\T{1};{}$ ${}\|k\Z\|q;{}$ ${}\|k\PP){}$\1\5
${}\\{code}\K\|m*\\{code}+\|a[\|k];{}$\2\6
\X4:Avoid the $n$-tuple encoded by \PB{\\{code}}\X;\6
\4${}\}{}$\2\par
\U2.\fi

\M{4}\B\X4:Avoid the $n$-tuple encoded by \PB{\\{code}}\X${}\E{}$\6
$\\{avoid}[\\{code}]\K\T{1};{}$\6
${}\|q\K\\{code}/\|m;{}$\6
${}\\{deg}[\|q]\MM;{}$\6
\&{if} ${}(\\{deg}[\|q]\E\T{0}){}$\1\5
${}\\{deg}[\|q]\K{-}\T{1},\39\\{link}[\|q]\K\\{top},\39\\{top}\K\|q,\39%
\\{count}\PP{}$;\2\par
\U3.\fi

\M{5}\B\X5:Decrease the degree of \PB{\|l}'s predecessors, and stack them if
their degree drops to zero\X${}\E{}$\6
\&{for} ${}(\|j\K\|m-\T{1};{}$ ${}\|j\G\T{0};{}$ ${}\|j\MM){}$\5
${}\{{}$\1\6
${}\|k\K\|l+\|j*\\{space};{}$\6
\&{if} ${}(\R\\{avoid}[\|k]){}$\5
${}\{{}$\1\6
${}\|q\K\|k/\|m;{}$\6
${}\\{deg}[\|q]\MM;{}$\6
\&{if} ${}(\\{deg}[\|q]\E\T{0}){}$\1\5
${}\\{deg}[\|q]\K\|l,\39\\{link}[\|q]\K\\{top},\39\\{top}\K\|q,\39\\{count}%
\PP;{}$\2\6
\4${}\}{}$\2\6
\4${}\}{}$\2\par
\U1.\fi

\M{6}Here I apologize for using a dirty trick:
The current value of \PB{\|k} happens to be the most recent value of \PB{\|l},
a vertex with no predecessors.

\Y\B\4\X6:Print out a longest path\X${}\E{}$\6
\\{printf}(\.{"Path:"});\6
\&{for} ${}(\\{code}\K\|k,\39\|j\K\T{1};{}$ ${}\|j<\|n;{}$ ${}\|j\PP){}$\5
${}\{{}$\1\6
${}\\{code}\K\\{code}*\|m,\39\|q\K\\{code}/\\{space};{}$\6
${}\\{printf}(\.{"\ \%d"},\39\|q);{}$\6
${}\\{code}\MRL{-{\K}}\|q*\\{space};{}$\6
\4${}\}{}$\2\6
\&{while} ${}(\\{deg}[\|k]\G\T{0}){}$\5
${}\{{}$\1\6
${}\\{printf}(\.{"\ \%d"},\39\\{deg}[\|k]\MOD\|m);{}$\6
${}\|k\K\\{deg}[\|k];{}$\6
\4${}\}{}$\2\par
\U1.\fi

\N{1}{7}Index.








\fi


\inx
\fin
\con
