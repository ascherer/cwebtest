\input cwebmac
\datethis

\N{1}{1}Intro. This program contains two implementations of the Koda--Ruskey
algorithm for generating all ideals of a given forest poset
[{\sl Journal of Algorithms\/ \bf15} (1993), 324--340].
The common goal of both implementations is, in essence, to generate all
binary strings $b_0\ldots b_{n-1}$ in which certain bits are
required to be less than or equal to specified bits that lie to their right.
(For some values of $j$ there is a value of $k>j$ such that
we don't allow $b_k$ to be 0 when $b_j=1$.)
Moreover, each binary string should differ from its predecessor in exactly
one bit position; the algorithm therefore
defines a generalized reflected Gray code.

The given forest is represented by $n$ pairs of nested parentheses. For
example, \.{()()()()()} represents five independent bits, while
\.{((((()))))} represents five bits with $b_0\le b_1\le b_2\le b_3\le b_4$.
A more interesting example, \.{((())(()()))}, represents six bits subject
to the conditions
$b_0\le b_1$, $b_1\le b_5$, $b_2\le b_4$, $b_3\le b_4$, $b_4\le b_5$.
Each pair of parentheses corresponds to a bit that must not exceed the
bit of its enclosing pair, if any, and the pairs are ordered by the
appearances of their right parentheses.

The first implementation uses $n$ coroutines, which call each other
in a hierarchical fashion. The second uses
multilinked data structures in a loopless way, so that each generation step
performs a bounded number of operations to obtain the next element.
I couldn't resist writing this program, because both implementations turn out
to be quite interesting and instructive.

Indeed, I think it's a worthwhile challenge for people who study the
science of computer programming to verify that these two implementations
both define the same sequence of bitstrings. Even more challenging
would be to derive the second implementation ``automatically'' from
the first.

\Y\B\8\#\&{include} \.{<stdio.h>}\6
\X4:Type definitions\X\6
\X3:Global variables\X\7
\&{int} \\{main}(\&{int} \\{argc}${},\39{}$\&{char} ${}{*}\\{argv}[\,]){}$\1\1%
\2\2\6
${}\{{}$\1\6
\&{register} \&{int} \|j${},{}$ \|k${},{}$ \|l;\7
\X2:Process the command line, parsing the given forest\X;\6
${}\\{printf}(\.{"Bitstrings\ generate}\)\.{d\ from\ \\"\%s\\":\\n"},\39%
\\{argv}[\T{1}]);{}$\6
\X9:Generate the strings with a coroutine implementation\X;\6
\\{printf}(\.{"\\nTrying\ again,\ loo}\)\.{plessly:\\n"});\6
\X15:Generate the strings with a loopless implementation\X;\6
\&{return} \T{0};\6
\4${}\}{}$\2\par
\fi

\M{2}In this step we parse the forest into an array of ``scopes'':
\PB{\\{scope}[\|j]} is the index of the smallest descendant of node~\PB{\|j},
including node~\PB{\|j} itself.

\Y\B\4\D$\\{abort}(\|m,\|i)$ \6
${}\{{}$\5
\1${}\\{fprintf}(\\{stderr},\39\|m,\39\\{argv}[\|i]){}$;\5
\&{return} ${}{-}\T{1}{}$;\5
${}\}{}$\2\par
\B\4\D$\\{stacksize}$ \5
\T{100}\C{ max levels in the forest }\par
\B\4\D$\\{forestsize}$ \5
\T{100}\C{ max nodes in the forest }\par
\Y\B\4\X2:Process the command line, parsing the given forest\X${}\E{}$\6
\&{if} ${}(\\{argc}\I\T{2}\V\\{argv}[\T{1}][\T{0}]\I\.{'('}){}$\1\5
${}\\{abort}(\.{"Usage:\ \%s\ \\"nestedp}\)\.{arens\\"\\n"},\39\T{0});{}$\2\6
\&{for} ${}(\|j\K\|k\K\|l\K\T{0};{}$ \\{argv}[\T{1}][\|k]; ${}\|k\PP){}$\1\6
\&{if} ${}(\\{argv}[\T{1}][\|k]\E\.{'('}){}$\5
${}\{{}$\1\6
${}\\{stack}[\|l\PP]\K\|j;{}$\6
\&{if} ${}(\|l\E\\{stacksize}){}$\1\5
${}\\{abort}(\.{"Stack\ overflow\ ---\ }\)\.{\\"\%s\\"\ is\ too\ deep\ f}\)%
\.{or\ me!\\n"},\39\T{1});{}$\2\6
\4${}\}{}$\5
\2\&{else} \&{if} ${}(\\{argv}[\T{1}][\|k]\E\.{')'}){}$\5
${}\{{}$\1\6
\&{if} ${}(\MM\|l<\T{0}){}$\1\5
${}\\{abort}(\.{"Extra\ right\ parenth}\)\.{esis\ in\ \\"\%s\\"!\\n"},\39%
\T{1});{}$\2\6
${}\\{scope}[\|j\PP]\K\\{stack}[\|l];{}$\6
\&{if} ${}(\|j\E\\{forestsize}){}$\1\5
${}\\{abort}(\.{"Memory\ overflow\ ---}\)\.{\ \\"\%s\\"\ is\ too\ big!\\}\)%
\.{n"},\39\T{1});{}$\2\6
\4${}\}{}$\5
\2\&{else}\1\5
${}\\{abort}(\.{"The\ forest\ spec\ \\"\%}\)\.{s\\"\ should\ contain\ o}\)%
\.{nly\ parentheses!\\n"},\39\T{1});{}$\2\2\6
\&{if} (\|l)\1\5
${}\\{abort}(\.{"Missing\ right\ paren}\)\.{thesis\ in\ \\"\%s\\"!\\n"},\39%
\T{1});{}$\2\6
${}\\{nn}\K\|j{}$;\par
\U1.\fi

\M{3}\B\X3:Global variables\X${}\E{}$\6
\&{int} \\{stack}[\\{stacksize}];\C{ nodes preceding each open leftparen, while
parsing }\6
\&{int} \\{scope}[\\{forestsize}];\C{ table that exhibits each rightparen's
influence }\6
\&{int} \\{nn};\C{ the actual number of nodes in the forest }\par
\As11\ET17.
\U1.\fi

\N{1}{4}The coroutine implementation. Our first implementation uses a system
of $n$ cooperating programs, each of which represents a node in the forest.
For convenience we will call the associated record a ``cnode.''
If \PB{\|p} points to a cnode, \PB{$\|p\MG\\{child}$} points to the cnode
representing
its rightmost child, and \PB{$\|p\MG\\{sib}$} points to the cnode representing
its nearest sibling on the left, in the given forest.

Each cnode corresponds to a coroutine whose job is to generate all the ideals
of the subforest it represents. Whenever the coroutine is invoked, it
either changes one of the bits in its scope and returns \PB{\\{true}}, or
it changes nothing and returns \PB{\\{false}}. Initially all the bits are~0;
when it first returns \PB{\\{false}}, it will have generated all legitimate
bit patterns, ending with some nonzero pattern. Subsequently it will
generate the patterns again in reverse order, ending with all 0s,
after which it will return \PB{\\{false}} a second time. Invoking it again and
again will repeat the same process, going forwards and backwards, ad infinitum.

Each coroutine has the same basic structure, which can be described
as follows in an ad hoc extension of \CEE/ language:
$$\vbox{\halign{#\hfil\cr
\&{coroutine} \PB{\|p(\,)}\cr
\quad\PB{$\{$}\cr
\quad\quad\PB{\&{while} (\T{1}) $\{$}\cr
\quad\quad\quad\PB{$\|p\MG\\{bit}\K\T{1}{}$; \&{return} \\{true};}\cr
\quad\quad\quad\PB{ \&{while} ${}(\|p\MG\\{child}(\,)){}$ \&{return} \\{true};%
\2}\cr
\quad\quad\quad\PB{\&{return} \|p${}\MG\\{sib}(\,);$}\cr
\quad\quad\quad\PB{ \&{while} ${}(\|p\MG\\{child}(\,)){}$ \&{return} \\{true};%
\2}\cr
\quad\quad\quad\PB{$\|p\MG\\{bit}\K\T{0}{}$; \&{return} \\{true};}\cr
\quad\quad\quad\PB{\&{return} \|p${}\MG\\{sib}(\,);$}\cr
\quad\quad\PB{$\}$}\cr
\quad\PB{$\}$}\cr
}}$$
If either \PB{$\|p\MG\\{child}$} or \PB{$\|p\MG\\{sib}$} is \PB{$\NULL$}, the
corresponding
coroutine \PB{$\NULL(\,)$} is assumed to simply return \PB{\\{false}}.

Suppose \PB{$\|p\MG\\{child}(\,)$} first returns \PB{\\{false}} after it has
been called \PB{\|r} times;
thus \PB{$\|p\MG\\{child}(\,)$} generates \PB{\|r} different patterns,
including the
initial pattern of all~0s. Similarly, suppose that \PB{$\|p\MG\\{sib}(\,)$}
generates \PB{\|l}
different patterns before first returning \PB{\\{false}}. Then the coroutine %
\PB{\|p(\,)}
itself will generate \PB{$\|l(\|r+\T{1})$} patterns in between the times when
it
returns \PB{\\{false}}.
The final bit pattern for \PB{\|p} will be the final bit pattern
for \PB{$\|p\MG\\{sib}$}, together with either
\PB{$\|p\MG\\{bit}\K\T{1}$} and the final bit pattern of \PB{$\|p\MG\\{child}$}
(if \PB{\|l} is odd)
or with \PB{$\|p\MG\\{bit}\K\T{0}$} and all 0s in \PB{$\|p\MG\\{child}$} (if %
\PB{\|l} is even).

\Y\B\4\X4:Type definitions\X${}\E{}$\6
\&{typedef} \&{enum} ${}\{{}$\1\6
${}\\{false},\39\\{true}{}$\2\6
${}\}{}$ \&{boolean};\7
\&{typedef} \&{struct} \&{cnode\_struct} ${}\{{}$\1\6
\&{char} \\{bit};\C{ either 0 or 1; always 1 when a child's bit is set }\6
\&{char} \\{state};\C{ the current place in this cnode's coroutine }\6
\&{struct} \&{cnode\_struct} ${}{*}\\{child}{}$;\C{ rightmost child in the
given forest }\6
\&{struct} \&{cnode\_struct} ${}{*}\\{sib}{}$;\C{ nearest left sibling in the
given forest }\6
\&{struct} \&{cnode\_struct} ${}{*}\\{caller}{}$;\C{ which coroutine invoked
this one }\2\6
${}\}{}$ \&{cnode};\par
\A14.
\U1.\fi

\M{5}When coroutine \PB{\|p} calls coroutine \PB{\|q}, it sets \PB{$\|p\MG%
\\{state}$} to an
appropriate number and also sets \PB{$\|q\MG\\{caller}\K\|p$}. Then control
passes
to \PB{\|q} at the place determined by \PB{$\|q\MG\\{state}$}.

When coroutine \PB{\|q} wants to return a boolean value, it sets \PB{%
\\{coresult}} to this
value; then it passes control to \PB{$\|p\K\|q\MG\\{caller}$} at the place
determined
by \PB{$\|p\MG\\{state}$}.

This program simulates coroutine linkage with a big switch statement.
Actually the notion of ``passing control'' really means that we simply
assign a value to the variable \PB{\\{cur\_cnode}}.

The value of \PB{$\|q\MG\\{caller}$} for every cnode \PB{\|q} is completely
determined by the
structure of the given forest, so we could set it once and for all during
the initialization instead of setting it dynamically as done here.
But what the heck.

\Y\B\4\D$\\{cocall}(\|q,\|s)$ \6
${}\{{}$\5
\1${}\\{cur\_cnode}\MG\\{state}\K\|s;{}$\6
\&{if} (\|q)\1\5
${}\|q\MG\\{caller}\K\\{cur\_cnode},\39\\{cur\_cnode}\K\|q;{}$\2\6
\&{else}\1\5
${}\\{coresult}\K\\{false};{}$\2\6
\&{goto} \\{cogo};\5
${}\}{}$\2\par
\B\4\D$\\{bitchange}(\|b,\|s)$ \6
${}\{{}$\5
\1${}\\{cur\_cnode}\MG\\{bit}\K\|b,\39\\{coresult}\K\\{true}{}$;\5
\\{coreturn}(\|s);\5
${}\}{}$\2\par
\B\4\D$\\{coreturn}(\|s)$ \6
${}\{{}$\5
\1${}\\{cur\_cnode}\MG\\{state}\K\|s,\39\\{cur\_cnode}\K\\{cur\_cnode}\MG%
\\{caller};{}$\6
\&{goto} \\{cogo};\5
${}\}{}$\2\par
\Y\B\4\X5:Repeatedly switch to the proper part of the current coroutine\X${}%
\E{}$\6
\4\\{cogo}:\5
\&{switch} ${}(\\{cur\_cnode}\MG\\{state}){}$\5
${}\{{}$\1\6
\X6:Cases for coroutine states\X;\6
\4\&{default}:\5
${}\\{abort}(\.{"\%s:\ Unknown\ state\ c}\)\.{ode\ (this\ can't\ happ}\)\.{en)!%
\\n"},\39\T{0});{}$\6
\4${}\}{}$\2\par
\U9.\fi

\M{6}In its initial state~0, a coroutine turns its bit on, returns~\PB{%
\\{true}},
and enters state~1.

\Y\B\4\X6:Cases for coroutine states\X${}\E{}$\6
\4\&{case} \T{0}:\5
${}\\{bitchange}(\T{1},\39\T{1}){}$;\par
\As7, 8\ETs12.
\U5.\fi

\M{7}The purpose of state 1 is to run through all bit patterns of the
current node's children, starting with all 0s and ending when they reach their
final pattern.
At that point we invoke the current node's nearest left sibling
and enter state~3. An intermediate state~2 is defined for the purpose
of examining the result after calling the child coroutine.

The purpose of state 3 is simply to return to whoever called us,
passing along the information in \PB{\\{coresult}}, which tells whether
any of our left siblings has changed one of its bits. Then we will
continue in state 4.

\Y\B\4\X6:Cases for coroutine states\X${}\mathrel+\E{}$\6
\4\&{case} \T{1}:\5
${}\\{cocall}(\\{cur\_cnode}\MG\\{child},\39\T{2});{}$\6
\4\&{case} \T{2}:\5
\&{if} (\\{coresult})\1\5
\\{coreturn}(\T{1});\2\6
${}\\{cocall}(\\{cur\_cnode}\MG\\{sib},\39\T{3});{}$\6
\4\&{case} \T{3}:\5
\\{coreturn}(\T{4});\par
\fi

\M{8}State 4 is rather like state 1, except that the child coroutine is now
running through its bit patterns in reverse order. Finally it reduces them
all to 0s, and returns \PB{\\{false}} the next time we attempt to invoke it.
At that point we reset the current bit, return \PB{\\{true}}, and enter
state~6.

State 6 invokes the sibling coroutine, leading to state 7.
And state 7 is like state 3, but it takes us back to state 0 instead
of state 4.

\Y\B\4\X6:Cases for coroutine states\X${}\mathrel+\E{}$\6
\4\&{case} \T{4}:\5
${}\\{cocall}(\\{cur\_cnode}\MG\\{child},\39\T{5});{}$\6
\4\&{case} \T{5}:\5
\&{if} (\\{coresult})\1\5
\\{coreturn}(\T{4});\2\6
${}\\{bitchange}(\T{0},\39\T{6});{}$\6
\4\&{case} \T{6}:\5
${}\\{cocall}(\\{cur\_cnode}\MG\\{sib},\39\T{7});{}$\6
\4\&{case} \T{7}:\5
\\{coreturn}(\T{0});\par
\fi

\M{9}Hey, the implementation is done already, except that we have to
get it started and write the code that controls it at the outermost level.

\Y\B\4\X9:Generate the strings with a coroutine implementation\X${}\E{}$\6
${}\{{}$\1\6
\&{register} \&{cnode} ${}{*}\\{cur\_cnode};{}$\7
\X10:Initialize the cnode structure\X;\6
\X5:Repeatedly switch to the proper part of the current coroutine\X;\6
\4${}\}{}$\2\par
\U1.\fi

\M{10}We allocate a special cnode to represent the external world outside of
the
given forest.

\Y\B\4\D$\\{root\_cnode}$ \5
$\\{cnode\_table}[\\{nn}].{}$\\{child}\par
\Y\B\4\X10:Initialize the cnode structure\X${}\E{}$\6
$\\{scope}[\\{nn}]\K\T{0};{}$\6
\&{for} ${}(\|k\K\T{0};{}$ ${}\|k\Z\\{nn};{}$ ${}\|k\PP){}$\1\6
\&{if} ${}(\\{scope}[\|k]<\|k){}$\5
${}\{{}$\1\6
${}\\{cnode\_table}[\|k].\\{child}\K\\{cnode\_table}+\|k-\T{1};{}$\6
\&{for} ${}(\|j\K\|k-\T{1};{}$ ${}\\{scope}[\|j]>\\{scope}[\|k];{}$ ${}\|j\K%
\\{scope}[\|j]-\T{1}){}$\1\5
${}\\{cnode\_table}[\|j].\\{sib}\K\\{cnode\_table}+\\{scope}[\|j]-\T{1};{}$\2\6
\4${}\}{}$\2\2\6
${}\\{cur\_cnode}\K\\{cnode\_table}+\\{nn};{}$\6
\&{goto} \\{upward\_step};\par
\U9.\fi

\M{11}\B\X3:Global variables\X${}\mathrel+\E{}$\6
\&{cnode} ${}\\{cnode\_table}[\\{forestsize}+\T{1}]{}$;\C{ the cnodes }\6
\&{boolean} \\{coresult};\C{ value returned by a coroutine }\par
\fi

\M{12}States 8 and greater are reserved for the external (outermost) level,
which
simply invokes the coroutine for the entire forest and prints out the
results, until the bit patterns have been generated in both the
forward and reverse directions.

\Y\B\4\X6:Cases for coroutine states\X${}\mathrel+\E{}$\6
\4\&{case} \T{8}:\5
\&{if} (\\{coresult})\5
${}\{{}$\1\6
\4\\{upward\_step}:\5
\X13:Print out all the current cnode bits\X;\6
${}\\{cocall}(\\{root\_cnode},\39\T{8});{}$\6
\4${}\}{}$\2\6
\\{printf}(\.{"...\ and\ now\ we\ gene}\)\.{rate\ them\ in\ reverse}\)\.{:%
\\n"});\6
\&{goto} \\{downward\_step};\6
\4\&{case} \T{9}:\5
\&{if} (\\{coresult})\5
${}\{{}$\1\6
\4\\{downward\_step}:\5
\X13:Print out all the current cnode bits\X;\6
${}\\{cocall}(\\{root\_cnode},\39\T{9});{}$\6
\4${}\}{}$\2\6
\&{break};\par
\fi

\M{13}\B\X13:Print out all the current cnode bits\X${}\E{}$\6
\&{for} ${}(\|k\K\T{0};{}$ ${}\|k<\\{nn};{}$ ${}\|k\PP){}$\1\5
${}\\{putchar}(\.{'0'}+\\{cnode\_table}[\|k].\\{bit});{}$\2\6
\\{putchar}(\.{'\\n'});\par
\U12.\fi

\N{1}{14}The loopless implementation. Our coroutine implementation solves
the generation problem in a nice and natural fashion, but it can be
inefficient if the given forest has numerous nodes of degree one.
For example, a one-tree forest like \.{((...()...))} with $n$ pairs
of parentheses will need approximately $n\choose2$ coroutine invocations
to generate $n+1$ bitstrings.

Our second implementation reduces the work in such cases to $O(n)$; in
fact, it needs only a bounded number of operations to generate each
bitstring after the first. It does, however, need a slightly more complex data
structure with four link fields.

The basic idea is to work with a dynamically varying list of nodes called
the current {\it fringe\/} of the forest. The fringe consists of all
node whose bit is~1, together with their children.
We maintain it as a doubly linked list,
so that \PB{$\|p\MG\\{left}$} and \PB{$\|p\MG\\{right}$} are
the neighbors of \PB{\|p} on the left and right. A special node \PB{\\{head}}
is
provided to make the list circular; thus \PB{$\\{head}\MG\\{right}$} and \PB{$%
\\{head}\MG\\{left}$}
are the leftmost and rightmost fringe nodes.

A fringe node is said to be said to be either {\it active\/} or {\it passive}.
Every node is active when it joins the fringe, but it becomes passive for at
least a short time when its bit changes value; at such times the node is
essentially shifting direction between going forward or backward, as
in the coroutine implementation. (A passive node corresponds roughly to
a coroutine that is asking its siblings to make the next move.)
We save time jumping across such call-chains by using a special link
field called the \PB{\\{focus}}: If \PB{\|p} is a passive fringe node whose
righthand
neighbor \PB{$\|p\MG\\{right}$} is active, \PB{$\|p\MG\\{focus}$} is the
rightmost active
node to the left of~\PB{\|p} in the fringe; otherwise \PB{$\|p\MG\\{focus}\K%
\|p$}. (The
special \PB{\\{head}} node is always considered to be active, for purposes of
this definition, but it is not strictly speaking a member of the fringe.)

The loopless implementation works with records called lnodes, just as
the coroutine implementation worked with cnodes.
Besides the dynamic \PB{\\{bit}} and \PB{\\{left}} and \PB{\\{right}} and \PB{%
\\{focus}} fields
already mentioned, each lnode also has a static field called
\PB{\\{lchild}}, representing its leftmost child.
(There is no need for an \PB{\\{rchild}} field, since \PB{$\|p\MG\\{rchild}\K%
\|p-\T{1}$} when
\PB{$\|p\MG\\{lchild}\I\NULL$}.)

If \PB{\|p} is not in the fringe, \PB{$\|p\MG\\{focus}$} should equal \PB{\|p}.
Also, \PB{$\|p\MG\\{left}$} and \PB{$\|p\MG\\{right}$} are assumed to
equal the nearest siblings of \PB{\|p} to the left and right, respectively,
if such siblings exist; otherwise \PB{$\|p\MG\\{left}$} and/or \PB{$\|p\MG%
\\{right}$} are undefined.

\Y\B\4\X4:Type definitions\X${}\mathrel+\E{}$\6
\&{typedef} \&{struct} \&{lnode\_struct} ${}\{{}$\1\6
\&{char} \\{bit};\C{ either 0 or 1; always 1 when a child's bit is set }\6
\&{struct} \&{lnode\_struct} ${}{*}\\{left},{}$ ${}{*}\\{right}{}$;\C{
neighbors in the forest and/or fringe }\6
\&{struct} \&{lnode\_struct} ${}{*}\\{lchild}{}$;\C{ leftmost child }\6
\&{struct} \&{lnode\_struct} ${}{*}\\{focus}{}$;\C{ red-tape cutter for
efficiency }\2\6
${}\}{}$ \&{lnode};\par
\fi

\M{15}Here now is the basic outline of the loopless implementation:

\Y\B\4\X15:Generate the strings with a loopless implementation\X${}\E{}$\6
${}\{{}$\1\6
\&{register} \&{lnode} ${}{*}\|p,{}$ ${}{*}\|q,{}$ ${}{*}\|r;{}$\7
\X16:Initialize the lnode structure, putting all roots into the fringe\X;\6
\&{while} (\T{1})\5
${}\{{}$\1\6
\X22:Print out all the current lnode bits\X;\6
\X18:Set \PB{\|p} to the rightmost active node of the fringe, and activate
everything to its right\X;\6
\&{if} ${}(\|p\I\\{head}){}$\5
${}\{{}$\1\6
\&{if} ${}(\|p\MG\\{bit}\E\T{0}){}$\5
${}\{{}$\1\6
${}\|p\MG\\{bit}\K\T{1}{}$;\C{ moving forward }\6
\X19:Insert the children of \PB{\|p} after \PB{\|p} in the fringe\X;\6
\4${}\}{}$\5
\2\&{else}\5
${}\{{}$\1\6
${}\|p\MG\\{bit}\K\T{0}{}$;\C{ moving backward }\6
\X20:Delete the children of \PB{\|p} from the fringe\X;\6
\4${}\}{}$\2\6
\4${}\}{}$\5
\2\&{else} \&{if} (\\{been\_there\_and\_done\_that})\1\5
\&{break};\2\6
\&{else}\5
${}\{{}$\1\6
\\{printf}(\.{"...\ and\ now\ we\ gene}\)\.{rate\ them\ in\ reverse}\)\.{:%
\\n"});\6
${}\\{been\_there\_and\_done\_that}\K\\{true}{}$;\5
\&{continue};\6
\4${}\}{}$\2\6
\X21:Make node \PB{\|p} passive\X;\6
\4${}\}{}$\2\6
\4${}\}{}$\2\par
\U1.\fi

\M{16}Initialization of the lnodes is similar to initialization of the cnodes,
but more links need to be set up.

\Y\B\4\D$\\{head}$ \5
$(\\{lnode\_table}+\\{nn}{}$)\par
\Y\B\4\X16:Initialize the lnode structure, putting all roots into the fringe%
\X${}\E{}$\6
\&{for} ${}(\|k\K\T{0};{}$ ${}\|k\Z\\{nn};{}$ ${}\|k\PP){}$\5
${}\{{}$\1\6
${}\\{lnode\_table}[\|k].\\{focus}\K\\{lnode\_table}+\|k;{}$\6
\&{if} ${}(\\{scope}[\|k]<\|k){}$\5
${}\{{}$\1\6
\&{for} ${}(\|j\K\|k-\T{1};{}$ ${}\\{scope}[\|j]>\\{scope}[\|k];{}$ ${}\|j\K%
\\{scope}[\|j]-\T{1}){}$\5
${}\{{}$\1\6
${}\\{lnode\_table}[\|j].\\{left}\K\\{lnode\_table}+\\{scope}[\|j]-\T{1};{}$\6
${}\\{lnode\_table}[\\{scope}[\|j]-\T{1}].\\{right}\K\\{lnode\_table}+\|j;{}$\6
\4${}\}{}$\2\6
${}\\{lnode\_table}[\|k].\\{lchild}\K\\{lnode\_table}+\|j;{}$\6
\4${}\}{}$\2\6
\4${}\}{}$\2\6
${}\\{head}\MG\\{left}\K\\{head}-\T{1},\39(\\{head}-\T{1})\MG\\{right}\K%
\\{head};{}$\6
${}\\{head}\MG\\{right}\K\\{head}\MG\\{lchild},\39\\{head}\MG\\{lchild}\MG%
\\{left}\K\\{head}{}$;\par
\U15.\fi

\M{17}\B\X3:Global variables\X${}\mathrel+\E{}$\6
\&{lnode} ${}\\{lnode\_table}[\\{forestsize}+\T{1}]{}$;\C{ the lnodes }\6
\&{boolean} \\{been\_there\_and\_done\_that};\par
\fi

\M{18}\B\X18:Set \PB{\|p} to the rightmost active node of the fringe, and
activate everything to its right\X${}\E{}$\6
$\|q\K\\{head}\MG\\{left};{}$\6
${}\|p\K\|q\MG\\{focus};{}$\6
${}\|q\MG\\{focus}\K\|q{}$;\par
\U15.\fi

\M{19}\B\X19:Insert the children of \PB{\|p} after \PB{\|p} in the fringe\X${}%
\E{}$\6
\&{if} ${}(\|p\MG\\{lchild}){}$\5
${}\{{}$\1\6
${}\|q\K\|p\MG\\{right};{}$\6
${}\|q\MG\\{left}\K\|p-\T{1},\39(\|p-\T{1})\MG\\{right}\K\|q;{}$\6
${}\|p\MG\\{right}\K\|p\MG\\{lchild},\39\|p\MG\\{lchild}\MG\\{left}\K\|p;{}$\6
\4${}\}{}$\2\par
\U15.\fi

\M{20}\B\X20:Delete the children of \PB{\|p} from the fringe\X${}\E{}$\6
\&{if} ${}(\|p\MG\\{lchild}){}$\5
${}\{{}$\1\6
${}\|q\K(\|p-\T{1})\MG\\{right};{}$\6
${}\|p\MG\\{right}\K\|q,\39\|q\MG\\{left}\K\|p;{}$\6
\4${}\}{}$\2\par
\U15.\fi

\M{21}At this point we know that \PB{$\|p\MG\\{right}$} is active.

\Y\B\4\X21:Make node \PB{\|p} passive\X${}\E{}$\6
$\|p\MG\\{focus}\K\|p\MG\\{left}\MG\\{focus};{}$\6
${}\|p\MG\\{left}\MG\\{focus}\K\|p\MG\\{left}{}$;\par
\U15.\fi

\M{22}\B\X22:Print out all the current lnode bits\X${}\E{}$\6
\&{for} ${}(\|k\K\T{0};{}$ ${}\|k<\\{nn};{}$ ${}\|k\PP){}$\1\5
${}\\{putchar}(\.{'0'}+\\{lnode\_table}[\|k].\\{bit});{}$\2\6
\\{putchar}(\.{'\\n'});\par
\U15.\fi

\M{23}I used the following code when debugging.

\Y\B\4\D$\\{rel}(\|f)$ \5
$(\\{lnode\_table}[\|k].\|f\?\\{lnode\_table}[\|k].\|f-\\{lnode\_table}:{-}%
\T{1}{}$)\par
\Y\B\4\X23:Print out the whole lnode structure\X${}\E{}$\6
\&{for} ${}(\|k\K\T{0};{}$ ${}\|k\Z\\{nn};{}$ ${}\|k\PP){}$\5
${}\{{}$\1\6
${}\\{printf}(\.{"lnode\ \%d:\ bit=\%d,\ "},\39\|k,\39\\{lnode\_table}[\|k].%
\\{bit});{}$\6
${}\\{printf}(\.{"focus=\%d,\ left=\%d,\ }\)\.{right=\%d,\ lchild=\%d\\}\)%
\.{n"},\39\\{rel}(\\{focus}),\39\\{rel}(\\{left}),\39\\{rel}(\\{right}),\39%
\\{rel}(\\{lchild}));{}$\6
\4${}\}{}$\2\par
\fi

\N{1}{24}Index.

\fi


\inx
\fin
\con
