\input cwebmac
\datethis




\N{1}{1}Introduction. This program inputs a directed graph.
It outputs a not-necessarily-reduced binary decision diagram
for the family of all simple oriented cycles in that graph.

The format of the output is described in another program,
{\mc SIMPATH-REDUCE}. Let me just say here that it is intended
only for computational convenience, not for human readability.

I've tried to make this program simple, whenever I had to
choose between simplicity and efficiency. But I haven't gone
out of my way to be inefficient.

(I hacked this code by extending {\mc SIMPATH}, the undirected version.)

\Y\B\4\D$\\{maxn}$ \5
\T{90}\C{ maximum number of vertices; at most 126 }\par
\B\4\D$\\{maxm}$ \5
\T{2000}\C{ maximum number of arcs }\par
\B\4\D$\\{logmemsize}$ \5
\T{27}\par
\B\4\D$\\{memsize}$ \5
$(\T{1}\LL\\{logmemsize}{}$)\C{ warning: we need $\PB{\\{maxn}}*\PB{%
\\{memsize}}\le2^{32}$ }\par
\B\4\D$\\{loghtsize}$ \5
\T{24}\par
\B\4\D$\\{htsize}$ \5
$(\T{1}\LL\\{loghtsize}{}$)\par
\Y\B\8\#\&{include} \.{<stdio.h>}\6
\8\#\&{include} \.{<stdlib.h>}\6
\8\#\&{include} \.{<string.h>}\6
\8\#\&{include} \.{"gb\_graph.h"}\6
\8\#\&{include} \.{"gb\_save.h"}\6
\&{char} \\{mem}[\\{memsize}];\C{ the big workspace }\6
\&{unsigned} \&{long} \&{long} \\{tail}${},{}$ \\{boundary}${},{}$ \\{head};\C{
queue pointers }\6
\&{unsigned} \&{int} \\{htable}[\\{htsize}];\C{ hash table }\6
\&{unsigned} \&{int} \\{htid};\C{ ``time stamp'' for hash entries }\6
\&{int} \\{htcount};\C{ number of entries in the hash table }\6
\&{Vertex} ${}{*}\\{vert}[\\{maxn}+\T{1}];{}$\6
\&{int} \\{arcto}[\\{maxm}];\C{ destination number of each arc }\6
\&{int} ${}\\{firstarc}[\\{maxn}+\T{2}]{}$;\C{ where arcs from a vertex start
in \PB{\\{arcto}} }\6
\&{char} ${}\\{mate}[\\{maxn}+\T{3}]{}$;\C{ encoded state }\6
\&{int} \\{serial}${},{}$ \\{newserial};\C{ state numbers }\7
\X12:Subroutines\X\7
\\{main}(\&{int} \\{argc}${},\39{}$\&{char} ${}{*}\\{argv}[\,]){}$\1\1\2\2\6
${}\{{}$\1\6
\&{register} \&{int} \|i${},{}$ \|j${},{}$ \\{jj}${},{}$ \\{jm}${},{}$ %
\|k${},{}$ \\{km}${},{}$ \|l${},{}$ \\{ll}${},{}$ \|m${},{}$ \|n${},{}$ \|t;\6
\&{register} \&{Graph} ${}{*}\|g;{}$\6
\&{register} \&{Arc} ${}{*}\|a,{}$ ${}{*}\|b;{}$\6
\&{register} \&{Vertex} ${}{*}\|u,{}$ ${}{*}\|v;{}$\7
\X2:Input the graph\X;\6
\X3:Renumber the vertices\X;\6
\X4:Reformat the arcs\X;\6
\X5:Do the algorithm\X;\6
\4${}\}{}$\2\par
\fi

\M{2}\B\X2:Input the graph\X${}\E{}$\6
\&{if} ${}(\\{argc}\I\T{2}){}$\5
${}\{{}$\1\6
${}\\{fprintf}(\\{stderr},\39\.{"Usage:\ \%s\ foo.gb\\n"},\39\\{argv}[%
\T{0}]);{}$\6
${}\\{exit}({-}\T{1});{}$\6
\4${}\}{}$\2\6
${}\|g\K\\{restore\_graph}(\\{argv}[\T{1}]);{}$\6
\&{if} ${}(\R\|g){}$\5
${}\{{}$\1\6
${}\\{fprintf}(\\{stderr},\39\.{"I\ can't\ input\ the\ g}\)\.{raph\ \%s\ (panic%
\ code\ }\)\.{\%ld)!\\n"},\39\\{argv}[\T{1}],\39\\{panic\_code});{}$\6
${}\\{exit}({-}\T{2});{}$\6
\4${}\}{}$\2\6
${}\|n\K\|g\MG\|n;{}$\6
\&{if} ${}(\|n>\\{maxn}){}$\5
${}\{{}$\1\6
${}\\{fprintf}(\\{stderr},\39\.{"Sorry,\ that\ graph\ h}\)\.{as\ \%d\ vertices;%
\ "},\39\|n);{}$\6
${}\\{fprintf}(\\{stderr},\39\.{"I\ can't\ handle\ more}\)\.{\ than\ \%d!\\n"},%
\39\\{maxn});{}$\6
${}\\{exit}({-}\T{3});{}$\6
\4${}\}{}$\2\6
\&{if} ${}(\|g\MG\|m>\\{maxm}){}$\5
${}\{{}$\1\6
${}\\{fprintf}(\\{stderr},\39\.{"Sorry,\ that\ graph\ h}\)\.{as\ \%d\ arcs;\
"},\39(\|g\MG\|m+\T{1})/\T{2});{}$\6
${}\\{fprintf}(\\{stderr},\39\.{"I\ can't\ handle\ more}\)\.{\ than\ \%d!\\n"},%
\39\\{maxm});{}$\6
${}\\{exit}({-}\T{3});{}$\6
\4${}\}{}$\2\par
\U1.\fi

\M{3}We create the inverse-arc list for each vertex~\PB{\|v} (the list of all
vertices that point to~\PB{\|v}). Then we use a breadth-first numbering scheme
to attach a serial number \PB{$\|v\MG\\{num}$}.

\Y\B\4\D$\\{num}$ \5
$\|z.{}$\|I\par
\B\4\D$\\{invarcs}$ \5
$\|y.{}$\|A\par
\Y\B\4\X3:Renumber the vertices\X${}\E{}$\6
\&{for} ${}(\|v\K\|g\MG\\{vertices};{}$ ${}\|v<\|g\MG\\{vertices}+\|n;{}$ ${}%
\|v\PP){}$\1\5
${}\|v\MG\\{num}\K\T{0},\39\|v\MG\\{invarcs}\K\NULL;{}$\2\6
\&{for} ${}(\|v\K\|g\MG\\{vertices};{}$ ${}\|v<\|g\MG\\{vertices}+\|n;{}$ ${}%
\|v\PP){}$\5
${}\{{}$\1\6
\&{for} ${}(\|a\K\|v\MG\\{arcs};{}$ \|a; ${}\|a\K\|a\MG\\{next}){}$\5
${}\{{}$\1\6
\&{register} \&{Arc} ${}{*}\|b\K\\{gb\_virgin\_arc}(\,);{}$\7
${}\|u\K\|a\MG\\{tip};{}$\6
${}\|b\MG\\{tip}\K\|v;{}$\6
${}\|b\MG\\{next}\K\|u\MG\\{invarcs};{}$\6
${}\|u\MG\\{invarcs}\K\|b;{}$\6
\4${}\}{}$\2\6
\4${}\}{}$\2\6
${}\\{vert}[\T{1}]\K\|g\MG\\{vertices},\39\|g\MG\\{vertices}\MG\\{num}\K%
\T{1};{}$\6
\&{for} ${}(\|j\K\T{0},\39\|k\K\T{1};{}$ ${}\|j<\|k;{}$ ${}\|j\PP){}$\5
${}\{{}$\1\6
${}\|v\K\\{vert}[\|j+\T{1}];{}$\6
\&{for} ${}(\|a\K\|v\MG\\{arcs};{}$ \|a; ${}\|a\K\|a\MG\\{next}){}$\5
${}\{{}$\1\6
${}\|u\K\|a\MG\\{tip};{}$\6
\&{if} ${}(\|u\MG\\{num}\E\T{0}){}$\1\5
${}\|u\MG\\{num}\K\PP\|k,\39\\{vert}[\|k]\K\|u;{}$\2\6
\4${}\}{}$\2\6
\&{for} ${}(\|a\K\|v\MG\\{invarcs};{}$ \|a; ${}\|a\K\|a\MG\\{next}){}$\5
${}\{{}$\1\6
${}\|u\K\|a\MG\\{tip};{}$\6
\&{if} ${}(\|u\MG\\{num}\E\T{0}){}$\1\5
${}\|u\MG\\{num}\K\PP\|k,\39\\{vert}[\|k]\K\|u;{}$\2\6
\4${}\}{}$\2\6
\4${}\}{}$\2\par
\U1.\fi

\M{4}The arcs will be either $j\to k$ or $j\gets k$ between vertex number~$j$
and vertex number~$k$, when $j<k$ and those vertices are adjacent in
the graph. We process them in order of increasing~$j$; but for fixed~$j$,
the values of~$k$ aren't necessarily increasing.

The $k$ values appear in the \PB{\\{arcto}} array, with $-k$ used for the arcs
that emanate from~$k$. The arcs for fixed~$j$
occur in positions \PB{\\{firstarc}[\|j]} through \PB{$\\{firstarc}[\|j+\T{1}]-%
\T{1}$} of that array.

After this step, we forget the GraphBase data structures and just work
with our homegrown integer-only representation.

\Y\B\4\X4:Reformat the arcs\X${}\E{}$\6
\&{for} ${}(\|m\K\T{0},\39\|k\K\T{1};{}$ ${}\|k\Z\|n;{}$ ${}\|k\PP){}$\5
${}\{{}$\1\6
${}\\{firstarc}[\|k]\K\|m;{}$\6
${}\|v\K\\{vert}[\|k];{}$\6
${}\\{printf}(\.{"\%d(\%s)\\n"},\39\|k,\39\|v\MG\\{name});{}$\6
\&{for} ${}(\|a\K\|v\MG\\{arcs};{}$ \|a; ${}\|a\K\|a\MG\\{next}){}$\5
${}\{{}$\1\6
${}\|u\K\|a\MG\\{tip};{}$\6
\&{if} ${}(\|u\MG\\{num}>\|k){}$\5
${}\{{}$\1\6
${}\\{arcto}[\|m\PP]\K\|u\MG\\{num};{}$\6
\&{if} ${}(\|a\MG\\{len}\E\T{1}){}$\1\5
${}\\{printf}(\.{"\ ->\ \%d(\%s)\ \#\%d\\n"},\39\|u\MG\\{num},\39\|u\MG%
\\{name},\39\|m);{}$\2\6
\&{else}\1\5
${}\\{printf}(\.{"\ ->\ \%d(\%s,\%d)\ \#\%d\\n}\)\.{"},\39\|u\MG\\{num},\39\|u%
\MG\\{name},\39\|a\MG\\{len},\39\|m);{}$\2\6
\4${}\}{}$\2\6
\4${}\}{}$\2\6
\&{for} ${}(\|a\K\|v\MG\\{invarcs};{}$ \|a; ${}\|a\K\|a\MG\\{next}){}$\5
${}\{{}$\1\6
${}\|u\K\|a\MG\\{tip};{}$\6
\&{if} ${}(\|u\MG\\{num}>\|k){}$\5
${}\{{}$\1\6
${}\\{arcto}[\|m\PP]\K{-}\|u\MG\\{num};{}$\6
\&{if} ${}(\|a\MG\\{len}\E\T{1}){}$\1\5
${}\\{printf}(\.{"\ <-\ \%d(\%s)\ \#\%d\\n"},\39\|u\MG\\{num},\39\|u\MG%
\\{name},\39\|m);{}$\2\6
\&{else}\1\5
${}\\{printf}(\.{"\ <-\ \%d(\%s,\%d)\ \#\%d\\n}\)\.{"},\39\|u\MG\\{num},\39\|u%
\MG\\{name},\39\|a\MG\\{len},\39\|m);{}$\2\6
\4${}\}{}$\2\6
\4${}\}{}$\2\6
\4${}\}{}$\2\6
${}\\{firstarc}[\|k]\K\|m{}$;\par
\U1.\fi

\N{1}{5}The algorithm.
Now comes the fun part. We systematically construct a binary decision
diagram for all simple paths by working top-down, considering the
arcs in \PB{\\{arcto}}, one by one.

When we're dealing with arc \PB{\|i}, we've already constructed a table of
all possible states that might arise when each of the previous arcs has
been chosen-or-not, except for states that obviously cannot be
part of a simple path.

Arc \PB{\|i} runs from vertex \PB{\|j} to vertex \PB{$\|k\K\\{arcto}[\|i]$},
or from \PB{$\|k\K{-}\\{arcto}[\|i]$} to~\PB{\|j}.
Let \PB{\|l} be the maximum vertex number in arcs less than~\PB{\|i}.

The state before we decide whether or not to include arc~\PB{\|i} is
represented by a table of values \PB{\\{mate}[\|t]}, for $j\le t\le l$,
with the following significance:
If \PB{$\\{mate}[\|t]\K\|t$}, the previous arcs haven't touched vertex \PB{%
\|t}.
If \PB{$\\{mate}[\|t]\K\|u$} and \PB{$\|u\I\|t$}, the previous arcs have made a
simple directed
path from \PB{\|t} to \PB{\|u}.
If \PB{$\\{mate}[\|t]\K{-}\|u$}, the previous arcs have made a simple directed
path from \PB{\|u} to \PB{\|t}.
If \PB{$\\{mate}[\|t]\K\T{0}$}, the previous arcs have ``saturated'' vertex~%
\PB{\|t}; we can't
touch it again.

The \PB{\\{mate}} information is all that we need to know about the behavior of
previous arcs. And it's easily updated when we add the \PB{\|i}th arc (or not).
So each ``state'' is equivalent to a \PB{\\{mate}} table, consisting of
\PB{$\|l+\T{1}-\|j$} numbers.

The states are stored in a queue, indexed by 64-bit numbers
\PB{\\{tail}}, \PB{\\{boundary}}, and \PB{\\{head}}, where \PB{$\\{tail}\Z%
\\{boundary}\Z\\{head}$}.
Between \PB{\\{tail}} and \PB{\\{boundary}} are the pre-arc-\PB{\|i} states
that haven't yet
been processed; between \PB{\\{boundary}} and \PB{\\{head}} are the post-arc-%
\PB{\|i} states
that will be considered later. The states before \PB{\\{boundary}}
are sequences of \PB{$\|s\K\|l+\T{1}-\|j$} bytes each, and the states after %
\PB{\\{boundary}}
are sequences of \PB{$\\{ss}\K\\{ll}+\T{1}-\\{jj}$} bytes each, where \PB{%
\\{ll}} and \PB{\\{jj}} are the values of
\PB{\|l} and \PB{\|j} for arc \PB{$\|i+\T{1}$}.

Bytes of the queue are stored in \PB{\\{mem}}, which wraps around modulo \PB{%
\\{memsize}}.
We ensure that \PB{$\\{head}-\\{tail}$} never exceeds \PB{\\{memsize}}.


\Y\B\4\X5:Do the algorithm\X${}\E{}$\6
\&{for} ${}(\|t\K\T{1};{}$ ${}\|t\Z\|n;{}$ ${}\|t\PP){}$\1\5
${}\\{mate}[\|t]\K\|t;{}$\2\6
\X6:Initialize the queue\X;\6
\&{for} ${}(\|i\K\T{0};{}$ ${}\|i<\|m;{}$ ${}\|i\PP){}$\5
${}\{{}$\1\6
${}\\{printf}(\.{"\#\%d:\\n"},\39\|i+\T{1}){}$;\C{ announce that we're
beginning a new arc }\6
${}\\{fprintf}(\\{stderr},\39\.{"Beginning\ arc\ \%d\ (s}\)\.{erial=%
\%d,head-tail=\%}\)\.{ld)\\n"},\39\|i+\T{1},\39\\{serial},\39\\{head}-%
\\{tail});{}$\6
\\{fflush}(\\{stderr});\6
\X7:Process arc \PB{\|i}\X;\6
\4${}\}{}$\2\par
\U1.\fi

\M{6}\B\X6:Initialize the queue\X${}\E{}$\6
$\\{jj}\K\\{ll}\K\T{1};{}$\6
${}\\{mem}[\T{0}]\K\\{mate}[\T{1}];{}$\6
${}\\{tail}\K\T{0},\39\\{head}\K\T{1};{}$\6
${}\\{serial}\K\T{2}{}$;\par
\U5.\fi

\M{7}Each state for a particular arc gets a distinguishing number.
Two states are special: 0 means the losing state, when a simple path
is impossible; 1 means the winning state, when a simple path has been
completed. The other states are 2 or more.

The output format on \PB{\\{stdout}} simply shows the identifying numbers of a
state
and its two succesors, in hexadecimal.

\Y\B\4\D$\\{trunc}(\\{addr})$ \5
$((\\{addr})\AND(\\{memsize}-\T{1}){}$)\par
\Y\B\4\X7:Process arc \PB{\|i}\X${}\E{}$\6
$\\{boundary}\K\\{head},\39\\{htcount}\K\T{0},\39\\{htid}\K(\|i+\T{1})\LL%
\\{logmemsize};{}$\6
${}\\{newserial}\K\\{serial}+((\\{head}-\\{tail})/(\\{ll}+\T{1}-\\{jj}));{}$\6
${}\|j\K\\{jj},\39\|k\K\\{arcto}[\|i],\39\|l\K\\{ll};{}$\6
\&{while} ${}(\\{jj}\Z\|n\W\\{firstarc}[\\{jj}+\T{1}]\E\|i+\T{1}){}$\1\5
${}\\{jj}\PP;{}$\2\6
${}\\{ll}\K(\|k>\|l\?\|k:{-}\|k>\|l\?{-}\|k:\|l);{}$\6
\&{while} ${}(\\{tail}<\\{boundary}){}$\5
${}\{{}$\1\6
${}\\{printf}(\.{"\%x:"},\39\\{serial});{}$\6
${}\\{serial}\PP;{}$\6
\X8:Unpack a state, and move \PB{\\{tail}} up\X;\6
\X10:Print the successor if arc \PB{\|i} is not chosen\X;\6
\\{printf}(\.{","});\6
\X9:Print the successor if arc \PB{\|i} is chosen\X;\6
\\{printf}(\.{"\\n"});\6
\4${}\}{}$\2\par
\U5.\fi

\M{8}If the target vertex hasn't entered the action yet (that is, if it
exceeds~\PB{\|l}), we must update its \PB{\\{mate}} entry at this point.

\Y\B\4\X8:Unpack a state, and move \PB{\\{tail}} up\X${}\E{}$\6
\&{for} ${}(\|t\K\|j;{}$ ${}\|t\Z\|l;{}$ ${}\|t\PP,\39\\{tail}\PP){}$\5
${}\{{}$\1\6
${}\\{mate}[\|t]\K\\{mem}[\\{trunc}(\\{tail})];{}$\6
\4${}\}{}$\2\par
\U7.\fi

\M{9}Here's where we update the mates. The order of doing this is carefully
chosen so that it works fine when \PB{$\\{mate}[\|j]\K\|j$} and/or \PB{$%
\\{mate}[\|k]\K\|k$}.

\Y\B\4\X9:Print the successor if arc \PB{\|i} is chosen\X${}\E{}$\6
\&{if} ${}(\|k>\T{0}){}$\5
${}\{{}$\1\6
${}\\{jm}\K\\{mate}[\|j],\39\\{km}\K\\{mate}[\|k];{}$\6
\&{if} ${}(\\{jm}\E\|j){}$\1\5
${}\\{jm}\K{-}\|j;{}$\2\6
\&{if} ${}(\\{jm}\G\T{0}\V\\{km}\Z\T{0}){}$\1\5
\\{printf}(\.{"0"});\C{ we mustn't touch a saturated vertex }\2\6
\&{else} \&{if} ${}(\\{jm}\E{-}\|k){}$\1\5
\X11:Print 1 or 0, depending on whether this arc wins or loses\X\2\6
\&{else}\5
${}\{{}$\1\6
${}\\{mate}[\|j]\K\T{0},\39\\{mate}[\|k]\K\T{0};{}$\6
${}\\{mate}[{-}\\{jm}]\K\\{km},\39\\{mate}[\\{km}]\K\\{jm};{}$\6
${}\\{printstate}(\|j,\39\\{jj},\39\\{ll});{}$\6
${}\\{mate}[{-}\\{jm}]\K\|j,\39\\{mate}[\\{km}]\K\|k,\39\\{mate}[\|j]\K\\{jm},%
\39\\{mate}[\|k]\K\\{km}{}$;\C{ restore original state }\6
\&{if} ${}(\\{mate}[\|j]\E{-}\|j){}$\1\5
${}\\{mate}[\|j]\K\|j;{}$\2\6
\4${}\}{}$\2\6
\4${}\}{}$\5
\2\&{else}\5
${}\{{}$\1\6
${}\\{jm}\K\\{mate}[\|j],\39\\{km}\K\\{mate}[{-}\|k];{}$\6
\&{if} ${}(\\{km}\E{-}\|k){}$\1\5
${}\\{km}\K\|k;{}$\2\6
\&{if} ${}(\\{jm}\Z\T{0}\V\\{km}\G\T{0}){}$\1\5
\\{printf}(\.{"0"});\C{ we mustn't touch a saturated vertex }\2\6
\&{else} \&{if} ${}(\\{km}\E{-}\|j){}$\1\5
\X11:Print 1 or 0, depending on whether this arc wins or loses\X\2\6
\&{else}\5
${}\{{}$\1\6
${}\\{mate}[\|j]\K\T{0},\39\\{mate}[{-}\|k]\K\T{0};{}$\6
${}\\{mate}[\\{jm}]\K\\{km},\39\\{mate}[{-}\\{km}]\K\\{jm};{}$\6
${}\\{printstate}(\|j,\39\\{jj},\39\\{ll});{}$\6
${}\\{mate}[\\{jm}]\K\|j,\39\\{mate}[\\{km}]\K{-}\|k,\39\\{mate}[\|j]\K\\{jm},%
\39\\{mate}[{-}\|k]\K\\{km}{}$;\C{ restore original state }\6
\&{if} ${}(\\{mate}[{-}\|k]\E\|k){}$\1\5
${}\\{mate}[{-}\|k]\K{-}\|k;{}$\2\6
\4${}\}{}$\2\6
\4${}\}{}$\2\par
\U7.\fi

\M{10}\B\X10:Print the successor if arc \PB{\|i} is not chosen\X${}\E{}$\6
$\\{printstate}(\|j,\39\\{jj},\39\\{ll}){}$;\par
\U7.\fi

\M{11}See the note below regarding a change that will restrict consideration
to Hamiltonian paths. A similar change is needed here.

\Y\B\4\X11:Print 1 or 0, depending on whether this arc wins or loses\X${}\E{}$\6
${}\{{}$\1\6
\&{for} ${}(\|t\K\|j+\T{1};{}$ ${}\|t\Z\\{ll};{}$ ${}\|t\PP){}$\1\6
\&{if} ${}(\|t\I(\|k>\T{0}\?\|k:{-}\|k)){}$\5
${}\{{}$\1\6
\&{if} ${}(\\{mate}[\|t]\W\\{mate}[\|t]\I\|t){}$\1\5
\&{break};\2\6
\4${}\}{}$\2\2\6
\&{if} ${}(\|t>\\{ll}){}$\1\5
\\{printf}(\.{"1"});\C{ we win: this cycle is all by itself }\2\6
\&{else}\1\5
\\{printf}(\.{"0"});\C{ we lose: there's junk outside this cycle }\2\6
\4${}\}{}$\2\par
\U9.\fi

\M{12}The \PB{\\{printstate}} subroutine does the rest of the work. It makes
sure
that no incomplete paths linger in positions \PB{\|j} through \PB{$\\{jj}-%
\T{1}$}, which
are about to disappear; and it puts the contents of \PB{\\{mate}[\\{jj}]}
through
\PB{\\{mate}[\\{ll}]} into the queue, checking to see if it was already there.

If `\PB{$\\{mate}[\|t]\I\|t$}' is removed from the condition below, we get
Hamiltonian paths only (I mean, simple paths that include every vertex).

\Y\B\4\X12:Subroutines\X${}\E{}$\6
\&{void} \\{printstate}(\&{int} \|j${},\39{}$\&{int} \\{jj}${},\39{}$\&{int} %
\\{ll})\1\1\2\2\6
${}\{{}$\1\6
\&{register} \&{int} \|h${},{}$ \\{hh}${},{}$ \\{ss}${},{}$ \|t${},{}$ %
\\{tt}${},{}$ \\{hash};\7
\&{for} ${}(\|t\K\|j;{}$ ${}\|t<\\{jj};{}$ ${}\|t\PP){}$\1\6
\&{if} ${}(\\{mate}[\|t]\W\\{mate}[\|t]\I\|t){}$\1\5
\&{break};\2\2\6
\&{if} ${}(\|t<\\{jj}){}$\1\5
\\{printf}(\.{"0"});\C{ incomplete junk mustn't be left hanging }\2\6
\&{else} \&{if} ${}(\\{ll}<\\{jj}){}$\1\5
\\{printf}(\.{"0"});\C{ nothing is viable }\2\6
\&{else}\5
${}\{{}$\1\6
${}\\{ss}\K\\{ll}+\T{1}-\\{jj};{}$\6
\&{if} ${}(\\{head}+\\{ss}-\\{tail}>\\{memsize}){}$\5
${}\{{}$\1\6
${}\\{fprintf}(\\{stderr},\39\.{"Oops,\ I'm\ out\ of\ me}\)\.{mory\ (memsize=%
\%d,\ se}\)\.{rial=\%d)!\\n"},\39\\{memsize},\39\\{serial});{}$\6
${}\\{exit}({-}\T{69});{}$\6
\4${}\}{}$\2\6
\X13:Move the current state into position after \PB{\\{head}}, and compute \PB{%
\\{hash}}\X;\6
\X14:Find the first match, \PB{\\{hh}}, for the current state after \PB{%
\\{boundary}}\X;\6
${}\|h\K\\{trunc}(\\{hh}-\\{boundary})/\\{ss};{}$\6
${}\\{printf}(\.{"\%x"},\39\\{newserial}+\|h);{}$\6
\4${}\}{}$\2\6
\4${}\}{}$\2\par
\U1.\fi

\M{13}\B\X13:Move the current state into position after \PB{\\{head}}, and
compute \PB{\\{hash}}\X${}\E{}$\6
\&{for} ${}(\|t\K\\{jj},\39\|h\K\\{trunc}(\\{head}),\39\\{hash}\K\T{0};{}$ ${}%
\|t\Z\\{ll};{}$ ${}\|t\PP,\39\|h\K\\{trunc}(\|h+\T{1})){}$\5
${}\{{}$\1\6
${}\\{mem}[\|h]\K\\{mate}[\|t];{}$\6
${}\\{hash}\K\\{hash}*\T{31415926525}+\\{mate}[\|t];{}$\6
\4${}\}{}$\2\par
\U12.\fi

\M{14}The hash table is automatically cleared whenever \PB{\\{htid}} is
increased,
because we store \PB{\\{htid}} with each relevant table entry.

\Y\B\4\X14:Find the first match, \PB{\\{hh}}, for the current state after \PB{%
\\{boundary}}\X${}\E{}$\6
\&{for} ${}(\\{hash}\K\\{hash}\AND(\\{htsize}-\T{1});{}$  ; ${}\\{hash}\K(%
\\{hash}+\T{1})\AND(\\{htsize}-\T{1})){}$\5
${}\{{}$\1\6
${}\\{hh}\K\\{htable}[\\{hash}];{}$\6
\&{if} ${}((\\{hh}\XOR\\{htid})\G\\{memsize}){}$\1\5
\X15:Insert new entry and \PB{\&{goto} \\{found}}\X;\2\6
${}\\{hh}\K\\{trunc}(\\{hh});{}$\6
\&{for} ${}(\|t\K\\{hh},\39\|h\K\\{trunc}(\\{head}),\39\\{tt}\K\\{trunc}(\|t+%
\\{ss}-\T{1});{}$  ; ${}\|t\K\\{trunc}(\|t+\T{1}),\39\|h\K\\{trunc}(\|h+%
\T{1})){}$\5
${}\{{}$\1\6
\&{if} ${}(\\{mem}[\|t]\I\\{mem}[\|h]){}$\1\5
\&{break};\2\6
\&{if} ${}(\|t\E\\{tt}){}$\1\5
\&{goto} \\{found};\2\6
\4${}\}{}$\2\6
\4${}\}{}$\2\6
\\{found}:\par
\U12.\fi

\M{15}\B\X15:Insert new entry and \PB{\&{goto} \\{found}}\X${}\E{}$\6
${}\{{}$\1\6
\&{if} ${}(\PP\\{htcount}>(\\{htsize}\GG\T{1})){}$\5
${}\{{}$\1\6
${}\\{fprintf}(\\{stderr},\39\.{"Sorry,\ the\ hash\ tab}\)\.{le\ is\ full\
(htsize=\%}\)\.{d,\ serial=\%d)!\\n"},\39\\{htsize},\39\\{serial});{}$\6
${}\\{exit}({-}\T{96});{}$\6
\4${}\}{}$\2\6
${}\\{hh}\K\\{trunc}(\\{head});{}$\6
${}\\{htable}[\\{hash}]\K\\{htid}+\\{hh};{}$\6
${}\\{head}\MRL{+{\K}}\\{ss};{}$\6
\&{goto} \\{found};\6
\4${}\}{}$\2\par
\U14.\fi

\N{1}{16}Index.
\fi

\inx
\fin
\con
