\input cwebmac
\datethis


\N{1}{1}Intro. This program is sort of a reverse of the preprocessor {\mc
SAT12}:
Suppose $F$ is a set of clauses for a satisfiability problem, and {\mc SAT12}
transforms $F$ to~$F'$ and outputs the file \.{/tmp/erp}. Then if some other
program finds a solution to $F'$, this program inputs that solution
(in \PB{\\{stdin}}) together with \.{/tmp/erp} and outputs a solution to~$F$.

The reader is supposed to be familiar with {\mc SAT12}, or at
least with those parts of {\mc SAT12} where
the input format and the \.{erp} file format are specified.

(I hacked this program in a big hurry. It has nothing complicated to do.)

\smallskip\noindent
{\it Note:}\enspace The standard {\mc UNIX} pipes aren't versatile enough
to use this program without auxiliary intermediate files. For instance,
$$\.{sat12 < foo.dat {\char124} sat11k {\char124} sat12-pre}$$
does not work; \.{sat12-pre} will start to read file \.{/tmp/erp}
before \.{sat12} has written it! Instead, you must say something like
$$\.{sat12 < foo.dat >! /tmp/bar.dat;
sat11k < /tmp/bar.dat {\char124} sat12-pre}$$
or
$$\.{sat12 < foo.dat {\char124} sat11k >! /tmp/bar.sol;
sat12-pre < /tmp/bar.sol}$$
to get the list of satisfying literals. The second alternative is generally
better, because \.{/tmp/bar.sol} is a one-line file with
at most as many literals as
there are variables in the reduced clauses, while \.{/tmp/bar.dat} has the full
set of those clauses.

I could probably get around this problem by using named pipes. But I don't
want to go to the trouble of creating and destroying them.

\Y\B\4\D$\|O$ \5
\.{"\%"}\C{ used for percent signs in format strings }\par
\Y\B\8\#\&{include} \.{<stdio.h>}\6
\8\#\&{include} \.{<stdlib.h>}\6
\8\#\&{include} \.{<string.h>}\6
\8\#\&{include} \.{"gb\_flip.h"}\6
\&{typedef} \&{unsigned} \&{int} \&{uint};\C{ a convenient abbreviation }\6
\&{typedef} \&{unsigned} \&{long} \&{long} \&{ullng};\C{ ditto }\7
\X4:Type definitions\X;\6
\X2:Global variables\X;\6
\X29:Subroutines\X;\7
\\{main}(\&{int} \\{argc}${},\39{}$\&{char} ${}{*}\\{argv}[\,]){}$\1\1\2\2\6
${}\{{}$\1\6
\&{register} \&{uint} \|c${},{}$ \|h${},{}$ \|i${},{}$ \|j${},{}$ \|k${},{}$ %
\\{kk}${},{}$ \|l${},{}$ \|p${},{}$ \|v${},{}$ \\{vv};\7
\X3:Process the command line\X;\6
\X7:Initialize everything\X;\6
\X8:Input the \.{erp} file\X;\6
\&{if} ${}(\R\\{clauses}){}$\1\5
${}\\{fprintf}(\\{stderr},\39\.{"(The\ erp\ file\ is\ em}\)\.{pty!)\\n"});{}$\2%
\6
\X21:Input the solution\X;\6
\X10:Check input anomalies\X;\6
\X22:Output the new solution\X;\6
\4${}\}{}$\2\par
\fi

\M{2}Here I'm mostly copying miscellaneous lines of code from {\mc SAT12},
editing it lightly, and
keeping more of it than actually necessary.

\Y\B\4\X2:Global variables\X${}\E{}$\6
\&{int} \\{random\_seed}${}\K\T{0}{}$;\C{ seed for the random words of \PB{%
\\{gb\_rand}} }\6
\&{int} \\{hbits}${}\K\T{8}{}$;\C{ logarithm of the number of the hash lists }\6
\&{int} \\{buf\_size}${}\K\T{1024}{}$;\C{ must exceed the length of the longest
erp input line }\6
\&{FILE} ${}{*}\\{erp\_file}{}$;\C{ file to allow reverse preprocessing }\6
\&{char} \\{erp\_file\_name}[\T{100}]${}\K\.{"/tmp/erp"}{}$;\C{ its name }\par
\As6\ET24.
\U1.\fi

\M{3}On the command line one can specify nondefault values for any of the
following parameters:
\smallskip
\item{$\bullet$}
`\.h$\langle\,$positive integer$\,\rangle$' to adjust the hash table size.
\item{$\bullet$}
`\.b$\langle\,$positive integer$\,\rangle$' to adjust the size of the input
buffer.
\item{$\bullet$}
`\.s$\langle\,$integer$\,\rangle$' to define the seed for any random numbers
that are used.
\item{$\bullet$}
`\.e$\langle\,$filename$\,\rangle$' to change the name
of the \.{erp} output file.

\Y\B\4\X3:Process the command line\X${}\E{}$\6
\&{for} ${}(\|j\K\\{argc}-\T{1},\39\|k\K\T{0};{}$ \|j; ${}\|j\MM){}$\1\6
\&{switch} (\\{argv}[\|j][\T{0}])\5
${}\{{}$\1\6
\4\&{case} \.{'h'}:\5
${}\|k\MRL{{\OR}{\K}}(\\{sscanf}(\\{argv}[\|j]+\T{1},\39\.{""}\|O\.{"d"},\39{%
\AND}\\{hbits})-\T{1}){}$;\5
\&{break};\6
\4\&{case} \.{'b'}:\5
${}\|k\MRL{{\OR}{\K}}(\\{sscanf}(\\{argv}[\|j]+\T{1},\39\.{""}\|O\.{"d"},\39{%
\AND}\\{buf\_size})-\T{1}){}$;\5
\&{break};\6
\4\&{case} \.{'s'}:\5
${}\|k\MRL{{\OR}{\K}}(\\{sscanf}(\\{argv}[\|j]+\T{1},\39\.{""}\|O\.{"d"},\39{%
\AND}\\{random\_seed})-\T{1}){}$;\5
\&{break};\6
\4\&{case} \.{'e'}:\5
${}\\{sprintf}(\\{erp\_file\_name},\39\.{""}\|O\.{".99s"},\39\\{argv}[\|j]+%
\T{1}){}$;\5
\&{break};\6
\4\&{default}:\5
${}\|k\K\T{1}{}$;\C{ unrecognized command-line option }\6
\4${}\}{}$\2\2\6
\&{if} ${}(\|k\V\\{hbits}<\T{0}\V\\{hbits}>\T{30}\V\\{buf\_size}<\T{11}){}$\5
${}\{{}$\1\6
${}\\{fprintf}(\\{stderr},\39\.{"Usage:\ "}\|O\.{"s\ [v<n>]\ [h<n>]\ [b<}\)%
\.{n>]\ [s<n>]\ [efoo.erp}\)\.{]\ [m<n>]"},\39\\{argv}[\T{0}]);{}$\6
${}\\{fprintf}(\\{stderr},\39\.{"\ [c<n>]\ <\ foo.dat\\n}\)\.{"});{}$\6
${}\\{exit}({-}\T{1});{}$\6
\4${}\}{}$\2\6
\&{if} ${}(\R(\\{erp\_file}\K\\{fopen}(\\{erp\_file\_name},\39\.{"r"}))){}$\5
${}\{{}$\1\6
${}\\{fprintf}(\\{stderr},\39\.{"I\ couldn't\ open\ fil}\)\.{e\ "}\|O\.{"s\ for%
\ reading!\\n"},\39\\{erp\_file\_name});{}$\6
${}\\{exit}({-}\T{16});{}$\6
\4${}\}{}$\2\par
\U1.\fi

\N{1}{4}The I/O wrapper. The following routines read the input and absorb it
into
temporary data areas from which all of the ``real'' data structures
can readily be initialized. My intent is to incorporate these routines in all
of the SAT-solvers in this series. Therefore I've tried to make the code
short and simple, yet versatile enough so that almost no restrictions are
placed on the sizes of problems that can be handled. These routines are
supposed to work properly unless there are more than
$2^{32}-1=4$,294,967,295 occurrences of literals in clauses,
or more than $2^{31}-1=2$,147,483,647 variables or clauses.

In these temporary tables, each variable is represented by four things:
its unique name; its serial number; the clause number (if any) in which it has
most recently appeared; and a pointer to the previous variable (if any)
with the same hash address. Several variables at a time
are represented sequentially in small chunks of memory called ``vchunks,''
which are allocated as needed (and freed later).

\Y\B\4\D$\\{vars\_per\_vchunk}$ \5
\T{341}\C{ preferably $(2^k-1)/3$ for some $k$ }\par
\Y\B\4\X4:Type definitions\X${}\E{}$\6
\&{typedef} \&{union} ${}\{{}$\1\6
\&{char} \\{ch8}[\T{8}];\6
\&{uint} \\{u2}[\T{2}];\6
\&{ullng} \\{lng};\2\6
${}\}{}$ \&{octa};\6
\&{typedef} \&{struct} \&{tmp\_var\_struct} ${}\{{}$\1\6
\&{octa} \\{name};\C{ the name (one to eight ASCII characters) }\6
\&{uint} \\{serial};\C{ 0 for the first variable, 1 for the second, etc. }\6
\&{int} \\{stamp};\C{ \PB{\|m} if positively in clause \PB{\|m}; \PB{${-}\|m$}
if negatively there }\6
\&{struct} \&{tmp\_var\_struct} ${}{*}\\{next}{}$;\C{ pointer for hash list }\2%
\6
${}\}{}$ \&{tmp\_var};\7
\&{typedef} \&{struct} \&{vchunk\_struct} ${}\{{}$\1\6
\&{struct} \&{vchunk\_struct} ${}{*}\\{prev}{}$;\C{ previous chunk allocated
(if any) }\6
\&{tmp\_var} \\{var}[\\{vars\_per\_vchunk}];\2\6
${}\}{}$ \&{vchunk};\par
\A5.
\U1.\fi

\M{5}Each clause in the temporary tables is represented by a sequence of
one or more pointers to the \PB{\&{tmp\_var}} nodes of the literals involved.
A negated literal is indicated by adding~1 to such a pointer.
The first literal of a clause is indicated by adding~2.
Several of these pointers are represented sequentially in chunks
of memory, which are allocated as needed and freed later.

\Y\B\4\D$\\{cells\_per\_chunk}$ \5
\T{511}\C{ preferably $2^k-1$ for some $k$ }\par
\Y\B\4\X4:Type definitions\X${}\mathrel+\E{}$\6
\&{typedef} \&{struct} \&{chunk\_struct} ${}\{{}$\1\6
\&{struct} \&{chunk\_struct} ${}{*}\\{prev}{}$;\C{ previous chunk allocated (if
any) }\6
\&{tmp\_var} ${}{*}\\{cell}[\\{cells\_per\_chunk}];{}$\2\6
${}\}{}$ \&{chunk};\par
\fi

\M{6}\B\X2:Global variables\X${}\mathrel+\E{}$\6
\&{char} ${}{*}\\{buf}{}$;\C{ buffer for reading the lines (clauses) of \PB{%
\\{erp\_file}} }\6
\&{tmp\_var} ${}{*}{*}\\{hash}{}$;\C{ heads of the hash lists }\6
\&{uint} \\{hash\_bits}[\T{93}][\T{8}];\C{ random bits for universal hash
function }\6
\&{vchunk} ${}{*}\\{cur\_vchunk}{}$;\C{ the vchunk currently being filled }\6
\&{tmp\_var} ${}{*}\\{cur\_tmp\_var}{}$;\C{ current place to create new \PB{%
\&{tmp\_var}} entries }\6
\&{tmp\_var} ${}{*}\\{bad\_tmp\_var}{}$;\C{ the \PB{\\{cur\_tmp\_var}} when we
need a new \PB{\&{vchunk}} }\6
\&{chunk} ${}{*}\\{cur\_chunk}{}$;\C{ the chunk currently being filled }\6
\&{tmp\_var} ${}{*}{*}\\{cur\_cell}{}$;\C{ current place to create new elements
of a clause }\6
\&{tmp\_var} ${}{*}{*}\\{bad\_cell}{}$;\C{ the \PB{\\{cur\_cell}} when we need
a new \PB{\&{chunk}} }\6
\&{ullng} \\{vars};\C{ how many distinct variables have we seen? }\6
\&{ullng} \\{clauses};\C{ how many clauses have we seen? }\6
\&{ullng} \\{cells};\C{ how many occurrences of literals in clauses? }\6
\&{int} \\{kkk};\C{ how many clauses should follow the current \.{erp} file
group }\par
\fi

\M{7}\B\X7:Initialize everything\X${}\E{}$\6
\\{gb\_init\_rand}(\\{random\_seed});\6
${}\\{buf}\K{}$(\&{char} ${}{*}){}$ \\{malloc}${}(\\{buf\_size}*\&{sizeof}(%
\&{char}));{}$\6
\&{if} ${}(\R\\{buf}){}$\5
${}\{{}$\1\6
${}\\{fprintf}(\\{stderr},\39\.{"Couldn't\ allocate\ t}\)\.{he\ input\ buffer\
(buf}\)\.{\_size="}\|O\.{"d)!\\n"},\39\\{buf\_size});{}$\6
${}\\{exit}({-}\T{2});{}$\6
\4${}\}{}$\2\6
${}\\{hash}\K{}$(\&{tmp\_var} ${}{*}{*}){}$ \\{malloc}${}(\&{sizeof}(\&{tmp%
\_var})\LL\\{hbits});{}$\6
\&{if} ${}(\R\\{hash}){}$\5
${}\{{}$\1\6
${}\\{fprintf}(\\{stderr},\39\.{"Couldn't\ allocate\ "}\|O\.{"d\ hash\ list\
heads\ (}\)\.{hbits="}\|O\.{"d)!\\n"},\39\T{1}\LL\\{hbits},\39\\{hbits});{}$\6
${}\\{exit}({-}\T{3});{}$\6
\4${}\}{}$\2\6
\&{for} ${}(\|h\K\T{0};{}$ ${}\|h<\T{1}\LL\\{hbits};{}$ ${}\|h\PP){}$\1\5
${}\\{hash}[\|h]\K\NULL{}$;\2\par
\A15.
\U1.\fi

\M{8}The hash address of each variable name has $h$ bits, where $h$ is the
value of the adjustable parameter \PB{\\{hbits}}.
Thus the average number of variables per hash list is $n/2^h$ when there
are $n$ different variables. A warning is printed if this average number
exceeds 10. (For example, if $h$ has its default value, 8, the program will
suggest that you might want to increase $h$ if your input has 2560
different variables or more.)

All the hashing takes place at the very beginning,
and the hash tables are actually recycled before any SAT-solving takes place;
therefore the setting of this parameter is by no means crucial. But I didn't
want to bother with fancy coding that would determine $h$ automatically.

\Y\B\4\X8:Input the \.{erp} file\X${}\E{}$\6
\&{while} (\T{1})\5
${}\{{}$\1\6
${}\|k\K\\{fscanf}(\\{erp\_file},\39\.{""}\|O\.{"10s\ <-"}\|O\.{"d"},\39%
\\{buf},\39{\AND}\\{kkk});{}$\6
\&{if} ${}(\|k\I\T{2}){}$\1\5
\&{break};\2\6
${}\\{clauses}\PP;{}$\6
\X20:Input one literal\X;\6
${}{*}(\\{cur\_cell}-\T{1})\K\\{hack\_in}({*}(\\{cur\_cell}-\T{1}),\39%
\T{4}){}$;\C{ special marker }\6
\&{if} ${}(\R\\{fgets}(\\{buf},\39\\{buf\_size},\39\\{erp\_file})\V\\{buf}[%
\T{0}]\I\.{'\\n'}){}$\1\5
\\{confusion}(\.{"erp\ group\ intro\ lin}\)\.{e\ format"});\2\6
\X9:Input \PB{\\{kkk}} clauses\X;\6
\4${}\}{}$\2\par
\U1.\fi

\M{9}\B\X9:Input \PB{\\{kkk}} clauses\X${}\E{}$\6
\&{for} ${}(\\{kk}\K\T{0};{}$ ${}\\{kk}<\\{kkk};{}$ ${}\\{kk}\PP){}$\5
${}\{{}$\1\6
\&{if} ${}(\R\\{fgets}(\\{buf},\39\\{buf\_size},\39\\{erp\_file})){}$\1\5
\&{break};\2\6
${}\\{clauses}\PP;{}$\6
\&{if} ${}(\\{buf}[\\{strlen}(\\{buf})-\T{1}]\I\.{'\\n'}){}$\5
${}\{{}$\1\6
${}\\{fprintf}(\\{stderr},\39\.{"The\ clause\ on\ line\ }\)\.{"}\|O\.{"lld\ ("}%
\|O\.{".20s...)\ is\ too\ lon}\)\.{g\ for\ me;\\n"},\39\\{clauses},\39%
\\{buf});{}$\6
${}\\{fprintf}(\\{stderr},\39\.{"\ my\ buf\_size\ is\ onl}\)\.{y\ "}\|O\.{"d!%
\\n"},\39\\{buf\_size});{}$\6
${}\\{fprintf}(\\{stderr},\39\.{"Please\ use\ the\ comm}\)\.{and-line\ option\
b<ne}\)\.{wsize>.\\n"});{}$\6
${}\\{exit}({-}\T{4});{}$\6
\4${}\}{}$\2\6
\X11:Input the clause in \PB{\\{buf}}\X;\6
\4${}\}{}$\2\6
\&{if} ${}(\\{kk}<\\{kkk}){}$\5
${}\{{}$\1\6
${}\\{fprintf}(\\{stderr},\39\.{"file\ "}\|O\.{"s\ ended\ prematurely}\)\.{:\
"}\|O\.{"d\ clauses\ missing!\\}\)\.{n"},\39\\{erp\_file\_name},\39\\{kkk}-%
\\{kk});{}$\6
${}\\{exit}({-}\T{667});{}$\6
\4${}\}{}$\2\par
\U8.\fi

\M{10}\B\X10:Check input anomalies\X${}\E{}$\6
\&{if} ${}((\\{vars}\GG\\{hbits})\G\T{10}){}$\5
${}\{{}$\1\6
${}\\{fprintf}(\\{stderr},\39\.{"There\ are\ "}\|O\.{"lld\ variables\ but\ o}\)%
\.{nly\ "}\|O\.{"d\ hash\ tables;\\n"},\39\\{vars},\39\T{1}\LL\\{hbits});{}$\6
\&{while} ${}((\\{vars}\GG\\{hbits})\G\T{10}){}$\1\5
${}\\{hbits}\PP;{}$\2\6
${}\\{fprintf}(\\{stderr},\39\.{"\ maybe\ you\ should\ u}\)\.{se\ command-line\
opti}\)\.{on\ h"}\|O\.{"d?\\n"},\39\\{hbits});{}$\6
\4${}\}{}$\2\6
\&{if} ${}(\\{clauses}\E\T{0}){}$\5
${}\{{}$\1\6
${}\\{fprintf}(\\{stderr},\39\.{"No\ clauses\ were\ inp}\)\.{ut!\\n"});{}$\6
${}\\{exit}({-}\T{77});{}$\6
\4${}\}{}$\2\6
\&{if} ${}(\\{vars}\G\T{\^80000000}){}$\5
${}\{{}$\1\6
${}\\{fprintf}(\\{stderr},\39\.{"Whoa,\ the\ input\ had}\)\.{\ "}\|O\.{"llu\
variables!\\n"},\39\\{cells});{}$\6
${}\\{exit}({-}\T{664});{}$\6
\4${}\}{}$\2\6
\&{if} ${}(\\{clauses}\G\T{\^80000000}){}$\5
${}\{{}$\1\6
${}\\{fprintf}(\\{stderr},\39\.{"Whoa,\ the\ input\ had}\)\.{\ "}\|O\.{"llu\
clauses!\\n"},\39\\{cells});{}$\6
${}\\{exit}({-}\T{665});{}$\6
\4${}\}{}$\2\6
\&{if} ${}(\\{cells}\G\T{\^100000000}){}$\5
${}\{{}$\1\6
${}\\{fprintf}(\\{stderr},\39\.{"Whoa,\ the\ input\ had}\)\.{\ "}\|O\.{"llu\
occurrences\ of\ }\)\.{literals!\\n"},\39\\{cells});{}$\6
${}\\{exit}({-}\T{666});{}$\6
\4${}\}{}$\2\par
\U1.\fi

\M{11}\B\X11:Input the clause in \PB{\\{buf}}\X${}\E{}$\6
\&{for} ${}(\|j\K\|k\K\T{0};{}$  ; \,)\5
${}\{{}$\1\6
\&{while} ${}(\\{buf}[\|j]\E\.{'\ '}){}$\1\5
${}\|j\PP{}$;\C{ scan to nonblank }\2\6
\&{if} ${}(\\{buf}[\|j]\E\.{'\\n'}){}$\1\5
\&{break};\2\6
\&{if} ${}(\\{buf}[\|j]<\.{'\ '}\V\\{buf}[\|j]>\.{'\~'}){}$\5
${}\{{}$\1\6
${}\\{fprintf}(\\{stderr},\39\.{"Illegal\ character\ (}\)\.{code\ \#"}\|O\.{"x)%
\ in\ the\ clause\ on}\)\.{\ line\ "}\|O\.{"lld!\\n"},\39\\{buf}[\|j],\39%
\\{clauses});{}$\6
${}\\{exit}({-}\T{5});{}$\6
\4${}\}{}$\2\6
\&{if} ${}(\\{buf}[\|j]\E\.{'\~'}){}$\1\5
${}\|i\K\T{1},\39\|j\PP;{}$\2\6
\&{else}\1\5
${}\|i\K\T{0};{}$\2\6
\X12:Scan and record a variable; negate it if \PB{$\|i\E\T{1}$}\X;\6
\4${}\}{}$\2\6
\&{if} ${}(\|k\E\T{0}){}$\5
${}\{{}$\1\6
${}\\{fprintf}(\\{stderr},\39\.{"Empty\ line\ "}\|O\.{"lld\ in\ file\ "}\|O%
\.{"s!\\n"},\39\\{clauses},\39\\{erp\_file\_name});{}$\6
${}\\{exit}({-}\T{663});{}$\6
\4${}\}{}$\2\6
${}\\{cells}\MRL{+{\K}}\|k{}$;\par
\U9.\fi

\M{12}We need a hack to insert the bit codes 1, 2, and/or 4 into a pointer
value.

\Y\B\4\D$\\{hack\_in}(\|q,\|t)$ \5
(\&{tmp\_var} ${}{*})(\|t\OR{}$(\&{ullng}) \|q)\par
\Y\B\4\X12:Scan and record a variable; negate it if \PB{$\|i\E\T{1}$}\X${}\E{}$%
\6
${}\{{}$\1\6
\&{register} \&{tmp\_var} ${}{*}\|p;{}$\7
\&{if} ${}(\\{cur\_tmp\_var}\E\\{bad\_tmp\_var}){}$\1\5
\X13:Install a new \PB{\&{vchunk}}\X;\2\6
\X16:Put the variable name beginning at \PB{\\{buf}[\|j]} in \PB{$\\{cur\_tmp%
\_var}\MG\\{name}$} and compute its hash code \PB{\|h}\X;\6
\X17:Find \PB{$\\{cur\_tmp\_var}\MG\\{name}$} in the hash table at \PB{\|p}\X;\6
\&{if} ${}(\|p\MG\\{stamp}\E\\{clauses}\V\|p\MG\\{stamp}\E{-}\\{clauses}){}$\5
${}\{{}$\1\6
${}\\{fprintf}(\\{stderr},\39\.{"Duplicate\ literal\ e}\)\.{ncountered\ on\
line\ "}\|O\.{"lld!\\n"},\39\\{clauses});{}$\6
${}\\{exit}({-}\T{669});{}$\6
\4${}\}{}$\5
\2\&{else}\5
${}\{{}$\1\6
${}\|p\MG\\{stamp}\K(\|i\?{-}\\{clauses}:\\{clauses});{}$\6
\&{if} ${}(\\{cur\_cell}\E\\{bad\_cell}){}$\1\5
\X14:Install a new \PB{\&{chunk}}\X;\2\6
${}{*}\\{cur\_cell}\K\|p;{}$\6
\&{if} ${}(\|i\E\T{1}){}$\1\5
${}{*}\\{cur\_cell}\K\\{hack\_in}({*}\\{cur\_cell},\39\T{1});{}$\2\6
\&{if} ${}(\|k\E\T{0}){}$\1\5
${}{*}\\{cur\_cell}\K\\{hack\_in}({*}\\{cur\_cell},\39\T{2});{}$\2\6
${}\\{cur\_cell}\PP,\39\|k\PP;{}$\6
\4${}\}{}$\2\6
\4${}\}{}$\2\par
\Us11\ET20.\fi

\M{13}\B\X13:Install a new \PB{\&{vchunk}}\X${}\E{}$\6
${}\{{}$\1\6
\&{register} \&{vchunk} ${}{*}\\{new\_vchunk};{}$\7
${}\\{new\_vchunk}\K{}$(\&{vchunk} ${}{*}){}$ \\{malloc}(\&{sizeof}(%
\&{vchunk}));\6
\&{if} ${}(\R\\{new\_vchunk}){}$\5
${}\{{}$\1\6
${}\\{fprintf}(\\{stderr},\39\.{"Can't\ allocate\ a\ ne}\)\.{w\ vchunk!%
\\n"});{}$\6
${}\\{exit}({-}\T{6});{}$\6
\4${}\}{}$\2\6
${}\\{new\_vchunk}\MG\\{prev}\K\\{cur\_vchunk},\39\\{cur\_vchunk}\K\\{new%
\_vchunk};{}$\6
${}\\{cur\_tmp\_var}\K{\AND}\\{new\_vchunk}\MG\\{var}[\T{0}];{}$\6
${}\\{bad\_tmp\_var}\K{\AND}\\{new\_vchunk}\MG\\{var}[\\{vars\_per%
\_vchunk}];{}$\6
\4${}\}{}$\2\par
\U12.\fi

\M{14}\B\X14:Install a new \PB{\&{chunk}}\X${}\E{}$\6
${}\{{}$\1\6
\&{register} \&{chunk} ${}{*}\\{new\_chunk};{}$\7
${}\\{new\_chunk}\K{}$(\&{chunk} ${}{*}){}$ \\{malloc}(\&{sizeof}(\&{chunk}));\6
\&{if} ${}(\R\\{new\_chunk}){}$\5
${}\{{}$\1\6
${}\\{fprintf}(\\{stderr},\39\.{"Can't\ allocate\ a\ ne}\)\.{w\ chunk!%
\\n"});{}$\6
${}\\{exit}({-}\T{7});{}$\6
\4${}\}{}$\2\6
${}\\{new\_chunk}\MG\\{prev}\K\\{cur\_chunk},\39\\{cur\_chunk}\K\\{new%
\_chunk};{}$\6
${}\\{cur\_cell}\K{\AND}\\{new\_chunk}\MG\\{cell}[\T{0}];{}$\6
${}\\{bad\_cell}\K{\AND}\\{new\_chunk}\MG\\{cell}[\\{cells\_per\_chunk}];{}$\6
\4${}\}{}$\2\par
\U12.\fi

\M{15}The hash code is computed via ``universal hashing,'' using the following
precomputed tables of random bits.

\Y\B\4\X7:Initialize everything\X${}\mathrel+\E{}$\6
\&{for} ${}(\|j\K\T{92};{}$ \|j; ${}\|j\MM){}$\1\6
\&{for} ${}(\|k\K\T{0};{}$ ${}\|k<\T{8};{}$ ${}\|k\PP){}$\1\5
${}\\{hash\_bits}[\|j][\|k]\K\\{gb\_next\_rand}(\,){}$;\2\2\par
\fi

\M{16}\B\X16:Put the variable name beginning at \PB{\\{buf}[\|j]} in \PB{$%
\\{cur\_tmp\_var}\MG\\{name}$} and compute its hash code \PB{\|h}\X${}\E{}$\6
$\\{cur\_tmp\_var}\MG\\{name}.\\{lng}\K\T{0};{}$\6
\&{for} ${}(\|h\K\|l\K\T{0};{}$ ${}\\{buf}[\|j+\|l]>\.{'\ '}\W\\{buf}[\|j+\|l]%
\Z\.{'\~'};{}$ ${}\|l\PP){}$\5
${}\{{}$\1\6
\&{if} ${}(\|l>\T{7}){}$\5
${}\{{}$\1\6
${}\\{fprintf}(\\{stderr},\39\.{"Variable\ name\ "}\|O\.{".9s...\ in\ the\
claus}\)\.{e\ on\ line\ "}\|O\.{"lld\ is\ too\ long!\\n"},\39\\{buf}+\|j,\39%
\\{clauses});{}$\6
${}\\{exit}({-}\T{8});{}$\6
\4${}\}{}$\2\6
${}\|h\MRL{{\XOR}{\K}}\\{hash\_bits}[\\{buf}[\|j+\|l]-\.{'!'}][\|l];{}$\6
${}\\{cur\_tmp\_var}\MG\\{name}.\\{ch8}[\|l]\K\\{buf}[\|j+\|l];{}$\6
\4${}\}{}$\2\6
\&{if} ${}(\|l\E\T{0}){}$\5
${}\{{}$\1\6
${}\\{fprintf}(\\{stderr},\39\.{"Illegal\ appearance\ }\)\.{of\ \~\ on\ line\
"}\|O\.{"lld!\\n"},\39\\{clauses});{}$\6
${}\\{exit}({-}\T{668});{}$\6
\4${}\}{}$\2\6
${}\|j\MRL{+{\K}}\|l;{}$\6
${}\|h\MRL{\AND{\K}}(\T{1}\LL\\{hbits})-\T{1}{}$;\par
\U12.\fi

\M{17}\B\X17:Find \PB{$\\{cur\_tmp\_var}\MG\\{name}$} in the hash table at \PB{%
\|p}\X${}\E{}$\6
\&{for} ${}(\|p\K\\{hash}[\|h];{}$ \|p; ${}\|p\K\|p\MG\\{next}){}$\1\6
\&{if} ${}(\|p\MG\\{name}.\\{lng}\E\\{cur\_tmp\_var}\MG\\{name}.\\{lng}){}$\1\5
\&{break};\2\2\6
\&{if} ${}(\R\|p){}$\5
${}\{{}$\C{ new variable found }\1\6
${}\|p\K\\{cur\_tmp\_var}\PP;{}$\6
${}\|p\MG\\{next}\K\\{hash}[\|h],\39\\{hash}[\|h]\K\|p;{}$\6
${}\|p\MG\\{serial}\K\\{vars}\PP;{}$\6
${}\|p\MG\\{stamp}\K\T{0};{}$\6
\4${}\}{}$\2\par
\U12.\fi

\M{18}\B\X18:Move \PB{\\{cur\_cell}} backward to the previous cell\X${}\E{}$\6
\&{if} ${}(\\{cur\_cell}>{\AND}\\{cur\_chunk}\MG\\{cell}[\T{0}]){}$\1\5
${}\\{cur\_cell}\MM;{}$\2\6
\&{else}\5
${}\{{}$\1\6
\&{register} \&{chunk} ${}{*}\\{old\_chunk}\K\\{cur\_chunk};{}$\7
${}\\{cur\_chunk}\K\\{old\_chunk}\MG\\{prev}{}$;\5
\\{free}(\\{old\_chunk});\6
${}\\{bad\_cell}\K{\AND}\\{cur\_chunk}\MG\\{cell}[\\{cells\_per\_chunk}];{}$\6
${}\\{cur\_cell}\K\\{bad\_cell}-\T{1};{}$\6
\4${}\}{}$\2\par
\Us26\ET27.\fi

\M{19}\B\X19:Move \PB{\\{cur\_tmp\_var}} backward to the previous temporary
variable\X${}\E{}$\6
\&{if} ${}(\\{cur\_tmp\_var}>{\AND}\\{cur\_vchunk}\MG\\{var}[\T{0}]){}$\1\5
${}\\{cur\_tmp\_var}\MM;{}$\2\6
\&{else}\5
${}\{{}$\1\6
\&{register} \&{vchunk} ${}{*}\\{old\_vchunk}\K\\{cur\_vchunk};{}$\7
${}\\{cur\_vchunk}\K\\{old\_vchunk}\MG\\{prev}{}$;\5
\\{free}(\\{old\_vchunk});\6
${}\\{bad\_tmp\_var}\K{\AND}\\{cur\_vchunk}\MG\\{var}[\\{vars\_per%
\_vchunk}];{}$\6
${}\\{cur\_tmp\_var}\K\\{bad\_tmp\_var}-\T{1};{}$\6
\4${}\}{}$\2\par
\U25.\fi

\M{20}\B\X20:Input one literal\X${}\E{}$\6
\&{if} ${}(\\{buf}[\T{0}]\E\.{'\~'}){}$\1\5
${}\|i\K\|j\K\T{1};{}$\2\6
\&{else}\1\5
${}\|i\K\|j\K\T{0};{}$\2\6
\X12:Scan and record a variable; negate it if \PB{$\|i\E\T{1}$}\X;\par
\Us8\ET21.\fi

\M{21}\B\X21:Input the solution\X${}\E{}$\6
$\\{clauses}\PP;{}$\6
${}\|k\K\T{0};{}$\6
\&{while} (\T{1})\5
${}\{{}$\1\6
\&{if} ${}(\\{scanf}(\.{""}\|O\.{"10s"},\39\\{buf})\I\T{1}){}$\1\5
\&{break};\2\6
\&{if} ${}(\\{buf}[\T{0}]\E\.{'\~'}\W\\{buf}[\T{1}]\E\T{0}){}$\5
${}\{{}$\1\6
\\{printf}(\.{"\~\\n"});\C{ it was unsatisfiable }\6
\\{exit}(\T{0});\6
\4${}\}{}$\2\6
\X20:Input one literal\X;\6
\4${}\}{}$\2\par
\U1.\fi

\N{1}{22}Doing it.
When the input phase is done, \PB{\|k} literals will have been stored as if
they are one huge clause. They are preceded by other groups of clauses,
where each group begins with a literal-to-be-defined, identified
by a hacked-in 4~bit.

We unwind that data, seeing it backwards as in other programs of this series.
Two trivial data structures make the process easy: One for the names of
the variables, and one for the current values of the literals.

\Y\B\4\X22:Output the new solution\X${}\E{}$\6
\X23:Allocate the main arrays\X;\6
\&{for} ${}(\|l\K\T{2};{}$ ${}\|l<\\{vars}+\\{vars}+\T{2};{}$ ${}\|l\PP){}$\1\5
${}\\{lmem}[\|l]\K\\{unknown};{}$\2\6
\X25:Copy all the temporary variable nodes to the \PB{\\{vmem}} array in proper
format\X;\6
\&{if} (\|k)\1\5
\X26:Absorb and echo the literals of the given solution\X;\2\6
\X27:Use the erp data to compute the rest of the solution\X;\6
\X28:Check consistency\X;\6
\\{printf}(\.{"\\n"});\par
\U1.\fi

\M{23}A single \PB{\&{octa}} is enough information for each variable,
and a single \PB{\&{char}} is (more than) enough for each literal.

\Y\B\4\D$\\{true}$ \5
\T{1}\par
\B\4\D$\\{false}$ \5
${-}{}$\T{1}\par
\B\4\D$\\{unknown}$ \5
\T{0}\par
\B\4\D$\\{thevar}(\|l)$ \5
$((\|l)\GG\T{1}{}$)\par
\B\4\D$\\{bar}(\|l)$ \5
$((\|l)\XOR\T{1}{}$)\C{ the complement of \PB{\|l} }\par
\B\4\D$\\{litname}(\|l)$ \5
$(\|l)\AND\T{1}\?\.{"\~"}:\.{""},\39\\{vmem}[\\{thevar}(\|l)].{}$\\{ch8}\C{
used in printouts }\par
\Y\B\4\X23:Allocate the main arrays\X${}\E{}$\6
$\\{vmem}\K{}$(\&{octa} ${}{*}){}$ \\{malloc}${}((\\{vars}+\T{1})*\&{sizeof}(%
\&{octa}));{}$\6
\&{if} ${}(\R\\{vmem}){}$\5
${}\{{}$\1\6
${}\\{fprintf}(\\{stderr},\39\.{"Oops,\ I\ can't\ alloc}\)\.{ate\ the\ vmem\
array!\\}\)\.{n"});{}$\6
${}\\{exit}({-}\T{10});{}$\6
\4${}\}{}$\2\6
${}\\{lmem}\K{}$(\&{char} ${}{*}){}$ \\{malloc}${}((\\{vars}+\\{vars}+\T{2})*%
\&{sizeof}(\&{char}));{}$\6
\&{if} ${}(\R\\{lmem}){}$\5
${}\{{}$\1\6
${}\\{fprintf}(\\{stderr},\39\.{"Oops,\ I\ can't\ alloc}\)\.{ate\ the\ lmem\
array!\\}\)\.{n"});{}$\6
\4${}\}{}$\2\par
\U22.\fi

\M{24}\B\X2:Global variables\X${}\mathrel+\E{}$\6
\&{octa} ${}{*}\\{vmem}{}$;\C{ array of variable names }\6
\&{char} ${}{*}\\{lmem}{}$;\C{ array of literal values }\par
\fi

\M{25}\B\X25:Copy all the temporary variable nodes to the \PB{\\{vmem}} array
in proper format\X${}\E{}$\6
\&{for} ${}(\|c\K\\{vars};{}$ \|c; ${}\|c\MM){}$\5
${}\{{}$\1\6
\X19:Move \PB{\\{cur\_tmp\_var}} backward to the previous temporary variable\X;%
\6
${}\\{vmem}[\|c].\\{lng}\K\\{cur\_tmp\_var}\MG\\{name}.\\{lng};{}$\6
\4${}\}{}$\2\par
\U22.\fi

\M{26}\B\D$\\{hack\_out}(\|q)$ \5
(((\&{ullng}) \|q)${}\AND\T{\^7}{}$)\par
\B\4\D$\\{hack\_clean}(\|q)$ \5
((\&{tmp\_var} ${}{*})({}$(\&{ullng}) \|q${}\AND{-}\T{8}){}$)\par
\Y\B\4\X26:Absorb and echo the literals of the given solution\X${}\E{}$\6
${}\{{}$\1\6
\&{for} ${}(\|i\K\T{0};{}$ ${}\|i<\T{2};{}$ \,)\5
${}\{{}$\1\6
\X18:Move \PB{\\{cur\_cell}} backward to the previous cell\X;\6
${}\|i\K\\{hack\_out}({*}\\{cur\_cell});{}$\6
${}\|p\K\\{hack\_clean}({*}\\{cur\_cell})\MG\\{serial};{}$\6
${}\|p\MRL{+{\K}}\|p+(\|i\AND\T{1})+\T{2};{}$\6
${}\\{printf}(\.{"\ "}\|O\.{"s"}\|O\.{"s"},\39\\{litname}(\|p));{}$\6
${}\\{lmem}[\|p]\K\\{true},\39\\{lmem}[\\{bar}(\|p)]\K\\{false};{}$\6
\4${}\}{}$\2\6
\4${}\}{}$\2\par
\U22.\fi

\M{27}At last we get to the heart of this program: Clauses are
evaluated (in reverse order of their appearance in the \.{erp} file)
until we come back to a definition point.

\Y\B\4\X27:Use the erp data to compute the rest of the solution\X${}\E{}$\6
$\|v\K\\{true};{}$\6
\&{for} ${}(\|c\K\\{clauses}-\T{1};{}$ \|c; ${}\|c\MM){}$\5
${}\{{}$\1\6
${}\\{vv}\K\\{false};{}$\6
\&{for} ${}(\|i\K\T{0};{}$ ${}\|i<\T{2};{}$ \,)\5
${}\{{}$\1\6
\X18:Move \PB{\\{cur\_cell}} backward to the previous cell\X;\6
${}\|i\K\\{hack\_out}({*}\\{cur\_cell});{}$\6
${}\|p\K\\{hack\_clean}({*}\\{cur\_cell})\MG\\{serial};{}$\6
${}\|p\MRL{+{\K}}\|p+(\|i\AND\T{1})+\T{2};{}$\6
\&{if} ${}(\|i\G\T{4}){}$\1\5
\&{break};\2\6
\&{if} ${}(\\{lmem}[\|p]\E\\{unknown}){}$\5
${}\{{}$\1\6
${}\\{printf}(\.{"\ "}\|O\.{"s"}\|O\.{"s"},\39\\{litname}(\|p)){}$;\C{ assign
an arbitrary value }\6
${}\\{lmem}[\|p]\K\\{true},\39\\{lmem}[\\{bar}(\|p)]\K\\{false};{}$\6
\4${}\}{}$\2\6
\&{if} ${}(\\{lmem}[\|p]\E\\{true}){}$\1\5
${}\\{vv}\K\\{true}{}$;\C{ \PB{\\{vv}} is {\mc OR} of literals in clause }\2\6
\4${}\}{}$\2\6
\&{if} ${}(\|i<\T{4}){}$\5
${}\{{}$\1\6
\&{if} ${}(\\{vv}\E\\{false}){}$\1\5
${}\|v\K\\{false}{}$;\C{ \PB{\|v} is {\mc AND} of clauses in group }\2\6
\4${}\}{}$\5
\2\&{else}\5
${}\{{}$\C{ defining an eliminated variable }\1\6
${}\\{lmem}[\|p]\K\|v,\39\\{lmem}[\\{bar}(\|p)]\K{-}\|v;{}$\6
\&{if} ${}(\|v\E\\{true}){}$\1\5
${}\\{printf}(\.{"\ "}\|O\.{"s"}\|O\.{"s"},\39\\{litname}(\|p));{}$\2\6
\&{else}\1\5
${}\\{printf}(\.{"\ "}\|O\.{"s"}\|O\.{"s"},\39\\{litname}(\\{bar}(\|p)));{}$\2\6
${}\|v\K\\{true};{}$\6
\4${}\}{}$\2\6
\4${}\}{}$\2\par
\U22.\fi

\M{28}\B\X28:Check consistency\X${}\E{}$\6
\&{if} ${}(\\{cur\_cell}\I{\AND}\\{cur\_chunk}\MG\\{cell}[\T{0}]\V\\{cur%
\_chunk}\MG\\{prev}\I\NULL\V\\{cur\_tmp\_var}\I{\AND}\\{cur\_vchunk}\MG\\{var}[%
\T{0}]\V\\{cur\_vchunk}\MG\\{prev}\I\NULL){}$\1\5
\\{confusion}(\.{"consistency"});\2\6
\\{free}(\\{cur\_chunk});\5
\\{free}(\\{cur\_vchunk});\par
\U22.\fi

\M{29}\B\X29:Subroutines\X${}\E{}$\6
\&{void} \\{confusion}(\&{char} ${}{*}\\{id}){}$\1\1\2\2\6
${}\{{}$\C{ an assertion has failed }\1\6
${}\\{fprintf}(\\{stderr},\39\.{"This\ can't\ happen\ (}\)\.{"}\|O\.{"s)!\\n"},%
\39\\{id});{}$\6
${}\\{exit}({-}\T{69});{}$\6
\4${}\}{}$\2\7
\&{void} \\{debugstop}(\&{int} \\{foo})\1\1\2\2\6
${}\{{}$\C{ can be inserted as a special breakpoint }\1\6
${}\\{fprintf}(\\{stderr},\39\.{"You\ rang("}\|O\.{"d)?\\n"},\39\\{foo});{}$\6
\4${}\}{}$\2\par
\U1.\fi

\N{1}{30}Index.
\fi

\inx
\fin
\con
