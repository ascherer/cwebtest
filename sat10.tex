\input cwebmac
\datethis

\N{1}{1}Intro. This program is part of a series of ``SAT-solvers'' that I'm
putting
together for my own education as I prepare to write Section 7.2.2.2 of
{\sl The Art of Computer Programming}. My intent is to have a variety of
compatible programs on which I can run experiments to learn how different
approaches work in practice.

After experience with ten previous approaches, I finally feel ready to
write the program that I plan to describe first: a very simple ``no-frills''
algorithm that does pretty well on not-too-large problems in spite of
being rather short and sweet. The model for this program is the
``fast one-level algorithm'' of Cynthia~A. Brown and Paul~W. Purdom, Jr.,
found in their paper ``An empirical comparison of backtracking algorithms,''
{\sl IEEE Transactions on Pattern Analysis and Machine Intelligence\/
\bf PAMI-4} (1982), 309--316. This almost-forgotten paper introduced
the idea of {\it watched literals},  a concept that became famous when it was
rediscovered and generalized almost two decades later.
Brown and Purdom noticed that
the operations of backtracking became quite simple when there is {\it one\/}
watched literal in each clause; later researchers, unaware of this previous
work, discovered how to speed up the process of so-called unit propagation
by having {\it two\/} watched literals per clause. By presenting
the Brown--Purdom algorithm first, I hope to introduce my readers to
this concept in a natural and gradual way.

[Note: This program, {\mc SAT10}, is essentially the prototype for
Algorithm 7.2.2.2D.]

If you have already read {\mc SAT0} (or some other program of this
series), you might as well skip now past all the code for the
``I/O wrapper,'' because you have seen it before.

The input on \PB{\\{stdin}} is a series of lines with one clause per line. Each
clause is a sequence of literals separated by spaces. Each literal is
a sequence of one to eight ASCII characters between \.{!} and \.{\}},
inclusive, not beginning with \.{\~},
optionally preceded by \.{\~} (which makes the literal ``negative'').
For example, Rivest's famous clauses on four variables,
found in 6.5--(13) and 7.1.1--(32) of {\sl TAOCP}, can be represented by the
following eight lines of input:
$$\chardef~=`\~
\vcenter{\halign{\tt#\cr
x2 x3 ~x4\cr
x1 x3 x4\cr
~x1 x2 x4\cr
~x1 ~x2 x3\cr
~x2 ~x3 x4\cr
~x1 ~x3 ~x4\cr
x1 ~x2 ~x4\cr
x1 x2 ~x3\cr}}$$
Input lines that begin with \.{\~\ } are ignored (treated as comments).
The output will be `\.{\~}' if the input clauses are unsatisfiable.
Otherwise it will be a list of noncontradictory literals that cover each
clause, separated by spaces. (``Noncontradictory'' means that we don't
have both a literal and its negation.) The input above would, for example,
yield `\.{\~}'; but if the final clause were omitted, the output would
be `\.{\~x1} \.{\~x2} \.{x3}', in some order, possibly together
with either \.{x4} or \.{\~x4} (but not both). No attempt is made to
find all solutions; at most one solution is given.

The running time in ``mems'' is also reported, together with the approximate
number of bytes needed for data storage. One ``mem'' essentially means a
memory access to a 64-bit word.
(These totals don't include the time or space needed to parse the
input or to format the output.)

\fi

\M{2}So here's the structure of the program. (Skip ahead if you are
impatient to see the interesting stuff.)

\Y\B\4\D$\|o$ \5
$\\{mems}\PP{}$\C{ count one mem }\par
\B\4\D$\\{oo}$ \5
$\\{mems}\MRL{+{\K}}{}$\T{2}\C{ count two mems }\par
\B\4\D$\\{ooo}$ \5
$\\{mems}\MRL{+{\K}}{}$\T{3}\C{ count three mems }\par
\B\4\D$\|O$ \5
\.{"\%"}\C{ used for percent signs in format strings }\par
\Y\B\8\#\&{include} \.{<stdio.h>}\6
\8\#\&{include} \.{<stdlib.h>}\6
\8\#\&{include} \.{<string.h>}\6
\8\#\&{include} \.{"gb\_flip.h"}\6
\&{typedef} \&{unsigned} \&{int} \&{uint};\C{ a convenient abbreviation }\6
\&{typedef} \&{unsigned} \&{long} \&{long} \&{ullng};\C{ ditto }\7
\X5:Type definitions\X;\6
\X3:Global variables\X;\6
\X28:Subroutines\X;\7
\\{main}(\&{int} \\{argc}${},\39{}$\&{char} ${}{*}\\{argv}[\,]){}$\1\1\2\2\6
${}\{{}$\1\6
\&{register} \&{uint} \|h${},{}$ \|i${},{}$ \|j${},{}$ \|l${},{}$ \|p${},{}$ %
\|q${},{}$ \|r${},{}$ \\{level}${},{}$ \\{kk}${},{}$ \\{pp}${},{}$ %
\\{qq}${},{}$ \\{ll}${},{}$ \\{force}${},{}$ \\{nextmove};\6
\&{register} \&{int} \|c${},{}$ \\{cc}${},{}$ \|k${},{}$ \\{v0}${},{}$ %
\|v${},{}$ \\{vv}${},{}$ \\{vvv};\7
\X4:Process the command line\X;\6
\X8:Initialize everything\X;\6
\X9:Input the clauses\X;\6
\&{if} ${}(\\{verbose}\AND\\{show\_basics}){}$\1\5
\X22:Report the successful completion of the input phase\X;\2\6
\X33:Set up the main data structures\X;\6
${}\\{imems}\K\\{mems},\39\\{mems}\K\T{0};{}$\6
\X40:Solve the problem\X;\6
\4\\{done}:\5
\&{if} ${}(\\{verbose}\AND\\{show\_basics}){}$\1\5
${}\\{fprintf}(\\{stderr},\39\.{"Altogether\ "}\|O\.{"llu+"}\|O\.{"llu\ mems,\
"}\|O\.{"llu\ bytes,\ "}\|O\.{"llu\ nodes.\\n"},\39\\{imems},\39\\{mems},\39%
\\{bytes},\39\\{nodes});{}$\2\6
\4${}\}{}$\2\par
\fi

\M{3}\B\D$\\{show\_basics}$ \5
\T{1}\C{ \PB{\\{verbose}} code for basic stats }\par
\B\4\D$\\{show\_choices}$ \5
\T{2}\C{ \PB{\\{verbose}} code for backtrack logging }\par
\B\4\D$\\{show\_details}$ \5
\T{4}\C{ \PB{\\{verbose}} code for further commentary }\par
\B\4\D$\\{show\_unused\_vars}$ \5
\T{8}\C{ \PB{\\{verbose}} code to list variables not in solution }\par
\Y\B\4\X3:Global variables\X${}\E{}$\6
\&{int} \\{random\_seed}${}\K\T{0}{}$;\C{ seed for the random words of \PB{%
\\{gb\_rand}} }\6
\&{int} \\{verbose}${}\K\\{show\_basics}+\\{show\_unused\_vars}{}$;\C{ level of
verbosity }\6
\&{int} \\{show\_choices\_max}${}\K\T{1000000}{}$;\C{ above this level, \PB{%
\\{show\_choices}} is ignored }\6
\&{int} \\{hbits}${}\K\T{8}{}$;\C{ logarithm of the number of the hash lists }\6
\&{int} \\{buf\_size}${}\K\T{1024}{}$;\C{ must exceed the length of the longest
input line }\6
\&{FILE} ${}{*}\\{out\_file}{}$;\C{ file for optional output }\6
\&{char} ${}{*}\\{out\_name}{}$;\C{ its name }\6
\&{FILE} ${}{*}\\{primary\_file}{}$;\C{ file for optional input }\6
\&{char} ${}{*}\\{primary\_name}{}$;\C{ its name }\6
\&{int} \\{primary\_vars};\C{ the number of primary variables }\6
\&{ullng} \\{imems}${},{}$ \\{mems};\C{ mem counts }\6
\&{ullng} \\{bytes};\C{ memory used by main data structures }\6
\&{ullng} \\{nodes};\C{ total number of branch nodes initiated }\6
\&{ullng} \\{thresh}${}\K\T{0}{}$;\C{ report when \PB{\\{mems}} exceeds this,
if \PB{$\\{delta}\I\T{0}$} }\6
\&{ullng} \\{delta}${}\K\T{0}{}$;\C{ report every \PB{\\{delta}} or so mems }\6
\&{ullng} \\{timeout}${}\K\T{\^1fffffffffffffff}{}$;\C{ give up after this many
mems }\6
\&{float} \\{eps}${}\K\T{0.1}{}$;\C{ parameter for the minimum score of a watch
list }\par
\As7\ET27.
\U2.\fi

\M{4}On the command line one can specify any or all of the following options:
\smallskip
\item{$\bullet$}
`\.v$\langle\,$integer$\,\rangle$' to enable various levels of verbose
output on \PB{\\{stderr}}.
\item{$\bullet$}
`\.c$\langle\,$positive integer$\,\rangle$' to limit the levels on which
clauses are shown.
\item{$\bullet$}
`\.h$\langle\,$positive integer$\,\rangle$' to adjust the hash table size.
\item{$\bullet$}
`\.b$\langle\,$positive integer$\,\rangle$' to adjust the size of the input
buffer.
\item{$\bullet$}
`\.s$\langle\,$integer$\,\rangle$' to define the seed for any random numbers
that are used.
\item{$\bullet$}
`\.d$\langle\,$integer$\,\rangle$' to set \PB{\\{delta}} for periodic state
reports.
\item{$\bullet$}
`\.e$\langle\,$float$\,\rangle$' to change the \PB{\\{eps}} parameter in
rankings of
variables for branching.
\item{$\bullet$}
`\.x$\langle\,$filename$\,\rangle$' to copy the input plus a
solution-eliminating clause to the specified file. If the given problem is
satisfiable in more than one way, a different solution can be obtained by
inputting that file.
\item{$\bullet$}
`\.V$\langle\,$filename$\,\rangle$' to input a file that lists the names
of all ``primary'' variables. A nonprimary variable will not be used for
branching unless its value is forced, or unless all of the primary variables
have already been assigned a value.
\item{$\bullet$}
`\.T$\langle\,$integer$\,\rangle$' to set \PB{\\{timeout}}: This program will
abruptly terminate, when it discovers that \PB{$\\{mems}>\\{timeout}$}.

\Y\B\4\X4:Process the command line\X${}\E{}$\6
\&{for} ${}(\|j\K\\{argc}-\T{1},\39\|k\K\T{0};{}$ \|j; ${}\|j\MM){}$\1\6
\&{switch} (\\{argv}[\|j][\T{0}])\5
${}\{{}$\1\6
\4\&{case} \.{'v'}:\5
${}\|k\MRL{{\OR}{\K}}(\\{sscanf}(\\{argv}[\|j]+\T{1},\39\.{""}\|O\.{"d"},\39{%
\AND}\\{verbose})-\T{1}){}$;\5
\&{break};\6
\4\&{case} \.{'c'}:\5
${}\|k\MRL{{\OR}{\K}}(\\{sscanf}(\\{argv}[\|j]+\T{1},\39\.{""}\|O\.{"d"},\39{%
\AND}\\{show\_choices\_max})-\T{1}){}$;\5
\&{break};\6
\4\&{case} \.{'h'}:\5
${}\|k\MRL{{\OR}{\K}}(\\{sscanf}(\\{argv}[\|j]+\T{1},\39\.{""}\|O\.{"d"},\39{%
\AND}\\{hbits})-\T{1}){}$;\5
\&{break};\6
\4\&{case} \.{'b'}:\5
${}\|k\MRL{{\OR}{\K}}(\\{sscanf}(\\{argv}[\|j]+\T{1},\39\.{""}\|O\.{"d"},\39{%
\AND}\\{buf\_size})-\T{1}){}$;\5
\&{break};\6
\4\&{case} \.{'s'}:\5
${}\|k\MRL{{\OR}{\K}}(\\{sscanf}(\\{argv}[\|j]+\T{1},\39\.{""}\|O\.{"d"},\39{%
\AND}\\{random\_seed})-\T{1}){}$;\5
\&{break};\6
\4\&{case} \.{'d'}:\5
${}\|k\MRL{{\OR}{\K}}(\\{sscanf}(\\{argv}[\|j]+\T{1},\39\.{""}\|O\.{"lld"},\39{%
\AND}\\{delta})-\T{1}){}$;\5
${}\\{thresh}\K\\{delta}{}$;\5
\&{break};\6
\4\&{case} \.{'e'}:\5
${}\|k\MRL{{\OR}{\K}}(\\{sscanf}(\\{argv}[\|j]+\T{1},\39\.{""}\|O\.{"f"},\39{%
\AND}\\{eps})-\T{1}){}$;\5
\&{break};\6
\4\&{case} \.{'x'}:\5
${}\\{out\_name}\K\\{argv}[\|j]+\T{1},\39\\{out\_file}\K\\{fopen}(\\{out%
\_name},\39\.{"w"});{}$\6
\&{if} ${}(\R\\{out\_file}){}$\1\5
${}\\{fprintf}(\\{stderr},\39\.{"I\ can't\ open\ file\ `}\)\.{"}\|O\.{"s'\ for\
output!\\n"},\39\\{out\_name});{}$\2\6
\&{break};\6
\4\&{case} \.{'V'}:\5
${}\\{primary\_name}\K\\{argv}[\|j]+\T{1},\39\\{primary\_file}\K\\{fopen}(%
\\{primary\_name},\39\.{"r"});{}$\6
\&{if} ${}(\R\\{primary\_file}){}$\1\5
${}\\{fprintf}(\\{stderr},\39\.{"I\ can't\ open\ file\ `}\)\.{"}\|O\.{"s'\ for\
input!\\n"},\39\\{primary\_name});{}$\2\6
\&{break};\6
\4\&{case} \.{'T'}:\5
${}\|k\MRL{{\OR}{\K}}(\\{sscanf}(\\{argv}[\|j]+\T{1},\39\.{"\%lld"},\39{\AND}%
\\{timeout})-\T{1}){}$;\5
\&{break};\6
\4\&{default}:\5
${}\|k\K\T{1}{}$;\C{ unrecognized command-line option }\6
\4${}\}{}$\2\2\6
\&{if} ${}(\|k\V\\{hbits}<\T{0}\V\\{hbits}>\T{30}\V\\{buf\_size}\Z\T{0}){}$\5
${}\{{}$\1\6
${}\\{fprintf}(\\{stderr},\39\.{"Usage:\ "}\|O\.{"s\ [v<n>]\ [c<n>]\ [h<}\)%
\.{n>]\ [b<n>]\ [s<n>]\ [d}\)\.{<n>]\ [e<f>]"},\39\\{argv}[\T{0}]);{}$\6
${}\\{fprintf}(\\{stderr},\39\.{"\ [x<foo>]\ [V<foo>]\ }\)\.{[T<n>]\ <\ foo.sat%
\\n"});{}$\6
${}\\{exit}({-}\T{1});{}$\6
\4${}\}{}$\2\par
\U2.\fi

\N{1}{5}The I/O wrapper. The following routines read the input and absorb it
into
temporary data areas from which all of the ``real'' data structures
can readily be initialized. My intent is to incorporate these routines into all
of the SAT-solvers in this series. Therefore I've tried to make the code
short and simple, yet versatile enough so that almost no restrictions are
placed on the sizes of problems that can be handled. These routines are
supposed to work properly unless there are more than
$2^{32}-1=4$,294,967,295 occurrences of literals in clauses,
or more than $2^{31}-1=2$,147,483,647 variables or clauses.

In these temporary tables, each variable is represented by four things:
its unique name; its serial number; the clause number (if any) in which it has
most recently appeared; and a pointer to the previous variable (if any)
with the same hash address. Several variables at a time
are represented sequentially in small chunks of memory called ``vchunks,''
which are allocated as needed (and freed later).

\Y\B\4\D$\\{vars\_per\_vchunk}$ \5
\T{341}\C{ preferably $(2^k-1)/3$ for some $k$ }\par
\Y\B\4\X5:Type definitions\X${}\E{}$\6
\&{typedef} \&{union} ${}\{{}$\1\6
\&{char} \\{ch8}[\T{8}];\6
\&{uint} \\{u2}[\T{2}];\6
\&{long} \&{long} \\{lng};\2\6
${}\}{}$ \&{octa};\6
\&{typedef} \&{struct} \&{tmp\_var\_struct} ${}\{{}$\1\6
\&{octa} \\{name};\C{ the name (one to eight ASCII characters) }\6
\&{uint} \\{serial};\C{ 0 for the first variable, 1 for the second, etc. }\6
\&{int} \\{stamp};\C{ \PB{\|m} if positively in clause \PB{\|m}; \PB{${-}\|m$}
if negatively there }\6
\&{struct} \&{tmp\_var\_struct} ${}{*}\\{next}{}$;\C{ pointer for hash list }\2%
\6
${}\}{}$ \&{tmp\_var};\7
\&{typedef} \&{struct} \&{vchunk\_struct} ${}\{{}$\1\6
\&{struct} \&{vchunk\_struct} ${}{*}\\{prev}{}$;\C{ previous chunk allocated
(if any) }\6
\&{tmp\_var} \\{var}[\\{vars\_per\_vchunk}];\2\6
${}\}{}$ \&{vchunk};\par
\As6, 24, 25\ETs26.
\U2.\fi

\M{6}Each clause in the temporary tables is represented by a sequence of
one or more pointers to the \PB{\&{tmp\_var}} nodes of the literals involved.
A negated literal is indicated by adding~1 to such a pointer.
The first literal of a clause is indicated by adding~2.
Several of these pointers are represented sequentially in chunks
of memory, which are allocated as needed and freed later.

\Y\B\4\D$\\{cells\_per\_chunk}$ \5
\T{511}\C{ preferably $2^k-1$ for some $k$ }\par
\Y\B\4\X5:Type definitions\X${}\mathrel+\E{}$\6
\&{typedef} \&{struct} \&{chunk\_struct} ${}\{{}$\1\6
\&{struct} \&{chunk\_struct} ${}{*}\\{prev}{}$;\C{ previous chunk allocated (if
any) }\6
\&{tmp\_var} ${}{*}\\{cell}[\\{cells\_per\_chunk}];{}$\2\6
${}\}{}$ \&{chunk};\par
\fi

\M{7}\B\X3:Global variables\X${}\mathrel+\E{}$\6
\&{char} ${}{*}\\{buf}{}$;\C{ buffer for reading the lines (clauses) of \PB{%
\\{stdin}} }\6
\&{tmp\_var} ${}{*}{*}\\{hash}{}$;\C{ heads of the hash lists }\6
\&{uint} \\{hash\_bits}[\T{93}][\T{8}];\C{ random bits for universal hash
function }\6
\&{vchunk} ${}{*}\\{cur\_vchunk}{}$;\C{ the vchunk currently being filled }\6
\&{tmp\_var} ${}{*}\\{cur\_tmp\_var}{}$;\C{ current place to create new \PB{%
\&{tmp\_var}} entries }\6
\&{tmp\_var} ${}{*}\\{bad\_tmp\_var}{}$;\C{ the \PB{\\{cur\_tmp\_var}} when we
need a new \PB{\&{vchunk}} }\6
\&{chunk} ${}{*}\\{cur\_chunk}{}$;\C{ the chunk currently being filled }\6
\&{tmp\_var} ${}{*}{*}\\{cur\_cell}{}$;\C{ current place to create new elements
of a clause }\6
\&{tmp\_var} ${}{*}{*}\\{bad\_cell}{}$;\C{ the \PB{\\{cur\_cell}} when we need
a new \PB{\&{chunk}} }\6
\&{ullng} \\{vars};\C{ how many distinct variables have we seen? }\6
\&{ullng} \\{clauses};\C{ how many clauses have we seen? }\6
\&{ullng} \\{nullclauses};\C{ how many of them were null? }\6
\&{ullng} \\{cells};\C{ how many occurrences of literals in clauses? }\6
\&{int} \\{non\_clause};\C{ is the current clause ignorable? }\par
\fi

\M{8}\B\X8:Initialize everything\X${}\E{}$\6
\\{gb\_init\_rand}(\\{random\_seed});\6
${}\\{buf}\K{}$(\&{char} ${}{*}){}$ \\{malloc}${}(\\{buf\_size}*\&{sizeof}(%
\&{char}));{}$\6
\&{if} ${}(\R\\{buf}){}$\5
${}\{{}$\1\6
${}\\{fprintf}(\\{stderr},\39\.{"Couldn't\ allocate\ t}\)\.{he\ input\ buffer\
(buf}\)\.{\_size="}\|O\.{"d)!\\n"},\39\\{buf\_size});{}$\6
${}\\{exit}({-}\T{2});{}$\6
\4${}\}{}$\2\6
${}\\{hash}\K{}$(\&{tmp\_var} ${}{*}{*}){}$ \\{malloc}${}(\&{sizeof}(\&{tmp%
\_var})\LL\\{hbits});{}$\6
\&{if} ${}(\R\\{hash}){}$\5
${}\{{}$\1\6
${}\\{fprintf}(\\{stderr},\39\.{"Couldn't\ allocate\ "}\|O\.{"d\ hash\ list\
heads\ (}\)\.{hbits="}\|O\.{"d)!\\n"},\39\T{1}\LL\\{hbits},\39\\{hbits});{}$\6
${}\\{exit}({-}\T{3});{}$\6
\4${}\}{}$\2\6
\&{for} ${}(\|h\K\T{0};{}$ ${}\|h<\T{1}\LL\\{hbits};{}$ ${}\|h\PP){}$\1\5
${}\\{hash}[\|h]\K\NULL{}$;\2\par
\A15.
\U2.\fi

\M{9}The hash address of each variable name has $h$ bits, where $h$ is the
value of the adjustable parameter \PB{\\{hbits}}.
Thus the average number of variables per hash list is $n/2^h$ when there
are $n$ different variables. A warning is printed if this average number
exceeds 10. (For example, if $h$ has its default value, 8, the program will
suggest that you might want to increase $h$ if your input has 2560
different variables or more.)

All the hashing takes place at the very beginning,
and the hash tables are actually recycled before any SAT-solving takes place;
therefore the setting of this parameter is by no means crucial. But I didn't
want to bother with fancy coding that would determine $h$ automatically.

\Y\B\4\X9:Input the clauses\X${}\E{}$\6
\&{if} (\\{primary\_file})\1\5
\X10:Input the primary variables\X;\2\6
\&{while} (\T{1})\5
${}\{{}$\1\6
\&{if} ${}(\R\\{fgets}(\\{buf},\39\\{buf\_size},\39\\{stdin})){}$\1\5
\&{break};\2\6
${}\\{clauses}\PP;{}$\6
\&{if} ${}(\\{buf}[\\{strlen}(\\{buf})-\T{1}]\I\.{'\\n'}){}$\5
${}\{{}$\1\6
${}\\{fprintf}(\\{stderr},\39\.{"The\ clause\ on\ line\ }\)\.{"}\|O\.{"lld\ ("}%
\|O\.{".20s...)\ is\ too\ lon}\)\.{g\ for\ me;\\n"},\39\\{clauses},\39%
\\{buf});{}$\6
${}\\{fprintf}(\\{stderr},\39\.{"\ my\ buf\_size\ is\ onl}\)\.{y\ "}\|O\.{"d!%
\\n"},\39\\{buf\_size});{}$\6
${}\\{fprintf}(\\{stderr},\39\.{"Please\ use\ the\ comm}\)\.{and-line\ option\
b<ne}\)\.{wsize>.\\n"});{}$\6
${}\\{exit}({-}\T{4});{}$\6
\4${}\}{}$\2\6
\X11:Input the clause in \PB{\\{buf}}\X;\6
\4${}\}{}$\2\6
\&{if} ${}(\R\\{primary\_file}){}$\1\5
${}\\{primary\_vars}\K\\{vars};{}$\2\6
\&{if} ${}((\\{vars}\GG\\{hbits})\G\T{10}){}$\5
${}\{{}$\1\6
${}\\{fprintf}(\\{stderr},\39\.{"There\ are\ "}\|O\.{"lld\ variables\ but\ o}\)%
\.{nly\ "}\|O\.{"d\ hash\ tables;\\n"},\39\\{vars},\39\T{1}\LL\\{hbits});{}$\6
\&{while} ${}((\\{vars}\GG\\{hbits})\G\T{10}){}$\1\5
${}\\{hbits}\PP;{}$\2\6
${}\\{fprintf}(\\{stderr},\39\.{"\ maybe\ you\ should\ u}\)\.{se\ command-line\
opti}\)\.{on\ h"}\|O\.{"d?\\n"},\39\\{hbits});{}$\6
\4${}\}{}$\2\6
${}\\{clauses}\MRL{-{\K}}\\{nullclauses};{}$\6
\&{if} ${}(\\{clauses}\E\T{0}){}$\5
${}\{{}$\1\6
${}\\{fprintf}(\\{stderr},\39\.{"No\ clauses\ were\ inp}\)\.{ut!\\n"});{}$\6
${}\\{exit}({-}\T{77});{}$\6
\4${}\}{}$\2\6
\&{if} ${}(\\{vars}\G\T{\^80000000}){}$\5
${}\{{}$\1\6
${}\\{fprintf}(\\{stderr},\39\.{"Whoa,\ the\ input\ had}\)\.{\ "}\|O\.{"llu\
variables!\\n"},\39\\{cells});{}$\6
${}\\{exit}({-}\T{664});{}$\6
\4${}\}{}$\2\6
\&{if} ${}(\\{clauses}\G\T{\^80000000}){}$\5
${}\{{}$\1\6
${}\\{fprintf}(\\{stderr},\39\.{"Whoa,\ the\ input\ had}\)\.{\ "}\|O\.{"llu\
clauses!\\n"},\39\\{cells});{}$\6
${}\\{exit}({-}\T{665});{}$\6
\4${}\}{}$\2\6
\&{if} ${}(\\{cells}\G\T{\^100000000}){}$\5
${}\{{}$\1\6
${}\\{fprintf}(\\{stderr},\39\.{"Whoa,\ the\ input\ had}\)\.{\ "}\|O\.{"llu\
occurrences\ of\ }\)\.{literals!\\n"},\39\\{cells});{}$\6
${}\\{exit}({-}\T{666});{}$\6
\4${}\}{}$\2\par
\U2.\fi

\M{10}We input from \PB{\\{primary\_file}} just as if it were the standard
input
file, except that all ``clauses'' are discarded. (Line numbers in
error messages are zero.) The effect is to place
the primary variables first in the list of all variables: A variable
is primary if and only if its index is \PB{$\Z$ \\{primary\_vars}}.

\Y\B\4\X10:Input the primary variables\X${}\E{}$\6
${}\{{}$\1\6
\&{while} (\T{1})\5
${}\{{}$\1\6
\&{if} ${}(\R\\{fgets}(\\{buf},\39\\{buf\_size},\39\\{primary\_file})){}$\1\5
\&{break};\2\6
\&{if} ${}(\\{buf}[\\{strlen}(\\{buf})-\T{1}]\I\.{'\\n'}){}$\5
${}\{{}$\1\6
${}\\{fprintf}(\\{stderr},\39\.{"The\ clause\ on\ line\ }\)\.{"}\|O\.{"lld\ ("}%
\|O\.{".20s...)\ is\ too\ lon}\)\.{g\ for\ me;\\n"},\39\\{clauses},\39%
\\{buf});{}$\6
${}\\{fprintf}(\\{stderr},\39\.{"\ my\ buf\_size\ is\ onl}\)\.{y\ "}\|O\.{"d!%
\\n"},\39\\{buf\_size});{}$\6
${}\\{fprintf}(\\{stderr},\39\.{"Please\ use\ the\ comm}\)\.{and-line\ option\
b<ne}\)\.{wsize>.\\n"});{}$\6
${}\\{exit}({-}\T{4});{}$\6
\4${}\}{}$\2\6
\X11:Input the clause in \PB{\\{buf}}\X;\6
\X19:Remove all variables of the current clause\X;\6
\4${}\}{}$\2\6
${}\\{cells}\K\\{nullclauses}\K\T{0};{}$\6
${}\\{primary\_vars}\K\\{vars};{}$\6
\&{if} ${}(\\{verbose}\AND\\{show\_basics}){}$\1\5
${}\\{fprintf}(\\{stderr},\39\.{"("}\|O\.{"d\ primary\ variables}\)\.{\ read\
from\ "}\|O\.{"s)\\n"},\39\\{primary\_vars},\39\\{primary\_name});{}$\2\6
\4${}\}{}$\2\par
\U9.\fi

\M{11}\B\X11:Input the clause in \PB{\\{buf}}\X${}\E{}$\6
\&{for} ${}(\|j\K\|k\K\\{non\_clause}\K\T{0};{}$ ${}\R\\{non\_clause};{}$ \,)\5
${}\{{}$\1\6
\&{while} ${}(\\{buf}[\|j]\E\.{'\ '}){}$\1\5
${}\|j\PP{}$;\C{ scan to nonblank }\2\6
\&{if} ${}(\\{buf}[\|j]\E\.{'\\n'}){}$\1\5
\&{break};\2\6
\&{if} ${}(\\{buf}[\|j]<\.{'\ '}\V\\{buf}[\|j]>\.{'\~'}){}$\5
${}\{{}$\1\6
${}\\{fprintf}(\\{stderr},\39\.{"Illegal\ character\ (}\)\.{code\ \#"}\|O\.{"x)%
\ in\ the\ clause\ on}\)\.{\ line\ "}\|O\.{"lld!\\n"},\39\\{buf}[\|j],\39%
\\{clauses});{}$\6
${}\\{exit}({-}\T{5});{}$\6
\4${}\}{}$\2\6
\&{if} ${}(\\{buf}[\|j]\E\.{'\~'}){}$\1\5
${}\|i\K\T{1},\39\|j\PP;{}$\2\6
\&{else}\1\5
${}\|i\K\T{0};{}$\2\6
\X12:Scan and record a variable; negate it if \PB{$\|i\E\T{1}$}\X;\6
\4${}\}{}$\2\6
\&{if} ${}(\|k\E\T{0}\W\R\\{non\_clause}){}$\5
${}\{{}$\1\6
${}\\{fprintf}(\\{stderr},\39\.{"(Empty\ line\ "}\|O\.{"lld\ is\ being\ ignore}%
\)\.{d)\\n"},\39\\{clauses});{}$\6
${}\\{nullclauses}\PP{}$;\C{ strictly speaking it would be unsatisfiable }\6
\4${}\}{}$\2\6
\&{if} (\\{non\_clause})\1\5
\X19:Remove all variables of the current clause\X;\2\6
${}\\{cells}\MRL{+{\K}}\|k{}$;\par
\Us9\ET10.\fi

\M{12}We need a hack to insert the bit codes 1 and/or 2 into a pointer value.

\Y\B\4\D$\\{hack\_in}(\|q,\|t)$ \5
(\&{tmp\_var} ${}{*})(\|t\OR{}$(\&{ullng}) \|q)\par
\Y\B\4\X12:Scan and record a variable; negate it if \PB{$\|i\E\T{1}$}\X${}\E{}$%
\6
${}\{{}$\1\6
\&{register} \&{tmp\_var} ${}{*}\|p;{}$\7
\&{if} ${}(\\{cur\_tmp\_var}\E\\{bad\_tmp\_var}){}$\1\5
\X13:Install a new \PB{\&{vchunk}}\X;\2\6
\X16:Put the variable name beginning at \PB{\\{buf}[\|j]} in \PB{$\\{cur\_tmp%
\_var}\MG\\{name}$} and compute its hash code \PB{\|h}\X;\6
\&{if} ${}(\R\\{non\_clause}){}$\5
${}\{{}$\1\6
\X17:Find \PB{$\\{cur\_tmp\_var}\MG\\{name}$} in the hash table at \PB{\|p}\X;\6
\&{if} ${}(\\{clauses}\W(\|p\MG\\{stamp}\E\\{clauses}\V\|p\MG\\{stamp}\E{-}%
\\{clauses})){}$\1\5
\X18:Handle a duplicate literal\X\2\6
\&{else}\5
${}\{{}$\1\6
${}\|p\MG\\{stamp}\K(\|i\?{-}\\{clauses}:\\{clauses});{}$\6
\&{if} ${}(\\{cur\_cell}\E\\{bad\_cell}){}$\1\5
\X14:Install a new \PB{\&{chunk}}\X;\2\6
${}{*}\\{cur\_cell}\K\|p;{}$\6
\&{if} ${}(\|i\E\T{1}){}$\1\5
${}{*}\\{cur\_cell}\K\\{hack\_in}({*}\\{cur\_cell},\39\T{1});{}$\2\6
\&{if} ${}(\|k\E\T{0}){}$\1\5
${}{*}\\{cur\_cell}\K\\{hack\_in}({*}\\{cur\_cell},\39\T{2});{}$\2\6
${}\\{cur\_cell}\PP,\39\|k\PP;{}$\6
\4${}\}{}$\2\6
\4${}\}{}$\2\6
\4${}\}{}$\2\par
\U11.\fi

\M{13}\B\X13:Install a new \PB{\&{vchunk}}\X${}\E{}$\6
${}\{{}$\1\6
\&{register} \&{vchunk} ${}{*}\\{new\_vchunk};{}$\7
${}\\{new\_vchunk}\K{}$(\&{vchunk} ${}{*}){}$ \\{malloc}(\&{sizeof}(%
\&{vchunk}));\6
\&{if} ${}(\R\\{new\_vchunk}){}$\5
${}\{{}$\1\6
${}\\{fprintf}(\\{stderr},\39\.{"Can't\ allocate\ a\ ne}\)\.{w\ vchunk!%
\\n"});{}$\6
${}\\{exit}({-}\T{6});{}$\6
\4${}\}{}$\2\6
${}\\{new\_vchunk}\MG\\{prev}\K\\{cur\_vchunk},\39\\{cur\_vchunk}\K\\{new%
\_vchunk};{}$\6
${}\\{cur\_tmp\_var}\K{\AND}\\{new\_vchunk}\MG\\{var}[\T{0}];{}$\6
${}\\{bad\_tmp\_var}\K{\AND}\\{new\_vchunk}\MG\\{var}[\\{vars\_per%
\_vchunk}];{}$\6
\4${}\}{}$\2\par
\U12.\fi

\M{14}\B\X14:Install a new \PB{\&{chunk}}\X${}\E{}$\6
${}\{{}$\1\6
\&{register} \&{chunk} ${}{*}\\{new\_chunk};{}$\7
${}\\{new\_chunk}\K{}$(\&{chunk} ${}{*}){}$ \\{malloc}(\&{sizeof}(\&{chunk}));\6
\&{if} ${}(\R\\{new\_chunk}){}$\5
${}\{{}$\1\6
${}\\{fprintf}(\\{stderr},\39\.{"Can't\ allocate\ a\ ne}\)\.{w\ chunk!%
\\n"});{}$\6
${}\\{exit}({-}\T{7});{}$\6
\4${}\}{}$\2\6
${}\\{new\_chunk}\MG\\{prev}\K\\{cur\_chunk},\39\\{cur\_chunk}\K\\{new%
\_chunk};{}$\6
${}\\{cur\_cell}\K{\AND}\\{new\_chunk}\MG\\{cell}[\T{0}];{}$\6
${}\\{bad\_cell}\K{\AND}\\{new\_chunk}\MG\\{cell}[\\{cells\_per\_chunk}];{}$\6
\4${}\}{}$\2\par
\U12.\fi

\M{15}The hash code is computed via ``universal hashing,'' using the following
precomputed tables of random bits.

\Y\B\4\X8:Initialize everything\X${}\mathrel+\E{}$\6
\&{for} ${}(\|j\K\T{92};{}$ \|j; ${}\|j\MM){}$\1\6
\&{for} ${}(\|k\K\T{0};{}$ ${}\|k<\T{8};{}$ ${}\|k\PP){}$\1\5
${}\\{hash\_bits}[\|j][\|k]\K\\{gb\_next\_rand}(\,){}$;\2\2\par
\fi

\M{16}\B\X16:Put the variable name beginning at \PB{\\{buf}[\|j]} in \PB{$%
\\{cur\_tmp\_var}\MG\\{name}$} and compute its hash code \PB{\|h}\X${}\E{}$\6
$\\{cur\_tmp\_var}\MG\\{name}.\\{lng}\K\T{0};{}$\6
\&{for} ${}(\|h\K\|l\K\T{0};{}$ ${}\\{buf}[\|j+\|l]>\.{'\ '}\W\\{buf}[\|j+\|l]%
\Z\.{'\~'};{}$ ${}\|l\PP){}$\5
${}\{{}$\1\6
\&{if} ${}(\|l>\T{7}){}$\5
${}\{{}$\1\6
${}\\{fprintf}(\\{stderr},\39\.{"Variable\ name\ "}\|O\.{".9s...\ in\ the\
claus}\)\.{e\ on\ line\ "}\|O\.{"lld\ is\ too\ long!\\n"},\39\\{buf}+\|j,\39%
\\{clauses});{}$\6
${}\\{exit}({-}\T{8});{}$\6
\4${}\}{}$\2\6
${}\|h\MRL{{\XOR}{\K}}\\{hash\_bits}[\\{buf}[\|j+\|l]-\.{'!'}][\|l];{}$\6
${}\\{cur\_tmp\_var}\MG\\{name}.\\{ch8}[\|l]\K\\{buf}[\|j+\|l];{}$\6
\4${}\}{}$\2\6
\&{if} ${}(\|l\E\T{0}){}$\1\5
${}\\{non\_clause}\K\T{1}{}$;\C{ `\.\~' by itself is like `true' }\2\6
\&{else}\1\5
${}\|j\MRL{+{\K}}\|l,\39\|h\MRL{\AND{\K}}(\T{1}\LL\\{hbits})-\T{1}{}$;\2\par
\U12.\fi

\M{17}\B\X17:Find \PB{$\\{cur\_tmp\_var}\MG\\{name}$} in the hash table at \PB{%
\|p}\X${}\E{}$\6
\&{for} ${}(\|p\K\\{hash}[\|h];{}$ \|p; ${}\|p\K\|p\MG\\{next}){}$\1\6
\&{if} ${}(\|p\MG\\{name}.\\{lng}\E\\{cur\_tmp\_var}\MG\\{name}.\\{lng}){}$\1\5
\&{break};\2\2\6
\&{if} ${}(\R\|p){}$\5
${}\{{}$\C{ new variable found }\1\6
${}\|p\K\\{cur\_tmp\_var}\PP;{}$\6
${}\|p\MG\\{next}\K\\{hash}[\|h],\39\\{hash}[\|h]\K\|p;{}$\6
${}\|p\MG\\{serial}\K\\{vars}\PP;{}$\6
${}\|p\MG\\{stamp}\K\T{0};{}$\6
\4${}\}{}$\2\par
\U12.\fi

\M{18}The most interesting aspect of the input phase is probably the
``unwinding''
that we might need to do when encountering a literal more than once
in the same clause.

\Y\B\4\X18:Handle a duplicate literal\X${}\E{}$\6
${}\{{}$\1\6
\&{if} ${}((\|p\MG\\{stamp}>\T{0})\E(\|i>\T{0})){}$\1\5
${}\\{non\_clause}\K\T{1}{}$;\C{ tautology }\2\6
\4${}\}{}$\2\par
\U12.\fi

\M{19}An input line that begins with `\.{\~\ }' is silently treated as a
comment.
Otherwise redundant clauses are logged, in case they were unintentional.
(One can, however, intentionally
use redundant clauses to force the order of the variables.)

\Y\B\4\X19:Remove all variables of the current clause\X${}\E{}$\6
${}\{{}$\1\6
\&{while} (\|k)\5
${}\{{}$\1\6
\X20:Move \PB{\\{cur\_cell}} backward to the previous cell\X;\6
${}\|k\MM;{}$\6
\4${}\}{}$\2\6
\&{if} ${}(\\{non\_clause}\W((\\{buf}[\T{0}]\I\.{'\~'})\V(\\{buf}[\T{1}]\I\.{'\
'}))){}$\1\5
${}\\{fprintf}(\\{stderr},\39\.{"(The\ clause\ on\ line}\)\.{\ "}\|O\.{"lld\ is%
\ always\ satis}\)\.{fied)\\n"},\39\\{clauses});{}$\2\6
${}\\{nullclauses}\PP;{}$\6
\4${}\}{}$\2\par
\Us10\ET11.\fi

\M{20}\B\X20:Move \PB{\\{cur\_cell}} backward to the previous cell\X${}\E{}$\6
\&{if} ${}(\\{cur\_cell}>{\AND}\\{cur\_chunk}\MG\\{cell}[\T{0}]){}$\1\5
${}\\{cur\_cell}\MM;{}$\2\6
\&{else}\5
${}\{{}$\1\6
\&{register} \&{chunk} ${}{*}\\{old\_chunk}\K\\{cur\_chunk};{}$\7
${}\\{cur\_chunk}\K\\{old\_chunk}\MG\\{prev}{}$;\5
\\{free}(\\{old\_chunk});\6
${}\\{bad\_cell}\K{\AND}\\{cur\_chunk}\MG\\{cell}[\\{cells\_per\_chunk}];{}$\6
${}\\{cur\_cell}\K\\{bad\_cell}-\T{1};{}$\6
\4${}\}{}$\2\par
\Us19\ET36.\fi

\M{21}\B\X21:Move \PB{\\{cur\_tmp\_var}} backward to the previous temporary
variable\X${}\E{}$\6
\&{if} ${}(\\{cur\_tmp\_var}>{\AND}\\{cur\_vchunk}\MG\\{var}[\T{0}]){}$\1\5
${}\\{cur\_tmp\_var}\MM;{}$\2\6
\&{else}\5
${}\{{}$\1\6
\&{register} \&{vchunk} ${}{*}\\{old\_vchunk}\K\\{cur\_vchunk};{}$\7
${}\\{cur\_vchunk}\K\\{old\_vchunk}\MG\\{prev}{}$;\5
\\{free}(\\{old\_vchunk});\6
${}\\{bad\_tmp\_var}\K{\AND}\\{cur\_vchunk}\MG\\{var}[\\{vars\_per%
\_vchunk}];{}$\6
${}\\{cur\_tmp\_var}\K\\{bad\_tmp\_var}-\T{1};{}$\6
\4${}\}{}$\2\par
\U37.\fi

\M{22}\B\X22:Report the successful completion of the input phase\X${}\E{}$\6
$\\{fprintf}(\\{stderr},\39\.{"("}\|O\.{"lld\ variables,\ "}\|O\.{"lld\
clauses,\ "}\|O\.{"llu\ literals\ succes}\)\.{sfully\ read)\\n"},\39\\{vars},%
\39\\{clauses},\39\\{cells}){}$;\par
\U2.\fi

\N{1}{23}SAT solving, version 10. Okay, now comes my hypothetical recreation
of the Brown--Purdom SAT solver. (It's unfortunate that no copy of
their original program survives.)

The algorithm below essentially solves a satisfiability problem
by backtracking. At each level it tries two possibilities for some
unset variable, unless it finds an unset variable for which there's
only one viable possibility based on previously set variables
(thus making a forced move), or unless it finds an unset variable
with {\it no\/} viable possibilities (in which case it backs up to the
previous level of branching).

The key idea is that the first literal in every clause is considered
to be ``watched,'' and the watched literal has not been set false.
If the algorithm does want to make that literal false, it must first
swap another literal of the clause into the first position.

This method can be implemented with extremely simple data structures:
\smallskip
\item{$\bullet$}For each clause $c$, there's a sequential list of
the literals in~$c$.
\item{$\bullet$}For each variable $v$, there are linked lists of the clauses
that are watching $v$ and $\bar v$.
\item{$\bullet$}Each variable is either set to true, set to false, or unknown.
\item{$\bullet$}There's a circular list containing all the unset variables
whose
literals are watched by at least one clause. (This list is called the active
ring; we're done when it becomes empty at decision time.)
\smallskip\noindent
And of course we remember the current trail of decisions made at each level of
the implicit backtrack tree.

\fi

\M{24}Each link is a 32-bit integer. (I don't use \CEE/ pointers in the main
data structures, because they occupy 64 bits and clutter up the caches.)
Links in the watch lists are indexes of clauses; links in the active ring are
indexes of variables. A zero link indicates the end of a list.

The literals within a clause, called ``cells,'' are 32-bit unsigned
integers kept in a big array called \PB{\\{mem}}. Variable number $k$,
for $1\le k\le vars$, corresponds to the literals numbered $2k$ and $2k+1$.

Each clause is represented by a pointer to its first cell and by a link
to the successor clause (if any) with the same watched literal.

\Y\B\4\X5:Type definitions\X${}\mathrel+\E{}$\6
\&{typedef} \&{struct} ${}\{{}$\1\6
\&{uint} \\{start};\C{ the address in \PB{\\{mem}} where the cells for this
clause start }\6
\&{uint} \\{wlink};\C{ link to another clause in the same watch list }\2\6
${}\}{}$ \&{clause};\par
\fi

\M{25}Several items are stored for each variable: The heads of its
two watch lists; the link to the next active variable; a spare field
for miscellaneous use; and the 8-byte symbolic name.

(We also keep the current values of variables in a separate array \PB{\\{val}},
with
one byte for each variable.)

\Y\B\4\D$\\{false}$ \5
\T{0}\C{ \PB{\\{val}} code for a false literal }\par
\B\4\D$\\{true}$ \5
\T{1}\C{ \PB{\\{val}} code for a true literal }\par
\B\4\D$\\{unknown}$ \5
${-}{}$\T{1}\C{ \PB{\\{val}} code for an unset literal }\par
\Y\B\4\X5:Type definitions\X${}\mathrel+\E{}$\6
\&{typedef} \&{struct} ${}\{{}$\1\6
\&{uint} \\{wlist0}${},{}$ \\{wlist1};\C{ heads of the watch lists }\6
\&{int} \\{next};\C{ next item in the ring of active variables }\6
\&{uint} \\{spare};\C{ extra field used only by \PB{\\{sanity}} at the moment }%
\6
\&{octa} \\{name};\C{ the variable's symbolic name }\2\6
${}\}{}$ \&{variable};\par
\fi

\M{26}The backtracking process maintains a sequential stack of state
information.

\Y\B\4\X5:Type definitions\X${}\mathrel+\E{}$\6
\&{typedef} \&{struct} ${}\{{}$\1\6
\&{int} \\{var};\C{ variable whose value is being set }\6
\&{int} \\{move};\C{ code for what we're setting it }\2\6
${}\}{}$ \&{state};\par
\fi

\M{27}\B\X3:Global variables\X${}\mathrel+\E{}$\6
\&{uint} ${}{*}\\{mem}{}$;\C{ the master array of cells }\6
\&{clause} ${}{*}\\{cmem}{}$;\C{ the master array of clauses }\6
\&{variable} ${}{*}\\{vmem}{}$;\C{ the master array of variables }\6
\&{char} ${}{*}\\{val}{}$;\C{ the master array of variable values }\6
\&{state} ${}{*}\\{smem}{}$;\C{ the stack of choices made so far }\6
\&{uint} \\{active};\C{ an item in the active ring, or zero if that ring is
empty }\par
\fi

\M{28}Here is a subroutine that prints a clause symbolically. It illustrates
some of the conventions of the data structures that have been explained above.
I use it only for debugging.

Incidentally, the clause numbers reported to the user after the input phase
may differ from the line numbers reported during the input phase,
when \PB{$\\{nullclauses}>\T{0}$}.

\Y\B\4\X28:Subroutines\X${}\E{}$\6
\&{void} \\{print\_clause}(\&{int} \|c)\1\1\2\2\6
${}\{{}$\1\6
\&{register} \&{uint} \|k${},{}$ \|l;\7
${}\\{printf}(\.{""}\|O\.{"d:"},\39\|c){}$;\C{ show the clause number }\6
\&{for} ${}(\|k\K\\{cmem}[\|c].\\{start};{}$ ${}\|k<\\{cmem}[\|c-\T{1}].%
\\{start};{}$ ${}\|k\PP){}$\5
${}\{{}$\1\6
${}\|l\K\\{mem}[\|k];{}$\6
${}\\{printf}(\.{"\ "}\|O\.{"s"}\|O\.{".8s"},\39\|l\AND\T{1}\?\.{"\~"}:\.{""},%
\39\\{vmem}[\|l\GG\T{1}].\\{name}.\\{ch8}){}$;\C{ $k$th literal }\6
\4${}\}{}$\2\6
\\{printf}(\.{"\\n"});\6
\4${}\}{}$\2\par
\As29, 30, 31\ETs32.
\U2.\fi

\M{29}Similarly we can print out all of the clauses that currently watch
a particular literal.

\Y\B\4\X28:Subroutines\X${}\mathrel+\E{}$\6
\&{void} \\{print\_watches\_for}(\&{int} \|l)\1\1\2\2\6
${}\{{}$\1\6
\&{register} \&{int} \|c;\7
\&{if} ${}(\|l\AND\T{1}){}$\1\5
${}\|c\K\\{vmem}[\|l\GG\T{1}].\\{wlist1};{}$\2\6
\&{else}\1\5
${}\|c\K\\{vmem}[\|l\GG\T{1}].\\{wlist0};{}$\2\6
\&{for} ( ; \|c; ${}\|c\K\\{cmem}[\|c].\\{wlink}){}$\1\5
\\{print\_clause}(\|c);\2\6
\4${}\}{}$\2\par
\fi

\M{30}\B\X28:Subroutines\X${}\mathrel+\E{}$\6
\&{void} \\{print\_ring}(\&{void})\1\1\2\2\6
${}\{{}$\1\6
\&{register} \&{int} \|p;\7
\\{printf}(\.{"Ring:"});\6
\&{if} (\\{active})\5
${}\{{}$\1\6
\&{for} ${}(\|p\K\\{vmem}[\\{active}].\\{next};{}$  ; ${}\|p\K\\{vmem}[\|p].%
\\{next}){}$\5
${}\{{}$\1\6
${}\\{printf}(\.{"\ "}\|O\.{".8s"},\39\\{vmem}[\|p].\\{name}.\\{ch8});{}$\6
\&{if} ${}(\|p\E\\{active}){}$\1\5
\&{break};\2\6
\4${}\}{}$\2\6
\4${}\}{}$\2\6
\\{printf}(\.{"\\n"});\6
\4${}\}{}$\2\par
\fi

\M{31}Speaking of debugging, here's a routine to check if the redundant
parts of our data structure have gone awry.

\Y\B\4\D$\\{sanity\_checking}$ \5
\T{0}\C{ set this to 1 if you suspect a bug }\par
\Y\B\4\X28:Subroutines\X${}\mathrel+\E{}$\6
\&{void} \\{sanity}(\&{void})\1\1\2\2\6
${}\{{}$\1\6
\&{register} \&{int} \|k${},{}$ \|l${},{}$ \|c${},{}$ \|v;\7
\&{if} (\\{active})\5
${}\{{}$\1\6
\&{for} ${}(\|v\K\\{vmem}[\\{active}].\\{next};{}$  ; ${}\|v\K\\{vmem}[\|v].%
\\{next}){}$\5
${}\{{}$\1\6
${}\\{vmem}[\|v].\\{spare}\K\T{1}{}$;\C{ all \PB{\\{spare}} fields assumed zero
otherwise }\6
\&{if} ${}(\|v\E\\{active}){}$\1\5
\&{break};\2\6
\4${}\}{}$\2\6
\4${}\}{}$\2\6
${}\|k\K\T{0};{}$\6
\&{for} ${}(\|v\K\T{1};{}$ ${}\|v\Z\\{vars};{}$ ${}\|v\PP){}$\5
${}\{{}$\1\6
\&{for} ${}(\|c\K\\{vmem}[\|v].\\{wlist0};{}$ \|c; ${}\|c\K\\{cmem}[\|c].%
\\{wlink}){}$\5
${}\{{}$\1\6
${}\|k\PP;{}$\6
\&{if} ${}(\\{mem}[\\{cmem}[\|c].\\{start}]\I\|v+\|v){}$\1\5
${}\\{fprintf}(\\{stderr},\39\.{"Clause\ "}\|O\.{"d\ watches\ "}\|O\.{"u,\ not\
"}\|O\.{"u!\\n"},\39\|c,\39\\{mem}[\\{cmem}[\|c].\\{start}],\39\|v+\|v);{}$\2\6
\&{else} \&{if} ${}(\\{val}[\|v]\E\\{false}){}$\1\5
${}\\{fprintf}(\\{stderr},\39\.{"Clause\ "}\|O\.{"d\ watches\ the\ false}\)\.{\
literal\ "}\|O\.{"u!\\n"},\39\|c,\39(\|v+\|v));{}$\2\6
\4${}\}{}$\2\6
\&{for} ${}(\|c\K\\{vmem}[\|v].\\{wlist1};{}$ \|c; ${}\|c\K\\{cmem}[\|c].%
\\{wlink}){}$\5
${}\{{}$\1\6
${}\|k\PP;{}$\6
\&{if} ${}(\\{mem}[\\{cmem}[\|c].\\{start}]\I\|v+\|v+\T{1}){}$\1\5
${}\\{fprintf}(\\{stderr},\39\.{"Clause\ "}\|O\.{"d\ watches\ "}\|O\.{"u,\ not\
"}\|O\.{"u!\\n"},\39\|c,\39\\{mem}[\\{cmem}[\|c].\\{start}],\39\|v+\|v+%
\T{1});{}$\2\6
\&{else} \&{if} ${}(\\{val}[\|v]\E\\{true}){}$\1\5
${}\\{fprintf}(\\{stderr},\39\.{"Clause\ "}\|O\.{"d\ watches\ the\ false}\)\.{\
literal\ "}\|O\.{"u!\\n"},\39\|c,\39(\|v+\|v+\T{1}));{}$\2\6
\4${}\}{}$\2\6
\&{if} ${}(\\{vmem}[\|v].\\{spare}\E\T{0}\W\\{val}[\|v]\E\\{unknown}\W(%
\\{vmem}[\|v].\\{wlist0}\V\\{vmem}[\|v].\\{wlist1})){}$\1\5
${}\\{fprintf}(\\{stderr},\39\.{"Variable\ "}\|O\.{".8s\ should\ be\ in\ th}\)%
\.{e\ active\ ring!\\n"},\39\\{vmem}[\|v].\\{name}.\\{ch8});{}$\2\6
\&{if} ${}(\\{vmem}[\|v].\\{spare}\E\T{1}\W(\\{val}[\|v]\I\\{unknown}\V((%
\\{vmem}[\|v].\\{wlist0}\OR\\{vmem}[\|v].\\{wlist1})\E\T{0}))){}$\1\5
${}\\{fprintf}(\\{stderr},\39\.{"Variable\ "}\|O\.{".8s\ should\ not\ be\ i}\)%
\.{n\ the\ active\ ring!\\n}\)\.{"},\39\\{vmem}[\|v].\\{name}.\\{ch8});{}$\2\6
${}\\{vmem}[\|v].\\{spare}\K\T{0};{}$\6
\4${}\}{}$\2\6
\&{if} ${}(\|k\I\\{clauses}){}$\1\5
${}\\{fprintf}(\\{stderr},\39\.{"Oops:\ "}\|O\.{"d\ of\ "}\|O\.{"lld\ clauses\
are\ bei}\)\.{ng\ watched!\\n"},\39\|k,\39\\{clauses});{}$\2\6
\4${}\}{}$\2\par
\fi

\M{32}In long runs it's helpful to know how far we've gotten.

\Y\B\4\X28:Subroutines\X${}\mathrel+\E{}$\6
\&{void} \\{print\_state}(\&{int} \|l)\1\1\2\2\6
${}\{{}$\1\6
\&{register} \&{int} \|k;\7
${}\\{fprintf}(\\{stderr},\39\.{"\ after\ "}\|O\.{"lld\ mems:"},\39%
\\{mems});{}$\6
\&{for} ${}(\|k\K\T{1};{}$ ${}\|k\Z\|l;{}$ ${}\|k\PP){}$\1\5
${}\\{fprintf}(\\{stderr},\39\.{""}\|O\.{"c"},\39\\{smem}[\|k].\\{move}+%
\.{'0'});{}$\2\6
${}\\{fprintf}(\\{stderr},\39\.{"\\n"});{}$\6
\\{fflush}(\\{stderr});\6
\4${}\}{}$\2\par
\fi

\N{1}{33}Initializing the real data structures.
We're ready now to convert the temporary chunks of data into the
form we want, and to recycle those chunks. The code below is, of course,
similar to what has worked in previous programs of this series.

\Y\B\4\X33:Set up the main data structures\X${}\E{}$\6
\X34:Allocate the main arrays\X;\6
\X35:Copy all the temporary cells to the \PB{\\{mem}} and \PB{\\{cmem}} arrays
in proper format\X;\6
\X37:Copy all the temporary variable nodes to the \PB{\\{vmem}} array in proper
format\X;\6
\X38:Check consistency\X;\6
\&{if} (\\{out\_file})\1\5
\X39:Copy all the clauses to \PB{\\{out\_file}}\X;\2\par
\U2.\fi

\M{34}\B\X34:Allocate the main arrays\X${}\E{}$\6
\\{free}(\\{buf});\5
\\{free}(\\{hash});\C{ a tiny gesture to make a little room }\6
${}\\{mem}\K{}$(\&{uint} ${}{*}){}$ \\{malloc}${}(\\{cells}*\&{sizeof}(%
\&{uint}));{}$\6
\&{if} ${}(\R\\{mem}){}$\5
${}\{{}$\1\6
${}\\{fprintf}(\\{stderr},\39\.{"Oops,\ I\ can't\ alloc}\)\.{ate\ the\ big\ mem%
\ arra}\)\.{y!\\n"});{}$\6
${}\\{exit}({-}\T{10});{}$\6
\4${}\}{}$\2\6
${}\\{bytes}\K\\{cells}*\&{sizeof}(\&{uint});{}$\6
${}\\{cmem}\K{}$(\&{clause} ${}{*}){}$ \\{malloc}${}((\\{clauses}+\T{1})*%
\&{sizeof}(\&{clause}));{}$\6
\&{if} ${}(\R\\{cmem}){}$\5
${}\{{}$\1\6
${}\\{fprintf}(\\{stderr},\39\.{"Oops,\ I\ can't\ alloc}\)\.{ate\ the\ cmem\
array!\\}\)\.{n"});{}$\6
${}\\{exit}({-}\T{11});{}$\6
\4${}\}{}$\2\6
${}\\{bytes}\MRL{+{\K}}(\\{clauses}+\T{1})*\&{sizeof}(\&{clause});{}$\6
${}\\{vmem}\K{}$(\&{variable} ${}{*}){}$ \\{malloc}${}((\\{vars}+\T{1})*%
\&{sizeof}(\&{variable}));{}$\6
\&{if} ${}(\R\\{vmem}){}$\5
${}\{{}$\1\6
${}\\{fprintf}(\\{stderr},\39\.{"Oops,\ I\ can't\ alloc}\)\.{ate\ the\ vmem\
array!\\}\)\.{n"});{}$\6
${}\\{exit}({-}\T{12});{}$\6
\4${}\}{}$\2\6
${}\\{bytes}\MRL{+{\K}}(\\{vars}+\T{1})*\&{sizeof}(\&{variable});{}$\6
${}\\{smem}\K{}$(\&{state} ${}{*}){}$ \\{malloc}${}((\\{vars}+\T{1})*%
\&{sizeof}(\&{state}));{}$\6
\&{if} ${}(\R\\{smem}){}$\5
${}\{{}$\1\6
${}\\{fprintf}(\\{stderr},\39\.{"Oops,\ I\ can't\ alloc}\)\.{ate\ the\ smem\
array!\\}\)\.{n"});{}$\6
${}\\{exit}({-}\T{13});{}$\6
\4${}\}{}$\2\6
${}\\{bytes}\MRL{+{\K}}(\\{vars}+\T{1})*\&{sizeof}(\&{state});{}$\6
${}\\{val}\K{}$(\&{char} ${}{*}){}$ \\{malloc}${}((\\{vars}+\T{1})*\&{sizeof}(%
\&{char}));{}$\6
\&{if} ${}(\R\\{val}){}$\5
${}\{{}$\1\6
${}\\{fprintf}(\\{stderr},\39\.{"Oops,\ I\ can't\ alloc}\)\.{ate\ the\ val\
array!\\n}\)\.{"});{}$\6
${}\\{exit}({-}\T{14});{}$\6
\4${}\}{}$\2\6
${}\\{bytes}\MRL{+{\K}}(\\{vars}+\T{1})*\&{sizeof}(\&{char}){}$;\par
\U33.\fi

\M{35}\B\X35:Copy all the temporary cells to the \PB{\\{mem}} and \PB{\\{cmem}}
arrays in proper format\X${}\E{}$\6
\&{for} ${}(\|j\K\T{1};{}$ ${}\|j\Z\\{vars};{}$ ${}\|j\PP){}$\5
${}\{{}$\1\6
${}\|o,\39\\{vmem}[\|j].\\{wlist0}\K\\{vmem}[\|j].\\{wlist1}\K\T{0};{}$\6
${}\|o,\39\\{val}[\|j]\K\\{unknown};{}$\6
\4${}\}{}$\2\6
\&{for} ${}(\|c\K\\{clauses},\39\|j\K\T{0};{}$ \|c; ${}\|c\MM){}$\5
${}\{{}$\1\6
${}\|o,\39\\{cmem}[\|c].\\{start}\K\|k\K\|j;{}$\6
\X36:Insert the cells for the literals of clause \PB{\|c}\X;\6
${}\|l\K\\{mem}[\|k];{}$\6
\&{if} ${}(\|l\AND\T{1}){}$\1\5
${}\\{ooo},\39\|p\K\\{vmem}[\|l\GG\T{1}].\\{wlist1},\39\\{cmem}[\|c].\\{wlink}%
\K\|p,\39\\{vmem}[\|l\GG\T{1}].\\{wlist1}\K\|c;{}$\2\6
\&{else}\1\5
${}\\{ooo},\39\|p\K\\{vmem}[\|l\GG\T{1}].\\{wlist0},\39\\{cmem}[\|c].\\{wlink}%
\K\|p,\39\\{vmem}[\|l\GG\T{1}].\\{wlist0}\K\|c;{}$\2\6
\4${}\}{}$\2\6
\&{if} ${}(\|j\I\\{cells}){}$\5
${}\{{}$\1\6
${}\\{fprintf}(\\{stderr},\39\.{"Oh\ oh,\ something\ ha}\)\.{ppened\ to\ "}\|O%
\.{"d\ cells!\\n"},\39{}$(\&{int}) \\{cells}${}-\|j);{}$\6
${}\\{exit}({-}\T{15});{}$\6
\4${}\}{}$\2\6
${}\|o,\39\\{cmem}[\|c].\\{start}\K\|j{}$;\par
\U33.\fi

\M{36}The basic idea is to ``unwind'' the steps that we went through while
building up the chunks.

\Y\B\4\D$\\{hack\_out}(\|q)$ \5
(((\&{ullng}) \|q)${}\AND\T{\^3}{}$)\par
\B\4\D$\\{hack\_clean}(\|q)$ \5
((\&{tmp\_var} ${}{*})({}$(\&{ullng}) \|q${}\AND{-}\T{4}){}$)\par
\Y\B\4\X36:Insert the cells for the literals of clause \PB{\|c}\X${}\E{}$\6
\&{for} ${}(\|i\K\T{0};{}$ ${}\|i<\T{2};{}$ \,)\5
${}\{{}$\1\6
\X20:Move \PB{\\{cur\_cell}} backward to the previous cell\X;\6
${}\|i\K\\{hack\_out}({*}\\{cur\_cell});{}$\6
${}\|p\K\\{hack\_clean}({*}\\{cur\_cell})\MG\\{serial};{}$\6
${}\|p\MRL{+{\K}}\|p+(\|i\AND\T{1})+\T{2};{}$\6
${}\|o,\39\\{mem}[\|j\PP]\K\|p;{}$\6
\4${}\}{}$\2\par
\U35.\fi

\M{37}\B\X37:Copy all the temporary variable nodes to the \PB{\\{vmem}} array
in proper format\X${}\E{}$\6
\&{for} ${}(\|c\K\\{vars};{}$ \|c; ${}\|c\MM){}$\5
${}\{{}$\1\6
\X21:Move \PB{\\{cur\_tmp\_var}} backward to the previous temporary variable\X;%
\6
${}\|o,\39\\{vmem}[\|c].\\{name}.\\{lng}\K\\{cur\_tmp\_var}\MG\\{name}.%
\\{lng};{}$\6
\4${}\}{}$\2\par
\U33.\fi

\M{38}We should now have unwound all the temporary data chunks back to their
beginnings.

\Y\B\4\X38:Check consistency\X${}\E{}$\6
\&{if} ${}(\\{cur\_cell}\I{\AND}\\{cur\_chunk}\MG\\{cell}[\T{0}]\V\\{cur%
\_chunk}\MG\\{prev}\I\NULL\V\\{cur\_tmp\_var}\I{\AND}\\{cur\_vchunk}\MG\\{var}[%
\T{0}]\V\\{cur\_vchunk}\MG\\{prev}\I\NULL){}$\5
${}\{{}$\1\6
${}\\{fprintf}(\\{stderr},\39\.{"This\ can't\ happen\ (}\)\.{consistency\ check%
\ fa}\)\.{ilure)!\\n"});{}$\6
${}\\{exit}({-}\T{14});{}$\6
\4${}\}{}$\2\6
\\{free}(\\{cur\_chunk});\5
\\{free}(\\{cur\_vchunk});\par
\U33.\fi

\M{39}\B\X39:Copy all the clauses to \PB{\\{out\_file}}\X${}\E{}$\6
${}\{{}$\1\6
\&{for} ${}(\|k\K\T{0},\39\|c\K\\{clauses};{}$ \|c; ${}\|c\MM){}$\5
${}\{{}$\1\6
\&{for} ( ; ${}\|k<\\{cmem}[\|c-\T{1}].\\{start};{}$ ${}\|k\PP){}$\5
${}\{{}$\1\6
${}\|l\K\\{mem}[\|k];{}$\6
${}\\{fprintf}(\\{out\_file},\39\.{"\ "}\|O\.{"s"}\|O\.{".8s"},\39\|l\AND\T{1}%
\?\.{"\~"}:\.{""},\39\\{vmem}[\|l\GG\T{1}].\\{name}.\\{ch8}){}$;\C{ $k$th
literal }\6
\4${}\}{}$\2\6
${}\\{fprintf}(\\{out\_file},\39\.{"\\n"});{}$\6
\4${}\}{}$\2\6
\\{fflush}(\\{out\_file});\C{ complete the copy of input clauses }\6
\4${}\}{}$\2\par
\U33.\fi

\N{1}{40}Doing it. Now comes ye olde basic backtrack, but simplified because
updates to the watch lists don't have to be undone.

At level \PB{\|l} of the backtrack process we record the variable, \PB{$%
\\{smem}[\|l].\\{var}$},
whose value is being specified, and its chosen value, \PB{$\\{smem}[\|l].%
\\{move}$}.
The latter value is 0 or 1 if we're making a binary branch and
we're trying first to make the variable true or false, respectively;
it is 3 or 2 if that move failed and we're trying the other alternative.
It is 4 or 5 if the move was forced and the variable had to be set
respectively to true or false.

\Y\B\4\X40:Solve the problem\X${}\E{}$\6
$\|o,\39\\{level}\K\T{0},\39\\{smem}[\T{0}].\\{move}\K\T{0};{}$\6
\X41:Initialize the active ring\X;\6
\4\\{choose}:\5
\&{if} (\\{sanity\_checking})\1\5
\\{sanity}(\,);\2\6
\&{if} ${}(\\{delta}\W(\\{mems}\G\\{thresh})){}$\1\5
${}\\{thresh}\MRL{+{\K}}\\{delta},\39\\{print\_state}(\\{level});{}$\2\6
\&{if} ${}(\\{mems}>\\{timeout}){}$\5
${}\{{}$\1\6
${}\\{fprintf}(\\{stderr},\39\.{"TIMEOUT!\\n"});{}$\6
\&{goto} \\{done};\6
\4${}\}{}$\2\6
\X42:Decide what to do next, going either to \PB{\\{branch}} or \PB{%
\\{forcedmove}} or \PB{\\{backup}} or \PB{\\{satisfied}}\X;\6
\4\\{branch}:\5
${}\|o,\39\\{nextmove}\K(\\{vmem}[\|v].\\{wlist0}\E\T{0}\V\\{vmem}[\|v].%
\\{wlist1}\I\T{0});{}$\6
${}\\{nodes}\PP;{}$\6
\4\\{forcedmove}:\5
${}\\{level}\PP{}$;\C{ at this point \PB{$\\{vmem}[\\{active}].\\{next}\K\|v$}
is the branch variable }\6
${}\|o,\39\\{smem}[\\{level}].\\{var}\K\|v,\39\\{smem}[\\{level}].\\{move}\K%
\\{nextmove};{}$\6
\&{if} ${}(\\{active}\E\|v){}$\1\5
${}\\{active}\K\T{0}{}$;\C{ the ring becomes empty }\2\6
\&{else}\1\5
${}\\{oo},\39\|h\K\\{vmem}[\|v].\\{next},\39\\{vmem}[\\{active}].\\{next}\K%
\|h{}$;\C{ delete \PB{\|v} from the ring }\2\6
\4\\{makemove}:\5
\X45:Set \PB{\|v} and update the watch lists for its new value\X;\6
\&{goto} \\{choose};\6
\4\\{backup}:\5
\X48:Backtrack to the most recent unforced move\X;\par
\U2.\fi

\M{41}\B\X41:Initialize the active ring\X${}\E{}$\6
\&{for} ${}(\\{active}\K\|j\K\T{0},\39\|k\K\\{vars};{}$ \|k; ${}\|k\MM){}$\1\6
\&{if} ${}((\|o,\39\\{vmem}[\|k].\\{wlist0})\V(\\{vmem}[\|k].\\{wlist1})){}$\5
${}\{{}$\1\6
\&{if} ${}(\\{active}\E\T{0}){}$\1\5
${}\\{active}\K\|k;{}$\2\6
${}\|o,\39\\{vmem}[\|k].\\{next}\K\|j,\39\|j\K\|k;{}$\6
\4${}\}{}$\2\2\6
\&{if} (\\{active})\1\5
${}\|o,\39\\{vmem}[\\{active}].\\{next}\K\|j{}$;\C{ complete the circle }\2\par
\U40.\fi

\M{42}The basic operation we need to do at each level is to decide which
variable in
the active ring should be set next. And experience with SAT problems shows
that, once we get going, there's usually at least one unset variable whose
value is forced by previous clauses.

A literal is forced to be true if and only if there's a clause in its watch
list
such that all other literals of that clause are already set to
false. Therefore we go through the watch list of every active variable until we
either find such a literal or discover that there is no forcing at the present
time.

When a forced literal is found, we'll want to resume searching for another one
at the same place where we left off, thus going cyclically through the active
ring. (For if we were to start searching again at the beginning of that list,
we'd be covering more or less the same ground as before.)

\Y\B\4\X42:Decide what to do next, going either to \PB{\\{branch}} or \PB{%
\\{forcedmove}} or \PB{\\{backup}} or \PB{\\{satisfied}}\X${}\E{}$\6
\&{if} ${}(\\{active}\E\T{0}){}$\1\5
\&{goto} \\{satisfied};\2\6
\&{if} ${}(\\{verbose}\AND\\{show\_details}){}$\5
${}\{{}$\1\6
${}\\{fprintf}(\\{stderr},\39\.{"\ active\ ring:"});{}$\6
\&{for} ${}(\|v\K\\{vmem}[\\{active}].\\{next};{}$  ; ${}\|v\K\\{vmem}[\|v].%
\\{next}){}$\5
${}\{{}$\1\6
${}\\{fprintf}(\\{stderr},\39\.{"\ "}\|O\.{".8s"},\39\\{vmem}[\|v].\\{name}.%
\\{ch8});{}$\6
\&{if} ${}(\|v\E\\{active}){}$\1\5
\&{break};\2\6
\4${}\}{}$\2\6
${}\\{fprintf}(\\{stderr},\39\.{"\\n"});{}$\6
\4${}\}{}$\2\6
${}\\{vv}\K\\{active},\39\\{vvv}\K\T{0};{}$\6
\4\\{newv}:\5
${}\|o,\39\|v\K\\{vmem}[\\{vv}].\\{next}{}$;\C{ during the search, \PB{\|v} is
one step ahead of \PB{\\{vv}} }\6
${}\\{force}\K\T{0};{}$\6
\X43:Set \PB{$\\{force}\K\T{1}$} if variable \PB{\|v} must be true\X;\6
\X44:Set \PB{$\\{force}\MRL{+{\K}}\T{2}$} if variable \PB{\|v} must be false\X;%
\6
\&{if} ${}(\\{force}\E\T{3}){}$\1\5
\&{goto} \\{backup};\2\6
\&{if} (\\{force})\5
${}\{{}$\1\6
${}\\{nextmove}\K\\{force}+\T{3};{}$\6
${}\\{active}\K\\{vv};{}$\6
\&{goto} \\{forcedmove};\6
\4${}\}{}$\2\6
\&{if} ${}(\\{vvv}\E\T{0}\W\|v\Z\\{primary\_vars}){}$\1\5
${}\\{vvv}\K\\{vv}{}$;\C{ \PB{\\{vvv}} precedes the first active primary
variable }\2\6
\&{if} ${}(\|v\E\\{active}){}$\5
${}\{{}$\1\6
\&{if} (\\{vvv})\1\5
${}\\{vv}\K\\{active}\K\\{vvv};{}$\2\6
${}\|v\K\\{vmem}[\\{active}].\\{next};{}$\6
\&{goto} \\{branch};\6
\4${}\}{}$\2\6
${}\\{vv}\K\|v{}$;\5
\&{goto} \\{newv};\par
\U40.\fi

\M{43}When literal \PB{\|l} is watched in clause \PB{\|c}, we know that \PB{%
\|l} is the first
literal of~\PB{\|c}. We scan through the other literals until either
reaching a literal that's currently unknown or true (whence nothing
is forced), or reaching the end (whence \PB{\|l} is forced).

If we encounter a true literal $l'$, we could swap it into first position,
thereby moving clause \PB{\|c} from the watch list of~\PB{\|l} to the watch
list
of $l'$, where it probably won't need to be examined as often.
But that's a complication that I will postpone for future study,
to be explored in variants of this program.

\Y\B\4\X43:Set \PB{$\\{force}\K\T{1}$} if variable \PB{\|v} must be true\X${}%
\E{}$\6
\&{for} ${}(\|o,\39\|c\K\\{vmem}[\|v].\\{wlist0};{}$ \|c; \|o${},\39\|c\K%
\\{cmem}[\|c].\\{wlink}){}$\5
${}\{{}$\1\6
\&{for} ${}(\\{oo},\39\|k\K\\{cmem}[\|c].\\{start}+\T{1};{}$ ${}\|k<\\{cmem}[%
\|c-\T{1}].\\{start};{}$ ${}\|k\PP){}$\1\6
\&{if} ${}(\\{oo},\39\\{val}[\\{mem}[\|k]\GG\T{1}]\I(\\{mem}[\|k]\AND\T{1})){}$%
\1\5
\&{goto} \\{unforced0};\2\2\6
\&{if} ${}(\\{verbose}\AND\\{show\_details}){}$\1\5
${}\\{fprintf}(\\{stderr},\39\.{"(Clause\ "}\|O\.{"d\ reduced\ to\ "}\|O%
\.{".8s)\\n"},\39\|c,\39\\{vmem}[\|v].\\{name}.\\{ch8});{}$\2\6
${}\\{force}\K\T{1};{}$\6
\&{goto} \\{forced0};\6
\4\\{unforced0}:\5
\&{continue};\6
\4${}\}{}$\2\6
\\{forced0}:\par
\U42.\fi

\M{44}\B\X44:Set \PB{$\\{force}\MRL{+{\K}}\T{2}$} if variable \PB{\|v} must be
false\X${}\E{}$\6
\&{for} ${}(\|o,\39\|c\K\\{vmem}[\|v].\\{wlist1};{}$ \|c; \|o${},\39\|c\K%
\\{cmem}[\|c].\\{wlink}){}$\5
${}\{{}$\1\6
\&{for} ${}(\\{oo},\39\|k\K\\{cmem}[\|c].\\{start}+\T{1};{}$ ${}\|k<\\{cmem}[%
\|c-\T{1}].\\{start};{}$ ${}\|k\PP){}$\1\6
\&{if} ${}(\\{oo},\39\\{val}[\\{mem}[\|k]\GG\T{1}]\I(\\{mem}[\|k]\AND\T{1})){}$%
\1\5
\&{goto} \\{unforced1};\2\2\6
\&{if} ${}(\\{verbose}\AND\\{show\_details}){}$\1\5
${}\\{fprintf}(\\{stderr},\39\.{"(Clause\ "}\|O\.{"d\ reduced\ to\ \~"}\|O%
\.{".8s)\\n"},\39\|c,\39\\{vmem}[\|v].\\{name}.\\{ch8});{}$\2\6
${}\\{force}\MRL{+{\K}}\T{2};{}$\6
\&{goto} \\{forced1};\6
\4\\{unforced1}:\5
\&{continue};\6
\4${}\}{}$\2\6
\\{forced1}:\par
\U42.\fi

\M{45}\B\X45:Set \PB{\|v} and update the watch lists for its new value\X${}%
\E{}$\6
\&{if} ${}((\\{verbose}\AND\\{show\_choices})\W\\{level}\Z\\{show\_choices%
\_max}){}$\5
${}\{{}$\1\6
${}\\{fprintf}(\\{stderr},\39\.{"Level\ "}\|O\.{"d,\ "},\39\\{level});{}$\6
\&{switch} (\\{nextmove})\5
${}\{{}$\1\6
\4\&{case} \T{0}:\5
${}\\{fprintf}(\\{stderr},\39\.{"trying\ "}\|O\.{".8s"},\39\\{vmem}[\|v].%
\\{name}.\\{ch8}){}$;\5
\&{break};\6
\4\&{case} \T{1}:\5
${}\\{fprintf}(\\{stderr},\39\.{"trying\ \~"}\|O\.{".8s"},\39\\{vmem}[\|v].%
\\{name}.\\{ch8}){}$;\5
\&{break};\6
\4\&{case} \T{2}:\5
${}\\{fprintf}(\\{stderr},\39\.{"retrying\ "}\|O\.{".8s"},\39\\{vmem}[\|v].%
\\{name}.\\{ch8}){}$;\5
\&{break};\6
\4\&{case} \T{3}:\5
${}\\{fprintf}(\\{stderr},\39\.{"retrying\ \~"}\|O\.{".8s"},\39\\{vmem}[\|v].%
\\{name}.\\{ch8}){}$;\5
\&{break};\6
\4\&{case} \T{4}:\5
${}\\{fprintf}(\\{stderr},\39\.{"forcing\ "}\|O\.{".8s"},\39\\{vmem}[\|v].%
\\{name}.\\{ch8}){}$;\5
\&{break};\6
\4\&{case} \T{5}:\5
${}\\{fprintf}(\\{stderr},\39\.{"forcing\ \~"}\|O\.{".8s"},\39\\{vmem}[\|v].%
\\{name}.\\{ch8}){}$;\5
\&{break};\6
\4${}\}{}$\2\6
${}\\{fprintf}(\\{stderr},\39\.{",\ "}\|O\.{"lld\ mems\\n"},\39\\{mems});{}$\6
\4${}\}{}$\2\6
\&{if} ${}(\\{nextmove}\AND\T{1}){}$\5
${}\{{}$\1\6
${}\|o,\39\\{val}[\|v]\K\\{false};{}$\6
${}\\{oo},\39\|c\K\\{vmem}[\|v].\\{wlist0},\39\\{vmem}[\|v].\\{wlist0}\K\T{0},%
\39\\{ll}\K\|v+\|v;{}$\6
\4${}\}{}$\5
\2\&{else}\5
${}\{{}$\1\6
${}\|o,\39\\{val}[\|v]\K\\{true};{}$\6
${}\\{oo},\39\|c\K\\{vmem}[\|v].\\{wlist1},\39\\{vmem}[\|v].\\{wlist1}\K\T{0},%
\39\\{ll}\K\|v+\|v+\T{1};{}$\6
\4${}\}{}$\2\6
\X46:Clear the watch list for \PB{\\{ll}} that starts at \PB{\|c}\X;\par
\U40.\fi

\M{46}\B\X46:Clear the watch list for \PB{\\{ll}} that starts at \PB{\|c}\X${}%
\E{}$\6
\&{for} ( ; \|c; ${}\|c\K\\{cc}){}$\5
${}\{{}$\1\6
${}\|o,\39\\{cc}\K\\{cmem}[\|c].\\{wlink};{}$\6
\&{for} ${}(\\{oo},\39\|j\K\\{cmem}[\|c].\\{start},\39\|k\K\|j+\T{1};{}$ ${}%
\|k<\\{cmem}[\|c-\T{1}].\\{start};{}$ ${}\|k\PP){}$\5
${}\{{}$\1\6
${}\|o,\39\|l\K\\{mem}[\|k];{}$\6
\&{if} ${}(\|o,\39\\{val}[\|l\GG\T{1}]\I(\|l\AND\T{1})){}$\1\5
\&{break};\2\6
\4${}\}{}$\2\6
\&{if} ${}(\|k\E\\{cmem}[\|c-\T{1}].\\{start}){}$\5
${}\{{}$\1\6
${}\\{fprintf}(\\{stderr},\39\.{"Clause\ "}\|O\.{"d\ can't\ be\ watched!}\)\.{%
\\n"},\39\|c){}$;\C{ ``can't happen'' }\6
${}\\{exit}({-}\T{18});{}$\6
\4${}\}{}$\2\6
${}\\{oo},\39\\{mem}[\|k]\K\\{ll},\39\\{mem}[\|j]\K\|l;{}$\6
\X47:Put \PB{\|c} into the watch list of \PB{\|l}\X;\6
\4${}\}{}$\2\par
\U45.\fi

\M{47}The variable corresponding to \PB{\|l} might become active at this
point, because it might not be watched anywhere else. In such a
case we insert it at the ``beginning'' of the active ring (that
is, just after \PB{\\{active}}). We always have \PB{$\\{vmem}[\\{active}].%
\\{next}\K\|h$}
at this point, unless \PB{$\\{active}\K\T{0}$}.

\Y\B\4\X47:Put \PB{\|c} into the watch list of \PB{\|l}\X${}\E{}$\6
\&{if} ${}(\\{verbose}\AND\\{show\_details}){}$\1\5
${}\\{fprintf}(\\{stderr},\39\.{"(Clause\ "}\|O\.{"d\ now\ watches\ "}\|O%
\.{"s"}\|O\.{".8s)\\n"},\39\|c,\39\|l\AND\T{1}\?\.{"\~"}:\.{""},\39\\{vmem}[\|l%
\GG\T{1}].\\{name}.\\{ch8});{}$\2\6
${}\|o,\39\|p\K\\{vmem}[\|l\GG\T{1}].\\{wlist0},\39\|q\K\\{vmem}[\|l\GG\T{1}].%
\\{wlist1};{}$\6
\&{if} ${}(\\{val}[\|l\GG\T{1}]\E\\{unknown}\W\|p\E\T{0}\W\|q\E\T{0}){}$\5
${}\{{}$\1\6
\&{if} ${}(\\{active}\E\T{0}){}$\1\5
${}\|o,\39\\{active}\K\|h\K\|l\GG\T{1},\39\\{vmem}[\\{active}].\\{next}\K%
\|h;{}$\2\6
\&{else}\1\5
${}\\{oo},\39\\{vmem}[\|l\GG\T{1}].\\{next}\K\|h,\39\|h\K\|l\GG\T{1},\39%
\\{vmem}[\\{active}].\\{next}\K\|h;{}$\2\6
\4${}\}{}$\2\6
\&{if} ${}(\|l\AND\T{1}){}$\1\5
${}\\{oo},\39\\{cmem}[\|c].\\{wlink}\K\|q,\39\\{vmem}[\|l\GG\T{1}].\\{wlist1}\K%
\|c;{}$\2\6
\&{else}\1\5
${}\\{oo},\39\\{cmem}[\|c].\\{wlink}\K\|p,\39\\{vmem}[\|l\GG\T{1}].\\{wlist0}\K%
\|c{}$;\2\par
\U46.\fi

\M{48}If variables need to be reactivated here, we put them just
before the place where a conflict was found.

\Y\B\4\X48:Backtrack to the most recent unforced move\X${}\E{}$\6
$\\{active}\K\\{vv},\39\|h\K\|v;{}$\6
\&{while} ${}(\|o,\39\\{smem}[\\{level}].\\{move}\G\T{2}){}$\5
${}\{{}$\1\6
${}\|v\K\\{smem}[\\{level}].\\{var};{}$\6
${}\|o,\39\\{val}[\|v]\K\\{unknown};{}$\6
\&{if} ${}((\|o,\39\\{vmem}[\|v].\\{wlist0}\I\T{0})\V(\\{vmem}[\|v].\\{wlist1}%
\I\T{0})){}$\1\5
${}\\{oo},\39\\{vmem}[\|v].\\{next}\K\|h,\39\|h\K\|v,\39\\{vmem}[\\{active}].%
\\{next}\K\|h;{}$\2\6
${}\\{level}\MM;{}$\6
\4${}\}{}$\2\6
\&{if} (\\{level})\5
${}\{{}$\1\6
${}\\{nextmove}\K\T{3}-\\{smem}[\\{level}].\\{move};{}$\6
${}\\{oo},\39\|v\K\\{smem}[\\{level}].\\{var},\39\\{smem}[\\{level}].\\{move}\K%
\\{nextmove};{}$\6
\&{goto} \\{makemove};\6
\4${}\}{}$\2\6
\&{if} (\T{1})\5
${}\{{}$\1\6
\\{printf}(\.{"\~\\n"});\C{ the formula was unsatisfiable }\6
\&{if} ${}(\\{verbose}\AND\\{show\_basics}){}$\1\5
${}\\{fprintf}(\\{stderr},\39\.{"UNSAT\\n"});{}$\2\6
\4${}\}{}$\5
\2\&{else}\5
${}\{{}$\1\6
\4\\{satisfied}:\5
\&{if} ${}(\\{verbose}\AND\\{show\_basics}){}$\1\5
${}\\{fprintf}(\\{stderr},\39\.{"!SAT!\\n"});{}$\2\6
\X49:Print the solution found\X;\6
\4${}\}{}$\2\par
\U40.\fi

\M{49}\B\X49:Print the solution found\X${}\E{}$\6
\&{for} ${}(\|k\K\T{1};{}$ ${}\|k\Z\\{level};{}$ ${}\|k\PP){}$\5
${}\{{}$\1\6
${}\|l\K(\\{smem}[\|k].\\{var}\LL\T{1})+(\\{smem}[\|k].\\{move}\AND\T{1});{}$\6
${}\\{printf}(\.{"\ "}\|O\.{"s"}\|O\.{".8s"},\39\|l\AND\T{1}\?\.{"\~"}:\.{""},%
\39\\{vmem}[\|l\GG\T{1}].\\{name}.\\{ch8});{}$\6
\&{if} (\\{out\_file})\1\5
${}\\{fprintf}(\\{out\_file},\39\.{"\ "}\|O\.{"s"}\|O\.{".8s"},\39\|l\AND\T{1}%
\?\.{""}:\.{"\~"},\39\\{vmem}[\|l\GG\T{1}].\\{name}.\\{ch8});{}$\2\6
\4${}\}{}$\2\6
\\{printf}(\.{"\\n"});\6
\&{if} ${}(\\{level}<\\{vars}){}$\5
${}\{{}$\1\6
\&{if} ${}(\\{verbose}\AND\\{show\_unused\_vars}){}$\1\5
\\{printf}(\.{"(Unused:"});\2\6
\&{for} ${}(\|v\K\T{0};{}$ ${}\|v<\\{vars};{}$ ${}\|v\PP){}$\1\6
\&{if} ${}(\\{val}[\|v]\E\\{unknown}){}$\5
${}\{{}$\1\6
\&{if} ${}(\\{verbose}\AND\\{show\_unused\_vars}){}$\1\5
${}\\{printf}(\.{"\ "}\|O\.{".8s"},\39\\{vmem}[\|v].\\{name}.\\{ch8});{}$\2\6
\&{if} (\\{out\_file})\1\5
${}\\{fprintf}(\\{out\_file},\39\.{"\ "}\|O\.{".8s"},\39\\{vmem}[\|v].\\{name}.%
\\{ch8});{}$\2\6
\4${}\}{}$\2\2\6
\&{if} ${}(\\{verbose}\AND\\{show\_unused\_vars}){}$\1\5
\\{printf}(\.{")\\n"});\2\6
\4${}\}{}$\2\6
\&{if} (\\{out\_file})\1\5
${}\\{fprintf}(\\{out\_file},\39\.{"\\n"}){}$;\2\par
\U48.\fi

\N{1}{50}Index.
\fi

\inx
\fin
\con
