\input cwebmac
\datethis
\def\SET/{{\mc SET\null}}


\N{1}{1}Introduction. This program finds all nonisomorphic sets of \SET/ cards
that contain no \SET/s.

In case you don't know what that means, a \SET/ card is a vector
$(c_1,c_2,c_3,c_4)$ where each $c_i$ is 0, 1, or~2. Thus there are 81
possible \SET/ cards. A~\SET/ is a set of three \SET/ cards that sums
to $(0,0,0,0)$ modulo~3. Equivalently, the numbers in each coordinate
position of the three vectors in a \SET/ are either all the same or all
different. (It's kind of a 4-dimensional tic-tac-toe with wraparound.)

My previous {\sc SETSET} program considered only isomorphisms that
apply directly to the semantics of the game of \SET/, namely
permutations of coordinates and permutations of individual coordinate
positions; there are $4!\times 3!^4=31104$ of those. But now I want
to consider the much larger collection of all isomorphisms that
preserve \SET/s. There are $81\cdot(81-1)\cdot(81-3)\cdot(81-9)\cdot(81-27)
=1{,}965{,}150{,}720$ of these; thus the method used in the
previous program, which had complexity $\Omega(\hbox{number of isomorphisms})$
in both time and space, would be quite inappropriate. Here I'm using
a possibly new method, with space requirement only
$O(\hbox{number of elements})^2D$, where $D$ bounds the partial
transitivity of the isomorphism group: The image of the first $D$ elements
is sufficient to determine the images of all. In our case, $D=5$.

A web page of David Van Brink states that you can't have more than 20 \SET/
cards without having a \SET/. He says that he proved this in 1997 with a
computer program that took about one week to run on a 90MHz Pentium machine.
I'm hoping to get the result faster by using ideas of isomorph rejection,
meanwhile also discovering all of the $k$-element \SET/-less solutions
for $k\le20$.

The theorem about at most 20 \SET/-free cards was actually proved in much
stronger form by G. Pellegrino, {\sl Matematiche\/ \bf25} (1971), 149--157,
without using computers. Pellegrino showed that any set of 21 points in
the projective space of $81+27+9+3+1$ elements, represented by nonzero
5-tuples in which $x$ and $-x$ are considered equivalent, has three
collinear points; this would correspond to sets of three distinct points
in which the third is the sum or difference of the first two.

Incidentally, I've written this program for my own instruction, not for
publication. I~still haven't had time to read the highly relevant
papers by Adalbert Kerber, Reinhard Laue, and their colleagues at Bayreuth,
^{Kerber, Adalbert}
^{Laue, Reinhard}
although I've
had those works in my files for many years. Members of that group
probably are quite familiar with equivalent or better methods.
Perhaps I'm being foolish, but I thought it would be most educational to try
my own hand before looking at other people's solutions. I seem to learn a new
subject best when I try to write code for it, because the computer is
such a demanding, unbluffable taskmaster.

[\SET/ is a registered trademark of SET Enterprises, Inc.]

\Y\B\8\#\&{include} \.{<stdio.h>}\6
\8\#\&{include} \.{<stdlib.h>}\6
\8\#\&{include} \.{<setjmp.h>}\6
\&{jmp\_buf} \\{restart\_point};\7
\X5:Type definitions\X\6
\X6:Global variables\X\6
\X22:Subroutines\X\7
\\{main}(\,)\1\1\2\2\6
${}\{{}$\1\6
\X40:Local variables\X\6
\X8:Initialize\X;\6
\X39:Enumerate and print all solutions\X;\6
\X47:Print the totals\X;\6
\4${}\}{}$\2\par
\fi

\M{2}Our basic approach is to define a linear ordering on solutions, and to
look only for solutions that are smallest in their isomorphism class.
In other words, we will count the sets $S$ such that $S\le\alpha S$ for
all automorphisms~$\alpha$. We'll also count the number $t$ of cases where
$S=\alpha S$; then the number of distinct solutions isomorphic to~$S$
is $1965150720/t$, so we will essentially have also enumerated the distinct
solutions.

The ordering we use is almost standard: Vectors are ordered
by weight, and vectors of equal weight are ordered lexicographically;
thus the sequence is $(0,0,0,0)$, $(0,0,0,1)$, $(0,0,1,0)$, $(0,1,0,0)$,
$(1,0,0,0)$, $(0,0,0,2)$, $(0,0,1,1)$, \dots, $(2,2,2,2)$.
Also, when $S$ and $T$ are both sets of $k$ \SET/ cards, we define
$S\le T$ by first sorting the vectors into order so that $s_1<\cdots<s_k$ and
$t_1<\cdots<t_k$, then we compare $(s_1,\ldots,s_k)$ lexicographically
to $(t_1,\ldots,t_k)$. (Equivalently, we compare the smallest elements
of $S$ and~$T$; if they are equal, we compare the second-smallest elements,
and so on, until we've either found inequality or established that $S=T$.)

\fi

\M{3}The automorphisms can be thought of as the collection of all linear
mappings that take $x\mapsto Ax+b$, where $x$ is the column vector
$(c_1,c_2,c_3,c_4)^T$, $A$ is a nonsingular $4\times4$
matrix (mod~3), and $b$ is an arbitrary vector. Alternatively we can
think of the mapping $x\mapsto Ax$, where each \SET/ card~$x$ is a
column vector of the form $(c_1,c_2,c_3,c_4,1)^T$
and $A$ is a nonsingular $5\times5$ matrix whose bottom row is $(0,0,0,0,1)$.
In either case
we have the so-called ``affine linear group'' in 4-space (mod~3).

For example, any set of five points that are {\it independent}, in the
sense that none is in the subspace spanned by the other four, can be
mapped into the smallest five cards
$$\{(0,0,0,0),\;(0,0,0,1),\;(0,0,1,0),\;(0,1,0,0),(1,0,0,0)\}.$$
The set consisting of the smallest four cards has $24\cdot54=1296$
automorphisms, hence there are exactly $1965150720/1296=1516320$ ways to
choose four \SET/ cards that are independent in this sense.
All other \SET/-less sets of four cards are isomorphic to the dependent set
$$\{(0,0,0,0),\;(0,0,0,1),\;(0,0,1,0),\;(0,0,1,1)\};$$
and this set has 31104 automorphisms, leading to an additional
$1965150720/31104=63180$ ways to choose a set of four \SET/-less cards.
This explains why the total number of such sets is $1516320+63180=1579500$,
a number that the {\mc SETSET} program was able to compute only after
it had laboriously considered 128 cases that were nonisomorphic
in the previous setup.

\fi

\M{4}We will generate the elements of a $k$-set in order. If we have
$s_1<\cdots<s_k$ and $\{s_1,\ldots,s_k\}\le\{\alpha s_1,\ldots,\alpha s_k\}$
for all $\alpha$, it is not hard to prove that $\{s_1,\ldots,s_j\}\le\{\alpha
s_1,\ldots,\alpha s_j\}$ for all $\alpha$ and $1\le j\le k$.
(The reason is that $S<T$ and $t\ge\max T$ implies
$S\cup\{s\}<S\cup\{\infty\}<T\cup\{t\}$, for all $s$.)
Therefore every canonical $k$-set is obtained by extending a unique
canonical $(k-1)$-set.

\fi

\N{1}{5}Permutations.
It's convenient to represent \SET/ card vectors in a compact code,
as an integer between 0 and 80.

\Y\B\4\X5:Type definitions\X${}\E{}$\6
\&{typedef} \&{char} \&{SETcard};\C{ a \SET/ card $(c_1,c_2,c_3,c_4)$
represented as $\PB{\\{encode}}[(c_1 c_2 c_3 c_4)_3]$ }\par
\A20.
\U1.\fi

\M{6} Lexicographic order would correspond
to ternary notation, but our weight-first ordering is slightly different.
Here we specify the card ranks in the desired order.

\Y\B\4\D$\\{nn}$ \5
\T{81}\C{ the total number of elements permuted by automorphisms }\par
\B\4\D$\\{nnn}$ \5
\T{128}\C{ the value of \PB{\\{nn}} rounded up to a power of 2, for efficiency
}\par
\Y\B\4\X6:Global variables\X${}\E{}$\6
\&{SETcard} \\{encode}[\\{nn}]${}\K\{\T{0},\39\T{1},\39\T{5},\39\T{2},\39\T{6},%
\39\T{15},\39\T{7},\39\T{16},\39\T{31},\3{-1}\39\T{3},\39\T{8},\39\T{17},\39%
\T{9},\39\T{18},\39\T{32},\39\T{19},\39\T{33},\39\T{50},\3{-1}\39\T{10},\39%
\T{20},\39\T{34},\39\T{21},\39\T{35},\39\T{51},\39\T{36},\39\T{52},\39\T{66},%
\3{-1}\39\T{4},\39\T{11},\39\T{22},\39\T{12},\39\T{23},\39\T{37},\39\T{24},\39%
\T{38},\39\T{53},\3{-1}\39\T{13},\39\T{25},\39\T{39},\39\T{26},\39\T{40},\39%
\T{54},\39\T{41},\39\T{55},\39\T{67},\3{-1}\39\T{27},\39\T{42},\39\T{56},\39%
\T{43},\39\T{57},\39\T{68},\39\T{58},\39\T{69},\39\T{76},\3{-1}\39\T{14},\39%
\T{28},\39\T{44},\39\T{29},\39\T{45},\39\T{59},\39\T{46},\39\T{60},\39\T{70},%
\3{-1}\39\T{30},\39\T{47},\39\T{61},\39\T{48},\39\T{62},\39\T{71},\39\T{63},\39%
\T{72},\39\T{77},\3{-1}\39\T{49},\39\T{64},\39\T{73},\39\T{65},\39\T{74},\39%
\T{78},\39\T{75},\39\T{79},\39\T{80}\}{}$;\par
\As7, 9, 11, 15, 16, 21, 23, 28, 41\ETs46.
\U1.\fi

\M{7}When we output a \SET/ card, however, we prefer a decimal code.

\Y\B\4\X6:Global variables\X${}\mathrel+\E{}$\6
\&{int} \\{decimalform}[\\{nn}]${}\K\{\T{0},\39\T{1},\39\T{2},\39\T{10},\39%
\T{11},\39\T{12},\39\T{20},\39\T{21},\39\T{22},\3{-1}\39\T{100},\39\T{101},\39%
\T{102},\39\T{110},\39\T{111},\39\T{112},\39\T{120},\39\T{121},\39\T{122},%
\3{-1}\39\T{200},\39\T{201},\39\T{202},\39\T{210},\39\T{211},\39\T{212},\39%
\T{220},\39\T{221},\39\T{222},\3{-1}\39\T{1000},\39\T{1001},\39\T{1002},\39%
\T{1010},\39\T{1011},\39\T{1012},\39\T{1020},\39\T{1021},\39\T{1022},\3{-1}\39%
\T{1100},\39\T{1101},\39\T{1102},\39\T{1110},\39\T{1111},\39\T{1112},\39%
\T{1120},\39\T{1121},\39\T{1122},\3{-1}\39\T{1200},\39\T{1201},\39\T{1202},\39%
\T{1210},\39\T{1211},\39\T{1212},\39\T{1220},\39\T{1221},\39\T{1222},\3{-1}\39%
\T{2000},\39\T{2001},\39\T{2002},\39\T{2010},\39\T{2011},\39\T{2012},\39%
\T{2020},\39\T{2021},\39\T{2022},\3{-1}\39\T{2100},\39\T{2101},\39\T{2102},\39%
\T{2110},\39\T{2111},\39\T{2112},\39\T{2120},\39\T{2121},\39\T{2122},\3{-1}\39%
\T{2200},\39\T{2201},\39\T{2202},\39\T{2210},\39\T{2211},\39\T{2212},\39%
\T{2220},\39\T{2221},\39\T{2222}\};{}$\6
\&{int} \\{decode}[\\{nn}];\par
\fi

\M{8}\B\X8:Initialize\X${}\E{}$\6
\&{for} ${}(\|k\K\T{0};{}$ ${}\|k<\\{nn};{}$ ${}\|k\PP){}$\1\5
${}\\{decode}[\\{encode}[\|k]]\K\\{decimalform}[\|k]{}$;\2\par
\As10, 12\ETs17.
\U1.\fi

\M{9}We will frequently need to find the third card of a \SET/,
given any two distinct cards $x$ and $y$, so we store the answers
in a precomputed table.

\Y\B\4\X6:Global variables\X${}\mathrel+\E{}$\6
\&{char} \|z[\T{3}][\T{3}]${}\K\{\{\T{0},\39\T{2},\39\T{1}\},\39\{\T{2},\39%
\T{1},\39\T{0}\},\39\{\T{1},\39\T{0},\39\T{2}\}\}{}$;\C{ $x+y+z\equiv0$ (mod 3)
}\6
\&{char} \\{third}[\\{nn}][\\{nnn}];\par
\fi

\M{10}\B\D$\\{pack}(\|a,\|b,\|c,\|d)$ \5
$\\{encode}[(((\|a)*\T{3}+(\|b))*\T{3}+(\|c))*\T{3}+(\|d){}$]\par
\Y\B\4\X8:Initialize\X${}\mathrel+\E{}$\6
${}\{{}$\1\6
\&{int} \|a${},{}$ \|b${},{}$ \|c${},{}$ \|d${},{}$ \|e${},{}$ \|f${},{}$ %
\|g${},{}$ \|h;\7
\&{for} ${}(\|a\K\T{0};{}$ ${}\|a<\T{3};{}$ ${}\|a\PP){}$\1\6
\&{for} ${}(\|b\K\T{0};{}$ ${}\|b<\T{3};{}$ ${}\|b\PP){}$\1\6
\&{for} ${}(\|c\K\T{0};{}$ ${}\|c<\T{3};{}$ ${}\|c\PP){}$\1\6
\&{for} ${}(\|d\K\T{0};{}$ ${}\|d<\T{3};{}$ ${}\|d\PP){}$\1\6
\&{for} ${}(\|e\K\T{0};{}$ ${}\|e<\T{3};{}$ ${}\|e\PP){}$\1\6
\&{for} ${}(\|f\K\T{0};{}$ ${}\|f<\T{3};{}$ ${}\|f\PP){}$\1\6
\&{for} ${}(\|g\K\T{0};{}$ ${}\|g<\T{3};{}$ ${}\|g\PP){}$\1\6
\&{for} ${}(\|h\K\T{0};{}$ ${}\|h<\T{3};{}$ ${}\|h\PP){}$\1\5
${}\\{third}[\\{pack}(\|a,\39\|b,\39\|c,\39\|d)][\\{pack}(\|e,\39\|f,\39\|g,\39%
\|h)]\K\\{pack}(\|z[\|a][\|e],\39\|z[\|b][\|f],\39\|z[\|c][\|g],\39\|z[\|d][%
\|h]);{}$\2\2\2\2\2\2\2\2\6
\4${}\}{}$\2\par
\fi

\M{11}The set of automorphisms is conveniently represented by a mapping
table, as in the author's paper ``Efficient representation of
perm groups,'' {\sl Combinatorica\/ \bf11} (1991), 33--43.
If there is an $\alpha$ such that $\alpha k=j$ and $\alpha$ fixes
all elements~$<j$, we let \PB{\\{perm}[\|j][\|k]} be one such permutation~$%
\alpha$,
represented as an array of 81 elements. In particular, \PB{\\{perm}[\|j][\|j]}
always is
the identity permutation. If no such $\alpha$ exists, however, we set
\PB{$\\{perm}[\|j][\|k][\T{0}]\K{-}\T{1}$}.

Our algorithm for finding nonisomorphic \SET/-less sets is based entirely
on the group of isomorphisms defined by a \PB{\\{perm}} table. If \PB{\\{perm}}
is
initialized to the definition of some other group, the rest of this program
should need no further modification except for counting the
total number of automorphisms. (Of course, not every \PB{\\{perm}} table
defines a
group of permutations; the set of all possible products
$\pi_0\pi_1\ldots\pi_{80}$, where each $\pi_j$ is one of the permutations
$\PB{\\{perm}[\|j][\|k]}$ for some $k\ge j$, must be closed under
multiplication.
We assume that this condition is satisfied.)

There is an integer~$D$ such that \PB{$\\{perm}[\|j][\|k]\K{-}\T{1}$} for
all $D\le j<k$; in other words, each $\alpha$ that fixes 0,~1, \dots, $D-1$
is the identity map. We needn't bother representing \PB{\\{perm}[\|j][\|k]} for
$j\ge D$.

{\sl Important Note [30 April 2001]:\/} No, the algorithm does {\it not\/}
work for arbitrary permutation groups. I thank Prof.\ Reinhard Laue for
^{Laue, Reinhard}
pointing out a serious error. However, I do think the special group dealt
with here is handled satisfactorily because of its highly transitive nature.

\Y\B\4\D$\\{dd}$ \5
\T{5}\C{ in our case $D=5$ }\par
\Y\B\4\X6:Global variables\X${}\mathrel+\E{}$\6
\&{char} \\{perm}[\\{dd}][\\{nnn}][\\{nnn}];\C{ mapping table }\par
\fi

\M{12}Now let's set up the mapping table for the affine transformations we
need.
The basic idea is simple. For example, the group of all $5\times 5$ matrices
that fix $(0,0,0,0)=0$ and $(0,0,0,1)=1$ is the set of all nonsingular~$A$
of the form
$$\pmatrix{\ast&\ast&\ast&0&0\cr
\ast&\ast&\ast&0&0\cr
\ast&\ast&\ast&0&0\cr
\ast&\ast&\ast&1&0\cr
0 &  0 &  0 &0&1\cr}\,;$$
and the image of $(0,0,1,0)$ is the third column. For every possible third
column~$k$ (which must not be zero in the top three rows, lest the matrix be
singular), we need to choose an appropriate setting of the first two columns.
Then we get an affine mapping that takes $(0,0,1,0)=2$ into $k$, and
the inverse of this mapping can be stored in \PB{\\{perm}[\T{2}][\|k]} because
it
maps $k\mapsto2$.

\Y\B\4\X8:Initialize\X${}\mathrel+\E{}$\6
$\\{aa}[\T{4}][\T{4}]\K\T{1};{}$\6
\&{for} ${}(\|j\K\T{0};{}$ ${}\|j<\T{5};{}$ ${}\|j\PP){}$\5
${}\{{}$\C{ we want to set up \PB{$\\{perm}[\T{4}-\|j]$} }\1\6
\&{int} \|a${},{}$ \|b${},{}$ \|c${},{}$ \|d${},{}$ \|e${},{}$ \|f${},{}$ %
\|g${},{}$ \|h${},{}$ \\{jj}${},{}$ \\{kk};\7
\&{for} ${}(\\{jj}\K\|j+\T{1};{}$ ${}\\{jj}<\T{5};{}$ ${}\\{jj}\PP){}$\1\6
\&{for} ${}(\|k\K\T{0};{}$ ${}\|k<\T{4};{}$ ${}\|k\PP){}$\1\5
${}\\{aa}[\\{jj}][\|k]\K(\|k\E\\{jj}\?\T{1}:\T{0});{}$\2\2\6
\&{for} ${}(\|a\K\T{0};{}$ ${}\|a<\T{3};{}$ ${}\|a\PP){}$\1\6
\&{for} ${}(\|b\K\T{0};{}$ ${}\|b<\T{3};{}$ ${}\|b\PP){}$\1\6
\&{for} ${}(\|c\K\T{0};{}$ ${}\|c<\T{3};{}$ ${}\|c\PP){}$\1\6
\&{for} ${}(\|d\K\T{0};{}$ ${}\|d<\T{3};{}$ ${}\|d\PP){}$\5
${}\{{}$\1\6
${}\\{aa}[\|j][\T{0}]\K\|a,\39\\{aa}[\|j][\T{1}]\K\|b,\39\\{aa}[\|j][\T{2}]\K%
\|c,\39\\{aa}[\|j][\T{3}]\K\|d;{}$\6
\&{for} ${}(\\{kk}\K\|j;{}$ ${}\\{kk}\G\T{0};{}$ ${}\\{kk}\MM){}$\1\6
\&{if} (\\{aa}[\|j][\\{kk}])\1\5
\&{break};\2\2\6
\&{if} ${}(\\{kk}<\T{0}){}$\1\5
${}\\{perm}[\T{4}-\|j][\\{pack}(\|a,\39\|b,\39\|c,\39\|d)][\T{0}]\K{-}\T{1};{}$%
\2\6
\&{else}\5
${}\{{}$\1\6
\X13:Complete \PB{\\{aa}} to a nonsingular matrix\X;\6
\X14:Use \PB{\\{aa}} to define the mapping \PB{$\\{perm}[\T{4}-\|j][\\{pack}(%
\|a,\|b,\|c,\|d)]$}\X;\6
\4${}\}{}$\2\6
\4${}\}{}$\2\2\2\2\6
\4${}\}{}$\2\par
\fi

\M{13}\B\X13:Complete \PB{\\{aa}} to a nonsingular matrix\X${}\E{}$\6
\&{for} ${}(\\{jj}\K\T{0};{}$ ${}\\{jj}<\|j;{}$ ${}\\{jj}\PP){}$\5
${}\{{}$\1\6
\&{for} ${}(\|k\K\T{0};{}$ ${}\|k<\T{4};{}$ ${}\|k\PP){}$\1\5
${}\\{aa}[\\{jj}][\|k]\K\T{0};{}$\2\6
${}\\{aa}[\\{jj}][(\\{jj}+\\{kk}+\T{1})\MOD(\|j+\T{1})]\K\T{1};{}$\6
\4${}\}{}$\2\par
\U12.\fi

\M{14}\B\X14:Use \PB{\\{aa}} to define the mapping \PB{$\\{perm}[\T{4}-\|j][%
\\{pack}(\|a,\|b,\|c,\|d)]$}\X${}\E{}$\6
$\\{kk}\K\\{pack}(\|a,\39\|b,\39\|c,\39\|d);{}$\6
\&{for} ${}(\|e\K\T{0};{}$ ${}\|e<\T{3};{}$ ${}\|e\PP){}$\1\6
\&{for} ${}(\|f\K\T{0};{}$ ${}\|f<\T{3};{}$ ${}\|f\PP){}$\1\6
\&{for} ${}(\|g\K\T{0};{}$ ${}\|g<\T{3};{}$ ${}\|g\PP){}$\1\6
\&{for} ${}(\|h\K\T{0};{}$ ${}\|h<\T{3};{}$ ${}\|h\PP){}$\5
${}\{{}$\1\6
\&{for} ${}(\|k\K\T{0};{}$ ${}\|k<\T{4};{}$ ${}\|k\PP){}$\1\5
${}\\{trit}[\|k]\K(\|e*\\{aa}[\T{0}][\|k]+\|f*\\{aa}[\T{1}][\|k]+\|g*\\{aa}[%
\T{2}][\|k]+\|h*\\{aa}[\T{3}][\|k]+\\{aa}[\T{4}][\|k])\MOD\T{3};{}$\2\6
${}\\{perm}[\T{4}-\|j][\\{kk}][\\{pack}(\\{trit}[\T{0}],\39\\{trit}[\T{1}],\39%
\\{trit}[\T{2}],\39\\{trit}[\T{3}])]\K\\{pack}(\|e,\39\|f,\39\|g,\39\|h);{}$\6
\4${}\}{}$\2\2\2\2\par
\U12.\fi

\M{15}\B\X6:Global variables\X${}\mathrel+\E{}$\6
\&{char} \\{trit}[\T{4}];\C{ four ternary digits }\6
\&{char} \\{aa}[\T{5}][\T{8}];\C{ matrix $A$, stored columnwise }\par
\fi

\M{16}The algorithm below also needs another table of permutations: For $j<D$
and
$k\ge j$, we let \PB{\\{minp}[\|j][\|k]} be a permutation that takes $k$ into
the
smallest possible value, among all permutations that fix all elements
less than~$j$.

\Y\B\4\X6:Global variables\X${}\mathrel+\E{}$\6
\&{char} ${}\\{minp}[\\{dd}][\\{nnn}][\\{nnn}+\\{nnn}]{}$;\par
\fi

\M{17}The \PB{\\{minp}} table is readily built from the \PB{\\{perm}} table,
working from
bottom to top. (In the affine linear group under the ordering we have chosen,
\PB{\\{minp}} actually is the same as \PB{\\{perm}} if we plug the identity in
place
of empty permutations. But this code tries to be general.)

We store the inverse permutation too: If \PB{$\|p\K\\{minp}[\|j][\|k]$}, then
\PB{\|p} maps $v$ to $p[v]$ and $p[v+\PB{\\{nnn}}]$ to $v$.

\Y\B\4\X8:Initialize\X${}\mathrel+\E{}$\6
\&{for} ${}(\|j\K\\{dd}-\T{1};{}$ ${}\|j\G\T{0};{}$ ${}\|j\MM){}$\1\6
\&{for} ${}(\|k\K\|j;{}$ ${}\|k<\\{nn};{}$ ${}\|k\PP){}$\5
${}\{{}$\1\6
\&{if} ${}(\\{perm}[\|j][\|k][\T{0}]\I{-}\T{1}){}$\5
${}\{{}$\C{ the best we can do is obviously $k\mapsto j$ }\1\6
\&{for} ${}(\|l\K\T{0};{}$ ${}\|l<\\{nn};{}$ ${}\|l\PP){}$\1\5
${}\\{minp}[\|j][\|k][\|l]\K\\{perm}[\|j][\|k][\|l];{}$\2\6
\4${}\}{}$\5
\2\&{else} \&{if} ${}(\|j\E\\{dd}-\T{1}){}$\5
${}\{{}$\1\6
\&{for} ${}(\|l\K\T{0};{}$ ${}\|l<\\{nn};{}$ ${}\|l\PP){}$\1\5
${}\\{minp}[\|j][\|k][\|l]\K\|l;{}$\2\6
\&{for} ${}(\|i\K\|j+\T{1};{}$ ${}\|i<\\{nn};{}$ ${}\|i\PP){}$\1\6
\&{if} ${}(\\{perm}[\|j][\|i][\T{0}]\I{-}\T{1}\W\\{perm}[\|j][\|i][\|k]<%
\\{minp}[\|j][\|k][\|k]){}$\1\6
\&{for} ${}(\|l\K\T{0};{}$ ${}\|l<\\{nn};{}$ ${}\|l\PP){}$\1\5
${}\\{minp}[\|j][\|k][\|l]\K\\{perm}[\|j][\|i][\|l];{}$\2\2\2\6
\4${}\}{}$\5
\2\&{else}\5
${}\{{}$\1\6
\&{register} \&{int} \\{kk};\7
\&{for} ${}(\|l\K\T{0};{}$ ${}\|l<\\{nn};{}$ ${}\|l\PP){}$\1\5
${}\\{minp}[\|j][\|k][\|l]\K\\{minp}[\|j+\T{1}][\|k][\|l];{}$\2\6
\&{for} ${}(\|i\K\|j+\T{1};{}$ ${}\|i<\\{nn};{}$ ${}\|i\PP){}$\1\6
\&{if} ${}(\\{perm}[\|j][\|i][\T{0}]\I{-}\T{1}){}$\5
${}\{{}$\1\6
${}\\{kk}\K\\{perm}[\|j][\|i][\|k];{}$\6
\&{if} ${}(\\{minp}[\|j+\T{1}][\\{kk}][\\{kk}]<\\{minp}[\|j][\|k][\|k]){}$\1\6
\&{for} ${}(\|l\K\T{0};{}$ ${}\|l<\\{nn};{}$ ${}\|l\PP){}$\1\5
${}\\{minp}[\|j][\|k][\|l]\K\\{minp}[\|j+\T{1}][\\{kk}][\\{perm}[\|j][\|i][%
\|l]];{}$\2\2\6
\4${}\}{}$\2\2\6
\4${}\}{}$\2\6
\&{for} ${}(\|l\K\T{0};{}$ ${}\|l<\\{nn};{}$ ${}\|l\PP){}$\1\5
${}\\{minp}[\|j][\|k][\\{minp}[\|j][\|k][\|l]+\\{nnn}]\K\|l;{}$\2\6
\4${}\}{}$\2\2\par
\fi

\M{18}The mapping table of a permutation group has many magical properties.
For example, consider the digraph with arcs from $(j,x)$ to $(j+1,y)$
whenever there is a $k$ with \PB{\\{perm}[\|j][\|k]} mapping $x$ to $y$.
This digraph has a path from $(j,x)$ to $(D,y)$ if and only if
the group has a permutation that maps $x$ to $y$ and fixes all elements
less than~$j$; hence such a path exists if and only if there is also a path
from $(j,y)$ to $(D,x)$.

Furthermore, if we let $S$ be the set of all elements $k$ such that
\PB{\\{perm}[\T{0}][\|k][\T{0}]}
is $-1$, namely the set of all elements that are not permutable into~0
(not in the ``orbit'' of~0), then the group never maps an element of~$S$
to an element not in~$S$. In other words, the group also induces a
permutation group on the elements of~$S$. We will make use of this
property in the algorithm below.

\fi

\N{1}{19}Data structures. I'm going to try to describe the algorithm
simultaneously as I explain its main data structures, because the
two are somewhat intertwined.

Our main task is to check that a set $S=\{x_1,\ldots,x_l\}$, where
$x_1<\cdots<x_l$, cannot be mapped into a smaller set $\alpha S$, for
any automorphism~$\alpha$. This breaks down into subtasks where we consider
cases in which $\alpha x_i$ is the smallest element of $\alpha S$. If we
can map $x_i$ into a number less than $x_1$, we reject the set~$S$.
If we cannot map $x_i$ into any number $\le x_1$, we need not consider this
subtask any further. But if there is an $\alpha$ such that $\alpha x_i=x_1$,
we fix our attention on one such $\alpha$ and we reduce the problem to
showing that the set $\alpha x_1,\ldots,\alpha x_{i-1},\alpha x_{i+1},\ldots,
\alpha x_l\}$ cannot be mapped into a set smaller than $\{x_2,\ldots,x_l\}$
in the subgroup of automorphisms that fix all elements $\le x_1$.
For this subproblem we first rule out all elements that could map into anything
{\it between\/} $x_1$ and $x_2$; then we consider only the subgroup of
automorphisms on the {\it remaining\/} elements.
(Only one $\alpha$ is needed for each $x_i$; this is a key point, which
is proved below.)

\fi

\M{20}The data structure we use to support such an approach has one node \PB{%
\|p} for
each subtask. If we are currently trying to see whether $\{y_1,\ldots,y_r\}$
can be mapped into a set smaller than $\{x_d,\ldots,x_l\}$, by automorphisms
that fix all elements $\le x_{d-1}$, then $p$ will be on level~$d$ of the
current tree of tasks and subtasks, and \PB{$\|p\MG\\{val}$} will be $y_i$ for
some~$i$.

The task tree is triply linked in the conventional way, with links
\PB{\\{par}}, \PB{\\{kid}}, and \PB{\\{sib}} for parent, child, and sibling,
respectively.
In other words, if \PB{\|p} is a subtask of~\PB{\|q}, we have \PB{$\|p\MG%
\\{par}\K\|q$},
and the subtasks
of~\PB{\|q} (including \PB{\|p} itself) are respectively \PB{$\|q\MG\\{kid}$}, %
\PB{$\|q\MG\\{kid}\MG\\{sib}$},
\PB{$\|q\MG\\{kid}\MG\\{sib}\MG\\{sib}$}, etc.
The youngest child of a family, namely the child
added most recently to the structure, is \PB{$\|q\MG\\{kid}$}, and the next
youngest
is \PB{$\|q\MG\\{kid}\MG\\{sib}$}.

An additional field \PB{$\|p\MG\\{trans}$}, if non-null, is the automorphism $%
\alpha$ by
which the values of \PB{\|p}'s subtasks $\{\alpha y_1,\ldots,\alpha y_{i-1},
\alpha y_{i+1},\ldots,\alpha y_r\}$ have been transformed. If \PB{$\|p\MG%
\\{trans}$}
is null it means that subtask~\PB{\|p} is ``dead'' because $y_i$ cannot be
transformed into anything $\le x_d$.

\Y\B\4\X5:Type definitions\X${}\mathrel+\E{}$\6
\&{typedef} \&{struct} \&{node\_struct} ${}\{{}$\1\6
\&{SETcard} \\{val};\C{ value that is assumed to be smallest after
          subsequent mapping }\6
\&{char} \\{level};\C{ state information for terminal nodes, see below }\6
\&{struct} \&{node\_struct} ${}{*}\\{par},{}$ ${}{*}\\{kid},{}$ ${}{*}%
\\{sib}{}$;\C{ pointers in triply linked tree }\6
\&{char} ${}{*}\\{trans}{}$;\C{ a permutation, or \PB{$\NULL$} }\2\6
${}\}{}$ \&{node};\par
\fi

\M{21}Part of the algorithm is somewhat subtle and requires proof, so I shall
try
to state the inductive assumptions carefully. But I will state them only in a
particular case, in order to keep the notation simple. The general case can
easily be inferred from the special case considered here.

Suppose \PB{\|q} is a node at level 2 that corresponds to the subtask for
permutations that map $x_3\mapsto x_1$ and $x_5\mapsto x_2$. Then we have
found particular
automorphisms $\alpha_1$ and $\alpha_2$ such that $\alpha_1 x_3=x_1$
and $\alpha_2\alpha_1 x_5=x_2$, where $\alpha_2$ fixes all elements $\le x_1$.
(Automorphism $\alpha_2$ will be stored in \PB{$\|q\MG\\{trans}$} and $%
\alpha_1$ will
be in \PB{$\|q\MG\\{par}\MG\\{trans}$}; $x_3$ will be \PB{$\|q\MG\\{par}\MG%
\\{val}$}, and \PB{$\|q\MG\\{val}$} will
be $\alpha_1x_5$. Also
\PB{$\|q\MG\\{sib}\MG\\{val}$}${}=\alpha_1x_4$,
\PB{$\|q\MG\\{sib}\MG\\{sib}\MG\\{val}$}${}=\alpha_1x_2$,
\PB{$\|q\MG\\{sib}\MG\\{sib}\MG\\{sib}\MG\\{val}$}${}=\alpha_1x_1$, and
\PB{$\|q\MG\\{sib}\MG\\{sib}\MG\\{sib}\MG\\{sib}\K\NULL$}.)

We assume that there is a set $X_q$ of ``legal'' elements that might be used
to extend the current set $\{x_1,\ldots,x_l\}$. This set has the property
that, if $\alpha$ is any automorphism with $\alpha x_3=x_1$ and $\alpha
x_5=x_2$ and $\alpha x=y$ for some~$y$, where $x\in X_q$, then there is
an automorphism $\alpha'$ that fixes all elements $\le x_2$ and satisfies
$\alpha'\alpha_2\alpha_1 x=y$. In other words, we assume that we can obtain
the images
of all legal elements that belong to our subtask by restricting consideration
to automorphisms of the form $\alpha\alpha_2\alpha_1$, where $\alpha$
fixes all elements $\le x_2$ and where $\alpha_1$ and $\alpha_2$ are the
particular automorphisms we have chosen..

Now one of the subtasks below \PB{\|q} will be node \PB{\|p}, having
\PB{$\|p\MG\\{val}$}${}=\alpha_2\alpha_1x_4$ and \PB{$\|p\MG\\{trans}$}${}=%
\alpha_3$, where $\alpha_3$
is some automorphism that fixes everything $\le x_2$ and satisfies
$\alpha_3\alpha_2\alpha_1x_4=x_3$. The set of legal elements $X_p$ is obtained
by deleting from $X_q$ all elements~$z$ such that $\alpha\alpha_2\alpha_1z
< x_3$ for some $\alpha$ fixing $\le x_2$.

To prove that $X_p$ satisfies the required inductive assumption, suppose
$\alpha$ is any automorphism with $\alpha x_3=x_1$, $\alpha x_5=x_2$,
$\alpha x_4=x_3$, and $\alpha x=y$, where $x\in X_p$. Then since $x\in X_q$,
there is $\alpha'$ fixing $\le x_2$ such that $\alpha'\alpha_2\alpha_1x=y$.
And by definition of $X_p$ we know that $y\ge x_3$; we can assume $x\ne x_4$,
hence $y>x_3$. We can write $\alpha'=
\beta\alpha''$, where $\beta$ is a product of $x_3-x_2$ elements from the
\PB{\\{perm}} table (one from each row~$j$ for $x_2\le j<x_3$) and $\alpha''$
fixes
all elements $\le x_3$.

Consider the group $G$ of all $\alpha$ that fix the elements $\le x_2$. Every
permutation $\pi$ in $G$ is a product $\pi'\pi''$, where $\pi'$ permutes
the elements that can map into $\{x_2,\ldots,x_3-1\}$ and $\pi''$ permutes
those that can't. Thus we have $\beta=\beta'\beta''$ and $\alpha_3=
\alpha_3'\alpha_3''$. We also have
$\pi\alpha_2\alpha_1x=\pi''\alpha_2\alpha_1x$ for any $\pi$ in~$G$.

Now the permutation $\alpha''\beta''\alpha_3''^{-}$ fixes all elements
$\le x_3$, and we have
$\alpha''\beta''\alpha_3''{}^{-}\alpha_3\alpha_2\alpha_1x=
\alpha''\beta''\alpha_3'\alpha_2\alpha_1x=
\alpha''\beta''\alpha_2\alpha_1x=
\alpha''\beta''\beta'\alpha_2\alpha_1x=
\alpha'\alpha_2\alpha_1x=y$.

{\sl Important Note [30 April 2001]:\/} Oops no, that permutations does
{\it not\/} necessarily fix~$x_3$. I don't know at present how to repair this
error without spoiling the efficiency of the algorithm.

Suppose $\{x_1,\ldots,x_l\}$ is non canonical because some $\alpha$ has
$\alpha x_3=x_1$, $\alpha x_5=x_2$, $\alpha x_4=x_3$, and $\alpha x_1<x_4$.
If $x_1$ is in $X_p$, we've proved that there must be $\alpha'$ fixing $\le
x_3$ such that $\alpha\alpha_3\alpha_2\alpha_1x_1<x_4$; and
$\alpha_3\alpha_2\alpha_1x_1$ will be the value of one of $p$'s children, so
we will discover the existence of $\alpha'$ by looking at the \PB{\\{minp}}
table.
If $x_1$ is in $X_q$ but not $X_p$, there must be $\alpha'$ fixing $\le x_2$
such that $\alpha\alpha_2\alpha_1x_1<x_3$; the fact that $\{x_1,\ldots,x_l\}$
is noncanonical will be discovered on a sibling task $r$ of~$p$, having
\PB{$\|r\MG\\{val}$}${}=\alpha_2\alpha_1x_1$. Similarly, if $x_1$ is not in
$X_q$ there
must be $\alpha'$ fixing $\le x_1$ such that $\alpha'\alpha_1x_1<x_2$, and
a sibling task of~$q$ will discover this.

That, for me, completes the proof.
Readers who do not believe that I lived up to my
promise of ``keeping the notation simple'' are encouraged to supply a nicer
argument; I~decided to use brute force here in order to familiarize myself with
the underlying structure.

(Speaking of notation, I must admit to being unhappy today with my former
choice, in
{\mc SETSET}, of writing $\alpha x$ instead of $x\alpha$ for the image of
$x$ under a permutation~$\alpha$. This has compelled me to write
$\alpha_2\alpha_1$
for the permutation in which $\alpha_1$ is applied before $\alpha_2$,
against my normal custom and preference. Certainly I'll use the other
order if I ever write this up.)

\Y\B\4\X6:Global variables\X${}\mathrel+\E{}$\6
\&{char} ${}\\{legal}[\\{nn}+\T{1}]{}$;\C{ nonzero when a card is legal in all
subtasks }\par
\fi

\M{22}Nodes come and go in a last-in-first-out fashion, so we can allocate them
sequentially.

\Y\B\4\D$\\{max\_node\_count}$ \5
\T{22000000}\par
\Y\B\4\X22:Subroutines\X${}\E{}$\6
\&{node} ${}{*}\\{new\_node}(\,){}$\1\1\2\2\6
${}\{{}$\1\6
\&{register} \&{node} ${}{*}\|p\K\\{node\_ptr}\PP;{}$\7
\&{if} ${}(\|p\G{\AND}\\{nodes}[\\{max\_node\_count}]){}$\5
${}\{{}$\1\6
${}\\{fprintf}(\\{stderr},\39\.{"Node\ memory\ overflo}\)\.{w!\\n"});{}$\6
${}\\{exit}({-}\T{3});{}$\6
\4${}\}{}$\2\6
${}\|p\MG\\{kid}\K\NULL;{}$\6
\&{return} \|p;\6
\4${}\}{}$\2\par
\As24, 25, 30, 34, 36, 37, 38\ETs44.
\U1.\fi

\M{23}\B\D$\\{root}$ \5
${\AND}\\{nodes}{}$[\T{0}]\par
\Y\B\4\X6:Global variables\X${}\mathrel+\E{}$\6
\&{node} \\{nodes}[\\{max\_node\_count}];\6
\&{node} ${}{*}\\{node\_ptr}\K{\AND}\\{nodes}[\T{1}]{}$;\par
\fi

\M{24}When we're processing a set $\{x_1,\ldots,x_l\}$, every active node on
level~$d$ of the tree has $l-d$ children. Thus when $l$ increases by~1, every
active node gains a child; this leads to an interesting recursive algorithm.

The \PB{\\{new\_kid}} procedure creates a new child of $p$ at level $d+1$ with
a given value~$v$.

If $d$ is sufficiently deep, we switch to another strategy described below.

\Y\B\4\X22:Subroutines\X${}\mathrel+\E{}$\6
\&{void} \\{launch}(\&{node} ${}{*},\39\&{int},\39{}$\&{node} ${}{*}){}$;\C{
prototype for a subroutine declared below }\6
\&{void} \\{new\_terminal\_kid}(\&{node} ${}{*},\39\&{SETcard}){}$;\C{ and
another }\7
\&{void} \\{new\_kid}(\&{node} ${}{*}\|p,\39{}$\&{int} \|d${},\39{}$\&{SETcard}
\|v)\1\1\2\2\6
${}\{{}$\1\6
\&{register} \&{node} ${}{*}\|q;{}$\7
\X29:Use special strategy for \PB{\\{new\_kid}} if \PB{\|d} is sufficiently
deep\X;\6
${}\|q\K\\{new\_node}(\,);{}$\6
${}\|q\MG\\{val}\K\|v,\39\|q\MG\\{par}\K\|p,\39\|q\MG\\{level}\K\T{0};{}$\6
${}\|q\MG\\{sib}\K\|p\MG\\{kid},\39\|p\MG\\{kid}\K\|q;{}$\6
${}\\{launch}(\|p,\39\|d+\T{1},\39\|q);{}$\6
\&{for} ${}(\|q\K\|q\MG\\{sib};{}$ \|q; ${}\|q\K\|q\MG\\{sib}){}$\1\6
\&{if} ${}(\|q\MG\\{trans}){}$\1\5
${}\\{new\_kid}(\|q,\39\|d+\T{1},\39\|q\MG\\{trans}[\|v]);{}$\2\2\6
\4${}\}{}$\2\par
\fi

\M{25}At this point we've reached the heart of the algorithm, the \PB{%
\\{launch}}
subroutine that initializes each new node created by \PB{\\{new\_kid}} (or by
\PB{\\{launch}} itself). When \PB{$\\{launch}(\|p,\|d,\|q)$} is called, node~%
\PB{\|q} is a child of~\PB{\|p}
on level~\PB{\|d} whose \PB{\\{val}} field has been set but the \PB{\\{trans}}
field has not.

\Y\B\4\X22:Subroutines\X${}\mathrel+\E{}$\6
\&{void} \\{launch}(\&{node} ${}{*}\|p,\39{}$\&{int} \|d${},\39{}$\&{node}
${}{*}\|q){}$\1\1\2\2\6
${}\{{}$\1\6
\&{register} \&{int} \|v${},{}$ \|w;\6
\&{register} \&{node} ${}{*}\|r,{}$ ${}{*}\|s,{}$ ${}{*}\|t;{}$\7
${}\|v\K\|q\MG\\{val};{}$\6
${}\|w\K\\{minp}[\|x[\|d-\T{1}]+\T{1}][\|v][\|v];{}$\6
\&{if} ${}(\|w<\|x[\|d]){}$\1\5
\X42:Reject the current set $\{x_1,\ldots,x_l\}$\X;\2\6
\&{if} ${}(\|w>\|x[\|d]){}$\1\5
${}\|q\MG\\{trans}\K\NULL{}$;\C{ this branch is dormant }\2\6
\&{else}\1\5
\X26:Launch a new node that maps $v$ to $x_d$\X;\2\6
\4${}\}{}$\2\par
\fi

\M{26}We can exclude values that will map between $x_{d-1}$ and $x_{d+1}$,
at least when $d<l$. By setting $x_{l+1}=x_l$ below, we make this
work also when $d=l$.

\Y\B\4\X26:Launch a new node that maps $v$ to $x_d$\X${}\E{}$\6
${}\{{}$\1\6
${}\|w\K\|x[\|d-\T{1}]+\T{1};{}$\6
${}\|q\MG\\{trans}\K\\{minp}[\|w][\|v];{}$\6
\&{for} ( ; ${}\|w<\|x[\|d+\T{1}];{}$ ${}\|w\PP){}$\1\6
\&{if} ${}(\|w\I\|x[\|d]){}$\1\5
\X27:Make sure $w$ is forbidden\X;\2\2\6
\X33:Use special strategy for \PB{\\{launch}} if \PB{\|d} is sufficiently deep%
\X;\6
\&{for} ${}(\|r\K\|p\MG\\{kid},\39\|s\K\NULL;{}$ \|r; ${}\|r\K\|r\MG\\{sib}){}$%
\1\6
\&{if} ${}(\|r\I\|q){}$\5
${}\{{}$\1\6
${}\|t\K\\{new\_node}(\,);{}$\6
${}\|t\MG\\{par}\K\|q;{}$\6
${}\|t\MG\\{val}\K\|q\MG\\{trans}[\|r\MG\\{val}],\39\|t\MG\\{level}\K\T{0};{}$\6
\&{if} (\|s)\1\5
${}\|s\MG\\{sib}\K\|t{}$;\5
\2\&{else}\1\5
${}\|q\MG\\{kid}\K\|t;{}$\2\6
${}\|s\K\|t;{}$\6
\4${}\}{}$\2\2\6
\&{if} (\|s)\1\5
${}\|s\MG\\{sib}\K\NULL;{}$\2\6
\&{else}\1\5
${}\\{auts}\PP{}$;\C{ when we've reached level $l$, we've found an automorphism
}\2\6
\&{for} ${}(\|r\K\|q\MG\\{kid};{}$ \|r; ${}\|r\K\|r\MG\\{sib}){}$\1\5
${}\\{launch}(\|q,\39\|d+\T{1},\39\|r);{}$\2\6
\4${}\}{}$\2\par
\U25.\fi

\M{27}The \PB{\\{forbidden}} table is used in the algorithm below to ensure
that no
\SET/ occurs among the elements $\{x_1,\ldots,x_l\}$. We also use it here
to forbid card values that will be transformed into~\PB{\|w}
by a sequence of permutations ending with \PB{$\|q\MG\\{trans}$}. (Since all
such
cases will ultimately lead to rejection, we can presumably save time
by ruling them out in advance. If I myself had more time to spend,
I'd check this to see just how much it helps.)

The reader should not confuse ``forbidden'' elements with elements that are
``illegal'' in the sense of the proof above. The two concepts are related,
but the algorithm would work even if the present step were omitted.

\Y\B\4\X27:Make sure $w$ is forbidden\X${}\E{}$\6
${}\{{}$\1\6
\&{for} ${}(\|r\K\|q,\39\|v\K\|w;{}$ ${}\|r\I\\{root};{}$ ${}\|r\K\|r\MG%
\\{par}){}$\1\5
${}\|v\K\|r\MG\\{trans}[\|v+\\{nnn}];{}$\2\6
${}\\{forbidden}[\|v]\K\T{1};{}$\6
\4${}\}{}$\2\par
\U26.\fi

\M{28}\B\X6:Global variables\X${}\mathrel+\E{}$\6
\&{SETcard} \|x[\T{22}];\C{ here's where we remember $x_1$, $x_2$, etc. }\6
\&{char} ${}\\{forbidden}[\\{nn}+\T{1}]{}$;\C{ nonzero for noncanonical choices
}\6
\&{int} \\{auts};\C{ automorphisms of $\{x_1,\ldots,x_l\}$ found }\6
\&{int} \|l;\C{ the current level }\par
\fi

\M{29}Eventually we get to a level so deep that only the identity mapping fixes
all elements $\le x_d$. Then the algorithm we have described so far,
although correct, begins to spin its wheels as it laboriously finds at
most one active child of each node.

Therefore we streamline the data structures for all such nodes, which we
call ``terminal,'' and we go into a different mode of operation when we
reach a terminal node. Such a node~\PB{\|q} is identified by the condition
\PB{$\|q\MG\\{level}>\T{0}$};
if \PB{$\|q\MG\\{level}\K\|k$} it means that all nodes $\{x_1,\ldots,x_{k-1}\}$
have
been matched among \PB{\|q} and its ancestors and older siblings. The next
younger siblings of~\PB{\|q} will therefore try to match $x_k$.

The \PB{\\{new\_kid}} procedure will add a new terminal node to the descendants
of~\PB{\|p}, if \PB{\|d} is sufficiently deep as described above; this will be
true
if and only if $x_d\ge D-1$, where $D$ is the \PB{\\{perm}} table depth.
If \PB{\|p} already has at least one child, all of its children are terminal,
and we use the \PB{\\{new\_terminal\_kid}} procedure to extend \PB{\|p}'s
family.
Otherwise we initiate the family, using the fact that \PB{\|l} must equal \PB{$%
\|d+\T{1}$}
if \PB{\|p} was previously childless.

\Y\B\4\X29:Use special strategy for \PB{\\{new\_kid}} if \PB{\|d} is
sufficiently deep\X${}\E{}$\6
\&{if} ${}(\|x[\|d]\G\\{dd}-\T{1}){}$\5
${}\{{}$\1\6
\&{if} ${}(\|p\MG\\{kid}){}$\1\5
${}\\{new\_terminal\_kid}(\|p,\39\|v);{}$\2\6
\&{else}\5
${}\{{}$\C{ \PB{$\|l\K\|d+\T{1}$} }\1\6
\&{if} ${}(\|v<\|x[\|l]){}$\1\5
\X42:Reject the current set $\{x_1,\ldots,x_l\}$\X;\2\6
${}\|q\K\\{new\_node}(\,);{}$\6
${}\|p\MG\\{kid}\K\|q,\39\|q\MG\\{val}\K\|v,\39\|q\MG\\{sib}\K\NULL,\39\|q\MG%
\\{par}\K\|p;{}$\6
\&{if} ${}(\|v\E\|x[\|l]){}$\1\5
${}\|q\MG\\{level}\K\|l+\T{1},\39\\{auts}\PP;{}$\2\6
\&{else}\1\5
${}\|q\MG\\{level}\K\|l;{}$\2\6
\4${}\}{}$\2\6
\&{return};\6
\4${}\}{}$\2\par
\U24.\fi

\M{30}The \PB{\\{kid}} links in terminal nodes are {\it not\/} used for
parenting; they jump across siblings known to be irrelevant in future
searches. (This is just a heuristic, designed to ameliorate the fact
that we don't want to complicate the backtracking process by updating
any existing links in a family when a new child is born.)

\Y\B\4\X22:Subroutines\X${}\mathrel+\E{}$\6
\&{void} \\{new\_terminal\_kid}(\&{node} ${}{*}\|q,\39{}$\&{SETcard} \|v)\1\1\2%
\2\6
${}\{{}$\1\6
\&{register} \&{node} ${}{*}\|r,{}$ ${}{*}\|p;{}$\6
\&{register} \&{int} \|k${},{}$ \|w;\7
${}\|r\K\\{new\_node}(\,);{}$\6
${}\|r\MG\\{val}\K\|v,\39\|r\MG\\{sib}\K\|q\MG\\{kid},\39\|q\MG\\{kid}\K\|r;{}$%
\6
\X31:Find the index $k$ such that we've matched $\{x_1,\ldots,x_{k-1}\}$ and
such that all other values exceed~$x_k$\X;\6
${}\|r\MG\\{level}\K\|k;{}$\6
\&{for} ${}(\|p\K\|r\MG\\{sib};{}$ \|p; ${}\|p\K\|p\MG\\{kid}){}$\1\6
\&{if} ${}(\|p\MG\\{val}>\|x[\|k]){}$\1\5
\&{break};\2\2\6
${}\|r\MG\\{kid}\K\|p;{}$\6
\4${}\}{}$\2\par
\fi

\M{31}\B\X31:Find the index $k$ such that we've matched $\{x_1,\ldots,x_{k-1}%
\}$ and such that all other values exceed~$x_k$\X${}\E{}$\6
$\|k\K\|r\MG\\{sib}\MG\\{level};{}$\6
\&{while} (\T{1})\5
${}\{{}$\C{ \PB{\|v} is the smallest value greater than $x_{k-1}$ }\1\6
\&{if} ${}(\|v<\|x[\|k]){}$\1\5
\X42:Reject the current set $\{x_1,\ldots,x_l\}$\X;\2\6
\&{if} ${}(\|v>\|x[\|k]){}$\1\5
\&{break};\2\6
\X32:Forbid values that will cause rejection if they propagate this far\X;\6
${}\|k\PP;{}$\6
\&{if} ${}(\|k>\|l){}$\5
${}\{{}$\1\6
${}\\{auts}\PP{}$;\C{ hey, we've matched everything }\6
\&{break};\6
\4${}\}{}$\2\6
\&{for} ${}(\|p\K\|r\MG\\{sib},\39\|v\K\\{nn};{}$ \|p; ${}\|p\K\|p\MG%
\\{kid}){}$\1\6
\&{if} ${}(\|p\MG\\{val}>\|x[\|k-\T{1}]\W\|p\MG\\{val}<\|v){}$\1\5
${}\|v\K\|p\MG\\{val};{}$\2\2\6
\4${}\}{}$\2\par
\U30.\fi

\M{32}We can exclude values that will map between $x_{k-1}$ and $x_{k+1}$,
at least when $k<l$. By setting $x_{l+1}=x_l$ below, we make this
work also when $k=l$.

\Y\B\4\X32:Forbid values that will cause rejection if they propagate this far%
\X${}\E{}$\6
\&{for} ${}(\|w\K\|x[\|k-\T{1}]+\T{1};{}$ ${}\|w<\|x[\|k+\T{1}];{}$ ${}\|w%
\PP){}$\1\6
\&{if} ${}(\|w\I\|x[\|k]){}$\5
${}\{{}$\1\6
\&{for} ${}(\|p\K\|q,\39\|v\K\|w;{}$ ${}\|p\I\\{root};{}$ ${}\|p\K\|p\MG%
\\{par}){}$\1\5
${}\|v\K\|p\MG\\{trans}[\|v+\\{nnn}];{}$\2\6
${}\\{forbidden}[\|v]\K\T{1};{}$\6
\4${}\}{}$\2\2\par
\U31.\fi

\M{33}When all children of a newly launched node \PB{\|q} are terminal,
we append them in reverse order. This is done only for convenience,
because the order
is unimportant in newly launched nodes. (Such nodes will disappear completely
when we backtrack.)

\Y\B\4\X33:Use special strategy for \PB{\\{launch}} if \PB{\|d} is sufficiently
deep\X${}\E{}$\6
\&{if} ${}(\|x[\|d]\G\\{dd}-\T{1}\W\|d<\|l){}$\5
${}\{{}$\1\6
${}\|s\K\|p\MG\\{kid}{}$;\5
\&{if} ${}(\|s\E\|q){}$\1\5
${}\|s\K\|s\MG\\{sib};{}$\2\6
${}\|v\K\|q\MG\\{trans}[\|s\MG\\{val}];{}$\6
\&{if} ${}(\|v<\|x[\|d+\T{1}]){}$\1\5
\X42:Reject the current set $\{x_1,\ldots,x_l\}$\X;\2\6
${}\|r\K\\{new\_node}(\,);{}$\6
${}\|q\MG\\{kid}\K\|r,\39\|r\MG\\{par}\K\|q;{}$\6
${}\|r\MG\\{val}\K\|v,\39\|r\MG\\{sib}\K\NULL;{}$\6
\&{if} ${}(\|v\E\|x[\|d+\T{1}]){}$\5
${}\{{}$\1\6
${}\|r\MG\\{level}\K\|d+\T{2};{}$\6
\&{if} ${}(\|d+\T{1}\E\|l){}$\1\5
${}\\{auts}\PP;{}$\2\6
\4${}\}{}$\5
\2\&{else}\1\5
${}\|r\MG\\{level}\K\|d+\T{1};{}$\2\6
\&{for} ${}(\|s\K\|s\MG\\{sib};{}$ \|s; ${}\|s\K\|s\MG\\{sib}){}$\1\6
\&{if} ${}(\|s\I\|q){}$\1\5
${}\\{new\_terminal\_kid}(\|q,\39\|q\MG\\{trans}[\|s\MG\\{val}]);{}$\2\2\6
\&{return};\6
\4${}\}{}$\2\par
\U26.\fi

\M{34}Here's a subroutine that I expect will be useful during the debugging
process.

\Y\B\4\X22:Subroutines\X${}\mathrel+\E{}$\6
\&{void} \\{print\_subtree}(\&{node} ${}{*}\|p,\39{}$\&{int} \|d)\1\1\2\2\6
${}\{{}$\1\6
\&{register} \&{node} ${}{*}\|r;{}$\6
\&{register} \&{int} \|k;\7
\&{for} ${}(\|k\K\T{0};{}$ ${}\|k<\|d;{}$ ${}\|k\PP){}$\1\5
${}\\{printf}(\|p\MG\\{level}\?\.{"\ "}:\.{"."});{}$\2\6
${}\\{printf}(\.{"\%04d"},\39\\{decode}[\|p\MG\\{val}]);{}$\6
\&{if} ${}(\|p\MG\\{level}){}$\5
${}\{{}$\1\6
${}\\{printf}(\.{",\%d"},\39\|p\MG\\{level});{}$\6
\&{if} ${}(\|p\MG\\{kid}){}$\1\5
${}\\{printf}(\.{"\ ->\%04d"},\39\\{decode}[\|p\MG\\{kid}\MG\\{val}]);{}$\2\6
\\{printf}(\.{"\\n"});\6
\4${}\}{}$\2\6
\&{else} \&{if} ${}(\|p\MG\\{trans}){}$\5
${}\{{}$\1\6
\X35:Print the current transform matrix\X;\6
\&{for} ${}(\|r\K\|p\MG\\{kid};{}$ \|r; ${}\|r\K\|r\MG\\{sib}){}$\1\5
${}\\{print\_subtree}(\|r,\39\|d+\T{1});{}$\2\6
\4${}\}{}$\5
\2\&{else}\1\5
\\{printf}(\.{"\\n"});\2\6
\4${}\}{}$\2\par
\fi

\M{35}We print the matrix by giving five vectors $y_i$ such that
$(x_1,x_2,x_3,x_4)\mapsto y_0+x_1y_1+x_2y_2+x_3y_3+x_4y_4$.

\Y\B\4\X35:Print the current transform matrix\X${}\E{}$\6
$\|k\K\|p\MG\\{trans}[\T{0}];{}$\6
${}\\{printf}(\.{"\ [\%04d,\%04d,\%04d,\%0}\)\.{4d,\%04d]\\n"},\39\\{decode}[%
\|k],\39\\{decode}[\\{third}[\|k][\\{third}[\T{0}][\|p\MG\\{trans}[\T{4}]]]],%
\39\\{decode}[\\{third}[\|k][\\{third}[\T{0}][\|p\MG\\{trans}[\T{3}]]]],\39%
\\{decode}[\\{third}[\|k][\\{third}[\T{0}][\|p\MG\\{trans}[\T{2}]]]],\39%
\\{decode}[\\{third}[\|k][\\{third}[\T{0}][\|p\MG\\{trans}[\T{1}]]]]){}$;\par
\U34.\fi

\M{36}\B\X22:Subroutines\X${}\mathrel+\E{}$\6
\&{void} \\{print\_trees}(\,)\1\1\2\2\6
${}\{{}$\1\6
\&{register} \&{node} ${}{*}\|r;{}$\7
\&{for} ${}(\|r\K(\\{root})\MG\\{kid};{}$ \|r; ${}\|r\K\|r\MG\\{sib}){}$\1\5
${}\\{print\_subtree}(\|r,\39\T{0});{}$\2\6
\4${}\}{}$\2\par
\fi

\M{37}More help for debugging: \PB{\\{nod}(\.{"123"})} gives the third-youngest
child of the
second-youngest child of the youngest child of the root.

\Y\B\4\X22:Subroutines\X${}\mathrel+\E{}$\6
\&{node} ${}{*}{}$\\{nod}(\&{char} ${}{*}\|s){}$\1\1\2\2\6
${}\{{}$\1\6
\&{register} \&{char} ${}{*}\|p;{}$\6
\&{register} \&{int} \|j;\6
\&{register} \&{node} ${}{*}\|q\K\\{root};{}$\7
\&{for} ${}(\|p\K\|s;{}$ ${}{*}\|p;{}$ ${}\|p\PP){}$\5
${}\{{}$\1\6
\&{if} ${}(\R\|q){}$\1\5
\&{return} ${}\NULL;{}$\2\6
\&{for} ${}(\|j\K{*}\|p-\.{'1'},\39\|q\K\|q\MG\\{kid};{}$ \|j; ${}\|j\MM){}$\5
${}\{{}$\1\6
\&{if} ${}(\R\|q){}$\1\5
\&{return} ${}\NULL;{}$\2\6
${}\|q\K\|q\MG\\{sib};{}$\6
\4${}\}{}$\2\6
\4${}\}{}$\2\6
\&{return} \|q;\6
\4${}\}{}$\2\7
\&{void} \\{dummy}(\,)\1\1\2\2\6
${}\{{}$\1\6
\\{malloc}(\T{1});\C{ loads a routine needed by \.{gdb} }\6
\4${}\}{}$\2\par
\fi

\M{38}And here's a sort of converse routine, \PB{\\{whoami}}.

\Y\B\4\X22:Subroutines\X${}\mathrel+\E{}$\6
\&{void} \\{print\_id}(\&{node} ${}{*}\|p){}$\1\1\2\2\6
${}\{{}$\1\6
\&{register} \&{node} ${}{*}\|q\K\|p\MG\\{par},{}$ ${}{*}\|r;{}$\6
\&{register} \&{char} \|j;\7
\&{if} (\|q)\5
${}\{{}$\1\6
\\{print\_id}(\|q);\6
\&{for} ${}(\|r\K\|q\MG\\{kid},\39\|j\K\.{'1'};{}$ ${}\|r\I\|p;{}$ ${}\|j%
\PP){}$\1\6
\&{if} (\|r)\1\5
${}\|r\K\|r\MG\\{sib};{}$\2\6
\&{else}\5
${}\{{}$\1\6
\\{printf}(\.{"???"});\5
\&{return};\6
\4${}\}{}$\2\2\6
${}\\{printf}(\.{"\%c"},\39\|j);{}$\6
\4${}\}{}$\2\6
\4${}\}{}$\2\7
\&{void} \\{whoami}(\&{node} ${}{*}\|p){}$\1\1\2\2\6
${}\{{}$\1\6
\\{print\_id}(\|p);\5
\\{printf}(\.{"\\n"});\6
\4${}\}{}$\2\par
\fi

\N{1}{39}Backtracking. Now we're ready to construct the tree of all canonical
\SET/-free sets $\{x_1,\ldots,x_l\}$.

\Y\B\4\X39:Enumerate and print all solutions\X${}\E{}$\6
$\|l\K\T{0}{}$;\5
${}\|j\K\T{0};{}$\6
${}\|x[\T{0}]\K{-}\T{1};{}$\6
\&{if} (\\{setjmp}(\\{restart\_point}))\1\5
\&{goto} \\{backup};\C{ get ready for \PB{\\{longjmp}} }\2\6
\4\\{moveup}:\5
\&{while} (\\{forbidden}[\|j])\1\5
${}\|j\PP;{}$\2\6
\&{if} ${}(\|j\E\\{nn}){}$\1\5
\&{goto} \\{backup};\2\6
\&{for} ${}(\|k\K\T{0};{}$ ${}\|k<\\{nn};{}$ ${}\|k\PP){}$\1\5
${}\\{forbidden\_back}[\|l][\|k]\K\\{forbidden}[\|k];{}$\2\6
${}\\{node\_ptr\_back}[\|l]\K\\{node\_ptr};{}$\6
${}\\{auts}\K\T{0};{}$\6
${}\|l\PP,\39\|x[\|l]\K\|x[\|l+\T{1}]\K\|j;{}$\6
${}\\{new\_kid}(\\{root},\39\T{0},\39\|x[\|l]);{}$\6
\X45:Record the current canonical $l$-set as a solution\X;\6
\X43:Forbid all \SET/s that include $\{x_k,x_l\}$ for $1\le k<l$\X;\6
${}\|j\K\|x[\|l]+\T{1}{}$;\5
\&{goto} \\{moveup};\6
\4\\{backup}:\5
${}\|l\MM;{}$\6
${}\\{node\_ptr}\K\\{node\_ptr\_back}[\|l];{}$\6
\\{prune}(\\{root});\6
\&{for} ${}(\|k\K\T{0};{}$ ${}\|k<\\{nn};{}$ ${}\|k\PP){}$\1\5
${}\\{forbidden}[\|k]\K\\{forbidden\_back}[\|l][\|k];{}$\2\6
${}\|j\K\|x[\|l+\T{1}]+\T{1};{}$\6
\&{if} (\|l)\1\5
\&{goto} \\{moveup};\2\par
\U1.\fi

\M{40}\B\X40:Local variables\X${}\E{}$\6
\&{register} \&{int} \|i${},{}$ \|j${},{}$ \|k;\C{ miscellaneous indices }\par
\U1.\fi

\M{41}\B\X6:Global variables\X${}\mathrel+\E{}$\6
\&{char} \\{forbidden\_back}[\T{22}][\\{nnn}];\C{ brute-force undoing }\6
\&{node} ${}{*}\\{node\_ptr\_back}[\T{22}]{}$;\par
\fi

\M{42}If the recursive procedures invoked by \PB{\\{new\_kid}} lead to a
non-canonical
situation, we leave them and back up by using \CEE/'s \PB{\\{longjmp}}
library function. (The code above will then cause control to pass
to the label \PB{\\{rejected}}.)

\Y\B\4\X42:Reject the current set $\{x_1,\ldots,x_l\}$\X${}\E{}$\6
$\\{longjmp}(\\{restart\_point},\39\T{1}){}$;\par
\Us25, 29, 31\ETs33.\fi

\M{43}\B\X43:Forbid all \SET/s that include $\{x_k,x_l\}$ for $1\le k<l$\X${}%
\E{}$\6
\&{for} ${}(\|k\K\T{1};{}$ ${}\|k<\|l;{}$ ${}\|k\PP){}$\1\5
${}\\{forbidden}[\\{third}[\|x[\|k]][\|x[\|l]]]\K\T{1}{}$;\2\par
\U39.\fi

\M{44}The data structures have been designed so that all changes invoked by
\PB{\\{new\_kid}} and \PB{\\{launch}} are easily undone. Indeed, if \PB{$\\{new%
\_kid}(\\{root},\T{0},\|x[\|l])$}
terminates normally, it has added precisely one child to each active node
(including any terminal nodes that are present), and the newly added child
will of course be the youngest. But if \PB{\\{new\_kid}} terminates abnormally
via \PB{\\{longjmp}}, some active nodes may not have been reached.

\Y\B\4\X22:Subroutines\X${}\mathrel+\E{}$\6
\&{void} \\{prune}(\&{node} ${}{*}\|p){}$\1\1\2\2\6
${}\{{}$\1\6
\&{register} \&{node} ${}{*}\|q\K\|p\MG\\{kid};{}$\6
\&{register} \&{node} ${}{*}\|r;{}$\7
\&{if} (\|q)\5
${}\{{}$\1\6
${}\|r\K\|q;{}$\6
\&{if} ${}(\|q\G\\{node\_ptr}){}$\1\5
${}\|p\MG\\{kid}\K\|q\K\|q\MG\\{sib}{}$;\C{ the youngest should be pruned away
}\2\6
\&{if} ${}(\R\|r\MG\\{level}){}$\1\6
\&{for} ( ; \|q; ${}\|q\K\|q\MG\\{sib}){}$\1\5
\\{prune}(\|q);\2\2\6
\4${}\}{}$\2\6
\4${}\}{}$\2\par
\fi

\N{1}{45}The totals. I want to know not only the nonisomorphic solutions but
also
the exact number of \SET/-less $k$ sets that are possible. Then I'll know the
precise odds of having no \SET/ in a random deal.

When the program reaches this point, \PB{\\{auts}} will have been set to the
number
of permutations of $\{x_1,\ldots,x_l\}$ that are achievable by automorphisms.
The true number of automorphisms of $\{x_1,\ldots,x_l\}$ will therefore be
\PB{\\{auts}} times the number of automorphisms that fix each of $\{x_1,%
\ldots,x_l\}$.

I don't know how to compute the latter quantity easily from the \PB{\\{perm}}
table
of a general permutation group. But in the affine linear group of interest
here, we need only determine the number of independent elements. This is
the smallest index~$k$ such that $x_{k+1}\ne k$.

\Y\B\4\X45:Record the current canonical $l$-set as a solution\X${}\E{}$\6
\&{for} ${}(\|j\K\T{1};{}$ ${}\|j<\|l;{}$ ${}\|j\PP){}$\1\5
\\{printf}(\.{"."});\2\6
${}\\{non\_iso\_count}[\|l]\PP;{}$\6
\&{for} ${}(\|k\K\T{0};{}$ ${}\|x[\|k+\T{1}]\E\|k;{}$ ${}\|k\PP){}$\1\5
;\2\6
${}\\{total\_count}[\|l]\MRL{+{\K}}\\{multiplier}[\|k-\T{1}]/{}$(\&{double}) %
\\{auts};\6
${}\\{printf}(\.{"\%04d\ (\%d:\%d)\ \%d\\n"},\39\\{decode}[\|x[\|l]],\39%
\\{auts},\39\|k,\39\\{node\_ptr}-\\{nodes}){}$;\par
\U39.\fi

\M{46}Integers of 32 bits are insufficient to hold the numbers we're counting,
but double precision floating point turns out to be good enough
for exact values in this problem.

\Y\B\4\X6:Global variables\X${}\mathrel+\E{}$\6
\&{int} \\{non\_iso\_count}[\T{30}];\C{ number of canonical solutions }\6
\&{double} \\{total\_count}[\T{30}];\C{ total number of solutions }\6
\&{double} \\{multiplier}[\T{5}]${}\K\{\T{81.0},\39\T{6480.0},\39\T{505440.0},%
\39\T{36391680.0},\39\T{1965150720.0}\}{}$;\par
\fi

\M{47}\B\X47:Print the totals\X${}\E{}$\6
\&{for} ${}(\|j\K\T{1};{}$ ${}\|j\Z\T{21};{}$ ${}\|j\PP){}$\1\5
${}\\{printf}(\.{"\%20.20g\ SETless\ \%d-}\)\.{sets\ (\%d\ cases)\\n"},\39%
\\{total\_count}[\|j],\39\|j,\39\\{non\_iso\_count}[\|j]){}$;\2\par
\U1.\fi

\N{1}{48}Index.

\fi


\inx
\fin
\con
