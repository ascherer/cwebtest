\input cwebmac
\srcloctrue
\datethis

\N[2 sat0.w]{1}{1}Intro. This program is part of a series of ``SAT-solvers''
that I'm putting
together for my own education as I prepare to write Section 7.2.2.2 of
{\sl The Art of Computer Programming}. My intent is to have a variety of
compatible programs on which I can run experiments to learn how different
approaches work in practice.

Indeed, this is the first of the series --- more precisely the zero-th. I've
tried to write it as a primitive baseline against which I'll be able to measure
various technical improvements that have been discovered in recent years.
This version represents what I think I would have written in the 1960s,
when I knew how to do basic backtracking with classical data structures
(but very little else). I have intentionally written it {\it before\/} having
read {\it any\/} of the literature about modern SAT-solving techniques;
in other words I'm starting with a personal ``tabula rasa.''
My plan is to write new versions as I read the literature, in more-or-less
historical order. The only thing that currently distinguishes me from a
programmer of forty years ago, SAT-solving-wise, is the knowledge that better
methods almost surely do exist.

[{\it Note:}\enspace The present code is slightly modified from the
original {\mc SAT0}. It now corresponds to what has become
Algorithm 7.2.2.2A, so that I can make the quantitative experiments
recorded in the book.]

Although this is the zero-level program, I'm taking care to adopt conventions
for input and output that will be essentially the same in all of the
fancier versions that are to come.

The input on \PB{\\{stdin}} is a series of lines with one clause per line. Each
clause is a sequence of literals separated by spaces. Each literal is
a sequence of one to eight ASCII characters between \.{!} and \.{\}},
inclusive, not beginning with \.{\~},
optionally preceded by \.{\~} (which makes the literal ``negative'').
For example, Rivest's famous clauses on four variables,
found in 6.5--(13) and 7.1.1--(32) of {\sl TAOCP}, can be represented by the
following eight lines of input:
$$\chardef~=`\~
\vcenter{\halign{\tt#\cr
x2 x3 ~x4\cr
x1 x3 x4\cr
~x1 x2 x4\cr
~x1 ~x2 x3\cr
~x2 ~x3 x4\cr
~x1 ~x3 ~x4\cr
x1 ~x2 ~x4\cr
x1 x2 ~x3\cr}}$$
Input lines that begin with \.{\~\ } are ignored (treated as comments).
The output will be `\.{\~}' if the input clauses are unsatisfiable.
Otherwise it will be a list of noncontradictory literals that cover each
clause, separated by spaces. (``Noncontradictory'' means that we don't
have both a literal and its negation.) The input above would, for example,
yield `\.{\~}'; but if the final clause were omitted, the output would
be `\.{\~x1} \.{\~x2} \.{x3}', in some order, possibly together
with either \.{x4} or \.{\~x4} (but not both). No attempt is made to
find all solutions; at most one solution is given.

The running time in ``mems'' is also reported, together with the approximate
number of bytes needed for data storage. One ``mem'' essentially means a
memory access to a 64-bit word.
(These totals don't include the time or space needed to parse the
input or to format the output.)

\fi

\M[64 sat0.w]{2}So here's the structure of the program. (Skip ahead if you are
impatient to see the interesting stuff.)

\Y\B\4\D$\|o$ \5
$\\{mems}\PP{}$\C{ count one mem }\par
\B\4\D$\\{oo}$ \5
$\\{mems}\MRL{+{\K}}{}$\T{2}\C{ count two mems }\par
\B\4\D$\\{ooo}$ \5
$\\{mems}\MRL{+{\K}}{}$\T{3}\C{ count three mems }\par
\Y\B\8\#\&{include} \.{<stdio.h>}\6
\8\#\&{include} \.{<stdlib.h>}\6
\8\#\&{include} \.{<string.h>}\6
\8\#\&{include} \.{"gb\_flip.h"}\6
\&{typedef} \&{unsigned} \&{int} \&{uint};\C{ a convenient abbreviation }\6
\&{typedef} \&{unsigned} \&{long} \&{long} \&{ullng};\C{ ditto }\7
\X5:Type definitions\X;\6
\X3:Global variables\X;\6
\X27:Subroutines\X;\7
\\{main}(\&{int} \\{argc}${},\39{}$\&{char} ${}{*}\\{argv}[\,]){}$\1\1\2\2\6
${}\{{}$\1\6
\&{register} \&{uint} \|c${},{}$ \|h${},{}$ \|i${},{}$ \|j${},{}$ \|k${},{}$ %
\|l${},{}$ \|p${},{}$ \|q${},{}$ \|r${},{}$ \\{level}${},{}$ \\{parity};\7
\X4:Process the command line\X;\6
\X8:Initialize everything\X;\6
\X9:Input the clauses\X;\6
\&{if} ${}(\\{verbose}\AND\\{show\_basics}){}$\1\5
\X21:Report the successful completion of the input phase\X;\2\6
\X30:Set up the main data structures\X;\6
${}\\{imems}\K\\{mems},\39\\{mems}\K\T{0};{}$\6
\X39:Solve the problem\X;\6
\4\\{done}:\5
\&{if} ${}(\\{verbose}\AND\\{show\_basics}){}$\1\5
${}\\{fprintf}(\\{stderr},\39\.{"Altogether\ \%llu+\%ll}\)\.{u\ mems,\ \%llu\
bytes,\ }\)\.{\%llu\ nodes.\\n"},\39\\{imems},\39\\{mems},\39\\{bytes},\39%
\\{nodes});{}$\2\6
\4${}\}{}$\2\par
\fi

\M[96 sat0.w]{3}\B\D$\\{show\_basics}$ \5
\T{1}\C{ \PB{\\{verbose}} code for basic stats }\par
\B\4\D$\\{show\_choices}$ \5
\T{2}\C{ \PB{\\{verbose}} code for backtrack logging }\par
\B\4\D$\\{show\_details}$ \5
\T{4}\C{ \PB{\\{verbose}} code for further commentary }\par
\Y\B\4\X3:Global variables\X${}\E{}$\6
\&{int} \\{random\_seed}${}\K\T{0}{}$;\C{ seed for the random words of \PB{%
\\{gb\_rand}} }\6
\&{int} \\{verbose}${}\K\\{show\_basics}{}$;\C{ level of verbosity }\6
\&{int} \\{show\_choices\_max}${}\K\T{1000000}{}$;\C{ above this level, \PB{%
\\{show\_choices}} is ignored }\6
\&{int} \\{hbits}${}\K\T{8}{}$;\C{ logarithm of the number of the hash lists }\6
\&{int} \\{buf\_size}${}\K\T{1024}{}$;\C{ must exceed the length of the longest
input line }\6
\&{ullng} \\{imems}${},{}$ \\{mems};\C{ mem counts }\6
\&{ullng} \\{bytes};\C{ memory used by main data structures }\6
\&{ullng} \\{nodes};\C{ total number of branch nodes initiated }\6
\&{ullng} \\{thresh}${}\K\T{0}{}$;\C{ report when \PB{\\{mems}} exceeds this,
if \PB{$\\{delta}\I\T{0}$} }\6
\&{ullng} \\{delta}${}\K\T{0}{}$;\C{ report every \PB{\\{delta}} or so mems }\6
\&{ullng} \\{timeout}${}\K\T{\^1fffffffffffffff}{}$;\C{ give up after this many
mems }\par
\As7\ET26.
\U2.\fi

\M[113 sat0.w]{4}On the command line one can say
\smallskip
\item{$\bullet$}
`\.v$\langle\,$integer$\,\rangle$' to enable various levels of verbose
output on \PB{\\{stderr}};
\item{$\bullet$}
`\.c$\langle\,$positive integer$\,\rangle$' to limit the levels on which
clauses are shown;
\item{$\bullet$}
`\.h$\langle\,$positive integer$\,\rangle$' to adjust the hash table size;
\item{$\bullet$}
`\.b$\langle\,$positive integer$\,\rangle$' to adjust the size of the input
buffer;
\item{$\bullet$}
`\.s$\langle\,$integer$\,\rangle$' to define the seed for any random numbers
that are used; and/or
\item{$\bullet$}
`\.d$\langle\,$integer$\,\rangle$' to set \PB{\\{delta}} for periodic state
reports.
\item{$\bullet$}
`\.T$\langle\,$integer$\,\rangle$' to set \PB{\\{timeout}}: This program will
abruptly terminate, when it discovers that \PB{$\\{mems}>\\{timeout}$}.

\Y\B\4\X4:Process the command line\X${}\E{}$\6
\&{for} ${}(\|j\K\\{argc}-\T{1},\39\|k\K\T{0};{}$ \|j; ${}\|j\MM){}$\1\6
\&{switch} (\\{argv}[\|j][\T{0}])\5
${}\{{}$\1\6
\4\&{case} \.{'v'}:\5
${}\|k\MRL{{\OR}{\K}}(\\{sscanf}(\\{argv}[\|j]+\T{1},\39\.{"\%d"},\39{\AND}%
\\{verbose})-\T{1}){}$;\5
\&{break};\6
\4\&{case} \.{'c'}:\5
${}\|k\MRL{{\OR}{\K}}(\\{sscanf}(\\{argv}[\|j]+\T{1},\39\.{"\%d"},\39{\AND}%
\\{show\_choices\_max})-\T{1}){}$;\5
\&{break};\6
\4\&{case} \.{'h'}:\5
${}\|k\MRL{{\OR}{\K}}(\\{sscanf}(\\{argv}[\|j]+\T{1},\39\.{"\%d"},\39{\AND}%
\\{hbits})-\T{1}){}$;\5
\&{break};\6
\4\&{case} \.{'b'}:\5
${}\|k\MRL{{\OR}{\K}}(\\{sscanf}(\\{argv}[\|j]+\T{1},\39\.{"\%d"},\39{\AND}%
\\{buf\_size})-\T{1}){}$;\5
\&{break};\6
\4\&{case} \.{'s'}:\5
${}\|k\MRL{{\OR}{\K}}(\\{sscanf}(\\{argv}[\|j]+\T{1},\39\.{"\%d"},\39{\AND}%
\\{random\_seed})-\T{1}){}$;\5
\&{break};\6
\4\&{case} \.{'d'}:\5
${}\|k\MRL{{\OR}{\K}}(\\{sscanf}(\\{argv}[\|j]+\T{1},\39\.{"\%lld"},\39{\AND}%
\\{delta})-\T{1}){}$;\5
${}\\{thresh}\K\\{delta}{}$;\5
\&{break};\6
\4\&{case} \.{'T'}:\5
${}\|k\MRL{{\OR}{\K}}(\\{sscanf}(\\{argv}[\|j]+\T{1},\39\.{"\%lld"},\39{\AND}%
\\{timeout})-\T{1}){}$;\5
\&{break};\6
\4\&{default}:\5
${}\|k\K\T{1}{}$;\C{ unrecognized command-line option }\6
\4${}\}{}$\2\2\6
\&{if} ${}(\|k\V\\{hbits}<\T{0}\V\\{hbits}>\T{30}\V\\{buf\_size}\Z\T{0}){}$\5
${}\{{}$\1\6
${}\\{fprintf}(\\{stderr},\39\.{"Usage:\ \%s\ [v<n>]\ [c}\)\.{<n>]\ [h<n>]\
[b<n>]\ [}\)\.{s<n>]\ [d<n>]\ [T<n>]\ }\)\.{<\ foo.sat\\n"},\39\\{argv}[%
\T{0}]);{}$\6
${}\\{exit}({-}\T{1});{}$\6
\4${}\}{}$\2\par
\U2.\fi

\N[152 sat0.w]{1}{5}The I/O wrapper. The following routines read the input and
absorb it into
temporary data areas from which all of the ``real'' data structures
can readily be initialized. My intent is to incorporate these routines in all
of the SAT-solvers in this series. Therefore I've tried to make the code
short and simple, yet versatile enough so that almost no restrictions are
placed on the sizes of problems that can be handled. These routines are
supposed to work properly unless there are more than
$2^{32}-1=4$,294,967,295 occurrences of literals in clauses,
or more than $2^{31}-1=2$,147,483,647 variables or clauses.

In these temporary tables, each variable is represented by four things:
its unique name; its serial number; the clause number (if any) in which it has
most recently appeared; and a pointer to the previous variable (if any)
with the same hash address. Several variables at a time
are represented sequentially in small chunks of memory called ``vchunks,''
which are allocated as needed (and freed later).

\Y\B\4\D$\\{vars\_per\_vchunk}$ \5
\T{341}\C{ preferably $(2^k-1)/3$ for some $k$ }\par
\Y\B\4\X5:Type definitions\X${}\E{}$\6
\&{typedef} \&{union} ${}\{{}$\1\6
\&{char} \\{ch8}[\T{8}];\6
\&{uint} \\{u2}[\T{2}];\6
\&{long} \&{long} \\{lng};\2\6
${}\}{}$ \&{octa};\6
\&{typedef} \&{struct} \&{tmp\_var\_struct} ${}\{{}$\1\6
\&{octa} \\{name};\C{ the name (one to eight ASCII characters) }\6
\&{uint} \\{serial};\C{ 0 for the first variable, 1 for the second, etc. }\6
\&{int} \\{stamp};\C{ \PB{\|m} if positively in clause \PB{\|m}; \PB{${-}\|m$}
if negatively there }\6
\&{struct} \&{tmp\_var\_struct} ${}{*}\\{next}{}$;\C{ pointer for hash list }\2%
\6
${}\}{}$ \&{tmp\_var};\7
\&{typedef} \&{struct} \&{vchunk\_struct} ${}\{{}$\1\6
\&{struct} \&{vchunk\_struct} ${}{*}\\{prev}{}$;\C{ previous chunk allocated
(if any) }\6
\&{tmp\_var} \\{var}[\\{vars\_per\_vchunk}];\2\6
${}\}{}$ \&{vchunk};\par
\As6, 23, 24\ETs25.
\U2.\fi

\M[189 sat0.w]{6}Each clause in the temporary tables is represented by a
sequence of
one or more pointers to the \PB{\&{tmp\_var}} nodes of the literals involved.
A negated literal is indicated by adding~1 to such a pointer.
The first literal of a clause is indicated by adding~2.
Several of these pointers are represented sequentially in chunks
of memory, which are allocated as needed and freed later.

\Y\B\4\D$\\{cells\_per\_chunk}$ \5
\T{511}\C{ preferably $2^k-1$ for some $k$ }\par
\Y\B\4\X5:Type definitions\X${}\mathrel+\E{}$\6
\&{typedef} \&{struct} \&{chunk\_struct} ${}\{{}$\1\6
\&{struct} \&{chunk\_struct} ${}{*}\\{prev}{}$;\C{ previous chunk allocated (if
any) }\6
\&{tmp\_var} ${}{*}\\{cell}[\\{cells\_per\_chunk}];{}$\2\6
${}\}{}$ \&{chunk};\par
\fi

\M[204 sat0.w]{7}\B\X3:Global variables\X${}\mathrel+\E{}$\6
\&{char} ${}{*}\\{buf}{}$;\C{ buffer for reading the lines (clauses) of \PB{%
\\{stdin}} }\6
\&{tmp\_var} ${}{*}{*}\\{hash}{}$;\C{ heads of the hash lists }\6
\&{uint} \\{hash\_bits}[\T{93}][\T{8}];\C{ random bits for universal hash
function }\6
\&{vchunk} ${}{*}\\{cur\_vchunk}{}$;\C{ the vchunk currently being filled }\6
\&{tmp\_var} ${}{*}\\{cur\_tmp\_var}{}$;\C{ current place to create new \PB{%
\&{tmp\_var}} entries }\6
\&{tmp\_var} ${}{*}\\{bad\_tmp\_var}{}$;\C{ the \PB{\\{cur\_tmp\_var}} when we
need a new \PB{\&{vchunk}} }\6
\&{chunk} ${}{*}\\{cur\_chunk}{}$;\C{ the chunk currently being filled }\6
\&{tmp\_var} ${}{*}{*}\\{cur\_cell}{}$;\C{ current place to create new elements
of a clause }\6
\&{tmp\_var} ${}{*}{*}\\{bad\_cell}{}$;\C{ the \PB{\\{cur\_cell}} when we need
a new \PB{\&{chunk}} }\6
\&{ullng} \\{vars};\C{ how many distinct variables have we seen? }\6
\&{ullng} \\{clauses};\C{ how many clauses have we seen? }\6
\&{ullng} \\{nullclauses};\C{ how many of them were null? }\6
\&{ullng} \\{cells};\C{ how many occurrences of literals in clauses? }\par
\fi

\M[219 sat0.w]{8}\B\X8:Initialize everything\X${}\E{}$\6
\\{gb\_init\_rand}(\\{random\_seed});\6
${}\\{buf}\K{}$(\&{char} ${}{*}){}$ \\{malloc}${}(\\{buf\_size}*\&{sizeof}(%
\&{char}));{}$\6
\&{if} ${}(\R\\{buf}){}$\5
${}\{{}$\1\6
${}\\{fprintf}(\\{stderr},\39\.{"Couldn't\ allocate\ t}\)\.{he\ input\ buffer\
(buf}\)\.{\_size=\%d)!\\n"},\39\\{buf\_size});{}$\6
${}\\{exit}({-}\T{2});{}$\6
\4${}\}{}$\2\6
${}\\{hash}\K{}$(\&{tmp\_var} ${}{*}{*}){}$ \\{malloc}${}(\&{sizeof}(\&{tmp%
\_var})\LL\\{hbits});{}$\6
\&{if} ${}(\R\\{hash}){}$\5
${}\{{}$\1\6
${}\\{fprintf}(\\{stderr},\39\.{"Couldn't\ allocate\ \%}\)\.{d\ hash\ list\
heads\ (h}\)\.{bits=\%d)!\\n"},\39\T{1}\LL\\{hbits},\39\\{hbits});{}$\6
${}\\{exit}({-}\T{3});{}$\6
\4${}\}{}$\2\6
\&{for} ${}(\|h\K\T{0};{}$ ${}\|h<\T{1}\LL\\{hbits};{}$ ${}\|h\PP){}$\1\5
${}\\{hash}[\|h]\K\NULL{}$;\2\par
\A14.
\U2.\fi

\M[235 sat0.w]{9}The hash address of each variable name has $h$ bits, where $h$
is the
value of the adjustable parameter \PB{\\{hbits}}.
Thus the average number of variables per hash list is $n/2^h$ when there
are $n$ different variables. A warning is printed if this average number
exceeds 10. (For example, if $h$ has its default value, 8, the program will
suggest that you might want to increase $h$ if your input has 2560
different variables or more.)

All the hashing takes place at the very beginning,
and the hash tables are actually recycled before any SAT-solving takes place;
therefore the setting of this parameter is by no means crucial. But I didn't
want to bother with fancy coding that would determine $h$ automatically.

\Y\B\4\X9:Input the clauses\X${}\E{}$\6
\&{while} (\T{1})\5
${}\{{}$\1\6
\&{if} ${}(\R\\{fgets}(\\{buf},\39\\{buf\_size},\39\\{stdin})){}$\1\5
\&{break};\2\6
${}\\{clauses}\PP;{}$\6
\&{if} ${}(\\{buf}[\\{strlen}(\\{buf})-\T{1}]\I\.{'\\n'}){}$\5
${}\{{}$\1\6
${}\\{fprintf}(\\{stderr},\39\.{"The\ clause\ on\ line\ }\)\.{\%lld\ (%
\%.20s...)\ is\ t}\)\.{oo\ long\ for\ me;\\n"},\39\\{clauses},\39\\{buf});{}$\6
${}\\{fprintf}(\\{stderr},\39\.{"\ my\ buf\_size\ is\ onl}\)\.{y\ \%d!\\n"},\39%
\\{buf\_size});{}$\6
${}\\{fprintf}(\\{stderr},\39\.{"Please\ use\ the\ comm}\)\.{and-line\ option\
b<ne}\)\.{wsize>.\\n"});{}$\6
${}\\{exit}({-}\T{4});{}$\6
\4${}\}{}$\2\6
\X10:Input the clause in \PB{\\{buf}}\X;\6
\4${}\}{}$\2\6
\&{if} ${}((\\{vars}\GG\\{hbits})\G\T{10}){}$\5
${}\{{}$\1\6
${}\\{fprintf}(\\{stderr},\39\.{"There\ are\ \%lld\ vari}\)\.{ables\ but\ only\
\%d\ ha}\)\.{sh\ tables;\\n"},\39\\{vars},\39\T{1}\LL\\{hbits});{}$\6
\&{while} ${}((\\{vars}\GG\\{hbits})\G\T{10}){}$\1\5
${}\\{hbits}\PP;{}$\2\6
${}\\{fprintf}(\\{stderr},\39\.{"\ maybe\ you\ should\ u}\)\.{se\ command-line\
opti}\)\.{on\ h\%d?\\n"},\39\\{hbits});{}$\6
\4${}\}{}$\2\6
${}\\{clauses}\MRL{-{\K}}\\{nullclauses};{}$\6
\&{if} ${}(\\{clauses}\E\T{0}){}$\5
${}\{{}$\1\6
${}\\{fprintf}(\\{stderr},\39\.{"No\ clauses\ were\ inp}\)\.{ut!\\n"});{}$\6
${}\\{exit}({-}\T{77});{}$\6
\4${}\}{}$\2\6
\&{if} ${}(\\{vars}\G\T{\^80000000}){}$\5
${}\{{}$\1\6
${}\\{fprintf}(\\{stderr},\39\.{"Whoa,\ the\ input\ had}\)\.{\ \%llu\
variables!\\n"},\39\\{cells});{}$\6
${}\\{exit}({-}\T{664});{}$\6
\4${}\}{}$\2\6
\&{if} ${}(\\{clauses}\G\T{\^80000000}){}$\5
${}\{{}$\1\6
${}\\{fprintf}(\\{stderr},\39\.{"Whoa,\ the\ input\ had}\)\.{\ \%llu\ clauses!%
\\n"},\39\\{cells});{}$\6
${}\\{exit}({-}\T{665});{}$\6
\4${}\}{}$\2\6
\&{if} ${}(\\{cells}\G\T{\^100000000}){}$\5
${}\{{}$\1\6
${}\\{fprintf}(\\{stderr},\39\.{"Whoa,\ the\ input\ had}\)\.{\ \%llu\
occurrences\ of}\)\.{\ literals!\\n"},\39\\{cells});{}$\6
${}\\{exit}({-}\T{666});{}$\6
\4${}\}{}$\2\par
\U2.\fi

\M[285 sat0.w]{10}\B\X10:Input the clause in \PB{\\{buf}}\X${}\E{}$\6
\&{for} ${}(\|j\K\|k\K\T{0};{}$  ; \,)\5
${}\{{}$\1\6
\&{while} ${}(\\{buf}[\|j]\E\.{'\ '}){}$\1\5
${}\|j\PP{}$;\C{ scan to nonblank }\2\6
\&{if} ${}(\\{buf}[\|j]\E\.{'\\n'}){}$\1\5
\&{break};\2\6
\&{if} ${}(\\{buf}[\|j]<\.{'\ '}\V\\{buf}[\|j]>\.{'\~'}){}$\5
${}\{{}$\1\6
${}\\{fprintf}(\\{stderr},\39\.{"Illegal\ character\ (}\)\.{code\ \#\%x)\ in\
the\ cla}\)\.{use\ on\ line\ \%lld!\\n"},\39\\{buf}[\|j],\39\\{clauses});{}$\6
${}\\{exit}({-}\T{5});{}$\6
\4${}\}{}$\2\6
\&{if} ${}(\\{buf}[\|j]\E\.{'\~'}){}$\1\5
${}\|i\K\T{1},\39\|j\PP;{}$\2\6
\&{else}\1\5
${}\|i\K\T{0};{}$\2\6
\X11:Scan and record a variable; negate it if \PB{$\|i\E\T{1}$}\X;\6
\4${}\}{}$\2\6
\&{if} ${}(\|k\E\T{0}){}$\5
${}\{{}$\1\6
${}\\{fprintf}(\\{stderr},\39\.{"(Empty\ line\ \%lld\ is}\)\.{\ being\ ignored)%
\\n"},\39\\{clauses});{}$\6
${}\\{nullclauses}\PP{}$;\C{ strictly speaking it would be unsatisfiable }\6
\4${}\}{}$\2\6
\&{goto} \\{clause\_done};\6
\4\\{empty\_clause}:\5
\X18:Remove all variables of the current clause\X;\6
\4\\{clause\_done}:\5
${}\\{cells}\MRL{+{\K}}\|k{}$;\par
\U9.\fi

\M[306 sat0.w]{11}We need a hack to insert the bit codes 1 and/or 2 into a
pointer value.

\Y\B\4\D$\\{hack\_in}(\|q,\|t)$ \5
(\&{tmp\_var} ${}{*})(\|t\OR{}$(\&{ullng}) \|q)\par
\Y\B\4\X11:Scan and record a variable; negate it if \PB{$\|i\E\T{1}$}\X${}\E{}$%
\6
${}\{{}$\1\6
\&{register} \&{tmp\_var} ${}{*}\|p;{}$\7
\&{if} ${}(\\{cur\_tmp\_var}\E\\{bad\_tmp\_var}){}$\1\5
\X12:Install a new \PB{\&{vchunk}}\X;\2\6
\X15:Put the variable name beginning at \PB{\\{buf}[\|j]} in \PB{$\\{cur\_tmp%
\_var}\MG\\{name}$} and compute its hash code \PB{\|h}\X;\6
\X16:Find \PB{$\\{cur\_tmp\_var}\MG\\{name}$} in the hash table at \PB{\|p}\X;\6
\&{if} ${}(\|p\MG\\{stamp}\E\\{clauses}\V\|p\MG\\{stamp}\E{-}\\{clauses}){}$\1\5
\X17:Handle a duplicate literal\X\2\6
\&{else}\5
${}\{{}$\1\6
${}\|p\MG\\{stamp}\K(\|i\?{-}\\{clauses}:\\{clauses});{}$\6
\&{if} ${}(\\{cur\_cell}\E\\{bad\_cell}){}$\1\5
\X13:Install a new \PB{\&{chunk}}\X;\2\6
${}{*}\\{cur\_cell}\K\|p;{}$\6
\&{if} ${}(\|i\E\T{1}){}$\1\5
${}{*}\\{cur\_cell}\K\\{hack\_in}({*}\\{cur\_cell},\39\T{1});{}$\2\6
\&{if} ${}(\|k\E\T{0}){}$\1\5
${}{*}\\{cur\_cell}\K\\{hack\_in}({*}\\{cur\_cell},\39\T{2});{}$\2\6
${}\\{cur\_cell}\PP,\39\|k\PP;{}$\6
\4${}\}{}$\2\6
\4${}\}{}$\2\par
\U10.\fi

\M[328 sat0.w]{12}\B\X12:Install a new \PB{\&{vchunk}}\X${}\E{}$\6
${}\{{}$\1\6
\&{register} \&{vchunk} ${}{*}\\{new\_vchunk};{}$\7
${}\\{new\_vchunk}\K{}$(\&{vchunk} ${}{*}){}$ \\{malloc}(\&{sizeof}(%
\&{vchunk}));\6
\&{if} ${}(\R\\{new\_vchunk}){}$\5
${}\{{}$\1\6
${}\\{fprintf}(\\{stderr},\39\.{"Can't\ allocate\ a\ ne}\)\.{w\ vchunk!%
\\n"});{}$\6
${}\\{exit}({-}\T{6});{}$\6
\4${}\}{}$\2\6
${}\\{new\_vchunk}\MG\\{prev}\K\\{cur\_vchunk},\39\\{cur\_vchunk}\K\\{new%
\_vchunk};{}$\6
${}\\{cur\_tmp\_var}\K{\AND}\\{new\_vchunk}\MG\\{var}[\T{0}];{}$\6
${}\\{bad\_tmp\_var}\K{\AND}\\{new\_vchunk}\MG\\{var}[\\{vars\_per%
\_vchunk}];{}$\6
\4${}\}{}$\2\par
\U11.\fi

\M[341 sat0.w]{13}\B\X13:Install a new \PB{\&{chunk}}\X${}\E{}$\6
${}\{{}$\1\6
\&{register} \&{chunk} ${}{*}\\{new\_chunk};{}$\7
${}\\{new\_chunk}\K{}$(\&{chunk} ${}{*}){}$ \\{malloc}(\&{sizeof}(\&{chunk}));\6
\&{if} ${}(\R\\{new\_chunk}){}$\5
${}\{{}$\1\6
${}\\{fprintf}(\\{stderr},\39\.{"Can't\ allocate\ a\ ne}\)\.{w\ chunk!%
\\n"});{}$\6
${}\\{exit}({-}\T{7});{}$\6
\4${}\}{}$\2\6
${}\\{new\_chunk}\MG\\{prev}\K\\{cur\_chunk},\39\\{cur\_chunk}\K\\{new%
\_chunk};{}$\6
${}\\{cur\_cell}\K{\AND}\\{new\_chunk}\MG\\{cell}[\T{0}];{}$\6
${}\\{bad\_cell}\K{\AND}\\{new\_chunk}\MG\\{cell}[\\{cells\_per\_chunk}];{}$\6
\4${}\}{}$\2\par
\U11.\fi

\M[354 sat0.w]{14}The hash code is computed via ``universal hashing,'' using
the following
precomputed tables of random bits.

\Y\B\4\X8:Initialize everything\X${}\mathrel+\E{}$\6
\&{for} ${}(\|j\K\T{92};{}$ \|j; ${}\|j\MM){}$\1\6
\&{for} ${}(\|k\K\T{0};{}$ ${}\|k<\T{8};{}$ ${}\|k\PP){}$\1\5
${}\\{hash\_bits}[\|j][\|k]\K\\{gb\_next\_rand}(\,){}$;\2\2\par
\fi

\M[361 sat0.w]{15}\B\X15:Put the variable name beginning at \PB{\\{buf}[\|j]}
in \PB{$\\{cur\_tmp\_var}\MG\\{name}$} and compute its hash code \PB{\|h}\X${}%
\E{}$\6
$\\{cur\_tmp\_var}\MG\\{name}.\\{lng}\K\T{0};{}$\6
\&{for} ${}(\|h\K\|l\K\T{0};{}$ ${}\\{buf}[\|j+\|l]>\.{'\ '}\W\\{buf}[\|j+\|l]%
\Z\.{'\~'};{}$ ${}\|l\PP){}$\5
${}\{{}$\1\6
\&{if} ${}(\|l>\T{7}){}$\5
${}\{{}$\1\6
${}\\{fprintf}(\\{stderr},\39\.{"Variable\ name\ \%.9s.}\)\.{..\ in\ the\
clause\ on\ }\)\.{line\ \%lld\ is\ too\ lon}\)\.{g!\\n"},\39\\{buf}+\|j,\39%
\\{clauses});{}$\6
${}\\{exit}({-}\T{8});{}$\6
\4${}\}{}$\2\6
${}\|h\MRL{{\XOR}{\K}}\\{hash\_bits}[\\{buf}[\|j+\|l]-\.{'!'}][\|l];{}$\6
${}\\{cur\_tmp\_var}\MG\\{name}.\\{ch8}[\|l]\K\\{buf}[\|j+\|l];{}$\6
\4${}\}{}$\2\6
\&{if} ${}(\|l\E\T{0}){}$\1\5
\&{goto} \\{empty\_clause};\C{ `\.\~' by itself is like `true' }\2\6
${}\|j\MRL{+{\K}}\|l;{}$\6
${}\|h\MRL{\AND{\K}}(\T{1}\LL\\{hbits})-\T{1}{}$;\par
\U11.\fi

\M[377 sat0.w]{16}\B\X16:Find \PB{$\\{cur\_tmp\_var}\MG\\{name}$} in the hash
table at \PB{\|p}\X${}\E{}$\6
\&{for} ${}(\|p\K\\{hash}[\|h];{}$ \|p; ${}\|p\K\|p\MG\\{next}){}$\1\6
\&{if} ${}(\|p\MG\\{name}.\\{lng}\E\\{cur\_tmp\_var}\MG\\{name}.\\{lng}){}$\1\5
\&{break};\2\2\6
\&{if} ${}(\R\|p){}$\5
${}\{{}$\C{ new variable found }\1\6
${}\|p\K\\{cur\_tmp\_var}\PP;{}$\6
${}\|p\MG\\{next}\K\\{hash}[\|h],\39\\{hash}[\|h]\K\|p;{}$\6
${}\|p\MG\\{serial}\K\\{vars}\PP;{}$\6
${}\|p\MG\\{stamp}\K\T{0};{}$\6
\4${}\}{}$\2\par
\U11.\fi

\M[387 sat0.w]{17}The most interesting aspect of the input phase is probably
the ``unwinding''
that we might need to do when encountering a literal more than once
in the same clause.

\Y\B\4\X17:Handle a duplicate literal\X${}\E{}$\6
${}\{{}$\1\6
\&{if} ${}((\|p\MG\\{stamp}>\T{0})\E(\|i>\T{0})){}$\1\5
\&{goto} \\{empty\_clause};\2\6
\4${}\}{}$\2\par
\U11.\fi

\M[396 sat0.w]{18}An input line that begins with `\.{\~\ }' is silently treated
as a comment.
Otherwise redundant clauses are logged, in case they were unintentional.
(One can, however, intentionally
use redundant clauses to force the order of the variables.)

\Y\B\4\X18:Remove all variables of the current clause\X${}\E{}$\6
\&{while} (\|k)\5
${}\{{}$\1\6
\X19:Move \PB{\\{cur\_cell}} backward to the previous cell\X;\6
${}\|k\MM;{}$\6
\4${}\}{}$\2\6
\&{if} ${}((\\{buf}[\T{0}]\I\.{'\~'})\V(\\{buf}[\T{1}]\I\.{'\ '})){}$\1\5
${}\\{fprintf}(\\{stderr},\39\.{"(The\ clause\ on\ line}\)\.{\ \%lld\ is\
always\ sati}\)\.{sfied)\\n"},\39\\{clauses});{}$\2\6
${}\\{nullclauses}\PP{}$;\par
\U10.\fi

\M[410 sat0.w]{19}\B\X19:Move \PB{\\{cur\_cell}} backward to the previous cell%
\X${}\E{}$\6
\&{if} ${}(\\{cur\_cell}>{\AND}\\{cur\_chunk}\MG\\{cell}[\T{0}]){}$\1\5
${}\\{cur\_cell}\MM;{}$\2\6
\&{else}\5
${}\{{}$\1\6
\&{register} \&{chunk} ${}{*}\\{old\_chunk}\K\\{cur\_chunk};{}$\7
${}\\{cur\_chunk}\K\\{old\_chunk}\MG\\{prev}{}$;\5
\\{free}(\\{old\_chunk});\6
${}\\{bad\_cell}\K{\AND}\\{cur\_chunk}\MG\\{cell}[\\{cells\_per\_chunk}];{}$\6
${}\\{cur\_cell}\K\\{bad\_cell}-\T{1};{}$\6
\4${}\}{}$\2\par
\Us18\ET33.\fi

\M[419 sat0.w]{20}\B\X20:Move \PB{\\{cur\_tmp\_var}} backward to the previous
temporary variable\X${}\E{}$\6
\&{if} ${}(\\{cur\_tmp\_var}>{\AND}\\{cur\_vchunk}\MG\\{var}[\T{0}]){}$\1\5
${}\\{cur\_tmp\_var}\MM;{}$\2\6
\&{else}\5
${}\{{}$\1\6
\&{register} \&{vchunk} ${}{*}\\{old\_vchunk}\K\\{cur\_vchunk};{}$\7
${}\\{cur\_vchunk}\K\\{old\_vchunk}\MG\\{prev}{}$;\5
\\{free}(\\{old\_vchunk});\6
${}\\{bad\_tmp\_var}\K{\AND}\\{cur\_vchunk}\MG\\{var}[\\{vars\_per%
\_vchunk}];{}$\6
${}\\{cur\_tmp\_var}\K\\{bad\_tmp\_var}-\T{1};{}$\6
\4${}\}{}$\2\par
\U37.\fi

\M[428 sat0.w]{21}\B\X21:Report the successful completion of the input phase%
\X${}\E{}$\6
$\\{fprintf}(\\{stderr},\39\.{"(\%lld\ variables,\ \%l}\)\.{ld\ clauses,\ \%llu%
\ lit}\)\.{erals\ successfully\ r}\)\.{ead)\\n"},\39\\{vars},\39\\{clauses},\39%
\\{cells}){}$;\par
\U2.\fi

\N[433 sat0.w]{1}{22}SAT solving, version 0. OK, now comes my hypothetical
recreation of
a 1960s-style SAT-solver. I knew about doubly linked lists, way back then;
but I hadn't yet realized the power of ``dancing links.'' This program
does invoke a mild form of the dancing-links principle, because I
think I probably would have discovered it if I'd actually worked
on satisfiability in those days. (The slightly modified program
{\mc SAT0-NODANCE} shows what my method would have been if I hadn't
foreseen dancing links so early.)

The algorithm below
essentially tries to solve a satisfiability problem on $n$
variables by first setting the $n$th variable to its most plausible value,
then using the same idea recursively on the remaining $(n-1)$-variable
problem. If this doesn't work, we try the other possibility for
the $n$th variable, and the result will either succeed or fail.

Data structures to support that method should allow us to do the
following things easily:
\smallskip
\item{$\bullet$}Know, for each variable, the clauses in which
that variable occurs, and in how many of them it occurs positively
or negatively (two counts).
\item{$\bullet$}Know, for each clause, the literals that it currently
contains.
\item{$\bullet$}Delete literals from clauses when they don't satisfy it.
\item{$\bullet$}Delete clauses that have already been satisfied.
\item{$\bullet$}Insert deleted literals and/or clauses when
backing up to reconsider previous choices.
\smallskip\noindent
The original clause sizes are known in advance. Therefore we can use a
combination of sequential and linked memory to accomplish all of these goals.

\fi

\M[465 sat0.w]{23}The basic unit in our data structure is called a cell.
There's one
cell for each literal in each clause, and there also are $2n$ special
cells explained below. Each cell belongs to a doubly linked
list for the corresponding literal, as well as to a sequential list
for the relevant clause. It also ``knows'' the number of its clause
and the number of its literal (which is $2k$ or $2k+1$ for the
positive and negative versions of variable~$k$).

Each link is a 32-bit integer. (I don't use \CEE/ pointers in the main
data structures, because they occupy 64 bits and clutter up the caches.)
The integer is an index into a monolithic array of cells called \PB{\\{mem}}.

\Y\B\4\D$\\{listsize}$ \5
\\{owner}\C{ alternative name for the \PB{\\{owner}} field }\par
\Y\B\4\X5:Type definitions\X${}\mathrel+\E{}$\6
\&{typedef} \&{struct} ${}\{{}$\1\6
\&{uint} \\{flink}${},{}$ \\{blink};\C{ forward and backward links for this
literal }\6
\&{uint} \\{owner};\C{ clause number, or size in the special list-head cells }\6
\&{uint} \\{litno};\C{ literal number }\2\6
${}\}{}$ \&{cell};\par
\fi

\M[486 sat0.w]{24}Each clause is represented by a pointer to its first cell and
by its
current size. My first draft of this program also included links to the
preceding and following clauses, in a doubly linked cyclic list
of all the active clauses that are currently active; but later I realized
that such a list is irrelevant, so it might as well be immaterial.

\Y\B\4\X5:Type definitions\X${}\mathrel+\E{}$\6
\&{typedef} \&{struct} ${}\{{}$\1\6
\&{uint} \\{start};\C{ the address in \PB{\\{mem}} where the cells for this
clause start }\6
\&{uint} \\{size};\C{ the current number of literals in this clause }\2\6
${}\}{}$ \&{clause};\par
\fi

\M[498 sat0.w]{25}If there are $n$ variables, there are $2n$ possible literals.
Hence we
reserve $2n$ special cells at the beginning of \PB{\\{mem}}, for the heads of
the lists that link all occurrences of the same literal together.

(Added later: Well, I now actually reserve $2n+2$ special cells, in order
to be consistent with the exposition in {\sl TAOCP}, where it was found
``friendlier'' to speak of $x_1$ through~$x_n$ instead of $x_0$ through
$x_{n-1}$ in the introductory examples.)

The lists for variable $k$ begin in locations $2k$ and $2k+1$, corresponding to
its positive and negative incarnations, for $1\le k\le n$. The \PB{\\{owner}}
field
in the list head tells the total size of the list.

A variable is also represented by its name, for purposes of output.
The name appears in a separate array \PB{\\{vmem}} of vertex nodes.

\Y\B\4\X5:Type definitions\X${}\mathrel+\E{}$\6
\&{typedef} \&{struct} ${}\{{}$\1\6
\&{octa} \\{name};\C{ the variable's symbolic name }\2\6
${}\}{}$ \&{variable};\par
\fi

\M[519 sat0.w]{26}\B\X3:Global variables\X${}\mathrel+\E{}$\6
\&{cell} ${}{*}\\{mem}{}$;\C{ the master array of cells }\6
\&{uint} \\{nonspec};\C{ address in \PB{\\{mem}} of the first non-special cell
}\6
\&{clause} ${}{*}\\{cmem}{}$;\C{ the master array of clauses }\6
\&{variable} ${}{*}\\{vmem}{}$;\C{ the master array of variables }\6
\&{char} ${}{*}\\{move}{}$;\C{ the stack of choices made so far }\6
\&{uint} \\{active};\C{ the total number of active clauses }\par
\fi

\M[527 sat0.w]{27}Here is a subroutine that prints a clause symbolically. It
illustrates
some of the conventions of the data structures that have been explained above.
I use it only for debugging.

Incidentally, the clause numbers reported to the user after the input phase
may differ from the line numbers reported during the input phase,
when \PB{$\\{nullclauses}>\T{0}$}.

\Y\B\4\X27:Subroutines\X${}\E{}$\6
\&{void} \\{print\_clause}(\&{uint} \|c)\1\1\2\2\6
${}\{{}$\1\6
\&{register} \&{uint} \|k${},{}$ \|l;\7
${}\\{printf}(\.{"\%d:"},\39\|c){}$;\C{ show the clause number }\6
\&{for} ${}(\|k\K\T{0};{}$ ${}\|k<\\{cmem}[\|c].\\{size};{}$ ${}\|k\PP){}$\5
${}\{{}$\1\6
${}\|l\K\\{mem}[\\{cmem}[\|c].\\{start}+\|k].\\{litno};{}$\6
${}\\{printf}(\.{"\ \%s\%.8s"},\39\|l\AND\T{1}\?\.{"\~"}:\.{""},\39\\{vmem}[\|l%
\GG\T{1}].\\{name}.\\{ch8}){}$;\C{ $k$th literal }\6
\4${}\}{}$\2\6
\\{printf}(\.{"\\n"});\6
\4${}\}{}$\2\par
\As28\ET29.
\U2.\fi

\M[546 sat0.w]{28}Similarly we can print out all of the clauses that use (or
originally used)
a particular literal.

\Y\B\4\X27:Subroutines\X${}\mathrel+\E{}$\6
\&{void} \\{print\_clauses\_for}(\&{int} \|l)\1\1\2\2\6
${}\{{}$\1\6
\&{register} \&{uint} \|p;\7
\&{for} ${}(\|p\K\\{mem}[\|l].\\{flink};{}$ ${}\|p\G\\{nonspec};{}$ ${}\|p\K%
\\{mem}[\|p].\\{flink}){}$\1\5
${}\\{print\_clause}(\\{mem}[\|p].\\{owner});{}$\2\6
\4${}\}{}$\2\par
\fi

\M[556 sat0.w]{29}In long runs it's helpful to know how far we've gotten.

\Y\B\4\X27:Subroutines\X${}\mathrel+\E{}$\6
\&{void} \\{print\_state}(\&{int} \|l)\1\1\2\2\6
${}\{{}$\1\6
\&{register} \&{int} \|k;\7
${}\\{fprintf}(\\{stderr},\39\.{"\ after\ \%lld\ mems:"},\39\\{mems});{}$\6
\&{for} ${}(\|k\K\T{1};{}$ ${}\|k\Z\|l;{}$ ${}\|k\PP){}$\1\5
${}\\{fprintf}(\\{stderr},\39\.{"\%c"},\39\\{move}[\|k]+\.{'0'});{}$\2\6
${}\\{fprintf}(\\{stderr},\39\.{"\\n"});{}$\6
\\{fflush}(\\{stderr});\6
\4${}\}{}$\2\par
\fi

\N[567 sat0.w]{1}{30}Initializing the real data structures.
Okay, we're ready now to convert the temporary chunks of data into the
form we want, and to recycle those chunks. The code below is intended to be
a prototype for similar tasks in later programs of this series.

\Y\B\4\X30:Set up the main data structures\X${}\E{}$\6
\X31:Allocate the main arrays\X;\6
\X32:Copy all the temporary cells to the \PB{\\{mem}} and \PB{\\{cmem}} arrays
in proper format\X;\6
\X37:Copy all the temporary variable nodes to the \PB{\\{vmem}} array in proper
format\X;\6
\X38:Check consistency\X;\par
\U2.\fi

\M[579 sat0.w]{31}The backtracking routine uses a small array called \PB{%
\\{move}} to record
its choices-so-far. We don't count the space for \PB{\\{move}} in \PB{%
\\{bytes}}, because
each \PB{\&{variable}} entry has a spare byte that could have been used.

\Y\B\4\X31:Allocate the main arrays\X${}\E{}$\6
\\{free}(\\{buf});\5
\\{free}(\\{hash});\C{ a tiny gesture to make a little room }\6
${}\\{nonspec}\K\\{vars}+\\{vars}+\T{2};{}$\6
\&{if} ${}(\\{nonspec}+\\{cells}\G\T{\^100000000}){}$\5
${}\{{}$\1\6
${}\\{fprintf}(\\{stderr},\39\.{"Whoa,\ nonspec+cells}\)\.{\ is\ too\ big\ for\
me!\\}\)\.{n"});{}$\6
${}\\{exit}({-}\T{667});{}$\6
\4${}\}{}$\2\6
${}\\{mem}\K{}$(\&{cell} ${}{*}){}$ \\{malloc}${}((\\{nonspec}+\\{cells})*%
\&{sizeof}(\&{cell}));{}$\6
\&{if} ${}(\R\\{mem}){}$\5
${}\{{}$\1\6
${}\\{fprintf}(\\{stderr},\39\.{"Oops,\ I\ can't\ alloc}\)\.{ate\ the\ big\ mem%
\ arra}\)\.{y!\\n"});{}$\6
${}\\{exit}({-}\T{10});{}$\6
\4${}\}{}$\2\6
${}\\{bytes}\K(\\{nonspec}+\\{cells})*\&{sizeof}(\&{cell});{}$\6
${}\\{cmem}\K{}$(\&{clause} ${}{*}){}$ \\{malloc}${}((\\{clauses}+\T{1})*%
\&{sizeof}(\&{clause}));{}$\6
\&{if} ${}(\R\\{cmem}){}$\5
${}\{{}$\1\6
${}\\{fprintf}(\\{stderr},\39\.{"Oops,\ I\ can't\ alloc}\)\.{ate\ the\ cmem\
array!\\}\)\.{n"});{}$\6
${}\\{exit}({-}\T{11});{}$\6
\4${}\}{}$\2\6
${}\\{bytes}\MRL{+{\K}}(\\{clauses}+\T{1})*\&{sizeof}(\&{clause});{}$\6
${}\\{vmem}\K{}$(\&{variable} ${}{*}){}$ \\{malloc}${}((\\{vars}+\T{1})*%
\&{sizeof}(\&{variable}));{}$\6
\&{if} ${}(\R\\{vmem}){}$\5
${}\{{}$\1\6
${}\\{fprintf}(\\{stderr},\39\.{"Oops,\ I\ can't\ alloc}\)\.{ate\ the\ vmem\
array!\\}\)\.{n"});{}$\6
${}\\{exit}({-}\T{12});{}$\6
\4${}\}{}$\2\6
${}\\{bytes}\MRL{+{\K}}(\\{vars}+\T{1})*\&{sizeof}(\&{variable});{}$\6
${}\\{move}\K{}$(\&{char} ${}{*}){}$ \\{malloc}${}((\\{vars}+\T{1})*\&{sizeof}(%
\&{char}));{}$\6
\&{if} ${}(\R\\{move}){}$\5
${}\{{}$\1\6
${}\\{fprintf}(\\{stderr},\39\.{"Oops,\ I\ can't\ alloc}\)\.{ate\ the\ move\
array!\\}\)\.{n"});{}$\6
${}\\{exit}({-}\T{13});{}$\6
\4${}\}{}$\2\par
\U30.\fi

\M[614 sat0.w]{32}\B\X32:Copy all the temporary cells to the \PB{\\{mem}} and %
\PB{\\{cmem}} arrays in proper format\X${}\E{}$\6
\&{for} ${}(\|j\K\T{0};{}$ ${}\|j<\\{nonspec};{}$ ${}\|j\PP){}$\1\5
${}\|o,\39\\{mem}[\|j].\\{flink}\K\T{0};{}$\2\6
\&{for} ${}(\|c\K\\{clauses};{}$ \|c; ${}\|c\MM){}$\5
${}\{{}$\1\6
${}\|o,\39\\{cmem}[\|c].\\{start}\K\|j,\39\\{cmem}[\|c].\\{size}\K\T{0};{}$\6
\X33:Insert the cells for the literals of clause \PB{\|c}\X;\6
\4${}\}{}$\2\6
${}\\{active}\K\\{clauses};{}$\6
\X34:Fix up the \PB{\\{blink}} fields and compute the list sizes\X;\6
\X35:Sort the literals within each clause\X;\par
\U30.\fi

\M[624 sat0.w]{33}The basic idea is to ``unwind'' the steps that we went
through while
building up the chunks.

\Y\B\4\D$\\{hack\_out}(\|q)$ \5
(((\&{ullng}) \|q)${}\AND\T{\^3}{}$)\par
\B\4\D$\\{hack\_clean}(\|q)$ \5
((\&{tmp\_var} ${}{*})({}$(\&{ullng}) \|q${}\AND{-}\T{4}){}$)\par
\Y\B\4\X33:Insert the cells for the literals of clause \PB{\|c}\X${}\E{}$\6
\&{for} ${}(\|i\K\T{0};{}$ ${}\|i<\T{2};{}$ ${}\|j\PP){}$\5
${}\{{}$\1\6
\X19:Move \PB{\\{cur\_cell}} backward to the previous cell\X;\6
${}\|i\K\\{hack\_out}({*}\\{cur\_cell});{}$\6
${}\|p\K\\{hack\_clean}({*}\\{cur\_cell})\MG\\{serial};{}$\6
${}\|p\MRL{+{\K}}\|p+(\|i\AND\T{1})+\T{2};{}$\6
${}\\{ooo},\39\\{mem}[\|j].\\{flink}\K\\{mem}[\|p].\\{flink},\39\\{mem}[\|p].%
\\{flink}\K\|j;{}$\6
${}\|o,\39\\{mem}[\|j].\\{owner}\K\|c,\39\\{mem}[\|j].\\{litno}\K\|p;{}$\6
\4${}\}{}$\2\par
\U32.\fi

\M[640 sat0.w]{34}\B\X34:Fix up the \PB{\\{blink}} fields and compute the list
sizes\X${}\E{}$\6
\&{for} ${}(\|k\K\T{2};{}$ ${}\|k<\\{nonspec};{}$ ${}\|k\PP){}$\5
${}\{{}$\1\6
\&{for} ${}(\|o,\39\|i\K\T{0},\39\|q\K\|k,\39\|p\K\\{mem}[\|k].\\{flink};{}$
${}\|p\G\\{nonspec};{}$ ${}\|i\PP,\39\|q\K\|p,\39\|o,\39\|p\K\\{mem}[\|p].%
\\{flink}){}$\1\5
${}\|o,\39\\{mem}[\|p].\\{blink}\K\|q;{}$\2\6
${}\\{oo},\39\\{mem}[\|k].\\{blink}\K\|q,\39\\{mem}[\|q].\\{flink}\K\|k;{}$\6
${}\|o,\39\\{mem}[\|k].\\{listsize}\K\|i,\39\\{mem}[\|k].\\{litno}\K\|k;{}$\6
\4${}\}{}$\2\par
\U32.\fi

\M[648 sat0.w]{35}The backtracking scheme we will use works nicely when the
literals
of a clause are arranged so that the first one to be tried comes last.
Then a false literal is removed from its clause simply by decreasing
the clause's \PB{\\{size}} field.

This program tries variable 0 first, then variable 1, etc.; so we want the
literals of each clause to be in decreasing order.

The following sorting scheme takes linear time, in the number of cells,
because of the characteristics of our data structures. The \PB{\\{size}} field
of each clause is initially zero.

\Y\B\4\X35:Sort the literals within each clause\X${}\E{}$\6
\&{for} ${}(\|k\K\\{nonspec}-\T{1};{}$ ${}\|k\G\T{2};{}$ ${}\|k\MM){}$\1\6
\&{for} ${}(\|o,\39\|j\K\\{mem}[\|k].\\{flink};{}$ ${}\|j\G\\{nonspec};{}$ %
\|o${},\39\|j\K\\{mem}[\|j].\\{flink}){}$\5
${}\{{}$\1\6
${}\|o,\39\|c\K\\{mem}[\|j].\\{owner};{}$\6
${}\|o,\39\|i\K\\{cmem}[\|c].\\{size},\39\|p\K\\{cmem}[\|c].\\{start}+\|i;{}$\6
\&{if} ${}(\|j\I\|p){}$\1\5
\X36:Swap cell \PB{\|j} with cell \PB{\|p}\X;\2\6
${}\|o,\39\\{cmem}[\|c].\\{size}\K\|i+\T{1};{}$\6
\4${}\}{}$\2\2\par
\U32.\fi

\M[670 sat0.w]{36}Sometimes doubly linked lists make me feel good, even when
spending 11 mems.
(For mem computation, \PB{\\{flink}} and \PB{\\{blink}} are assumed to be
stored in
a single 64-bit word.)

\Y\B\4\X36:Swap cell \PB{\|j} with cell \PB{\|p}\X${}\E{}$\6
${}\{{}$\1\6
${}\|o,\39\|q\K\\{mem}[\|p].\\{flink},\39\|r\K\\{mem}[\|p].\\{blink};{}$\6
${}\\{oo},\39\\{mem}[\|p].\\{flink}\K\\{mem}[\|j].\\{flink},\39\\{mem}[\|p].%
\\{blink}\K\\{mem}[\|j].\\{blink};{}$\6
${}\\{oo},\39\\{mem}[\\{mem}[\|j].\\{flink}].\\{blink}\K\\{mem}[\\{mem}[\|j].%
\\{blink}].\\{flink}\K\|p;{}$\6
${}\|o,\39\\{mem}[\|j].\\{flink}\K\|q,\39\\{mem}[\|j].\\{blink}\K\|r;{}$\6
${}\\{oo},\39\\{mem}[\|q].\\{blink}\K\|j,\39\\{mem}[\|r].\\{flink}\K\|j;{}$\6
${}\\{oo},\39\\{mem}[\|j].\\{litno}\K\\{mem}[\|p].\\{litno};{}$\6
${}\|o,\39\\{mem}[\|p].\\{litno}\K\|k;{}$\6
${}\|j\K\|p;{}$\6
\4${}\}{}$\2\par
\U35.\fi

\M[686 sat0.w]{37}\B\X37:Copy all the temporary variable nodes to the \PB{%
\\{vmem}} array in proper format\X${}\E{}$\6
\&{for} ${}(\|c\K\\{vars};{}$ \|c; ${}\|c\MM){}$\5
${}\{{}$\1\6
\X20:Move \PB{\\{cur\_tmp\_var}} backward to the previous temporary variable\X;%
\6
${}\|o,\39\\{vmem}[\|c].\\{name}.\\{lng}\K\\{cur\_tmp\_var}\MG\\{name}.%
\\{lng};{}$\6
\4${}\}{}$\2\par
\U30.\fi

\M[692 sat0.w]{38}We should now have unwound all the temporary data chunks back
to their
beginnings.

\Y\B\4\X38:Check consistency\X${}\E{}$\6
\&{if} ${}(\\{cur\_cell}\I{\AND}\\{cur\_chunk}\MG\&{cell}[\T{0}]\V\\{cur%
\_chunk}\MG\\{prev}\I\NULL\V\\{cur\_tmp\_var}\I{\AND}\\{cur\_vchunk}\MG\\{var}[%
\T{0}]\V\\{cur\_vchunk}\MG\\{prev}\I\NULL){}$\5
${}\{{}$\1\6
${}\\{fprintf}(\\{stderr},\39\.{"This\ can't\ happen\ (}\)\.{consistency\ check%
\ fa}\)\.{ilure)!\\n"});{}$\6
${}\\{exit}({-}\T{14});{}$\6
\4${}\}{}$\2\6
\\{free}(\\{cur\_chunk});\5
\\{free}(\\{cur\_vchunk});\par
\U30.\fi

\N[705 sat0.w]{1}{39}Doing it. Now comes ye olde basic backtrack.

A choice is recorded in the \PB{\\{move}} array, as the number 0 or 1 if we're
trying first to set the current variable true or false, respectively;
it is 3 or 2 if that move failed and we're trying the other alternative.

One slightly nontrivial point arises here: If we can satisfy all of the
remaining clauses at some level of the computation, we can do it
with our first choice for the relevant variable, because the other
choice satisfies at most as many clauses. Thus the test for success
doesn't need to be redone at label \PB{\\{try\_again}} below.

Furthermore, if the complement of our first choice doesn't appear in any
active clause, we need not try it. (In other words, a ``pure literal'' can be
assumed to be true. I~didn't know about pure literals when I first wrote {\mc
SAT0}, but I've put that knowledge into Algorithm 7.2.2.2A.)
Such cases are encoded as moves 4 and~5 instead of 0 and~1.

\Y\B\4\X39:Solve the problem\X${}\E{}$\6
$\\{level}\K\T{1}{}$;\C{ I used to start at level 0, but Algorithm 7.2.2.2A
does this }\6
\4\\{newlevel}:\5
${}\\{ooo},\39\\{move}[\\{level}]\K(\\{mem}[\\{level}+\\{level}].\\{listsize}\Z%
\\{mem}[\\{level}+\\{level}+\T{1}].\\{listsize});{}$\6
\&{if} ${}((\\{verbose}\AND\\{show\_choices})\W\\{level}\Z\\{show\_choices%
\_max}){}$\5
${}\{{}$\1\6
${}\\{fprintf}(\\{stderr},\39\.{"Level\ \%d,\ trying\ \%s}\)\.{\%.8s"},\39%
\\{level},\39\\{move}[\\{level}]\?\.{"\~"}:\.{""},\39\\{vmem}[\\{level}].%
\\{name}.\\{ch8});{}$\6
\&{if} ${}(\\{verbose}\AND\\{show\_details}){}$\1\5
${}\\{fprintf}(\\{stderr},\39\.{"\ (\%d:\%d,\ \%d\ active,}\)\.{\ \%lld\
mems)"},\39\\{mem}[\\{level}+\\{level}].\\{listsize},\39\\{mem}[\\{level}+%
\\{level}+\T{1}].\\{listsize},\39\\{active},\39\\{mems});{}$\2\6
${}\\{fprintf}(\\{stderr},\39\.{"\\n"});{}$\6
\4${}\}{}$\2\6
${}\\{parity}\K\\{move}[\\{level}]\AND\T{1};{}$\6
\&{if} ${}(\\{mem}[\\{level}+\\{level}+\T{1}-\\{parity}].\\{listsize}\E%
\T{0}){}$\1\5
${}\\{move}[\\{level}]\MRL{+{\K}}\T{4}{}$;\C{ pure literal; see above }\2\6
\&{else}\1\5
${}\\{nodes}\PP;{}$\2\6
\&{if} ${}(\\{delta}\W(\\{mems}\G\\{thresh})){}$\1\5
${}\\{thresh}\MRL{+{\K}}\\{delta},\39\\{print\_state}(\\{level});{}$\2\6
\&{if} ${}(\\{active}\E\\{mem}[\\{level}+\\{level}+\\{parity}].\\{listsize}){}$%
\1\5
\&{goto} \\{satisfied};\C{ success! }\2\6
\&{if} ${}(\\{mems}>\\{timeout}){}$\5
${}\{{}$\1\6
${}\\{fprintf}(\\{stderr},\39\.{"TIMEOUT!\\n"});{}$\6
\&{goto} \\{done};\6
\4${}\}{}$\2\6
\4\\{tryit}:\5
${}\\{parity}\K\\{move}[\\{level}]\AND\T{1};{}$\6
\X40:Remove variable \PB{\\{level}} from the clauses in the non-chosen list; %
\PB{\&{goto} \\{try\_again}} if that would make a clause empty\X;\6
\X41:Inactivate all clauses of the chosen list\X;\6
${}\\{level}\PP{}$;\5
\&{goto} \\{newlevel};\6
\4\\{try\_again}:\5
\&{if} ${}(\|o,\39\\{move}[\\{level}]<\T{2}){}$\5
${}\{{}$\1\6
${}\|o,\39\\{move}[\\{level}]\K\T{3}-\\{move}[\\{level}];{}$\6
\&{if} ${}((\\{verbose}\AND\\{show\_choices})\W\\{level}\Z\\{show\_choices%
\_max}){}$\5
${}\{{}$\1\6
${}\\{fprintf}(\\{stderr},\39\.{"Level\ \%d,\ trying\ ag}\)\.{ain"},\39%
\\{level});{}$\6
\&{if} ${}(\\{verbose}\AND\\{show\_details}){}$\1\5
${}\\{fprintf}(\\{stderr},\39\.{"\ (\%d\ active,\ \%lld\ m}\)\.{ems)\\n"},\39%
\\{active},\39\\{mems});{}$\2\6
\&{else}\1\5
${}\\{fprintf}(\\{stderr},\39\.{"\\n"});{}$\2\6
\4${}\}{}$\2\6
\&{goto} \\{tryit};\6
\4${}\}{}$\2\6
\&{if} ${}(\\{level}>\T{1}){}$\1\5
\X42:Backtrack to the previous level\X;\2\6
\&{if} (\T{1})\5
${}\{{}$\1\6
\\{printf}(\.{"\~\\n"});\C{ the formula was unsatisfiable }\6
\&{if} ${}(\\{verbose}\AND\\{show\_basics}){}$\1\5
${}\\{fprintf}(\\{stderr},\39\.{"UNSAT\\n"});{}$\2\6
\4${}\}{}$\5
\2\&{else}\5
${}\{{}$\1\6
\4\\{satisfied}:\5
\&{if} ${}(\\{verbose}\AND\\{show\_basics}){}$\1\5
${}\\{fprintf}(\\{stderr},\39\.{"!SAT!\\n"});{}$\2\6
\X45:Print the solution found\X;\6
\4${}\}{}$\2\par
\U2.\fi

\M[769 sat0.w]{40}Here's where the fact that clauses are sorted really pays
off.

\Y\B\4\X40:Remove variable \PB{\\{level}} from the clauses in the non-chosen
list; \PB{\&{goto} \\{try\_again}} if that would make a clause empty\X${}\E{}$\6
\&{for} ${}(\|o,\39\|k\K\\{mem}[\\{level}+\\{level}+\T{1}-\\{parity}].%
\\{flink};{}$ ${}\|k\G\\{nonspec};{}$ \|o${},\39\|k\K\\{mem}[\|k].\\{flink}){}$%
\5
${}\{{}$\1\6
${}\\{oo},\39\|c\K\\{mem}[\|k].\\{owner},\39\|i\K\\{cmem}[\|c].\\{size};{}$\6
\&{if} ${}(\|i\E\T{1}){}$\5
${}\{{}$\1\6
\&{if} ${}(\\{verbose}\AND\\{show\_details}){}$\1\5
${}\\{fprintf}(\\{stderr},\39\.{"(Clause\ \%d\ now\ unsa}\)\.{tisfied)\\n"},\39%
\|c);{}$\2\6
\&{for} ${}(\|o,\39\|k\K\\{mem}[\|k].\\{blink};{}$ ${}\|k\G\\{nonspec};{}$ %
\|o${},\39\|k\K\\{mem}[\|k].\\{blink}){}$\5
${}\{{}$\1\6
${}\\{oo},\39\|c\K\\{mem}[\|k].\\{owner},\39\|i\K\\{cmem}[\|c].\\{size};{}$\6
${}\|o,\39\\{cmem}[\|c].\\{size}\K\|i+\T{1};{}$\6
\4${}\}{}$\2\6
\&{goto} \\{try\_again};\6
\4${}\}{}$\2\6
${}\|o,\39\\{cmem}[\|c].\\{size}\K\|i-\T{1};{}$\6
\4${}\}{}$\2\par
\U39.\fi

\M[786 sat0.w]{41}The links dance a little here.

\Y\B\4\X41:Inactivate all clauses of the chosen list\X${}\E{}$\6
\&{for} ${}(\|o,\39\|k\K\\{mem}[\\{level}+\\{level}+\\{parity}].\\{flink};{}$
${}\|k\G\\{nonspec};{}$ \|o${},\39\|k\K\\{mem}[\|k].\\{flink}){}$\5
${}\{{}$\1\6
${}\\{oo},\39\|c\K\\{mem}[\|k].\\{owner},\39\|i\K\\{cmem}[\|c].\\{size},\39\|j%
\K\\{cmem}[\|c].\\{start};{}$\6
\&{for} ${}(\|p\K\|j;{}$ ${}\|p<\|j+\|i-\T{1};{}$ ${}\|p\PP){}$\5
${}\{{}$\1\6
${}\|o,\39\|q\K\\{mem}[\|p].\\{flink},\39\|r\K\\{mem}[\|p].\\{blink};{}$\6
${}\\{oo},\39\\{mem}[\|q].\\{blink}\K\|r,\39\\{mem}[\|r].\\{flink}\K\|q;{}$\6
${}\\{ooo},\39\\{mem}[\\{mem}[\|p].\\{litno}].\\{listsize}\MM;{}$\6
\4${}\}{}$\2\6
\4${}\}{}$\2\6
${}\|o,\39\\{active}\MRL{-{\K}}\\{mem}[\|k].\\{listsize}{}$;\par
\U39.\fi

\M[799 sat0.w]{42}\B\X42:Backtrack to the previous level\X${}\E{}$\6
${}\{{}$\1\6
${}\\{level}\MM;{}$\6
\X43:Reactivate all clauses of the chosen list\X;\6
\X44:Put variable \PB{\\{level}} back into all clauses on the non-chosen list%
\X;\6
\&{goto} \\{try\_again};\6
\4${}\}{}$\2\par
\U39.\fi

\M[807 sat0.w]{43}Here the dancing links protocol requires us to traverse the
list
in the reverse direction from what we had before.

\Y\B\4\X43:Reactivate all clauses of the chosen list\X${}\E{}$\6
$\|o,\39\\{parity}\K\\{move}[\\{level}]\AND\T{1};{}$\6
\&{for} ${}(\|o,\39\|k\K\\{mem}[\\{level}+\\{level}+\\{parity}].\\{blink};{}$
${}\|k\G\\{nonspec};{}$ \|o${},\39\|k\K\\{mem}[\|k].\\{blink}){}$\5
${}\{{}$\1\6
${}\\{oo},\39\|c\K\\{mem}[\|k].\\{owner},\39\|i\K\\{cmem}[\|c].\\{size},\39\|j%
\K\\{cmem}[\|c].\\{start};{}$\6
\&{for} ${}(\|p\K\|j;{}$ ${}\|p<\|j+\|i-\T{1};{}$ ${}\|p\PP){}$\5
${}\{{}$\1\6
${}\|o,\39\|q\K\\{mem}[\|p].\\{flink},\39\|r\K\\{mem}[\|p].\\{blink};{}$\6
${}\\{oo},\39\\{mem}[\|q].\\{blink}\K\|p,\39\\{mem}[\|r].\\{flink}\K\|p;{}$\6
${}\\{ooo},\39\\{mem}[\\{mem}[\|p].\\{litno}].\\{listsize}\PP;{}$\6
\4${}\}{}$\2\6
\4${}\}{}$\2\6
${}\|o,\39\\{active}\MRL{+{\K}}\\{mem}[\|k].\\{listsize}{}$;\par
\U42.\fi

\M[822 sat0.w]{44}\B\X44:Put variable \PB{\\{level}} back into all clauses on
the non-chosen list\X${}\E{}$\6
\&{for} ${}(\|o,\39\|k\K\\{mem}[\\{level}+\\{level}+\T{1}-\\{parity}].%
\\{flink};{}$ ${}\|k\G\\{nonspec};{}$ \|o${},\39\|k\K\\{mem}[\|k].\\{flink}){}$%
\5
${}\{{}$\1\6
${}\\{oo},\39\|c\K\\{mem}[\|k].\\{owner},\39\|i\K\\{cmem}[\|c].\\{size};{}$\6
${}\|o,\39\\{cmem}[\|c].\\{size}\K\|i+\T{1};{}$\6
\4${}\}{}$\2\par
\U42.\fi

\M[828 sat0.w]{45}\B\X45:Print the solution found\X${}\E{}$\6
\&{for} ${}(\|k\K\T{1};{}$ ${}\|k\Z\\{level};{}$ ${}\|k\PP){}$\1\5
${}\\{printf}(\.{"\ \%s\%.8s"},\39\\{move}[\|k]\AND\T{1}\?\.{"\~"}:\.{""},\39%
\\{vmem}[\|k].\\{name}.\\{ch8});{}$\2\6
\\{printf}(\.{"\\n"});\par
\U39.\fi

\N[833 sat0.w]{1}{46}Index.
\fi

\inx
\fin
\con
