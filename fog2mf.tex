\input cwebmac
\datethis
\font\logo=logo10
\def\MF{{\logo META}\-{\logo FONT}}


\N{1}{1}Introduction. This program is a quick-and-dirty hack to convert
Fontographer Type~3 font output to \MF\ code. I assume that the input
file has been hand-converted to a mixture of the \.{afm} file and
the \.{ps} file output by Fontographer; I also assume that the output
file will be hand-tailored to make a complete \MF\ program.

More precisely, this program reads blocks of material like
$$\vbox{\halign{\tt#\cr
C 33 ; WX 220 ; N exclam ; B 34 442 187 664 ;\cr
/exclam\char`\{220 0 4.0426 422.858 217.66 683.983 Cache\cr
187.702 623.194 moveto\cr
-0.297607 -29.4158 -111.106 -124.124 -24 -28 rrcurveto\cr
-19 10 rlineto\cr
16.42 45.8054 35.7433 110.536 23 36 rrcurveto\cr
8.12329 12.7366 11.0454 6.84058 15.4363 -0.159424 rrcurveto\cr
23.3605 -0.159424 21.7024 -18.0087 -0.297607 -23.8059 rrcurveto\cr
closepath\cr
0 FillStroke\cr
\char`\}def\cr
}}$$
and writes corresponding blocks of material like
$$\vbox{\halign{\tt#\cr
beginchar(33,220u\char`#,664u\char`#,0u\char`#);\cr
stroke (188,623)\cr
\ ...(187,594,76,470,52,442)\cr
\ --(33,452)\cr
\ ...(50,497,85,608,108,644)\cr
\ ...(117,657,128,664,143,663)\cr
\ ...(166,663,188,645,188,621)\cr
\ --cycle;\cr
endchar;\cr
}}$$
(operating from standard input to standard output).

It does absolutely nothing fancy.

On a closed shape like the letter O, the user has to change some \.{stroke}
commands to \.{unstroke}, because Fontographer gives the outside contour and
then the inside contour (in opposite directions). The inside contour needs
to be erased, not filled, so we want to \.{unstroke} it.

\Y\B\8\#\&{include} \.{<stdio.h>}\6
\&{char} \\{buffer}[\T{100}]${},{}$ ${}{*}\\{pos}\K\\{buffer};{}$\6
\&{char} \\{token}[\T{100}];\7
\X2:Subroutines for input\X\7
\\{main}(\,)\1\1\2\2\6
${}\{{}$\1\6
\&{register} \&{int} \|j${},{}$ \|k;\6
\&{double} \|x${},{}$ \|y${},{}$ \|z;\6
\&{register} \&{char} ${}{*}\|p,{}$ ${}{*}\|q;{}$\7
\&{while} (\T{1})\5
${}\{{}$\1\6
\X4:Process font metric info\X;\6
\X5:Process stroke info\X;\6
\4${}\}{}$\2\6
\4${}\}{}$\2\par
\fi

\N{1}{2}Low-level input. At the bottom I need a way to parse the input into
tokens.
A token is either a number or a string of nonblank characters.

To make things simple, \PB{\\{get\_token}} just finds a string of nonblank
characters. The calling routine will easily be able to convert a numeric
string to the actual number.

\Y\B\4\D$\\{get\_token}$ \5
\\{gtok}(\,)\par
\Y\B\4\X2:Subroutines for input\X${}\E{}$\6
\\{gtok}(\,)\1\1\2\2\6
${}\{{}$\1\6
\&{register} \&{char} ${}{*}\|p;{}$\7
\&{if} ${}({*}\\{pos}\E\T{0}\V{*}\\{pos}\E\.{'\\n'}){}$\5
${}\{{}$\1\6
\&{if} ${}(\R\\{fgets}(\\{buffer},\39\T{100},\39\\{stdin})){}$\1\5
\\{exit}(\T{0});\C{ normal exit at end of file }\2\6
${}\\{pos}\K\\{buffer};{}$\6
\4${}\}{}$\2\6
\&{for} ( ; ${}{*}\\{pos}\E\.{'\ '};{}$ ${}\\{pos}\PP){}$\1\5
;\C{ move to next nonspace }\2\6
\&{for} ${}(\|p\K\\{token};{}$ ${}{*}\\{pos}\I\.{'\ '}\W{*}\\{pos}\W{*}\\{pos}%
\I\.{'\\n'};{}$ ${}\|p\PP,\39\\{pos}\PP){}$\1\5
${}{*}\|p\K{*}\\{pos};{}$\2\6
${}{*}\|p\K\T{0};{}$\6
\&{for} ( ; ${}{*}\\{pos}\E\.{'\ '};{}$ ${}\\{pos}\PP){}$\1\5
;\2\6
\4${}\}{}$\2\par
\A3.
\U1.\fi

\M{3}If the input contains any surprises, we give up immediately.

\Y\B\4\D$\\{get\_num}$ \5
\\{gnum}(\,)\par
\B\4\D$\\{panic}(\\{str})$ \6
${}\{{}$\1\6
${}\\{fprintf}(\\{stderr},\39\.{"Oops!\ \%s:\\n\%s"},\39\\{str},\39%
\\{buffer});{}$\6
${}\\{exit}({-}\T{1});{}$\6
\4${}\}{}$\2\par
\Y\B\4\X2:Subroutines for input\X${}\mathrel+\E{}$\6
\&{double} \\{gnum}(\,)\1\1\2\2\6
${}\{{}$\1\6
\&{double} \\{xx};\7
\&{if} ${}(\\{sscanf}(\\{token},\39\.{"\%lf"},\39{\AND}\\{xx})\I\T{1}){}$\1\5
\\{panic}(\.{"Unreadable\ number"});\2\6
\&{return} \\{xx};\6
\4${}\}{}$\2\par
\fi

\N{1}{4}Reading the font metrics. If the first line of the input is, say,
$$\.{C 36 ; WX 482 ; N dollar ; B 23 -205 437 751 ;}$$
we want to define character number 36, whose width is 482 units.
The name of the character is unimportant (Fontographer assigned it
based solely of the character number). The bounding box is also
mostly unimportant except for the $y$ coordinates; in this example
we give the character a depth of 205 units and a height of 751.

Another line such as
$$\.{/dollar\char`\{482 0 -44.1428 -260.8 504.143 806.8 Cache}$$
immediately follows in the input, but we totally ignore it.

\Y\B\4\D$\\{check}(\\{str},\\{err})$ \6
${}\{{}$\1\6
\\{get\_token};\6
\&{if} ${}(\\{strcmp}(\\{token},\39\\{str})\I\T{0}){}$\1\5
\\{panic}(\\{err});\2\6
\4${}\}{}$\2\par
\Y\B\4\X4:Process font metric info\X${}\E{}$\6
$\\{check}(\.{"C"},\39\.{"Expected\ `C'"});{}$\6
\\{get\_token};\6
${}\\{printf}(\.{"beginchar(\%s,"},\39\\{token});{}$\6
${}\\{check}(\.{";"},\39\.{"Expected\ `;'"});{}$\6
${}\\{check}(\.{"WX"},\39\.{"Expected\ `WX'"})\\{get\_token};{}$\6
${}\\{printf}(\.{"\%su\#,"},\39\\{token});{}$\6
${}\\{check}(\.{";"},\39\.{"Expected\ `;'"});{}$\6
${}\\{check}(\.{"N"},\39\.{"Expected\ `N'"});{}$\6
\\{get\_token};\6
${}\\{check}(\.{";"},\39\.{"Expected\ `;'"});{}$\6
${}\\{check}(\.{"B"},\39\.{"Expected\ `B'"});{}$\6
\\{get\_token};\6
\\{get\_token};\6
${}\|k\K(\&{int})(\\{get\_num}+\T{.5});{}$\6
\&{if} ${}(\|k>\T{0}){}$\1\5
${}\|k\K\T{0};{}$\2\6
\&{else}\1\5
${}\|k\K{-}\|k;{}$\2\6
\\{get\_token};\6
\\{get\_token};\6
${}\\{printf}(\.{"\%su\#,\%du\#);\\n"},\39\\{token},\39\|k);{}$\6
${}\\{check}(\.{";"},\39\.{"Expected\ `;'"});{}$\6
\\{get\_token};\6
${}\\{check}(\.{"0"},\39\.{"Expected\ `0'"});{}$\6
\\{get\_token};\6
\\{get\_token};\6
\\{get\_token};\6
\\{get\_token};\6
${}\\{check}(\.{"Cache"},\39\.{"Expected\ `Cache'"}){}$;\par
\U1.\fi

\N{1}{5}The strokes. Each shape to be filled is presented as a sequence
of lines beginning with `$x$ $y$ \.{moveto}' and followed by lines
that say either `$x$ $y$ \.{rlineto}' or `$x_1$ $y_1$ $x_2$ $y_2$ $x_3$ $y_3$
\.{rrcurveto}'; finally `\.{closepath}' ends the shape.
Each pair $(x,y)$ is an increment to be added to the previous coordinates.

The final stroke is followed by `\.{0} \.{Fillstroke} \.{\char`\}def}'.

\Y\B\4\X5:Process stroke info\X${}\E{}$\6
\&{while} (\T{1})\5
${}\{{}$\1\6
\\{get\_token};\6
${}\|x\K\\{get\_num};{}$\6
\\{get\_token};\6
\&{if} ${}(\\{strcmp}(\\{token},\39\.{"FillStroke"})\E\T{0}){}$\1\5
\&{break};\2\6
${}\|y\K\\{get\_num};{}$\6
${}\\{check}(\.{"moveto"},\39\.{"Expected\ `moveto'"});{}$\6
${}\\{printf}(\.{"stroke\ (\%d,\%d)\\n"},\39(\&{int})(\|x+\T{.5}),\39(\&{int})(%
\|y+\T{.5}));{}$\6
\&{while} (\T{1})\5
${}\{{}$\1\6
\\{get\_token};\6
\&{if} ${}(\\{strcmp}(\\{token},\39\.{"closepath"})\E\T{0}){}$\1\5
\&{break};\2\6
${}\|x\MRL{+{\K}}\\{get\_num};{}$\6
\\{get\_token};\6
${}\|y\MRL{+{\K}}\\{get\_num};{}$\6
\\{get\_token};\6
\&{if} ${}(\\{strcmp}(\\{token},\39\.{"rlineto"})\E\T{0}){}$\1\5
${}\\{printf}(\.{"\ --(\%d,\%d)\\n"},\39(\&{int})(\|x+\T{.5}),\39(\&{int})(\|y+%
\T{.5}));{}$\2\6
\&{else}\5
${}\{{}$\1\6
${}\\{printf}(\.{"\ ...(\%d,\%d"},\39(\&{int})(\|x+\T{.5}),\39(\&{int})(\|y+%
\T{.5}));{}$\6
${}\|x\MRL{+{\K}}\\{get\_num};{}$\6
\\{get\_token};\6
${}\|y\MRL{+{\K}}\\{get\_num};{}$\6
${}\\{printf}(\.{",\%d,\%d"},\39(\&{int})(\|x+\T{.5}),\39(\&{int})(\|y+%
\T{.5}));{}$\6
\\{get\_token};\6
${}\|x\MRL{+{\K}}\\{get\_num};{}$\6
\\{get\_token};\6
${}\|y\MRL{+{\K}}\\{get\_num};{}$\6
${}\\{printf}(\.{",\%d,\%d)\\n"},\39(\&{int})(\|x+\T{.5}),\39(\&{int})(\|y+%
\T{.5}));{}$\6
${}\\{check}(\.{"rrcurveto"},\39\.{"Expected\ `rrcurveto}\)\.{'"});{}$\6
\4${}\}{}$\2\6
\4${}\}{}$\2\6
;\6
\\{printf}(\.{"\ --cycle;\\n"});\6
\4${}\}{}$\2\6
\\{printf}(\.{"endchar;\\n"});\6
${}\\{check}(\.{"\}def"},\39\.{"Expected\ `\}def'"}){}$;\par
\U1.\fi

\N{1}{6}Index.
\fi

\inx
\fin
\con
