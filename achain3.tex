\input cwebmac
\datethis

\N{1}{1}Intro. This program is a sequel to {\mc ACHAIN2}, which you should
read first. I'm experimenting with a brand-new way to find shortest
addition chains. Maybe it will be good, maybe not; but in either
case the results should be interesting (at least to me).
At the end of this program I shall discuss the observed running time.

The new idea is to generalize the problem to $l_k(n)$, the minimum length
of an addition chain for which $a_j=2^j$ for $0\le j\le k$,
assuming that $n\ge2^k$. Clearly $l_0(n)=l_1(n)=l(n)$ is the ordinary
function, and we have $l_k(n)\le l_{k+1}(n)$. Furthermore the
dual of the binary method (exercise 4.6.3--34) shows that
$l_k(n)\le\lfloor\lg n\rfloor+\nu n-1$. A slightly less obvious fact is
the inequality $l_{k+1}(2n)\le l_k(n)+1$; because if
1, 2, \dots, $2^k$, $a_{k+1}$, \dots,~$n$ is an addition chain, so is
1, 2, \dots, $2^k$, $2^{k+1}$, $2a_{k+1}$, \dots,~$2n$.

When I first thought of defining $l_k(n)$, I conjectured that
$l_{k+1}(n)\le l_k(n)+1$; but I'm tending to believe this less and less,
the more I think about it. If it fails, we would have
$l_{k+1}(n)>l_{k+1}(2n)$, by the previous inequality; but addition
chains are full of surprises.

Two parameters are given on the given line. If they are \.{foo} and
\.{bar}, this program reads from files \.{foo-1}, \.{foo-2}, etc.,
and writes to files \.{bar-1}, \.{bar-2}, etc., with bytes of
the $k$th file giving values of $f_k(n)$ for $n=2^k$, $2^k+1$, $2^k+2$, etc.
An input file that doesn't exist, or that is too short to contain information
about the number $n$ being worked on, is simply disregarded; but if
data is present in an input file, it is believed to be true without
further checking.

\Y\B\4\D$\\{nmax}$ \5
$(\T{1}\LL\T{20}{}$)\C{ should be less than $2^{24}$ on a 32-bit machine }\par
\Y\B\8\#\&{include} \.{<stdio.h>}\6
\8\#\&{include} \.{<stdlib.h>}\6
\8\#\&{include} \.{<time.h>}\6
\&{char} \|l[\T{20}][\\{nmax}];\6
\&{int} \|a[\T{128}]${},{}$ \|b[\T{128}];\6
\&{unsigned} \&{int} ${}\\{undo}[\T{128}*\T{128}];{}$\6
\&{int} \\{ptr};\C{ this many items of the \PB{\\{undo}} stack are in use }\6
\&{struct} ${}\{{}$\1\6
\&{int} \\{lbp}${},{}$ \\{lbq}${},{}$ \\{ubq}${},{}$ \|r${},{}$ \\{ptrp}${},{}$
\\{ptrq};\2\6
${}\}{}$ \\{stack}[\T{128}];\6
\&{int} \\{tail}[\T{128}]${},{}$ \\{outdeg}[\T{128}]${},{}$ \\{outsum}[%
\T{128}]${},{}$ \\{limit}[\T{128}];\6
\&{FILE} ${}{*}\\{infile}[\T{64}],{}$ ${}{*}\\{outfile}[\T{64}];{}$\6
\&{char} \\{buf}[\T{100}];\7
\&{int} \\{main}(\&{int} \\{argc}${},\39{}$\&{char} ${}{*}\\{argv}[\,]){}$\1\1%
\2\2\6
${}\{{}$\1\6
\&{register} \&{int} \|i${},{}$ \|j${},{}$ \|n${},{}$ \|p${},{}$ \|q${},{}$ %
\|r${},{}$ \|s${},{}$ \\{ubq}${},{}$ \\{lbp}${},{}$ \\{lbq}${},{}$ %
\\{ptrp}${},{}$ \\{ptrq};\6
\&{int} \\{lg2n}${},{}$ \\{kk}${},{}$ \\{lb}${},{}$ \\{ub}${},{}$ \\{timer}${}%
\K\T{0};{}$\7
\X2:Process the command line\X;\6
${}\|a[\T{0}]\K\|b[\T{0}]\K\T{1},\39\|a[\T{1}]\K\|b[\T{1}]\K\T{2}{}$;\C{ an
addition chain always begins like this }\6
\&{for} ${}(\|n\K\T{2};{}$ ${}\|n<\\{nmax};{}$ ${}\|n\PP){}$\5
${}\{{}$\1\6
\X5:Determine $\lfloor\lg n\rfloor$ and the binary upper bound\X;\6
\&{for} ${}(\\{kk}\K\\{lg2n};{}$ \\{kk}; ${}\\{kk}\MM){}$\5
${}\{{}$\1\6
\X4:Try to input $l_k(n)$; \PB{\&{goto} \\{done}} if successful\X;\6
\X6:Backtrack until $l_k(n)$ is known\X;\6
\4\\{done}:\5
\X3:Output the value of $l_k(n)$\X;\6
\4${}\}{}$\2\6
\&{if} ${}(\|n\MOD\T{1000}\E\T{0}){}$\5
${}\{{}$\1\6
${}\|j\K\\{clock}(\,);{}$\6
${}\\{printf}(\.{"\%d..\%d\ done\ in\ \%.5g}\)\.{\ minutes\\n"},\39\|n-\T{999},%
\39\|n,\39(\&{double})(\|j-\\{timer})/(\T{60}*\.{CLOCKS\_PER\_SEC}));{}$\6
${}\\{timer}\K\|j;{}$\6
\4${}\}{}$\2\6
\4${}\}{}$\2\6
\4${}\}{}$\2\par
\fi

\M{2}\B\X2:Process the command line\X${}\E{}$\6
\&{if} ${}(\\{argc}\I\T{3}){}$\5
${}\{{}$\1\6
${}\\{fprintf}(\\{stderr},\39\.{"Usage:\ \%s\ foo\ bar\\n}\)\.{"},\39\\{argv}[%
\T{0}]);{}$\6
${}\\{exit}({-}\T{1});{}$\6
\4${}\}{}$\2\par
\U1.\fi

\M{3}\B\X3:Output the value of $l_k(n)$\X${}\E{}$\6
\&{if} ${}(\R\\{outfile}[\\{kk}]){}$\5
${}\{{}$\1\6
${}\\{sprintf}(\\{buf},\39\.{"\%s-\%d"},\39\\{argv}[\T{2}],\39\\{kk});{}$\6
${}\\{outfile}[\\{kk}]\K\\{fopen}(\\{buf},\39\.{"w"});{}$\6
\&{if} ${}(\R\\{outfile}[\\{kk}]){}$\5
${}\{{}$\1\6
${}\\{fprintf}(\\{stderr},\39\.{"Can't\ open\ file\ `\%s}\)\.{'\ for\ writing!%
\\n"},\39\\{buf});{}$\6
${}\\{exit}({-}\T{2});{}$\6
\4${}\}{}$\2\6
\4${}\}{}$\2\6
${}\\{fprintf}(\\{outfile}[\\{kk}],\39\.{"\%c"},\39\|l[\\{kk}][\|n]+\.{'\
'});{}$\6
\\{fflush}(\\{outfile}[\\{kk}]);\C{ make sure the result is viewable
immediately }\par
\U1.\fi

\M{4}Note that the input file for $l_1(n)$ starts with $n=2$, not $n=1$ as
in the previous programs.

\Y\B\4\X4:Try to input $l_k(n)$; \PB{\&{goto} \\{done}} if successful\X${}\E{}$%
\6
\&{if} ${}(\R\\{infile}[\\{kk}]){}$\5
${}\{{}$\1\6
${}\\{sprintf}(\\{buf},\39\.{"\%s-\%d"},\39\\{argv}[\T{1}],\39\\{kk});{}$\6
${}\\{infile}[\\{kk}]\K\\{fopen}(\\{buf},\39\.{"r"});{}$\6
\&{if} ${}(\R\\{infile}[\\{kk}]){}$\1\5
${}\\{infile}[\\{kk}]\K{}$(\&{FILE} ${}{*}){}$ \T{1};\2\6
\4${}\}{}$\2\6
\&{if} ${}(\\{infile}[\\{kk}]\I{}$(\&{FILE} ${}{*}){}$ \T{1})\5
${}\{{}$\1\6
${}\|l[\\{kk}][\|n]\K\\{fgetc}(\\{infile}[\\{kk}])-\.{'\ '};{}$\6
\&{if} ${}(\|l[\\{kk}][\|n]\Z\T{0}){}$\1\5
${}\\{infile}[\\{kk}]\K{}$(\&{FILE} ${}{*}){}$ \T{1};\C{ shut down input when
something fails }\2\6
\&{goto} \\{done};\C{ accept the input value unquestioningly }\6
\4${}\}{}$\2\par
\U1.\fi

\M{5}\B\X5:Determine $\lfloor\lg n\rfloor$ and the binary upper bound\X${}\E{}$%
\6
\&{for} ${}(\|q\K\|n,\39\|i\K{-}\T{1},\39\|j\K\T{0};{}$ \|q; ${}\|q\MRL{{\GG}{%
\K}}\T{1},\39\|i\PP){}$\1\5
${}\|j\MRL{+{\K}}\|q\AND\T{1};{}$\2\6
${}\\{lg2n}\K\|i,\39\\{ub}\K\|i+\|j-\T{1}{}$;\par
\U1.\fi

\N{1}{6}The interesting part.
The canonical-chain reduction of {\mc ACHAIN2} works for $l_k(n)$ as
well as for $l(n)$, because the first $k$ steps of an $l_k$~chain
are always reduced in the digraph. So I've taken it over here without
change.

Well, there is one change: In the former method, I started with a lower
bound and worked upward until achieving success;
now I'm going to start at an upper-bound-less-1 and continue until
failing (as in {\mc ACHAIN0}). This switch causes only minor
modifications, in spite of what I believed when I wrote {\mc ACHAIN1}.

At the top level, when $k=\lfloor\lg n\rfloor$, there's nothing to do,
because \PB{\\{ub}} clearly contains the optimal value.
For smaller values of~$k$, we start at $l_{k+1}(n)-1$, and we
also set $b[k+1]\gets2^{k+1}-$, because we know that the value
$2^{k+1}$ has been ruled out.

\Y\B\4\X6:Backtrack until $l_k(n)$ is known\X${}\E{}$\6
\4\\{loop}:\5
${}\|l[\\{kk}][\|n]\K\\{ub};{}$\6
\&{if} ${}(\\{kk}\E\\{lg2n}){}$\1\5
\&{goto} \\{done};\2\6
${}\\{lb}\K\\{ub}-\T{1}{}$;\C{ \PB{\\{lb}} isn't really a lower bound, it's
just a holdover from {\mc ACHAIN2} }\6
\&{if} ${}(\\{lb}\Z\\{kk}+\T{1}){}$\1\5
\&{goto} \\{done};\2\6
\&{for} ${}(\|i\K\T{0};{}$ ${}\|i\Z\\{lb};{}$ ${}\|i\PP){}$\1\5
${}\\{outdeg}[\|i]\K\\{outsum}[\|i]\K\T{0};{}$\2\6
${}\|a[\\{lb}]\K\|b[\\{lb}]\K\|n;{}$\6
\&{for} ${}(\|i\K\T{2};{}$ ${}\|i\Z\\{kk};{}$ ${}\|i\PP){}$\1\5
${}\|a[\|i]\K\|b[\|i]\K\T{1}\LL\|i;{}$\2\6
${}\|a[\|i]\K\|a[\\{kk}]+\T{1},\39\|b[\|i]\K(\T{1}\LL\|i)-\T{1};{}$\6
\&{for} ${}(\|i\PP;{}$ ${}\|i<\\{lb};{}$ ${}\|i\PP){}$\1\5
${}\|a[\|i]\K\|a[\|i-\T{1}]+\T{1},\39\|b[\|i]\K\|b[\|i-\T{1}]\LL\T{1};{}$\2\6
\&{for} ${}(\|i\K\\{lb}-\T{1};{}$ ${}\|i>\\{kk};{}$ ${}\|i\MM){}$\5
${}\{{}$\1\6
\&{if} ${}((\|a[\|i]\LL\T{1})<\|a[\|i+\T{1}]){}$\1\5
${}\|a[\|i]\K(\|a[\|i+\T{1}]+\T{1})\GG\T{1};{}$\2\6
\&{if} ${}(\|b[\|i]\G\|b[\|i+\T{1}]){}$\1\5
${}\|b[\|i]\K\|b[\|i+\T{1}]-\T{1};{}$\2\6
\4${}\}{}$\2\6
\&{if} ${}(\|a[\\{lb}-\T{1}]>\|b[\\{lb}-\T{1}]){}$\1\5
\&{goto} \\{done};\2\6
\X7:Try to fix the rest of the chain; \PB{\&{goto} \\{done}} if it's impossible%
\X;\6
${}\\{ub}\K\\{lb};{}$\6
\&{goto} \\{loop};\par
\U1.\fi

\M{7}The only change to this algorithm for {\mc ACHAIN2} occurs when we happen
to encounter an empty slot (namely when \PB{$\\{outdeg}[\|s]\E\T{0}$} and \PB{%
\|s}
isn't the top level). Then we simply reject the current solution.
Reason: If it could be completed with the empty slot, that's great; but
we'll discover the fact later. Meanwhile there certainly are canonical
solutions with all slots nonempty, and they should be easy to find.

\Y\B\4\X7:Try to fix the rest of the chain; \PB{\&{goto} \\{done}} if it's
impossible\X${}\E{}$\6
$\\{ptr}\K\T{0}{}$;\C{ clear the \PB{\\{undo}} stack }\6
\&{for} ${}(\|r\K\|s\K\\{lb};{}$ ${}\|s>\\{kk};{}$ ${}\|s\MM){}$\5
${}\{{}$\1\6
\&{if} ${}(\\{outdeg}[\|s]\E\T{0}\W\|s<\\{lb}){}$\1\5
\&{goto} \\{backup};\2\6
\&{if} ${}(\\{outdeg}[\|s]\E\T{1}){}$\1\5
${}\\{limit}[\|s]\K\\{tail}[\\{outsum}[\|s]]{}$;\5
\2\&{else}\1\5
${}\\{limit}[\|s]\K\T{1};{}$\2\6
\&{for} ( ; ${}\|r>\T{1}\W\|a[\|r-\T{1}]\E\|b[\|r-\T{1}];{}$ ${}\|r\MM){}$\1\5
;\2\6
\&{if} ${}(\\{outdeg}[\|s-\T{1}]\E\T{0}\W(\|a[\|s]\AND\T{1})){}$\1\5
${}\|q\K\|a[\|s]/\T{3}{}$;\5
\2\&{else}\1\5
${}\|q\K\|a[\|s]\GG\T{1};{}$\2\6
\&{for} ${}(\|p\K\|a[\|s]-\|q;{}$ ${}\|p\Z\|b[\|s-\T{1}];{}$ \,)\5
${}\{{}$\1\6
\&{if} ${}(\|p>\|b[\|r-\T{1}]){}$\5
${}\{{}$\1\6
\&{while} ${}(\|p>\|a[\|r]){}$\1\5
${}\|r\PP{}$;\C{ this step keeps \PB{$\|r<\|s$} }\2\6
${}\|p\K\|a[\|r],\39\|q\K\|a[\|s]-\|p,\39\|r\PP;{}$\6
\4${}\}{}$\2\6
\&{if} ${}(\|q<\\{limit}[\|s]){}$\1\5
\&{goto} \\{backup};\2\6
\X10:Find bounds $(\PB{\\{lbp}},\PB{\\{ubq}})$ and $(\PB{\\{lbq}},\PB{%
\\{ubq}})$ on where \PB{\|p} and \PB{\|q} can be inserted; but go to \PB{%
\\{failpq}} if they can't both be accommodated\X;\6
${}\\{ptrp}\K\\{ptr};{}$\6
\&{for} ( ; ${}\\{ubq}\G\\{lbp};{}$ ${}\\{ubq}\MM){}$\5
${}\{{}$\1\6
\X12:Put \PB{\|p} into the chain at location \PB{\\{ubq}}; \PB{\&{goto} %
\\{failp}} if there's a problem\X;\6
\&{if} ${}(\|p\E\|q){}$\1\5
\&{goto} \\{happiness};\2\6
\&{if} ${}(\\{ubq}\G\\{ubq}){}$\1\5
${}\\{ubq}\K\\{ubq}-\T{1};{}$\2\6
${}\\{ptrq}\K\\{ptr};{}$\6
\&{for} ( ; ${}\\{ubq}\G\\{lbq};{}$ ${}\\{ubq}\MM){}$\5
${}\{{}$\1\6
\X14:Put \PB{\|q} into the chain at location \PB{\\{ubq}}; \PB{\&{goto} %
\\{failq}} if there's a problem\X;\6
\4\\{happiness}:\5
\X8:Put local variables on the stack and update outdegrees\X;\6
\&{goto} \\{onward};\C{ now \PB{\|a[\|s]} is covered; try to fill in \PB{$\|a[%
\|s-\T{1}]$} }\6
\4\\{backup}:\5
${}\|s\PP;{}$\6
\&{if} ${}(\|s>\\{lb}){}$\1\5
\&{goto} \\{done};\2\6
\X9:Restore local variables from the stack and downdate outdegrees\X;\6
\&{if} ${}(\|p\E\|q){}$\1\5
\&{goto} \\{failp};\2\6
\4\\{failq}:\5
\&{while} ${}(\\{ptr}>\\{ptrq}){}$\1\5
\X11:Undo a change\X;\2\6
\4${}\}{}$\2\6
\4\\{failp}:\5
\&{while} ${}(\\{ptr}>\\{ptrp}){}$\1\5
\X11:Undo a change\X;\2\6
\4${}\}{}$\2\6
\4\\{failpq}:\5
\&{if} ${}(\|p\E\|q){}$\5
${}\{{}$\1\6
\&{if} ${}(\\{outdeg}[\|s-\T{1}]\E\T{0}){}$\1\5
${}\|q\K\|a[\|s]/\T{3}+\T{1}{}$;\C{ will be decreased momentarily }\2\6
\&{if} ${}(\|q>\|b[\|s-\T{2}]){}$\1\5
${}\|q\K\|b[\|s-\T{2}];{}$\2\6
\&{else}\1\5
${}\|q\MM;{}$\2\6
${}\|p\K\|a[\|s]-\|q;{}$\6
\4${}\}{}$\5
\2\&{else}\1\5
${}\|p\PP,\39\|q\MM;{}$\2\6
\4${}\}{}$\2\6
\&{goto} \\{backup};\6
\4\\{onward}:\5
\&{continue};\6
\4${}\}{}$\2\6
\\{possible}:\par
\U6.\fi

\M{8}\B\X8:Put local variables on the stack and update outdegrees\X${}\E{}$\6
$\\{tail}[\|s]\K\|q,\39\\{stack}[\|s].\|r\K\|r;{}$\6
${}\\{outdeg}[\\{ubq}]\PP,\39\\{outsum}[\\{ubq}]\MRL{+{\K}}\|s;{}$\6
${}\\{outdeg}[\\{ubq}]\PP,\39\\{outsum}[\\{ubq}]\MRL{+{\K}}\|s;{}$\6
${}\\{stack}[\|s].\\{lbp}\K\\{lbp},\39\\{stack}[\|s].\\{ubq}\K\\{ubq};{}$\6
${}\\{stack}[\|s].\\{lbq}\K\\{lbq},\39\\{stack}[\|s].\\{ubq}\K\\{ubq};{}$\6
${}\\{stack}[\|s].\\{ptrp}\K\\{ptrp},\39\\{stack}[\|s].\\{ptrq}\K\\{ptrq}{}$;%
\par
\U7.\fi

\M{9}\B\X9:Restore local variables from the stack and downdate outdegrees\X${}%
\E{}$\6
$\\{ptrq}\K\\{stack}[\|s].\\{ptrq},\39\\{ptrp}\K\\{stack}[\|s].\\{ptrp};{}$\6
${}\\{lbq}\K\\{stack}[\|s].\\{lbq},\39\\{ubq}\K\\{stack}[\|s].\\{ubq};{}$\6
${}\\{lbp}\K\\{stack}[\|s].\\{lbp},\39\\{ubq}\K\\{stack}[\|s].\\{ubq};{}$\6
${}\\{outdeg}[\\{ubq}]\MM,\39\\{outsum}[\\{ubq}]\MRL{-{\K}}\|s;{}$\6
${}\\{outdeg}[\\{ubq}]\MM,\39\\{outsum}[\\{ubq}]\MRL{-{\K}}\|s;{}$\6
${}\|q\K\\{tail}[\|s],\39\|p\K\|a[\|s]-\|q,\39\|r\K\\{stack}[\|s].\|r{}$;\par
\U7.\fi

\M{10}After the test in this step is passed, we'll have \PB{$\\{ubq}>\\{ubq}$}
and \PB{$\\{lbp}>\\{lbq}$}.

\Y\B\4\X10:Find bounds $(\PB{\\{lbp}},\PB{\\{ubq}})$ and $(\PB{\\{lbq}},\PB{%
\\{ubq}})$ on where \PB{\|p} and \PB{\|q} can be inserted; but go to \PB{%
\\{failpq}} if they can't both be accommodated\X${}\E{}$\6
$\\{lbp}\K\|l[\\{kk}][\|p];{}$\6
\&{if} ${}(\\{lbp}\G\\{lb}){}$\1\5
\&{goto} \\{failpq};\2\6
\&{while} ${}(\|b[\\{lbp}]<\|p){}$\1\5
${}\\{lbp}\PP;{}$\2\6
\&{if} ${}(\|a[\\{lbp}]>\|p){}$\1\5
\&{goto} \\{failpq};\2\6
\&{for} ${}(\\{ubq}\K\\{lbp};{}$ ${}\|a[\\{ubq}+\T{1}]\Z\|p;{}$ ${}\\{ubq}%
\PP){}$\1\5
;\2\6
\&{if} ${}(\\{ubq}\E\|s-\T{1}){}$\1\5
${}\\{lbp}\K\\{ubq};{}$\2\6
\&{if} ${}(\|p\E\|q){}$\1\5
${}\\{lbq}\K\\{lbp},\39\\{ubq}\K\\{ubq};{}$\2\6
\&{else}\5
${}\{{}$\1\6
${}\\{lbq}\K\|l[\\{kk}][\|q];{}$\6
\&{if} ${}(\\{lbq}\G\\{ubq}){}$\1\5
\&{goto} \\{failpq};\2\6
\&{while} ${}(\|b[\\{lbq}]<\|q){}$\1\5
${}\\{lbq}\PP;{}$\2\6
\&{if} ${}(\\{lbq}\G\\{ubq}){}$\1\5
\&{goto} \\{failpq};\2\6
\&{if} ${}(\|a[\\{lbq}]>\|q){}$\1\5
\&{goto} \\{failpq};\2\6
\&{for} ${}(\\{ubq}\K\\{lbq};{}$ ${}\|a[\\{ubq}+\T{1}]\Z\|q\W\\{ubq}+\T{1}<%
\\{ubq};{}$ ${}\\{ubq}\PP){}$\1\5
;\2\6
\&{if} ${}(\\{lbp}\E\\{lbq}){}$\1\5
${}\\{lbp}\PP;{}$\2\6
\4${}\}{}$\2\par
\U7.\fi

\M{11}The undoing mechanism is very simple: When changing \PB{\|a[\|j]}, we
put \PB{$(\|j\LL\T{24})+\|x$} on the \PB{\\{undo}} stack, where \PB{\|x} was
the former value.
Similarly, when changing \PB{\|b[\|j]}, we stack the value \PB{$(\T{1}\LL%
\T{31})+(\|j\LL\T{24})+\|x$}.

\Y\B\4\D$\\{newa}(\|j,\|y)$ \5
$\\{undo}[\\{ptr}\PP]\K(\|j\LL\T{24})+\|a[\|j],\39\|a[\|j]\K{}$\|y\par
\B\4\D$\\{newb}(\|j,\|y)$ \5
$\\{undo}[\\{ptr}\PP]\K(\T{1}\LL\T{31})+(\|j\LL\T{24})+\|b[\|j],\39\|b[\|j]%
\K{}$\|y\par
\Y\B\4\X11:Undo a change\X${}\E{}$\6
${}\{{}$\1\6
${}\|i\K\\{undo}[\MM\\{ptr}];{}$\6
\&{if} ${}(\|i\G\T{0}){}$\1\5
${}\|a[\|i\GG\T{24}]\K\|i\AND\T{\^ffffff};{}$\2\6
\&{else}\1\5
${}\|b[(\|i\AND\T{\^3fffffff})\GG\T{24}]\K\|i\AND\T{\^ffffff};{}$\2\6
\4${}\}{}$\2\par
\U7.\fi

\M{12}At this point we know that $a[ubq]\le p\le b[ubq]$.

\Y\B\4\X12:Put \PB{\|p} into the chain at location \PB{\\{ubq}}; \PB{\&{goto} %
\\{failp}} if there's a problem\X${}\E{}$\6
\&{if} ${}(\|a[\\{ubq}]\I\|p){}$\5
${}\{{}$\1\6
${}\\{newa}(\\{ubq},\39\|p);{}$\6
\&{for} ${}(\|j\K\\{ubq}-\T{1};{}$ ${}(\|a[\|j]\LL\T{1})<\|a[\|j+\T{1}];{}$ ${}%
\|j\MM){}$\5
${}\{{}$\1\6
${}\|i\K(\|a[\|j+\T{1}]+\T{1})\GG\T{1};{}$\6
\&{if} ${}(\|i>\|b[\|j]){}$\1\5
\&{goto} \\{failp};\2\6
${}\\{newa}(\|j,\39\|i);{}$\6
\4${}\}{}$\2\6
\&{for} ${}(\|j\K\\{ubq}+\T{1};{}$ ${}\|a[\|j]\Z\|a[\|j-\T{1}];{}$ ${}\|j%
\PP){}$\5
${}\{{}$\1\6
${}\|i\K\|a[\|j-\T{1}]+\T{1};{}$\6
\&{if} ${}(\|i>\|b[\|j]){}$\1\5
\&{goto} \\{failp};\2\6
${}\\{newa}(\|j,\39\|i);{}$\6
\4${}\}{}$\2\6
\4${}\}{}$\2\6
\&{if} ${}(\|b[\\{ubq}]\I\|p){}$\5
${}\{{}$\1\6
${}\\{newb}(\\{ubq},\39\|p);{}$\6
\&{for} ${}(\|j\K\\{ubq}-\T{1};{}$ ${}\|b[\|j]\G\|b[\|j+\T{1}];{}$ ${}\|j%
\MM){}$\5
${}\{{}$\1\6
${}\|i\K\|b[\|j+\T{1}]-\T{1};{}$\6
\&{if} ${}(\|i<\|a[\|j]){}$\1\5
\&{goto} \\{failp};\2\6
${}\\{newb}(\|j,\39\|i);{}$\6
\4${}\}{}$\2\6
\&{for} ${}(\|j\K\\{ubq}+\T{1};{}$ ${}\|b[\|j]>\|b[\|j-\T{1}]\LL\T{1};{}$ ${}%
\|j\PP){}$\5
${}\{{}$\1\6
${}\|i\K\|b[\|j-\T{1}]\LL\T{1};{}$\6
\&{if} ${}(\|i<\|a[\|j]){}$\1\5
\&{goto} \\{failp};\2\6
${}\\{newb}(\|j,\39\|i);{}$\6
\4${}\}{}$\2\6
\4${}\}{}$\2\6
\X13:Make forced moves if \PB{\|p} has a special form\X;\par
\U7.\fi

\M{13}If, say, we've just set \PB{$\|a[\T{8}]\K\|b[\T{8}]\K\T{132}$}, special
considerations apply,
because the only addition chains of length~8 for 132 are
$$\eqalign{
&1,2,4,8,16,32,64,128,132;\cr
&1,2,4,8,16,32,64,68,132;\cr
&1,2,4,8,16,32,64,66,132;\cr
&1,2,4,8,16,32,34,66,132;\cr
&1,2,4,8,16,32,33,66,132;\cr
&1,2,4,8,16,17,33,66,132.\cr}$$
The values of \PB{\|a[\T{4}]} and \PB{\|b[\T{4}]} must therefore be 16; and
then, of course,
we also must have \PB{$\|a[\T{3}]\K\|b[\T{3}]\K\T{8}$}, etc. Similar reasoning
applies
whenever we set $a[j]=b[j]=2^j+2^k$ for $k\le j-4$.

Such cases may seem extremely special. But they are especially
useful in ruling out cases that have no good $l_k(n)$.

\Y\B\4\X13:Make forced moves if \PB{\|p} has a special form\X${}\E{}$\6
$\|i\K\|p-(\T{1}\LL(\\{ubq}-\T{1}));{}$\6
\&{if} ${}(\|i\W((\|i\AND(\|i-\T{1}))\E\T{0})\W(\|i\LL\T{4})<\|p){}$\5
${}\{{}$\1\6
\&{for} ${}(\|j\K\\{ubq}-\T{2};{}$ ${}(\|i\AND\T{1})\E\T{0};{}$ ${}\|i\MRL{{%
\GG}{\K}}\T{1},\39\|j\MM){}$\1\5
;\2\6
\&{if} ${}(\|b[\|j]<(\T{1}\LL\|j)){}$\1\5
\&{goto} \\{failp};\2\6
\&{for} ( ; ${}\|a[\|j]<(\T{1}\LL\|j);{}$ ${}\|j\MM){}$\1\5
${}\\{newa}(\|j,\39\T{1}\LL\|j);{}$\2\6
\4${}\}{}$\2\par
\U12.\fi

\M{14}At this point we had better not assume that $a[ubq]\le q\le b[ubq]$,
because \PB{\|p} has just been inserted. That insertion can mess up the
bounds that we looked at when \PB{\\{lbq}} and \PB{\\{ubq}} were computed.

\Y\B\4\X14:Put \PB{\|q} into the chain at location \PB{\\{ubq}}; \PB{\&{goto} %
\\{failq}} if there's a problem\X${}\E{}$\6
\&{if} ${}(\|a[\\{ubq}]\I\|q){}$\5
${}\{{}$\1\6
\&{if} ${}(\|a[\\{ubq}]>\|q){}$\1\5
\&{goto} \\{failq};\2\6
${}\\{newa}(\\{ubq},\39\|q);{}$\6
\&{for} ${}(\|j\K\\{ubq}-\T{1};{}$ ${}(\|a[\|j]\LL\T{1})<\|a[\|j+\T{1}];{}$ ${}%
\|j\MM){}$\5
${}\{{}$\1\6
${}\|i\K(\|a[\|j+\T{1}]+\T{1})\GG\T{1};{}$\6
\&{if} ${}(\|i>\|b[\|j]){}$\1\5
\&{goto} \\{failq};\2\6
${}\\{newa}(\|j,\39\|i);{}$\6
\4${}\}{}$\2\6
\&{for} ${}(\|j\K\\{ubq}+\T{1};{}$ ${}\|a[\|j]\Z\|a[\|j-\T{1}];{}$ ${}\|j%
\PP){}$\5
${}\{{}$\1\6
${}\|i\K\|a[\|j-\T{1}]+\T{1};{}$\6
\&{if} ${}(\|i>\|b[\|j]){}$\1\5
\&{goto} \\{failq};\2\6
${}\\{newa}(\|j,\39\|i);{}$\6
\4${}\}{}$\2\6
\4${}\}{}$\2\6
\&{if} ${}(\|b[\\{ubq}]\I\|q){}$\5
${}\{{}$\1\6
\&{if} ${}(\|b[\\{ubq}]<\|q){}$\1\5
\&{goto} \\{failq};\2\6
${}\\{newb}(\\{ubq},\39\|q);{}$\6
\&{for} ${}(\|j\K\\{ubq}-\T{1};{}$ ${}\|b[\|j]\G\|b[\|j+\T{1}];{}$ ${}\|j%
\MM){}$\5
${}\{{}$\1\6
${}\|i\K\|b[\|j+\T{1}]-\T{1};{}$\6
\&{if} ${}(\|i<\|a[\|j]){}$\1\5
\&{goto} \\{failq};\2\6
${}\\{newb}(\|j,\39\|i);{}$\6
\4${}\}{}$\2\6
\&{for} ${}(\|j\K\\{ubq}+\T{1};{}$ ${}\|b[\|j]>\|b[\|j-\T{1}]\LL\T{1};{}$ ${}%
\|j\PP){}$\5
${}\{{}$\1\6
${}\|i\K\|b[\|j-\T{1}]\LL\T{1};{}$\6
\&{if} ${}(\|i<\|a[\|j]){}$\1\5
\&{goto} \\{failq};\2\6
${}\\{newb}(\|j,\39\|i);{}$\6
\4${}\}{}$\2\6
\4${}\}{}$\2\6
\X15:Make forced moves if \PB{\|q} has a special form\X;\par
\U7.\fi

\M{15}\B\X15:Make forced moves if \PB{\|q} has a special form\X${}\E{}$\6
$\|i\K\|q-(\T{1}\LL(\\{ubq}-\T{1}));{}$\6
\&{if} ${}(\|i\W((\|i\AND(\|i-\T{1}))\E\T{0})\W(\|i\LL\T{4})<\|q){}$\5
${}\{{}$\1\6
\&{for} ${}(\|j\K\\{ubq}-\T{2};{}$ ${}(\|i\AND\T{1})\E\T{0};{}$ ${}\|i\MRL{{%
\GG}{\K}}\T{1},\39\|j\MM){}$\1\5
;\2\6
\&{if} ${}(\|b[\|j]<(\T{1}\LL\|j)){}$\1\5
\&{goto} \\{failq};\2\6
\&{for} ( ; ${}\|a[\|j]<(\T{1}\LL\|j);{}$ ${}\|j\MM){}$\1\5
${}\\{newa}(\|j,\39\T{1}\LL\|j);{}$\2\6
\4${}\}{}$\2\par
\U14.\fi

\M{16}The bottom line: Alas, this method turns out to be by far the
slowest of all. But maybe somebody will find a use for it?
The most interesting thing I noticed is that $l_1(n)=l_2(n)$
for $4\le n<14759$; in other words, when $n$ is small there's
always a way to get by without using `3' in the chain.
But all four addition chains of length 17 for $n=14759$
start with 1,~2,~3. For example, one of them is
1, 2, 3, 5, 10, 13, 23, 46, 92, 184, 368, 736, 1472, 2944, 2957, 5901, 8858,
14759.

(I learned subsequently that Sch\"onhage had conjectured $l_1(n)=l_2(n)$
in 1975. Moreover, Bleichenbacher and Flammenkamp mentioned
the first three counterexamples in an unpublished preprint of 1997.)

\fi

\N{1}{17}Index.
\fi

\inx
\fin
\con
