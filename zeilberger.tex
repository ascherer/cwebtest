\input cwebmac
% (c) 2005 D E Knuth; use and enjoy, but please don't blame me for mistakes...
\datethis
\def\title{ZEILBERGER}
\def\[#1]{[\hbox{$\mkern1mu\thickmuskip=\thinmuskip#1\mkern1mu$}]} % Iverson


\N{1}{1}Introduction. This little program implements Zeilberger's bijection
between $n$-node forests with pruning order~$s$ and
$n$-node binary trees with Strahler number~$s$. [Doron Zeilberger,
``A bijection from ordered trees to binary trees that sends the
pruning order to the Strahler number,'' {\sl Discrete Mathematics\/
\bf82} (1990), 89--92.]

As Zeilberger says in his paper, ``First, definitions!''
I won't define forests
(by which I always mean {\it ordered forests}, as in {\sl The Art of
Computer Programming\/}), nor need I define binary trees.
But I should explain the
concepts of pruning order and Strahler number, since they are
less familiar.

A {\it filament\/} of a forest is a sequence of one or more nodes
$x_1$, \dots,~$x_k$, where $x_{j+1}$ is the only child of $x_j$ for
$1\le j<k$, and where $x_k$ is a leaf. Given a forest, we can {\it prune\/}
it by removing all of its filaments. The number of times we must
do this before reaching the empty forest is called the {\it pruning
order\/} of the original forest.

The {\it Strahler number\/} of a binary tree is 0 if the binary
tree is empty; otherwise it is $\max(s_l,s_r)+\[s_l=s_r]$,
when $s_l$ and~$s_r$ are the Strahler numbers of the
left and right binary subtrees of the root.

Zeilberger's bijection proves constructively the remarkable fact,
discovered by Mireille Vauchaussade de Chaumont in her thesis (Bordeaux,
1985), that the pruning order and Strahler number have precisely
the same distribution, when forests and binary trees are chosen
uniformly at random.

Furthermore, as we shall see, his bijection has
another significant property: When forests are represented in the
natural way within a computer, as binary trees with left links to
the leftmost child of a node and with right links to a node's right
sibling, Zeilberger's transformation preserves all of the left links:
Node $x$ is the leftmost child of~$y$ in the original forest
if and only if $x$ is the left child of~$y$ in the binary tree that is
produced by Zeilberger's procedure. In particular, the number
of leaves in the forest equals the number of ``left leaves'' in the
corresponding binary tree.

This programs runs through all forests with a given number of nodes,
and computes the corresponding binary tree. It checks that the
pruning order of the former equals the Strahler number of the latter;
and then it applies the inverse bijection, thus verifying that
the original forest can indeed be uniquely reconstructed.

\vfill\eject

\fi

\M{2}Skarbek's algorithm (Algorithm 7.2.1.6B in {\sl The Art of
Computer Programming}, Volume~4) is used to run through
all linked binary trees, thereby running through all forests
in their natural representation.

\Y\B\4\D$\|n$ \5
\T{17}\C{ nodes in the forest }\par
\Y\B\8\#\&{include} \.{<stdio.h>}\6
\&{int} ${}\|l[\|n+\T{2}],{}$ ${}\|r[\|n+\T{1}]{}$;\C{ leftmost child and right
sibling }\6
\&{int} ${}\\{ll}[\|n+\T{1}],{}$ ${}\\{rr}[\|n+\T{1}]{}$;\C{ links of the
binary tree }\6
\&{int} ${}\\{lll}[\|n+\T{1}],{}$ ${}\\{rrr}[\|n+\T{1}]{}$;\C{ links of the
inverse forest }\6
\&{int} ${}\|q[\|n+\T{1}],{}$ ${}\|s[\|n+\T{1}]{}$;\C{ data needed by
Zeilberger's algorithm }\6
\&{int} \\{serial};\C{ total number of cases checked }\6
\&{int} \\{count}[\T{10}];\C{ individual counts by pruning order }\7
\X5:Subroutines\X\7
\\{main}(\,)\1\1\2\2\6
${}\{{}$\1\6
\&{register} \&{int} \|j${},{}$ \|k${},{}$ \|y${},{}$ \|p;\7
${}\\{printf}(\.{"Checking\ all\ forest}\)\.{s\ with\ \%d\ nodes...\\n}\)\.{"},%
\39\|n);{}$\6
${}\|q[\T{0}]\K\T{0},\39\|s[\T{0}]\K\T{1000000}{}$;\C{ see below }\6
\&{for} ${}(\|k\K\T{1};{}$ ${}\|k<\|n;{}$ ${}\|k\PP){}$\1\5
${}\|l[\|k]\K\|k+\T{1},\39\|r[\|k]\K\T{0};{}$\2\6
${}\|l[\|n]\K\|r[\|n]\K\T{0}{}$;\C{ we start with the 1-filament forest }\6
${}\|l[\|n+\T{1}]\K\T{1}{}$;\C{ now Skarbek's algorithm is ready to go }\6
\&{while} (\T{1})\5
${}\{{}$\1\6
\X6:Find the binary tree \PB{$(\\{ll},\\{rr})$} corresponding to the forest %
\PB{$(\|l,\|r)$}\X;\6
\X13:Check the pruning order and Strahler number\X;\6
\X14:Check the inverse bijection\X;\6
\X3:Move to the next forest \PB{$(\|l,\|r)$}, or \PB{\&{break}}\X\6
\4${}\}{}$\2\6
\&{for} ${}(\|p\K\T{1};{}$ \\{count}[\|p]; ${}\|p\PP){}$\1\5
${}\\{printf}(\.{"Altogether\ \%d\ cases}\)\.{\ with\ pruning\ order\ }\)\.{%
\%d.\\n"},\39\\{count}[\|p],\39\|p);{}$\2\6
\4${}\}{}$\2\par
\fi

\M{3}\B\X3:Move to the next forest \PB{$(\|l,\|r)$}, or \PB{\&{break}}\X${}%
\E{}$\6
\&{for} ${}(\|j\K\T{1};{}$ ${}\R\|l[\|j];{}$ ${}\|j\PP){}$\1\5
${}\|r[\|j]\K\T{0},\39\|l[\|j]\K\|j+\T{1};{}$\2\6
\&{if} ${}(\|j>\|n){}$\1\5
\&{break};\2\6
\&{for} ${}(\|k\K\T{0},\39\|y\K\|l[\|j];{}$ \|r[\|y]; ${}\|k\K\|y,\39\|y\K\|r[%
\|y]){}$\1\5
;\2\6
\&{if} ${}(\|k>\T{0}){}$\1\5
${}\|r[\|k]\K\T{0}{}$;\5
\2\&{else}\1\5
${}\|l[\|j]\K\T{0};{}$\2\6
${}\|r[\|y]\K\|r[\|j],\39\|r[\|j]\K\|y{}$;\par
\U2.\fi

\N{1}{4}The main algorithm. The nodes in the forest are represented by
positive integers, according to their rank in preorder. For each
node~$x>0$, we let $l[x]$ be its leftmost child (namely $x+1$),
if it has one, but $l[x]=0$ when $x$ is childless. Similarly,
$r[x]$ is $x$'s right sibling, or $r[x]=0$ when $x$ is rightmost
in its family.

Zeilberger's method implicitly begins by
determining the pruning order of the given forest and its
principal subforests. We can carry this out by defining three numbers
$p[x]$, $q[x]$, and $s[x]$ for each node~$x$; here $p[x]$ is
the pruning order of the subtree rooted at~$x$,
$q[x]$ is the maximum pruning order of $x$ and all of its right siblings,
and $s[x]$ is the number of nodes in which that maximum occurs.
(Zeilberger called $s[l[x]]$ the number of ``skewers'' of~$x$.)

Formally, we can compute $p[x]$, $q[x]$, and $s[x]$ via
the following recursive definitions, if we set $q[0]=0$ and $s[0]=\infty$:
$$\openup1\jot\eqalign{
p[x]&=q[l[x]]+\[s[l[x]]>1];\cr
q[x]&=\max(p[x],q[r[x]]);\cr
s[x]&=\cases{1,&if $p[x]>q[r[x]]$;\cr
s[r[x]]+1,&if $p[x]=q[r[x]]$;\cr
s[r[x]],&if $p[x]<q[r[x]]$.\cr}\cr}$$

\fi

\M{5}The following recursive procedure computes
$(p[x],q[x],s[x])$ at each node~$x$ of the forest. It turns out
that $p[x]$ need not be stored in memory.

\Y\B\4\X5:Subroutines\X${}\E{}$\6
\&{void} \\{label}(\&{register} \&{int} \|x)\1\1\2\2\6
${}\{{}$\1\6
\&{register} \&{int} \|p${},{}$ \\{qr};\7
\&{if} (\|l[\|x])\1\5
\\{label}(\|l[\|x]);\2\6
\&{if} (\|r[\|x])\1\5
\\{label}(\|r[\|x]);\2\6
${}\|p\K\|q[\|l[\|x]]+(\|s[\|l[\|x]]>\T{1});{}$\6
${}\\{qr}\K\|q[\|r[\|x]];{}$\6
\&{if} ${}(\|p>\\{qr}){}$\1\5
${}\|q[\|x]\K\|p,\39\|s[\|x]\K\T{1};{}$\2\6
\&{else}\1\5
${}\|q[\|x]\K\\{qr},\39\|s[\|x]\K\|s[\|r[\|x]]+(\|p\E\\{qr});{}$\2\6
\4${}\}{}$\2\par
\As8, 12\ETs15.
\U2.\fi

\M{6}Zeilberger's bijection is recursively defined too, and we will
implement it by writing another recursive procedure. But before
we do so, let's examine how that procedure will be invoked in the program.

\Y\B\4\X6:Find the binary tree \PB{$(\\{ll},\\{rr})$} corresponding to the
forest \PB{$(\|l,\|r)$}\X${}\E{}$\6
\&{for} ${}(\|k\K\T{1};{}$ ${}\|k\Z\|n;{}$ ${}\|k\PP){}$\1\5
${}\\{ll}[\|k]\K\|l[\|k],\39\\{rr}[\|k]\K\|r[\|k]{}$;\C{ clone the forest }\2\6
\\{label}(\T{1});\C{ compute all the $q$'s and $s$'s }\6
\\{zeil}(\T{1});\C{ transform the binary tree }\par
\U2.\fi

\M{7}When subroutine \PB{\\{zeil}(\|x)} is called, \PB{\|x} is a node of a
forest, represented
via left-child and right-sibling links \PB{\\{ll}} and \PB{\\{rr}}. The mission
of \PB{\\{zeil}(\|x)} is to transform that forest in such a way that \PB{\|x}
will
be a node of the final binary tree, while \PB{\\{ll}[\|x]} and \PB{\\{rr}[\|x]}
will
be the binary subtrees that are obtained by applying \PB{\\{zeil}} to
two subforests.

Zeilberger's paper considered three cases, depending on the relative
sizes of \PB{\|q[\|x]}, \PB{\|q[\\{ll}[\|x]]}, and \PB{\|q[\\{rr}[\|x]]}. In
Case 1, no change
is needed; in Case 2, some of \PB{\|x}'s children are promoted to be
siblings of~\PB{\|x}; in Case 3, some of \PB{\|x}'s children swap places with
all of the right siblings.

\fi

\M{8}\B\X5:Subroutines\X${}\mathrel+\E{}$\6
\&{void} \\{zeil}(\&{register} \&{int} \|x)\1\1\2\2\6
${}\{{}$\1\6
\&{register} \&{int} \|k${},{}$ \\{ql}${},{}$ \\{qr}${},{}$ \|p${},{}$ %
\|y${},{}$ \\{yy}${},{}$ \|z${},{}$ \\{zz}${},{}$ \\{ss};\7
${}\|k\K\|q[\|x],\39\\{ql}\K\|q[\\{ll}[\|x]],\39\\{qr}\K\|q[\\{rr}[\|x]];{}$\6
\&{if} ${}(\\{ql}\E\|k){}$\1\5
\X10:Do Zeilberger's Case 3\X\2\6
\&{else} \&{if} ${}(\\{qr}<\|k-\T{1}){}$\1\5
\X9:Do Zeilberger's Case 2\X;\C{ otherwise we do Zeilberger's Case 1, which
involves no action }\2\6
\&{if} (\\{ll}[\|x])\1\5
\\{zeil}(\\{ll}[\|x]);\2\6
\&{if} (\\{rr}[\|x])\1\5
\\{zeil}(\\{rr}[\|x]);\2\6
\4${}\}{}$\2\par
\fi

\M{9}Case 2, the tricky case, detaches all but the first of \PB{\|x}'s
``skewers.''

When right links change, we must update the \PB{\|q} and \PB{\|s}
numbers in each subnode of~\PB{\|x} before applying \PB{\\{zeil}} to
the new subforests.
In particular, we must recompute all the \PB{\|q}'s and \PB{\|s}'s for
the children of~\PB{\|x} that lie between the first and second skewer,
because those nodes will no longer lie in the shadow of the
second skewer.

This recomputation is a bit tricky because
it must be done bottom-up, while our links go downward.
An auxiliary recursive procedure could be introduced at this point,
in order to achieve bottom-up behavior.
But there's a more efficient (and more fun) way to do the job,
namely to reverse the links temporarily
as we descend the chain, and to reverse them again as we go back up.

\Y\B\4\X9:Do Zeilberger's Case 2\X${}\E{}$\6
${}\{{}$\1\6
\&{for} ${}(\|y\K\\{ll}[\|x],\39\\{ss}\K\|s[\|y];{}$ ${}\|s[\|y]\E\\{ss};{}$
${}\\{yy}\K\|y,\39\|y\K\\{rr}[\\{yy}]){}$\1\5
${}\|s[\|y]\K\T{1}{}$;\C{ at this point node \PB{\\{yy}} is the first skewer of
\PB{\|x} }\2\6
\&{for} ${}(\|z\K\\{yy},\39\\{ss}\MM;{}$ ${}\|s[\\{rr}[\|y]]\E\\{ss}\W\|q[%
\\{rr}[\|y]]\E\\{ql};\3{-1}{}$ ${}\\{yy}\K\|z,\39\|z\K\|y,\39\|y\K\\{rr}[\|z],%
\39\\{rr}[\|z]\K\\{yy}){}$\1\5
;\C{ reverse links }\2\6
\hbox{}\C{ now node \PB{\|y} is the second skewer of \PB{\|x} }\6
\hbox{}\C{ its left sibling, \PB{\|z}, will become the last child of \PB{\|x} }%
\6
\&{if} ${}(\|z\E\\{yy}){}$\1\5
${}\\{rr}[\|z]\K\T{0}{}$;\C{ easy case: the skewers were adjacent }\2\6
\&{else}\5
\1\&{for} ${}(\\{zz}\K\T{0};{}$ ${}\\{rr}[\|z]\I\\{zz};{}$ ${}\\{yy}\K\\{zz},%
\39\\{zz}\K\|z,\39\|z\K\\{rr}[\\{zz}],\39\\{rr}[\\{zz}]\K\\{yy}){}$\5
${}\{{}$\1\6
${}\|p\K\|q[\\{ll}[\|z]]+(\|s[\\{ll}[\|z]]>\T{1}){}$;\C{ we will recompute \PB{%
\|q[\|z]} and \PB{\|s[\|z]} }\6
\&{if} ${}(\|p>\|q[\\{zz}]){}$\1\5
${}\|q[\|z]\K\|p,\39\|s[\|z]\K\T{1};{}$\2\6
\&{else}\1\5
${}\|q[\|z]\K\|q[\\{zz}],\39\|s[\|z]\K\|s[\\{zz}]+(\|p\E\|q[\\{zz}]);{}$\2\6
\4${}\}{}$\2\2\6
\&{for} ${}(\\{zz}\K\|x,\39\|z\K\\{rr}[\\{zz}];{}$ \|z; ${}\\{zz}\K\|z,\39\|z\K%
\\{rr}[\\{zz}]){}$\1\5
${}\|q[\|z]\K\\{ql},\39\|s[\|z]\K\\{ss};{}$\2\6
${}\\{rr}[\\{zz}]\K\|y{}$;\C{ \PB{\|y} and its right siblings become \PB{\|x}'s
final siblings }\6
\4${}\}{}$\2\par
\U8.\fi

\M{10}In Case 3, \PB{\|x} has only one skewer. But the transformation
in this case demotes siblings to children, where they might become
additional skewers. It also might promote some children to siblings.

\Y\B\4\X10:Do Zeilberger's Case 3\X${}\E{}$\6
${}\{{}$\1\6
\&{if} ${}(\\{qr}\E\|k){}$\1\5
${}\\{ss}\K\|s[\\{rr}[\|x]]+\T{1}{}$;\5
\2\&{else}\1\5
${}\\{ss}\K\T{1};{}$\2\6
\&{for} ${}(\|y\K\\{ll}[\|x];{}$ ${}\|q[\|y]\E\|k;{}$ ${}\\{yy}\K\|y,\39\|y\K%
\\{rr}[\\{yy}]){}$\1\5
${}\|s[\|y]\K\\{ss};{}$\2\6
${}\\{rr}[\\{yy}]\K\\{rr}[\|x],\39\\{rr}[\|x]\K\|y;{}$\6
\4${}\}{}$\2\par
\U8.\fi

\N{1}{11}Checking the Strahler number. If our implementation of Zeilberger's
transformation is correct, it will have set \PB{\|q[\|x]} to the Strahler
number of the binary subtree rooted at \PB{\|x} with respect to the
\PB{\\{ll}} and \PB{\\{rr}} links, for every node~\PB{\|x}.

Therefore we want to check this condition. And we might as well
do the checking by brute force, so that the evidence is convincing.

\fi

\M{12}\B\X5:Subroutines\X${}\mathrel+\E{}$\6
\&{int} \\{strahler}(\&{register} \&{int} \|x)\1\1\2\2\6
${}\{{}$\1\6
\&{register} \&{int} \\{sl}${},{}$ \\{sr}${},{}$ \|s;\7
\&{if} (\\{ll}[\|x])\1\5
${}\\{sl}\K\\{strahler}(\\{ll}[\|x]);{}$\2\6
\&{else}\1\5
${}\\{sl}\K\T{0};{}$\2\6
\&{if} (\\{rr}[\|x])\1\5
${}\\{sr}\K\\{strahler}(\\{rr}[\|x]);{}$\2\6
\&{else}\1\5
${}\\{sr}\K\T{0};{}$\2\6
${}\|s\K(\\{sl}>\\{sr}\?\\{sl}:\\{sl}<\\{sr}\?\\{sr}:\\{sl}+\T{1});{}$\6
\&{if} ${}(\|q[\|x]\I\|s){}$\1\5
${}\\{fprintf}(\\{stderr},\39\.{"I\ goofed\ at\ binary\ }\)\.{tree\ node\ \%d,\
case\ \%}\)\.{d.\\n"},\39\|x,\39\\{serial});{}$\2\6
\&{return} \|s;\6
\4${}\}{}$\2\par
\fi

\M{13}\B\X13:Check the pruning order and Strahler number\X${}\E{}$\6
$\\{count}[\\{strahler}(\T{1})]\PP;{}$\6
${}\\{serial}\PP{}$;\par
\U2.\fi

\N{1}{14}The inverse algorithm. The evidence of correctness is mounting.
But our argument in favor of Zeilberger's algorithm
is still not compelling, because there are lots
of ways to convert a forest
with pruning order~$s$ into a binary tree with Strahler number~$s$.
For example, we need only compute the pruning order, then choose
our favorite binary tree that has the desired Strahler number.

A bijection must do more than this: It must not destroy information. We
must be able to go back from each binary tree to the original forest
that produced it. Thus, our implementation of Zeilberger's procedure is
incomplete until we have also implemented its inverse, \PB{\\{unzeil}}.

Our \PB{\\{unzeil}} algorithm will recreate the forest in new arrays \PB{%
\\{lll}} and \PB{\\{rrr}},
just to emphasize that no cheating is going on.

\Y\B\4\X14:Check the inverse bijection\X${}\E{}$\6
\&{for} ${}(\|k\K\T{1};{}$ ${}\|k\Z\|n;{}$ ${}\|k\PP){}$\1\5
${}\\{lll}[\|k]\K\\{ll}[\|k],\39\\{rrr}[\|k]\K\\{rr}[\|k],\39\|q[\|k]\K\|s[\|k]%
\K\T{0}{}$;\C{ clone the binary tree }\2\6
\\{unzeil}(\T{1});\C{ attempt to recreate the original forest }\6
\&{for} ${}(\|k\K\T{1};{}$ ${}\|k\Z\|n;{}$ ${}\|k\PP){}$\1\6
\&{if} ${}(\\{lll}[\|k]\I\|l[\|k]\V\\{rrr}[\|k]\I\|r[\|k]){}$\1\5
${}\\{fprintf}(\\{stderr},\39\.{"Rejection\ at\ node\ \%}\)\.{d\ of\ case\ \%d!%
\\n"},\39\|k,\39\\{serial}){}$;\2\2\par
\U2.\fi

\M{15}The \PB{\\{unzeil}} procedure also computes the \PB{\|q} and \PB{\|s}
values at each node
of the reconstructed forest.

\Y\B\4\X5:Subroutines\X${}\mathrel+\E{}$\6
\&{void} \\{unzeil}(\&{register} \&{int} \|x)\1\1\2\2\6
${}\{{}$\1\6
\&{register} \&{int} \\{ql}${},{}$ \\{qr}${},{}$ \|p${},{}$ \|y${},{}$ %
\\{yy}${},{}$ \|z${},{}$ \\{zz}${},{}$ \\{ss};\7
\&{if} (\\{rrr}[\|x])\1\5
\\{unzeil}(\\{rrr}[\|x]);\2\6
\&{if} (\\{lll}[\|x])\1\5
\\{unzeil}(\\{lll}[\|x]);\2\6
${}\\{ql}\K\|q[\\{lll}[\|x]],\39\\{qr}\K\|q[\\{rrr}[\|x]];{}$\6
\&{if} ${}(\\{ql}>\\{qr}){}$\1\5
\X16:Undo Zeilberger's Case 3\X\2\6
\&{else} \&{if} ${}(\\{ql}\E\\{qr}\W\|s[\\{lll}[\|x]]\E\T{1}){}$\1\5
\X17:Undo Zeilberger's Case 2\X;\C{ otherwise we undo Zeilberger's Case 1,
which involves no action }\2\6
${}\|p\K\\{ql}+(\|s[\\{lll}[\|x]]>\T{1});{}$\6
\&{if} ${}(\|p>\\{qr}){}$\1\5
${}\|q[\|x]\K\|p,\39\|s[\|x]\K\T{1};{}$\2\6
\&{else}\1\5
${}\|q[\|x]\K\\{qr},\39\|s[\|x]\K\|s[\\{rrr}[\|x]]+(\|p\E\\{qr});{}$\2\6
\4${}\}{}$\2\par
\fi

\M{16}\B\X16:Undo Zeilberger's Case 3\X${}\E{}$\6
${}\{{}$\1\6
\&{if} ${}(\|s[\\{lll}[\|x]]>\T{1}){}$\5
${}\{{}$\1\6
\&{for} ${}(\|y\K\\{lll}[\|x];{}$ ${}\|s[\\{rrr}[\|y]]\E\|s[\|y];{}$ ${}\|y\K%
\\{rrr}[\|y]){}$\1\5
${}\|s[\|y]\K\T{1};{}$\2\6
${}\|s[\|y]\K\T{1};{}$\6
\4${}\}{}$\5
\2\&{else}\5
\1\&{for} ${}(\|y\K\\{lll}[\|x];{}$ ${}\|q[\\{rrr}[\|y]]\E\|q[\|y];{}$ ${}\|y\K%
\\{rrr}[\|y]){}$\1\5
;\2\2\6
${}\\{yy}\K\\{rrr}[\|y],\39\\{rrr}[\|y]\K\\{rrr}[\|x],\39\\{rrr}[\|x]\K%
\\{yy}{}$;\C{ swap siblings with children }\6
${}\\{qr}\K\|q[\\{yy}]{}$;\C{ the subsequent program assumes that \PB{$\\{qr}\K%
\|q[\\{rrr}[\|x]]$} }\6
\4${}\}{}$\2\par
\U15.\fi

\M{17}We have saved the other tricky case for last:
Again we do a double-reversal in order to recompute any \PB{\|q}'s and \PB{%
\|s}'s
that have been obliterated. In this case the recomputation affects
the near siblings of~\PB{\|x}.

\Y\B\4\X17:Undo Zeilberger's Case 2\X${}\E{}$\6
${}\{{}$\1\6
\&{for} ${}(\|z\K\\{rrr}[\|x],\39\\{zz}\K\|x;{}$ ${}\|s[\\{rrr}[\|z]]\E\|s[\|z]%
\W\|q[\\{rrr}[\|z]]\E\|q[\|z];\3{-1}{}$ ${}\\{yy}\K\\{zz},\39\\{zz}\K\|z,\39\|z%
\K\\{rrr}[\\{zz}],\39\\{rrr}[\\{zz}]\K\\{yy}){}$\1\5
;\2\6
\&{if} ${}(\\{zz}\E\|x){}$\1\5
${}\\{yy}\K\\{rrr}[\|x]\K\T{0};{}$\2\6
\&{else}\5
\1\&{for} ${}(\\{yy}\K\T{0};{}$ ${}\\{zz}\I\|x;{}$ ${}\|y\K\\{zz},\39\\{zz}\K%
\\{rrr}[\|y],\39\\{rrr}[\|y]\K\\{yy},\39\\{yy}\K\|y){}$\5
${}\{{}$\1\6
${}\|p\K\|q[\\{lll}[\\{zz}]]+(\|s[\\{lll}[\\{zz}]]>\T{1}){}$;\C{ we will
recompute \PB{\|q[\\{zz}]} and \PB{\|s[\\{zz}]} }\6
\&{if} ${}(\|p>\|q[\\{yy}]){}$\1\5
${}\|q[\\{zz}]\K\|p,\39\|s[\\{zz}]\K\T{1};{}$\2\6
\&{else}\1\5
${}\|q[\\{zz}]\K\|q[\\{yy}],\39\|s[\\{zz}]\K\|s[\\{yy}]\K(\|p\E\|q[\\{yy}]);{}$%
\2\6
\4${}\}{}$\C{ \PB{\|x}'s former siblings are now undone }\2\2\6
${}\\{qr}\K\|q[\\{yy}],\39\\{ss}\K\|s[\|z]+\T{1}{}$;\C{ at this point \PB{$%
\\{yy}\K\\{rrr}[\|x]$} }\6
\&{for} ${}(\\{yy}\K\\{lll}[\|x],\39\|y\K\\{rrr}[\\{yy}];{}$ \|y; ${}\\{yy}\K%
\|y,\39\|y\K\\{rrr}[\\{yy}]){}$\5
${}\{{}$\1\6
${}\|s[\\{yy}]\K\\{ss};{}$\6
\&{if} ${}(\|q[\\{yy}]\E\\{ql}){}$\1\5
${}\\{ss}\MM;{}$\2\6
\4${}\}{}$\2\6
${}\|s[\\{yy}]\K\\{ss},\39\\{rrr}[\\{yy}]\K\|z{}$;\C{ unpromote \PB{\|x}'s
former children }\6
\4${}\}{}$\2\par
\U15.\fi

\N{1}{18}Index.

\fi


\inx
\fin
\con
