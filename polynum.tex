\input cwebmac
\datethis
\ifpdftex \input supp-pdf \else \input epsf \fi

\def\caret/{\smash{\lower4pt\hbox to0pt{\hss$\scriptscriptstyle\land$\hss}}}
\def\qcaret/{`\thinspace\caret/\thinspace'} % quoted caret


\N{1}{1}Introduction. The purpose of this program is to enumerate fixed
polyominoes
of up to 55 cells (although I won't really be able to get that far until
Moore's law carries on for a few more years). The method is essentially
that of Iwan Jensen [{\tt arXiv:cond-mat/0007239}, to appear in {\sl Journal of
Statistical Physics}, spring 2001], who discovered that the important
techniques of Andrew Conway [{\sl Journal of Physics\/ \bf A28} (1995),
335--349] can be vastly improved in the special case of polyomino
enumeration.

The basic idea is quite simple: We will count the number of fixed polyominoes
that span a rectangle that is $h$ cells high and $w$ cells wide, where
$h$ and $w$ are as small as possible; then we will add the totals for
all relevant $h$ and $w$. We can assume that $h\ge w$. For each $h$ and $w$
that we need, we enumerate the spanning polyominoes by considering
one cell at a time of an $h\times w$ array, working from left to right
and top to bottom, deciding whether or not that cell is occupied, and
combining results for all boundary configurations that are equivalent
as far as future decisions are concerned. For example, we might have a
polyomino that starts out like this:
$$\ifpdftex\convertMPtoPDF{polyomino.2}{1}{1}\else\epsfbox{polyomino.2}\fi$$
(This partial polyomino obviously has more than 54 cells already, but large
examples will help clarify the concepts that are needed in the program below.)

Most of the details of the upper part of this pattern have no effect on
whether the yet-undetermined cells will form a suitable polyomino or not.
All we really need to know in order to answer that question is
whether the bottom cells of each column-so-far are occupied or
unoccupied, and which of the occupied ones are already connected to
each other. We also need to know whether the left column and right
column are still blank.

In this case the 26 columns have occupancy pattern
\.{01001001001010110011010110} at the bottom,
and the occupied cells belong to six connected components, namely
$$\vbox{\halign{\.{#}\cr
01000000000000000000010000\cr
00001000001000110000000000\cr
00000001000000000000000000\cr
00000000000010000000000000\cr
00000000000000000011000000\cr
00000000000000000000000110\cr}}$$
Fortunately the fact that polyominoes lie in a plane forces the components
to be nested within each other; we can't have ``crossings'' like
\.{1000100} with \.{0010001}. Therefore we can encode the component and
occupancy information conveniently as
$$\.{0(00(00100-010-)00()0)0()0}$$
using a five-character alphabet:
$$\vbox{\halign{\.#\enspace&#\hfil\cr
0&means the cell is unoccupied;\cr
1&means the cell is a single-cell component;\cr
(&means the cell is the leftmost of a multi-cell component;\cr
)&means the cell is the rightmost of a multi-cell component;\cr
-&means the cell is in the midst of a multi-cell component.\cr}}$$
Furthermore we can treat the cases where the entire leftmost column is
nonblank by considering that the left edge of the array belongs to the leftmost
cell component; and the rightmost column can be treated similarly.
With these conventions, we encode the boundary condition at the
lower fringe of the partially filled array above by the 26-character string
$$\.{0(00(00100-010-\caret/)00()0)0(-0}\,,\eqno(*)$$
using \qcaret/ to show where the last partial row ends.

\vskip1pt
A string like $(*)$ represents a so-called {\it configuration}. If no
\qcaret/ appears, the partial row is assumed to be a
complete row. The number of rows above a given occupancy pattern is implicitly
part of the configuration but not explicitly shown in the notation,
because this program never has to deal simultaneously with configurations
that have different numbers of rows above the current partial row.

\fi

\M{2}A bit of theory may help the reader internalize these rules: It turns out
that the set of all connectivity/occupancy configuration codes at the end of a
row has the interesting unambiguous context-free grammar
$$\eqalign{
S&\to L\,\.0J_0R \mid Z\.-I\.-Z\mid Z\.-Z\cr
Z&\to\epsilon\mid\.0Z\cr
L&\to\epsilon\mid Z\,\.)\mid Z\.-I\.)\cr
R&\to\epsilon\mid \.(Z\mid \.(I\.-Z\cr
I_0&\to\epsilon\mid\.0J_0\cr
J_0&\to I_0\mid A\,\.0J_0\cr
I&\to\epsilon\mid\.-I\mid\.0J\cr
J&\to I\mid A\,\.0J\cr
A&\to 1\mid \.(I\.)\cr
}$$
[Translation: $I$ is any sequence of \.0's and \.-'s and $A$'s with each
$A$ preceded and followed by~\.0; $I_0$ is similar, but with no \.- at
top level.] The number $s_n$ of strings of length $n$
in $S$ has the generating function
$$\sum s_nz^n={1-4z^2-4z^3+z^4-(1+z)(1-z^2)\sqrt{\mkern1mu1+z}
\sqrt{\mkern1mu1-3z}\over2z^3(1-z)}=2z+6z^2+16z^3+\cdots{};$$
hence $s_n$ is asymptotically proportional to $3^n\!/n^{3/2}$.

\fi

\M{3}Any polyomino with the configuration $(*)$ in the midst of row 10 must
have
occupied at least $28+18+1+1+2+4=54$ cells so far, and 16 more will be
needed to connect it up and make it touch the left edge of the array.
Moreover, the array has only 10 rows so far, but it has 26 columns, and we
will only be interested in arrays for which $h\ge w$. Therefore at least 15
additional cells must be occupied if we are going to complete a polyomino in
an $h\times26$ array.

In other words, we would need to consider the boundary configuration
$(*)$ only if we were enumerating polyominoes of
at least $54+16+15=85$ cells. Such considerations greatly limit the number of
configurations that can arise; Jensen's experiments show that the total number
of cases we must deal with grows approximately as $c^n$ where $c$ is
slightly larger than~1.4. This is exponentially large, but it is considerably
smaller than the $3^{n/2}$ configurations that arise in Conway's original
method; and it's vastly smaller than the total number of polyominoes,
which turns out to be approximately $4.06^n$ times $0.3/n$.

\fi

\M{4}This program doesn't actually do the whole job of enumeration;
it only outputs a set of instructions that tell how to do it.
Another program reads those instructions and completes the computation.

\Y\B\8\#\&{include} \.{<stdio.h>}\6
\X12:Type definitions\X\6
\X7:Global variables\X\6
\X5:Subroutines\X\7
\\{main}(\&{int} \\{argc}${},\39{}$\&{char} ${}{*}\\{argv}[\,]){}$\1\1\2\2\6
${}\{{}$\1\6
\X67:Local variables\X;\6
\X6:Scan the command line\X;\6
\X15:Initialize\X;\6
\X78:Output instructions for the postprocessor\X;\6
\X72:Print statistics about this run\X;\6
\X16:Empty the buffer and close the output file\X;\6
\\{exit}(\T{0});\6
\4${}\}{}$\2\par
\fi

\M{5}\B\X5:Subroutines\X${}\E{}$\6
\&{void} \\{panic}(\&{char} ${}{*}\\{mess}){}$\1\1\2\2\6
${}\{{}$\1\6
${}\\{fprintf}(\\{stderr},\39\.{"\%s!\\n"},\39\\{mess});{}$\6
${}\\{exit}({-}\T{1});{}$\6
\4${}\}{}$\2\par
\As9, 10, 11, 13, 14, 17, 30, 31, 32, 35, 55, 56, 59, 63, 64, 69, 73, 75\ETs76.
\U4.\fi

\M{6}The user specifies the maximum size of polyominoes to be counted, $n$,
and the desired width, $w$, on the command line. All $h\times w$ rectangles
spanned by polyominoes of $n$ cells or less will be counted, for
$w\le h\le n+1-w$. (No solutions are possible for $h>n+1-w$.)

The present version of this program restricts \PB{\|w} to be at most
23, for simplicity. But the packing and unpacking routines below could be
adapted via \.{CWEB} change files in order to deal with
values of \PB{\|w} as large as 27, when we're pushing the envelope.

The command line should also specify the amount of memory allocated for
configurations in this program and for individual counters in the output.
Statistics will be printed for guidance in the choice of those parameters.

The base name of the output file should be given as the final command-line
argument. This name will actually be extended by \.{.0}, \.{.1}, \dots,
as explained below, because there might be an enormous amount of output.

\Y\B\4\D$\\{wmax}$ \5
\T{23}\C{ for quinary/octal packing into two tetrabytes }\par
\B\4\D$\\{nmax}$ \5
$(\\{wmax}+\\{wmax}+\T{126}{}$)\par
\B\4\D$\\{bad}(\|k,\|v)$ \5
$\\{sscanf}(\\{argv}[\|k],\39\.{"\%d"},\39{\AND}\|v)\I{}$\T{1}\par
\Y\B\4\X6:Scan the command line\X${}\E{}$\6
\&{if} ${}(\\{argc}\I\T{6}\V\\{bad}(\T{1},\39\|n)\V\\{bad}(\T{2},\39\|w)\V%
\\{bad}(\T{3},\39\\{conf\_size})\V\\{bad}(\T{4},\39\\{slave\_size})){}$\5
${}\{{}$\1\6
${}\\{fprintf}(\\{stderr},\39\.{"Usage:\ \%s\ n\ w\ confs}\)\.{ize\ slavesize\
outfil}\)\.{ename\\n"},\39\\{argv}[\T{0}]);{}$\6
${}\\{exit}({-}\T{2});{}$\6
\4${}\}{}$\2\6
\&{if} ${}(\|w>\\{wmax}){}$\1\5
\\{panic}(\.{"Sorry,\ that\ w\ is\ to}\)\.{o\ big\ for\ this\ imple}\)%
\.{mentation"});\2\6
\&{if} ${}(\|w<\T{2}){}$\1\5
\\{panic}(\.{"No,\ w\ must\ be\ at\ le}\)\.{ast\ 2"});\2\6
\&{if} ${}(\|n<\|w+\|w-\T{1}){}$\1\5
\\{panic}(\.{"There\ are\ no\ soluti}\)\.{ons\ for\ such\ a\ small}\)\.{\ n"});%
\2\6
\&{if} ${}(\|n>\|w+\|w+\T{126}){}$\1\5
\\{panic}(\.{"Eh?\ That\ n\ is\ incre}\)\.{dible"});\2\6
${}\\{base\_name}\K\\{argv}[\T{5}]{}$;\par
\U4.\fi

\M{7}\B\X7:Global variables\X${}\E{}$\6
\&{int} \|n;\C{ we will count polyominoes of $n$ or fewer cells }\6
\&{int} \|w;\C{ provided that they span a rectangle of width $w$          and
height $\ge w$ }\6
\&{int} \\{conf\_size};\C{ the number of \&{config} structures in our memory }\6
\&{int} \\{slave\_size};\C{ the number of counter positions          in the
slave program memory }\par
\As8, 18, 27, 34, 39, 57, 65, 68, 71, 85, 98, 108\ETs110.
\U4.\fi

\N{1}{8}Output. Let's get the basics of output out of the way first, so that
we know where we're heading. The postprocessing program {\mc POLYSLAVE}
will interpret instructions according to a compact binary format,
with either one or four bytes per instruction.

Several gigabytes might well be generated,
and my Linux system is not real happy with files of length greater than
$2^{31}-1=2147483647$. Therefore this program breaks the output up
into a sequence of files called \.{foo.0}, \.{foo.1}, \dots, each
at most one large gigabyte in size. (That's one GGbyte${}=2^{30}$~bytes.)

Some unfortunate hardware failures led me to add a \PB{\\{checksum}} feature.

\Y\B\4\D$\\{filelength\_threshold}$ \5
$(\T{1}\LL\T{30}{}$)\C{ maximum file size in bytes }\par
\B\4\D$\\{buf\_size}$ \5
$(\T{1}\LL\T{16}{}$)\C{ buffer size, should be a divisor of \PB{\\{filelength%
\_threshold}} }\par
\Y\B\4\X7:Global variables\X${}\mathrel+\E{}$\6
\&{FILE} ${}{*}\\{out\_file}{}$;\C{ the output file }\6
\&{union} ${}\{{}$\1\6
\&{unsigned} \&{char} ${}\\{buf}[\\{buf\_size}+\T{4}]{}$;\C{ place for binary
output }\6
\&{int} \\{foo};\C{ force \PB{$\\{out}.\\{buf}$} to be aligned somewhat
sensibly }\2\6
${}\}{}$ \\{out};\6
\&{unsigned} \&{char} ${}{*}\\{buf\_ptr}{}$;\C{ our current place in the buffer
}\6
\&{int} \\{bytes\_out};\C{ the number of bytes so far in the current output
file }\6
\&{unsigned} \&{int} \\{checksum};\C{ a way to help identify bad I/O }\6
\&{FILE} ${}{*}\\{ck\_file}{}$;\C{ the checksum file }\6
\&{int} \\{file\_extension};\C{ the number of GGbytes output }\6
\&{char} ${}{*}\\{base\_name},{}$ \\{filename}[\T{100}];\par
\fi

\M{9}\B\X5:Subroutines\X${}\mathrel+\E{}$\6
\&{void} \\{open\_it}(\,)\1\1\2\2\6
${}\{{}$\1\6
${}\\{sprintf}(\\{filename},\39\.{"\%.90s.\%d"},\39\\{base\_name},\39\\{file%
\_extension});{}$\6
${}\\{out\_file}\K\\{fopen}(\\{filename},\39\.{"wb"});{}$\6
\&{if} ${}(\R\\{out\_file}){}$\5
${}\{{}$\1\6
${}\\{fprintf}(\\{stderr},\39\.{"I\ can't\ open\ file\ \%}\)\.{s"},\39%
\\{filename});{}$\6
\\{panic}(\.{"\ for\ output"});\6
\4${}\}{}$\2\6
${}\\{bytes\_out}\K\\{checksum}\K\T{0};{}$\6
\4${}\}{}$\2\par
\fi

\M{10}\B\X5:Subroutines\X${}\mathrel+\E{}$\6
\&{void} \\{close\_it}(\,)\1\1\2\2\6
${}\{{}$\1\6
\&{if} ${}(\\{fwrite}({\AND}\\{checksum},\39{}$\&{sizeof}(\&{unsigned} %
\&{int})${},\39\T{1},\39\\{ck\_file})\I\T{1}){}$\1\5
\\{panic}(\.{"I\ couldn't\ write\ th}\)\.{e\ check\ sum"});\2\6
\&{if} ${}(\\{fclose}(\\{out\_file})\I\T{0}){}$\1\5
\\{panic}(\.{"I\ couldn't\ close\ th}\)\.{e\ output\ file"});\2\6
${}\\{printf}(\.{"[\%d\ bytes\ written\ o}\)\.{n\ file\ \%s,\ checksum\ }\)\.{%
\%u.]\\n"},\39\\{bytes\_out},\39\\{filename},\39\\{checksum});{}$\6
\4${}\}{}$\2\par
\fi

\M{11}\B\X5:Subroutines\X${}\mathrel+\E{}$\6
\&{void} \\{write\_it}(\&{int} \\{bytes})\1\1\2\2\6
${}\{{}$\1\6
\&{register} \&{int} \|k;\6
\&{register} \&{unsigned} \&{int} \|s;\7
\&{if} ${}(\\{bytes\_out}\G\\{filelength\_threshold}){}$\5
${}\{{}$\1\6
\&{if} ${}(\\{bytes\_out}\I\\{filelength\_threshold}){}$\1\5
\\{panic}(\.{"Improper\ buffer\ siz}\)\.{e"});\2\6
\\{close\_it}(\,);\6
${}\\{file\_extension}\PP;{}$\6
\\{open\_it}(\,);\6
\4${}\}{}$\2\6
\&{if} ${}(\\{fwrite}({\AND}\\{out}.\\{buf},\39{}$\&{sizeof}(\&{unsigned} %
\&{char})${},\39\\{bytes},\39\\{out\_file})\I\\{bytes}){}$\1\5
\\{panic}(\.{"Bad\ write"});\2\6
${}\\{bytes\_out}\MRL{+{\K}}\\{bytes};{}$\6
\&{for} ${}(\|k\K\|s\K\T{0};{}$ ${}\|k<\\{bytes};{}$ ${}\|k\PP){}$\1\5
${}\|s\K(\|s\LL\T{1})+\\{out}.\\{buf}[\|k];{}$\2\6
${}\\{checksum}\MRL{+{\K}}\|s;{}$\6
\4${}\}{}$\2\par
\fi

\M{12}A four-byte instruction has the binary form $(0xaaaaaa)_2$,
$(bbbbbbbb)_2$,
$(cccccccc)_2$, $(dddddddd)_2$, where
$(aaaaaabbbbbbbbccccccccdddddddd)_2$ is a 30-bit address specified
in big-endian fashion.
If $x=0$ it means, ``This is the new source address $s$.''
If $x=1$ it means, ``This is the new target address $t$.''

A one-byte instruction has the binary form $(1ooopppp)_2$, with a 3-bit
opcode $(ooo)_2$ and a 4-bit parameter $(pppp)_2$. If the parameter is zero,
the following byte is regarded as an 8-bit parameter $(pppppppp)_2$, and
it should not be zero. (In that case the ``one-byte instruction'' actually
occupies two bytes.)

In the instruction definitions below, $p$ stands for the parameter,
$s$ stands for the current source address, and $t$ stands for the
current target address. The slave processor operates on a large
array called \PB{\\{count}}.

Opcode 0 (\PB{\\{sync}}) means, ``We have just finished row \PB{\|p}.'' A
report
is given to the user.

Opcode 1 (\PB{\\{clear}}) means, ``Set \PB{$\\{count}[\|t+\|j]\K\T{0}$} for $0%
\le j< p$.''

Opcode 2 (\PB{\\{copy}}) means, ``Set \PB{$\\{count}[\|t+\|j]\K\\{count}[\|s+%
\|j]$} for $0\le j<p$.''

Opcode 3 (\PB{\\{add}}) means, ``Set \PB{$\\{count}[\|t+\|j]\MRL{+{\K}}%
\\{count}[\|s+\|j]$} for $0\le j<p$.''

Opcode 4 (\PB{\\{inc\_src}}) means, ``Set \PB{$\|s\MRL{+{\K}}\|p$}.''

Opcode 5 (\PB{\\{dec\_src}}) means, ``Set \PB{$\|s\MRL{-{\K}}\|p$}.''

Opcode 6 (\PB{\\{inc\_trg}}) means, ``Set \PB{$\|t\MRL{+{\K}}\|p$}.''

Opcode 7 (\PB{\\{dec\_trg}}) means, ``Set \PB{$\|t\MRL{-{\K}}\|p$}.''

\Y\B\4\D$\\{targ\_bit}$ \5
\T{\^40000000}\C{ specifies \PB{\|t} in a four-byte instruction }\par
\Y\B\4\X12:Type definitions\X${}\E{}$\6
\&{typedef} \&{enum} ${}\{{}$\1\6
${}\\{sync},\39\\{clear},\39\\{copy},\39\\{add},\39\\{inc\_src},\39\\{dec%
\_src},\39\\{inc\_trg},\39\\{dec\_trg}{}$\2\6
${}\}{}$ \&{opcode};\par
\As26, 28, 29, 33, 54, 60\ETs61.
\U4.\fi

\M{13}\B\D$\\{end\_of\_buffer}$ \5
${\AND}\\{out}.\\{buf}{}$[\\{buf\_size}]\par
\Y\B\4\X5:Subroutines\X${}\mathrel+\E{}$\6
\&{void} \\{put\_inst}(\&{unsigned} \&{char} \|o${},\39{}$\&{unsigned} \&{char}
\|p)\1\1\2\2\6
${}\{{}$\1\6
\&{register} \&{unsigned} \&{char} ${}{*}\|b\K\\{buf\_ptr};{}$\7
${}{*}\|b\PP\K\T{\^80}+(\|o\LL\T{4})+(\|p<\T{16}\?\|p:\T{0});{}$\6
\&{if} ${}(\|p\G\T{16}){}$\1\5
${}{*}\|b\PP\K\|p;{}$\2\6
\&{if} ${}(\|b\G\\{end\_of\_buffer}){}$\5
${}\{{}$\1\6
\\{write\_it}(\\{buf\_size});\6
${}\\{out}.\\{buf}[\T{0}]\K\\{out}.\\{buf}[\\{buf\_size}];{}$\6
${}\|b\MRL{-{\K}}\\{buf\_size};{}$\6
\4${}\}{}$\2\6
${}\\{buf\_ptr}\K\|b;{}$\6
\4${}\}{}$\2\par
\fi

\M{14}\B\X5:Subroutines\X${}\mathrel+\E{}$\6
\&{void} \\{put\_four}(\&{register} \&{unsigned} \&{int} \|x)\1\1\2\2\6
${}\{{}$\1\6
\&{register} \&{unsigned} \&{char} ${}{*}\|b\K\\{buf\_ptr};{}$\7
${}{*}\|b\K\|x\GG\T{24};{}$\6
${}{*}(\|b+\T{1})\K(\|x\GG\T{16})\AND\T{\^ff};{}$\6
${}{*}(\|b+\T{2})\K(\|x\GG\T{8})\AND\T{\^ff};{}$\6
${}{*}(\|b+\T{3})\K\|x\AND\T{\^ff};{}$\6
${}\|b\MRL{+{\K}}\T{4};{}$\6
\&{if} ${}(\|b\G\\{end\_of\_buffer}){}$\5
${}\{{}$\1\6
\\{write\_it}(\\{buf\_size});\6
${}\\{out}.\\{buf}[\T{0}]\K\\{out}.\\{buf}[\\{buf\_size}];{}$\6
${}\\{out}.\\{buf}[\T{1}]\K\\{out}.\\{buf}[\\{buf\_size}+\T{1}];{}$\6
${}\\{out}.\\{buf}[\T{2}]\K\\{out}.\\{buf}[\\{buf\_size}+\T{2}];{}$\6
${}\|b\MRL{-{\K}}\\{buf\_size};{}$\6
\4${}\}{}$\2\6
${}\\{buf\_ptr}\K\|b;{}$\6
\4${}\}{}$\2\par
\fi

\M{15}The first six bytes of the instruction file are, however, special.
Byte~0 is the number $n$ of cells in the largest polyominoes being
enumerated. When a \PB{\\{sync}} is interpreted, {\mc POLYSLAVE}
outputs the current values of \PB{\\{count}[\|j]} for $1\le j\le n$.

Byte 1 is the number of the final row. If this number is $r$, {\mc
POLYSLAVE} will terminate after interpreting the instruction \PB{\\{sync}}~$r$.

Bytes 2--5 specify the (big-endian) number of elements in the \PB{\\{count}}
array.

Initially \PB{$\|s\K\|t\K\T{0}$}, \PB{$\\{count}[\T{0}]\K\T{1}$}, and \PB{%
\\{count}[\|j]} is assumed to be zero
for $1\le j\le n$.

\Y\B\4\X15:Initialize\X${}\E{}$\6
$\\{sprintf}(\\{filename},\39\.{"\%.90s.ck"},\39\\{base\_name});{}$\6
${}\\{ck\_file}\K\\{fopen}(\\{filename},\39\.{"wb"});{}$\6
\&{if} ${}(\R\\{ck\_file}){}$\1\5
\\{panic}(\.{"I\ can't\ open\ the\ ch}\)\.{ecksum\ file"});\2\6
\\{open\_it}(\,);\6
${}\\{out}.\\{buf}[\T{0}]\K\|n;{}$\6
${}\\{out}.\\{buf}[\T{1}]\K\|n+\T{2}-\|w;{}$\6
${}\\{buf\_ptr}\K{\AND}\\{out}.\\{buf}[\T{2}];{}$\6
\\{put\_four}(\\{slave\_size});\par
\As58, 62, 66\ETs77.
\U4.\fi

\M{16}Here's what we'll do when it's all over.

\Y\B\4\X16:Empty the buffer and close the output file\X${}\E{}$\6
\&{if} ${}(\\{buf\_ptr}\I{\AND}\\{out}.\\{buf}[\T{0}]){}$\1\5
${}\\{write\_it}(\\{buf\_ptr}-{\AND}\\{out}.\\{buf}[\T{0}]);{}$\2\6
\\{close\_it}(\,);\par
\Us4\ET106.\fi

\M{17}Most of the output is generated by the \PB{\\{basic\_inst}} routine.

\Y\B\4\X5:Subroutines\X${}\mathrel+\E{}$\6
\&{void} \\{basic\_inst}(\&{int} \\{op}${},\39{}$\&{int} \\{src\_addr}${},%
\39{}$\&{int} \\{trg\_addr}${},\39{}$\&{unsigned} \&{char} \\{count})\1\1\2\2\6
${}\{{}$\1\6
\&{register} \&{int} \\{del};\7
\&{if} ${}(\\{verbose}>\T{1}){}$\5
${}\{{}$\1\6
\&{if} ${}(\\{op}\E\\{clear}){}$\1\5
${}\\{printf}(\.{"\{clear\ \%d\ ->\%d\}\\n"},\39\\{count},\39\\{trg\_addr});{}$%
\2\6
\&{else}\1\5
${}\\{printf}(\.{"\{\%s\ \%d\ \%d->\%d\}\\n"},\39\\{sym}[\\{op}],\39\\{count},%
\39\\{src\_addr},\39\\{trg\_addr});{}$\2\6
\4${}\}{}$\2\6
${}\\{del}\K\\{src\_addr}-\\{cur\_src};{}$\6
\&{if} ${}(\\{del}>\T{0}\W\\{del}<\T{256}){}$\1\5
${}\\{put\_inst}(\\{inc\_src},\39\\{del});{}$\2\6
\&{else} \&{if} ${}(\\{del}<\T{0}\W\\{del}>{-}\T{256}){}$\1\5
${}\\{put\_inst}(\\{dec\_src},\39{-}\\{del});{}$\2\6
\&{else} \&{if} (\\{del})\1\5
\\{put\_four}(\\{src\_addr});\2\6
${}\\{cur\_src}\K\\{src\_addr};{}$\6
${}\\{del}\K\\{trg\_addr}-\\{cur\_trg};{}$\6
\&{if} ${}(\\{del}>\T{0}\W\\{del}<\T{256}){}$\1\5
${}\\{put\_inst}(\\{inc\_trg},\39\\{del});{}$\2\6
\&{else} \&{if} ${}(\\{del}<\T{0}\W\\{del}>{-}\T{256}){}$\1\5
${}\\{put\_inst}(\\{dec\_trg},\39{-}\\{del});{}$\2\6
\&{else} \&{if} (\\{del})\1\5
${}\\{put\_four}(\\{trg\_addr}+\\{targ\_bit});{}$\2\6
${}\\{cur\_trg}\K\\{trg\_addr};{}$\6
${}\\{put\_inst}(\\{op},\39\\{count});{}$\6
\4${}\}{}$\2\par
\fi

\M{18}\B\X7:Global variables\X${}\mathrel+\E{}$\6
\&{char} ${}{*}\\{sym}[\T{4}]\K\{\.{"sync"},\39\.{"clear"},\39\.{"copy"},\39%
\.{"add"}\};{}$\6
\&{int} \\{cur\_src}${},{}$ \\{cur\_trg};\C{ current source and target
addresses in the slave }\6
\&{int} \\{verbose}${}\K\T{0}{}$;\C{ set nonzero when debugging }\par
\fi

\N{1}{19}Connectivity. The hardest task that confronts us is to figure out
how to determine the cutoff threshold:
Given a configuration like \.{0(00(00100-010-\caret/)00()0)0(-0},
what is the minimum number of additional cells that are needed to connect
it up and to make it stretch out to at least a given number of further rows?
We claimed above, without proof, that this particular configuration
needs at least 16 more cells before it will be connected and touch
the left boundary. Now we want to prove that claim, and solve the general
problem as~well.

Some cases of this problem are easy. For example, let's consider first
the case when we are at the beginning or end of a complete row. Then
it is clear that a configuration like \.{00-)0(0--)00(--0} needs
at least $3+4$ more cells to become occupied. [Well, if this {\it isn't\/}
clear, please stop now and think about it until it {\it is}. Remember
that it stands for a pattern with three connected components of
occupied cells; the left component is connected to the left edge,
the right component is connected to the right edge, and
the middle component is standing alone.]

Suppose we have a pattern like $\.0^{g_0}\alpha_1\.0^{g_1}\alpha_2\ldots
\.0^{g_{k-1}}\alpha_k\.0^{g_k}$, where each $\alpha_j$ is a separate
component beginning with \.( and ending with \.). For example,
a typical $\alpha$ might be \.{()} or \.{(-)} or \.{(-00--00-0-0)}, etc.
Again the problem we face is easily solved: We need to occupy $g_0+1$ cells
in order to connect
$\alpha_1$ to the left edge, $g_j+2$ cells to connect $\alpha_j$ to
$\alpha_{j+1}$ for $1\le j<k$, and $g_k+1$ cells to connect $\alpha_k$ to the
right edge. The same formula holds if any $\alpha_j$ is simply \.1, denoting a
singleton component, except that we can subtract~1 for every such $\alpha_j$.
For example, the cost of connecting up \.{0100(0-)00010(0)0} is
$2+4+5+3+2-2$.

If $\alpha_1$ is already connected to the left edge, we save $g_0+1$,
but we cannot take the bonus if $\alpha_1$ is~\.1; a similar consideration
applies at the right.

\fi

\M{20}The situation gets more interesting when components are nested. Suppose,
for
example, that $\alpha_1$, $\alpha_2$, \dots, $\alpha_{k-1}$ are distinct,
but $\alpha_k$ is part of the same component as $\alpha_1$. Then we still
must pay $g_0+1$ to connect $\alpha_1$ at the left and $g_k+1$ to
connect $\alpha_k\equiv\alpha_1$ at the right; but in this new case we are
allowed to keep $\alpha_j$ disconnected from $\alpha_{j+1}$ for any single
choice of $j$ we like, in the range $1\le j<k$. That will save $g_j+2$
from the formula stated above, except that it will cost one or two bonus
points if $\alpha_j$ and/or $\alpha_{j+1}$ had length~1.
For example, to connect the configuration \.{0(00010(0)00-0)00}, which
has the form $\.0^1\alpha_1\.0^3\alpha_2\.0^1\alpha_3\.0^2\alpha_4\.0^2$
with $\alpha_1\equiv\alpha_4$ and potential bonuses at $\alpha_1$ and
$\alpha_2$, we have three options. Disconnecting $\alpha_1$ from $\alpha_2$
costs $2+0+3+4+3-0$; disconnecting $\alpha_2$ from $\alpha_3$ costs
$2+5+0+4+3-1$; disconnecting $\alpha_3$ from $\alpha_4$ costs
$2+5+3+0+3-2$. The third alternative is best, even though it doesn't
disconnect the largest gap~$\.0^3$, because it retains the 2 bonus points.

\fi

\M{21}Now look at the configuration \.{-00(010-010)00()00()00-}, which is
connected to left and right edges and which also contains the subcomponent
\.{(010-010)}. The best way to handle the subcomponent is to occupy 5 cells
below it, spanning the middle region \.{10-01}. But then we need
$4+4+4$ additional cells to connect up the whole diagram. If instead we
use 6~cells within the subcomponent, spanning \.{(01} at the left and \.{10)}
at the right, we need only $3+3+4$ additional cells to finish. Thus the
environment can affect the optimal behavior within a subcomponent.

These examples give us one way to think about the minimum connection cost for
the general pattern
$\.0^{g_0}\alpha_1\.0^{g_1}\alpha_2\ldots\.0^{g_{k-1}}\alpha_k\.0^{g_k}$,
when $\alpha_i$ is already connected to $\alpha_j$ for certain pairs $(i,j)$,
namely to start by charging $(g_0+1)+(g_1+2)+\cdots+(g_{k-1}+2)+(g_k+1)$
and then to deduct some of the terms for gaps $g_j$ that are legitimately
left unconnected: The term $(g_0+1)$ can be deducted if $\alpha_1$ is connected
to the left edge, and $(g_k+1)$ can be deducted if $\alpha_k$ is connected to
the
right edge. If $\alpha_i\equiv\alpha_j$ for some $j>i$ and if the
components $\alpha_{i+1}$, \dots, $\alpha_j$ are mutually disconnected, then
we are allowed to deduct any one of the terms $(g_i+2)$, \dots, $(g_{j-1}+2)$,
after which we can treat $\alpha_i\ldots\alpha_j$ as a {\it single\/} component
with respect to further deductions. Finally after choosing a subset of terms
to deduct, we get a bonus for each $\alpha_j$ of length~1 such that neither
$g_{j-1}$ nor $g_j$ were left disconnected. (Length~1 means that the code for
$\alpha_j$ is a single character, either \.1 or \.( or \.) or \.-.)

\fi

\M{22}A recursive strategy can be used to solve the minimum connectivity
problem
in linear time, but we must design it carefully because of the examples
considered earlier. The key idea will be to associate four costs $c_{ij}$ with
each subcomponent, where $0\le i,j\le1$. Cost $c_{ij}$ is the minimum number
of future occupied cells needed to connect everything up within the component,
with the further proviso that there is a cell below the leftmost cell if
$i=1$, and a cell below the rightmost cell if $j=1$. The $2\times2$ matrix
$(c_{ij})$ will then represent all we need to know about connecting this
component at a higher level.

\vskip1pt
\def\mx#1#2#3#4{{#1\,#2\choose#3\,#4}}
For example, the cost matrix for a single-character component is
$\mx0111$, and the cost matrix for a multi-character component
with no internal subcomponents is $\mx0112$. (The 0 in the upper
left corner signifies that we don't need any cells to connect it up,
since it's already connected. But we may have to occupy one or two cells
as ``hooks'' at the left and/or the right if the environment wants them.)

\vskip1pt
\def\gmp#1#2{\mathbin{^#1_#2}}
The theory that underlies the algorithm below is best understood in
terms of {\it min-plus matrix multiplication\/} $C=AB$, where $c_{ij}=
\min_k(a_{ik}+b_{kj})$. (I should probably use a special symbol to denote this
multiplication, like $A\gmp\land+B$ instead of $AB$; see, for example,
exercise 1.3.1$'$--32 in {\sl The Art of Computer Programming}, Fascicle~1.
But that would clutter up a huge number of the formulas below. And I have
no use for ordinary matrix multiplication in the present program.
Therefore min-plus multiplication will be assumed to need no special marking
in the following discussion.)

\vskip1pt
\def\mtx#1#2#3#4{\hbox{$\bigl({#1\atop#2}\ {#3\atop#4}\bigr)$}}
Let $X_g$ be the matrix \mtx\infty\infty\infty g. Then if $\alpha_1$, \dots,
$\alpha_k$ are distinct subcomponents with cost matrices $A_1$, \dots,~$A_k$,
respectively, the cost of connecting up
$\.0^{g_0}\alpha_1\.0^{g_1}\alpha_2\ldots\.0^{g_{k-1}}\alpha_k\.0^{g_k}$
and touching the left and right edges is the upper left corner entry of
$$OX_{g_0}A_1X_{g_1}\ldots X_{g_{k-1}}A_kX_{g_k}O,\qquad \qquad
\hbox{where $O=\mtx0000$},$$
because matrix $X_g$ essentially means ``occupy $g$ cells below and
require the cells to the left and right of these cells to be occupied as
well.'' Notice that this rule yields
$(g_0+1)+(g_1+2)+\cdots+(g_{k-1}+2)+(g_k+1)$
in the special case when each matrix $A_j$ is $\mx0112$, and there's a
bonus of~1 whenever we replace $A_j$ by $\mx0111$. Notice also
that $X_gX_h=X_{g+h}$; thus we can essentially think of each \.0 in
the configuration as a multiplication by~$X_1$.

\fi

\M{23}If $\alpha$ and $\beta$ are subcomponents that are already connected to
each
other, having cost matrices $A$ and~$B$ respectively, the cost matrix for
$\alpha\,\.0^g\beta$ as a single component is $ANB$, where
$N=\mtx0\infty\infty\infty$ is the matrix meaning ``don't occupy anything
below the $\.0^g$ and don't insist that the cells to the left and right of that
gap must be occupied.'' For example, this rule gives
$\mx0111N\mx0111=\mx0112$ as it should.

And in general, suppose $\alpha_0$, $\alpha_1$, \dots, $\alpha_k$ are
subcomponents that are distinct except that $\alpha_0$ is already connected to
$\alpha_k$. Then the cost matrix for
$\alpha_0\.0^{g_1}\alpha_1\ldots\alpha_{k-1}\.0^{g_k}\alpha_k$
as a single subcomponent
can be expressed in terms of the individual cost matrices $A_0$, $A_1$, \dots,
$A_k$ by using the formula
$$\min_{1\le j\le k}\bigl(A_0X_{g_1}\ldots X_{g_{j-1}}A_{j-1}NA_j
X_{g_{j+1}}\ldots X_{g_k}A_k\bigr),   \eqno(**)$$
because we are allowed to leave any one of the gaps $\.0^{g_j}$
disconnected. Rule $(**)$ is the basic principle that makes a
recursive algorithm work.

\fi

\M{24}Our discussion so far has assumed for simplicity that we're at the end
of a row. But we also need to consider the case that a gap $\.0^g$ might be
$\.0^a\caret/\.0^b$ in the vicinity of the place where a partial row is being
filled. Fortunately the same theory applies with only a slight variation:
Instead of $X_g=X_{a+b}$, we use $X_{a+b+1}$ if $a>0$, or
we use $Y_b=\mtx\infty\infty b\infty$ if $a=0$; either $a$ or $b$ or both
might be zero. Matrix $Y_b$ means, ``Occupy the cells below $\.0^b$
and require the next cell on the right (but not necessarily the left)
to be occupied as well.'' The extra~1 in $X_{a+b+1}$ is needed when $a>0$
because one of the newly occupied cells must pass through the current
row~$r$ in order to reach row~$r+1$.

\fi

\M{25}After calculating the minimum cost of connection, we might also need to
add the cost of extension to span a $w\times w$ square. Thus, if row~$r$
is partially filled and row $r-1$ is completely filled, and if $r<w$,
we must add $w-r-1$ cells if the minimum connection can be achieved with at
least one cell in row~$r+1$; otherwise we must add $w-r$ cells.

Consider, for example, the configuration \.{-0\caret/010)00}. The best way
to connect this up is to occupy the five cells in row~$r$ that occur below the
\.{10)00}. But then we need $w-r$ more for extension. On the other hand,
in a configuration like \.{-0\caret/010000)00}, there are two essentially
different ways to do the connection with eight more cells, and we must choose
the one that uses cells of row~$r+1$. (A configuration like
\.{-0\caret/01000)00} can be connected in seven cells without using
row $r+1$, or in eight with the use of that row, so it's a tossup.)

The configurations \.{-01\caret/00-} and \.{-01\caret/00)0(} also
present interesting tradeoffs that our algorithm must handle properly.

The solution adopted here is to add $\epsilon$ to the cell costs after
the \qcaret/, so that newly occupied cells in row~$r$
are slightly more expensive than cells in row~$r+1$. This makes the
latter cells more attractive, in cases when they need to~be.

\fi

\M{26}OK, we've got a reasonably clean theory; now it's time to put it into
practice. The first step is to define the encoding of our alphabet
\.0, \.1, \.(, \.), and \.-, to which we'll add an ``edge of line''
character. The following encoding is designed to make it easy to test
whether a character is either \.- or \.):

\Y\B\4\D$\\{mid\_or\_rt}(\|x)$ \5
$(((\|x)\AND\T{2})\E\T{2}{}$)\par
\Y\B\4\X12:Type definitions\X${}\mathrel+\E{}$\6
\&{typedef} \&{enum} ${}\{{}$\1\6
${}\\{zero},\39\\{one},\39\\{rt},\39\\{mid},\39\\{lft},\39\\{eol}{}$\2\6
${}\}{}$ \&{code};\C{ \.0, \.1, \.), \.-, \.(, or edge delimiter in a
configuration string }\par
\fi

\M{27}\B\X7:Global variables\X${}\mathrel+\E{}$\6
\&{char} \\{decode}[\T{5}]${}\K\{\.{'0'},\39\.{'1'},\39\.{')'},\39\.{'-'},\39%
\.{'('}\};{}$\6
\&{code} \\{reflect}[\T{5}]${}\K\{\\{zero},\39\\{one},\39\\{lft},\39\\{mid},\39%
\\{rt}\}{}$;\par
\fi

\M{28}Then in the cost matrices and in our answer describing the minimum
connection cost of a given configuration, we will represent numbers
$a+b\epsilon$ as short integers, knowing that $a$ and $b$ will never exceed 27.

\Y\B\4\D$\\{unity}$ \5
$(\T{1}\LL\T{8}{}$)\par
\B\4\D$\\{epsilon}$ \5
\T{1}\par
\B\4\D$\\{uunity}$ \5
$(\\{unity}+\\{epsilon}{}$)\par
\B\4\D$\\{int\_part}(\|x)$ \5
$((\|x)\GG\T{8}{}$)\par
\B\4\D$\\{eps\_part}(\|x)$ \5
$((\|x)\AND\T{\^ff}{}$)\par
\Y\B\4\X12:Type definitions\X${}\mathrel+\E{}$\6
\&{typedef} \&{unsigned} \&{short} \&{cost};\C{ $a+b\epsilon$ represented as %
\PB{$(\|a\LL\T{8})+\|b$} }\7
\&{typedef} \&{struct} ${}\{{}$\1\6
\&{cost} \|c[\T{2}][\T{2}];\2\6
${}\}{}$ \&{cost\_matrix};\par
\fi

\M{29}We will need to distinguish between the cost matrices
$X_g=\mtx\infty\infty\infty {g_{\mathstrut}}$
and $Y_g=\mtx\infty\infty g\infty$,
as well as a third case that arises when a configuration is already
connected to the left edge.

\Y\B\4\X12:Type definitions\X${}\mathrel+\E{}$\6
\&{typedef} \&{enum} ${}\{{}$\1\6
${}\\{ytyp},\39\\{xtyp},\39\\{otyp}{}$\2\6
${}\}{}$ \&{gap\_type};\par
\fi

\M{30}A few easy subroutines do the basic operations on cost matrices that we
will
need.

\Y\B\4\X5:Subroutines\X${}\mathrel+\E{}$\6
\&{cost\_matrix} \\{a\_n\_b}(\&{cost\_matrix} \|a${},\39{}$\&{cost\_matrix} %
\|b)\C{ computes $A N B$ }\6
${}\{{}$\1\6
\&{cost\_matrix} \|c;\7
${}\|c.\|c[\T{0}][\T{0}]\K\|a.\|c[\T{0}][\T{0}]+\|b.\|c[\T{0}][\T{0}];{}$\6
${}\|c.\|c[\T{0}][\T{1}]\K\|a.\|c[\T{0}][\T{0}]+\|b.\|c[\T{0}][\T{1}];{}$\6
${}\|c.\|c[\T{1}][\T{0}]\K\|a.\|c[\T{1}][\T{0}]+\|b.\|c[\T{0}][\T{0}];{}$\6
${}\|c.\|c[\T{1}][\T{1}]\K\|a.\|c[\T{1}][\T{0}]+\|b.\|c[\T{0}][\T{1}];{}$\6
\&{return} \|c;\6
\4${}\}{}$\2\par
\fi

\M{31}\B\X5:Subroutines\X${}\mathrel+\E{}$\6
\&{cost\_matrix} \\{a\_x\_b}(\&{cost\_matrix} \|a${},\39{}$\&{cost\_matrix} %
\|b${},\39{}$\&{gap\_type} \\{typ}${},\39{}$\&{cost} \|g)\C{ computes $A X_{g%
\,}B$, $A Y_{g\,} B$,  or $B$, depending on \PB{\\{typ}} }\6
${}\{{}$\1\6
\&{cost\_matrix} \|c;\7
\&{if} ${}(\\{typ}\E\\{otyp}){}$\1\5
\&{return} \|b;\2\6
${}\|c.\|c[\T{0}][\T{0}]\K\|a.\|c[\T{0}][\\{typ}]+\|g+\|b.\|c[\T{1}][\T{0}];{}$%
\6
${}\|c.\|c[\T{0}][\T{1}]\K\|a.\|c[\T{0}][\\{typ}]+\|g+\|b.\|c[\T{1}][\T{1}];{}$%
\6
${}\|c.\|c[\T{1}][\T{0}]\K\|a.\|c[\T{1}][\\{typ}]+\|g+\|b.\|c[\T{1}][\T{0}];{}$%
\6
${}\|c.\|c[\T{1}][\T{1}]\K\|a.\|c[\T{1}][\\{typ}]+\|g+\|b.\|c[\T{1}][\T{1}];{}$%
\6
\&{return} \|c;\6
\4${}\}{}$\2\par
\fi

\M{32}\B\X5:Subroutines\X${}\mathrel+\E{}$\6
\&{void} \\{min\_mat}(\&{cost\_matrix} ${}{*}\|a,\39{}$\&{cost\_matrix} \|b)\C{
sets $A=\min(A,B)$ }\6
${}\{{}$\1\6
\&{if} ${}(({*}\|a).\|c[\T{0}][\T{0}]>\|b.\|c[\T{0}][\T{0}]){}$\1\5
${}({*}\|a).\|c[\T{0}][\T{0}]\K\|b.\|c[\T{0}][\T{0}];{}$\2\6
\&{if} ${}(({*}\|a).\|c[\T{0}][\T{1}]>\|b.\|c[\T{0}][\T{1}]){}$\1\5
${}({*}\|a).\|c[\T{0}][\T{1}]\K\|b.\|c[\T{0}][\T{1}];{}$\2\6
\&{if} ${}(({*}\|a).\|c[\T{1}][\T{0}]>\|b.\|c[\T{1}][\T{0}]){}$\1\5
${}({*}\|a).\|c[\T{1}][\T{0}]\K\|b.\|c[\T{1}][\T{0}];{}$\2\6
\&{if} ${}(({*}\|a).\|c[\T{1}][\T{1}]>\|b.\|c[\T{1}][\T{1}]){}$\1\5
${}({*}\|a).\|c[\T{1}][\T{1}]\K\|b.\|c[\T{1}][\T{1}];{}$\2\6
\4${}\}{}$\2\par
\fi

\M{33}The algorithm we use is inherently recursive, but we implement it
iteratively using a stack because it involves only simple algebraic
operations. Each stack entry typically represents a string of \.0's
and a subcomponent that has not been completely scanned as yet;
the string of \.0's is represented by a cost matrix $X_g$ or $Y_g$,
and the incomplete subcomponent is represented by a partial evaluation
of formula $(**)$.

At stack level~0, however, there is no incomplete subcomponent and
only the \PB{\\{closed\_cost}} field is relevant. This field corresponds to
the cost matrix of all components scanned so far.

\Y\B\4\X12:Type definitions\X${}\mathrel+\E{}$\6
\&{typedef} \&{struct} ${}\{{}$\1\6
\&{cost} \\{gap};\C{ zeros to cover in the previous gap }\6
\&{gap\_type} \\{gap\_typ};\C{ type of previous gap (\PB{\\{xtyp}} or \PB{%
\\{ytyp}} or \PB{\\{otyp}}) }\6
\&{cost\_matrix} \\{closed\_cost};\C{ cost matrix with no gaps }\6
\&{cost\_matrix} \\{open\_cost};\C{ cost matrix with optional gap }\2\6
${}\}{}$ \&{stack\_entry};\par
\fi

\M{34}The given configuration string will appear in \PB{\|c[\T{1}]} through %
\PB{\|c[\|w]};
also \PB{\|c[\T{0}]} and \PB{$\|c[\|w+\T{1}]$} will be set to \PB{\\{eol}}.

\Y\B\4\X7:Global variables\X${}\mathrel+\E{}$\6
\&{code} \|c[\T{64}];\C{ codes of the current configuration string }\6
\&{stack\_entry} \\{stk}[\T{64}];\C{ partially evaluated costs }\par
\fi

\M{35}This program operates in two phases that are almost identical:
Before \PB{\\{row\_end}} has been sensed, the cost of a connection cell is
\PB{\\{unity}}, but afterwards it is \PB{$\\{unity}+\\{epsilon}$}. In order to
streamline
the code I'm using a little trick explained in Example~7 of my paper
``Structured programming with \PB{\&{goto}} statements'': I make two almost
identical copies of the code, one for the actions to be taken before
\PB{$\|k\E\\{row\_end}$} and one for the actions to be taken subsequently.
The first copy jumps to the second as soon as the condition \PB{$\|k\E\\{row%
\_end}$} is
sensed. This avoids all kinds of conditional coding and makes ``variables''
into ``constants,'' although it does have the somewhat disconcerting feature
of jumping from one loop into the body of another.

The program could be made faster if I would look at high speed for
special cases like subcomponents of the form \.{(0--00-)}, since
that subcomponent is equivalent to \.{()} with respect to the
connectivity measure we are computing. But I~intentially avoided tricky
optimizations in order to keep this program simpler and easier to verify.

On the other hand, I do
reduce \.0-less forms like \.{(---)} to \.{()},
and \.{(-\caret/--)} to \.{(\caret/)}, etc.
Such optimizations aren't strictly necessary but I found them irresistible,
because they arise so frequently.

Variable \PB{\\{row\_end}} in the following routine points to the character
just
following the \qcaret/.

\Y\B\4\X5:Subroutines\X${}\mathrel+\E{}$\6
\&{cost} \\{connectivity}(\&{register} \&{int} \\{row\_end})\1\1\2\2\6
${}\{{}$\1\6
\&{register} \&{int} \|k;\C{ our place in the string }\6
\&{register} \&{int} \|s;\C{ the number of open items on the stack }\6
\&{int} \|g;\C{ the current gap size }\6
\&{gap\_type} \\{typ};\C{ its type }\6
\&{int} \\{open};\C{ set nonzero if previous token was \.( or \.- }\7
\X36:Get ready to compute connectivity\X;\6
\4\\{scan\_zeros0}:\5
\X37:Scan for zeros in Phase 0\X;\6
\4\\{scan\_tokens0}:\5
\X38:Scan a nonzero token cluster in Phase 0\X;\6
\4\\{scan\_zeros1}:\5
\X48:Scan for zeros in Phase 1\X;\6
\4\\{scan\_tokens1}:\5
\X49:Scan a nonzero token cluster in Phase 1\X;\6
\X52:Finish the connectivity bound calculation and return the answer\X;\6
\4${}\}{}$\2\par
\fi

\M{36}In practice \PB{\\{row\_end}} will not be zero. But I decided to make the
algorithm
general enough to work correctly also in that case, because only one more
line of code was needed.

\Y\B\4\X36:Get ready to compute connectivity\X${}\E{}$\6
$\|s\K\\{open}\K\T{0};{}$\6
${}\\{stk}[\T{0}].\\{closed\_cost}\K\\{zero\_cost};{}$\6
${}\|k\K\T{1};{}$\6
\&{if} (\\{mid\_or\_rt}(\|c[\T{1}]))\5
${}\{{}$\1\6
${}\|g\K\T{0},\39\\{typ}\K\\{xtyp}{}$;\C{ \PB{\\{xtyp}} and \PB{\\{ytyp}} are
equivalent at the left edge }\6
\&{if} ${}(\\{row\_end}\E\T{1}){}$\1\5
\&{goto} \\{scan\_tokens1};\2\6
\&{else}\1\5
\&{goto} \\{scan\_tokens0};\2\6
\4${}\}{}$\2\par
\U35.\fi

\M{37}\B\X37:Scan for zeros in Phase 0\X${}\E{}$\6
\&{if} ${}(\|k\E\\{row\_end}){}$\5
${}\{{}$\1\6
${}\|g\K\T{0},\39\\{typ}\K\\{ytyp};{}$\6
\&{goto} \\{scan\_zeros1x};\6
\4${}\}{}$\2\6
\&{if} (\|c[\|k])\1\5
\\{panic}(\.{"Syntax\ error,\ 0\ exp}\)\.{ected"});\2\6
${}\\{typ}\K\\{xtyp},\39\|g\K\\{unity},\39\|k\PP;{}$\6
\&{while} ${}(\|c[\|k]\E\T{0}){}$\5
${}\{{}$\1\6
\&{if} ${}(\|k\E\\{row\_end}){}$\5
${}\{{}$\1\6
${}\|g\MRL{+{\K}}\\{unity}+\\{uunity},\39\|k\PP{}$;\C{ correction for
straddling rows }\6
\&{goto} \\{scan\_zeros1x};\6
\4${}\}{}$\2\6
${}\|g\MRL{+{\K}}\\{unity},\39\|k\PP;{}$\6
\4${}\}{}$\2\6
\&{if} ${}(\|k\E\\{row\_end}){}$\5
${}\{{}$\1\6
${}\|g\MRL{+{\K}}\\{unity};{}$\6
\&{goto} \\{scan\_tokens1};\6
\4${}\}{}$\2\par
\U35.\fi

\M{38}\B\X38:Scan a nonzero token cluster in Phase 0\X${}\E{}$\6
$\\{cm}\K\\{base\_cost0};{}$\6
${}\|k\PP;$ \&{switch} ${}(\|c[\|k-\T{1}])$ $\{$ \6
\4\&{case} \\{lft}:\5
\&{if} ${}(\R\\{mid\_or\_rt}(\|c[\|k])){}$\1\5
\&{goto} \\{scan\_open0};\2\6
\X40:Compress $\.-^*\.)$ following \.( into a single token in Phase 0\X \6
\&{if} ${}(\|c[\|k-\T{1}]\I\\{rt}){}$\1\5
\&{goto} \\{scan\_open0};\2\6
\4\&{case} \\{one}:\5
\X42:Append \PB{\\{cm}} to the current partial component\X;\6
${}\\{open}\K\T{0}{}$;\5
\&{goto} \\{scan\_zeros0};\6
\4\&{case} \\{mid}:\5
\&{if} ${}(\R(\\{mid\_or\_rt}(\|c[\|k]))){}$\1\5
\&{goto} \\{scan\_mid0};\2\6
\X41:Compress $\.-^*\.)$ following \.- into a single token in Phase 0\X \6
\&{if} ${}(\|c[\|k-\T{1}]\I\\{rt}){}$\1\5
\&{goto} \\{scan\_mid0};\2\6
\4\&{case} \\{rt}:\5
\&{if} ${}(\R\|s){}$\5
${}\{{}$\1\6
\&{if} ${}(\\{stk}[\T{0}].\\{closed\_cost}.\|c[\T{1}][\T{1}]){}$\1\5
\\{panic}(\.{"Unmatched\ )"});\2\6
${}\\{stk}[\T{0}].\\{closed\_cost}\K\\{cm}{}$;\C{ already connected to left
edge }\6
\4${}\}{}$\5
\2\&{else}\5
${}\{{}$\1\6
\X43:Append \PB{\\{cm}} to \PB{$\\{stk}[\|s].\\{open\_cost}$}\X;\6
\X44:Combine the top two items on the stack\X;\6
\4${}\}{}$\2\6
${}\\{open}\K\T{0}{}$;\5
\&{goto} \\{scan\_zeros0};\6
\4\&{case} \\{eol}:\5
\&{goto} \\{scan\_eol0};\6
\4\&{default}:\5
\\{panic}(\.{"Illegal\ code"}); $\}$ \6
\4\\{scan\_open0}:\5
\X46:Finish processing \PB{\\{lft}}\X;\5
\&{goto} \\{check\_eol0};\6
\4\\{scan\_mid0}:\5
\X47:Finish processing \PB{\\{mid}}\X;\6
\4\\{check\_eol0}:\5
${}\\{open}\K\T{1};{}$\6
\&{if} ${}(\|c[\|k]\I\\{eol}){}$\1\5
\&{goto} \\{scan\_zeros0};\2\6
\&{if} ${}(\|k\E\\{row\_end}){}$\1\5
\&{goto} \\{scan\_eol1};\2\6
\4\\{scan\_eol0}:\5
\\{panic}(\.{"Row\ end\ missed"});\par
\U35.\fi

\M{39}\B\X7:Global variables\X${}\mathrel+\E{}$\6
\&{gap\_type} \\{typ};\C{ the current type of gap \PB{\|g} }\6
\&{cost\_matrix} \\{cm};\C{ the current cost matrix }\6
\&{cost\_matrix} \\{acm};\C{ another cost matrix }\7
\&{const} \&{cost\_matrix} \\{zero\_cost}${}\K\{\T{0},\39\T{0},\39\T{0},\39%
\T{0}\}{}$;\C{ $0\,0\choose0\,0$ }\6
\&{const} \&{cost\_matrix} \\{base\_cost0}${}\K\{\T{0},\39\\{unity},\39%
\\{unity},\39\\{unity}\}{}$;\C{ $0\,1\choose1\,1$ }\6
\&{const} \&{cost\_matrix} \\{base\_cost1}${}\K\{\T{0},\39\\{uunity},\39%
\\{uunity},\39\\{uunity}\}{}$;\C{ $\bigl({0\atop1+\epsilon}\ {1+\epsilon\atop1+%
\epsilon}\bigr)$ }\par
\fi

\M{40}\B\X40:Compress $\.-^*\.)$ following \.( into a single token in Phase 0%
\X${}\E{}$\6
${}\{{}$\1\6
\&{do}\5
${}\{{}$\1\6
\&{if} ${}(\|k\E\\{row\_end}){}$\1\5
\&{goto} \\{scan\_tokens1a};\2\6
${}\|k\PP;{}$\6
\&{if} ${}(\|c[\|k-\T{1}]\E\\{rt}){}$\1\5
\&{break};\2\6
\4${}\}{}$\5
\2\5
\&{while} (\\{mid\_or\_rt}(\|c[\|k]));\6
${}\\{cm}.\|c[\T{1}][\T{1}]\K\\{unity}+\\{unity}{}$;\C{ now \PB{\\{cm}} is $0%
\,1\choose1\,2$ }\6
\4${}\}{}$\2\par
\U38.\fi

\M{41}\B\X41:Compress $\.-^*\.)$ following \.- into a single token in Phase 0%
\X${}\E{}$\6
${}\{{}$\1\6
\&{do}\5
${}\{{}$\1\6
\&{if} ${}(\|k\E\\{row\_end}){}$\1\5
\&{goto} \\{scan\_tokens1b};\2\6
${}\|k\PP;{}$\6
\&{if} ${}(\|c[\|k-\T{1}]\E\\{rt}){}$\1\5
\&{break};\2\6
\4${}\}{}$\5
\2\5
\&{while} (\\{mid\_or\_rt}(\|c[\|k]));\6
${}\\{cm}.\|c[\T{1}][\T{1}]\K\\{unity}+\\{unity}{}$;\C{ now \PB{\\{cm}} is $0%
\,1\choose1\,2$ }\6
\4${}\}{}$\2\par
\U38.\fi

\M{42}\B\X42:Append \PB{\\{cm}} to the current partial component\X${}\E{}$\6
\&{if} (\|s)\1\5
\X43:Append \PB{\\{cm}} to \PB{$\\{stk}[\|s].\\{open\_cost}$}\X;\2\6
${}\\{stk}[\|s].\\{closed\_cost}\K\\{a\_x\_b}(\\{stk}[\|s].\\{closed\_cost},\39%
\\{cm},\39\\{typ},\39\|g){}$;\par
\Us38\ET49.\fi

\M{43}\B\X43:Append \PB{\\{cm}} to \PB{$\\{stk}[\|s].\\{open\_cost}$}\X${}\E{}$%
\6
${}\{{}$\1\6
${}\\{acm}\K\\{a\_n\_b}(\\{stk}[\|s].\\{closed\_cost},\39\\{cm});{}$\6
${}\\{stk}[\|s].\\{open\_cost}\K\\{a\_x\_b}(\\{stk}[\|s].\\{open\_cost},\39%
\\{cm},\39\\{typ},\39\|g);{}$\6
${}\\{min\_mat}({\AND}\\{stk}[\|s].\\{open\_cost},\39\\{acm});{}$\6
\4${}\}{}$\2\par
\Us38, 42, 47\ETs49.\fi

\M{44}\B\X44:Combine the top two items on the stack\X${}\E{}$\6
$\|s\MM;{}$\6
\&{if} (\|s)\1\5
\X45:Append \PB{$\\{stk}[\|s+\T{1}].\\{open\_cost}$} to \PB{$\\{stk}[\|s].%
\\{open\_cost}$}\X;\2\6
${}\\{stk}[\|s].\\{closed\_cost}\K\\{a\_x\_b}(\\{stk}[\|s].\\{closed\_cost},\39%
\\{stk}[\|s+\T{1}].\\{open\_cost},\39\\{stk}[\|s+\T{1}].\\{gap\_typ},\39%
\\{stk}[\|s+\T{1}].\\{gap}){}$;\par
\Us38, 49\ETs52.\fi

\M{45}\B\X45:Append \PB{$\\{stk}[\|s+\T{1}].\\{open\_cost}$} to \PB{$\\{stk}[%
\|s].\\{open\_cost}$}\X${}\E{}$\6
${}\{{}$\1\6
${}\\{acm}\K\\{a\_n\_b}(\\{stk}[\|s].\\{closed\_cost},\39\\{stk}[\|s+\T{1}].%
\\{open\_cost});{}$\6
${}\\{stk}[\|s].\\{open\_cost}\K\\{a\_x\_b}(\\{stk}[\|s].\\{open\_cost},\39%
\\{stk}[\|s+\T{1}].\\{open\_cost},\39\\{stk}[\|s+\T{1}].\\{gap\_typ},\39%
\\{stk}[\|s+\T{1}].\\{gap});{}$\6
${}\\{min\_mat}({\AND}\\{stk}[\|s].\\{open\_cost},\39\\{acm});{}$\6
\4${}\}{}$\2\par
\U44.\fi

\M{46}\B\X46:Finish processing \PB{\\{lft}}\X${}\E{}$\6
$\\{stk}[\PP\|s].\\{gap\_typ}\K\\{typ};{}$\6
${}\\{stk}[\|s].\\{gap}\K\|g;{}$\6
${}\\{stk}[\|s].\\{closed\_cost}\K\\{stk}[\|s].\\{open\_cost}\K\\{cm}{}$;\par
\Us38\ET49.\fi

\M{47}\B\X47:Finish processing \PB{\\{mid}}\X${}\E{}$\6
\&{if} ${}(\R\|s){}$\5
${}\{{}$\1\6
\&{if} ${}(\\{stk}[\T{0}].\\{closed\_cost}.\|c[\T{1}][\T{1}]){}$\1\5
\\{panic}(\.{"Unmatched\ -"});\2\6
${}\|s\K\T{1},\39\\{stk}[\T{1}].\\{gap\_typ}\K\\{otyp},\39\\{stk}[\T{1}].%
\\{closed\_cost}\K\\{stk}[\T{1}].\\{open\_cost}\K\\{cm};{}$\6
\4${}\}{}$\5
\2\&{else}\5
${}\{{}$\1\6
\X43:Append \PB{\\{cm}} to \PB{$\\{stk}[\|s].\\{open\_cost}$}\X;\6
${}\\{stk}[\|s].\\{closed\_cost}\K\\{stk}[\|s].\\{open\_cost};{}$\6
\4${}\}{}$\2\par
\Us38\ET49.\fi

\M{48}\B\X48:Scan for zeros in Phase 1\X${}\E{}$\6
\&{if} (\|c[\|k])\1\5
\\{panic}(\.{"Syntax\ error,\ 0\ exp}\)\.{ected"});\2\6
${}\\{typ}\K\\{xtyp},\39\|g\K\\{uunity},\39\|k\PP;{}$\6
\4\\{scan\_zeros1x}:\5
\&{while} ${}(\|c[\|k]\E\T{0}){}$\5
${}\{{}$\1\6
${}\|g\MRL{+{\K}}\\{uunity},\39\|k\PP;{}$\6
\4${}\}{}$\2\par
\U35.\fi

\M{49}\B\X49:Scan a nonzero token cluster in Phase 1\X${}\E{}$\6
$\\{cm}\K\\{base\_cost1};{}$\6
${}\|k\PP;$ \&{switch} ${}(\|c[\|k-\T{1}])$ $\{$ \6
\4\&{case} \\{lft}:\5
\&{if} ${}(\R\\{mid\_or\_rt}(\|c[\|k])){}$\1\5
\&{goto} \\{scan\_open1};\2\6
\X50:Compress $\.-^*\.)$ following \.( into a single token in Phase 1\X \6
\&{if} ${}(\|c[\|k-\T{1}]\I\\{rt}){}$\1\5
\&{goto} \\{scan\_open1};\2\6
\4\&{case} \\{one}:\5
\X42:Append \PB{\\{cm}} to the current partial component\X;\6
${}\\{open}\K\T{0}{}$;\5
\&{goto} \\{scan\_zeros1};\6
\4\&{case} \\{mid}:\5
\&{if} ${}(\R(\\{mid\_or\_rt}(\|c[\|k]))){}$\1\5
\&{goto} \\{scan\_mid1};\2\6
\X51:Compress $\.-^*\.)$ following \.- into a single token in Phase 1\X \6
\&{if} ${}(\|c[\|k-\T{1}]\I\\{rt}){}$\1\5
\&{goto} \\{scan\_mid1};\2\6
\4\&{case} \\{rt}:\5
\&{if} ${}(\R\|s){}$\5
${}\{{}$\1\6
\&{if} ${}(\\{stk}[\T{0}].\\{closed\_cost}.\|c[\T{1}][\T{1}]){}$\1\5
\\{panic}(\.{"Unmatched\ )"});\2\6
${}\\{stk}[\T{0}].\\{closed\_cost}\K\\{cm}{}$;\C{ already connected to left
edge }\6
\4${}\}{}$\5
\2\&{else}\5
${}\{{}$\1\6
\X43:Append \PB{\\{cm}} to \PB{$\\{stk}[\|s].\\{open\_cost}$}\X;\6
\X44:Combine the top two items on the stack\X;\6
\4${}\}{}$\2\6
${}\\{open}\K\T{0}{}$;\5
\&{goto} \\{scan\_zeros1};\6
\4\&{case} \\{eol}:\5
\&{goto} \\{scan\_eol1};\6
\4\&{default}:\5
\\{panic}(\.{"Illegal\ code"}); $\}$ \6
\4\\{scan\_open1}:\5
\X46:Finish processing \PB{\\{lft}}\X;\5
\&{goto} \\{check\_eol1};\6
\4\\{scan\_mid1}:\5
\X47:Finish processing \PB{\\{mid}}\X;\6
\4\\{check\_eol1}:\5
${}\\{open}\K\T{1};{}$\6
\&{if} ${}(\|c[\|k]\I\\{eol}){}$\1\5
\&{goto} \\{scan\_zeros1};\2\6
\4\\{scan\_eol1}:\C{ fall through to return the answer }\par
\U35.\fi

\M{50}\B\X50:Compress $\.-^*\.)$ following \.( into a single token in Phase 1%
\X${}\E{}$\6
${}\{{}$\1\6
\4\\{scan\_tokens1a}:\5
\&{do}\5
${}\{{}$\1\6
${}\|k\PP;{}$\6
\&{if} ${}(\|c[\|k-\T{1}]\E\\{rt}){}$\1\5
\&{break};\2\6
\4${}\}{}$\5
\2\5
\&{while} (\\{mid\_or\_rt}(\|c[\|k]));\6
${}\\{cm}.\|c[\T{0}][\T{1}]\K\\{uunity};{}$\6
${}\\{cm}.\|c[\T{1}][\T{1}]\MRL{+{\K}}\\{uunity}{}$;\C{ now \PB{\\{cm}} is
 $\bigl({0\atop1}\ {1+\epsilon\atop2+\epsilon}\bigr)$ or       $\bigl({0\atop1+%
\epsilon}\ {1+\epsilon\atop2+2\epsilon}\bigr)$ }\6
\4${}\}{}$\2\par
\U49.\fi

\M{51}\B\X51:Compress $\.-^*\.)$ following \.- into a single token in Phase 1%
\X${}\E{}$\6
${}\{{}$\1\6
\4\\{scan\_tokens1b}:\5
\&{do}\5
${}\{{}$\1\6
${}\|k\PP;{}$\6
\&{if} ${}(\|c[\|k-\T{1}]\E\\{rt}){}$\1\5
\&{break};\2\6
\4${}\}{}$\5
\2\5
\&{while} (\\{mid\_or\_rt}(\|c[\|k]));\6
${}\\{cm}.\|c[\T{0}][\T{1}]\K\\{uunity};{}$\6
${}\\{cm}.\|c[\T{1}][\T{1}]\MRL{+{\K}}\\{uunity}{}$;\C{ now \PB{\\{cm}} is
 $\bigl({0\atop1}\ {1+\epsilon\atop2+\epsilon}\bigr)$ or       $\bigl({0\atop1+%
\epsilon}\ {1+\epsilon\atop2+2\epsilon}\bigr)$ }\6
\4${}\}{}$\2\par
\U49.\fi

\M{52}The cost of reaching the right edge is \PB{\|g} if \PB{\|g} has the form
$b+b\epsilon$, but it is $a+b$ if \PB{\|g} has the exceptional form
$a+1+b+b\epsilon$ that arises when filling an unfilled row.

\Y\B\4\X52:Finish the connectivity bound calculation and return the answer\X${}%
\E{}$\6
\&{if} (\\{open})\5
${}\{{}$\1\6
\X44:Combine the top two items on the stack\X;\6
\&{if} (\|s)\1\5
\\{panic}(\.{"Missing\ )"});\2\6
\&{return} \\{stk}[\T{0}]${}.\\{closed\_cost}.\|c[\T{0}][\T{0}];{}$\6
\4${}\}{}$\5
\2\&{else}\5
${}\{{}$\C{ we need to reach the right edge }\1\6
\&{if} (\|s)\1\5
\\{panic}(\.{"Missing\ )"});\2\6
\&{if} ${}(\\{int\_part}(\|g)\E\\{eps\_part}(\|g)){}$\1\5
\&{return} \\{stk}[\T{0}]${}.\\{closed\_cost}.\|c[\T{0}][\\{typ}]+\|g;{}$\2\6
\&{return} \\{stk}[\T{0}]${}.\\{closed\_cost}.\|c[\T{0}][\T{1}]+((\\{int%
\_part}(\|g)-\T{1})\LL\T{8});{}$\6
\4${}\}{}$\2\par
\U35.\fi

\N{1}{53}Data structures. If we want to count $n$-ominoes efficiently for
large~$n$,
our most precious resource turns out to be the random-access memory that
is available. I think at least 20 bytes of memory are needed per active
configuration, and that limits us to about 50 million active configurations per
gigabyte of memory. Under such circumstances I don't mind recomputing results
several times in order to save space. For example, many configurations will
occur on several different rows, and this program recomputes their
connectivity cost each time they appear.

The main loop of our computation will consist of taking a viable configuration
$\alpha$ and looking at its two successors $\alpha_0$ and $\alpha_1$. The
value of \PB{\\{row\_end}} in $\alpha_0$ and $\alpha_1$ (the position of
\qcaret/) will be one greater than its value in~$\alpha$;
$\alpha_0$ will leave the new cell empty, but $\alpha_1$ will occupy it.

We gain time when $\alpha_0$ and/or $\alpha_1$ have been seen before;
a hash table helps us determine whether or not they are d\'ej\`a vu.
Newly seen configurations are subjected to the \PB{\\{connectivity}} bound, and
accepted into the computation only if they are found to be viable.

Each configuration $\alpha$ has an associated {\it generating function\/}
$g(\alpha)$. For example, the generating function $5z^8+2z^9$ would mean that
there are 5 ways to reach $\alpha$ with 8 cells occupied and 2 ways to reach
it with 9 cells occupied (so far). We will add $g(\alpha)$ to $g(\alpha_0)$ if
$\alpha_0$ is viable, and $zg(\alpha)$ to $g(\alpha_1)$ if $\alpha_1$ is
viable. After that we are able to forget $\alpha$ and $g(\alpha)$,
reclaiming precious memory space.

Indeed, we don't actually have enough space to deal with the generating
functions $g(\alpha)$. Therefore this program compiles and outputs a sequence
of instructions that will be interpreted by another program, {\mc POLYSLAVE};
that program will subsequently do the actual additions, without needing a hash
table or other space-hungry data.

The algorithm proceeds row by row, starting in row $r=1$, and continues until
no viable configurations remain. In each row it makes $w$ complete passes
over the existing configurations, with \PB{\\{row\_end}} running from 0 to
$w-1$.
Every such pass processes and discards all configurations that were generated
on the previous pass, and we are free to process them in any convenient
order. Therefore we adopt a ``rolling'' strategy that uses memory with
near-maximum efficiency: Suppose one pass has produced $M$ configurations
in the first $M$ slots $\alpha^{(1)}$, \dots, $\alpha^{(M)}$ of a memory pool
consisting of $N=\PB{\\{conf\_size}}$ total slots. We start by finding the
successors $\alpha^{(M)}_0$ and $\alpha^{(M)}_1$ of $\alpha^{(M)}$, putting
them into slots $N$ and $N-1$. Then slot~$M$ is free and we turn to
$\alpha^{(M-1)}$, etc., thereby running out of memory only if $N$
configurations are fully in use. On the following pass we reverse direction,
filling slots from 1~upwards instead of from $N$ downwards.

The same strategy makes it easy to allocate the memory space needed for
generating functions in the {\mc POLYSLAVE} program. Indeed, this rolling
allocation scheme fills memory almost perfectly, even though each generating
function occupies a variable number of bytes. (Sometimes holes do appear,
if a generating function turns out to need more bytes than we thought it
would. But the behavior in general is quite satisfactory.)

\fi

\M{54}Eight bytes suffice to encode any configuration $c_0\ldots c_{w-1}$ in a
5-letter alphabet, since $5^{27}<2^{64}<5^{28}$ and we are assuming that $w$ is
at most~27. We define \&{cstring} to a union type, so that the hash function
can readily access its individual bytes, yet packing and unpacking can be done
with radix 5 or 8. (The hash function will evaluate differently on a
big-endian machine versus a little-endian machine, but that doesn't matter.)

\Y\B\4\X12:Type definitions\X${}\mathrel+\E{}$\6
\&{typedef} \&{struct} ${}\{{}$\1\6
\&{unsigned} \&{int} \|h${},{}$ \|l;\C{ high-order and low-order halves }\2\6
${}\}{}$ \&{octa};\C{ two tetrabytes make one octabyte }\6
\&{typedef} \&{union} ${}\{{}$\1\6
\&{octa} \|o;\6
\&{unsigned} \&{char} \\{byte}[\T{8}];\2\6
${}\}{}$ \&{cstring};\C{ packed version of a configuration string }\par
\fi

\M{55}The \PB{\\{pack}} subroutine produces a \PB{\&{cstring}} from the codes
in array \PB{\|c}.
Since we are assuming provisionally that \PB{\|w} is at most 23, we can use
octal notation to put 10 codes in one tetrabyte and quinary notation
to put 13 in the other.
But quinary notation could obviously be used in both tetrabytes, or we
could use pure quinary on octabytes, in variants of
this program designed for larger values of~\PB{\|w}.

\Y\B\4\X5:Subroutines\X${}\mathrel+\E{}$\6
\&{cstring} \\{packit}(\,)\1\1\2\2\6
${}\{{}$\1\6
\&{register} \&{int} \|j${},{}$ \|k;\6
\&{cstring} \\{packed};\7
${}\|k\K\|w-\T{1},\39\|j\K\|c[\|w];{}$\6
\&{if} ${}(\|w\Z\T{10}){}$\1\5
${}\\{packed}.\|o.\|h\K\T{0};{}$\2\6
\&{else}\5
${}\{{}$\1\6
\&{for} ( ; ${}\|k>\T{10};{}$ ${}\|k\MM){}$\1\5
${}\|j\K(\|j\LL\T{2})+\|j+\|c[\|k];{}$\2\6
${}\\{packed}.\|o.\|h\K\|j;{}$\6
${}\|k\K\T{9},\39\|j\K\|c[\T{10}];{}$\6
\4${}\}{}$\2\6
\&{for} ( ; ${}\|k>\T{0};{}$ ${}\|k\MM){}$\1\5
${}\|j\K(\|j\LL\T{3})+\|c[\|k];{}$\2\6
${}\\{packed}.\|o.\|l\K\|j;{}$\6
\&{return} \\{packed};\6
\4${}\}{}$\2\par
\fi

\M{56}That which can be packed can be unpacked. This routine puts the
results into two arrays, \PB{\\{sc}} and \PB{\|c}, because it is used only when
unpacking a new source configuration $\alpha$.

There's an all-binary way to divide by 5 that is faster than division
on some machines (see {\sl TAOCP\/} exercise 4.4--9).
But the simple `/5' works best on my computer, winning also
over floating point division.

Curiously, I also found that `\PB{$\|x*\T{5}$}' is slower than `$(x<<2)+x$',
but
`\PB{$\|y-\T{5}*\|x$}' is faster than `\PB{$\|y-((\|x\LL\T{2})+\|x)$}'. Some
quirk of pipelining
probably underlies these phenomena. I am content to leave
such mysteries unexplained for now, because the present speed is acceptable.

\Y\B\4\X5:Subroutines\X${}\mathrel+\E{}$\6
\&{void} \\{unpackit}(\&{cstring} \|s)\1\1\2\2\6
${}\{{}$\1\6
\&{register} \&{int} \|j${},{}$ \|k${},{}$ \|q;\7
\&{if} ${}(\|w>\T{10}){}$\5
${}\{{}$\1\6
\&{for} ${}(\|k\K\T{1},\39\|j\K\|s.\|o.\|l;{}$ ${}\|k<\T{10};{}$ ${}\|k\PP){}$\5
${}\{{}$\1\6
${}\\{sc}[\|k]\K\|c[\|k]\K\|j\AND\T{7};{}$\6
${}\|j\MRL{{\GG}{\K}}\T{3};{}$\6
\4${}\}{}$\2\6
${}\\{sc}[\T{10}]\K\|c[\T{10}]\K\|j;{}$\6
\&{for} ${}(\|k\K\T{11},\39\|j\K\|s.\|o.\|h;{}$ ${}\|k<\|w;{}$ ${}\|k\PP){}$\5
${}\{{}$\1\6
${}\|q\K\|j/\T{5};{}$\6
${}\\{sc}[\|k]\K\|c[\|k]\K\|j-\T{5}*\|q;{}$\6
${}\|j\K\|q;{}$\6
\4${}\}{}$\2\6
\4${}\}{}$\5
\2\&{else}\1\6
\&{for} ${}(\|k\K\T{1},\39\|j\K\|s.\|o.\|l;{}$ ${}\|k<\|w;{}$ ${}\|k\PP){}$\5
${}\{{}$\1\6
${}\\{sc}[\|k]\K\|c[\|k]\K\|j\AND\T{7};{}$\6
${}\|j\MRL{{\GG}{\K}}\T{3};{}$\6
\4${}\}{}$\2\2\6
${}\\{sc}[\|k]\K\|c[\|k]\K\|j;{}$\6
\4${}\}{}$\2\par
\fi

\M{57}\B\X7:Global variables\X${}\mathrel+\E{}$\6
\&{code} \\{sc}[\T{64}];\C{ codes of the current source configuration }\par
\fi

\M{58}\B\X15:Initialize\X${}\mathrel+\E{}$\6
$\|c[\T{0}]\K\\{sc}[\T{0}]\K\|c[\|w+\T{1}]\K\\{sc}[\|w+\T{1}]\K\\{eol}{}$;\par
\fi

\M{59}\B\X5:Subroutines\X${}\mathrel+\E{}$\6
\&{void} \\{print\_config}(\&{int} \\{row\_end})\1\1\2\2\6
${}\{{}$\1\6
\&{register} \&{int} \|k;\7
\&{for} ${}(\|k\K\T{1};{}$ ${}\|k\Z\|w;{}$ ${}\|k\PP){}$\5
${}\{{}$\1\6
\&{if} ${}(\\{row\_end}\E\|k){}$\1\5
\\{printf}(\.{"\^"});\2\6
\&{if} ${}(\|c[\|k]<\\{eol}){}$\1\5
${}\\{printf}(\.{"\%c"},\39\\{decode}[\|c[\|k]]);{}$\2\6
\&{else}\1\5
\\{printf}(\.{"?"});\2\6
\4${}\}{}$\2\6
\4${}\}{}$\2\par
\fi

\M{60}A configuration can be in three states: Normally it is \PB{\\{active}},
with a generating function represented as a sequence of counters;
first, however, it is \PB{\\{raw}}, meaning that the space for counters has
been
allocated but not yet cleared to zero. An inactive target node is
marked \PB{\\{deleted}} when its memory space has been recycled and made
available for reuse.

\Y\B\4\X12:Type definitions\X${}\mathrel+\E{}$\6
\&{typedef} \&{enum} ${}\{{}$\1\6
${}\\{active},\39\\{raw},\39\\{deleted}{}$\2\6
${}\}{}$ \&{status};\par
\fi

\M{61}Here then are the 20 precious bytes that represent a configuration.
The rolling strategy allows us to get by with only one link field.

\Y\B\4\X12:Type definitions\X${}\mathrel+\E{}$\6
\&{typedef} \&{struct} \&{conf\_struct} ${}\{{}$\1\6
\&{cstring} \|s;\C{ the configuration name }\6
\&{unsigned} \&{int} \\{addr};\C{ where the slave keeps the generating function
}\6
\&{struct} \&{conf\_struct} ${}{*}\\{link}{}$;\C{ the next item in a hash chain
or hole list }\6
\&{char} \\{lo};\C{ smallest exponent of $z$ in the current generating function
}\6
\&{char} \\{hi};\C{ largest exponent of $z$ in the current generating function
}\6
\&{char} \\{lim};\C{ largest viable exponent of $z$, if this is a target }\6
\&{status} \\{state};\C{ \PB{\\{active}}, \PB{\\{raw}}, or \PB{\\{deleted}} }\2%
\6
${}\}{}$ \&{config};\par
\fi

\M{62}\B\X15:Initialize\X${}\mathrel+\E{}$\6
$\\{conf}\K{}$(\&{config} ${}{*}){}$ \\{calloc}${}(\\{conf\_size},\39%
\&{sizeof}(\&{config}));{}$\6
\&{if} ${}(\R\\{conf}){}$\1\5
\\{panic}(\.{"I\ can't\ allocate\ th}\)\.{e\ config\ table"});\2\6
${}\\{conf\_end}\K\\{conf}+\\{conf\_size}{}$;\par
\fi

\M{63}The main high-level routine, called \PB{\\{update}}, is used to add terms
\PB{$\|p\MG\\{lo}$} through \PB{\\{hi}} of the generating function for
configuration~\PB{\|p}
to the generating function for configuration~\PB{\|q}. The special case
\PB{$\|q\K\NULL$} is used to update the counters for polyominoes that have been
completed.

\Y\B\4\X5:Subroutines\X${}\mathrel+\E{}$\6
\&{void} \\{update}(\&{config} ${}{*}\|p,\39{}$\&{config} ${}{*}\|q,\39{}$%
\&{char} \\{hi})\1\1\2\2\6
${}\{{}$\1\6
\&{if} ${}(\R\|q){}$\1\5
${}\\{basic\_inst}(\\{add},\39\|p\MG\\{addr},\39\|p\MG\\{lo},\39\\{hi}+\T{1}-%
\|p\MG\\{lo});{}$\2\6
\&{else} \&{if} ${}(\|q\MG\\{state}\E\\{raw}){}$\5
${}\{{}$\1\6
${}\|q\MG\\{state}\K\\{active};{}$\6
\&{if} ${}(\|q\MG\\{lo}\I\|p\MG\\{lo}\V\|q\MG\\{hi}\I\\{hi}){}$\1\5
${}\\{basic\_inst}(\\{clear},\39\\{cur\_src},\39\|q\MG\\{addr},\39\|q\MG\\{hi}+%
\T{1}-\|q\MG\\{lo});{}$\2\6
${}\\{basic\_inst}(\\{copy},\39\|p\MG\\{addr},\39\|q\MG\\{addr}+\|p\MG\\{lo}-%
\|q\MG\\{lo},\39\\{hi}+\T{1}-\|p\MG\\{lo});{}$\6
\4${}\}{}$\5
\2\&{else}\1\5
${}\\{basic\_inst}(\\{add},\39\|p\MG\\{addr},\39\|q\MG\\{addr}+\|p\MG\\{lo}-\|q%
\MG\\{lo},\39\\{hi}+\T{1}-\|p\MG\\{lo});{}$\2\6
\4${}\}{}$\2\par
\fi

\M{64}``Universal hashing'' ({\sl TAOCP\/} exercise 6.4--72) is used to get a
good hash function, because most of the key bits tend to be zero.

\Y\B\4\D$\\{hash\_width}$ \5
\T{20}\C{ lg of hash table size }\par
\B\4\D$\\{hash\_mask}$ \5
$((\T{1}\LL\\{hash\_width})-\T{1}{}$)\par
\Y\B\4\X5:Subroutines\X${}\mathrel+\E{}$\6
\&{int} \\{mangle}(\&{cstring} \|s)\1\1\2\2\6
${}\{{}$\1\6
\&{register} \&{unsigned} \&{int} \|h${},{}$ \|l;\7
\&{for} ${}(\|l\K\T{1},\39\|h\K\\{hash\_bits}[\T{0}][\|s.\\{byte}[\T{0}]];{}$
${}\|l<\T{8};{}$ ${}\|l\PP){}$\1\5
${}\|h\MRL{+{\K}}\\{hash\_bits}[\|l][\|s.\\{byte}[\|l]];{}$\2\6
\&{return} \|h${}\AND\\{hash\_mask};{}$\6
\4${}\}{}$\2\par
\fi

\M{65}\B\X7:Global variables\X${}\mathrel+\E{}$\6
\&{unsigned} \&{int} \\{hash\_bits}[\T{8}][\T{256}];\C{ random bits for
universal hashing }\6
\&{config} ${}{*}\\{hash\_table}[\\{hash\_mask}+\T{1}]{}$;\C{ heads of the hash
chains }\par
\fi

\M{66}The random number generator used here doesn't have to be of sensational
quality. We can keep \PB{$\\{hash\_bits}[\|j][\T{0}]\K\T{0}$} without loss of
universality.

\Y\B\4\X15:Initialize\X${}\mathrel+\E{}$\6
$\\{row\_end}\K\T{314159265}{}$;\C{ borrow a register temporarily (bad style,
sorry) }\6
\&{for} ${}(\|j\K\T{0};{}$ ${}\|j<\T{8};{}$ ${}\|j\PP){}$\1\6
\&{for} ${}(\|k\K\T{1};{}$ ${}\|k<\T{256};{}$ ${}\|k\PP){}$\5
${}\{{}$\1\6
${}\\{row\_end}\K\T{69069}*\\{row\_end}+\T{1};{}$\6
${}\\{hash\_bits}[\|j][\|k]\K\\{row\_end}\GG(\T{32}-\\{hash\_width});{}$\6
\4${}\}{}$\2\2\par
\fi

\M{67}\B\X67:Local variables\X${}\E{}$\6
\&{register} \&{int} \|j${},{}$ \|k;\C{ all-purpose indices }\6
\&{register} \&{int} \\{row\_end};\C{ size of the current partial row \PB{\|r}
}\par
\A99.
\U4.\fi

\M{68}On odd-numbered passes, \PB{\\{src}} runs down towards \PB{\\{conf}},
while \PB{\\{trg}}
starts at \PB{$\\{conf\_end}-\T{1}$} and proceeds downward. On even-numbered
passes,
\PB{\\{src}} runs up towards \PB{$\\{conf\_end}-\T{1}$} and \PB{\\{trg}} starts
up from \PB{\\{conf}}.
The variables \PB{\\{ssrc}} and \PB{\\{strg}} have a similar significance but
they refer to addresses in the slave memory.

\Y\B\4\X7:Global variables\X${}\mathrel+\E{}$\6
\&{config} ${}{*}\\{conf}{}$;\C{ first item in the pool of configuration nodes
}\6
\&{config} ${}{*}\\{conf\_end}{}$;\C{ last item (plus 1) in the pool of
configuration nodes }\6
\&{config} ${}{*}\\{src}{}$;\C{ the current configuration $\alpha$ about to be
recycled }\6
\&{config} ${}{*}\\{trg}{}$;\C{ the first unused configuration slot }\6
\&{int} \\{ssrc}${},{}$ \\{strg};\C{ allocation pointers for slave counts }\par
\fi

\M{69}When a new configuration is created, we allocate space for its
generating function in the slave module. Later on, we might discover
that more space is needed because another generating function (with
more terms) must be combined with it. At such times, we copy the data to
another slot, leaving a hole in the configuration array and in the slave's
array of counters. All holes of a given size are linked together,
so that they can hopefully be plugged again soon.

Here is the basic subroutine that allocates space for a configuration with
\PB{$\|s+\T{1}$} terms in its generating function. The subroutine has two
versions,
one for passes in which allocation goes upward and the other for
passes in which allocation goes downward. It maintains statistics so
that we can judge how fragmented the memory has become at
the most stressful times.

\Y\B\4\X5:Subroutines\X${}\mathrel+\E{}$\6
\&{config} ${}{*}{}$\\{get\_slot\_up}(\&{register} \&{int} \|s)\1\1\2\2\6
${}\{{}$\1\6
\&{register} \&{config} ${}{*}\|p\K\\{slot}[\|s];{}$\7
\&{if} (\|p)\5
${}\{{}$\1\6
${}\\{slot}[\|s]\K\|p\MG\\{link};{}$\6
${}\\{holes}\MM,\39\\{sholes}\MRL{-{\K}}\|s+\T{1};{}$\6
\4${}\}{}$\5
\2\&{else}\5
${}\{{}$\1\6
${}\|p\K\\{trg}\PP;{}$\6
\X70:Allocate \PB{$\|p\MG\\{addr}$} (upward) and check that memory hasn't
overflowed\X;\6
\4${}\}{}$\2\6
${}\|p\MG\\{state}\K\\{raw};{}$\6
\&{return} \|p;\6
\4${}\}{}$\2\par
\fi

\M{70}\B\X70:Allocate \PB{$\|p\MG\\{addr}$} (upward) and check that memory
hasn't overflowed\X${}\E{}$\6
${}\{{}$\1\6
\&{if} ${}(\\{src}-\\{trg}<\\{min\_space}){}$\5
${}\{{}$\1\6
${}\\{min\_space}\K\\{src}-\\{trg};{}$\6
\&{if} ${}(\\{min\_space}<\T{0}){}$\1\5
\\{panic}(\.{"Memory\ overflow"});\2\6
${}\\{min\_holes}\K\\{holes},\39\\{space\_row}\K\|r,\39\\{space\_col}\K%
\\{re};{}$\6
\4${}\}{}$\2\6
${}\|p\MG\\{addr}\K\\{strg};{}$\6
${}\\{strg}\MRL{+{\K}}\|s+\T{1};{}$\6
\&{if} ${}(\\{ssrc}-\\{strg}<\\{min\_sspace}){}$\5
${}\{{}$\1\6
${}\\{min\_sspace}\K\\{ssrc}-\\{strg};{}$\6
\&{if} ${}(\\{min\_sspace}<\T{0}){}$\1\5
\\{panic}(\.{"Slave\ memory\ overfl}\)\.{ow"});\2\6
${}\\{min\_sholes}\K\\{sholes},\39\\{slave\_row}\K\|r,\39\\{slave\_col}\K%
\\{re};{}$\6
\4${}\}{}$\2\6
\4${}\}{}$\2\par
\U69.\fi

\M{71}\B\X7:Global variables\X${}\mathrel+\E{}$\6
\&{int} \\{holes};\C{ current number of holes in the target area }\6
\&{int} \\{sholes};\C{ current number of vacated counters in slave target area
}\6
\&{int} \\{min\_space}${}\K\T{1000000000}{}$;\C{ how close did \PB{\\{src}} and
\PB{\\{trg}} get? }\6
\&{int} \\{min\_holes};\C{ and how many holes were present at that time? }\6
\&{int} \\{space\_row}${},{}$ \\{space\_col};\C{ and where were we then? }\6
\&{int} \\{min\_sspace}${}\K\T{1000000000}{}$;\C{ how close did \PB{\\{ssrc}}
and \PB{\\{strg}} get? }\6
\&{int} \\{min\_sholes};\C{ and how many wasted counters were present then? }\6
\&{int} \\{slave\_row}${},{}$ \\{slave\_col};\C{ and where were we then? }\6
\&{int} \\{moves};\C{ the number of times a hole was created }\6
\&{int} \\{configs};\C{ total configurations recorded so far, mod $10^9$ }\6
\&{int} \\{hconfigs};\C{ billions of configurations so far }\6
\&{int} \|r;\C{ number of the partially filled row }\6
\&{int} \\{re};\C{ non-register copy of \PB{\\{row\_end}} }\6
\&{config} ${}{*}\\{slot}[\\{nmax}+\T{1}]{}$;\C{ heads of the available-slot
chains }\par
\fi

\M{72}\B\X72:Print statistics about this run\X${}\E{}$\6
\\{printf}(\.{"Altogether\ "});\6
\&{if} (\\{hconfigs})\1\5
${}\\{printf}(\.{"\%d\%09d"},\39\\{hconfigs},\39\\{configs});{}$\2\6
\&{else}\1\5
${}\\{printf}(\.{"\%d"},\39\\{configs});{}$\2\6
\\{printf}(\.{"\ viable\ configurati}\)\.{ons\ examined;\\n"});\6
${}\\{printf}(\.{"\ \%d\ slots\ needed\ (w}\)\.{ith\ \%d\ holes)\ in\ pos}\)%
\.{ition\ (\%d,\%d);\\n"},\39\\{conf\_size}-\\{min\_space},\39\\{min\_holes},%
\39\\{space\_row},\39\\{space\_col});{}$\6
${}\\{printf}(\.{"\ \%d\ counters\ needed}\)\.{\ (with\ \%d\ wasted)\ in}\)\.{\
position\ (\%d,\%d);\\n}\)\.{"},\39\\{slave\_size}-\\{min\_sspace},\39\\{min%
\_sholes},\39\\{slave\_row},\39\\{slave\_col});{}$\6
${}\\{printf}(\.{"\ \%d\ moves.\\n"},\39\\{moves}){}$;\par
\Us4\ET105.\fi

\M{73}\B\X5:Subroutines\X${}\mathrel+\E{}$\6
\&{config} ${}{*}{}$\\{get\_slot\_down}(\&{register} \&{int} \|s)\1\1\2\2\6
${}\{{}$\1\6
\&{register} \&{config} ${}{*}\|p\K\\{slot}[\|s];{}$\7
\&{if} (\|p)\5
${}\{{}$\1\6
${}\\{slot}[\|s]\K\|p\MG\\{link};{}$\6
${}\\{holes}\MM,\39\\{sholes}\MRL{-{\K}}\|s+\T{1};{}$\6
\4${}\}{}$\5
\2\&{else}\5
${}\{{}$\1\6
${}\|p\K\\{trg}\MM;{}$\6
\X74:Allocate \PB{$\|p\MG\\{addr}$} (downward) and check that memory hasn't
overflowed\X;\6
\4${}\}{}$\2\6
${}\|p\MG\\{state}\K\\{raw};{}$\6
\&{return} \|p;\6
\4${}\}{}$\2\par
\fi

\M{74}\B\X74:Allocate \PB{$\|p\MG\\{addr}$} (downward) and check that memory
hasn't overflowed\X${}\E{}$\6
${}\{{}$\1\6
\&{if} ${}(\\{trg}-\\{src}<\\{min\_space}){}$\5
${}\{{}$\1\6
${}\\{min\_space}\K\\{trg}-\\{src};{}$\6
\&{if} ${}(\\{min\_space}<\T{0}){}$\1\5
\\{panic}(\.{"Memory\ overflow"});\2\6
${}\\{min\_holes}\K\\{holes},\39\\{space\_row}\K\|r,\39\\{space\_col}\K%
\\{re};{}$\6
\4${}\}{}$\2\6
${}\\{strg}\MRL{-{\K}}\|s+\T{1};{}$\6
\&{if} ${}(\\{strg}-\\{ssrc}<\\{min\_sspace}){}$\5
${}\{{}$\1\6
${}\\{min\_sspace}\K\\{strg}-\\{ssrc};{}$\6
\&{if} ${}(\\{min\_sspace}<\T{0}){}$\1\5
\\{panic}(\.{"Slave\ memory\ overfl}\)\.{ow"});\2\6
${}\\{min\_sholes}\K\\{sholes},\39\\{slave\_row}\K\|r,\39\\{slave\_col}\K%
\\{re};{}$\6
\4${}\}{}$\2\6
${}\|p\MG\\{addr}\K\\{strg}+\T{1};{}$\6
\4${}\}{}$\2\par
\U73.\fi

\M{75}The \PB{\\{move\_down}} and \PB{\\{move\_up}} subroutines are invoked
when an
active target configuration \PB{\|p} needs more space for its generating
function. The global variable \PB{\\{hash}} will have been set so that
\PB{$\\{hash\_table}[\\{hash}]\K\|p$}; we effectively move that configuration
to another
place in the sequential list of targets,
and return a pointer to the new place.
The former node \PB{\|p} is now marked \PB{\\{deleted}}, but its \PB{\\{addr}}
field remains valid (in case \\{get\_slot} is able to reuse it).

\Y\B\4\X5:Subroutines\X${}\mathrel+\E{}$\6
\&{config} ${}{*}{}$\\{move\_down}(\&{config} ${}{*}\|p,\39{}$\&{int} %
\\{lo}${},\39{}$\&{int} \\{hi})\1\1\2\2\6
${}\{{}$\1\6
\&{register} \&{config} ${}{*}\|q,{}$ ${}{*}\|r;{}$\6
\&{register} \&{int} \|s${}\K\|p\MG\\{lo},{}$ \|t${}\K\|p\MG\\{hi};{}$\7
${}\|r\K\|p\MG\\{link};{}$\6
${}\|p\MG\\{link}\K\\{slot}[\|t-\|s],\39\\{slot}[\|t-\|s]\K\|p;{}$\6
${}\|p\MG\\{state}\K\\{deleted};{}$\6
${}\\{holes}\PP,\39\\{sholes}\MRL{+{\K}}\|t-\|s+\T{1};{}$\6
\&{if} ${}(\|s>\\{lo}){}$\1\5
${}\|s\K\\{lo};{}$\2\6
\&{if} ${}(\|t<\\{hi}){}$\1\5
${}\|t\K\\{hi};{}$\2\6
${}\|q\K\\{get\_slot\_down}(\|t-\|s);{}$\6
${}\|q\MG\\{lo}\K\|s,\39\|q\MG\\{hi}\K\|t;{}$\6
${}\|q\MG\|s\K\|p\MG\|s,\39\|q\MG\\{lim}\K\|p\MG\\{lim};{}$\6
${}\\{hash\_table}[\\{hash}]\K\|q,\39\|q\MG\\{link}\K\|r;{}$\6
${}\\{update}(\|p,\39\|q,\39\|p\MG\\{hi});{}$\6
${}\\{moves}\PP;{}$\6
\&{return} \|q;\6
\4${}\}{}$\2\par
\fi

\M{76}\B\X5:Subroutines\X${}\mathrel+\E{}$\6
\&{config} ${}{*}{}$\\{move\_up}(\&{config} ${}{*}\|p,\39{}$\&{int} \\{lo}${},%
\39{}$\&{int} \\{hi})\1\1\2\2\6
${}\{{}$\1\6
\&{register} \&{config} ${}{*}\|q,{}$ ${}{*}\|r;{}$\6
\&{register} \&{int} \|s${}\K\|p\MG\\{lo},{}$ \|t${}\K\|p\MG\\{hi};{}$\7
${}\|r\K\|p\MG\\{link};{}$\6
${}\|p\MG\\{link}\K\\{slot}[\|t-\|s],\39\\{slot}[\|t-\|s]\K\|p;{}$\6
${}\|p\MG\\{state}\K\\{deleted};{}$\6
${}\\{holes}\PP,\39\\{sholes}\MRL{+{\K}}\|t-\|s+\T{1};{}$\6
\&{if} ${}(\|s>\\{lo}){}$\1\5
${}\|s\K\\{lo};{}$\2\6
\&{if} ${}(\|t<\\{hi}){}$\1\5
${}\|t\K\\{hi};{}$\2\6
${}\|q\K\\{get\_slot\_up}(\|t-\|s);{}$\6
${}\|q\MG\\{lo}\K\|s,\39\|q\MG\\{hi}\K\|t;{}$\6
${}\|q\MG\|s\K\|p\MG\|s,\39\|q\MG\\{lim}\K\|p\MG\\{lim};{}$\6
${}\\{hash\_table}[\\{hash}]\K\|q,\39\|q\MG\\{link}\K\|r;{}$\6
${}\\{update}(\|p,\39\|q,\39\|p\MG\\{hi});{}$\6
${}\\{moves}\PP;{}$\6
\&{return} \|q;\6
\4${}\}{}$\2\par
\fi

\N{1}{77}The main loop. Now that we have some infrastructure in place,
we can map out the top levels of this program's main processing cycle.

We start with an all-\PB{\\{zero}} configuration, at the very top of the
width-$w$ array of cells that we will conceptually traverse; this configuration
serves as the great-$\,\ldots\,$-great grandparent of all other
configurations that will arise later. The slave module will begin
by giving this configuration the trivial generating function `1' (namely
$z^0$) in its counter cell number~0, meaning that there's just one way to
reach the initial configuration, and that no cells are occupied so far.

There is no need to initialize \PB{$\\{conf}[\T{0}].\\{lim}$} or \PB{$\\{conf}[%
\T{0}].\\{link}$}, because
those fields are used only when a configuration is a target.
The other fields---namely \PB{$\\{conf}[\T{0}].\|s$}, \PB{$\\{conf}[\T{0}].%
\\{addr}$}, \PB{$\\{conf}[\T{0}].\\{lo}$},
\PB{$\\{conf}[\T{0}].\\{hi}$}, and \PB{$\\{conf}[\T{0}].\\{state}$}--- are
initially zero by the
conventions of \CEE/, and luckily those zeros happen to be just what we want.

\Y\B\4\X15:Initialize\X${}\mathrel+\E{}$\6
$\|r\K\T{0},\39\\{row\_end}\K\|w;{}$\6
${}\\{trg}\K\\{conf}+\T{1}{}$;\C{ pretend that the previous pass produced a
single result }\6
${}\\{strg}\K\T{1}{}$;\par
\fi

\M{78}Once again it seems best to write two nearly identical pieces of
code, depending on whether the allocation is actually moving upward or
downward. (We're supposed to be hoarding memory, but the space required
for this program is small potatoes.)

\Y\B\4\X78:Output instructions for the postprocessor\X${}\E{}$\6
\&{while} (\T{1})\5
${}\{{}$\1\6
\X79:Get ready for a downward pass, or \PB{\&{break}} when done\X;\6
\X80:Pass downward over all configurations created on the previous pass\X;\6
\X82:Get ready for an upward pass, or \PB{\&{break}} when done\X;\6
\X83:Pass upward over all configurations created on the previous pass\X;\6
\4${}\}{}$\2\par
\U4.\fi

\M{79}\B\X79:Get ready for a downward pass, or \PB{\&{break}} when done\X${}%
\E{}$\6
\&{if} ${}(\\{row\_end}<\|w){}$\5
${}\{{}$\1\6
${}\\{row\_end}\PP;{}$\6
${}\\{printf}(\.{"Beginning\ column\ \%d}\)\.{"},\39\\{row\_end});{}$\6
\X81:Print current stats and clear the hash/slot tables\X;\6
\4${}\}{}$\5
\2\&{else}\5
${}\{{}$\1\6
\&{if} (\|r)\5
${}\{{}$\1\6
${}\\{printf}(\.{"Finished\ row\ \%d"},\39\|r);{}$\6
\X81:Print current stats and clear the hash/slot tables\X;\6
\&{if} ${}(\|r>\|w){}$\1\5
${}\\{put\_inst}(\\{sync},\39\|r);{}$\2\6
\4${}\}{}$\2\6
\X105:Check if this run has gone on too long\X;\6
${}\|r\PP,\39\\{row\_end}\K\T{1};{}$\6
\4${}\}{}$\2\6
\&{if} ${}(\\{trg}\E\\{conf}){}$\1\5
\&{break};\C{ the previous pass was sterile }\2\6
${}\\{src}\K\\{trg}-\T{1}{}$;\C{ start the source pointer at the highest
occupied node }\6
${}\\{ssrc}\K\\{strg}-\T{1}{}$;\C{ and the highest occupied counter position }\6
${}\\{trg}\K\\{conf\_end}-\T{1}{}$;\C{ start the target pointer at the highest
unoccupied node }\6
${}\\{strg}\K\\{slave\_size}-\T{1}{}$;\C{ and the highest unoccupied counter
position }\6
${}\\{re}\K\\{row\_end}{}$;\par
\U78.\fi

\M{80}\B\X80:Pass downward over all configurations created on the previous pass%
\X${}\E{}$\6
\&{while} ${}(\\{src}\G\\{conf}){}$\5
${}\{{}$\1\6
\&{if} ${}(\\{src}\MG\\{state}\E\\{active}){}$\5
${}\{{}$\1\6
${}\\{unpackit}(\\{src}\MG\|s){}$;\C{ Put the source configuration $\alpha$
into \PB{\\{sc}} and \PB{\|c} }\6
\&{if} (\\{verbose})\5
${}\{{}$\1\6
\\{print\_config}(\\{row\_end});\5
\\{printf}(\.{"\\n"});\6
\4${}\}{}$\2\6
\X84:Change array \PB{\|c} for target $\alpha_0$\X;\6
\&{if} (\\{viable})\1\5
\X97:Process target configuration \PB{\|c} (downward)\X;\2\6
\&{for} ${}(\|k\K\T{1};{}$ ${}\|k\Z\|w;{}$ ${}\|k\PP){}$\1\5
${}\|c[\|k]\K\\{sc}[\|k];{}$\2\6
\X90:Change array \PB{\|c} for target $\alpha_1$\X;\6
\&{if} (\\{viable})\1\5
\X97:Process target configuration \PB{\|c} (downward)\X;\2\6
\4${}\}{}$\2\6
${}\\{ssrc}\K\\{src}\MG\\{addr}-\T{1};{}$\6
${}\\{src}\MM{}$;\C{ the old \PB{\\{src}} node is now outta here }\6
\4${}\}{}$\2\par
\U78.\fi

\M{81}Timely progress reports let the user know that we are still chugging
along.
We are about to start an upward pass if and only if \PB{$\\{src}\E\\{conf}-%
\T{1}$}.

\Y\B\4\X81:Print current stats and clear the hash/slot tables\X${}\E{}$\6
\&{if} ${}(\\{src}\E\\{conf}-\T{1}){}$\1\5
${}\\{printf}(\.{"\ (\%d,\%d,"},\39\\{conf\_end}-\T{1}-\\{trg},\39\\{slave%
\_size}-\T{1}-\\{strg});{}$\2\6
\&{else}\1\5
${}\\{printf}(\.{"\ (\%d,\%d,"},\39\\{trg}-\\{conf},\39\\{strg}-\|n-\T{1});{}$%
\2\6
${}\\{printf}(\.{"\%d,\%d,\%d,\%d,\%d)\\n"},\39\\{conf\_size}-\\{min\_space},%
\39\\{min\_holes},\39\\{slave\_size}-\\{min\_sspace},\39\\{min\_sholes},\39%
\\{bytes\_out});{}$\6
\X109:Print and clear the hash/slot tables\X;\6
\\{fflush}(\\{stdout});\par
\Us79\ET82.\fi

\M{82}Counter positions 1 through $n$ in the slave memory are reserved for
the final polyomino counts.

\Y\B\4\X82:Get ready for an upward pass, or \PB{\&{break}} when done\X${}\E{}$\6
\&{if} ${}(\\{row\_end}<\|w){}$\5
${}\{{}$\1\6
${}\\{row\_end}\PP;{}$\6
${}\\{printf}(\.{"Beginning\ column\ \%d}\)\.{"},\39\\{row\_end});{}$\6
\X81:Print current stats and clear the hash/slot tables\X;\6
\4${}\}{}$\5
\2\&{else}\5
${}\{{}$\1\6
\&{if} (\|r)\5
${}\{{}$\1\6
${}\\{printf}(\.{"Finished\ row\ \%d"},\39\|r);{}$\6
\X81:Print current stats and clear the hash/slot tables\X;\6
\&{if} ${}(\|r>\|w){}$\1\5
${}\\{put\_inst}(\\{sync},\39\|r);{}$\2\6
\4${}\}{}$\2\6
${}\|r\PP,\39\\{row\_end}\K\T{1};{}$\6
\4${}\}{}$\2\6
\&{if} ${}(\\{trg}\E\\{conf\_end}-\T{1}){}$\1\5
\&{break};\C{ the previous pass was sterile }\2\6
${}\\{src}\K\\{trg}+\T{1}{}$;\C{ start the source pointer at the lowest
occupied node }\C{ and we'll soon set \PB{\\{ssrc}} to \PB{$\\{src}\MG%
\\{addr}$},     which equals \PB{$\\{strg}+\T{1}$}, the lowest occupied counter
}\6
${}\\{trg}\K\\{conf}{}$;\C{ start the target pointer at the lowest unoccupied
node }\6
${}\\{strg}\K\|n+\T{1}{}$;\C{ and the lowest unoccupied counter position }\6
${}\\{re}\K\\{row\_end}{}$;\par
\U78.\fi

\M{83}\B\X83:Pass upward over all configurations created on the previous pass%
\X${}\E{}$\6
\&{while} ${}(\\{src}<\\{conf\_end}){}$\5
${}\{{}$\1\6
\&{if} ${}(\\{src}\MG\\{state}\E\\{active}){}$\5
${}\{{}$\1\6
${}\\{ssrc}\K\\{src}\MG\\{addr};{}$\6
${}\\{unpackit}(\\{src}\MG\|s){}$;\C{ Put the source configuration $\alpha$
into \PB{\\{sc}} and \PB{\|c} }\6
\&{if} (\\{verbose})\5
${}\{{}$\1\6
\\{print\_config}(\\{row\_end});\5
\\{printf}(\.{"\\n"});\6
\4${}\}{}$\2\6
\X84:Change array \PB{\|c} for target $\alpha_0$\X;\6
\&{if} (\\{viable})\1\5
\X103:Process target configuration \PB{\|c} (upward)\X;\2\6
\&{for} ${}(\|k\K\T{1};{}$ ${}\|k\Z\|w;{}$ ${}\|k\PP){}$\1\5
${}\|c[\|k]\K\\{sc}[\|k];{}$\2\6
\X90:Change array \PB{\|c} for target $\alpha_1$\X;\6
\&{if} (\\{viable})\1\5
\X103:Process target configuration \PB{\|c} (upward)\X;\2\6
\4${}\}{}$\2\6
${}\\{src}\PP{}$;\C{ the old \PB{\\{src}} node is now outta here }\6
\4${}\}{}$\2\par
\U78.\fi

\N{1}{84}Nitty-gritty. The basic logic of a so-called ``transfer-matrix''
approach
is embedded in the following program steps, which change a configuration string
when a new cell in row~$r$ is or is not to be occupied. Here, for example,
we observe that when the previous configuration has `\.{1\caret/(}' at
the end of a partial row, and if we occupy the new cell, the new configuration
has `\.{(-}\caret/\thinspace' instead. But if we don't occupy that cell,
the new configuration has `\.{10}\caret/\thinspace' and a further change
must also be made because of the \.( that has disappeared.

\def\sp(#1){\rlap{$^{\rm\,#1}$}}
\def\key(#1) {&\omit&\multispan5\quad\sp(#1)\quad}
Most of the cases that arise are completely straightforward. But each
of the thirty combinations of two adjacent codes must of course be handled
perfectly. The following chart summarizes what the program is supposed to do
when the new cell is being left vacant.
$$\vbox{\offinterlineskip
\halign{\strut\hfil\tt#\quad&\vrule#&
\hbox to4em{\hfil\tt#\hfil}&
\hbox to4em{\hfil\tt#\hfil}&
\hbox to4em{\hfil\tt#\hfil}&
\hbox to4em{\hfil\tt#\hfil}&
\hbox to4em{\hfil\tt#\hfil}&
\vrule#\cr
&\omit&0&1&(&-&)\cr
\noalign{\vskip2pt}
\omit&&\multispan5\hrulefill&\cr
\omit&height3pt&&&&&&\cr
0&&00&\sp(a)&00\sp(b)&00\sp(c)&00\sp(d)&\cr
1&&10&\sp(a)&10\sp(b)&10\relax&10\sp(d)&\cr
(&&(0&\sp(e)&  \sp(e)&(0\relax&(0\sp(d)&\cr
-&&-0&\sp(a)&-0\sp(b)&-0\relax&-0\sp(d)&\cr
)&&)0&\sp(a)&-0\sp(b)&)0\relax&-0\sp(d)&\cr
\rm left edge&&0&\sp(e)&\sp(e)&0\sp(c)&\sp(a)&\cr
\omit&height1pt&&&&&&\cr
\omit&&\multispan5\hrulefill&\cr
\noalign{\smallskip}
\key(a) Not viable\hfil\cr
\key(b) Downgrade the successor of \.(\hfil\cr
\key(c) A polyomino may have been completed\hfil\cr
\key(d) Downgrade the predecessor of \.)\hfil\cr
\key(e) Impossible case\hfil\cr}}$$

Special handling is necessary when a \.- is eliminated, if it is preceded or
followed by nothing but zeros; for example, \.{0(010-} must become
\.{010(00}, and \.{-010()0-0} must become either \.{00)0()0(0} or
\.{)010(-000}. These somewhat unusual cases are approached cautiously
in the program below.

\Y\B\4\D$\|f(\|x,\|y)$ \5
$((\|x\LL\T{3})+\|y{}$)\par
\Y\B\4\X84:Change array \PB{\|c} for target $\alpha_0$\X${}\E{}$\6
$\\{pair}\K\|f(\\{sc}[\\{row\_end}-\T{1}],\39\\{sc}[\\{row\_end}]);{}$\6
${}\|c[\\{row\_end}]\K\\{zero};{}$\6
${}\\{viable}\K\T{1};{}$\6
\&{switch} (\\{pair})\5
${}\{{}$\1\6
\4\&{case} \|f${}(\\{zero},\39\\{one}){}$:\5
\&{case} \|f${}(\\{one},\39\\{one}){}$:\5
\&{case} \|f${}(\\{mid},\39\\{one}){}$:\5
\&{case} \|f${}(\\{rt},\39\\{one}){}$:\5
\&{case} \|f${}(\\{eol},\39\\{rt}){}$:\5
${}\\{viable}\K\T{0}{}$;\C{ component would be isolated }\6
\4\&{case} \|f${}(\\{zero},\39\\{zero}){}$:\5
\&{case} \|f${}(\\{one},\39\\{zero}){}$:\5
\&{case} \|f${}(\\{lft},\39\\{zero}){}$:\5
\&{case} \|f${}(\\{mid},\39\\{zero}){}$:\5
\&{case} \|f${}(\\{rt},\39\\{zero}){}$:\5
\&{case} \|f${}(\\{eol},\39\\{zero}){}$:\5
\&{case} \|f${}(\\{lft},\39\\{mid}){}$:\5
\&{case} \|f${}(\\{mid},\39\\{mid}){}$:\5
\&{break};\6
\4\&{case} \|f${}(\\{zero},\39\\{lft}){}$:\5
\&{case} \|f${}(\\{one},\39\\{lft}){}$:\5
\&{case} \|f${}(\\{mid},\39\\{lft}){}$:\5
\&{case} \|f${}(\\{rt},\39\\{lft}){}$:\5
\X86:Downgrade the successor of the \PB{\\{lft}}\X;\5
\&{break};\6
\4\&{case} \|f${}(\\{zero},\39\\{rt}){}$:\5
\&{case} \|f${}(\\{one},\39\\{rt}){}$:\5
\&{case} \|f${}(\\{lft},\39\\{rt}){}$:\5
\&{case} \|f${}(\\{mid},\39\\{rt}){}$:\5
\&{case} \|f${}(\\{rt},\39\\{rt}){}$:\5
\X87:Downgrade the predecessor of the \PB{\\{rt}}\X;\5
\&{break};\6
\4\&{case} \|f${}(\\{zero},\39\\{mid}){}$:\5
\&{case} \|f${}(\\{eol},\39\\{mid}){}$:\5
\X88:Cautiously delete a \PB{\\{mid}} that may be leftmost\X;\6
\4\&{case} \|f${}(\\{one},\39\\{mid}){}$:\5
\&{case} \|f${}(\\{rt},\39\\{mid}){}$:\5
\X89:Cautiously delete a \PB{\\{mid}} that may be rightmost\X;\6
\&{break};\6
\4\&{case} \|f${}(\\{lft},\39\\{one}){}$:\5
\&{case} \|f${}(\\{lft},\39\\{lft}){}$:\5
\&{case} \|f${}(\\{eol},\39\\{one}){}$:\5
\&{case} \|f${}(\\{eol},\39\\{lft}){}$:\5
\\{panic}(\.{"Impossible\ configur}\)\.{ation"});\6
\4\&{default}:\5
\\{panic}(\.{"Impossible\ pair"});\6
\4${}\}{}$\2\par
\Us80\ET83.\fi

\M{85}\B\X7:Global variables\X${}\mathrel+\E{}$\6
\&{int} \\{pair};\C{ the two codes surrounding the \qcaret/ in \PB{\\{sc}} }\6
\&{int} \\{viable};\C{ might the target configuration lead to a relevant
polyomino? }\par
\fi

\M{86}In this step, we have just zeroed out a left parenthesis.
If that \.( is followed by a \.-, we change the \.- to \.(;
if it is followed by a \.), we change the \.) to \.1.

Here ``followed by'' really means ``followed on the same level by,'' because
nested subcomponents may intervene. We therefore need a level counter, \PB{%
\|j},
as we scan to the right.

If the \.( had no successor because it simply marked a connection to the
right edge, we shouldn't have deleted it; its component is now disconnected,
so we set \PB{\\{viable}} to zero.

\Y\B\4\X86:Downgrade the successor of the \PB{\\{lft}}\X${}\E{}$\6
\&{for} ${}(\|k\K\\{row\_end}+\T{1},\39\|j\K\T{0};{}$  ; ${}\|k\PP){}$\5
${}\{{}$\1\6
\&{switch} (\|c[\|k])\5
${}\{{}$\1\6
\4\&{case} \\{lft}:\5
${}\|j\PP;{}$\6
\4\&{case} \\{zero}:\5
\&{case} \\{one}:\5
\&{continue};\6
\4\&{case} \\{mid}:\5
\&{if} (\|j)\1\5
\&{continue};\2\6
${}\|c[\|k]\K\\{lft}{}$;\5
\&{break};\6
\4\&{case} \\{rt}:\5
\&{if} (\|j)\5
${}\{{}$\5
\1${}\|j\MM{}$;\5
\&{continue};\5
${}\}{}$\2\6
${}\|c[\|k]\K\\{one}{}$;\5
\&{break};\6
\4\&{case} \\{eol}:\5
\&{if} (\|j)\1\5
\\{panic}(\.{"Unexpected\ eol"});\2\6
${}\\{viable}\K\T{0};{}$\6
\4${}\}{}$\2\6
\&{break};\6
\4${}\}{}$\2\par
\U84.\fi

\M{87}Contrariwise, an erased \.) is like an erased \.( but vice versa.

\Y\B\4\X87:Downgrade the predecessor of the \PB{\\{rt}}\X${}\E{}$\6
\&{for} ${}(\|k\K\\{row\_end}-\T{1},\39\|j\K\T{0};{}$  ; ${}\|k\MM){}$\5
${}\{{}$\1\6
\&{switch} (\|c[\|k])\5
${}\{{}$\1\6
\4\&{case} \\{rt}:\5
${}\|j\PP;{}$\6
\4\&{case} \\{zero}:\5
\&{case} \\{one}:\5
\&{continue};\6
\4\&{case} \\{mid}:\5
\&{if} (\|j)\1\5
\&{continue};\2\6
${}\|c[\|k]\K\\{rt}{}$;\5
\&{break};\6
\4\&{case} \\{lft}:\5
\&{if} (\|j)\5
${}\{{}$\5
\1${}\|j\MM{}$;\5
\&{continue};\5
${}\}{}$\2\6
${}\|c[\|k]\K\\{one}{}$;\5
\&{break};\6
\4\&{case} \\{eol}:\5
\&{if} (\|j)\1\5
\\{panic}(\.{"Unexpected\ eol"});\2\6
${}\\{viable}\K\T{0};{}$\6
\4${}\}{}$\2\6
\&{break};\6
\4${}\}{}$\2\par
\U84.\fi

\M{88}\B\X88:Cautiously delete a \PB{\\{mid}} that may be leftmost\X${}\E{}$\6
\&{for} ${}(\|k\K\\{row\_end}-\T{1};{}$ ${}\|c[\|k]\E\\{zero};{}$ ${}\|k\MM){}$%
\1\5
;\2\6
\&{if} ${}(\|c[\|k]\E\\{eol}){}$\5
${}\{{}$\C{ yes, the \PB{\\{mid}} was leftmost }\1\6
\&{for} ${}(\|k\K\\{row\_end}+\T{1};{}$ ${}\|c[\|k]\E\\{zero};{}$ ${}\|k\PP){}$%
\1\5
;\2\6
\&{switch} (\|c[\|k])\5
${}\{{}$\1\6
\4\&{case} \\{mid}:\5
\&{case} \\{rt}:\5
\&{case} \\{eol}:\5
\&{break};\C{ no problem }\6
\4\&{default}:\5
\&{if} ${}(\|c[\|k]\E\\{one}){}$\1\5
${}\|c[\|k]\K\\{rt},\39\|j\K\T{0};{}$\2\6
\&{else}\1\5
${}\|c[\|k]\K\\{mid},\39\|j\K\T{1}{}$;\C{ \PB{\|c[\|k]} was \PB{\\{lft}} }\2\6
\&{for} ${}(\|k\PP;{}$  ; ${}\|k\PP){}$\5
${}\{{}$\C{ we must downgrade the successor of the \PB{\\{mid}} }\1\6
\&{switch} (\|c[\|k])\5
${}\{{}$\1\6
\4\&{case} \\{lft}:\5
${}\|j\PP;{}$\6
\4\&{case} \\{zero}:\5
\&{case} \\{one}:\5
\&{continue};\6
\4\&{case} \\{mid}:\5
\&{if} (\|j)\1\5
\&{continue};\2\6
${}\|c[\|k]\K\\{lft}{}$;\5
\&{break};\6
\4\&{case} \\{rt}:\5
\&{if} (\|j)\5
${}\{{}$\5
\1${}\|j\MM{}$;\5
\&{continue};\5
${}\}{}$\2\6
${}\|c[\|k]\K\\{one}{}$;\5
\&{break};\6
\4\&{case} \\{eol}:\5
\\{panic}(\.{"This\ can't\ happen"});\6
\4${}\}{}$\2\6
\&{break};\6
\4${}\}{}$\2\6
\4${}\}{}$\2\6
\4${}\}{}$\2\par
\U84.\fi

\M{89}\B\X89:Cautiously delete a \PB{\\{mid}} that may be rightmost\X${}\E{}$\6
\&{for} ${}(\|k\K\\{row\_end}+\T{1};{}$ ${}\|c[\|k]\E\\{zero};{}$ ${}\|k\PP){}$%
\1\5
;\2\6
\&{if} ${}(\|c[\|k]\E\\{eol}){}$\5
${}\{{}$\C{ yes, the \PB{\\{mid}} was rightmost }\1\6
\&{for} ${}(\|k\K\\{row\_end}-\T{1};{}$ ${}\|c[\|k]\E\\{zero};{}$ ${}\|k\MM){}$%
\1\5
;\2\6
\&{switch} (\|c[\|k])\5
${}\{{}$\1\6
\4\&{case} \\{mid}:\5
\&{case} \\{lft}:\5
\&{case} \\{eol}:\5
\&{break};\C{ no problem }\6
\4\&{default}:\5
\&{if} ${}(\|c[\|k]\E\\{one}){}$\1\5
${}\|c[\|k]\K\\{lft},\39\|j\K\T{0};{}$\2\6
\&{else}\1\5
${}\|c[\|k]\K\\{mid},\39\|j\K\T{1}{}$;\C{ \PB{\|c[\|k]} was \PB{\\{rt}} }\2\6
\&{for} ${}(\|k\MM;{}$  ; ${}\|k\MM){}$\5
${}\{{}$\C{ we must downgrade the predecessor of the \PB{\\{mid}} }\1\6
\&{switch} (\|c[\|k])\5
${}\{{}$\1\6
\4\&{case} \\{rt}:\5
${}\|j\PP;{}$\6
\4\&{case} \\{zero}:\5
\&{case} \\{one}:\5
\&{continue};\6
\4\&{case} \\{mid}:\5
\&{if} (\|j)\1\5
\&{continue};\2\6
${}\|c[\|k]\K\\{rt}{}$;\5
\&{break};\6
\4\&{case} \\{lft}:\5
\&{if} (\|j)\5
${}\{{}$\5
\1${}\|j\MM{}$;\5
\&{continue};\5
${}\}{}$\2\6
${}\|c[\|k]\K\\{one}{}$;\5
\&{break};\6
\4\&{case} \\{eol}:\5
\\{panic}(\.{"This\ can't\ happen"});\6
\4${}\}{}$\2\6
\&{break};\6
\4${}\}{}$\2\6
\4${}\}{}$\2\6
\4${}\}{}$\2\par
\U84.\fi

\M{90}A different kind of excitement awaits us when we consider
occupying the new cell.

If the cases \PB{$\|f(\\{lft},\\{mid})$}, \PB{$\|f(\\{lft},\\{rt})$}, \PB{$\|f(%
\\{mid},\\{mid})$}, and \PB{$\|f(\\{mid},\\{rt})$}
are modified here to set \PB{$\\{viable}\K\T{0}$}, the program will count
{\it polyomino trees\/} instead of normal polyominoes. (In a
polyomino tree there is exactly one way to get from one cell to another
via rook moves.) These are the four cases in which already-connected
cells are connected again.
$$\vbox{\offinterlineskip
\halign{\strut\hfil\tt#\quad&\vrule#&
\hbox to4em{\hfil\tt#\hfil}&
\hbox to4em{\hfil\tt#\hfil}&
\hbox to4em{\hfil\tt#\hfil}&
\hbox to4em{\hfil\tt#\hfil}&
\hbox to4em{\hfil\tt#\hfil}&
\vrule#\cr
&\omit&0&1&(&-&)\cr
\noalign{\vskip2pt}
\omit&&\multispan5\hrulefill&\cr
\omit&height3pt&&&&&&\cr
0&&01\sp(k)&01&    0(&0-\relax&0)\relax&\cr
1&&()&    ()&(-\relax&--\relax&-)\relax&\cr
(&&(-&\sp(e)&  \sp(e)&(-\sp(f)&()\sp(f)&\cr
-&&--&    --&--\sp(g)&--\sp(f)&-)\sp(f)&\cr
)&&-)&    -)&--\relax&--\sp(h)&-)\sp(h)&\cr
\rm left edge&&)\sp(i)&\sp(e)&\sp(e)&-&)&\cr
\omit&height1pt&&&&&&\cr
\omit&&\multispan5\hrulefill&\cr
\noalign{\smallskip}
\key(e) Impossible case\hfil\cr
\key(f) Not viable in polyomino trees\hfil\cr
\key(g) Merge with mate of \.(\hfil\cr
\key(h) Merge with mate of \.)\hfil\cr
\key(i) Downgrade the next component if open\hfil\cr
\key(j) Downgrade the previous component if open\hfil\cr
\key(k) Or possibly \.{0)\sp(i)} or \.{0(\sp(j)}\hfil\cr
}}$$
The somewhat unusual case \.{01\sp(k)}\ \ becomes \.{0)\sp(i)} if
no nonzero cells lie to the left but the left edge has already
been occupied somewhere in the rows above. It becomes \.{0(\sp(j)}
if no nonzero cells lie to the right but the right edge has already
been occupied somewhere in the rows above.

\Y\B\4\X90:Change array \PB{\|c} for target $\alpha_1$\X${}\E{}$\6
$\\{viable}\K\T{1};{}$\6
${}\\{src}\MG\\{lo}\PP,\39\\{src}\MG\\{hi}\PP{}$;\C{ implicitly multiply the
generating function $g(\alpha)$ by $z$ }\6
\&{switch} (\\{pair})\5
${}\{{}$\1\6
\4\&{case} \|f${}(\\{one},\39\\{zero}){}$:\5
\&{case} \|f${}(\\{one},\39\\{one}){}$:\5
${}\|c[\\{row\_end}-\T{1}]\K\\{lft},\39\|c[\\{row\_end}]\K\\{rt};{}$\6
\4\&{case} \|f${}(\\{zero},\39\\{one}){}$:\5
\&{case} \|f${}(\\{zero},\39\\{lft}){}$:\5
\&{case} \|f${}(\\{zero},\39\\{mid}){}$:\5
\&{case} \|f${}(\\{zero},\39\\{rt}){}$:\5
\&{case} \|f${}(\\{lft},\39\\{mid}){}$:\5
\&{case} \|f${}(\\{lft},\39\\{rt}){}$:\5
\&{case} \|f${}(\\{mid},\39\\{mid}){}$:\5
\&{case} \|f${}(\\{mid},\39\\{rt}){}$:\5
\&{case} \|f${}(\\{eol},\39\\{mid}){}$:\5
\&{case} \|f${}(\\{eol},\39\\{rt}){}$:\5
\&{break};\6
\4\&{case} \|f${}(\\{one},\39\\{lft}){}$:\5
${}\|c[\\{row\_end}-\T{1}]\K\\{lft},\39\|c[\\{row\_end}]\K\\{mid}{}$;\5
\&{break};\6
\4\&{case} \|f${}(\\{one},\39\\{mid}){}$:\5
\&{case} \|f${}(\\{one},\39\\{rt}){}$:\5
${}\|c[\\{row\_end}-\T{1}]\K\\{mid}{}$;\5
\&{break};\6
\4\&{case} \|f${}(\\{lft},\39\\{zero}){}$:\5
\&{case} \|f${}(\\{mid},\39\\{zero}){}$:\5
\&{case} \|f${}(\\{mid},\39\\{one}){}$:\5
${}\|c[\\{row\_end}]\K\\{mid}{}$;\5
\&{break};\6
\4\&{case} \|f${}(\\{mid},\39\\{lft}){}$:\5
${}\|c[\\{row\_end}]\K\\{mid};{}$\6
\X91:Merge with the mate of the former \PB{\\{lft}}\X;\5
\&{break};\6
\4\&{case} \|f${}(\\{rt},\39\\{zero}){}$:\5
\&{case} \|f${}(\\{rt},\39\\{one}){}$:\5
${}\|c[\\{row\_end}-\T{1}]\K\\{mid},\39\|c[\\{row\_end}]\K\\{rt}{}$;\5
\&{break};\6
\4\&{case} \|f${}(\\{rt},\39\\{lft}){}$:\5
${}\|c[\\{row\_end}-\T{1}]\K\|c[\\{row\_end}]\K\\{mid}{}$;\5
\&{break};\6
\4\&{case} \|f${}(\\{rt},\39\\{mid}){}$:\5
\&{case} \|f${}(\\{rt},\39\\{rt}){}$:\5
${}\|c[\\{row\_end}-\T{1}]\K\\{mid};{}$\6
\X92:Merge with the mate of the former \PB{\\{rt}}\X;\5
\&{break};\6
\4\&{case} \|f${}(\\{eol},\39\\{zero}){}$:\5
${}\|c[\\{row\_end}]\K\\{rt};{}$\6
\X93:Downgrade the next component if it is open\X;\5
\&{break};\6
\4\&{case} \|f${}(\\{zero},\39\\{zero}){}$:\5
\X94:Cautiously introduce a new \PB{\\{one}}\X;\5
\&{break};\6
\4\&{case} \|f${}(\\{lft},\39\\{one}){}$:\5
\&{case} \|f${}(\\{lft},\39\\{lft}){}$:\5
\&{case} \|f${}(\\{eol},\39\\{one}){}$:\5
\&{case} \|f${}(\\{eol},\39\\{lft}){}$:\5
\\{panic}(\.{"Impossible\ configur}\)\.{ation"});\6
\4\&{default}:\5
\\{panic}(\.{"Impossible\ pair"});\6
\4${}\}{}$\2\6
\&{if} ${}(\\{row\_end}\E\|w){}$\1\5
\X95:Make special corrections at the right edge\X;\2\par
\Us80\ET83.\fi

\M{91}\B\X91:Merge with the mate of the former \PB{\\{lft}}\X${}\E{}$\6
\&{for} ${}(\|k\K\\{row\_end}+\T{1},\39\|j\K\T{0};{}$  ; ${}\|k\PP){}$\5
${}\{{}$\1\6
\&{switch} (\|c[\|k])\5
${}\{{}$\1\6
\4\&{case} \\{lft}:\5
${}\|j\PP;{}$\6
\4\&{case} \\{zero}:\5
\&{case} \\{one}:\5
\&{case} \\{mid}:\5
\&{continue};\6
\4\&{case} \\{rt}:\5
\&{if} ${}(\R\|j){}$\1\5
\&{break};\2\6
${}\|j\MM{}$;\5
\&{continue};\6
\4\&{case} \\{eol}:\5
\\{panic}(\.{"Unexpected\ eol"});\6
\4${}\}{}$\2\6
${}\|c[\|k]\K\\{mid}{}$;\5
\&{break};\6
\4${}\}{}$\2\par
\U90.\fi

\M{92}\B\X92:Merge with the mate of the former \PB{\\{rt}}\X${}\E{}$\6
\&{for} ${}(\|k\K\\{row\_end}-\T{2},\39\|j\K\T{0};{}$  ; ${}\|k\MM){}$\5
${}\{{}$\1\6
\&{switch} (\|c[\|k])\5
${}\{{}$\1\6
\4\&{case} \\{rt}:\5
${}\|j\PP;{}$\6
\4\&{case} \\{zero}:\5
\&{case} \\{one}:\5
\&{case} \\{mid}:\5
\&{continue};\6
\4\&{case} \\{lft}:\5
\&{if} ${}(\R\|j){}$\1\5
\&{break};\2\6
${}\|j\MM{}$;\5
\&{continue};\6
\4\&{case} \\{eol}:\5
\\{panic}(\.{"Unexpected\ eol"});\6
\4${}\}{}$\2\6
${}\|c[\|k]\K\\{mid}{}$;\5
\&{break};\6
\4${}\}{}$\2\par
\U90.\fi

\M{93}\B\X93:Downgrade the next component if it is open\X${}\E{}$\6
\&{for} ${}(\|k\K\T{2};{}$  ; ${}\|k\PP){}$\5
${}\{{}$\1\6
\&{switch} (\|c[\|k])\5
${}\{{}$\1\6
\4\&{case} \\{zero}:\5
\&{continue};\6
\4\&{case} \\{mid}:\5
${}\|c[\|k]\K\\{lft};{}$\6
\4\&{case} \\{one}:\5
\&{case} \\{lft}:\5
\&{case} \\{eol}:\5
\&{break};\6
\4\&{case} \\{rt}:\5
${}\|c[\|k]\K\\{one};{}$\6
\4${}\}{}$\2\6
\&{break};\6
\4${}\}{}$\2\par
\U90.\fi

\M{94}\B\X94:Cautiously introduce a new \PB{\\{one}}\X${}\E{}$\6
$\|c[\\{row\_end}]\K\\{one};{}$\6
\&{for} ${}(\|k\K\\{row\_end}-\T{2};{}$ ${}\|c[\|k]\E\\{zero};{}$ ${}\|k\MM){}$%
\1\5
;\2\6
\&{if} ${}(\R\|k){}$\5
${}\{{}$\C{ we're introducing a new leftmost \.1 }\1\6
\&{for} ${}(\|k\K\\{row\_end}+\T{1};{}$  ; ${}\|k\PP){}$\5
${}\{{}$\1\6
\&{switch} (\|c[\|k])\5
${}\{{}$\1\6
\4\&{case} \\{zero}:\5
\&{continue};\6
\4\&{case} \\{mid}:\5
${}\|c[\|k]\K\\{lft},\39\|c[\\{row\_end}]\K\\{rt};{}$\6
\4\&{case} \\{one}:\5
\&{case} \\{lft}:\5
\&{case} \\{eol}:\5
\&{break};\6
\4\&{case} \\{rt}:\5
${}\|c[\|k]\K\\{one},\39\|c[\\{row\_end}]\K\\{rt};{}$\6
\4${}\}{}$\2\6
\&{break};\6
\4${}\}{}$\2\6
\4${}\}{}$\5
\2\&{else}\5
${}\{{}$\1\6
\&{for} ${}(\|j\K\\{row\_end}+\T{1};{}$ ${}\|c[\|j]\E\\{zero};{}$ ${}\|j\PP){}$%
\1\5
;\2\6
\&{if} ${}(\|c[\|j]\E\\{eol}){}$\5
${}\{{}$\C{ we're introducing a new rightmost \.1 }\1\6
\&{if} ${}(\|c[\|k]\E\\{mid}){}$\1\5
${}\|c[\|k]\K\\{rt},\39\|c[\\{row\_end}]\K\\{lft};{}$\2\6
\&{else} \&{if} ${}(\|c[\|k]\E\\{lft}){}$\1\5
${}\|c[\|k]\K\\{one},\39\|c[\\{row\_end}]\K\\{lft};{}$\2\6
\4${}\}{}$\2\6
\4${}\}{}$\2\par
\U90.\fi

\M{95}\B\X95:Make special corrections at the right edge\X${}\E{}$\6
\&{switch} (\|c[\\{row\_end}])\5
${}\{{}$\1\6
\4\&{case} \\{rt}:\5
${}\|c[\\{row\_end}]\K\\{mid};{}$\6
\4\&{case} \\{zero}:\5
\&{case} \\{mid}:\5
\&{case} \\{lft}:\5
\&{break};\6
\4\&{case} \\{one}:\5
${}\|c[\\{row\_end}]\K\\{lft};{}$\6
\X96:Downgrade the previous component if it is open\X;\6
\4${}\}{}$\2\par
\U90.\fi

\M{96}\B\X96:Downgrade the previous component if it is open\X${}\E{}$\6
\&{for} ${}(\|k\K\\{row\_end}-\T{1};{}$  ; ${}\|k\MM){}$\5
${}\{{}$\1\6
\&{switch} (\|c[\|k])\5
${}\{{}$\1\6
\4\&{case} \\{zero}:\5
\&{continue};\6
\4\&{case} \\{mid}:\5
${}\|c[\|k]\K\\{rt};{}$\6
\4\&{case} \\{one}:\5
\&{case} \\{rt}:\5
\&{case} \\{eol}:\5
\&{break};\6
\4\&{case} \\{lft}:\5
${}\|c[\|k]\K\\{one};{}$\6
\4${}\}{}$\2\6
\&{break};\6
\4${}\}{}$\2\par
\U95.\fi

\N{1}{97}Nittier-and-grittier. The last nontrivial hurdle facing us
is the problem of what to do after a target configuration has
been constructed in \PB{\|c[\T{1}]} through \PB{\|c[\|w]}. It's not a Big
Problem,
but it does require care, especially with respect to the
generating function arithmetic.

An all-\PB{\\{zero}} target configuration is preserved only in the
first few passes, before we've reached the end of row~1.

\Y\B\4\X97:Process target configuration \PB{\|c} (downward)\X${}\E{}$\6
${}\{{}$\1\6
\&{if} ${}(\\{row\_end}\E\|w){}$\1\5
\X100:Canonize the configuration\X;\2\6
${}\\{target}\K\\{packit}(\,);{}$\6
\&{if} ${}(\\{target}.\|o.\|l\V\\{target}.\|o.\|h\V(\|r\E\T{1}\W\\{row\_end}<%
\|w)){}$\5
${}\{{}$\1\6
\X101:If \PB{\\{target}} is already present, make \PB{\|p} point to it\X;\6
\&{if} ${}(\R\|p){}$\1\5
\X102:Get a downward slot for \PB{\\{target}} if it is really viable\X;\2\6
\&{if} ${}(\|p\W(\\{src}\MG\\{lo}\Z(\|j\K\|p\MG\\{lim}))){}$\5
${}\{{}$\1\6
\&{if} ${}(\\{src}\MG\\{hi}<\|j){}$\1\5
${}\|j\K\\{src}\MG\\{hi};{}$\2\6
\&{if} ${}(\|j>\|p\MG\\{hi}\V\\{src}\MG\\{lo}<\|p\MG\\{lo}){}$\1\5
${}\|p\K\\{move\_down}(\|p,\39\\{src}\MG\\{lo},\39\|j);{}$\2\6
\&{if} (\\{verbose})\5
${}\{{}$\1\6
\\{printf}(\.{"\ ->\ "});\5
${}\\{print\_config}(\\{row\_end}+\T{1}){}$;\5
\\{printf}(\.{"\\n"});\6
\4${}\}{}$\2\6
${}\\{update}(\\{src},\39\|p,\39\|j);{}$\6
\4${}\}{}$\2\6
\4${}\}{}$\5
\2\&{else} \&{if} ${}(\|r>\|w){}$\5
${}\{{}$\1\6
\&{if} (\\{verbose})\1\5
\\{printf}(\.{"\ ->\ 0\\n"});\2\6
${}\\{update}(\\{src},\39\NULL,\39\\{src}\MG\\{hi}){}$;\C{ polyominoes
completed }\6
\4${}\}{}$\2\6
\4${}\}{}$\2\par
\U80.\fi

\M{98}\B\X7:Global variables\X${}\mathrel+\E{}$\6
\&{cstring} \\{target};\C{ the packed name of the current target configuration
}\6
\&{int} \\{hash};\C{ its hash address }\par
\fi

\M{99}\B\X67:Local variables\X${}\mathrel+\E{}$\6
\&{register} \&{config} ${}{*}\|p{}$;\C{ current target of interest }\par
\fi

\M{100}At the end of a row we change the configuration to its left-right
reflection, if the reflection is lexicographically smaller. This reduction to
a canonical form reduces the number of active configurations by a factor of
nearly~2, at least for the next few passes. (The reduction can be justified
by observing that we could have operated from right to left instead of from
left to right, on each row that follows a left-heavy row.)

Notice that a code like \.{0(0} will be reflected to \.{0)0}.

\Y\B\4\X100:Canonize the configuration\X${}\E{}$\6
${}\{{}$\1\6
\&{for} ${}(\|j\K\T{1},\39\|k\K\|w;{}$ ${}\|j\Z\|k;{}$ ${}\|j\PP,\39\|k\MM){}$%
\1\6
\&{if} ${}(\|c[\|j]\I\\{reflect}[\|c[\|k]]){}$\1\5
\&{break};\2\2\6
\&{if} ${}(\|c[\|j]>\\{reflect}[\|c[\|k]]){}$\1\6
\&{for} ( ; ${}\|j\Z\|k;{}$ ${}\|j\PP,\39\|k\MM){}$\5
${}\{{}$\1\6
\&{register} \&{int} \|i${}\K\|c[\|k];{}$\7
${}\|c[\|k]\K\\{reflect}[\|c[\|j]];{}$\6
${}\|c[\|j]\K\\{reflect}[\|i];{}$\6
\4${}\}{}$\2\2\6
\4${}\}{}$\2\par
\Us97\ET103.\fi

\M{101}We take care to move \PB{\|p} to the top of its hash list, when present,
and to set the global variable \PB{\\{hash}} as required by \PB{\\{move\_down}}
and \PB{\\{move\_up}}.

\Y\B\4\X101:If \PB{\\{target}} is already present, make \PB{\|p} point to it%
\X${}\E{}$\6
$\\{hash}\K\\{mangle}(\\{target});{}$\6
${}\|p\K\\{hash\_table}[\\{hash}];{}$\6
\&{if} ${}(\|p\W\R(\|p\MG\|s.\|o.\|l\E\\{target}.\|o.\|l\W\|p\MG\|s.\|o.\|h\E%
\\{target}.\|o.\|h)){}$\5
${}\{{}$\1\6
\&{register} \&{config} ${}{*}\|q;{}$\7
\&{for} ${}(\|q\K\|p,\39\|p\K\|p\MG\\{link};{}$ \|p; ${}\|q\K\|p,\39\|p\K\|p\MG%
\\{link}){}$\1\6
\&{if} ${}(\|p\MG\|s.\|o.\|l\E\\{target}.\|o.\|l\W\|p\MG\|s.\|o.\|h\E%
\\{target}.\|o.\|h){}$\1\5
\&{break};\2\2\6
\&{if} (\|p)\5
${}\{{}$\1\6
${}\|q\MG\\{link}\K\|p\MG\\{link}{}$;\C{ remove \PB{\|p} from its former place
in the list }\6
${}\|p\MG\\{link}\K\\{hash\_table}[\\{hash}]{}$;\C{ and insert it at the front
}\6
${}\\{hash\_table}[\\{hash}]\K\|p;{}$\6
\4${}\}{}$\2\6
\4${}\}{}$\2\par
\Us97\ET103.\fi

\M{102}If the target is viable, \PB{\|p} will be set to a fresh node with
\PB{$\|p\MG\\{state}\K\\{raw}$}. The reader can verify that \PB{\\{move\_down}}
will not
then be necessary.

\Y\B\4\X102:Get a downward slot for \PB{\\{target}} if it is really viable\X${}%
\E{}$\6
${}\{{}$\1\6
${}\|j\K\\{connectivity}(\\{row\_end}+\T{1});{}$\6
\&{if} ${}(\|r\G\|w){}$\1\5
${}\|j\K\\{int\_part}(\|j);{}$\2\6
\&{else} \&{if} ${}(\\{int\_part}(\|j)\E\\{eps\_part}(\|j)){}$\1\5
${}\|j\K\\{int\_part}(\|j)+(\|w-\|r);{}$\2\6
\&{else}\1\5
${}\|j\K\\{int\_part}(\|j)+(\|w-\T{1}-\|r){}$;\C{ \PB{\|j} more cells are
needed in a valid polyomino }\2\6
\&{if} ${}(\\{src}\MG\\{lo}+\|j\Z\|n){}$\5
${}\{{}$\1\6
\&{if} ${}(\PP\\{configs}\E\T{1000000000}){}$\1\5
${}\\{configs}\K\T{0},\39\\{hconfigs}\PP;{}$\2\6
${}\|p\K\\{get\_slot\_down}((\\{src}\MG\\{hi}>\|n-\|j\?\|n-\|j:\\{src}\MG%
\\{hi})-\\{src}\MG\\{lo});{}$\6
${}\|p\MG\\{link}\K\\{hash\_table}[\\{hash}],\39\\{hash\_table}[\\{hash}]\K%
\|p;{}$\6
${}\|p\MG\|s\K\\{target};{}$\6
${}\|p\MG\\{lo}\K\\{src}\MG\\{lo},\39\|p\MG\\{hi}\K\\{src}\MG\\{hi},\39\|p\MG%
\\{lim}\K\|n-\|j;{}$\6
\&{if} ${}(\|p\MG\\{hi}>\|p\MG\\{lim}){}$\1\5
${}\|p\MG\\{hi}\K\|p\MG\\{lim};{}$\2\6
\4${}\}{}$\2\6
\4${}\}{}$\2\par
\U97.\fi

\M{103}\B\X103:Process target configuration \PB{\|c} (upward)\X${}\E{}$\6
${}\{{}$\1\6
\&{if} ${}(\\{row\_end}\E\|w){}$\1\5
\X100:Canonize the configuration\X;\2\6
${}\\{target}\K\\{packit}(\,);{}$\6
\&{if} ${}(\\{target}.\|o.\|l\V\\{target}.\|o.\|h\V(\|r\E\T{1}\W\\{row\_end}<%
\|w)){}$\5
${}\{{}$\1\6
\X101:If \PB{\\{target}} is already present, make \PB{\|p} point to it\X;\6
\&{if} ${}(\R\|p){}$\1\5
\X104:Get an upward slot for \PB{\\{target}} if it is really viable\X;\2\6
\&{if} ${}(\|p\W(\\{src}\MG\\{lo}\Z(\|j\K\|p\MG\\{lim}))){}$\5
${}\{{}$\1\6
\&{if} ${}(\\{src}\MG\\{hi}<\|j){}$\1\5
${}\|j\K\\{src}\MG\\{hi};{}$\2\6
\&{if} ${}(\|j>\|p\MG\\{hi}\V\\{src}\MG\\{lo}<\|p\MG\\{lo}){}$\1\5
${}\|p\K\\{move\_up}(\|p,\39\\{src}\MG\\{lo},\39\|j);{}$\2\6
\&{if} (\\{verbose})\5
${}\{{}$\1\6
\\{printf}(\.{"\ ->\ "});\5
${}\\{print\_config}(\\{row\_end}+\T{1}){}$;\5
\\{printf}(\.{"\\n"});\6
\4${}\}{}$\2\6
${}\\{update}(\\{src},\39\|p,\39\|j);{}$\6
\4${}\}{}$\2\6
\4${}\}{}$\5
\2\&{else} \&{if} ${}(\|r>\|w){}$\5
${}\{{}$\1\6
\&{if} (\\{verbose})\1\5
\\{printf}(\.{"\ ->\ 0\\n"});\2\6
${}\\{update}(\\{src},\39\NULL,\39\\{src}\MG\\{hi}){}$;\C{ polyominoes
completed }\6
\4${}\}{}$\2\6
\4${}\}{}$\2\par
\U83.\fi

\M{104}\B\X104:Get an upward slot for \PB{\\{target}} if it is really viable%
\X${}\E{}$\6
${}\{{}$\1\6
${}\|j\K\\{connectivity}(\\{row\_end}+\T{1});{}$\6
\&{if} ${}(\|r\G\|w){}$\1\5
${}\|j\K\\{int\_part}(\|j);{}$\2\6
\&{else} \&{if} ${}(\\{int\_part}(\|j)\E\\{eps\_part}(\|j)){}$\1\5
${}\|j\K\\{int\_part}(\|j)+(\|w-\|r);{}$\2\6
\&{else}\1\5
${}\|j\K\\{int\_part}(\|j)+(\|w-\T{1}-\|r){}$;\C{ \PB{\|j} more cells are
needed in a valid polyomino }\2\6
\&{if} ${}(\\{src}\MG\\{lo}+\|j\Z\|n){}$\5
${}\{{}$\1\6
\&{if} ${}(\PP\\{configs}\E\T{1000000000}){}$\1\5
${}\\{configs}\K\T{0},\39\\{hconfigs}\PP;{}$\2\6
${}\|p\K\\{get\_slot\_up}((\\{src}\MG\\{hi}>\|n-\|j\?\|n-\|j:\\{src}\MG\\{hi})-%
\\{src}\MG\\{lo});{}$\6
${}\|p\MG\\{link}\K\\{hash\_table}[\\{hash}],\39\\{hash\_table}[\\{hash}]\K%
\|p;{}$\6
${}\|p\MG\|s\K\\{target};{}$\6
${}\|p\MG\\{lo}\K\\{src}\MG\\{lo},\39\|p\MG\\{hi}\K\\{src}\MG\\{hi},\39\|p\MG%
\\{lim}\K\|n-\|j;{}$\6
\&{if} ${}(\|p\MG\\{hi}>\|p\MG\\{lim}){}$\1\5
${}\|p\MG\\{hi}\K\|p\MG\\{lim};{}$\2\6
\4${}\}{}$\2\6
\4${}\}{}$\2\par
\U103.\fi

\N{1}{105}Checkpointing. One of the goals of this program is to establish
new world records. Thus, local resources
are probably being stretched to their current limits, and several days
of running time might well be involved.

It's prudent therefore to make the program stop at a suitable
``checkpoint,'' firming up what has been accomplished so far;
then we won't have to go back to square one when recovering
from a disaster. We should also use {\mc POLYSLAVE} to reduce the
intermediate data at such times, thereby freeing up nearly all
of the disk space we've been filling before we proceed to fill some more.

The code in this section is executed at a particularly convenient
time: A new row is about to begin, and so is a new downward pass.
It's as good a time as any to dump out the configuration-table-so-far
in a form that can easily be used by a special version of this program
to get going again when we're ready to resume. (See the change file
\.{polynum-restart.ch} for details.)

\Y\B\4\D$\\{gig\_threshold}$ \5
\T{5}\C{ try to avoid filling more than about twice this      many gigabytes of
disk space }\par
\Y\B\4\X105:Check if this run has gone on too long\X${}\E{}$\6
\&{if} ${}(\\{file\_extension}\G\\{gig\_threshold}\W\\{trg}\I\\{conf}){}$\5
${}\{{}$\1\6
\X106:Shut down the {\mc POLYSLAVE} process\X;\6
${}\\{sprintf}(\\{filename},\39\.{"\%.90s.dump"},\39\\{base\_name});{}$\6
${}\\{out\_file}\K\\{fopen}(\\{filename},\39\.{"wb"});{}$\6
\&{if} ${}(\R\\{out\_file}){}$\1\5
\\{panic}(\.{"I\ can't\ open\ the\ du}\)\.{mp\ file"});\2\6
\X107:Dump all information needed to restart\X;\6
\X72:Print statistics about this run\X;\6
${}\\{printf}(\.{"[\%d\ bytes\ written\ o}\)\.{n\ file\ \%s.]\\n"},\39%
\\{ftell}(\\{out\_file}),\39\\{filename});{}$\6
\\{exit}(\T{1});\6
\4${}\}{}$\2\par
\U79.\fi

\M{106}A special \PB{\\{sync}} instruction with parameter 255 tells {\mc
POLYSLAVE}
that it should invoke its own checkpointing activity.

\Y\B\4\X106:Shut down the {\mc POLYSLAVE} process\X${}\E{}$\6
$\\{put\_inst}(\\{sync},\39\T{255});{}$\6
\X16:Empty the buffer and close the output file\X;\6
\\{printf}(\.{"Checkpoint\ stop:\ Pl}\)\.{ease\ process\ that\ da}\)\.{ta\ with%
\ polyslave,\\n}\)\.{"});\6
\\{printf}(\.{"then\ resume\ the\ com}\)\.{putation\ with\ polynu}\)%
\.{m-restart.\\n"});\par
\U105.\fi

\M{107}Since we're at the beginning of a downward pass, the user will be
able to restart this program with different values of \PB{\\{conf\_size}} and
\PB{\\{slave\_size}} if desired.

\Y\B\4\X107:Dump all information needed to restart\X${}\E{}$\6
$\\{dump\_data}[\T{0}]\K\|n;{}$\6
${}\\{dump\_data}[\T{1}]\K\|w;{}$\6
${}\\{dump\_data}[\T{2}]\K\|r;{}$\6
${}\\{dump\_data}[\T{3}]\K\\{trg}-\\{conf};{}$\6
${}\\{dump\_data}[\T{4}]\K\\{strg};{}$\6
\&{if} ${}(\\{fwrite}(\\{dump\_data},\39\&{sizeof}(\&{int}),\39\T{5},\39\\{out%
\_file})\I\T{5}){}$\1\5
\\{panic}(\.{"Bad\ write\ at\ beginn}\)\.{ing\ of\ dump"});\2\6
\&{if} ${}(\\{fwrite}(\\{conf},\39\&{sizeof}(\&{config}),\39\\{trg}-\\{conf},%
\39\\{out\_file})\I\\{trg}-\\{conf}){}$\1\5
\\{panic}(\.{"Couldn't\ dump\ the\ c}\)\.{onfiguration\ table"});\2\par
\U105.\fi

\M{108}\B\X7:Global variables\X${}\mathrel+\E{}$\6
\&{int} \\{dump\_data}[\T{5}];\C{ parameters needed to restart }\par
\fi

\N{1}{109}Computational experience. With a suitable change file
it is not difficult to convert this program to a one-pass routine
that does the evaluation directly, provided that $n$ and $w$ are
reasonably small. For example, when $n=30$ and $2\le w\le15$,
all the computations were completed in 192 seconds (on 12 December 2000).
The most difficult case, which took 68 seconds to complete, was
for $w=13$, when
100,488 slots (with 0 holes)
and 218980 counters (with 4114 wasted)
were needed.

The resulting number of $n$-ominoes for $n\le30$ agreed perfectly with the
answers obtained from a completely different algorithm, using my now-obsolete
program {\mc POLYENUM}. That program had taken more than 15 hours
to count 30-ominoes, so Jensen's method ran almost 300 times faster.

Setting $n=47$ led to much more of an adventure, of course, since
all space and time requirements grow exponentially. Some runs lasted
several days, and various glitches and hardware failures added to the
excitement. Detailed statistics about the performance, including the
histograms computed here, were helpful for planning and
for diagnosing various problems.

\Y\B\4\D$\\{hist\_size}$ \5
\T{100}\par
\Y\B\4\X109:Print and clear the hash/slot tables\X${}\E{}$\6
\&{for} ${}(\|k\K\T{0};{}$ ${}\|k<\\{hist\_size};{}$ ${}\|k\PP){}$\1\5
${}\\{hhist}[\|k]\K\T{0};{}$\2\6
\&{for} ${}(\|k\K\T{0};{}$ ${}\|k\Z\\{nmax};{}$ ${}\|k\PP){}$\1\5
${}\\{chist}[\|k]\K\T{0};{}$\2\6
${}\\{jj}\K\T{0};{}$\6
\&{for} ${}(\|k\K\T{0};{}$ ${}\|k\Z\\{hash\_mask};{}$ ${}\|k\PP){}$\5
${}\{{}$\1\6
\&{for} ${}(\|p\K\\{hash\_table}[\|k],\39\|j\K\T{0};{}$ \|p; ${}\|p\K\|p\MG%
\\{link},\39\|j\PP){}$\1\5
${}\\{chist}[\|p\MG\\{hi}-\|p\MG\\{lo}]\PP;{}$\2\6
\&{if} ${}(\|j>\\{jj}){}$\5
${}\{{}$\1\6
\&{if} ${}(\|j\G\\{hist\_size}){}$\1\5
${}\|j\K\\{hist\_size}-\T{1};{}$\2\6
${}\\{jj}\K\|j;{}$\6
\4${}\}{}$\2\6
${}\\{hhist}[\|j]\PP;{}$\6
${}\\{hash\_table}[\|k]\K\NULL;{}$\6
\4${}\}{}$\2\6
\\{printf}(\.{"Hash\ histogram:"});\6
\&{for} ${}(\|j\K\T{1};{}$ ${}\|j\Z\\{jj};{}$ ${}\|j\PP){}$\1\5
${}\\{printf}(\.{"\ \%d"},\39\\{hhist}[\|j]);{}$\2\6
\\{printf}(\.{"\\nCounters:"});\6
\&{for} ${}(\|k\K\\{nmax};{}$ ${}\|k\G\T{0};{}$ ${}\|k\MM){}$\1\6
\&{if} (\\{chist}[\|k])\1\5
\&{break};\2\2\6
\&{for} ${}(\|j\K\T{0};{}$ ${}\|j\Z\|k;{}$ ${}\|j\PP){}$\1\5
${}\\{printf}(\.{"\ \%d"},\39\\{chist}[\|j]);{}$\2\6
\&{for} ${}(\|k\K\\{nmax};{}$ ${}\|k\G\T{0};{}$ ${}\|k\MM){}$\1\6
\&{if} (\\{slot}[\|k])\1\5
\&{break};\2\2\6
\&{if} ${}(\|k\G\T{0}){}$\5
${}\{{}$\1\6
\\{printf}(\.{"\\nHoles:"});\6
\&{for} ${}(\|j\K\T{0};{}$ ${}\|j\Z\|k;{}$ ${}\|j\PP){}$\5
${}\{{}$\1\6
\&{for} ${}(\|p\K\\{slot}[\|j],\39\\{jj}\K\T{0};{}$ \|p; ${}\|p\K\|p\MG%
\\{link},\39\\{jj}\PP){}$\1\5
;\2\6
${}\\{printf}(\.{"\ \%d"},\39\\{jj});{}$\6
${}\\{slot}[\|j]\K\NULL;{}$\6
\4${}\}{}$\2\6
\4${}\}{}$\2\6
\\{printf}(\.{"\\n"});\6
${}\\{holes}\K\\{sholes}\K\T{0}{}$;\par
\U81.\fi

\M{110}\B\X7:Global variables\X${}\mathrel+\E{}$\6
\&{int} \\{hhist}[\\{hist\_size}];\C{ histogram of hash chain lengths }\6
\&{int} ${}\\{chist}[\\{nmax}+\T{1}]{}$;\C{ histogram of counter table lengths
}\6
\&{int} \\{jj};\C{ auxiliary variable for statistics calculations }\par
\fi

\M{111}The greatest difficulty for $n=47$ occurred when $w=20$; indeed,
more than 100 gigabytes of data were passed to {\mc POLYSLAVE} in
that case, and the computation lasted several days,
so the checkpointing algorithm proved to be particularly helpful.

Here is a summary of the main statistics from those runs. The number of
``configs'' is the total of distinct configurations, summed over all passes.
The number of ``moves'' is the number of times \PB{\\{move\_up}} or \PB{\\{move%
\_down}} was
called to increase the space allocated to a generating function.
$$\vbox{\halign{&\quad\hfil#\cr
$w\ $&slots&counters&configs&moves&bytes&\mc POLYNUM&\mc POLYSLAVE\cr
\noalign{\vskip2pt}
23&    0.3M&    0.3M&   109M&   3M&  0.4G&     14 min&        5 min\cr
22&    6.2M&    8.0M&  2150M& 129M& 10.2G&    267 min\rlap*& 35 min\rlap*\cr
21&   28.6M&   46.5M&  9481M&1053M& 58.8G&   1911 min\rlap*&314 min\rlap*\cr
20&   40.2M&   94.0M& 12852M&2267M&103.6G&   2960 min\rlap*&574 min\rlap*\cr
19&   31.5M&  105.6M&  9183M&2099M& 86.8G&   1803 min\rlap*&497 min\rlap*\cr
18&   19.4M&   85.5M&  5220M&1318M& 52.9G&    749 min\rlap*&324 min\rlap*\cr
17&   10.1M&   58.3M&  2514M& 678M& 26.7G&    280 min\rlap*&172 min\rlap*\cr
16&    4.5M&   34.2M&  1091M& 308M& 11.9G&    137 min\rlap*&104 min\rlap*\cr
15&    1.9M&   18.2M&   437M& 127M&  4.9G&     51 min\rlap*& 44 min\rlap*\cr
14&    0.7M&    8.8M&   167M&  49M&  2.0G&     22 min&       31 min\cr
13&    0.3M&    4.1M&    62M&  19M&  0.8G&      8 min&       12 min\cr
12&     98K&    1.8M&    23M&   7M&  0.3G&      3 min&        5 min\cr
11&     37K&    789K&   8.5M& 2.6M&  0.1G&     84 sec&        2 min\cr
10&     14K&    334K&   3.1M& 0.9M&   40M&     45 sec&       21 sec\cr
9&      5K&    148K&  1124K& 340K&   15M&     30 sec&        7 sec\cr
8&      2K&     59K&   402K& 119K&    5M&     23 sec&        2 sec\cr
7&     875&     24K&   144K&  43K&  1.9M&     20 sec&        1 sec\cr
6&     350&     10K&    50K&  14K&  618K&     17 sec&        0 sec\cr
5&     146&      4K&    17K&   5K&  206K&     14 sec&        0 sec\cr
4&      57&    1484&   5410& 1460&   59K&     11 sec&        0 sec\cr
3&      22&     553&   1658&  430&   17K&      8 sec&        0 sec\cr
2&       7&     180&    318&   76&    3K&      6 sec&        0 sec\cr
}}$$
* Done on computers with 1 gigabyte of memory, thanks to
Andy Kacsmar of Stanford's database group.

\smallskip\noindent (The case $w=24$ was omitted because of the formula
$$
8{h+w-2\choose w-1}-3hw+2h+2w-8,
$$
which gives the total number of $n$-ominoes spanning an $h\times w$
rectangle when $n=h+w-1$ and $h>1$ and $w>1$.)

\fi

\M{112}I was glad to see that the allocation system used here for
variable-length
nodes, allowing ``holes,'' worked quite well: More than 98 percent of
the memory space was typically being put to good use when it was needed.
In fact, only 8 holes were present when the maximum demand of 40,219,325
slots occurred in the runs for $n=47$ (row 10 and column 11
when $w=20$); only 105 counters were wasted when the maximum demand of
105,578,552 counters occurred (row 12 and column 10 when $w=19$).

\fi

\M{113}Joke: George P\'olya was a polymath who worked on polyominoes.

\fi

\N{1}{114}Index.
\fi

\inx
\fin
\con
