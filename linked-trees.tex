\input cwebmac
\hypertextrue\srcloctrue
\datethis

\N[2 linked-trees.w]{1}{1}Intro. This program tests an amazingly simple
algorithm that
generates all $n$-node trees with the property that the $k$th node in preorder
has $t_k$ link fields. (A link field is either zero or it points to
a subtree.) In the special case that $t_k=2$ for $1\le k\le n$, we
get Skarbek's algorithm for binary trees, Algorithm 7.2.1.6B. In the
special case that $t_k=t$ for $1\le k\le n$, we get an algorithm that
was sent to me by James Korsh in December 2004. I~happened to notice
that Korsh's idea works in the general case considered here; but
I'll let him have the fun of constructing a formal proof,
because the basic insights are essentially his.

The number of trees generated does not appear to have a simple formula
in general. But one can show bijectively that such trees are equivalent
to integer sequences $a_1\ldots a_{n-1}$ with the property
that $a_1\ge\cdots\ge a_{n-1}\ge0$ and $a_k\le\sum_{j=1}^{n-k}(t_j-1)$.

The numbers $t_1$, \dots, $t_n$ are input on the command line.

\Y\B\4\D$\\{maxn}$ \5
\T{100}\C{ $n$ should be less than this }\par
\Y\B\8\#\&{include} \.{<stdio.h>}\6
\&{int} \|h[\\{maxn}];\C{ the table of degrees; $h_k=t_k-1$ }\6
\&{int} \|l[\\{maxn}][\\{maxn}];\C{ the links (right to left) }\6
\&{int} \\{count};\7
\\{main}(\&{int} \\{argc}${},\39{}$\&{char} ${}{*}\\{argv}[\,]){}$\1\1\2\2\6
${}\{{}$\1\6
\&{register} \&{int} \|j${},{}$ \|k${},{}$ \|n${},{}$ \|r${},{}$ \|x${},{}$ %
\|y;\7
\X2:Read the \PB{\|t}'s from the command line\X;\6
\&{for} ${}(\|j\K\T{1};{}$ ${}\|j<\|n;{}$ ${}\|j\PP){}$\1\5
${}\|l[\|j][\|h[\|j]]\K\|j+\T{1};{}$\2\6
\&{while} (\T{1})\5
${}\{{}$\1\6
\X3:Visit the current tree\X;\6
\&{for} ${}(\|j\K\T{1},\39\|x\K\|l[\T{1}][\T{0}];{}$ ${}\|x\E\|j+\T{1};{}$ ${}%
\|j\K\|x,\39\|x\K\|l[\|j][\T{0}]){}$\1\5
${}\|l[\|j][\T{0}]\K\T{0},\39\|l[\|j][\|h[\|j]]\K\|x;{}$\2\6
\&{if} ${}(\|j\E\|n){}$\1\5
\&{break};\2\6
\&{for} ${}(\|r\K\T{1};{}$ ${}\|l[\|j][\|r]\E\T{0};{}$ ${}\|r\PP){}$\1\5
;\2\6
\&{for} ${}(\|k\K\T{0},\39\|y\K\|l[\|j][\|r];{}$ \|l[\|y][\T{0}]; ${}\|k\K\|y,%
\39\|y\K\|l[\|y][\T{0}]){}$\1\5
;\2\6
\&{if} (\|k)\1\5
${}\|l[\|k][\T{0}]\K\T{0}{}$;\5
\2\&{else}\1\5
${}\|l[\|j][\|r]\K\T{0};{}$\2\6
${}\|l[\|j][\T{0}]\K\T{0},\39\|l[\|j][\|r-\T{1}]\K\|y,\39\|l[\|y][\T{0}]\K%
\|x;{}$\6
\4${}\}{}$\2\6
\4${}\}{}$\2\par
\fi

\M[44 linked-trees.w]{2}\B\D$\\{abort}(\|m)$ \6
${}\{{}$\5
\1${}\\{fprintf}(\\{stderr},\39\.{"\%s!\\n"},\39\|m);{}$\6
\\{exit}(\|j);\5
${}\}{}$\2\par
\Y\B\4\X2:Read the \PB{\|t}'s from the command line\X${}\E{}$\6
$\|n\K\\{argc}-\T{1};{}$\6
\&{if} ${}(\|n\E\T{0}){}$\5
${}\{{}$\1\6
${}\\{fprintf}(\\{stderr},\39\.{"Usage:\ \%s\ t1\ t2\ ...}\)\.{\ tn\\n"},\39%
\\{argv}[\T{0}]);{}$\6
\\{exit}(\T{0});\6
\4${}\}{}$\2\6
\&{if} ${}(\|n\G\\{maxn}){}$\1\5
\\{abort}(\.{"I\ can't\ handle\ that}\)\.{\ many\ degrees"});\2\6
\&{for} ${}(\|j\K\T{1};{}$ ${}\|j\Z\|n;{}$ ${}\|j\PP){}$\5
${}\{{}$\1\6
\&{if} ${}(\\{sscanf}(\\{argv}[\|j],\39\.{"\%d"},\39{\AND}\|h[\|j])\I\T{1}){}$%
\1\5
\\{abort}(\.{"unreadable\ degree"});\2\6
${}\|h[\|j]\MM;{}$\6
\&{if} ${}(\|h[\|j]<\T{0}){}$\1\5
\\{abort}(\.{"Each\ degree\ must\ be}\)\.{\ positive"});\2\6
\&{if} ${}(\|h[\|j]\G\\{maxn}){}$\1\5
\\{abort}(\.{"Degree\ is\ too\ large}\)\.{"});\2\6
\4${}\}{}$\2\par
\U1.\fi

\M[61 linked-trees.w]{3}For each tree, we print out the array of links, with
link~0 last.

\Y\B\4\X3:Visit the current tree\X${}\E{}$\6
$\\{count}\PP;{}$\6
${}\\{printf}(\.{"\%d:"},\39\\{count});{}$\6
\&{for} ${}(\|j\K\T{1};{}$ ${}\|j\Z\|n;{}$ ${}\|j\PP){}$\5
${}\{{}$\1\6
\&{for} ${}(\|k\K\|h[\|j];{}$ ${}\|k\G\T{0};{}$ ${}\|k\MM){}$\1\5
${}\\{printf}(\.{"\ \%d"},\39\|l[\|j][\|k]);{}$\2\6
\&{if} ${}(\|j<\|n){}$\1\5
\\{printf}(\.{";"});\2\6
\4${}\}{}$\2\6
\\{printf}(\.{"\\n"});\par
\U1.\fi

\N[72 linked-trees.w]{1}{4}Index.

\fi


\inx
\fin
\con
