\input cwebmac
\datethis

\N{1}{1}Intro. This program constructs segments of the ``sieve of
Erasthosthenes,''
and outputs the largest prime gaps that it finds. More precisely, it
works with sets of prime numbers between $s_i$ and $s_{i+1}=s_i+\delta$,
represented as an array of bits, and it examines these arrays for
$t$ consecutive intervals beginning with $s_i$ for $i=0$, 1, \dots~$t-1$.
Thus it scans all primes between $s_0$ and $s_t$.

Let $p_k$ be the $k$th prime number. The sieve of Eratotosthenes determines
all primes $\le N$ by starting with the set $\{2,3,\ldots,N\}$ and striking
out the nonprimes: After we know $p_1$ through $p_{k-1}$, the next remaining
element is $p_k$, and we strike out the numbers $p_k^2$, $p_k(p_k+1)$,
$p_k(p_k+2)$, etc. The sieve is complete when we've found the
first prime with $p_k^2>N$.

In this program it's convenient to deal with the nonprimes instead of the
primes, and to assume that we already know all of the ``small'' primes~$p_k$
for which $p_k^2\le s_t$.
And of course we might as well restrict consideration to odd numbers.
Thus, we'll represent the integers between $s_i$ and $s_{i+1}$ by
$\delta/2$ bits; these bits will appear in $\delta/128$ 64-bit numbers
\PB{\\{sieve}[\|j]}, where
$$\PB{\\{sieve}[\|j]}=\sum_{n=s_i+128j}^{s_i+128(j+1)} 2^{(n-s_i-128j-1)/2}
\,\hbox{$\bigl[n$ is an odd multiple of some odd prime
$\le\sqrt{\mathstrut s_{i+1}}\,\bigr]$}.$$

We choose the segment size $\delta$ to be a multiple of 128.
We also assume that $s_0$ is even, and $s_0\ge\sqrt\delta$. It follows
that $s_i$ is even for all~$i$, and that $(s_i+1)^2=s_i^2+s_i+s_{i+1}-\delta
\ge s_i+s_{i+1}>s_{i+1}$. Consequently we have
$$\PB{\\{sieve}[\|j]}=\sum_{n=s_i+128j}^{s_i+128(j+1)} 2^{(n-s_i-128j-1)/2}
\,\hbox{$\bigl[n$ is odd and not prime$\bigr]$},$$
because $n$ appears if and only if it is divisible by some prime~$p$
where $p\le\sqrt{\mathstrut s_{i+1}}<s_i+1\le n$.

\fi

\M{2}The sieve size $\delta$ is specified at compile time,
but $s_0$ and $t$ are specified on the command line when this program
is run. There also are two additional command-line parameters,
which name the input and output files.

The input file should contain
all prime numbers $p_1$, $p_2$, \dots, up to the first prime such
that $p_k^2>s_t$; it may also contain further primes, which are ignored.
It is a binary file, with each prime given as an \PB{\&{unsigned} \&{int}}.
(There are 203,280,221 primes less than $2^{32}$, the largest of which
is $2^{32}-5$. Thus I'm implicitly assuming that $s_t<(2^{32}-5)^2
\approx 1.8\times10^{19}$.)

The output file is a short text file that reports large gaps.
Whenever the program discovers consecutive primes for which the gap
$p_{k+1}-p_k$ is greater than or equal to all previously seen gaps,
this gap is output (unless it is smaller than 256).
The smallest and largest
primes between $s_0$ and $s_t$ are also output, so that we can keep
track of gaps between primes that are
found by different instances of this program.

\Y\B\4\D$\\{del}$ \5
\T{1000000000\$L\$L}\C{ the segment size $\delta$, a multiple of 128 }\par
\B\4\D$\\{kmax}$ \5
\T{51000000}\C{ an index such that $p_{kmax}^2>s_t$ }\par
\Y\B\8\#\&{include} \.{<stdio.h>}\6
\8\#\&{include} \.{<stdlib.h>}\6
\8\#\&{include} \.{<time.h>}\6
\&{FILE} ${}{*}\\{infile},{}$ ${}{*}\\{outfile};{}$\6
\&{unsigned} \&{int} \\{prime}[\\{kmax}];\C{ $\PB{\\{prime}}[k]=p_{k+1}$ }\6
\&{unsigned} \&{int} \\{start}[\\{kmax}];\C{ indices for initializing a segment
}\6
\&{unsigned} \&{long} \&{long} ${}\\{sieve}[\T{2}+\\{del}/\T{128}];{}$\6
\&{unsigned} \&{long} \&{long} \\{s0};\C{ beginning of the first segment }\6
\&{int} \\{tt};\C{ number of segments }\6
\&{unsigned} \&{long} \&{long} \\{st};\C{ ending of the last segment }\6
\&{unsigned} \&{long} \&{long} \\{lastprime};\C{ largest prime so far, if any }%
\6
\&{int} \\{bestgap}${}\K\T{256}{}$;\C{ lower bound for gap reporting }\6
\&{unsigned} \&{long} \&{long} \\{sv}[\T{11}];\C{ bit patterns for the smallest
primes }\6
\&{int} \\{rem}[\T{11}];\C{ shift amounts for the smallest primes }\6
\&{char} \\{nu}[\T{\^10000}];\C{ table for counting bits }\6
\&{int} \\{timer};\7
\\{main}(\&{int} \\{argc}${},\39{}$\&{char} ${}{*}\\{argv}[\,]){}$\1\1\2\2\6
${}\{{}$\1\6
\&{register} \|j${},{}$ \|k;\6
\&{unsigned} \&{long} \&{long} \|x${},{}$ \|y${},{}$ \|z${},{}$ \|s${},{}$ %
\\{ss};\6
\&{int} \|d${},{}$ \\{ii}${},{}$ \\{kk};\7
${}\\{timer}\K\\{time}(\T{0});{}$\6
\X17:Initialize the bit-counting table\X;\6
\X3:Process the command line and input the primes\X;\6
\X7:Get ready for the first segment\X;\6
\&{for} ${}(\\{ii}\K\T{0};{}$ ${}\\{ii}<\\{tt};{}$ ${}\\{ii}\PP){}$\1\5
\X8:Do segment \PB{\\{ii}}\X;\2\6
\X19:Report the final prime\X;\6
${}\\{printf}(\.{"(Finished;\ the\ last}\)\.{\ segment\ took\ \%d\ sec}\)\.{.)%
\\n"},\39\\{time}(\T{0})-\\{timer});{}$\6
\4${}\}{}$\2\par
\fi

\M{3}\B\X3:Process the command line and input the primes\X${}\E{}$\6
\&{if} ${}(\\{argc}\I\T{5}\V\\{sscanf}(\\{argv}[\T{1}],\39\.{"\%llu"},\39{\AND}%
\\{s0})\I\T{1}\V\\{sscanf}(\\{argv}[\T{2}],\39\.{"\%d"},\39{\AND}\\{tt})\I%
\T{1}){}$\5
${}\{{}$\1\6
${}\\{fprintf}(\\{stderr},\39\.{"Usage:\ \%s\ s[0]\ t\ in}\)\.{putfile\
outputfile\\n}\)\.{"},\39\\{argv}[\T{0}]);{}$\6
${}\\{exit}({-}\T{1});{}$\6
\4${}\}{}$\2\6
${}\\{infile}\K\\{fopen}(\\{argv}[\T{3}],\39\.{"rb"});{}$\6
\&{if} ${}(\R\\{infile}){}$\5
${}\{{}$\1\6
${}\\{fprintf}(\\{stderr},\39\.{"I\ can't\ open\ \%s\ for}\)\.{\ binary\ input!%
\\n"},\39\\{argv}[\T{3}]);{}$\6
${}\\{exit}({-}\T{2});{}$\6
\4${}\}{}$\2\6
${}\\{outfile}\K\\{fopen}(\\{argv}[\T{4}],\39\.{"w"});{}$\6
\&{if} ${}(\R\\{outfile}){}$\5
${}\{{}$\1\6
${}\\{fprintf}(\\{stderr},\39\.{"I\ can't\ open\ \%s\ for}\)\.{\ text\ output!%
\\n"},\39\\{argv}[\T{4}]);{}$\6
${}\\{exit}({-}\T{3});{}$\6
\4${}\}{}$\2\6
${}\\{st}\K\\{s0}+\\{tt}*\\{del};{}$\6
\&{if} ${}(\\{del}\MOD\T{128}){}$\5
${}\{{}$\1\6
${}\\{fprintf}(\\{stderr},\39\.{"Oops:\ The\ sieve\ siz}\)\.{e\ \%d\ isn't\ a\
multipl}\)\.{e\ of\ 128!\\n"},\39\\{del});{}$\6
${}\\{exit}({-}\T{4});{}$\6
\4${}\}{}$\2\6
\&{if} ${}(\\{s0}\AND\T{1}){}$\5
${}\{{}$\1\6
${}\\{fprintf}(\\{stderr},\39\.{"The\ starting\ point\ }\)\.{\%llu\ isn't\
even!\\n"},\39\\{s0});{}$\6
${}\\{exit}({-}\T{5});{}$\6
\4${}\}{}$\2\6
\&{if} ${}(\\{s0}*\\{s0}<\\{del}){}$\5
${}\{{}$\1\6
${}\\{fprintf}(\\{stderr},\39\.{"The\ starting\ point\ }\)\.{\%llu\ is\ less\
than\ sq}\)\.{rt(\%llu)!\\n"},\39\\{s0},\39\\{del});{}$\6
${}\\{exit}({-}\T{6});{}$\6
\4${}\}{}$\2\6
\X4:Input the primes\X;\6
${}\\{printf}(\.{"Sieving\ between\ s[0}\)\.{]=\%llu\ and\ s[t]=\%llu}\)\.{:%
\\n"},\39\\{s0},\39\\{st}){}$;\par
\U2.\fi

\M{4}\B\X4:Input the primes\X${}\E{}$\6
\&{for} ${}(\|k\K\T{0};{}$  ; ${}\|k\PP){}$\5
${}\{{}$\1\6
\&{if} ${}(\|k\G\\{kmax}){}$\5
${}\{{}$\1\6
${}\\{fprintf}(\\{stderr},\39\.{"Oops:\ Please\ recomp}\)\.{ile\ me\ with\
kmax>\%d!}\)\.{\\n"},\39\\{kmax});{}$\6
${}\\{exit}({-}\T{7});{}$\6
\4${}\}{}$\2\6
\&{if} ${}(\\{fread}({\AND}\\{prime}[\|k],\39{}$\&{sizeof}(\&{unsigned} %
\&{int})${},\39\T{1},\39\\{infile})\I\T{1}){}$\5
${}\{{}$\1\6
${}\\{fprintf}(\\{stderr},\39\.{"The\ input\ file\ ende}\)\.{d\ prematurely\ (%
\%d\^2<}\)\.{\%llu)!\\n"},\39\|k\?\\{prime}[\|k-\T{1}]:\T{0},\39\\{st});{}$\6
${}\\{exit}({-}\T{8});{}$\6
\4${}\}{}$\2\6
\&{if} ${}(\|k\E\T{0}\W\\{prime}[\T{0}]\I\T{2}){}$\5
${}\{{}$\1\6
${}\\{fprintf}(\\{stderr},\39\.{"The\ input\ file\ begi}\)\.{ns\ with\ \%d,\
not\ 2!\\n}\)\.{"},\39\\{prime}[\T{0}]);{}$\6
${}\\{exit}({-}\T{9});{}$\6
\4${}\}{}$\2\6
\&{else} \&{if} ${}(\|k>\T{0}\W\\{prime}[\|k]\Z\\{prime}[\|k-\T{1}]){}$\5
${}\{{}$\1\6
${}\\{fprintf}(\\{stderr},\39\.{"The\ input\ file\ has\ }\)\.{consecutive\
entries\ }\)\.{\%d,\%d!\\n"},\39\\{prime}[\|k-\T{1}],\39\\{prime}[\|k]);{}$\6
${}\\{exit}({-}\T{10});{}$\6
\4${}\}{}$\2\6
\&{if} (((\&{unsigned} \&{long} \&{long}) \\{prime}[\|k])${}*\\{prime}[\|k]>%
\\{st}){}$\1\5
\&{break};\2\6
\4${}\}{}$\2\6
${}\\{printf}(\.{"\%d\ primes\ successfu}\)\.{lly\ loaded\ from\ \%s\\n}\)%
\.{"},\39\|k,\39\\{argv}[\T{3}]){}$;\par
\U3.\fi

\N{1}{5}Sieving. Let's say that the prime $p_k$ is ``active'' if
$p_k^2<s_{i+1}$.
Variable \PB{\\{kk}} is the index of the first inactive prime.
The main task of sieving is to mark the multiples of all active
primes in the current segment.

For each active prime $p_k$, let $n_k$ be the smallest odd multiple of~$p_k$
that exceeds $s_i$. We let \PB{\\{start}[\|k]} be $(n_k-s_i-1)/2$, the bit
offset
of the first such multiple that needs to be marked.

At the beginning, we compute \PB{\\{start}[\|k]} by division. But we'll
be able to compute \PB{\\{start}[\|k]} for subsequent segments as a byproduct
of
sieving, without division; that's why we bother to keep \PB{\\{start}[\|k]} in
memory.

\Y\B\4\X5:Initialize the active primes\X${}\E{}$\6
\&{for} ${}(\|k\K\T{1};{}$ ((\&{unsigned} \&{long} \&{long}) \\{prime}[%
\|k])${}*\\{prime}[\|k]<\\{s0};{}$ ${}\|k\PP){}$\5
${}\{{}$\1\6
${}\|j\K\\{s0}\MOD\\{prime}[\|k];{}$\6
\&{if} ${}(\|j\AND\T{1}){}$\1\5
${}\\{start}[\|k]\K\\{prime}[\|k]-((\|j+\T{1})\GG\T{1});{}$\2\6
\&{else}\1\5
${}\\{start}[\|k]\K(\\{prime}[\|k]-\|j-\T{1})\GG\T{1};{}$\2\6
\4${}\}{}$\2\6
${}\\{kk}\K\|k;{}$\6
\X6:Initialize the tiny active primes\X;\par
\U7.\fi

\M{6}Primes less than 32 will appear at least twice in every octabyte of
the sieve. So we handle them in a slightly more efficient way, unless
they're initially inactive.

\Y\B\4\X6:Initialize the tiny active primes\X${}\E{}$\6
\&{for} ${}(\|k\K\T{1};{}$ ${}\\{prime}[\|k]<\T{32}\W\|k<\\{kk};{}$ ${}\|k%
\PP){}$\5
${}\{{}$\1\6
\&{for} ${}(\|x\K\T{0},\39\|y\K\T{1\$L\$L}\LL\\{start}[\|k];{}$ ${}\|x\I\|y;{}$
${}\|x\K\|y,\39\|y\MRL{{\OR}{\K}}\|y\LL\\{prime}[\|k]){}$\1\5
;\2\6
${}\\{sv}[\|k]\K\|x,\39\\{rem}[\|k]\K\T{64}\MOD\\{prime}[\|k];{}$\6
\4${}\}{}$\2\6
${}\|d\K\|k{}$;\C{ \PB{\|d} is the smallest nontiny prime }\par
\U5.\fi

\M{7}\B\X7:Get ready for the first segment\X${}\E{}$\6
\X5:Initialize the active primes\X;\6
${}\\{ss}\K\\{s0}{}$;\C{ base address of the next segment }\6
${}\\{sieve}[\T{1}+\\{del}/\T{128}]\K{-}\T{1}{}$;\C{ store a sentinel }\par
\U2.\fi

\M{8}\B\X8:Do segment \PB{\\{ii}}\X${}\E{}$\6
${}\{{}$\1\6
${}\|s\K\\{ss},\39\\{ss}\K\|s+\\{del}{}$;\C{ $s=s_i$, $\PB{\\{ss}}=s_{i+1}$ }\6
${}\|j\K\\{time}(\T{0});{}$\6
${}\\{printf}(\.{"Beginning\ segment\ \%}\)\.{llu\ (after\ \%d\ sec)\\n}\)%
\.{"},\39\|s,\39\|j-\\{timer});{}$\6
\\{fflush}(\\{stdout});\6
${}\\{timer}\K\|j;{}$\6
\X9:Initialize the sieve from the tiny primes\X;\6
\X10:Sieve in the previously active primes\X;\6
\X11:Sieve in the newly active primes\X;\6
\X12:Look for large gaps\X;\6
\4${}\}{}$\2\par
\U2.\fi

\M{9}\B\X9:Initialize the sieve from the tiny primes\X${}\E{}$\6
\&{for} ${}(\|j\K\T{0};{}$ ${}\|j<\\{del}/\T{128};{}$ ${}\|j\PP){}$\5
${}\{{}$\1\6
\&{for} ${}(\|z\K\T{0},\39\|k\K\T{1};{}$ ${}\|k<\|d;{}$ ${}\|k\PP){}$\5
${}\{{}$\1\6
${}\|z\MRL{{\OR}{\K}}\\{sv}[\|k];{}$\6
${}\\{sv}[\|k]\K(\\{sv}[\|k]\LL(\\{prime}[\|k]-\\{rem}[\|k]))\OR(\\{sv}[\|k]\GG%
\\{rem}[\|k]);{}$\6
\4${}\}{}$\2\6
${}\\{sieve}[\|j]\K\|z;{}$\6
\4${}\}{}$\2\par
\U8.\fi

\M{10}Now we want to set 1 bits for every odd multiple of \PB{\\{prime}[\|k]}
in the current segment, whenever \PB{\\{prime}[\|k]} is active.
The bit for the integer $s_i+2j+1$ is
\PB{$\T{1}\LL(\|j\AND\T{\^3f})$} in \PB{$\\{sieve}[\|j\GG\T{6}]$}, for $0\le j<%
\delta/2$.

\Y\B\4\X10:Sieve in the previously active primes\X${}\E{}$\6
\&{for} ${}(\|k\K\|d;{}$ ${}\|k<\\{kk};{}$ ${}\|k\PP){}$\5
${}\{{}$\1\6
\&{for} ${}(\|j\K\\{start}[\|k];{}$ ${}\|j<\\{del}/\T{2};{}$ ${}\|j\MRL{+{\K}}%
\\{prime}[\|k]){}$\1\5
${}\\{sieve}[\|j\GG\T{6}]\MRL{{\OR}{\K}}\T{1\$L\$L}\LL(\|j\AND\T{\^3f});{}$\2\6
${}\\{start}[\|k]\K\|j-\\{del}/\T{2};{}$\6
\4${}\}{}$\2\par
\U8.\fi

\M{11}\B\X11:Sieve in the newly active primes\X${}\E{}$\6
\&{while} (((\&{unsigned} \&{long} \&{long}) \\{prime}[\|k])${}*\\{prime}[\|k]<%
\\{ss}){}$\5
${}\{{}$\1\6
\&{for} ${}(\|j\K{}$(((\&{unsigned} \&{long} \&{long}) \\{prime}[\|k])${}*%
\\{prime}[\|k]-\|s-\T{1})\GG\T{1};{}$ ${}\|j<\\{del}/\T{2};{}$ ${}\|j\MRL{+{%
\K}}\\{prime}[\|k]){}$\1\5
${}\\{sieve}[\|j\GG\T{6}]\MRL{{\OR}{\K}}\T{1\$L\$L}\LL(\|j\AND\T{\^3f});{}$\2\6
${}\\{start}[\|k]\K\|j-\\{del}/\T{2};{}$\6
${}\|k\PP;{}$\6
\4${}\}{}$\2\6
${}\\{kk}\K\|k{}$;\par
\U8.\fi

\N{1}{12}Processing gaps.
If $p_{k+1}-p_k\ge256$, we're bound to find an octabyte of all 1s in the
sieve between the 0~for~$p_k$ and the 0~for~$p_{k+1}$. In such cases,
we check to see if this gap breaks or ties the current record.

Complications occur if the gap appears at the very beginning or end of
a segment, or if an entire segment is prime-free. I've tried to get the
logic correct, without slowing the program down. But if any
bugs are present in this code, I suppose they are due to a fallacy
in this aspect of my reasoning.

Two sentinels appear at the end of the sieve, in order to speed up
loop termination: \PB{$\\{sieve}[\\{del}/\T{128}]\K\T{0}$} and \PB{$\\{sieve}[%
\T{1}+\\{del}/\T{128}]\K{-}\T{1}$}.

\Y\B\4\X12:Look for large gaps\X${}\E{}$\6
$\|j\K\T{0};{}$\6
\X13:Identify the first prime in this segment, if necessary\X;\6
\&{while} (\T{1})\5
${}\{{}$\C{ at this point \PB{$\|j<\\{del}/\T{128}$} and \PB{$\\{sieve}[\|j]%
\I{-}\T{1}$} }\1\6
\&{for} ${}(\|j\PP;{}$ ${}\\{sieve}[\|j]\I{-}\T{1};{}$ ${}\|j\PP){}$\1\5
;\2\6
\&{if} ${}(\|j<\\{del}/\T{128}){}$\5
${}\{{}$\1\6
${}\|k\K\|j-\T{1};{}$\6
\&{for} ${}(\|j\PP;{}$ ${}\\{sieve}[\|j]\E{-}\T{1};{}$ ${}\|j\PP){}$\1\5
;\2\6
\&{if} ${}(\|j\E\\{del}/\T{128}){}$\1\5
\&{break};\2\6
\X14:Check for a potentially interesting gap\X;\6
\4${}\}{}$\5
\2\&{else}\5
${}\{{}$\C{ \PB{$\|j\K\T{1}+\\{del}/\T{128}$} and \PB{$\\{sieve}[\\{del}/%
\T{128}-\T{1}]\I{-}\T{1}$} }\1\6
${}\|k\K\\{del}/\T{128}-\T{1}{}$;\5
\&{break};\6
\4${}\}{}$\2\6
\4${}\}{}$\2\6
\X15:Set \PB{\\{lastprime}} to the largest prime in \PB{\\{sieve}[\|k]}\X;\6
\4\\{donewithseg}:\par
\U8.\fi

\M{13}We don't need to figure out the exact value of the first prime
greater than~\PB{\|s} unless the present segment begins with an octabyte
of all 1s, or the previous segment ends with such an octabyte,
or we're in the first segment.

But in any case we'll want to go immediately to \PB{\\{donewithseg}} if
the current segment is entirely prime-free. And we always want to
end this step with \PB{\|j} equal to the smallest index such that \PB{$%
\\{sieve}[\|j]\I{-}\T{1}$}.

\Y\B\4\X13:Identify the first prime in this segment, if necessary\X${}\E{}$\6
\&{if} ${}(\\{lastprime}\Z\|s-\T{128}\V\\{sieve}[\|j]\E{-}\T{1}){}$\5
${}\{{}$\1\6
\&{for} ( ; ${}\\{sieve}[\|j]\E{-}\T{1};{}$ ${}\|j\PP){}$\1\5
;\2\6
\&{if} ${}(\|j\E\\{del}/\T{128}){}$\1\5
\&{goto} \\{donewithseg};\2\6
${}\|y\K\CM\\{sieve}[\|j];{}$\6
${}\|y\K\|y\AND{-}\|y{}$;\C{ extract the rightmost 1 bit }\6
\X16:Change \PB{\|y} to its binary logarithm\X;\6
${}\|x\K\|s+(\|j\LL\T{7})+\|y+\|y+\T{1}{}$;\C{ this is the first prime of the
segment }\6
\&{if} (\\{lastprime})\1\5
\X18:Report a gap, if it's big enough\X\2\6
\&{else}\5
${}\{{}$\1\6
${}\|k\K\|x-\\{s0};{}$\6
${}\\{fprintf}(\\{outfile},\39\.{"The\ first\ prime\ is\ }\)\.{\%llu\ =\ s[0]+%
\%d\\n"},\39\|x,\39\|k);{}$\6
\\{fflush}(\\{outfile});\6
\4${}\}{}$\2\6
\4${}\}{}$\2\par
\U12.\fi

\M{14}When \PB{$\\{sieve}[\|k]\I{-}\T{1}$} and \PB{$\\{sieve}[\|j]\I{-}\T{1}$}
and everything between them
is \PB{${-}\T{1}$} (all ones), there's a gap of size~$g$ where
$128\vert j-k\vert-126\le g\le128\vert j-k\vert+126$.

\Y\B\4\X14:Check for a potentially interesting gap\X${}\E{}$\6
\&{if} ${}(((\|j-\|k)\LL\T{7})+\T{126}\G\\{bestgap}){}$\5
${}\{{}$\1\6
${}\|y\K\CM\\{sieve}[\|j];{}$\6
${}\|y\K\|y\AND{-}\|y{}$;\C{ extract the rightmost 1 bit }\6
\X16:Change \PB{\|y} to its binary logarithm\X;\6
${}\|x\K\|s+(\|j\LL\T{7})+\|y+\|y+\T{1}{}$;\C{ this is the first prime after
the gap }\6
\X15:Set \PB{\\{lastprime}} to the largest prime in \PB{\\{sieve}[\|k]}\X;\6
\X18:Report a gap, if it's big enough\X;\6
\4${}\}{}$\2\par
\U12.\fi

\M{15}\B\X15:Set \PB{\\{lastprime}} to the largest prime in \PB{\\{sieve}[\|k]}%
\X${}\E{}$\6
\&{for} ${}(\|y\K\CM\\{sieve}[\|k],\39\|z\K\|y\AND(\|y-\T{1});{}$ \|z; ${}\|y\K%
\|z,\39\|z\K\|y\AND(\|y-\T{1})){}$\1\5
;\2\6
\X16:Change \PB{\|y} to its binary logarithm\X;\6
${}\\{lastprime}\K\|s+(\|k\LL\T{7})+\|y+\|y+\T{1}{}$;\par
\Us12\ET14.\fi

\M{16}As far as I know, the following method is the fastest way to compute
binary logarithms on an Opteron computer (which is the machine
I'm targeting here).

\Y\B\4\X16:Change \PB{\|y} to its binary logarithm\X${}\E{}$\6
$\|y\MM;{}$\6
${}\|y\K\\{nu}[\|y\AND\T{\^ffff}]+\\{nu}[(\|y\GG\T{16})\AND\T{\^ffff}]+\\{nu}[(%
\|y\GG\T{32})\AND\T{\^ffff}]+\\{nu}[(\|y\GG\T{48})\AND\T{\^ffff}]{}$;\par
\Us13, 14\ETs15.\fi

\M{17}With a more extensive table, I could count the 1s in an arbitrary
binary word. But seventeen table entries are sufficient for present purposes.

\Y\B\4\X17:Initialize the bit-counting table\X${}\E{}$\6
\&{for} ${}(\|j\K\T{0};{}$ ${}\|j\Z\T{16};{}$ ${}\|j\PP){}$\1\5
${}\\{nu}[((\T{1}\LL\|j)-\T{1})]\K\|j{}$;\2\par
\U2.\fi

\M{18}\B\X18:Report a gap, if it's big enough\X${}\E{}$\6
${}\{{}$\1\6
\&{if} ${}(\|x-\\{lastprime}\G\\{bestgap}){}$\5
${}\{{}$\1\6
${}\\{bestgap}\K\|x-\\{lastprime};{}$\6
${}\\{fprintf}(\\{outfile},\39\.{"\%llu\ is\ followed\ by}\)\.{\ a\ gap\ of\
length\ \%d\\}\)\.{n"},\39\\{lastprime},\39\\{bestgap});{}$\6
\\{fflush}(\\{outfile});\6
\4${}\}{}$\2\6
\4${}\}{}$\2\par
\Us13\ET14.\fi

\M{19}\B\X19:Report the final prime\X${}\E{}$\6
\&{if} (\\{lastprime})\5
${}\{{}$\1\6
${}\|k\K\\{st}-\\{lastprime};{}$\6
${}\\{fprintf}(\\{outfile},\39\.{"The\ final\ prime\ is\ }\)\.{\%llu\ =\ s[t]-%
\%d.\\n"},\39\\{lastprime},\39\|k);{}$\6
\4${}\}{}$\5
\2\&{else}\1\5
${}\\{fprintf}(\\{outfile},\39\.{"No\ prime\ numbers\ ex}\)\.{ist\ between\
s[0]\ and}\)\.{\ s[t].\\n"}){}$;\2\par
\U2.\fi

\N{1}{20}Index.
\fi

\inx
\fin
\con
