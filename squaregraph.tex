\input cwebmac
\datethis
\def\sqprod{\setbox0=\hbox{\kern-.13em$\times$\kern-.13em}
     \dimen0=\ht0 \advance\dimen0 -.09em \ht0=\dimen0
     \dimen0=\dp0 \advance\dimen0 -.09em \dp0=\dimen0
     \mathbin{\vcenter{\hrule\kern-.4pt
       \hbox{\vrule\kern-.4pt\phantom{$\box0$}\kern-.4pt\vrule}\kern-.4pt
       \hrule}}}


\N{1}{1}Intro. This program enumerates biconnected squaregraphs of small order,
more or less by brute force.

Every such graph of perimeter $2p$ is definable by a set partition of
$2p$ elements into $p$ blocks of size~2. Equivalently, it's definable
by a restricted growth string of a very simple kind, namely
a permutation of the multiset $\{0,0,1,1,\ldots,p-1,p-1\}$ in which the first
appearance of $j>0$ is preceded by $j-1$. For example, the restricted
growth string $01023132$ corresponds to the set partition
$13\mid26\mid48\mid57$.

However, the set partition is considered to be {\it circular}, so that
it corresponds to taking $2p$ points around a circle and connecting them
by chords. Positions $(1,2,\ldots,8)$ around a circle are equivalent to
positions $(2,\ldots,8,1)$; so the set partition above is also equivalent
to $24\mid37\mid51\mid68$, which has the restricted growth string
$01210323$. And there are six others, because we consider $2p$
cyclic shifts.

Therefore it turns out that restricted growth strings are not the
best representations of the set partitions in this application. Instead
we use the cyclic distances or ``deltas'' to the next occurrence of each
symbol; for instance, the growth string $01023132$ corresponds to the delta
sequence $24642464$. Cyclic shifting of the growth sequence is complicated,
because the digits need to be renamed; the delta sequence, however,
simply shifts cyclically. For instance, the delta sequence of $01210323$
is $46424642$.

A squaregraph is represented by its {\it canonical
sequence}, which is the lexicographically smallest of the $2p$
associated delta sequences. In our example, the smallest
turns out to be $24642464$. (And only four of the eight possible shifts
actually lead to different results in this case, because the
graph has a nontrivial automorphism.)

Two additional restrictions must be satisfied. First, the restricted
growth sequence must not contain a subword of the form $abcabc$ (not
necessarily consecutive). Our example satisfies this restriction, because
the relevant subwords of $01023132$ are $010212$, $010313$, $002332$, $123132$,
and because the condition is preserved under cyclic shifting. When this
restriction is met, there's a unique way to subdivide the circle into regions
by connecting the chords. (The algorithm below demonstrates this fact
constructively.)

Each of the regions just mentioned will have two or more vertices,
where the paths between consecutive vertices are either segments of the
circle or segments of a chord. Furthermore, a region whose boundaries
are interior --- not segments of the enclosing circle --- will always have
four or more vertices. (Again, the algorithm below will prove this.)

The squaregraph that corresponds to a canonical string that meets
this conditions is defined to be
the dual of the region graph, namely the graph whose vertices are the
regions, with adjacency defined as sharing the same segment of a chord.
For example, $24642464$ turns out to be the graph of the
so-called ``straight tromino,'' also known as $P_4\sqprod K_2$.

A final constraint is that none of the regions may include more than one
circle segment. For example, set partitions like $0011$ and $01012323$ are
disallowed. This restriction makes the squaregraph biconnected,
because it is equivalent to saying that there are no
articulation points.

The reader who tries to draw the resulting graphs will soon understand
why the name ``squaregraphs'' is brilliant.
(Also the Wikipedia already has a nice example picture!)

One might argue whether or not the shortest possible sequence, $00$,
is a biconnected squaregraph. Is the ``bridge'' $K_2$ biconnected?
In fact, should the empty string perhaps also qualify?
Anyway this program starts out with $p=3$, because the smaller cases
are obvious. (For example, the only biconnected squaregraph with $p=2$
is the square $C_4$, which corresponds to the canonical sequence $2222$.)

For each squaregraph we give its canonical sequence, $\delta_0\delta_1\ldots
\delta_{2p-1}$; the number of interior vertices, $q$; and the number
of squares, $s$. The graph then has $2p+q$ vertices altogether, and
$p+2s$ edges. We also compute the automorphisms, which are ``dihedral'': cyclic
and/or reflected.  In verbose mode, the full graph is listed as well.

If a squaregraph is not isomorphic to its reflection,
we list it only once, by taking the minimum of the
two canonical sequences.

\fi

\M{2}OK, here we go.

One simplification is obvious: The first element $\delta_0$ of the delta
sequence can't be~1, because that would produce an articulation point.
And because the sequence is canonical, we must have $\delta_j\ge\delta_0$
for all $j$. In particular, $\delta_0\le p$; and in fact $\delta_0=p$
is impossible, when $p>2$, because of the $abcabc$ condition.

When reporting totals, we consider both ``unlabeled'' delta sequences
(reduced by symmetries to canonical form) and ``labeled'' ones
(not reduced by symmetries). For example, $24642464$ corresponds to
three different labeled solutions, namely $46424642$ and $64246424$
in addition to itself.

The labeling condition essentially corresponds to assigning a direction
to one edge on the periphery of the squarefree graph. That edge and
that direction tell us what the delta sequence is; the delta sequence
determines all other vertices and edges uniquely, in any squaregraph.

\Y\B\4\D$\\{maxp}$ \5
\T{10}\C{ upper bound on $p$ }\par
\B\4\D$\\{verbose}$ \5
$(\\{argc}>\T{1}{}$)\C{ a command-line parameter triggers verbosity }\par
\Y\B\8\#\&{include} \.{<stdio.h>}\6
\8\#\&{include} \.{<stdlib.h>}\6
\X3:Global variables\X;\7
\&{int} \\{pp};\C{ the current perimeter }\6
\&{int} \\{ptot}${},{}$ \\{pltot};\C{ how many found with the current perimeter
}\6
\&{int} ${}\\{vtot}[\\{maxp}*\\{maxp}],{}$ ${}\\{vltot}[\\{maxp}*\\{maxp}],{}$
${}\\{stot}[\\{maxp}*\\{maxp}],{}$ ${}\\{sltot}[\\{maxp}*\\{maxp}]{}$;\C{ how
many graphs found with a given number of vertices or squares }\7
\X6:Subroutines\X;\7
\\{main}(\&{int} \\{argc})\1\1\2\2\6
${}\{{}$\1\6
\&{register} \&{int} \|j${},{}$ \|k${},{}$ \|l${},{}$ \|p${},{}$ \|q${},{}$ %
\|s;\6
\&{int} \\{rfl}${},{}$ \\{rot};\7
\&{for} ${}(\|p\K\T{3};{}$ ${}\|p\Z\\{maxp};{}$ ${}\|p\PP){}$\5
${}\{{}$\1\6
${}\\{pp}\K\|p+\|p;{}$\6
${}\\{ptot}\K\\{pltot}\K\T{0};{}$\6
\&{for} ${}(\|j\K\T{2};{}$ ${}\|j<\|p;{}$ ${}\|j\PP){}$\5
${}\{{}$\1\6
\X4:Initialize the delta sequence, with $\delta_0=j$\X;\6
\X5:Generate all answers for $(p,j)$\X;\6
\4${}\}{}$\2\6
${}\\{printf}(\.{"\%d\ (\%d\ labeled)\ wit}\)\.{h\ perimeter\ \%d;\ "},\39%
\\{ptot},\39\\{pltot},\39\\{pp});{}$\6
${}\\{printf}(\.{"\%d\ (\%d\ labeled)\ wit}\)\.{h\ \%d\ vertices;\ "},\39%
\\{vtot}[\\{pp}],\39\\{vltot}[\\{pp}],\39\\{pp});{}$\6
${}\\{printf}(\.{"\%d\ (\%d\ labeled)\ wit}\)\.{h\ \%d\ vertices;\ "},\39%
\\{vtot}[\\{pp}+\T{1}],\39\\{vltot}[\\{pp}+\T{1}],\39\\{pp}+\T{1});{}$\6
${}\\{printf}(\.{"\%d\ (\%d\ labeled)\ wit}\)\.{h\ \%d\ squares.\\n"},\39%
\\{stot}[\|p-\T{1}],\39\\{sltot}[\|p-\T{1}],\39\|p-\T{1});{}$\6
\4${}\}{}$\2\6
\4${}\}{}$\2\par
\fi

\N{1}{3}Generating the deltas.
The delta sequence is stored in array \PB{\\{del}}, and another array \PB{%
\\{occ}}
records the cells of the implicit restricted growth string that are
currently known (``occupied''). If the first occurrence of $j$
in the growth string is in cell $i_j$, we set $\PB{\\{back}}[i_j]=i_{j-1}$
for $0<j<p$.

\Y\B\4\X3:Global variables\X${}\E{}$\6
\&{int} ${}\\{del}[\T{4}*\\{maxp}];{}$\6
\&{int} ${}\\{occ}[\T{3}*\\{maxp}+\T{1}];{}$\6
\&{int} ${}\\{back}[\T{2}*\\{maxp}]{}$;\par
\A12.
\U2.\fi

\M{4}\B\X4:Initialize the delta sequence, with $\delta_0=j$\X${}\E{}$\6
$\\{del}[\T{0}]\K\|j,\39\\{del}[\|j]\K\\{pp}-\|j;{}$\6
${}\\{occ}[\T{0}]\K\\{occ}[\|j]\K\T{1};{}$\6
${}\|k\K\T{1},\39\|s\K\T{0}{}$;\par
\U2.\fi

\M{5}This program is one of those backtrack jobs
where I still like to use \PB{\&{goto}} statements,
forty years after the ``structured programming revolution.''

The value of \PB{\|s} points to the most recent entry that we've entered
into the \PB{\\{del}} table. It's essentially a stack pointer.

\Y\B\4\X5:Generate all answers for $(p,j)$\X${}\E{}$\6
\4\\{advance}:\5
\&{while} (\\{occ}[\|k])\1\5
${}\|k\PP;{}$\2\6
\&{if} ${}(\|k\E\\{pp}){}$\1\5
\X8:Print the current \PB{\\{del}} table if it's a solution, then \PB{\&{goto} %
\\{next}}\X;\2\6
${}\\{occ}[\|k]\K\T{1},\39\\{back}[\|k]\K\|s;{}$\6
${}\|l\K\|k+\|j{}$;\C{ the first conceivable way to place the mate of cell \PB{%
\|k} }\6
\4\\{nextslot}:\5
\&{while} (\\{occ}[\|l])\1\5
${}\|l\PP;{}$\2\6
\&{if} ${}(\|l\G\\{pp}){}$\1\5
\&{goto} \\{backtrack};\C{ no more places to match cell \PB{\|k} }\2\6
\&{if} ${}(\\{abcabc}(\|k,\39\|l)){}$\5
${}\{{}$\1\6
${}\|l\PP{}$;\5
\&{goto} \\{nextslot};\6
\4${}\}{}$\2\6
${}\\{del}[\|k]\K\|l-\|k,\39\\{del}[\|l]\K\\{pp}-\|l+\|k,\39\\{occ}[\|l]\K%
\T{1};{}$\6
${}\|s\K\|k\PP;{}$\6
\&{goto} \\{advance};\6
\4\\{next}:\5
${}\\{occ}[\|s+\\{del}[\|s]]\K\T{0};{}$\6
${}\|k\K\|s;{}$\6
\4\\{backtrack}:\5
${}\\{occ}[\|k]\K\T{0};{}$\6
${}\|k\K\\{back}[\|k];{}$\6
\&{if} (\|k)\5
${}\{{}$\1\6
${}\|l\K\|k+\\{del}[\|k],\39\\{occ}[\|l]\K\T{0},\39\|l\PP;{}$\6
\&{goto} \\{nextslot};\6
\4${}\}{}$\2\6
${}\\{occ}[\|j]\K\T{0}{}$;\par
\U2.\fi

\M{6}\B\X6:Subroutines\X${}\E{}$\6
\&{int} \\{abcabc}(\&{int} \|k${},\39{}$\&{int} \|l)\1\1\2\2\6
${}\{{}$\1\6
\&{register} \&{int} \|i${},{}$ \|j;\7
\&{for} ${}(\|j\K\\{back}[\|k];{}$ \|j; ${}\|j\K\\{back}[\|j]){}$\5
${}\{{}$\1\6
\&{if} ${}(\|l<\|j+\\{del}[\|j]){}$\1\5
\&{continue};\2\6
\&{for} ${}(\|i\K\\{back}[\|j];{}$  ; ${}\|i\K\\{back}[\|i]){}$\5
${}\{{}$\1\6
\&{if} ${}(\|j+\\{del}[\|j]<\|i+\\{del}[\|i]){}$\1\5
\&{goto} \.{OK};\2\6
\&{if} ${}(\|i+\\{del}[\|i]<\|k){}$\1\5
\&{goto} \.{OK};\2\6
\&{return} \T{1};\C{ $abcabc$ failure: $i<j<k<i+\delta_i<j+\delta_j<l$ }\6
\4\.{OK}:\5
\&{if} ${}(\|i\E\T{0}){}$\1\5
\&{break};\2\6
\4${}\}{}$\2\6
\4${}\}{}$\2\6
\&{return} \T{0};\C{ no problem }\6
\4${}\}{}$\2\par
\As13, 15, 18, 19\ETs20.
\U2.\fi

\N{1}{7}Testing canonicity. We want to reject a delta sequence that isn't
canonical, namely when it (or its reflection) has a cyclic shift
that's smaller. While we're doing this we can also determine any
automorphisms (symmetries) that are present.

When a sequence is canonical, we set \PB{$\\{rfl}\K\T{1}$} if it has reflection
symmetry,
and \PB{$\\{rot}\K\|q$} if it has \PB{\|q}-fold rotation symmetry. (The latter
means
that shifting by $2p/q$ yields the same result, where $q$ is as large
as possible.)

The reflection of a delta sequence $\delta_0\delta_1\ldots\delta_{2p-1}$ is
$(2p-\delta_{2p-1})\ldots(2p-\delta_1)(2p-\delta_0)$. The delta sequence
$24642464$, for example, has reflection $42464246$.
Hence \PB{$\\{rfl}\K\T{1}$} and \PB{$\\{rot}\K\T{2}$}, corresponding to the
four symmetries
of the straight tromino.

Incidentally, simply connected polyominoes are always squaregraphs.
The L-tromino corresponds to delta sequence $23635256$; it has
two symmetries, exhibited by \PB{$\\{rfl}\K\T{1}$} and \PB{$\\{rot}\K\T{1}$}.

\fi

\M{8}\B\X8:Print the current \PB{\\{del}} table if it's a solution, then \PB{%
\&{goto} \\{next}}\X${}\E{}$\6
${}\{{}$\1\6
\X9:Determine the automorphisms, but \PB{\&{goto} \\{next}} if \PB{\\{del}}
isn't canonical\X;\6
\X16:Construct the graph corresponding to the chords of \PB{\\{del}}\X;\6
\X22:Mark the regions, but \PB{\&{goto} \\{next}} if there's an articulation
problem\X;\6
\X23:Print \PB{\\{del}} and its characteristics\X;\6
\X25:Update the totals\X;\6
\&{goto} \\{next};\6
\4${}\}{}$\2\par
\U5.\fi

\M{9}\B\X9:Determine the automorphisms, but \PB{\&{goto} \\{next}} if \PB{%
\\{del}} isn't canonical\X${}\E{}$\6
\&{for} ${}(\|k\K\T{0};{}$ ${}\|k<\\{pp};{}$ ${}\|k\PP){}$\1\5
${}\\{del}[\\{pp}+\|k]\K\\{del}[\|k];{}$\2\6
\&{for} ${}(\|k\K\T{1};{}$ ${}\|k<\\{pp};{}$ ${}\|k\PP){}$\5
${}\{{}$\C{ try cyclic shifting by \PB{\|k} }\1\6
\&{for} ${}(\|q\K\|k;{}$ ${}\|q<\|k+\\{pp};{}$ ${}\|q\PP){}$\1\6
\&{if} ${}(\\{del}[\|q]\I\\{del}[\|q-\|k]){}$\1\5
\&{break};\2\2\6
\&{if} ${}(\\{del}[\|q]<\\{del}[\|q-\|k]){}$\1\5
\&{goto} \\{next};\C{ not canonical }\2\6
\&{if} ${}(\|q\E\|k+\\{pp}){}$\1\5
\&{break};\C{ match found }\2\6
\4${}\}{}$\2\6
${}\\{rot}\K\\{pp}/\|k{}$;\C{ \PB{\\{pp}} is a multiple of \PB{\|k} }\6
\X10:Check the reflected delta sequence for canonicity and possible identity\X;%
\par
\U8.\fi

\M{10}The reflected sequence is unchanged by a $k$-shift if and only if the
unreflected sequence is.

\Y\B\4\D$\\{rdel}(\|x)$ \5
$\\{pp}-\\{del}[\\{pp}+\\{pp}-\|x{}$]\par
\Y\B\4\X10:Check the reflected delta sequence for canonicity and possible
identity\X${}\E{}$\6
$\\{rfl}\K\T{0};{}$\6
\&{for} ${}(\|k\K\T{1};{}$ ${}\|k\Z\\{pp};{}$ ${}\|k\PP){}$\5
${}\{{}$\C{ try cyclic shifting by \PB{\|k} }\1\6
\&{for} ${}(\|q\K\|k;{}$ ${}\|q<\|k+\\{pp};{}$ ${}\|q\PP){}$\1\6
\&{if} ${}(\\{rdel}(\|q)\I\\{del}[\|q-\|k]){}$\1\5
\&{break};\2\2\6
\&{if} ${}(\|q\E\|k+\\{pp}){}$\5
${}\{{}$\1\6
${}\\{rfl}\K\T{1};{}$\6
\&{break};\6
\4${}\}{}$\2\6
\&{if} ${}(\\{rdel}(\|q)<\\{del}[\|q-\|k]){}$\1\5
\&{goto} \\{next};\C{ not canonical }\2\6
\4${}\}{}$\2\par
\U9.\fi

\N{1}{11}Constructing the graph. Now comes the fun part, where we actually
build the squaregraph vertices and edges, by constructing the regions
defined by the chords.

The data structure used here is undoubtedly overkill, but this program
was written with an intent to favor transparency over trickery.

Each region is represented as a cyclic sequence of edges, with
\PB{$\|e[\|j].\\{succ}$} the index of the successor to edge~$j$. There are two
kinds of edges, external and internal. External edge~$j$, for
$1\le j\le2p$, simply corresponds to a segment of the enclosing circle,
from point $j-1$ to point~$j$ (modulo~$2p$). An internal edge~$j$, on the
other hand, can be distinguished from the external edges because $j>2p$;
it represents a segment of the chord that runs between points \PB{$\|e[\|j].%
\\{from}$}
and \PB{$\|e[\|j].\\{to}$}. It also has a ``mate,'' $e[j\pm1]$, which runs the
other way on the same chord segment and belongs to an adjacent region.

After the whole graph has been constructed we will mark each edge
with the number of the region to which it belongs.

\fi

\M{12}\B\X3:Global variables\X${}\mathrel+\E{}$\6
\&{struct} ${}\{{}$\1\6
\&{int} \\{succ};\C{ index of the next edge in this region }\6
\&{int} \\{from}${},{}$ \\{to};\C{ points that define the chord, if internal }\6
\&{int} \\{reg};\C{ region number }\2\6
${}\}{}$ ${}\|e[\T{2}*\\{maxp}*\\{maxp}];{}$\6
\&{int} \\{eptr};\C{ address of first unused entry in \PB{\|e} }\par
\fi

\M{13}New edges are allocated in mated pairs.

The \PB{\|e} array, with \PB{$\T{2}*\\{maxp}*\\{maxp}$} entries, should have
plenty of
room for all the edges we need. But we check it, just to be sure.

\Y\B\4\X6:Subroutines\X${}\mathrel+\E{}$\6
\&{int} \\{newedge}(\&{int} \|s${},\39{}$\&{int} \|t)\1\1\2\2\6
${}\{{}$\1\6
${}\|e[\\{eptr}].\\{from}\K\|e[\\{eptr}+\T{1}].\\{to}\K\|s;{}$\6
${}\|e[\\{eptr}].\\{to}\K\|e[\\{eptr}+\T{1}].\\{from}\K\|t;{}$\6
${}\|e[\\{eptr}].\\{reg}\K\|e[\\{eptr}+\T{1}].\\{reg}\K\T{0};{}$\6
${}\\{eptr}\MRL{+{\K}}\T{2};{}$\6
\&{if} ${}(\\{eptr}\G\T{2}*\\{maxp}*\\{maxp}){}$\5
${}\{{}$\1\6
${}\\{fprint}(\\{stderr},\39\.{"Memory\ overflow!\\n"});{}$\6
${}\\{exit}({-}\T{2});{}$\6
\4${}\}{}$\2\6
\&{return} \\{eptr}${}-\T{2};{}$\6
\4${}\}{}$\2\par
\fi

\M{14}For convenience we assume that \PB{\\{eptr}} is always even, so that the
mate
of edge~$k$ is obtained by simply complementing the units bit of~$k$.

\Y\B\4\D$\\{mate}(\|k)$ \5
$((\|k)\XOR\T{1}{}$)\par
\fi

\M{15}We could have made the region lists doubly linked, but they tend to
be short. Therefore it probably doesn't hurt to search sequentially
for the predecessor of an edge.

\Y\B\4\X6:Subroutines\X${}\mathrel+\E{}$\6
\&{int} \\{pred}(\&{int} \|k)\1\1\2\2\6
${}\{{}$\1\6
\&{register} \&{int} \|j;\7
\&{for} ${}(\|j\K\|k;{}$ ${}\|e[\|j].\\{succ}\I\|k;{}$ ${}\|j\K\|e[\|j].%
\\{succ}){}$\1\5
;\2\6
\&{return} \|j;\6
\4${}\}{}$\2\par
\fi

\M{16}The task of building the graph consists of starting with just the
enclosing circle, then inserting the chords one by one.

\Y\B\4\X16:Construct the graph corresponding to the chords of \PB{\\{del}}\X${}%
\E{}$\6
\X17:Initialize for chord placement\X;\6
\&{for} ${}(\|k\K\|s;{}$  ; ${}\|k\K\\{back}[\|k]){}$\5
${}\{{}$\1\6
${}\\{newchord}(\|k,\39(\|k+\\{del}[\|k])\MOD\\{pp});{}$\6
\&{if} ${}(\|k\E\T{0}){}$\1\5
\&{break};\2\6
\4${}\}{}$\2\par
\U8.\fi

\M{17}\B\X17:Initialize for chord placement\X${}\E{}$\6
\&{for} ${}(\|k\K\T{1};{}$ ${}\|k<\\{pp};{}$ ${}\|k\PP){}$\1\5
${}\|e[\|k].\\{succ}\K\|k+\T{1},\39\|e[\|k].\\{reg}\K\T{0};{}$\2\6
${}\|e[\\{pp}].\\{succ}\K\T{1},\39\|e[\\{pp}].\\{reg}\K\T{0};{}$\6
${}\\{eptr}\K\\{pp}+\T{2}{}$;\C{ we want \PB{\\{eptr}} to be even, as mentioned
earlier }\6
${}\|e[\\{pp}+\T{1}].\\{reg}\K{-}\T{1}{}$;\C{ make the discarded edge unusable
}\par
\U16.\fi

\M{18}The \PB{\\{newchord}} subroutine, which inserts a chord from point \PB{%
\|s} to
point~\PB{\|t}, is the heart of the computation. It subdivides existing regions
that are crossed by the new chord.

Consider, for example, the insertion of the very first chord. Then
we want to insert a new edge, and its mate, so that there are two regions.
External edge \PB{$\|s+\T{1}$} will then be succeeded by the new edge from \PB{%
\|s} to~\PB{\|t},
and external edge \PB{$\|t+\T{1}$} will be succeeded by the new edge from \PB{%
\|t} to~\PB{\|s}.

Every chord insertion process ends in a similar way, with the
insertion of a new edge from \PB{\|t} to~\PB{\|s} that succeeds edge \PB{$\|t+%
\T{1}$},
and insertion of the mate edge from \PB{\|s} to~\PB{\|t} that succeeds some
other edge~\PB{\|q}.

\Y\B\4\X6:Subroutines\X${}\mathrel+\E{}$\6
\&{void} \\{finishchord}(\&{int} \|q${},\39{}$\&{int} \|s${},\39{}$\&{int} \|t)%
\1\1\2\2\6
${}\{{}$\1\6
\&{register} \&{int} \|m${}\K\\{newedge}(\|s,\39\|t);{}$\7
${}\|e[\|m].\\{succ}\K\|e[\|t+\T{1}].\\{succ},\39\|e[\|m+\T{1}].\\{succ}\K\|e[%
\|q].\\{succ};{}$\6
${}\|e[\|t+\T{1}].\\{succ}\K\|m+\T{1},\39\|e[\|q].\\{succ}\K\|m;{}$\6
\4${}\}{}$\2\par
\fi

\M{19}A more complex operation is needed when we split a region by introducing
an interior vertex. Then we need to create two new pairs of mated edges.
One of these, from \PB{\|t} to~\PB{\|s}, becomes the successor of some
{\it interior\/} edge~\PB{\|p}, which is being cut into two edges; its mate,
from \PB{\|s} to~\PB{\|t}, becomes the successor of a given edge~\PB{\|q},
which
doesn't need to be cut. The second mated pair of new edges becomes
the other half of edge~\PB{\|p} (and its mate) after cutting.

The following subroutine returns the index of the internal edge that
can be used to continue chord insertion on the next region to be
encountered between \PB{\|s} and~\PB{\|t}.

\Y\B\4\X6:Subroutines\X${}\mathrel+\E{}$\6
\&{int} \\{internalchord}(\&{int} \|p${},\39{}$\&{int} \|q${},\39{}$\&{int} %
\|s${},\39{}$\&{int} \|t)\1\1\2\2\6
${}\{{}$\1\6
\&{register} \&{int} \|m${}\K\\{newedge}(\|s,\39\|t),{}$ \\{mm}${}\K%
\\{newedge}(\|e[\|p].\\{from},\39\|e[\|p].\\{to});{}$\6
\&{register} \&{int} \\{pp}${}\K\\{mate}(\|p),{}$ \\{ppp}${}\K\\{pred}(%
\\{pp});{}$\7
${}\|e[\|m].\\{succ}\K\\{mm},\39\|e[\|m+\T{1}].\\{succ}\K\|e[\|q].\\{succ};{}$\6
${}\|e[\\{mm}].\\{succ}\K\|e[\|p].\\{succ},\39\|e[\\{mm}+\T{1}].\\{succ}\K%
\\{pp};{}$\6
${}\|e[\|p].\\{succ}\K\|m+\T{1},\39\|e[\|q].\\{succ}\K\|m,\39\|e[\\{ppp}].%
\\{succ}\K\\{mm}+\T{1};{}$\6
\&{return} \\{mm}${}+\T{1};{}$\6
\4${}\}{}$\2\par
\fi

\M{20}Now we're ready to insert a new chord in its entirety.

We use the fact that points $(a,b,c)$ are ``cyclically ordered'' if and
only if $(b-a)\bmod\PB{\\{pp}}<(c-a)\bmod\PB{\\{pp}}$.

\Y\B\4\X6:Subroutines\X${}\mathrel+\E{}$\6
\&{void} \\{newchord}(\&{int} \|s${},\39{}$\&{int} \|t)\1\1\2\2\6
${}\{{}$\1\6
\&{register} \&{int} \|p${},{}$ \|q;\7
\&{for} ${}(\|q\K\|s+\T{1};{}$  ; \,)\5
${}\{{}$\1\6
${}\|p\K\|e[\|q].\\{succ}{}$;\C{ the edge following \PB{\|q} will never be cut
}\6
\&{for} ${}(\|p\K\|e[\|p].\\{succ};{}$  ; ${}\|p\K\|e[\|p].\\{succ}){}$\5
${}\{{}$\1\6
\&{if} ${}(\|p\E\|q){}$\5
${}\{{}$\1\6
${}\\{fprintf}(\\{stderr},\39\.{"This\ can't\ happen\ (}\)\.{newchord\ loop)!%
\\n"});{}$\6
${}\\{exit}({-}\T{1});{}$\6
\4${}\}{}$\2\6
\&{if} ${}(\|p\Z\\{pp}){}$\5
${}\{{}$\C{ exterior edge }\1\6
\&{if} ${}(\|p\E\|t+\T{1}){}$\1\5
\&{break};\2\6
\4${}\}{}$\5
\2\&{else}\5
${}\{{}$\C{ interior edge }\1\6
\&{if} ${}(((\|t-\|e[\|p].\\{from}+\\{pp})\MOD\\{pp})<((\|e[\|p].\\{to}-\|e[%
\|p].\\{from}+\\{pp})\MOD\\{pp})){}$\1\5
\&{break};\2\6
\4${}\}{}$\2\6
\4${}\}{}$\2\6
\&{if} ${}(\|p\Z\\{pp}){}$\1\5
\&{break};\2\6
${}\|q\K\\{internalchord}(\|p,\39\|q,\39\|s,\39\|t){}$;\C{ see the discussion
below }\6
\4${}\}{}$\2\6
${}\\{finishchord}(\|q,\39\|s,\39\|t);{}$\6
\4${}\}{}$\2\par
\fi

\M{21}The situation that arises when the \PB{\\{internalchord}} routine is
called in the loop above needs some justification and further explanation.

Edge \PB{\|p} is an internal edge that goes from $s'$ to $t'$, say. We also
know that the points $(s,s',t,t')$ appear in that order as we traverse the
outer circle.

If the predecessor of edge \PB{\|p} is also an internal edge, say from
$s''$ to $t''$, then the points $(s,s'',s',t'',t')$ are in cyclic order.
In order to be sure that the new chord from $s$ to~$t$ is forced to cut
edge~$p$, we need to know that $(t'',t,t')$ are in cyclic order. The
alternative is the cyclic order $(s,s'',s',t,t'',t')$; if that were true,
the edge to cut would be ambiguous. Fortunately, however, the $abcabc$
condition has ruled out such a possibility.

A similar argument applies if the successor of edge \PB{\|p} is internal;
again, we conclude that the \PB{\\{newchord}} algorithm is justified
in cutting edge~\PB{\|p}.

A fancier argument would set up an invariant relation on each region
that arises during the construction process, showing that the
endpoints of each internal edge maintain a simple cyclic relationship.
Instead of making formal definitions, an example should suffice here:
Consider a region with edges $(k_1,k_2,\ldots,k_8)$ in cyclic order,
where $k_1$, $k_2$, and $k_5$ are external, while $k_3$ is
internal from $s_3$ to $t_3$, etc. Then $k_2-1=k_1$, $s_3=k_2$, $t_4=k_5-1$,
$s_6=k_5$, and $t_8=k_1-1$; and the points
$$(k_1,k_2,s_4,t_3,k_5-1,k_5,s_7,t_6,s_8,t_7,k_1-1)$$
are in cyclic order. Think about it.

\fi

\M{22}Once we've inserted the chords, we can construct the graph by
giving each region an identifying number.

At the same time we can reject cases with articulation points ---
namely, when a region has more than one exterior edge.

When this computation is finished, we will have discovered the total number of
regions, \PB{\|l}, which is also the total number of vertices in the
corresponding
squaregraph.

Note: We execute this code at a time when the local variables \PB{\|p}, \PB{%
\|j},
and~\PB{\|s} must not be changed. (They are part of the backtracking mechanism
for delta sequences.) The author apologies for not subroutinizing everything.

\Y\B\4\X22:Mark the regions, but \PB{\&{goto} \\{next}} if there's an
articulation problem\X${}\E{}$\6
\&{for} ${}(\|l\K\T{0},\39\|k\K\T{1};{}$ ${}\|k<\\{eptr};{}$ ${}\|k\PP){}$\1\6
\&{if} ${}(\|e[\|k].\\{reg}\E\T{0}){}$\5
${}\{{}$\1\6
${}\|e[\|k].\\{reg}\K\PP\|l;{}$\6
\&{for} ${}(\|q\K\|e[\|k].\\{succ};{}$ ${}\|e[\|q].\\{reg}\E\T{0};{}$ ${}\|q\K%
\|e[\|q].\\{succ}){}$\5
${}\{{}$\1\6
\&{if} ${}(\|q\Z\\{pp}){}$\1\5
\&{goto} \\{next};\2\6
${}\|e[\|q].\\{reg}\K\|l;{}$\6
\4${}\}{}$\2\6
\4${}\}{}$\2\2\par
\U8.\fi

\N{1}{23}Finishing up. The rest of this program is easy.

When we're ready to print a delta sequence, we know the total number
of vertices in the corresponding squarefree graph, \PB{\|l}. The total number
of edges is also essentially known, because there are
\PB{$(\\{eptr}-\\{pp}-\T{2})/\T{2}$} pairs of mates. Furthermore, each square
has four sides;
hence the number of squares, times 4, is the number of perimeter edges
plus twice the number of nonperimeter edges.

\Y\B\4\X23:Print \PB{\\{del}} and its characteristics\X${}\E{}$\6
$\\{ptot}\PP;{}$\6
${}\\{printf}(\.{"\%d:"},\39\\{ptot});{}$\6
\&{for} ${}(\|k\K\T{0};{}$ ${}\|k<\\{pp};{}$ ${}\|k\PP){}$\1\5
${}\\{printf}(\.{"\ \%d"},\39\\{del}[\|k]);{}$\2\6
${}\\{printf}(\.{"\ (\%d\%s),"},\39\\{rot},\39\\{rfl}\?\.{"R"}:\.{""});{}$\6
${}\\{printf}(\.{"\ \%d\ v,\ \%d\ e,\ \%d\ iv,}\)\.{\ \%d\ sq\\n"},\39\|l,\39(%
\\{eptr}\GG\T{1})-\T{1}-\|p,\39\|l-\\{pp},\39(\\{eptr}\GG\T{2})-\|p){}$;\C{
that's vertices, edges, internal vertices, and squares }\6
\&{if} (\\{verbose})\1\5
\X24:Print the graph\X;\2\par
\U8.\fi

\M{24}\B\X24:Print the graph\X${}\E{}$\6
\&{for} ${}(\|k\K\|q\K\T{1};{}$ ${}\|k\Z\|l;{}$ ${}\|k\PP){}$\5
${}\{{}$\1\6
\&{register} \&{int} \|r;\7
${}\\{printf}(\.{"\ \%d\ --"},\39\|k);{}$\6
\&{while} ${}(\|e[\|q].\\{reg}\I\|k){}$\1\5
${}\|q\PP;{}$\2\6
\&{for} ${}(\|r\K\|e[\|q].\\{succ};{}$  ; ${}\|r\K\|e[\|r].\\{succ}){}$\5
${}\{{}$\1\6
\&{if} ${}(\|r\Z\\{pp}){}$\1\5
\&{break};\2\6
${}\\{printf}(\.{"\ \%d"},\39\|e[\\{mate}(\|r)].\\{reg});{}$\6
\&{if} ${}(\|r\E\|q){}$\1\5
\&{break};\2\6
\4${}\}{}$\2\6
\\{printf}(\.{"\\n"});\6
\4${}\}{}$\2\par
\U23.\fi

\M{25}\B\X25:Update the totals\X${}\E{}$\6
$\|q\K\\{pp}/\\{rot};{}$\6
\&{if} ${}(\\{rfl}\E\T{0}){}$\1\5
${}\|q\MRL{{\LL}{\K}}\T{1}{}$;\C{ now \PB{\|q} is the number of labeled
varieties of \PB{\\{del}} }\2\6
${}\\{pltot}\MRL{+{\K}}\|q;{}$\6
${}\\{vtot}[\|l]\PP;{}$\6
${}\\{vltot}[\|l]\MRL{+{\K}}\|q;{}$\6
${}\\{stot}[(\\{eptr}\GG\T{2})-\|p]\PP;{}$\6
${}\\{sltot}[(\\{eptr}\GG\T{2})-\|p]\MRL{+{\K}}\|q{}$;\par
\U8.\fi

\N{1}{26}Index.
\fi

\inx
\fin
\con
