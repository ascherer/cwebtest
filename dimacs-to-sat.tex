\input cwebmac

\N{1}{1}Intro. This is a simple filter that inputs the well-known DIMACS format
for
satisfiability problems and outputs the format used by {\mc SAT0}, {\mc SAT1},
etc.

DIMACS format begins with zero or more optional comment lines, indicated by
their first character `\.c'. The next line should say `\.p \.{cnf} $n$ $m$',
where $n$ is the number of variables and $m$ is the number of clauses.
Then comes a string of $m$ ``clauses,'' which are sequences of
nonzero integers of absolute value $\le n$, followed by zero.
A literal for the $k$th variable is represented by $k$; its complement
is represented by $-k$.

SAT format is explained in the programs cited above.

\Y\B\4\D$\\{buf\_size}$ \5
\T{1024}\par
\Y\B\8\#\&{include} \.{<stdio.h>}\6
\8\#\&{include} \.{<stdlib.h>}\6
\&{char} \\{buf}[\\{buf\_size}];\6
\&{int} \|m${},{}$ \|n${},{}$ \|v;\7
\\{main}(\,)\1\1\2\2\6
${}\{{}$\1\6
\&{register} \&{int} \|j${},{}$ \|k${},{}$ \|l${},{}$ \|p;\7
\X2:Read the preamble\X;\6
\&{if} ${}(\|m\E\T{0}\V\|n\E\T{0}){}$\5
${}\{{}$\1\6
${}\\{fprintf}(\\{stderr},\39\.{"I\ didn't\ find\ m\ or\ }\)\.{n!\\n"});{}$\6
${}\\{exit}({-}\T{1});{}$\6
\4${}\}{}$\2\6
\X4:Read and translate the clauses\X;\6
\4${}\}{}$\2\par
\fi

\M{2}\B\X2:Read the preamble\X${}\E{}$\6
\&{while} (\T{1})\5
${}\{{}$\1\6
\&{if} ${}(\R\\{fgets}(\\{buf},\39\\{buf\_size},\39\\{stdin})){}$\1\5
\&{break};\2\6
\&{if} ${}(\\{buf}[\T{0}]\E\.{'c'}){}$\1\5
\X3:Process a comment line and \PB{\&{continue}}\X;\2\6
\&{if} ${}(\\{buf}[\T{0}]\I\.{'p'}\V\\{buf}[\T{1}]\I\.{'\ '}\V\\{buf}[\T{2}]\I%
\.{'c'}\V\\{buf}[\T{3}]\I\.{'n'}\V\\{buf}[\T{4}]\I\.{'f'}){}$\5
${}\{{}$\1\6
${}\\{fprintf}(\\{stderr},\39\.{"Unrecognized\ input\ }\)\.{line:\ \%s\\n"},\39%
\\{buf});{}$\6
${}\\{exit}({-}\T{2});{}$\6
\4${}\}{}$\2\6
${}\\{sscanf}(\\{buf}+\T{5},\39\.{"\%i\ \%i"},\39{\AND}\|n,\39{\AND}\|m);{}$\6
\&{break};\6
\4${}\}{}$\2\par
\U1.\fi

\M{3}\B\X3:Process a comment line and \PB{\&{continue}}\X${}\E{}$\6
${}\{{}$\1\6
${}\\{printf}(\.{"\~\ \%s"},\39\\{buf}+\T{1});{}$\6
\&{continue};\6
\4${}\}{}$\2\par
\U2.\fi

\M{4}\B\X4:Read and translate the clauses\X${}\E{}$\6
$\|j\K\|k\K\|l\K\|p\K\T{0};{}$\6
\&{while} ${}(\\{fscanf}(\\{stdin},\39\.{"\%i"},\39{\AND}\|v)\E\T{1}){}$\5
${}\{{}$\1\6
\&{if} ${}(\|v\E\T{0}){}$\5
${}\{{}$\1\6
\&{if} ${}(\|j\E\T{0}){}$\1\5
${}\\{fprintf}(\\{stderr},\39\.{"Warning:\ Clause\ \%d\ }\)\.{is\ empty!\\n"},%
\39\|k+\T{1});{}$\2\6
\\{printf}(\.{"\\n"});\6
\&{if} ${}(\|k\E\|m){}$\5
${}\{{}$\1\6
${}\\{fprintf}(\\{stderr},\39\.{"Too\ many\ clauses\ (m}\)\.{ore\ than\ \%d)!%
\\n"},\39\|m);{}$\6
${}\\{exit}({-}\T{3});{}$\6
\4${}\}{}$\2\6
${}\|k\PP,\39\|j\K\T{0};{}$\6
\4${}\}{}$\5
\2\&{else} \&{if} ${}(\|v>\T{0}){}$\5
${}\{{}$\1\6
\&{if} ${}(\|v>\|n){}$\5
${}\{{}$\1\6
${}\\{fprintf}(\\{stderr},\39\.{"Too\ many\ variables\ }\)\.{(\%d\ >\ \%d)!%
\\n"},\39\|v,\39\|n);{}$\6
${}\\{exit}({-}\T{4});{}$\6
\4${}\}{}$\2\6
${}\\{printf}(\.{"\ \%d"},\39\|v);{}$\6
\&{if} ${}(\|v>\|p){}$\1\5
${}\|p\K\|v;{}$\2\6
${}\|j\PP,\39\|l\PP;{}$\6
\4${}\}{}$\5
\2\&{else}\5
${}\{{}$\1\6
\&{if} ${}(\|v<{-}\|n){}$\5
${}\{{}$\1\6
${}\\{fprintf}(\\{stderr},\39\.{"Too\ many\ variables\ }\)\.{(\%d\ <\ -\%d)!%
\\n"},\39\|v,\39\|n);{}$\6
${}\\{exit}({-}\T{5});{}$\6
\4${}\}{}$\2\6
${}\\{printf}(\.{"\ \~\%d"},\39{-}\|v);{}$\6
\&{if} ${}({-}\|v>\|p){}$\1\5
${}\|p\K{-}\|v;{}$\2\6
${}\|j\PP,\39\|l\PP;{}$\6
\4${}\}{}$\2\6
\4${}\}{}$\2\6
\&{if} (\|j)\5
${}\{{}$\1\6
${}\\{fprintf}(\\{stderr},\39\.{"The\ last\ clause\ did}\)\.{n't\ end\ with\ 0!%
\\n"});{}$\6
\\{printf}(\.{"\\n"});\6
${}\|k\PP;{}$\6
\4${}\}{}$\2\6
\&{if} ${}(\|k<\|m){}$\1\5
${}\\{fprintf}(\\{stderr},\39\.{"Too\ few\ clauses\ (\%d}\)\.{\ <\ \%d)!\\n"},%
\39\|k,\39\|m);{}$\2\6
${}\\{fprintf}(\\{stderr},\39\.{"OK,\ I\ wrote\ out\ \%d\ }\)\.{literals\ in\ %
\%d\ claus}\)\.{es\ on\ \%d\ variables.\\}\)\.{n"},\39\|l,\39\|k,\39\|p){}$;\par
\U1.\fi

\N{1}{5}Index.

\fi


\inx
\fin
\con
