\input cwebmac
\datethis




\N{1}{1}Introduction. This program combines the ideas of {\mc GRAYSPAN}
and {\mc SPSPAN}, resulting in a glorious routine that generates
all spanning trees of a given graph, changing only one edge at a time,
with ``guaranteed efficiency''---in the sense that the total running
time is $O(m+n+t)$ when there are $m$ edges, $n$ vertices, and $t$
spanning trees.

The reader should be familiar with both {\mc GRAYSPAN} and {\mc SPSPAN},
because their principles of operation are not repeated here.

The first command line argument is the name of a file that specifies
an undirected graph in Stanford GraphBase
{\mc SAVE\_GRAPH} format. Additional command line
arguments are ignored except that they cause more verbose output.
The least verbose output contains only overall statistics about the total
number of spanning trees found and the total number of mems used.

\Y\B\4\D$\\{verbose}$ \5
$(\\{xargc}>\T{2}{}$)\par
\B\4\D$\\{extraverbose}$ \5
$(\\{xargc}>\T{3}{}$)\par
\B\4\D$\|o$ \5
$\\{mems}\PP{}$\par
\B\4\D$\\{oo}$ \5
$\\{mems}\MRL{+{\K}}{}$\T{2}\par
\B\4\D$\\{ooo}$ \5
$\\{mems}\MRL{+{\K}}{}$\T{3}\par
\B\4\D$\\{oooo}$ \5
$\\{mems}\MRL{+{\K}}{}$\T{4}\par
\B\4\D$\\{ooooo}$ \5
$\\{mems}\MRL{+{\K}}{}$\T{5}\par
\B\4\D$\\{oooooo}$ \5
$\\{mems}\MRL{+{\K}}{}$\T{6}\par
\Y\B\8\#\&{include} \.{"gb\_graph.h"}\6
\8\#\&{include} \.{"gb\_save.h"}\6
\ATH\6
\&{double} \\{mems};\C{ memory references made }\6
\&{double} \\{count};\C{ trees found }\6
\&{Graph} ${}{*}\\{gg}{}$;\C{ a global copy of \PB{\|g} }\6
\&{int} \\{xargc};\C{ a global copy of \PB{\\{argc}} }\7
\X5:Type definitions\X\6
\X7:Global variables\X\6
\X9:Subroutines\X\7
\\{main}(\&{int} \\{argc}${},\39{}$\&{char} ${}{*}\\{argv}[\,]){}$\1\1\2\2\6
${}\{{}$\1\6
\X3:Local variables\X;\6
\X2:Input the graph\X;\6
\&{if} ${}(\|n>\T{1}){}$\1\5
\X25:Generate all spanning trees\X;\2\6
${}\\{printf}(\.{"Altogether\ \%.15g\ sp}\)\.{anning\ trees,\ using\ }\)\.{%
\%.15g\ mems.\\n"},\39\\{count},\39\\{mems});{}$\6
\\{exit}(\T{0});\6
\4${}\}{}$\2\par
\fi

\M{2}\B\X2:Input the graph\X${}\E{}$\6
\&{if} ${}(\\{argc}<\T{2}){}$\5
${}\{{}$\1\6
${}\\{fprintf}(\\{stderr},\39\.{"Usage:\ \%s\ foo.gb\ [[}\)\.{gory]\ details]%
\\n"},\39\\{argv}[\T{0}]);{}$\6
${}\\{exit}({-}\T{11});{}$\6
\4${}\}{}$\2\6
${}\\{xargc}\K\\{argc};{}$\6
${}\\{gg}\K\|g\K\\{restore\_graph}(\\{argv}[\T{1}]);{}$\6
\&{if} ${}(\R\|g){}$\5
${}\{{}$\1\6
${}\\{fprintf}(\\{stderr},\39\.{"Sorry,\ can't\ create}\)\.{\ the\ graph\ from\
file}\)\.{\ \%s!\ (error\ code\ \%d)}\)\.{\\n"},\39\\{argv}[\T{1}],\39\\{panic%
\_code});{}$\6
${}\\{exit}({-}\T{2});{}$\6
\4${}\}{}$\2\6
\X6:Allocate additional storage\X;\6
\X18:Check the graph for validity and prepare it for action\X;\par
\U1.\fi

\M{3}\B\X3:Local variables\X${}\E{}$\6
\&{register} \&{Graph} ${}{*}\|g{}$;\C{ the graph we're dealing with }\6
\&{register} \&{int} \|n;\C{ the number of vertices }\6
\&{register} \&{int} \|m;\C{ the number of edges }\6
\&{register} \&{int} \|l;\C{ the current level of shrinkage }\6
\&{register} \&{int} \|k;\C{ current integer of interest }\6
\&{register} \&{Vertex} ${}{*}\|u,{}$ ${}{*}\|v,{}$ ${}{*}\|w{}$;\C{ current
vertices of interest }\6
\&{register} \&{Arc} ${}{*}\|e,{}$ ${}{*}\\{ee},{}$ ${}{*}\|f,{}$ ${}{*}%
\\{ff}{}$;\C{ current edges of interest }\6
\&{register} \&{Bond} ${}{*}\|b{}$;\C{ current interest-bearing bond }\par
\U1.\fi

\N{1}{4}The method and its data structures. The basic idea of this program,
which goes back to Malcolm Smith's M.S. thesis, ``Generating spanning trees''
(University of Victoria, 1997), is to modify the {\mc GRAYSPAN} algorithm,
replacing edges by series-parallel subgraphs that we will call ``bonds.''

A {\it bond\/} between vertices \PB{\|u} and \PB{\|v} is either an edge of the
given graph, or a sequence of bonds joined in series, or a set of bonds
joined in parallel.
The {\mc GRAYSPSPAN} algorithm essentially uses
the {\mc GRAYSPAN} method to generate spanning trees with respect to bonds,
together with the {\mc SPSPAN} method to generate all of the corresponding
spanning trees with respect to the original edges.

Suppose we start by considering each edge to be a bond of the simplest kind.
Then we can remove multiple bonds, if any, by combining them in parallel.
And we can remove vertices of degree~2 by combining their adjacent bonds
in series. A vertex of degree~1 and its adjacent bond can be removed
from the graph, if we require that bond to be present in every spanning tree.
Under the assumption that the given graph is connected, repeated reductions
of this kind will eventually lead to a simple graph that is {\it irreducible},
either trivial (with only one point) or with all vertices of degree
3~or~more.

For example, consider the smallest nontrivial irreducible graph,
having four vertices and bonds between any two of them. Suppose the
vertices are $v_1$, $v_2$, $v_3$, $v_4$ and the bonds are $b_{12}$,
$b_{13}$, $b_{14}$, $b_{23}$, $b_{24}$, $b_{34}$. If the {\mc GRAYSPAN}
algorithm produces the spanning tree $b_{12}b_{13}b_{24}$, say, the
{\mc SPSPAN} algorithm is used to cycle through all the 1-configs of
$b_{12}$, $b_{13}$, and $b_{24}$ together with all the 0-configs
of the other bonds $b_{14}$, $b_{23}$, $b_{34}$. Then if {\mc GRAYSPAN}
produces its next spanning tree by changing $b_{24}$ to $b_{34}$,
this change corresponds to removing the designated leaf of bond~$b_{24}$
and including the designated leaf of $b_{34}$, after which {\mc SPSPAN}
runs through the 1-configs of $b_{12}$, $b_{13}$, $b_{34}$ with the
0-configs of $b_{14}$, $b_{23}$, $b_{24}$.

The actual operation of this program is not quite the same as
just described, because the
operation of shrinking $b_{12}$ causes the resulting graph to reduce to a
triviality; the {\mc GRAYSPAN} algorithm never gets to the
case $n=2$ that used to be its goal. But conceptually we make progress
on the graph as a whole
by shrinking edges and removing nonbridges just as in {\mc GRAYSPAN},
and each reduction in the number of bonds takes more of the computation
into the highly efficient {\mc SPSPAN} domain.

\fi

\M{5}A suitable data structure to support all this machinery can be
fashioned from a struct of type \&{Bond}, which contains the fields
\PB{\\{typ}}, \PB{\\{val}}, \PB{\\{des}}, \PB{\\{done}}, \PB{\\{focus}}, and %
\PB{\\{rsib}} that we used in
{\mc SPSPAN}. We also need \PB{\\{left}} and \PB{\\{right}} pointers, because
we
must keep branch nodes in a doubly linked list instead of allocating
them sequentially and statically as we did before. Then there are \PB{%
\\{lchild}},
\PB{\\{lsib}}, and \PB{\\{scope}} pointers for constructing new bonds from old
ones;
also \PB{\\{lhist}} and \PB{\\{rhist}} for deconstructing when bonds are being
undone.
Finally there's yet another doubly linked list, containing the top-level
bonds of the current spanning tree; its link fields are called \PB{\\{up}} and %
\PB{\\{down}}.

\Y\B\4\X5:Type definitions\X${}\E{}$\6
\&{typedef} \&{struct} \&{bond\_struct} ${}\{{}$\1\6
\&{int} \\{typ};\C{ 1 for series bonds, 0 for parallel bonds, 2 for deletion
records }\6
\&{int} \\{val};\C{ 1 if the bond is in the current spanning tree }\6
\&{struct} \&{bond\_struct} ${}{*}\\{lchild}{}$;\C{ leftmost child; \PB{$%
\NULL$} for a leaf }\6
\&{struct} \&{bond\_struct} ${}{*}\\{lsib}{}$;\C{ left sibling; wraps around
cyclically }\6
\&{struct} \&{bond\_struct} ${}{*}\\{rsib}{}$;\C{ right sibling; wraps around
cyclically }\6
\&{struct} \&{bond\_struct} ${}{*}\\{des}{}$;\C{ designated child }\6
\&{struct} \&{bond\_struct} ${}{*}\\{done}{}$;\C{ final designated child in a
sweep }\6
\&{struct} \&{bond\_struct} ${}{*}\\{scope}{}$;\C{ last branch descendant in
preorder }\6
\&{struct} \&{bond\_struct} ${}{*}\\{focus}{}$;\C{ magic pointer control for
Gray products }\6
\&{struct} \&{bond\_struct} ${}{*}\\{left}{}$;\C{ left neighbor in the action
list }\6
\&{struct} \&{bond\_struct} ${}{*}\\{right}{}$;\C{ right neighbor in the action
list }\6
\&{struct} \&{bond\_struct} ${}{*}\\{up}{}$;\C{ upper neighbor in the tree list
}\6
\&{struct} \&{bond\_struct} ${}{*}\\{down}{}$;\C{ lower neighbor in the tree
list }\6
\&{struct} \&{bond\_struct} ${}{*}\\{lsave},{}$ ${}{*}\\{rsave}{}$;\C{ needed
for restoration in one case }\6
\&{Arc} ${}{*}\\{lhist},{}$ ${}{*}\\{rhist}{}$;\C{ the edges that spawned this
bond }\2\6
${}\}{}$ \&{Bond};\par
\U1.\fi

\M{6}If the graph has \PB{\|m} edges, we put the basic one-edge bonds into
\PB{$\\{bondbase}+\|k$} for $1\le k\le m$; series and parallel combinations,
and ``deletion records'' for undoing a bond deletion, go into the
subsequent locations, in a last-in-first-out manner.

\Y\B\4\D$\\{isleaf}(\|b)$ \5
$((\|b)\Z\\{topleaf}{}$)\C{ is \PB{\|b} simply an edge? }\par
\B\4\D$\\{action}$ \5
\\{bondbase}\C{ head of the action list }\par
\B\4\D$\\{tree}$ \5
$({\AND}\\{treehead}{}$)\C{ head of the tree list }\par
\Y\B\4\X6:Allocate additional storage\X${}\E{}$\6
$\|m\K\|g\MG\|m/\T{2};{}$\6
${}\\{bondbase}\K{}$(\&{Bond} ${}{*}){}$ \\{calloc}${}(\T{3}*\|m,\39\&{sizeof}(%
\&{Bond}));{}$\6
${}\\{action}\MG\\{left}\K\\{action}\MG\\{right}\K\\{action},\39\\{action}\MG%
\\{typ}\K\T{2},\39\\{action}\MG\\{focus}\K\\{action}{}$;\C{ the action list
starts empty }\6
${}\\{tree}\MG\\{up}\K\\{tree}\MG\\{down}\K\\{tree}{}$;\C{ and so does the tree
list }\par
\U2.\fi

\M{7}\B\X7:Global variables\X${}\E{}$\6
\&{Bond} ${}{*}\\{bondbase}{}$;\C{ base address for almost all \&{Bond} structs
}\6
\&{int} \\{bondcount};\C{ the number of bonds currently defined }\6
\&{Bond} ${}{*}\\{topleaf}{}$;\C{ dividing line between leaves and branches }\6
\&{Bond} \\{treehead};\C{ header for the tree list }\6
\&{Bond} ${}{*}\\{inbond},{}$ ${}{*}\\{outbond}{}$;\C{ bonds to be included and
excluded next }\6
\&{Bond} \\{James};\C{ my little joke }\par
\U1.\fi

\M{8}The bonds between vertices are represented as \&{Arc} structs in
almost the usual way; namely, a bond between \PB{\|u} and \PB{\|v} appears as
an arc \PB{\|e} with \PB{$\|e\MG\\{tip}\K\|v$} in the list \PB{$\|u\MG%
\\{arcs}$}, and as an arc \PB{\|f}
with \PB{$\|f\MG\\{tip}\K\|u$} in the list \PB{$\|v\MG\\{arcs}$}, where \PB{$%
\|f\K\\{mate}(\|e)$} and \PB{$\|e\K\\{mate}(\|f)$}.
However, as in {\mc GRAYSPAN}, we doubly link the arc list,
making \PB{$\|e\MG\\{prev}\MG\\{next}\K\|e\MG\\{next}\MG\\{prev}\K\|e$} and
starting the list with
a dummy entry called its header.

We also include a new field
\PB{$\|e\MG\\{bond}$} pointing to the \&{Bond} struct that carries further
information.
This one has to be coerced to type \PB{(\&{Bond} ${}{*})$} when used in the %
\CEE/ code.

\Y\B\4\D$\\{deg}$ \5
$\|u.{}$\|I\C{ utility field \PB{\|u} of each vertex holds its current degree }%
\par
\B\4\D$\\{prev}$ \5
$\|a.{}$\|A\C{ utility field \PB{\|a} of each arc holds its backpointer }\par
\B\4\D$\\{bond}$ \5
$\|b.{}$\|A\C{ utility field \PB{\|b} of each arc leads to the corresponding %
\&{Bond} }\par
\B\4\D$\\{mate}(\|e)$ \5
$(\\{edge\_trick}\AND(\&{siz\_t})(\|e)\?(\|e)-\T{1}:(\|e)+\T{1}{}$)\par
\B\4\D$\\{bn}(\|b)$ \5
$((\|b)-\\{bondbase}{}$)\C{ the number of bond \PB{\|b}, for printout }\par
\B\4\D$\\{ebn}(\|e)$ \5
\\{bn}((\&{Bond} ${}{*})((\|e)\MG\\{bond}){}$)\par
\B\4\D$\\{delete}(\|e)$ \5
$\\{ee}\K\|e,\39\\{oooo},\39\\{ee}\MG\\{prev}\MG\\{next}\K\\{ee}\MG\\{next},\39%
\\{ee}\MG\\{next}\MG\\{prev}\K\\{ee}\MG{}$\\{prev}\par
\B\4\D$\\{undelete}(\|e)$ \5
$\\{ee}\K\|e,\39\\{oooo},\39\\{ee}\MG\\{next}\MG\\{prev}\K\\{ee},\39\\{ee}\MG%
\\{prev}\MG\\{next}\K{}$\\{ee}\par
\fi

\N{1}{9}Diagnostic routines. Here are a few handy ways to look at the current
data.

\Y\B\4\X9:Subroutines\X${}\E{}$\6
\&{void} \\{printgraph}(\&{void})\C{ prints the current graph }\6
${}\{{}$\1\6
\&{register} \&{Vertex} ${}{*}\|v;{}$\6
\&{register} \&{Arc} ${}{*}\|e;{}$\7
\&{for} ${}(\|v\K\\{gg}\MG\\{vertices};{}$ ${}\|v<\\{gg}\MG\\{vertices}+\\{gg}%
\MG\|n;{}$ ${}\|v\PP){}$\1\6
\&{if} ${}(\|v\MG\\{mark}\G\T{0}){}$\5
${}\{{}$\1\6
${}\\{printf}(\.{"Bonds\ from\ \%s:"},\39\|v\MG\\{name});{}$\6
\&{for} ${}(\|e\K\|v\MG\\{arcs}\MG\\{next};{}$ ${}\|e\MG\\{tip};{}$ ${}\|e\K\|e%
\MG\\{next}){}$\1\5
${}\\{printf}(\.{"\ \%s(\%d)"},\39\|e\MG\\{tip}\MG\\{name},\39\\{ebn}(\|e));{}$%
\2\6
\\{printf}(\.{"\\n"});\6
\4${}\}{}$\2\2\6
\4${}\}{}$\2\par
\As10, 11, 12, 31, 38\ETs39.
\U1.\fi

\M{10}\B\X9:Subroutines\X${}\mathrel+\E{}$\6
\&{void} \\{printbond}(\&{Bond} ${}{*}\|b){}$\1\1\2\2\6
${}\{{}$\1\6
${}\\{printf}(\.{"\%d:\%c"},\39\\{bn}(\|b),\39\\{isleaf}(\|b)\?\.{'l'}:\|b\MG%
\\{typ}\E\T{2}\?\.{'r'}:\|b\MG\\{typ}\E\T{1}\?\.{'s'}:\.{'p'});{}$\6
\&{if} ${}(\|b\MG\\{typ}\E\T{2}){}$\5
${}\{{}$\1\6
${}\\{printf}(\.{"\ lhist=\%d"},\39\\{ebn}(\|b\MG\\{lhist}));{}$\6
\&{if} ${}(\|b\MG\\{rhist}){}$\1\5
${}\\{printf}(\.{"\ rhist=\%d"},\39\\{ebn}(\|b\MG\\{lhist}));{}$\2\6
\4${}\}{}$\5
\2\&{else}\5
${}\{{}$\1\6
${}\\{printf}(\.{"\%c"},\39\|b\MG\\{val}+\.{'0'});{}$\6
\&{if} ${}(\|b\MG\\{lsib}){}$\1\5
${}\\{printf}(\.{"\ lsib=\%d,\ rsib=\%d"},\39\\{bn}(\|b\MG\\{lsib}),\39\\{bn}(%
\|b\MG\\{rsib}));{}$\2\6
\&{if} ${}(\R\\{isleaf}(\|b)){}$\5
${}\{{}$\1\6
\&{if} ${}(\|b\MG\\{focus}\I\|b){}$\1\5
${}\\{printf}(\.{"\ focus=\%d"},\39\\{bn}(\|b\MG\\{focus}));{}$\2\6
${}\\{printf}(\.{"\ lchild=\%d,\ scope=\%}\)\.{d,\ des=\%d,\ done=\%d"},\39%
\\{bn}(\|b\MG\\{lchild}),\39\\{bn}(\|b\MG\\{scope}),\39\\{bn}(\|b\MG\\{des}),%
\39\\{bn}(\|b\MG\\{done}));{}$\6
\4${}\}{}$\2\6
\4${}\}{}$\2\6
\\{printf}(\.{"\\n"});\6
\4${}\}{}$\2\par
\fi

\M{11}\B\X9:Subroutines\X${}\mathrel+\E{}$\6
\&{void} \\{printaction}(\&{void})\C{ prints the current action list }\6
${}\{{}$\1\6
\&{register} \&{Bond} ${}{*}\|b;{}$\7
\&{for} ${}(\|b\K\\{action}\MG\\{right};{}$ ${}\|b\I\\{action};{}$ ${}\|b\K\|b%
\MG\\{right}){}$\1\5
\\{printbond}(\|b);\2\6
\4${}\}{}$\2\7
\&{void} \\{printtree}(\&{void})\C{ prints the current tree list }\6
${}\{{}$\1\6
\&{register} \&{Bond} ${}{*}\|b;{}$\7
\&{for} ${}(\|b\K\\{tree}\MG\\{down};{}$ ${}\|b\I\\{tree};{}$ ${}\|b\K\|b\MG%
\\{down}){}$\1\5
\\{printbond}(\|b);\2\6
\4${}\}{}$\2\7
\&{void} \\{printleaves}(\&{void})\C{ prints all the leaves }\6
${}\{{}$\1\6
\&{register} \&{Bond} ${}{*}\|b;{}$\7
\&{for} ${}(\|b\K\\{bondbase}+\T{1};{}$ ${}\|b\Z\\{topleaf};{}$ ${}\|b\PP){}$\1%
\5
\\{printbond}(\|b);\2\6
\4${}\}{}$\2\par
\fi

\M{12}Since there's so much redundancy in the data structures, I need
reassurance
that I haven't slipped up and forgotten to keep everything shipshape.
``A~data structure is only as strong as its weakest link.''

The \PB{\\{sanitycheck}}
routine is designed to print out most discrepancies between my assumptions and
the true state of affairs. I~used it to locate lapses when this program
was being debugged, and it remains as testimony to the most vital structural
assumptions that are being made.

\Y\B\4\X9:Subroutines\X${}\mathrel+\E{}$\6
\X16:Declare the recursive routine \PB{\\{bondsanity}}\X;\7
\&{int} \\{sanitycheck}(\&{int} \\{flags})\1\1\2\2\6
${}\{{}$\1\6
\&{register} \&{Vertex} ${}{*}\|u,{}$ ${}{*}\|v;{}$\6
\&{register} \&{Arc} ${}{*}\|e;{}$\6
\&{register} \&{Bond} ${}{*}\|b;{}$\6
\&{register} \&{int} \|k${},{}$ \|n;\6
\&{int} \\{bugs}${}\K\T{0};{}$\7
\&{if} ${}(\\{flags}\AND\T{1}){}$\1\5
\X13:Do a sanity check on the graph\X;\2\6
\&{if} ${}(\\{flags}\AND\T{2}){}$\1\5
\X15:Do a sanity check on the action list\X;\2\6
\&{if} ${}(\\{flags}\AND\T{4}){}$\1\5
\X17:Do a sanity check on the tree list\X;\2\6
\&{return} \\{bugs};\6
\4${}\}{}$\2\par
\fi

\M{13}\B\X13:Do a sanity check on the graph\X${}\E{}$\6
${}\{{}$\1\6
\&{for} ${}(\|n\K\T{0},\39\|v\K\\{gg}\MG\\{vertices};{}$ ${}\|v<\\{gg}\MG%
\\{vertices}+\\{gg}\MG\|n;{}$ ${}\|v\PP){}$\1\6
\&{if} ${}(\|v\MG\\{mark}\G\T{0}){}$\1\5
\X14:Do a sanity check on \PB{\|v}'s bond list\X;\2\2\6
\4${}\}{}$\2\par
\U12.\fi

\M{14}Some of the ``bugs'' detected in this routine are, of course,
harmless in certain contexts. My goal is to call attention to
things that might be unexpected, but to keep going in any case.

\Y\B\4\X14:Do a sanity check on \PB{\|v}'s bond list\X${}\E{}$\6
${}\{{}$\1\6
${}\|n\PP;{}$\6
\&{if} ${}(\|v\MG\\{mark}){}$\1\5
${}\\{bugs}\PP,\39\\{printf}(\.{"Vertex\ \%s\ is\ marked}\)\.{\\n"},\39\|v\MG%
\\{name});{}$\2\6
\&{if} ${}((\|v\MG\\{deg}<\T{3}\W\|v\MG\\{deg}\I\T{0})\V\|v\MG\\{deg}\G\\{gg}%
\MG\|n){}$\1\5
${}\\{bugs}\PP,\39\\{printf}(\.{"Vertex\ \%s\ has\ degre}\)\.{e\ \%d\\n"},\39%
\|v\MG\\{name},\39\|v\MG\\{deg});{}$\2\6
\&{for} ${}(\|k\K\T{0},\39\|e\K\|v\MG\\{arcs};{}$ ${}\|k\Z\|v\MG\\{deg};{}$ ${}%
\|k\PP,\39\|e\K\|e\MG\\{next}){}$\5
${}\{{}$\1\6
\&{if} ${}(\|e\MG\\{prev}\MG\\{next}\I\|e\V\|e\MG\\{next}\MG\\{prev}\I\|e){}$\1%
\5
${}\\{bugs}\PP,\39\\{printf}(\.{"Link\ failure\ at\ ver}\)\.{tex\ \%s,\ bond\ %
\%d\\n"},\39\|v\MG\\{name},\39\|k);{}$\2\6
\&{if} ${}(\|k>\T{0}){}$\5
${}\{{}$\1\6
${}\|b\K{}$(\&{Bond} ${}{*}){}$ \|e${}\MG\\{bond};{}$\6
\&{if} ${}(\|b\Z\\{bondbase}\V\|b>\\{bondbase}+\\{bondcount}){}$\1\5
${}\\{bugs}\PP,\39\\{printf}(\.{"Vertex\ \%s\ has\ bad\ b}\)\.{ond\ \%d\\n"},%
\39\|v\MG\\{name},\39\|k);{}$\2\6
\&{else} \&{if} ${}(\|b\MG\\{lhist}\I\|e\W\|b\MG\\{lhist}\I\\{mate}(\|e)){}$\1\5
${}\\{bugs}\PP,\39\\{printf}(\.{"Bond\ \%d\ has\ bad\ lhi}\)\.{st\ pointer%
\\n"},\39\\{bn}(\|b));{}$\2\6
\&{if} ${}(\\{mate}(\|e)\MG\\{bond}\I\|e\MG\\{bond}){}$\1\5
${}\\{bugs}\PP,\39\\{printf}(\.{"Vertex\ \%s\ has\ unmat}\)\.{ed\ bond\ \%d%
\\n"},\39\|v\MG\\{name},\39\|k);{}$\2\6
\&{if} ${}(\\{mate}(\|e)\MG\\{tip}\I\|v){}$\1\5
${}\\{bugs}\PP,\39\\{printf}(\.{"Vertex\ \%s's\ bond\ \%d}\)\.{\ has\ wrong\
mate\ tip\\}\)\.{n"},\39\|v\MG\\{name},\39\|k);{}$\2\6
${}\|u\K\|e\MG\\{tip};{}$\6
\&{if} ${}(\R\|u){}$\1\5
${}\\{bugs}\PP,\39\\{printf}(\.{"Vertex\ \%s\ has\ bad\ t}\)\.{ip\ \%d\\n"},\39%
\|v\MG\\{name},\39\|k);{}$\2\6
\&{else} \&{if} ${}(\|u\MG\\{mark}<\T{0}){}$\1\5
${}\\{bugs}\PP,\39\\{printf}(\.{"Vertex\ \%s\ points\ to}\)\.{\ deleted\ vertex%
\ \%s\\n}\)\.{"},\39\|v\MG\\{name},\39\|u\MG\\{name});{}$\2\6
\4${}\}{}$\2\6
\4${}\}{}$\2\6
\&{if} ${}(\|e\I\|v\MG\\{arcs}){}$\1\5
${}\\{bugs}\PP,\39\\{printf}(\.{"Vertex\ \%s\ has\ more\ }\)\.{than\ \%d\ bonds%
\\n"},\39\|v\MG\\{name},\39\|v\MG\\{deg});{}$\2\6
\4${}\}{}$\2\par
\U13.\fi

\M{15}The action list is essentially a forest of bonds, in preorder.

\Y\B\4\X15:Do a sanity check on the action list\X${}\E{}$\6
${}\{{}$\1\6
\&{if} ${}(\\{action}\MG\\{left}\MG\\{right}\I\\{action}\V\\{action}\MG%
\\{right}\MG\\{left}\I\\{action}){}$\1\5
${}\\{bugs}\PP,\39\\{printf}(\.{"Link\ failure\ at\ hea}\)\.{d\ of\ action\
list\\n"});{}$\2\6
\&{for} ${}(\|b\K\\{action}\MG\\{right};{}$ ${}\|b\I\\{action};{}$ ${}\|b\K\|b%
\MG\\{right}){}$\5
${}\{{}$\1\6
\&{if} ${}(\|b\MG\\{val}\E\T{1}\W(\R\|b\MG\\{up}\V\|b\MG\\{up}\MG\\{down}\I%
\|b)){}$\1\5
${}\\{bugs}\PP,\39\\{printf}(\.{"Bond\ \%d\ isn't\ prope}\)\.{rly\ in\ the\
tree\ list}\)\.{\\n"},\39\\{bn}(\|b));{}$\2\6
\&{if} ${}(\|b\MG\\{lsib}\V\|b\MG\\{rsib}){}$\1\5
${}\\{bugs}\PP,\39\\{printf}(\.{"Top\ level\ bond\ \%d\ h}\)\.{as\ siblings%
\\n"},\39\\{bn}(\|b));{}$\2\6
\&{if} (\\{isleaf}(\|b))\1\5
${}\\{bugs}\PP,\39\\{printf}(\.{"Leaf\ \%d\ is\ in\ the\ a}\)\.{ction\ list%
\\n"},\39\\{bn}(\|b));{}$\2\6
\&{else}\5
${}\{{}$\1\6
${}\\{bugs}\MRL{+{\K}}\\{bondsanity}(\|b);{}$\6
${}\|b\K\|b\MG\\{scope};{}$\6
\4${}\}{}$\2\6
\4${}\}{}$\2\6
\4${}\}{}$\2\par
\U12.\fi

\M{16}\B\X16:Declare the recursive routine \PB{\\{bondsanity}}\X${}\E{}$\6
\&{int} \\{bondsanity}(\&{Bond} ${}{*}\|b){}$\1\1\2\2\6
${}\{{}$\1\6
\&{int} \\{bugs}${}\K\T{0};{}$\6
\&{register} \&{Bond} ${}{*}\|a,{}$ ${}{*}\\{extent};{}$\6
\&{register} \&{int} \|j${},{}$ \|k;\7
${}\\{extent}\K\|b;{}$\6
\&{if} ${}(\|b\MG\\{left}\MG\\{right}\I\|b\V\|b\MG\\{right}\MG\\{left}\I\|b){}$%
\1\5
${}\\{bugs}\PP,\39\\{printf}(\.{"Link\ failure\ at\ bon}\)\.{d\ \%d\\n"},\39%
\\{bn}(\|b));{}$\2\6
\&{for} ${}(\|a\K\|b\MG\\{lchild},\39\|j\K\|k\K\T{0};{}$ ${}\|a\I\|b\MG%
\\{lchild}\V\|k\E\T{0};{}$ ${}\|a\K\|a\MG\\{rsib},\39\|k\PP){}$\5
${}\{{}$\1\6
\&{if} ${}(\|a\Z\\{bondbase}\V\|a>\\{bondbase}+\\{bondcount}){}$\5
${}\{{}$\1\6
${}\\{bugs}\PP,\39\\{printf}(\.{"Bond\ \%d\ has\ a\ child}\)\.{\ out\ of\ range%
\\n"},\39\\{bn}(\|b));{}$\6
\&{break};\6
\4${}\}{}$\2\6
\&{if} ${}(\|a\MG\\{lsib}\MG\\{rsib}\I\|a\V\|a\MG\\{rsib}\MG\\{lsib}\I\|a){}$\1%
\5
${}\\{bugs}\PP,\39\\{printf}(\.{"Sibling\ link\ failur}\)\.{e\ at\ bond\ \%d%
\\n"},\39\\{bn}(\|a));{}$\2\6
\&{if} ${}(\|a\E\|b\MG\\{des}){}$\1\5
${}\|j\K\T{1};{}$\2\6
\&{else} \&{if} ${}(\|a\MG\\{val}\I\|b\MG\\{typ}){}$\1\5
${}\\{bugs}\PP,\39\\{printf}(\.{"Bond\ \%d\ should\ have}\)\.{\ value\ \%d%
\\n"},\39\\{bn}(\|a),\39\|b\MG\\{typ});{}$\2\6
\&{if} ${}(\R\\{isleaf}(\|a)){}$\5
${}\{{}$\1\6
\&{if} ${}(\|a\MG\\{left}\I\\{extent}){}$\1\5
${}\\{bugs}\PP,\39\\{printf}(\.{"Preorder\ failure:\ b}\)\.{ond\ \%d\ doesn't\
follo}\)\.{w\ \%d\\n"},\39\\{bn}(\|a),\39\\{bn}(\\{extent}));{}$\2\6
${}\\{bugs}\MRL{+{\K}}\\{bondsanity}(\|a);{}$\6
${}\\{extent}\K\|a\MG\\{scope};{}$\6
\4${}\}{}$\2\6
\4${}\}{}$\2\6
\&{if} ${}(\R\|j){}$\1\5
${}\\{bugs}\PP,\39\\{printf}(\.{"Bond\ \%d\ doesn't\ des}\)\.{ignate\ any\ of\
its\ ch}\)\.{ildren\\n"},\39\\{bn}(\|b));{}$\2\6
\&{else} \&{if} ${}(\|b\MG\\{done}\I\|b\MG\\{des}\MG\\{lsib}){}$\1\5
${}\\{bugs}\PP,\39\\{printf}(\.{"Bond\ \%d\ should\ be\ d}\)\.{one\ at\ \%d%
\\n"},\39\\{bn}(\|b),\39\\{bn}(\|b\MG\\{des}\MG\\{lsib}));{}$\2\6
\&{if} ${}(\|b\MG\\{scope}\I\\{extent}){}$\1\5
${}\\{bugs}\PP,\39\\{printf}(\.{"Bond\ \%d\ should\ have}\)\.{\ scope\ \%d%
\\n"},\39\\{bn}(\|b),\39\\{bn}(\\{extent}));{}$\2\6
\&{return} \\{bugs};\6
\4${}\}{}$\2\par
\U12.\fi

\M{17}If \PB{$\\{flags}\AND\T{1}$}, we've computed the number \PB{\|n} of
current vertices.

\Y\B\4\X17:Do a sanity check on the tree list\X${}\E{}$\6
${}\{{}$\1\6
\&{if} ${}(\\{tree}\MG\\{up}\MG\\{down}\I\\{tree}\V\\{tree}\MG\\{down}\MG\\{up}%
\I\\{tree}){}$\1\5
${}\\{bugs}\PP,\39\\{printf}(\.{"Link\ failure\ at\ hea}\)\.{d\ of\ tree\ list%
\\n"});{}$\2\6
\&{for} ${}(\|b\K\\{tree}\MG\\{down},\39\|k\K\T{0};{}$ ${}\|b\I\\{tree};{}$ ${}%
\|k\PP,\39\|b\K\|b\MG\\{down}){}$\5
${}\{{}$\1\6
\&{if} ${}(\|b\MG\\{up}\MG\\{down}\I\|b\V\|b\MG\\{down}\MG\\{up}\I\|b){}$\1\5
${}\\{bugs}\PP,\39\\{printf}(\.{"Link\ failure\ in\ tre}\)\.{e\ list\ at\ bond\
\%d\\n"},\39\\{bn}(\|b));{}$\2\6
\&{if} ${}(\|b\MG\\{val}\I\T{1}){}$\1\5
${}\\{bugs}\PP,\39\\{printf}(\.{"Bond\ \%d\ in\ the\ tree}\)\.{\ list\ has\
value\ 0\\n"},\39\\{bn}(\|b));{}$\2\6
\4${}\}{}$\2\6
\&{if} ${}((\\{flags}\AND\T{1})\W(\|k\I\|n-\T{1}-(\\{inbond}\E\NULL))){}$\1\5
${}\\{bugs}\PP,\39\\{printf}(\.{"The\ tree\ list\ holds}\)\.{\ \%d\ bonds,\ not%
\ \%d\\n"},\39\|k,\39\|n-\T{1}-(\\{inbond}\E\NULL));{}$\2\6
\4${}\}{}$\2\par
\U12.\fi

\N{1}{18}Graph preparation. At the beginning, we want to make sure that the
given graph is truly undirected, and that we can find mates of its
edges using the infamous \PB{\\{edge\_trick}}. We also need to change its
representation from singly linked edges to doubly linked bonds,
and to compute vertex degrees,
as well as to find an initial spanning tree.

All of these things can be done easily with a single
search of the graph. The following program sets \PB{$\|v\MG\\{mark}\K\T{2}$}
for each
vertex that has been seen, and keeps a sequential stack of vertices
that have been seen but not yet explored. (The search has aspects
of both bread-first and depth-first approaches: When we explore a vertex
we see all of its successors before exploring another, but we select
new vertices for exploration in the last-seen-first-explored manner.)

Why is the \PB{\\{mark}} field set to 2? Because any nonzero value will do,
and it turns out that we'll want to set it to~2 shortly after
this part of the program is done.

\Y\B\4\D$\\{stack}(\|k)$ \5
$(\|g\MG\\{vertices}+(\|k))\MG\|z.{}$\|V\C{ utility field \PB{\|z} is used for
a stack }\par
\B\4\D$\\{mark}$ \5
$\|v.{}$\|I\C{ utility field \PB{\|v} of each vertex holds a mark }\par
\Y\B\4\X18:Check the graph for validity and prepare it for action\X${}\E{}$\6
\&{for} ${}(\|v\K\|g\MG\\{vertices}+\T{1};{}$ ${}\|v\Z\|g\MG\\{vertices}+\|g\MG%
\|n;{}$ ${}\|v\PP){}$\1\5
${}\|v\MG\\{mark}\K\T{0};{}$\2\6
\&{if} (\\{verbose})\1\5
${}\\{printf}(\.{"Graph\ \%s\ has\ the\ fo}\)\.{llowing\ edges:\\n"},\39\|g\MG%
\\{id});{}$\2\6
${}\|v\K\|g\MG\\{vertices},\39\\{stack}(\T{0})\K\|v,\39\|k\K\T{1},\39\|v\MG%
\\{mark}\K\T{2},\39\|n\K\T{1},\39\|m\K\T{0};{}$\6
\&{while} (\|k)\5
${}\{{}$\C{ $k$ is the number of items on the stack }\1\6
${}\|o,\39\|v\K\\{stack}(\MM\|k);{}$\6
${}\|f\K\\{gb\_virgin\_arc}(\,);{}$\6
${}\|f\MG\\{next}\K\|v\MG\\{arcs}{}$;\C{ the new header node }\6
\&{for} ${}(\|v\MG\\{deg}\K\T{0},\39\|e\K\|v\MG\\{arcs},\39\|v\MG\\{arcs}\K%
\|f;{}$ \|e; ${}\|v\MG\\{deg}\PP,\39\|f\K\|e,\39\|e\K\|e\MG\\{next}){}$\5
${}\{{}$\1\6
\4\\{looky}:\5
${}\|u\K\|e\MG\\{tip};{}$\6
\&{if} ${}(\|u\E\|v){}$\5
${}\{{}$\C{ self-loops are silently ignored }\1\6
${}\|f\MG\\{next}\K\|e\K\|e\MG\\{next};{}$\6
\&{if} ${}(\R\|e){}$\1\5
\&{break};\2\6
\&{goto} \\{looky};\6
\4${}\}{}$\2\6
${}\|e\MG\\{prev}\K\|f;{}$\6
\&{if} ${}(\\{mate}(\|e)\MG\\{tip}\I\|v){}$\5
${}\{{}$\1\6
${}\\{fprintf}(\\{stderr},\39\.{"Graph\ \%s\ has\ an\ arc}\)\.{\ from\ \%s\ to\
\%s,\\n"},\39\|g\MG\\{id},\39\|u\MG\\{name},\39\|v\MG\\{name});{}$\6
${}\\{fprintf}(\\{stderr},\39\.{"\ but\ the\ edge\ trick}\)\.{\ doesn't\ find\
the\ op}\)\.{posite\ arc!\\n"});{}$\6
${}\\{exit}({-}\T{3});{}$\6
\4${}\}{}$\2\6
\&{if} ${}(\|o,\39\R\|e\MG\\{bond}){}$\5
${}\{{}$\1\6
${}\\{ooo},\39\|m\PP,\39\|b\K\\{bondbase}+\|m,\39\|e\MG\\{bond}\K\\{mate}(\|e)%
\MG\\{bond}\K{}$(\&{Arc} ${}{*}){}$ \|b${},\39\|b\MG\\{lhist}\K\|e;{}$\6
\&{if} (\\{verbose})\1\5
${}\\{printf}(\.{"\ \%d:\ \%s\ --\ \%s\\n"},\39\|m,\39\|v\MG\\{name},\39\|u\MG%
\\{name});{}$\2\6
\4${}\}{}$\2\6
\&{if} ${}(\|o,\39\R\|u\MG\\{mark}){}$\5
${}\{{}$\1\6
${}\\{ooo},\39\|u\MG\\{mark}\K\T{2},\39\\{stack}(\|k\PP)\K\|u,\39\|b\K{}$(%
\&{Bond} ${}{*}){}$ \|e${}\MG\\{bond},\39\|b\MG\\{val}\K\T{1};{}$\6
${}\\{ooo},\39\|n\PP,\39\|b\MG\\{up}\K\\{tree}\MG\\{up},\39\\{tree}\MG\\{up}\MG%
\\{down}\K\|b,\39\\{tree}\MG\\{up}\K\|b,\39\|b\MG\\{down}\K\\{tree};{}$\6
\4${}\}{}$\2\6
\4${}\}{}$\2\6
${}\|v\MG\\{arcs}\MG\\{prev}\K\|f,\39\|f\MG\\{next}\K\|v\MG\\{arcs}{}$;\C{
complete the double linking }\6
\4${}\}{}$\2\6
\&{if} ${}(\|n<\|g\MG\|n){}$\5
${}\{{}$\1\6
${}\\{fprintf}(\\{stderr},\39\.{"Oops,\ the\ graph\ isn}\)\.{'t\ connected!%
\\n"});{}$\6
${}\\{exit}({-}\T{4});{}$\6
\4${}\}{}$\2\6
${}\|o,\39\\{topleaf}\K\\{bondbase}+\|m,\39\\{bondcount}\K\|m{}$;\par
\U2.\fi

\N{1}{19}Reduction. Let's start to write real code now. This program reaches
a crucial step when we get to the label called \PB{\\{reduce}}, corresponding
roughly to the point where {\mc GRAYSPAN} gets to its label called \\{enter}.

When \PB{\\{reduce}} is reached, we're in the following state:

\smallskip\item{1)} Bonds $a_1\ldots a_l$ of the current graph have
been shrunk, represented as the array \PB{\\{aa}(\T{1})} through \PB{\\{aa}(%
\|l)}.
After we've found all spanning trees in
the shrunken graph, we'll want to unshrink them.

\smallskip\item{2)} The \PB{\\{tree}} list specifies a set of bonds that form
a spanning tree on the current graph, if \PB{$\\{inbond}\I\NULL$},
or a near-spanning tree if \PB{$\\{inbond}\K\NULL$}. The next spanning tree
we generate is supposed to include all of those bonds.

\smallskip\item{3)} If the current graph contains any parallel edges,
they are adjacent to vertices \PB{\|v} for which \PB{$\|v\MG\\{mark}\K\T{2}$}.

\smallskip\item{4)} If the current graph contains any vertices \PB{\|v} of
degree
less than~3, we have \PB{$\|v\MG\\{mark}>\T{0}$}.

\smallskip\item{5)} All marked vertices appear on the stack, which currently
holds \PB{\|k} items.

\smallskip\item{6)} All current bonds and subbonds appear in locations
\PB{$\\{bondbase}+\T{1}$} through \PB{$\\{bondbase}+\\{bondcount}$}. The ones
created before
reaching level~\PB{\|l} are less than or equal to \PB{$\\{bondbase}+\\{bonds}(%
\|l)$}.

\smallskip\noindent
Our job is to zero out the marks, continuing to reduce the graph
either by forming more complex bonds or by shrinking bonds of
the tree list, until the graph finally becomes trivial.

\Y\B\4\D$\\{bonds}(\|l)$ \5
$(\|g\MG\\{vertices}+\|l)\MG\|y.{}$\|I\C{ utility field \PB{\|y} of the vertex
array }\par
\B\4\D$\\{aa}(\|l)$ \5
$(\|g\MG\\{vertices}+\|l)\MG\|x.{}$\|A\C{ utility field \PB{\|x} holds $a_l$ }%
\par
\Y\B\4\X19:Reduce the graph until it's trivial\X${}\E{}$\6
\4\\{reduce}:\5
\&{while} (\|k)\5
${}\{{}$\1\6
${}\|o,\39\|v\K\\{stack}(\MM\|k);{}$\6
\&{if} ${}(\|o,\39\|v\MG\\{mark}>\T{1}){}$\1\5
\X20:Parallelize duplicate bonds from \PB{\|v}\X;\2\6
\&{if} ${}(\|o,\39\|v\MG\\{deg}<\T{3}){}$\1\5
\X21:Eliminate \PB{\|v}, then \PB{\&{goto} \\{trivialgraph}} if only one vertex
is left\X\2\6
\&{else}\1\5
${}\|o,\39\|v\MG\\{mark}\K\T{0};{}$\2\6
\4${}\}{}$\2\6
\hbox{}\C{ now all relevant marks are zero and the graph still isn't trivial }\6
${}\|o,\39\|l\PP,\39\\{bonds}(\|l)\K\\{bondcount};{}$\6
\&{if} (\\{extraverbose})\1\5
${}\\{printf}(\.{"Entering\ level\ \%d\\n}\)\.{"},\39\|l);{}$\2\6
${}\\{oooo},\39\|b\K\\{tree}\MG\\{down},\39\\{tree}\MG\\{down}\K\|b\MG\\{down},%
\39\\{tree}\MG\\{down}\MG\\{up}\K\\{tree};{}$\6
${}\\{oo},\39\|e\K\|b\MG\\{lhist},\39\\{aa}(\|l)\K\|e{}$;\C{ we have \PB{$\|b%
\MG\\{lhist}\MG\\{bond}\K{}$(\&{Arc} ${}{*}){}$ \|b} }\6
\X24:Shrink bond \PB{\|e}\X;\6
\&{goto} \\{reduce};\par
\U25.\fi

\M{20}\B\D$\\{dup}$ \5
$\|w.{}$\|A\C{ utility field \PB{\|w} points to a previous edge }\par
\Y\B\4\X20:Parallelize duplicate bonds from \PB{\|v}\X${}\E{}$\6
${}\{{}$\1\6
\&{for} ${}(\\{oo},\39\|e\K\|v\MG\\{arcs}\MG\\{next};{}$ \|o${},\39\|e\MG%
\\{tip};{}$ \|o${},\39\|e\K\|e\MG\\{next}){}$\1\5
${}\|o,\39\|e\MG\\{tip}\MG\\{dup}\K\NULL;{}$\2\6
\&{for} ${}(\|e\K\|v\MG\\{arcs}\MG\\{next};{}$ \|o${},\39\|e\MG\\{tip};{}$ %
\|o${},\39\|e\K\|e\MG\\{next}){}$\5
${}\{{}$\1\6
${}\|u\K\|e\MG\\{tip};{}$\6
\&{if} ${}(\|o,\39\|u\MG\\{dup}){}$\5
${}\{{}$\1\6
${}\\{makeparallel}(\|u\MG\\{dup},\39\|e){}$;\C{ create a new parallel bond }\6
\&{if} ${}(\|o,\39\R\|u\MG\\{mark}){}$\1\5
${}\\{oo},\39\|u\MG\\{mark}\K\T{1},\39\\{stack}(\|k\PP)\K\|u;{}$\2\6
\4${}\}{}$\5
\2\&{else}\1\5
${}\|o,\39\|u\MG\\{dup}\K\|e;{}$\2\6
\4${}\}{}$\2\6
\4${}\}{}$\2\par
\U19.\fi

\M{21}A deleted vertex is marked $-1$, for debugging purposes only.

\Y\B\4\X21:Eliminate \PB{\|v}, then \PB{\&{goto} \\{trivialgraph}} if only one
vertex is left\X${}\E{}$\6
${}\{{}$\1\6
${}\|v\MG\\{mark}\K{-}\T{1};{}$\6
\&{if} ${}(\|v\MG\\{deg}\E\T{2}){}$\1\5
\X23:Serialize the bonds from \PB{\|v}\X\2\6
\&{else} \&{if} ${}(\|v\MG\\{deg}\E\T{1}){}$\1\5
\X22:Require the bond from \PB{\|v}\X\2\6
\&{else}\5
${}\{{}$\1\6
${}\|v\MG\\{mark}\K\T{0}{}$;\C{ the last vertex doesn't go away }\6
\&{goto} \\{trivialgraph};\C{ \PB{$\|v\MG\\{deg}\K\T{0}$}, hence no other
vertices remain }\6
\4${}\}{}$\2\6
\4${}\}{}$\2\par
\U19.\fi

\M{22}If the single bond touching \PB{\|v} isn't in the tree list,
we know that the tree list must specify only a near-spanning tree.
So we set \PB{\\{inbond}}, which will complete a spanning tree later.
And we eliminate \PB{\|v}; thus the tree list henceforth is a spanning
tree on the remaining vertices.

\Y\B\4\X22:Require the bond from \PB{\|v}\X${}\E{}$\6
${}\{{}$\1\6
${}\\{ooo},\39\|e\K\|v\MG\\{arcs}\MG\\{next},\39\|b\K{}$(\&{Bond} ${}{*}){}$ %
\|e${}\MG\\{bond};{}$\6
${}\|o,\39\|u\K\|e\MG\\{tip};{}$\6
${}\\{deletebonds}(\\{mate}(\|e),\39\NULL){}$;\C{ remove \PB{\\{mate}(\|e)} and
decrease \PB{$\|u\MG\\{deg}$} }\6
\&{if} ${}(\|o,\39\|b\MG\\{val}){}$\1\5
${}\\{oooo},\39\|b\MG\\{up}\MG\\{down}\K\|b\MG\\{down},\39\|b\MG\\{down}\MG%
\\{up}\K\|b\MG\\{up};{}$\2\6
\&{else}\5
${}\{{}$\1\6
\&{if} (\\{inbond})\5
${}\{{}$\1\6
${}\\{fprintf}(\\{stderr},\39\.{"I've\ goofed\ (inbond}\)\.{\ doubly\ set)!%
\\n"});{}$\6
${}\\{exit}({-}\T{5});{}$\6
\4${}\}{}$\2\6
${}\\{inbond}\K\|b;{}$\6
\4${}\}{}$\2\6
\&{if} ${}(\|o,\39\R\|u\MG\\{mark}){}$\1\5
${}\\{oo},\39\|u\MG\\{mark}\K\T{1},\39\\{stack}(\|k\PP)\K\|u;{}$\2\6
\4${}\}{}$\2\par
\U21.\fi

\M{23}A subtle point arises here: We might be serializing two bonds not
in the spanning tree, if \PB{\|v} happens to be the only vertex not reachable
from the current near-spanning tree. In that case we want to set
\PB{\\{inbond}} to the subbond of the new series bond that will {\it not\/} be
designated by \PB{\\{makeseries}}.

\Y\B\4\X23:Serialize the bonds from \PB{\|v}\X${}\E{}$\6
${}\{{}$\1\6
${}\\{oooooo},\39\|e\K\|v\MG\\{arcs}\MG\\{next},\39\|f\K\|e\MG\\{next},\39\|u\K%
\|e\MG\\{tip},\39\|w\K\|f\MG\\{tip},\39\|b\K{}$(\&{Bond} ${}{*}){}$ \|e${}\MG%
\\{bond};{}$\6
\&{if} ${}(\|o,\39\R\|b\MG\\{val}){}$\5
${}\{{}$\1\6
${}\|o,\39\|b\K{}$(\&{Bond} ${}{*}){}$ \|f${}\MG\\{bond};{}$\6
\&{if} ${}(\|o,\39\R\|b\MG\\{val}){}$\5
${}\{{}$\1\6
\&{if} (\\{inbond})\5
${}\{{}$\1\6
${}\\{fprintf}(\\{stderr},\39\.{"I've\ doubly\ goofed\ }\)\.{(inbond\ set)!%
\\n"});{}$\6
${}\\{exit}({-}\T{6});{}$\6
\4${}\}{}$\2\6
${}\\{inbond}\K\|b;{}$\6
\4${}\}{}$\2\6
\4${}\}{}$\2\6
${}\\{makeseries}(\|e,\39\|f){}$;\C{ create a new series bond }\6
\&{if} ${}(\|o,\39\|u\MG\\{mark}){}$\1\5
${}\|o,\39\|u\MG\\{mark}\K\T{2};{}$\2\6
\&{else}\1\5
${}\\{oo},\39\|u\MG\\{mark}\K\T{2},\39\\{stack}(\|k\PP)\K\|u;{}$\2\6
\&{if} ${}(\|o,\39\|w\MG\\{mark}){}$\1\5
${}\|o,\39\|w\MG\\{mark}\K\T{2};{}$\2\6
\&{else}\1\5
${}\\{oo},\39\|w\MG\\{mark}\K\T{2},\39\\{stack}(\|k\PP)\K\|w;{}$\2\6
\4${}\}{}$\2\par
\U21.\fi

\M{24}At this point the graph is irreducible, so \PB{\|v} appears only once
in \PB{\|u}'s list.

\Y\B\4\X24:Shrink bond \PB{\|e}\X${}\E{}$\6
$\\{oo},\39\|u\K\|e\MG\\{tip},\39\|v\K\\{mate}(\|e)\MG\\{tip};{}$\6
\&{for} ${}(\\{oo},\39\|f\K\|u\MG\\{arcs}\MG\\{next};{}$ \|o${},\39\|f\MG%
\\{tip};{}$ \|o${},\39\|f\K\|f\MG\\{next}){}$\1\6
\&{if} ${}(\|f\E\\{mate}(\|e)){}$\1\5
\\{delete}(\|f);\2\6
\&{else}\1\5
${}\|o,\39\\{mate}(\|f)\MG\\{tip}\K\|v;{}$\2\2\6
\\{delete}(\|e);\6
${}\\{ooo},\39\|v\MG\\{deg}\MRL{+{\K}}\|u\MG\\{deg}-\T{2};{}$\6
${}\|o,\39\\{ee}\K\|v\MG\\{arcs}{}$;\C{ now \PB{$\|f\K\|u\MG\\{arcs}$} }\6
${}\\{oooo},\39\|f\MG\\{prev}\MG\\{next}\K\\{ee}\MG\\{next},\39\\{ee}\MG%
\\{next}\MG\\{prev}\K\|f\MG\\{prev};{}$\6
${}\\{ooo},\39\|f\MG\\{next}\MG\\{prev}\K\\{ee},\39\\{ee}\MG\\{next}\K\|f\MG%
\\{next};{}$\6
\&{if} (\\{extraverbose})\1\5
${}\\{printf}(\.{"\ shrinking\ \%d;\ now\ }\)\.{\%s\ has\ degree\ \%d\\n"},\39%
\\{ebn}(\|e),\39\|v\MG\\{name},\39\|v\MG\\{deg});{}$\2\6
${}\\{oo},\39\|u\MG\\{mark}\K{-}\T{1},\39\|v\MG\\{mark}\K\T{2},\39\|k\K\T{1},%
\39\\{stack}(\T{0})\K\|v{}$;\par
\U19.\fi

\N{1}{25}The main algorithm. Now that we understand reduction, we're ready to
complete the {\mc GRAYSPAN} portion of this program, except for low-level
details.

\Y\B\4\X25:Generate all spanning trees\X${}\E{}$\6
${}\{{}$\1\6
\&{for} ${}(\|k\K\T{0};{}$ ${}\|k<\|n;{}$ ${}\|k\PP){}$\1\5
${}\|o,\39\\{stack}(\|k)\K\|g\MG\\{vertices}+\|k{}$;\C{ all vertices are
initially suspect }\2\6
${}\|o,\39\|b\K\\{tree}\MG\\{up},\39\|b\MG\\{val}\K\T{0}{}$;\C{ we delete the
last edge of the preliminary tree }\6
${}\\{oo},\39\\{tree}\MG\\{up}\K\|b\MG\\{up},\39\\{tree}\MG\\{up}\MG\\{down}\K%
\\{tree};{}$\6
\&{if} (\\{verbose})\5
${}\{{}$\1\6
\\{printf}(\.{"Start\ with\ the\ near}\)\.{-spanning\ edges"});\6
\&{for} ${}(\|b\K\\{tree}\MG\\{down};{}$ ${}\|b\I\\{tree};{}$ ${}\|b\K\|b\MG%
\\{down}){}$\1\5
${}\\{printf}(\.{"\ \%d"},\39\\{bn}(\|b));{}$\2\6
\\{printf}(\.{"\\n"});\6
\4${}\}{}$\2\6
${}\|l\K\T{0};{}$\6
\X19:Reduce the graph until it's trivial\X;\6
\4\\{trivialgraph}:\5
\X30:Obtain a new spanning tree by changing \PB{\\{outbond}} and \PB{%
\\{inbond}}\X;\6
\X47:Do the {\mc SPSPAN} algorithm on the action list\X;\6
\&{while} (\|l)\5
${}\{{}$\1\6
\X29:Undo all changes to the graph since entering level \PB{\|l}\X;\6
${}\|o,\39\|e\K\\{aa}(\|l);{}$\6
\X26:Unshrink bond \PB{\|e}\X;\6
${}\|l\MM;{}$\6
\X27:If \PB{\|e} is not a bridge, delete it, set \PB{$\\{outbond}\K\|e\MG%
\\{bond}$}, and \PB{\&{goto} \\{reduce}}\X;\6
${}\\{oooooo},\39\|b\K{}$(\&{Bond} ${}{*}){}$ \|e${}\MG\\{bond},\39\|b\MG\\{up}%
\K\\{tree}\MG\\{up},\39\\{tree}\MG\\{up}\K\|b\MG\\{up}\MG\\{down}\K\|b,\39\|b%
\MG\\{down}\K\\{tree};{}$\6
\4${}\}{}$\2\6
\4${}\}{}$\2\par
\U1.\fi

\M{26}After unshrinking, the graph will still be irreducible.

\Y\B\4\X26:Unshrink bond \PB{\|e}\X${}\E{}$\6
$\\{ooooo},\39\|u\K\|e\MG\\{tip},\39\|v\K\\{mate}(\|e)\MG\\{tip},\39\|v\MG%
\\{deg}\MRL{-{\K}}\|u\MG\\{deg}-\T{2};{}$\6
${}\\{oo},\39\\{ee}\K\|v\MG\\{arcs},\39\|f\K\|u\MG\\{arcs};{}$\6
${}\\{oooo},\39\\{ee}\MG\\{next}\K\|f\MG\\{prev}\MG\\{next},\39\\{ee}\MG%
\\{next}\MG\\{prev}\K\\{ee};{}$\6
${}\\{ooo},\39\|f\MG\\{prev}\MG\\{next}\K\|f,\39\|f\MG\\{next}\MG\\{prev}\K%
\|f;{}$\6
\&{for} ${}(\|o,\39\|f\K\|f\MG\\{next};{}$ \|o${},\39\|f\MG\\{tip};{}$ \|o${},%
\39\|f\K\|f\MG\\{next}){}$\1\5
${}\|o,\39\\{mate}(\|f)\MG\\{tip}\K\|u;{}$\2\6
${}\\{undelete}(\|e),\39\\{undelete}(\\{mate}(\|e));{}$\6
\&{if} (\\{extraverbose})\1\5
${}\\{printf}(\.{"\ unshrinking\ \%d;\ no}\)\.{w\ \%s\ has\ degree\ \%d\\n}\)%
\.{"},\39\\{ebn}(\|e),\39\|v\MG\\{name},\39\|v\MG\\{deg});{}$\2\6
${}\|u\MG\\{mark}\K\T{0}{}$;\C{ it was $-1$ }\par
\U25.\fi

\M{27}We use a field \PB{\\{bfs}} that shares space with \PB{\\{mark}}, because
\PB{\\{mark}}
is zero in all relevant vertices at this time.

\Y\B\4\D$\\{bfs}$ \5
$\|v.{}$\|V\C{ link for the breadth-first search: nonnull if vertex seen }\par
\Y\B\4\X27:If \PB{\|e} is not a bridge, delete it, set \PB{$\\{outbond}\K\|e\MG%
\\{bond}$}, and \PB{\&{goto} \\{reduce}}\X${}\E{}$\6
$\\{oo},\39\|u\K\|e\MG\\{tip},\39\|v\K\\{mate}(\|e)\MG\\{tip};{}$\6
\&{for} ${}(\|o,\39\|u\MG\\{bfs}\K\|v,\39\|w\K\|u;{}$ ${}\|u\I\|v;{}$ \|o${},%
\39\|u\K\|u\MG\\{bfs}){}$\5
${}\{{}$\1\6
\&{for} ${}(\\{oo},\39\|f\K\|u\MG\\{arcs}\MG\\{next};{}$ \|o${},\39\|f\MG%
\\{tip};{}$ \|o${},\39\|f\K\|f\MG\\{next}){}$\1\6
\&{if} ${}(\|o,\39\|f\MG\\{tip}\MG\\{bfs}\E\NULL){}$\5
${}\{{}$\1\6
\&{if} ${}(\|f\MG\\{tip}\E\|v){}$\5
${}\{{}$\1\6
\&{if} ${}(\|f\I\\{mate}(\|e)){}$\1\5
\X28:Nullify all \PB{\\{bfs}} links, delete \PB{\|e}, and \PB{\&{goto} %
\\{reduce}}\X;\2\6
\4${}\}{}$\5
\2\&{else}\1\5
${}\\{oo},\39\|f\MG\\{tip}\MG\\{bfs}\K\|v,\39\|w\MG\\{bfs}\K\|f\MG\\{tip},\39%
\|w\K\|f\MG\\{tip};{}$\2\6
\4${}\}{}$\2\2\6
\4${}\}{}$\2\6
\&{if} (\\{extraverbose})\1\5
${}\\{printf}(\.{"Leaving\ level\ \%d:\ \%}\)\.{d\ is\ a\ bridge\\n"},\39\|l+%
\T{1},\39\\{ebn}(\|e));{}$\2\6
\&{for} ${}(\|o,\39\|u\K\|e\MG\\{tip};{}$ ${}\|u\I\|v;{}$ \|o${},\39\|u\MG%
\\{bfs}\K\NULL,\39\|u\K\|w){}$\1\5
${}\|o,\39\|w\K\|u\MG\\{bfs}{}$;\2\par
\U25.\fi

\M{28}We have discovered that \PB{\|e} is not a bridge.

\Y\B\4\X28:Nullify all \PB{\\{bfs}} links, delete \PB{\|e}, and \PB{\&{goto} %
\\{reduce}}\X${}\E{}$\6
${}\{{}$\1\6
\&{for} ${}(\|o,\39\|u\K\|e\MG\\{tip};{}$ ${}\|u\I\|v;{}$ \|o${},\39\|u\MG%
\\{bfs}\K\NULL,\39\|u\K\|w){}$\1\5
${}\|o,\39\|w\K\|u\MG\\{bfs};{}$\2\6
${}\\{outbond}\K{}$(\&{Bond} ${}{*})(\|e\MG\\{bond});{}$\6
${}\\{deletebonds}(\|e,\39\\{mate}(\|e));{}$\6
${}\\{oooo},\39\|k\K\T{2},\39\\{stack}(\T{0})\K\|e\MG\\{tip},\39\\{stack}(%
\T{1})\K\\{mate}(\|e)\MG\\{tip};{}$\6
${}\\{oo},\39\|e\MG\\{tip}\MG\\{mark}\K\\{mate}(\|e)\MG\\{tip}\MG\\{mark}\K%
\T{1};{}$\6
\&{goto} \\{reduce};\6
\4${}\}{}$\2\par
\U27.\fi

\M{29}\B\X29:Undo all changes to the graph since entering level \PB{\|l}\X${}%
\E{}$\6
\|o;\5
\&{while} ${}(\\{bondcount}>\\{bonds}(\|l)){}$\1\5
\\{unbuildbond}(\,);\2\par
\U25.\fi

\M{30}When the program reaches \PB{\\{trivialgraph}}, the variables \PB{%
\\{outbond}}
and \PB{\\{inbond}} point to bonds whose values are to become 0 and~1,
respectively, in the next spanning tree.

(Exception: On the first
iteration, \PB{\\{outbond}} is null and we've already printed $n-2$ edges
of the first spanning tree, as sort of a pump-priming process.)

Note that \PB{\\{inbond}} might not be a top-level bond, because of the
subtle point mentioned earlier. But \PB{\\{outbond}} always at the top level,
because it was removed from the graph when \PB{\\{outbond}} was set.

\Y\B\4\X30:Obtain a new spanning tree by changing \PB{\\{outbond}} and \PB{%
\\{inbond}}\X${}\E{}$\6
$\\{count}\PP;{}$\6
\&{if} (\\{verbose})\1\5
${}\\{printf}(\.{"\%.15g:"},\39\\{count});{}$\2\6
\&{if} (\\{outbond})\5
${}\{{}$\1\6
\&{for} ${}(\|b\K\\{outbond};{}$  ; \|o${},\39\|b\K\|b\MG\\{des}){}$\5
${}\{{}$\1\6
${}\|o,\39\|b\MG\\{val}\K\T{0};{}$\6
\&{if} (\\{isleaf}(\|b))\1\5
\&{break};\2\6
\4${}\}{}$\2\6
\&{if} (\\{verbose})\1\5
${}\\{printf}(\.{"-\%d"},\39\\{bn}(\|b));{}$\2\6
\4${}\}{}$\2\6
\&{if} ${}(\R\\{inbond}){}$\5
${}\{{}$\1\6
${}\\{fprintf}(\\{stderr},\39\.{"Internal\ error\ (no\ }\)\.{inbond)!\\n"});{}$%
\6
${}\\{exit}({-}\T{7});{}$\6
\4${}\}{}$\2\6
\&{for} ${}(\|b\K\\{inbond};{}$  ; \|o${},\39\|b\K\|b\MG\\{des}){}$\5
${}\{{}$\1\6
${}\|o,\39\|b\MG\\{val}\K\T{1};{}$\6
\&{if} (\\{isleaf}(\|b))\1\5
\&{break};\2\6
\4${}\}{}$\2\6
\&{if} (\\{verbose})\5
${}\{{}$\1\6
${}\\{printf}(\.{"+\%d"},\39\\{bn}(\|b));{}$\6
\&{if} (\\{extraverbose})\5
${}\{{}$\1\6
\\{printf}(\.{"\ ("});\6
\&{for} ${}(\|b\K\\{bondbase}+\T{1};{}$ ${}\|b\Z\\{topleaf};{}$ ${}\|b\PP){}$\1%
\6
\&{if} ${}(\|b\MG\\{val}){}$\1\5
${}\\{printf}(\.{"\ \%d"},\39\\{bn}(\|b));{}$\2\2\6
\\{printf}(\.{"\ )\\n"});\6
\4${}\}{}$\5
\2\&{else}\1\5
\\{printf}(\.{"\\n"});\2\6
\4${}\}{}$\2\6
${}\\{inbond}\K\\{outbond}\K\NULL{}$;\par
\U25.\fi

\N{1}{31}Construction. Reducing the main graph means constructing more bonds.

The down side of having a complex data structure is that
we have to do tedious maintenance work when conditions change. The following
part of the program was least fun to write, and it is the most likely
place where silly errors might have crept in. But I gritted my teeth
and I think I've gotten the job done. (As Anne Lamott would say,
this part of the program was written ``bird by bird.'')

The two subroutines \PB{\\{makeseries}} and \PB{\\{makeparallel}} are each
called
only once, so I needn't have made them subroutines. They do share a
lot of common code, however, so I've simplified my task by writing
a combined routine that handles both cases. Hopefully this will reduce
the chances of error. But mems are computed as if the subroutine had
been expanded inline and customized for the cases \PB{$\|t\K\T{0}$} and \PB{$%
\|t\K\T{1}$}.

\Y\B\4\D$\\{makeseries}(\|e,\|f)$ \5
$\\{buildbond}(\T{1},\39\|e,\39\|f{}$)\par
\B\4\D$\\{makeparallel}(\|e,\|f)$ \5
$\\{buildbond}(\T{0},\39\|e,\39\|f{}$)\par
\Y\B\4\X9:Subroutines\X${}\mathrel+\E{}$\6
\&{void} \\{buildbond}(\&{int} \|t${},\39{}$\&{Arc} ${}{*}\\{aa},\39{}$\&{Arc}
${}{*}\\{bb}){}$\1\1\2\2\6
${}\{{}$\1\6
\&{register} \&{Bond} ${}{*}\|a,{}$ ${}{*}\|b,{}$ ${}{*}\|c,{}$ ${}{*}\|d;{}$\6
\&{register} \&{int} \\{at}${},{}$ \\{av}${},{}$ \\{bt}${},{}$ \\{bv};\6
\&{register} \&{Vertex} ${}{*}\|u,{}$ ${}{*}\|v;{}$\6
\&{register} \&{Arc} ${}{*}\\{ee}{}$;\C{ used by the \PB{\\{delete}} macro }\7
${}\\{bondcount}\PP,\39\|c\K\\{bondbase}+\\{bondcount};{}$\6
${}\\{oooo},\39\|c\MG\\{typ}\K\|t,\39\|c\MG\\{lhist}\K\\{aa},\39\|c\MG\\{rhist}%
\K\\{bb},\39\|c\MG\\{focus}\K\|c;{}$\6
${}\\{oo},\39\|a\K{}$(\&{Bond} ${}{*}){}$ \\{aa}${}\MG\\{bond},\39\|b\K{}$(%
\&{Bond} ${}{*}){}$ \\{bb}${}\MG\\{bond};{}$\6
${}\\{oo},\39\\{av}\K\|a\MG\\{val},\39\\{at}\K\|a\MG\\{typ},\39\\{bv}\K\|b\MG%
\\{val},\39\\{bt}\K\|b\MG\\{typ};{}$\6
\&{if} (\|t)\1\5
\X32:Update \PB{\\{aa}} and \PB{\\{bb}} for a new series bond\X\2\6
\&{else}\1\5
\X33:Update \PB{\\{aa}} and \PB{\\{bb}} for a new parallel bond\X;\2\6
${}\\{oo},\39\\{mate}(\\{bb})\MG\\{bond}\K\\{aa}\MG\\{bond}{}$;\C{ remember the
bond that's going away }\6
${}\\{oo},\39\\{aa}\MG\\{bond}\K\\{mate}(\\{aa})\MG\\{bond}\K{}$(\&{Arc}
${}{*}){}$ \|c;\C{ \PB{$\|c\MG\\{lhist}\MG\\{bond}$} points to \PB{\|c} }\6
\&{if} (\\{isleaf}(\|a))\1\6
\&{if} (\\{isleaf}(\|b))\1\5
\X34:Build leaf with leaf\X\2\6
\&{else}\1\5
\X35:Build leaf with branch\X\2\2\6
\&{else} \&{if} (\\{isleaf}(\|b))\1\5
\X36:Build branch with leaf\X\2\6
\&{else}\1\5
\X37:Build branch with branch\X;\2\6
\&{if} (\\{av})\1\5
${}\\{oooo},\39\|a\MG\\{up}\MG\\{down}\K\|a\MG\\{down},\39\|a\MG\\{down}\MG%
\\{up}\K\|a\MG\\{up};{}$\2\6
\&{if} (\\{bv})\1\5
${}\\{oooo},\39\|b\MG\\{up}\MG\\{down}\K\|b\MG\\{down},\39\|b\MG\\{down}\MG%
\\{up}\K\|b\MG\\{up};{}$\2\6
\&{if} ${}(\\{av}\E\\{bv}\V\\{bv}\E\|t){}$\1\6
\&{if} ${}(\\{isleaf}(\|a)\V\\{at}\I\|t){}$\1\5
${}\|o,\39\|c\MG\\{des}\K\|a;{}$\2\6
\&{else}\1\5
${}\\{oo},\39\|c\MG\\{des}\K\|a\MG\\{des};{}$\2\2\6
\&{else} \&{if} ${}(\\{isleaf}(\|b)\V\\{bt}\I\|t){}$\1\5
${}\|o,\39\|c\MG\\{des}\K\|b;{}$\2\6
\&{else}\1\5
${}\\{oo},\39\|c\MG\\{des}\K\|b\MG\\{des};{}$\2\6
${}\\{oooo},\39\|c\MG\\{val}\K\|c\MG\\{des}\MG\\{val},\39\|c\MG\\{done}\K\|c\MG%
\\{des}\MG\\{lsib};{}$\6
\&{if} ${}(\|c\MG\\{val}){}$\1\5
${}\\{ooooo},\39\|c\MG\\{up}\K\\{tree}\MG\\{up},\39\\{tree}\MG\\{up}\K\|c\MG%
\\{up}\MG\\{down}\K\|c,\39\|c\MG\\{down}\K\\{tree};{}$\2\6
\4${}\}{}$\2\par
\fi

\M{32}The new series bond has to agree with its mate. So we move
the arc node \PB{\\{aa}} from one list to another.

\Y\B\4\X32:Update \PB{\\{aa}} and \PB{\\{bb}} for a new series bond\X${}\E{}$\6
${}\{{}$\1\6
\&{if} (\\{extraverbose})\1\5
${}\\{printf}(\.{"\ \%d=series(\%d,\%d)\ b}\)\.{etween\ \%s\ and\ \%s\\n"},\39%
\\{bn}(\|c),\39\\{bn}(\|a),\39\\{bn}(\|b),\39\\{aa}\MG\\{tip}\MG\\{name},\39%
\\{bb}\MG\\{tip}\MG\\{name});{}$\2\6
${}\\{oooo},\39\\{ee}\K\\{mate}(\\{bb}),\39\\{aa}\MG\\{prev}\K\\{ee}\MG%
\\{prev},\39\\{aa}\MG\\{next}\K\\{ee}\MG\\{next};{}$\6
${}\\{oo},\39\\{aa}\MG\\{prev}\MG\\{next}\K\\{aa}\MG\\{next}\MG\\{prev}\K%
\\{aa};{}$\6
${}\\{oo},\39\\{mate}(\\{aa})\MG\\{tip}\K\\{bb}\MG\\{tip};{}$\6
\4${}\}{}$\2\par
\U31.\fi

\M{33}\B\X33:Update \PB{\\{aa}} and \PB{\\{bb}} for a new parallel bond\X${}%
\E{}$\6
${}\{{}$\1\6
${}\\{oo},\39\|u\K\\{aa}\MG\\{tip},\39\|v\K\\{mate}(\\{aa})\MG\\{tip};{}$\6
\&{if} (\\{extraverbose})\1\5
${}\\{printf}(\.{"\ \%d=parallel(\%d,\%d)}\)\.{\ between\ \%s\ and\ \%s\\n}\)%
\.{"},\39\\{bn}(\|c),\39\\{bn}(\|a),\39\\{bn}(\|b),\39\|u\MG\\{name},\39\|v\MG%
\\{name});{}$\2\6
${}\\{oooo},\39\\{delete}(\\{bb}),\39\\{delete}(\\{mate}(\\{bb})),\39\|u\MG%
\\{deg}\MM,\39\|v\MG\\{deg}\MM;{}$\6
\4${}\}{}$\2\par
\U31.\fi

\M{34}\B\X34:Build leaf with leaf\X${}\E{}$\6
${}\{{}$\1\6
${}\\{oooo},\39\|a\MG\\{lsib}\K\|a\MG\\{rsib}\K\|b,\39\|b\MG\\{lsib}\K\|b\MG%
\\{rsib}\K\|a;{}$\6
${}\|o,\39\|c\MG\\{lchild}\K\|a;{}$\6
${}\\{ooo},\39\|c\MG\\{right}\K\\{action}\MG\\{right},\39\|c\MG\\{right}\MG%
\\{left}\K\|c;{}$\6
${}\\{oo},\39\|c\MG\\{left}\K\\{action},\39\\{action}\MG\\{right}\K\|c;{}$\6
${}\|o,\39\|c\MG\\{scope}\K\|c;{}$\6
\4${}\}{}$\2\par
\U31.\fi

\M{35}\B\X35:Build leaf with branch\X${}\E{}$\6
${}\{{}$\1\6
\&{if} ${}(\\{bt}\I\|t){}$\5
${}\{{}$\1\6
${}\\{oooo},\39\|a\MG\\{lsib}\K\|a\MG\\{rsib}\K\|b,\39\|b\MG\\{lsib}\K\|b\MG%
\\{rsib}\K\|a;{}$\6
${}\\{ooo},\39\|c\MG\\{right}\K\|b,\39\|c\MG\\{scope}\K\|b\MG\\{scope};{}$\6
\4${}\}{}$\5
\2\&{else}\5
${}\{{}$\1\6
${}\\{oooo},\39\|a\MG\\{rsib}\K\|b\MG\\{lchild},\39\|a\MG\\{lsib}\K\|b\MG%
\\{lchild}\MG\\{lsib};{}$\6
${}\\{oo},\39\|a\MG\\{rsib}\MG\\{lsib}\K\|a\MG\\{lsib}\MG\\{rsib}\K\|a;{}$\6
${}\\{oooo},\39\|c\MG\\{right}\K\|b\MG\\{right},\39\|c\MG\\{scope}\K(\|b\MG%
\\{scope}\E\|b\?\|c:\|b\MG\\{scope});{}$\6
\4${}\}{}$\2\6
${}\|o,\39\|c\MG\\{lchild}\K\|a;{}$\6
${}\\{oo},\39\|c\MG\\{left}\K\|b\MG\\{left};{}$\6
${}\\{oo},\39\|c\MG\\{left}\MG\\{right}\K\|c\MG\\{right}\MG\\{left}\K\|c;{}$\6
\4${}\}{}$\2\par
\U31.\fi

\M{36}\B\X36:Build branch with leaf\X${}\E{}$\6
${}\{{}$\1\6
\&{if} ${}(\\{at}\I\|t){}$\5
${}\{{}$\1\6
${}\\{oooo},\39\|a\MG\\{lsib}\K\|a\MG\\{rsib}\K\|b,\39\|b\MG\\{lsib}\K\|b\MG%
\\{rsib}\K\|a;{}$\6
${}\\{oooo},\39\|c\MG\\{right}\K\|c\MG\\{lchild}\K\|a,\39\|c\MG\\{scope}\K\|a%
\MG\\{scope};{}$\6
\4${}\}{}$\5
\2\&{else}\5
${}\{{}$\1\6
${}\\{oooo},\39\|b\MG\\{rsib}\K\|a\MG\\{lchild},\39\|b\MG\\{lsib}\K\|a\MG%
\\{lchild}\MG\\{lsib};{}$\6
${}\\{oo},\39\|b\MG\\{rsib}\MG\\{lsib}\K\|b\MG\\{lsib}\MG\\{rsib}\K\|b;{}$\6
${}\\{oooo},\39\|c\MG\\{right}\K\|a\MG\\{right},\39\|c\MG\\{lchild}\K\|a\MG%
\\{lchild};{}$\6
${}\\{oo},\39\|c\MG\\{scope}\K(\|a\MG\\{scope}\E\|a\?\|c:\|a\MG\\{scope});{}$\6
\4${}\}{}$\2\6
${}\\{oo},\39\|c\MG\\{left}\K\|a\MG\\{left};{}$\6
${}\\{oo},\39\|c\MG\\{left}\MG\\{right}\K\|c\MG\\{right}\MG\\{left}\K\|c;{}$\6
\4${}\}{}$\2\par
\U31.\fi

\M{37}\B\X37:Build branch with branch\X${}\E{}$\6
${}\{{}$\1\6
${}\\{ooo},\39\|d\K\|a\MG\\{scope},\39\|c\MG\\{rsave}\K\|d\MG\\{right};{}$\6
${}\\{oooo},\39\|a\MG\\{left}\MG\\{right}\K\|d\MG\\{right},\39\|d\MG\\{right}%
\MG\\{left}\K\|a\MG\\{left};{}$\6
${}\\{ooo},\39\|c\MG\\{left}\K\|b\MG\\{left},\39\|c\MG\\{left}\MG\\{right}\K%
\|c;{}$\6
\&{if} ${}(\\{at}\I\|t){}$\5
${}\{{}$\1\6
${}\\{oo},\39\|c\MG\\{lsave}\K\|a\MG\\{left};{}$\6
${}\\{ooo},\39\|c\MG\\{lchild}\K\|a,\39\|c\MG\\{right}\K\|a,\39\|a\MG\\{left}\K%
\|c;{}$\6
\&{if} ${}(\\{bt}\I\|t){}$\5
${}\{{}$\1\6
${}\\{oooo},\39\|a\MG\\{lsib}\K\|a\MG\\{rsib}\K\|b,\39\|b\MG\\{lsib}\K\|b\MG%
\\{rsib}\K\|a;{}$\6
${}\\{oo},\39\|d\MG\\{right}\K\|b,\39\|b\MG\\{left}\K\|d;{}$\6
${}\\{ooo},\39\|c\MG\\{scope}\K\|b\MG\\{scope};{}$\6
\4${}\}{}$\5
\2\&{else}\5
${}\{{}$\1\6
${}\\{oooo},\39\|a\MG\\{rsib}\K\|b\MG\\{lchild},\39\|a\MG\\{lsib}\K\|b\MG%
\\{lchild}\MG\\{lsib};{}$\6
${}\\{oo},\39\|a\MG\\{lsib}\MG\\{rsib}\K\|a\MG\\{rsib}\MG\\{lsib}\K\|a;{}$\6
${}\\{ooo},\39\|d\MG\\{right}\K\|b\MG\\{right},\39\|d\MG\\{right}\MG\\{left}\K%
\|d;{}$\6
${}\\{oo},\39\|c\MG\\{scope}\K(\|b\MG\\{scope}\E\|b\?\|d:\|b\MG\\{scope});{}$\6
\4${}\}{}$\2\6
\4${}\}{}$\5
\2\&{else} \&{if} ${}(\\{bt}\I\|t){}$\5
${}\{{}$\1\6
${}\\{ooo},\39\|c\MG\\{lchild}\K\|b\MG\\{rsib}\K\|a\MG\\{lchild};{}$\6
${}\\{oo},\39\|b\MG\\{lsib}\K\|a\MG\\{lchild}\MG\\{lsib};{}$\6
${}\\{oo},\39\|b\MG\\{lsib}\MG\\{rsib}\K\|b\MG\\{rsib}\MG\\{lsib}\K\|b;{}$\6
\&{if} ${}(\|d\E\|a){}$\1\5
${}\|o,\39\|c\MG\\{right}\K\|b;{}$\2\6
\&{else}\1\5
${}\\{oooo},\39\|c\MG\\{right}\K\|a\MG\\{right},\39\|d\MG\\{right}\K\|b,\39\|b%
\MG\\{left}\K\|d;{}$\2\6
${}\|o,\39\|c\MG\\{right}\MG\\{left}\K\|c;{}$\6
${}\\{oo},\39\|c\MG\\{scope}\K\|b\MG\\{scope};{}$\6
\4${}\}{}$\5
\2\&{else}\5
${}\{{}$\1\6
\&{if} ${}(\|d\E\|a){}$\1\5
${}\\{oo},\39\|c\MG\\{right}\K\|b\MG\\{right};{}$\2\6
\&{else}\1\5
${}\\{ooooo},\39\|c\MG\\{right}\K\|a\MG\\{right},\39\|d\MG\\{right}\K\|b\MG%
\\{right},\39\|b\MG\\{right}\MG\\{left}\K\|d;{}$\2\6
${}\|o,\39\|c\MG\\{right}\MG\\{left}\K\|c;{}$\6
${}\\{oo},\39\|c\MG\\{lchild}\K\|a\MG\\{lchild};{}$\6
${}\\{oooo},\39\|d\K\|a\MG\\{lchild}\MG\\{lsib},\39\|a\MG\\{lchild}\MG\\{lsib}%
\K\|b\MG\\{lchild}\MG\\{lsib};{}$\6
${}\\{ooo},\39\|b\MG\\{lchild}\MG\\{lsib}\MG\\{rsib}\K\|a\MG\\{lchild},\39\|b%
\MG\\{lchild}\MG\\{lsib}\K\|d,\39\|d\MG\\{rsib}\K\|b\MG\\{lchild};{}$\6
${}\\{oo},\39\|c\MG\\{scope}\K(\|b\MG\\{scope}\E\|b\?(\|a\MG\\{scope}\E\|a\?%
\|c:\|a\MG\\{scope}):\|b\MG\\{scope});{}$\6
\4${}\}{}$\2\6
\4${}\}{}$\2\par
\U31.\fi

\M{38}A much simpler subroutine is used to record the fact that one or
two bonds are temporarily being deleted from the graph.

\Y\B\4\X9:Subroutines\X${}\mathrel+\E{}$\6
\&{void} \\{deletebonds}(\&{Arc} ${}{*}\\{aa},\39{}$\&{Arc} ${}{*}\\{bb}){}$\1%
\1\2\2\6
${}\{{}$\1\6
\&{register} \&{Bond} ${}{*}\|a,{}$ ${}{*}\|c;{}$\6
\&{register} \&{Vertex} ${}{*}\|u,{}$ ${}{*}\|v;{}$\6
\&{register} \&{Arc} ${}{*}\\{ee};{}$\7
${}\\{bondcount}\PP,\39\|c\K\\{bondbase}+\\{bondcount};{}$\6
${}\\{ooo},\39\|c\MG\\{typ}\K\T{2},\39\|c\MG\\{lhist}\K\\{aa},\39\|c\MG%
\\{rhist}\K\\{bb};{}$\6
${}\\{oo},\39\|u\K\\{mate}(\\{aa})\MG\\{tip},\39\|a\K{}$(\&{Bond} ${}{*}){}$ %
\\{aa}${}\MG\\{bond};{}$\6
${}\|o,\39\|v\K(\\{bb}\?\\{mate}(\\{bb})\MG\\{tip}:\\{aa}\MG\\{tip});{}$\6
\&{if} (\\{extraverbose})\1\5
${}\\{printf}(\.{"\ \%d:\ deleting\ bond\ }\)\.{\%d\ between\ \%s\ and\ \%s}\)%
\.{\%s\\n"},\39\\{bn}(\|c),\39\\{bn}(\|a),\39\|u\MG\\{name},\39\\{bb}\?\.{""}:%
\.{"endpoint\ "},\39\|v\MG\\{name});{}$\2\6
${}\\{oo},\39\\{delete}(\\{aa}),\39\|u\MG\\{deg}\MM;{}$\6
\&{if} (\\{bb})\1\5
${}\\{oo},\39\\{delete}(\\{bb}),\39\|v\MG\\{deg}\MM;{}$\2\6
\4${}\}{}$\2\par
\fi

\N{1}{39}Deconstruction. Every change to the graph after we first reach level~1
must be undone again before we finish the program. But fortunately
the changes are always made in a strictly last-done-first-undone
manner. Therefore we can use the ``dancing links'' principle to
exploit of the fact that pointers within deleted structures
still retain the values to which they should be restored.

Thus I could write \PB{\\{unbuildbond}} by just looking at the code for \PB{%
\\{buildbond}}
and unchanging everything that it changed.
(These remarks apply only to changes
to \PB{\\{prev}}, \PB{\\{next}}, \PB{\\{bond}}, \PB{\\{lchild}}, \PB{\\{lsib}},
\PB{\\{rsib}}, \PB{\\{scope}}, \PB{\\{left}},
and \PB{\\{right}}, which govern the state of the bonds. The \PB{\\{val}}, \PB{%
\\{des}}, \PB{\\{done}},
\PB{\\{up}}, and \PB{\\{down}} fields vary dynamically with the spanning tree
and
should not be restored blindly. The \PB{\\{focus}} fields always point to self
when
construction or deconstruction is being done.)

(The programming task still was tedious and error-prone though; sigh.)

\Y\B\4\X9:Subroutines\X${}\mathrel+\E{}$\6
\&{void} \\{unbuildbond}(\,)\1\1\2\2\6
${}\{{}$\1\6
\&{register} \&{Bond} ${}{*}\|a,{}$ ${}{*}\|b,{}$ ${}{*}\|c,{}$ ${}{*}\|d;{}$\6
\&{register} \&{int} \|t${},{}$ \\{at}${},{}$ \\{bt};\6
\&{register} \&{Vertex} ${}{*}\|u,{}$ ${}{*}\|v;{}$\6
\&{register} \&{Arc} ${}{*}\\{aa},{}$ ${}{*}\\{bb},{}$ ${}{*}\\{ee}{}$;\C{ used
by the \PB{\\{undelete}} macro }\7
${}\|c\K\\{bondbase}+\\{bondcount},\39\\{bondcount}\MM;{}$\6
${}\\{ooo},\39\|t\K\|c\MG\\{typ},\39\\{aa}\K\|c\MG\\{lhist},\39\\{bb}\K\|c\MG%
\\{rhist};{}$\6
\&{if} ${}(\|t>\T{1}){}$\1\5
\X46:Restore one or two deleted bonds and \PB{\&{return}}\X\2\6
${}\\{oo},\39\|a\K{}$(\&{Bond} ${}{*}){}$ \\{mate}(\\{bb})${}\MG\\{bond},\39\|b%
\K{}$(\&{Bond} ${}{*}){}$ \\{bb}${}\MG\\{bond};{}$\6
${}\|o,\39\\{mate}(\\{bb})\MG\\{bond}\K{}$(\&{Arc} ${}{*}){}$ \|b;\6
${}\\{oo},\39\\{aa}\MG\\{bond}\K\\{mate}(\\{aa})\MG\\{bond}\K{}$(\&{Arc}
${}{*}){}$ \|a;\6
${}\\{oo},\39\\{at}\K\|a\MG\\{typ},\39\\{bt}\K\|b\MG\\{typ};{}$\6
\&{if} (\|t)\1\5
\X40:Downdate \PB{\\{aa}} and \PB{\\{bb}} from an old series bond\X\2\6
\&{else}\1\5
\X41:Downdate \PB{\\{aa}} and \PB{\\{bb}} from an old parallel bond\X;\2\6
\&{if} (\\{isleaf}(\|a))\1\6
\&{if} (\\{isleaf}(\|b))\1\5
\X42:Unbuild leaf with leaf\X\2\6
\&{else}\1\5
\X43:Unbuild leaf with branch\X\2\2\6
\&{else} \&{if} (\\{isleaf}(\|b))\1\5
\X44:Unbuild branch with leaf\X\2\6
\&{else}\1\5
\X45:Unbuild branch with branch\X;\2\6
\&{if} ${}(\|c\MG\\{val}){}$\1\5
${}\\{oooo},\39\|c\MG\\{up}\MG\\{down}\K\|c\MG\\{down},\39\|c\MG\\{down}\MG%
\\{up}\K\|c\MG\\{up};{}$\2\6
\&{if} ${}(\|a\MG\\{val}){}$\1\5
${}\\{ooooo},\39\|a\MG\\{up}\K\\{tree}\MG\\{up},\39\\{tree}\MG\\{up}\K\|a\MG%
\\{up}\MG\\{down}\K\|a,\39\|a\MG\\{down}\K\\{tree};{}$\2\6
\&{if} ${}(\|b\MG\\{val}){}$\1\5
${}\\{ooooo},\39\|b\MG\\{up}\K\\{tree}\MG\\{up},\39\\{tree}\MG\\{up}\K\|b\MG%
\\{up}\MG\\{down}\K\|b,\39\|b\MG\\{down}\K\\{tree};{}$\2\6
\4${}\}{}$\2\par
\fi

\M{40}\B\X40:Downdate \PB{\\{aa}} and \PB{\\{bb}} from an old series bond\X${}%
\E{}$\6
${}\{{}$\1\6
\\{undelete}(\\{mate}(\\{bb}));\C{ now \PB{$\\{ee}\K\\{mate}(\\{bb})$} }\6
${}\\{oo},\39\|v\K\\{ee}\MG\\{tip},\39\\{mate}(\\{aa})\MG\\{tip}\K\|v,\39\|v\MG%
\\{mark}\K\T{0};{}$\6
${}\\{ooo},\39\\{aa}\MG\\{prev}\K\|v\MG\\{arcs},\39\\{aa}\MG\\{next}\K%
\\{bb};{}$\6
\&{if} (\\{extraverbose})\1\5
${}\\{printf}(\.{"\ removing\ \%d=series}\)\.{(\%d,\%d)\ between\ \%s\ a}\)%
\.{nd\ \%s\\n"},\39\\{bn}(\|c),\39\\{bn}(\|a),\39\\{bn}(\|b),\39\\{aa}\MG%
\\{tip}\MG\\{name},\39\\{bb}\MG\\{tip}\MG\\{name});{}$\2\6
\4${}\}{}$\2\par
\U39.\fi

\M{41}\B\X41:Downdate \PB{\\{aa}} and \PB{\\{bb}} from an old parallel bond%
\X${}\E{}$\6
${}\{{}$\1\6
${}\\{oo},\39\|u\K\\{aa}\MG\\{tip},\39\|v\K\\{mate}(\\{aa})\MG\\{tip};{}$\6
${}\\{oooo},\39\\{undelete}(\\{mate}(\\{bb})),\39\\{undelete}(\\{bb}),\39\|u\MG%
\\{deg}\PP,\39\|v\MG\\{deg}\PP;{}$\6
\&{if} (\\{extraverbose})\1\5
${}\\{printf}(\.{"\ removing\ \%d=parall}\)\.{el(\%d,\%d)\ between\ \%s}\)\.{\
and\ \%s\\n"},\39\\{bn}(\|c),\39\\{bn}(\|a),\39\\{bn}(\|b),\39\|u\MG\\{name},%
\39\|v\MG\\{name});{}$\2\6
\4${}\}{}$\2\par
\U39.\fi

\M{42}Sibling links of bonds at the top level are never examined.
This program nullifies them only to be tidy and possibly catch
errors, but no mems are counted.

\Y\B\4\X42:Unbuild leaf with leaf\X${}\E{}$\6
${}\{{}$\1\6
${}\|a\MG\\{lsib}\K\|a\MG\\{rsib}\K\|b\MG\\{lsib}\K\|b\MG\\{rsib}\K\NULL;{}$\6
${}\\{ooo},\39\\{action}\MG\\{right}\K\|c\MG\\{right},\39\\{action}\MG\\{right}%
\MG\\{left}\K\\{action};{}$\6
\4${}\}{}$\2\par
\U39.\fi

\M{43}The nonobvious part here is the calculation of \PB{\\{des}} when a
disabled bond reenters the picture.

\Y\B\4\X43:Unbuild leaf with branch\X${}\E{}$\6
${}\{{}$\1\6
\&{if} ${}(\\{bt}\I\|t){}$\5
${}\{{}$\1\6
${}\|a\MG\\{lsib}\K\|a\MG\\{rsib}\K\|b\MG\\{lsib}\K\|b\MG\\{rsib}\K\NULL;{}$\6
${}\\{oo},\39\|b\MG\\{left}\K\|c\MG\\{left};{}$\6
\4${}\}{}$\5
\2\&{else}\5
${}\{{}$\1\6
\&{if} ${}(\|o,\39\|c\MG\\{des}\E\|a){}$\1\5
${}\|o,\39\|b\MG\\{val}\K\\{bt};{}$\2\6
\&{else}\1\5
${}\\{oo},\39\|b\MG\\{val}\K\|c\MG\\{val};{}$\2\6
\&{if} ${}(\|b\MG\\{val}\I\\{bt}){}$\1\5
${}\|b\MG\\{des}\K\|c\MG\\{des}{}$;\C{ mem is charged below }\2\6
${}\\{oooo},\39\|b\MG\\{lchild}\MG\\{lsib}\K\|a\MG\\{lsib},\39\|a\MG\\{lsib}\MG%
\\{rsib}\K\|b\MG\\{lchild};{}$\6
${}\\{ooo},\39\|b\MG\\{done}\K\|b\MG\\{des}\MG\\{lsib};{}$\6
${}\|a\MG\\{lsib}\K\|a\MG\\{rsib}\K\NULL;{}$\6
${}\\{oo},\39\|b\MG\\{right}\MG\\{left}\K\|b;{}$\6
\4${}\}{}$\2\6
${}\\{oo},\39\|b\MG\\{left}\MG\\{right}\K\|b;{}$\6
\4${}\}{}$\2\par
\U39.\fi

\M{44}\B\X44:Unbuild branch with leaf\X${}\E{}$\6
${}\{{}$\1\6
\&{if} ${}(\\{at}\I\|t){}$\5
${}\{{}$\1\6
${}\|a\MG\\{lsib}\K\|a\MG\\{rsib}\K\|b\MG\\{lsib}\K\|b\MG\\{rsib}\K\NULL;{}$\6
${}\\{oo},\39\|a\MG\\{left}\K\|c\MG\\{left};{}$\6
\4${}\}{}$\5
\2\&{else}\5
${}\{{}$\1\6
\&{if} ${}(\|o,\39\|c\MG\\{des}\E\|b){}$\1\5
${}\|o,\39\|a\MG\\{val}\K\\{at};{}$\2\6
\&{else}\1\5
${}\\{oo},\39\|a\MG\\{val}\K\|c\MG\\{val};{}$\2\6
\&{if} ${}(\|a\MG\\{val}\I\\{at}){}$\1\5
${}\|a\MG\\{des}\K\|c\MG\\{des};{}$\2\6
${}\\{oooo},\39\|a\MG\\{lchild}\MG\\{lsib}\K\|b\MG\\{lsib},\39\|b\MG\\{lsib}\MG%
\\{rsib}\K\|a\MG\\{lchild};{}$\6
${}\\{ooo},\39\|a\MG\\{done}\K\|a\MG\\{des}\MG\\{lsib};{}$\6
${}\|b\MG\\{rsib}\K\|b\MG\\{lsib}\K\NULL;{}$\6
${}\\{oo},\39\|a\MG\\{right}\MG\\{left}\K\|a;{}$\6
\4${}\}{}$\2\6
${}\\{oo},\39\|a\MG\\{left}\MG\\{right}\K\|a;{}$\6
\4${}\}{}$\2\par
\U39.\fi

\M{45}If we previously combined two series bonds into a larger series bond, or
two parallel bonds into a larger parallel bond, we may have to search through
the children in order to figure out where \PB{$\|c\MG\\{des}$} lies.

\Y\B\4\X45:Unbuild branch with branch\X${}\E{}$\6
${}\{{}$\1\6
${}\\{ooo},\39\|d\K\|a\MG\\{scope},\39\|d\MG\\{right}\K\|c\MG\\{rsave};{}$\6
\&{if} ${}(\\{at}\I\|t){}$\5
${}\{{}$\1\6
${}\\{oo},\39\|a\MG\\{left}\K\|c\MG\\{lsave};{}$\6
\&{if} ${}(\\{bt}\I\|t){}$\5
${}\{{}$\1\6
${}\|a\MG\\{lsib}\K\|a\MG\\{rsib}\K\|b\MG\\{lsib}\K\|b\MG\\{rsib}\K\NULL;{}$\6
${}\\{ooo},\39\|b\MG\\{left}\K\|c\MG\\{left},\39\|b\MG\\{left}\MG\\{right}\K%
\|b;{}$\6
\4${}\}{}$\5
\2\&{else}\5
${}\{{}$\1\6
\&{if} ${}(\|o,\39\|c\MG\\{des}\E\|a){}$\1\5
${}\|o,\39\|b\MG\\{val}\K\\{bt};{}$\2\6
\&{else}\1\5
${}\\{oo},\39\|b\MG\\{val}\K\|c\MG\\{val};{}$\2\6
\&{if} ${}(\|b\MG\\{val}\I\\{bt}){}$\1\5
${}\|b\MG\\{des}\K\|c\MG\\{des};{}$\2\6
${}\\{oooo},\39\|b\MG\\{lchild}\MG\\{lsib}\K\|a\MG\\{lsib},\39\|a\MG\\{lsib}\MG%
\\{rsib}\K\|b\MG\\{lchild};{}$\6
${}\\{ooo},\39\|b\MG\\{done}\K\|b\MG\\{des}\MG\\{lsib};{}$\6
${}\\{oooo},\39\|b\MG\\{left}\MG\\{right}\K\|b\MG\\{right}\MG\\{left}\K\|b;{}$\6
${}\|a\MG\\{lsib}\K\|a\MG\\{rsib}\K\NULL;{}$\6
\4${}\}{}$\2\6
\4${}\}{}$\5
\2\&{else}\5
${}\{{}$\1\6
\&{if} ${}(\|d\I\|a){}$\1\5
${}\\{oo},\39\|a\MG\\{right}\MG\\{left}\K\|a;{}$\2\6
\&{if} ${}(\\{bt}\I\|t){}$\5
${}\{{}$\1\6
\&{if} ${}(\|o,\39\|c\MG\\{des}\E\|b){}$\1\5
${}\|o,\39\|a\MG\\{val}\K\\{at};{}$\2\6
\&{else}\1\5
${}\\{oo},\39\|a\MG\\{val}\K\|c\MG\\{val};{}$\2\6
\&{if} ${}(\|a\MG\\{val}\I\\{at}){}$\1\5
${}\|a\MG\\{des}\K\|c\MG\\{des};{}$\2\6
${}\\{oooo},\39\|a\MG\\{lchild}\MG\\{lsib}\K\|b\MG\\{lsib},\39\|b\MG\\{lsib}\MG%
\\{rsib}\K\|a\MG\\{lchild};{}$\6
${}\\{ooo},\39\|a\MG\\{done}\K\|a\MG\\{des}\MG\\{lsib};{}$\6
${}\\{oooo},\39\|b\MG\\{left}\K\|c\MG\\{left},\39\|b\MG\\{left}\MG\\{right}\K%
\|b;{}$\6
${}\|b\MG\\{lsib}\K\|b\MG\\{rsib}\K\NULL;{}$\6
\4${}\}{}$\5
\2\&{else}\5
${}\{{}$\1\6
\&{if} ${}(\|c\MG\\{val}\E\|t){}$\1\5
${}\\{oo},\39\|a\MG\\{val}\K\|b\MG\\{val}\K\|t;{}$\2\6
\&{else}\5
\1\&{for} ${}(\\{ooo},\39\|d\K\|c\MG\\{des};{}$  ; \|o${},\39\|d\K\|d\MG%
\\{lsib}){}$\1\6
\&{if} ${}(\|d\E\|a\MG\\{lchild}){}$\5
${}\{{}$\1\6
${}\\{ooo},\39\|a\MG\\{val}\K\T{1}-\|t,\39\|b\MG\\{val}\K\|t,\39\|a\MG\\{des}\K%
\|c\MG\\{des}{}$;\5
\&{break};\6
\4${}\}{}$\5
\2\&{else} \&{if} ${}(\|d\E\|b\MG\\{lchild}){}$\5
${}\{{}$\1\6
${}\\{ooo},\39\|b\MG\\{val}\K\T{1}-\|t,\39\|a\MG\\{val}\K\|t,\39\|b\MG\\{des}\K%
\|c\MG\\{des}{}$;\5
\&{break};\6
\4${}\}{}$\2\2\2\6
${}\\{oooooo},\39\|d\K\|a\MG\\{lchild}\MG\\{lsib},\39\|a\MG\\{lchild}\MG%
\\{lsib}\K\|b\MG\\{lchild}\MG\\{lsib},\39\|b\MG\\{lchild}\MG\\{lsib}\K\|d;{}$\6
${}\\{oo},\39\|d\MG\\{rsib}\K\|b\MG\\{lchild},\39\|a\MG\\{lchild}\MG\\{lsib}\MG%
\\{rsib}\K\|a\MG\\{lchild};{}$\6
${}\\{ooooo},\39\|a\MG\\{done}\K\|a\MG\\{des}\MG\\{lsib},\39\|b\MG\\{done}\K\|b%
\MG\\{des}\MG\\{lsib};{}$\6
${}\|d\K\|a\MG\\{scope};{}$\6
${}\\{oooo},\39\|b\MG\\{left}\MG\\{right}\K\|b\MG\\{right}\MG\\{left}\K\|b;{}$\6
\4${}\}{}$\2\6
\4${}\}{}$\2\6
${}\\{oooo},\39\|a\MG\\{left}\MG\\{right}\K\|a,\39\|d\MG\\{right}\MG\\{left}\K%
\|d;{}$\6
\4${}\}{}$\2\par
\U39.\fi

\M{46}\B\X46:Restore one or two deleted bonds and \PB{\&{return}}\X${}\E{}$\6
${}\{{}$\1\6
${}\\{oo},\39\|u\K\\{mate}(\\{aa})\MG\\{tip},\39\|a\K{}$(\&{Bond} ${}{*}){}$ %
\\{aa}${}\MG\\{bond};{}$\6
${}\|o,\39\|v\K(\\{bb}\?\\{mate}(\\{bb})\MG\\{tip}:\\{aa}\MG\\{tip});{}$\6
\&{if} (\\{bb})\1\5
${}\\{oo},\39\\{undelete}(\\{bb}),\39\|v\MG\\{deg}\PP;{}$\2\6
\&{else}\1\5
${}\|v\MG\\{mark}\K\T{0}{}$;\C{ for debugging, remove its negative mark }\2\6
${}\\{oo},\39\\{undelete}(\\{aa}),\39\|u\MG\\{deg}\PP;{}$\6
\&{if} (\\{extraverbose})\1\5
${}\\{printf}(\.{"\ restoring\ bond\ \%d\ }\)\.{between\ \%s\ and\ \%s\%s\\}\)%
\.{n"},\39\\{bn}(\|a),\39\|u\MG\\{name},\39\\{bb}\?\.{""}:\.{"endpoint\ "},\39%
\|v\MG\\{name});{}$\2\6
\&{if} ${}(\|a\MG\\{val}){}$\1\5
${}\\{ooooo},\39\|a\MG\\{up}\K\\{tree}\MG\\{up},\39\\{tree}\MG\\{up}\K\|a\MG%
\\{up}\MG\\{down}\K\|a,\39\|a\MG\\{down}\K\\{tree};{}$\2\6
\&{return};\6
\4${}\}{}$\2\par
\U39.\fi

\N{1}{47}Pulsing the action list. OK, all the hard stuff is done; only one
more piece of the program needs to be put in place. And from {\mc SPSPAN},
we know how to do the remaining job nicely.

\Y\B\4\D$\\{easy}(\|b)$ \5
$\|o,\39\|b\MG\\{typ}\E\|b\MG{}$\\{val}\par
\Y\B\4\X47:Do the {\mc SPSPAN} algorithm on the action list\X${}\E{}$\6
\&{while} (\T{1})\5
${}\{{}$\1\6
\&{register} \&{Bond} ${}{*}\|q,{}$ ${}{*}\|l,{}$ ${}{*}\|r;{}$\7
\&{for} ${}(\|o,\39\|r\K\\{action}\MG\\{left};{}$ \\{easy}(\|r); \|o${},\39\|r%
\K\|r\MG\\{left}){}$\1\5
;\C{ find the rightmost uneasy node }\2\6
${}\\{oo},\39\|b\K\|r\MG\\{focus},\39\|r\MG\\{focus}\K\|r{}$;\C{ steps (1) and
(3) }\6
\&{if} ${}(\|b\E\\{action}){}$\1\5
\&{break};\2\6
\X49:Change \PB{$\|b\MG\\{des}$} and visit a new spanning tree\X;\6
\&{if} ${}(\|o,\39\|b\MG\\{des}\E\|b\MG\\{done}){}$\1\5
\X48:Passivate \PB{\|b}\X;\2\6
\4${}\}{}$\2\par
\U25.\fi

\M{48}All uneasy nodes to the right of \PB{\|b} are now active, and \PB{\|l} is
the
former \PB{$\|b\MG\\{des}$}.

\Y\B\4\X48:Passivate \PB{\|b}\X${}\E{}$\6
${}\{{}$\1\6
${}\|o,\39\|b\MG\\{done}\K\|l;{}$\6
\&{for} ${}(\|o,\39\|l\K\|b\MG\\{left};{}$ \\{easy}(\|l); \|o${},\39\|l\K\|l\MG%
\\{left}){}$\1\5
;\C{ find the first uneasy node to the left }\2\6
${}\\{ooo},\39\|b\MG\\{focus}\K\|l\MG\\{focus},\39\|l\MG\\{focus}\K\|l;{}$\6
\4${}\}{}$\2\par
\U47.\fi

\M{49}If the user has asked for \PB{\\{verbose}} output, we print only the
edge that has entered the spanning tree and the edge that has left.

\Y\B\4\X49:Change \PB{$\|b\MG\\{des}$} and visit a new spanning tree\X${}\E{}$\6
$\\{count}\PP;{}$\6
${}\\{oo},\39\|l\K\|b\MG\\{des},\39\|r\K\|l\MG\\{rsib};{}$\6
${}\|o,\39\|k\K\|b\MG\\{val}{}$;\C{ \PB{$\|k\K\|l\MG\\{val}\I\|r\MG\\{val}$} }\6
\&{for} ${}(\|q\K\|l;{}$  ; \|o${},\39\|q\K\|q\MG\\{des}){}$\5
${}\{{}$\1\6
${}\|o,\39\|q\MG\\{val}\K\|k\XOR\T{1};{}$\6
\&{if} (\\{isleaf}(\|q))\1\5
\&{break};\2\6
\4${}\}{}$\2\6
\&{if} (\\{verbose})\1\5
${}\\{printf}(\.{"\%.15g:\ \%c\%d"},\39\\{count},\39\|k\?\.{'-'}:\.{'+'},\39%
\\{bn}(\|q));{}$\2\6
\&{for} ${}(\|q\K\|r;{}$  ; \|o${},\39\|q\K\|q\MG\\{des}){}$\5
${}\{{}$\1\6
${}\|o,\39\|q\MG\\{val}\K\|k;{}$\6
\&{if} (\\{isleaf}(\|q))\1\5
\&{break};\2\6
\4${}\}{}$\2\6
\&{if} (\\{verbose})\5
${}\{{}$\1\6
${}\\{printf}(\.{"\%c\%d"},\39\|k\?\.{'+'}:\.{'-'},\39\\{bn}(\|q));{}$\6
\&{if} (\\{extraverbose})\5
${}\{{}$\1\6
\\{printf}(\.{"\ ("});\6
\&{for} ${}(\|q\K\\{bondbase}+\T{1};{}$ ${}\|q\Z\\{topleaf};{}$ ${}\|q\PP){}$\1%
\6
\&{if} ${}(\|q\MG\\{val}){}$\1\5
${}\\{printf}(\.{"\ \%d"},\39\\{bn}(\|q));{}$\2\2\6
\\{printf}(\.{"\ )\\n"});\6
\4${}\}{}$\5
\2\&{else}\1\5
\\{printf}(\.{"\\n"});\2\6
\4${}\}{}$\2\6
${}\|o,\39\|b\MG\\{des}\K\|r{}$;\C{ ``visiting'' }\par
\U47.\fi

\M{50}Note: I haven't proved the efficiency claim made in the opening
paragraph.
I don't have time to write the proof down now, sorry. But it is based on
the fact that a connected graph on $n$ vertices
with all vertices of degree 3~or~more
always has more than $2^{n/2}$ spanning trees. Therefore the time
that is spent checking for bridges, etc., which is polynomial in the
number of vertices, is asymptotically
less than the time needed to spew out the trees.

\fi

\N{1}{51}Index.
\fi

\inx
\fin
\con
