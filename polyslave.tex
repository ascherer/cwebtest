\input cwebmac
\datethis


\N{1}{1}Introduction. This program obediently carries out the wishes of
{\mc POLYNUM}, which has compiled a set of one-byte and four-byte
instructions for us to interpret.

But instead of producing high-precision output, it does all its
arithmetic modulo a given number $m\le256$.
That trick keeps memory usage small; it allows the user to reconstruct
true answers of almost unlimited size just by trying sufficiently
many different values of~$m$. And if anybody is concerned about
bits getting clobbered by cosmic radiation, they can gain additional
confidence in the accuracy by running the calculations for
more moduli than strictly necessary.

\Y\B\8\#\&{include} \.{<stdio.h>}\6
\8\#\&{include} \.{<setjmp.h>}\6
\&{jmp\_buf} \\{restart\_point};\7
\X2:Type definitions\X\6
\X5:Global variables\X\6
\X3:Subroutines\X\7
\\{main}(\&{int} \\{argc}${},\39{}$\&{char} ${}{*}\\{argv}[\,]){}$\1\1\2\2\6
${}\{{}$\1\6
\X6:Local variables\X;\6
\X4:Scan the command line\X;\6
\\{setjmp}(\\{restart\_point});\C{ \PB{\\{longjmp}} will return here if
necessary }\6
\X15:Initialize\X;\6
\X17:Interpret the instructions in the input\X;\6
\X25:Print statistics\X;\6
\\{exit}(\T{0});\6
\4${}\}{}$\2\par
\fi

\M{2}It is easy to adapt this program to work with counters that occupy either
one byte (\PB{\&{unsigned} \&{char}}), two bytes (\PB{\&{unsigned} \&{short}}),
or
four bytes (\PB{\&{unsigned} \&{int}}), depending on how much memory is
available.

Even if we limit ourselves to one-byte counters, exact results of up to 362
bits can be determined. For example,
the eleven moduli 256, 253, 251, 247, 245, 243, 241, 239, 233, 229, 227
will suffice to enumerate $n$-ominoes for $n\le46$; and the additional modulus
223 will carry us through $n\le50$. (If some day we have the resources to
go even higher, the next moduli to try would be 211, 199, and 197.)

However, the author's experience with the case \PB{$\|n\K\T{47}$} showed that
the memory
space needed for counters in this program was not as precious as the memory
space needed for configurations in {\mc POLYNUM}. Therefore the
four-byte moduli $2^{31}=2147483648$, $2^{31}-1$ (which is prime), and
$2^{31}-3$ (which equals $5\cdot19\cdot22605091$) worked out best.
Together they reach nearly to $10^{28}$, which would actually be large
enough to count 49-ominoes.

With a little extra work I could have allowed moduli up to $2^{32}$.
But I didn't bother, because $2^{31}$ turned out to be plenty big.

\Y\B\4\D$\\{maxm}$ \5
$(\T{1}\LL\T{31}{}$)\C{ the modulus $m$ must not exceed this }\par
\Y\B\4\X2:Type definitions\X${}\E{}$\6
\&{typedef} \&{unsigned} \&{int} \&{counter};\C{ the main data type in our
arrays }\par
\A11.
\U1.\fi

\M{3}The program checks frequently that everything in the input file
is legitimate. If not, too bad; the run is terminated (although
a debugger can help diagnose nonobvious problems). Extensive checks like this
have helped the author to detect errors in the program as well as errors in
the input.

\Y\B\4\X3:Subroutines\X${}\E{}$\6
\&{void} \\{panic}(\&{char} ${}{*}\\{mess}){}$\1\1\2\2\6
${}\{{}$\1\6
${}\\{fprintf}(\\{stderr},\39\.{"\%s!\\n"},\39\\{mess});{}$\6
${}\\{exit}({-}\T{1});{}$\6
\4${}\}{}$\2\par
\As8, 9, 10\ETs12.
\U1.\fi

\M{4}Several gigabytes of input might be needed, so the input file name will
be extended by \.{.0}, \.{.1}, \dots, just as in {\mc POLYNUM}.

Output data suitable for processing by {\it Mathematica\/} will be
written on a file with the same name as the input but extended by
the modulus and \.{.m}.

\Y\B\4\X4:Scan the command line\X${}\E{}$\6
\&{if} ${}(\\{argc}\I\T{3}\V\\{sscanf}(\\{argv}[\T{2}],\39\.{"\%u"},\39{\AND}%
\\{modulus})\I\T{1}){}$\5
${}\{{}$\1\6
${}\\{fprintf}(\\{stderr},\39\.{"Usage:\ \%s\ infilenam}\)\.{e\ modulus\\n"},%
\39\\{argv}[\T{0}]);{}$\6
${}\\{exit}({-}\T{2});{}$\6
\4${}\}{}$\2\6
${}\\{base\_name}\K\\{argv}[\T{1}];{}$\6
\&{if} ${}(\\{modulus}<\T{2}\V\\{modulus}>\\{maxm}){}$\1\5
\\{panic}(\.{"Improper\ modulus"});\2\6
${}\|m\K\\{modulus};{}$\6
${}\\{sprintf}(\\{filename},\39\.{"\%.90s-\%u.m"},\39\\{base\_name},\39%
\\{modulus});{}$\6
${}\\{math\_file}\K\\{fopen}(\\{filename},\39\.{"w"});{}$\6
\&{if} ${}(\R\\{math\_file}){}$\1\5
\\{panic}(\.{"I\ can't\ open\ the\ ou}\)\.{tput\ file"});\2\par
\U1.\fi

\M{5}\B\X5:Global variables\X${}\E{}$\6
\&{unsigned} \&{int} \\{modulus};\C{ results will discard multiples of this
number }\6
\&{char} ${}{*}\\{base\_name},{}$ \\{filename}[\T{100}];\6
\&{FILE} ${}{*}\\{math\_file}{}$;\C{ the output file }\par
\As7, 14, 16\ETs28.
\U1.\fi

\M{6}\B\X6:Local variables\X${}\E{}$\6
\&{register} \&{int} \|k;\C{ all-purpose index register }\6
\&{register} \&{unsigned} \&{int} \|m;\C{ register copy of \PB{\\{modulus}} }%
\par
\A18.
\U1.\fi

\N{1}{7}Input. Let's start with the basic routines that are needed to
read instructions from the input file(s).
As soon as $2^{30}$ bytes of data have been read from file \.{foo.0},
we'll turn to file \.{foo.1}, etc.

\Y\B\4\D$\\{filelength\_threshold}$ \5
$(\T{1}\LL\T{30}{}$)\C{ should match     the corresponding number in {\mc
POLYNUM}  }\par
\B\4\D$\\{buf\_size}$ \5
$(\T{1}\LL\T{16}{}$)\C{ should be a divisor of \PB{\\{filelength\_threshold}} }%
\par
\Y\B\4\X5:Global variables\X${}\mathrel+\E{}$\6
\&{FILE} ${}{*}\\{in\_file}{}$;\C{ the input file }\6
\&{union} ${}\{{}$\1\6
\&{unsigned} \&{char} ${}\\{buf}[\\{buf\_size}+\T{10}]{}$;\C{ place for binary
input }\6
\&{unsigned} \&{int} \\{foo};\C{ force \PB{$\\{in}.\\{buf}$} to be aligned
somewhat sensibly }\2\6
${}\}{}$ \\{in};\6
\&{unsigned} \&{char} ${}{*}\\{buf\_ptr}{}$;\C{ our current place in the buffer
}\6
\&{int} \\{bytes\_in};\C{ the number of bytes seen so far in the current input
file }\6
\&{unsigned} \&{int} \\{checksum};\C{ a way to help identify bad I/O }\6
\&{FILE} ${}{*}\\{ck\_file}{}$;\C{ the checksum file }\6
\&{unsigned} \&{int} \\{checkbuf};\C{ a check sum for comparison }\6
\&{int} \\{file\_extension};\C{ the number of GGbytes input }\par
\fi

\M{8}\B\X3:Subroutines\X${}\mathrel+\E{}$\6
\&{void} \\{open\_it}(\,)\1\1\2\2\6
${}\{{}$\1\6
${}\\{sprintf}(\\{filename},\39\.{"\%.90s.\%d"},\39\\{base\_name},\39\\{file%
\_extension});{}$\6
${}\\{in\_file}\K\\{fopen}(\\{filename},\39\.{"rb"});{}$\6
\&{if} ${}(\R\\{in\_file}){}$\5
${}\{{}$\1\6
${}\\{fprintf}(\\{stderr},\39\.{"I\ can't\ open\ file\ \%}\)\.{s"},\39%
\\{filename});{}$\6
\\{panic}(\.{"\ for\ input"});\6
\4${}\}{}$\2\6
${}\\{bytes\_in}\K\\{checksum}\K\T{0};{}$\6
\4${}\}{}$\2\par
\fi

\M{9}If the check sum is bad, we go back to the beginning. Some incorrect
definitions may have been output to \PB{\\{math\_file}}, but we'll append
new definitions that override them.

\Y\B\4\X3:Subroutines\X${}\mathrel+\E{}$\6
\&{void} \\{close\_it}(\,)\1\1\2\2\6
${}\{{}$\1\6
\&{if} ${}(\\{fread}({\AND}\\{checkbuf},\39{}$\&{sizeof}(\&{unsigned} %
\&{int})${},\39\T{1},\39\\{ck\_file})\I\T{1}){}$\1\5
\\{panic}(\.{"I\ couldn't\ read\ the}\)\.{\ check\ sum"});\2\6
\&{if} ${}(\\{fclose}(\\{in\_file})\I\T{0}){}$\1\5
\\{panic}(\.{"I\ couldn't\ close\ th}\)\.{e\ input\ file"});\2\6
${}\\{printf}(\.{"[\%d\ bytes\ read\ from}\)\.{\ file\ \%s,\ checksum\ \%}\)%
\.{u.]\\n"},\39\\{bytes\_in},\39\\{filename},\39\\{checksum});{}$\6
\&{if} ${}(\\{checkbuf}\I\\{checksum}){}$\5
${}\{{}$\1\6
\\{printf}(\.{"Checksum\ mismatch!\ }\)\.{Restarting...\\n"});\6
${}\\{longjmp}(\\{restart\_point},\39\T{1});{}$\6
\4${}\}{}$\2\6
\\{fflush}(\\{stdout});\6
\4${}\}{}$\2\par
\fi

\M{10}My first draft of this program simply used \PB{\\{fread}} to input one or
three
bytes at a time. But that turned out to be incredibly slow on my system,
so now I'm doing my own buffering.

The program here uses the fact that six consecutive zero bytes cannot be
present in a valid input; thus we need not make a special check for premature
end-of-file.

\Y\B\4\D$\\{end\_of\_buffer}$ \5
${\AND}\\{in}.\\{buf}[\\{buf\_size}+\T{4}{}$]\par
\Y\B\4\X3:Subroutines\X${}\mathrel+\E{}$\6
\&{void} \\{read\_it}(\,)\1\1\2\2\6
${}\{{}$\1\6
\&{register} \&{int} \|t${},{}$ \|k;\6
\&{register} \&{unsigned} \&{int} \|s;\7
\&{if} ${}(\\{bytes\_in}\G\\{filelength\_threshold}){}$\5
${}\{{}$\1\6
\&{if} ${}(\\{bytes\_in}\I\\{filelength\_threshold}){}$\1\5
\\{panic}(\.{"Improper\ buffer\ siz}\)\.{e"});\2\6
\\{close\_it}(\,);\6
${}\\{file\_extension}\PP;{}$\6
\\{open\_it}(\,);\6
\4${}\}{}$\2\6
${}\|t\K\\{fread}(\\{in}.\\{buf}+\T{4},\39{}$\&{sizeof}(\&{unsigned} %
\&{char})${},\39\\{buf\_size},\39\\{in\_file});{}$\6
\&{if} ${}(\|t<\\{buf\_size}){}$\1\5
${}\\{in}.\\{buf}[\|t+\T{4}]\K\\{in}.\\{buf}[\|t+\T{5}]\K\\{in}.\\{buf}[\|t+%
\T{6}]\K\\{in}.\\{buf}[\|t+\T{7}]\K\\{in}.\\{buf}[\|t+\T{8}]\K\T{\^81}{}$;\C{
will cause \PB{\\{sync}} 1 error if read }\2\6
${}\\{bytes\_in}\MRL{+{\K}}\|t;{}$\6
\&{for} ${}(\|k\K\|s\K\T{0};{}$ ${}\|k<\|t;{}$ ${}\|k\PP){}$\1\5
${}\|s\K(\|s\LL\T{1})+\\{in}.\\{buf}[\|k+\T{4}];{}$\2\6
${}\\{checksum}\MRL{+{\K}}\|s;{}$\6
\4${}\}{}$\2\par
\fi

\M{11}A four-byte instruction has the binary form $(0xaaaaaa)_2$,
$(bbbbbbbb)_2$,
$(cccccccc)_2$, $(dddddddd)_2$, where
$(aaaaaabbbbbbbbccccccccdddddddd)_2$ is a 30-bit address specified
in big-endian fashion.
If $x=0$ it means, ``This is the new source address $s$.''
If $x=1$ it means, ``This is the new target address $t$.''

A one-byte instruction has the binary form $(1ooopppp)_2$, with a 3-bit
opcode $(ooo)_2$ and a 4-bit parameter $(pppp)_2$. If the parameter is zero,
the following byte is regarded as an 8-bit parameter $(pppppppp)_2$, and
it should not be zero. (In that case the ``one-byte instruction'' actually
occupies two bytes.)

In the instruction definitions below, $p$ stands for the parameter,
$s$ stands for the current source address, and $t$ stands for the
current target address. The slave processor operates on a large
array called \PB{\\{count}}.

Opcode 0 (\PB{\\{sync}}) means, ``We have just finished row \PB{\|p}.'' A
report
is given to the user.

Opcode 1 (\PB{\\{clear}}) means, ``Set \PB{$\\{count}[\|t+\|j]\K\T{0}$} for $0%
\le j< p$.''

Opcode 2 (\PB{\\{copy}}) means, ``Set \PB{$\\{count}[\|t+\|j]\K\\{count}[\|s+%
\|j]$} for $0\le j<p$.''

Opcode 3 (\PB{\\{add}}) means, ``Set \PB{$\\{count}[\|t+\|j]\MRL{+{\K}}%
\\{count}[\|s+\|j]$} for $0\le j<p$.''

Opcode 4 (\PB{\\{inc\_src}}) means, ``Set \PB{$\|s\MRL{+{\K}}\|p$}.''

Opcode 5 (\PB{\\{dec\_src}}) means, ``Set \PB{$\|s\MRL{-{\K}}\|p$}.''

Opcode 6 (\PB{\\{inc\_trg}}) means, ``Set \PB{$\|t\MRL{+{\K}}\|p$}.''

Opcode 7 (\PB{\\{dec\_trg}}) means, ``Set \PB{$\|t\MRL{-{\K}}\|p$}.''

\Y\B\4\D$\\{targ\_bit}$ \5
\T{\^40000000}\C{ specifies \PB{\|t} in a four-byte instruction }\par
\Y\B\4\X2:Type definitions\X${}\mathrel+\E{}$\6
\&{typedef} \&{enum} ${}\{{}$\1\6
${}\\{sync},\39\\{clear},\39\\{copy},\39\\{add},\39\\{inc\_src},\39\\{dec%
\_src},\39\\{inc\_trg},\39\\{dec\_trg}{}$\2\6
${}\}{}$ \&{opcode};\par
\fi

\M{12}The \PB{\\{get\_inst}} routine reads the next instruction from the input
and returns the value of its parameter, also storing the opcode
in the global variable \PB{\\{op}}. Changes to \PB{\|s} and \PB{\|t} are taken
care
of automatically, so that \PB{\\{op}} is reduced to either \PB{\\{sync}},
\PB{\\{clear}}, \PB{\\{copy}}, or \PB{\\{add}}.

\Y\B\4\D$\\{advance\_b}$ \6
\&{if} ${}(\PP\|b\E\\{end\_of\_buffer}){}$\5
${}\{{}$\5
\1\\{read\_it}(\,);\5
${}\|b\K{\AND}\\{in}.\\{buf}[\T{4}]{}$;\5
${}\}{}$\2\par
\Y\B\4\X3:Subroutines\X${}\mathrel+\E{}$\6
\&{opcode} \\{get\_inst}(\,)\1\1\2\2\6
${}\{{}$\1\6
\&{register} \&{unsigned} \&{char} ${}{*}\|b\K\\{buf\_ptr};{}$\6
\&{register} \&{opcode} \|o;\6
\&{register} \&{int} \|p;\7
\4\\{restart}:\5
\\{advance\_b};\6
\&{if} ${}(\R({*}\|b\AND\T{\^80})){}$\1\5
\X13:Change the source or target address and \PB{\&{goto} \\{restart}}\X;\2\6
${}\|o\K({*}\|b\GG\T{4})\AND\T{7};{}$\6
${}\|p\K{*}\|b\AND\T{\^f};{}$\6
\&{if} ${}(\R\|p){}$\5
${}\{{}$\1\6
\\{advance\_b};\6
${}\|p\K{*}\|b;{}$\6
\&{if} ${}(\R\|p){}$\1\5
\\{panic}(\.{"Parameter\ is\ zero"});\2\6
\4${}\}{}$\2\6
\&{switch} (\|o)\5
${}\{{}$\1\6
\4\&{case} \\{inc\_src}:\5
${}\\{cur\_src}\MRL{+{\K}}\|p{}$;\5
\&{goto} \\{restart};\6
\4\&{case} \\{dec\_src}:\5
${}\\{cur\_src}\MRL{-{\K}}\|p{}$;\5
\&{goto} \\{restart};\6
\4\&{case} \\{inc\_trg}:\5
${}\\{cur\_trg}\MRL{+{\K}}\|p{}$;\5
\&{goto} \\{restart};\6
\4\&{case} \\{dec\_trg}:\5
${}\\{cur\_trg}\MRL{-{\K}}\|p{}$;\5
\&{goto} \\{restart};\6
\4\&{default}:\5
${}\\{op}\K\|o;{}$\6
\4${}\}{}$\2\6
\&{if} (\\{verbose})\5
${}\{{}$\1\6
\&{if} ${}(\\{op}\E\\{clear}){}$\1\5
${}\\{printf}(\.{"\{clear\ \%d\ ->\%d\}\\n"},\39\|p,\39\\{cur\_trg});{}$\2\6
\&{else} \&{if} ${}(\\{op}>\\{clear}){}$\1\5
${}\\{printf}(\.{"\{\%s\ \%d\ \%d->\%d\}\\n"},\39\\{sym}[\\{op}],\39\|p,\39%
\\{cur\_src},\39\\{cur\_trg});{}$\2\6
\4${}\}{}$\2\6
${}\\{buf\_ptr}\K\|b;{}$\6
\&{return} \|p;\6
\4${}\}{}$\2\par
\fi

\M{13}\B\X13:Change the source or target address and \PB{\&{goto} \\{restart}}%
\X${}\E{}$\6
${}\{{}$\1\6
\&{if} ${}(\|b+\T{3}\G\\{end\_of\_buffer}){}$\5
${}\{{}$\1\6
${}{*}(\|b-\\{buf\_size})\K{*}\|b,\39{*}(\|b+\T{1}-\\{buf\_size})\K{*}(\|b+%
\T{1}),\39{*}(\|b+\T{2}-\\{buf\_size})\K{*}(\|b+\T{2});{}$\6
\\{read\_it}(\,);\6
${}\|b\MRL{-{\K}}\\{buf\_size};{}$\6
\4${}\}{}$\2\6
${}\|p\K(({*}\|b\AND\T{\^3f})\LL\T{24})+({*}(\|b+\T{1})\LL\T{16})+({*}(\|b+%
\T{2})\LL\T{8})+{*}(\|b+\T{3});{}$\6
\&{if} ${}({*}\|b\AND\T{\^40}){}$\1\5
${}\\{cur\_trg}\K\|p;{}$\2\6
\&{else}\1\5
${}\\{cur\_src}\K\|p;{}$\2\6
${}\|b\MRL{+{\K}}\T{3};{}$\6
\&{goto} \\{restart};\6
\4${}\}{}$\2\par
\U12.\fi

\M{14}\B\X5:Global variables\X${}\mathrel+\E{}$\6
\&{opcode} \\{op};\C{ operation code found by \PB{\\{get\_inst}} }\6
\&{int} \\{verbose}${}\K\T{0}{}$;\C{ set nonzero when debugging }\6
\&{char} ${}{*}\\{sym}[\T{4}]\K\{\.{"sync"},\39\.{"clear"},\39\.{"copy"},\39%
\.{"add"}\};{}$\6
\&{int} \\{cur\_src}${},{}$ \\{cur\_trg};\C{ current source and target
addresses, \PB{\|s} and \PB{\|t} }\par
\fi

\M{15}The first six bytes of the instruction file are, however, special.
Byte~0 is the number $n$ of cells in the largest polyominoes being
enumerated. When a \PB{\\{sync}} is interpreted, {\mc POLYSLAVE}
outputs the current values of \PB{\\{count}[\|j]} for $1\le j\le n$.

Byte 1 is the number of the final row. If this number is $r$, {\mc
POLYSLAVE} will terminate after interpreting the instruction \PB{\\{sync}}~$r$.

Bytes 2--5 specify the (big-endian) number of elements in the \PB{\\{count}}
array.

Initially \PB{$\|s\K\|t\K\T{0}$}, \PB{$\\{count}[\T{0}]\K\T{1}$}, and \PB{%
\\{count}[\|j]} is assumed to be zero
for $1\le j\le n$.

\Y\B\4\X15:Initialize\X${}\E{}$\6
$\\{sprintf}(\\{filename},\39\.{"\%.90s.ck"},\39\\{base\_name});{}$\6
${}\\{ck\_file}\K\\{fopen}(\\{filename},\39\.{"rb"});{}$\6
\&{if} ${}(\R\\{ck\_file}){}$\1\5
\\{panic}(\.{"I\ can't\ open\ the\ ch}\)\.{ecksum\ file"});\2\6
\\{open\_it}(\,);\6
\\{read\_it}(\,);\6
${}\|n\K\\{in}.\\{buf}[\T{4}];{}$\6
${}\\{last\_row}\K\\{in}.\\{buf}[\T{5}];{}$\6
${}\\{prev\_row}\K\T{0};{}$\6
${}\\{slave\_size}\K(\\{in}.\\{buf}[\T{6}]\LL\T{24})+(\\{in}.\\{buf}[\T{7}]\LL%
\T{16})+(\\{in}.\\{buf}[\T{8}]\LL\T{8})+\\{in}.\\{buf}[\T{9}];{}$\6
${}\\{buf\_ptr}\K{\AND}\\{in}.\\{buf}[\T{9}];{}$\6
${}\|w\K\|n+\T{2}-\\{last\_row};{}$\6
\&{if} ${}(\|w<\T{2}\V\|n<\|w+\|w-\T{1}\V\|n>\|w+\|w+\T{126}){}$\1\5
\\{panic}(\.{"Bad\ bytes\ at\ the\ be}\)\.{ginning"});\2\6
${}\\{count}\K{}$(\&{counter} ${}{*}){}$ \\{calloc}${}(\\{slave\_size},\39%
\&{sizeof}(\&{counter}));{}$\6
\&{if} ${}(\R\\{count}){}$\1\5
\\{panic}(\.{"I\ couldn't\ allocate}\)\.{\ the\ counter\ array"});\2\6
${}\\{count}[\T{0}]\K\T{1}{}$;\C{ prime the pump }\6
${}\\{cur\_src}\K\\{cur\_trg}\K\T{0};{}$\6
${}\\{scount}\K{}$(\&{counter} ${}{*}){}$ \\{calloc}${}(\|n+\T{1},\39%
\&{sizeof}(\&{counter}));{}$\6
\&{if} ${}(\R\\{scount}){}$\1\5
\\{panic}(\.{"I\ couldn't\ allocate}\)\.{\ the\ array\ of\ totals}\)\.{"});\2%
\par
\U1.\fi

\M{16}\B\X5:Global variables\X${}\mathrel+\E{}$\6
\&{int} \|n;\C{ the maximum polyomino size of interest }\6
\&{int} \\{last\_row};\C{ the row whose end will complete our mission }\6
\&{int} \\{prev\_row};\C{ the row whose end we've most recently seen }\6
\&{int} \|w;\C{ width of polynominoes being counted             (deduced from %
\PB{\|n} and \PB{\\{last\_row}}) }\6
\&{int} \\{slave\_size};\C{ the number of counters in memory }\6
\&{counter} ${}{*}\\{count}{}$;\C{ base address of The Big Table }\6
\&{counter} ${}{*}\\{scount}{}$;\C{ base address of totals captured at \PB{%
\\{sync}} commands }\par
\fi

\N{1}{17}Servitude. This program is so easy to write, I could even have
done it without the use of literate programming. (But of course
it wouldn't be nearly as much fun without \.{CWEB}.)

\Y\B\4\X17:Interpret the instructions in the input\X${}\E{}$\6
\&{while} (\T{1})\5
${}\{{}$\1\6
${}\|p\K\\{get\_inst}(\,);{}$\6
\&{if} ${}(\\{cur\_trg}+\|p>\\{slave\_size}\W\\{op}\G\\{clear}){}$\1\5
\\{panic}(\.{"Target\ address\ out\ }\)\.{of\ range"});\2\6
\&{if} ${}(\\{cur\_src}+\|p>\\{slave\_size}\W\\{op}\G\\{copy}){}$\1\5
\\{panic}(\.{"Source\ address\ out\ }\)\.{of\ range"});\2\6
\&{switch} (\\{op})\5
${}\{{}$\1\6
\4\&{case} \\{sync}:\5
\X22:Finish a row; \PB{\&{goto} \\{done}} if it was the last\X;\5
\&{break};\6
\4\&{case} \\{clear}:\5
\X19:Clear \PB{\|p} counters\X;\5
\&{break};\6
\4\&{case} \\{copy}:\5
\X20:Copy \PB{\|p} counters\X;\5
\&{break};\6
\4\&{case} \\{add}:\5
\X21:Add \PB{\|p} counters\X;\5
\&{break};\6
\4${}\}{}$\2\6
\4${}\}{}$\2\6
\4\\{done}:\par
\U1.\fi

\M{18}\B\X6:Local variables\X${}\mathrel+\E{}$\6
\&{register} \&{int} \|p;\C{ parameter of the current instruction }\6
\&{register} \&{unsigned} \&{int} \|a;\C{ an accumulator for arithmetic }\par
\fi

\M{19}\B\X19:Clear \PB{\|p} counters\X${}\E{}$\6
\&{for} ${}(\|k\K\T{0};{}$ ${}\|k<\|p;{}$ ${}\|k\PP){}$\1\5
${}\\{count}[\\{cur\_trg}+\|k]\K\T{0}{}$;\2\par
\U17.\fi

\M{20}\B\X20:Copy \PB{\|p} counters\X${}\E{}$\6
\&{for} ${}(\|k\K\T{0};{}$ ${}\|k<\|p;{}$ ${}\|k\PP){}$\1\5
${}\\{count}[\\{cur\_trg}+\|k]\K\\{count}[\\{cur\_src}+\|k]{}$;\2\par
\U17.\fi

\M{21}I wonder what kind of machine language code my \CEE/ compiler
is giving me here, but I'm afraid to look.

\Y\B\4\X21:Add \PB{\|p} counters\X${}\E{}$\6
\&{for} ${}(\|k\K\T{0};{}$ ${}\|k<\|p;{}$ ${}\|k\PP){}$\5
${}\{{}$\1\6
${}\|a\K\\{count}[\\{cur\_trg}+\|k]+\\{count}[\\{cur\_src}+\|k];{}$\6
\&{if} ${}(\|a\G\|m){}$\1\5
${}\|a\MRL{-{\K}}\|m;{}$\2\6
${}\\{count}[\\{cur\_trg}+\|k]\K\|a;{}$\6
\4${}\}{}$\2\par
\U17.\fi

\M{22}The \PB{\\{sync}} instruction, at least, gives me a little chance to be
creative,
especially with respect to checking the sanity of the source file.

\Y\B\4\X22:Finish a row; \PB{\&{goto} \\{done}} if it was the last\X${}\E{}$\6
\X23:Check that \PB{\|p} has the correct value\X;\6
\X24:Output the relevant counters for completed polyominoes\X;\6
\&{for} ${}(\|k\K\T{1};{}$ ${}\|k\Z\|n;{}$ ${}\|k\PP){}$\1\5
${}\\{scount}[\|k]\K\\{count}[\|k];{}$\2\6
\&{if} ${}(\|p\E\\{last\_row}){}$\1\5
\&{goto} \\{done};\2\par
\U17.\fi

\M{23}\B\X23:Check that \PB{\|p} has the correct value\X${}\E{}$\6
\&{if} ${}(\|p\E\T{255}){}$\1\5
\X26:Go into special shutdown mode\X;\2\6
\&{if} ${}(\|p\E\T{1}){}$\1\5
\\{panic}(\.{"File\ read\ error"});\C{ see \PB{\\{read\_it}} }\2\6
\&{if} ${}(\R\\{prev\_row}){}$\5
${}\{{}$\1\6
\&{if} ${}(\|p\I\|w+\T{1}){}$\1\5
\\{panic}(\.{"Bad\ first\ sync"});\2\6
\4${}\}{}$\5
\2\&{else} \&{if} ${}(\|p\I\\{prev\_row}+\T{1}){}$\1\5
\\{panic}(\.{"Out\ of\ sync"});\2\6
${}\\{prev\_row}\K\|p{}$;\par
\U22.\fi

\M{24}\B\X24:Output the relevant counters for completed polyominoes\X${}\E{}$\6
$\\{printf}(\.{"Polyominoes\ that\ sp}\)\.{an\ \%dx\%d\ rectangles\ }\)\.{(mod\
\%u):\\n"},\39\|p-\T{1},\39\|w,\39\|m);{}$\6
${}\\{fprintf}(\\{math\_file},\39\.{"p[\%d,\%d,\%u]=\{0"},\39\|p-\T{1},\39\|w,%
\39\|m);{}$\6
\&{for} ${}(\|k\K\T{2};{}$ ${}\|k<\|w+\|p-\T{2};{}$ ${}\|k\PP){}$\1\5
${}\\{fprintf}(\\{math\_file},\39\.{",0"});{}$\2\6
\&{for} ( ; ${}\|k\Z\|n;{}$ ${}\|k\PP){}$\5
${}\{{}$\1\6
\&{if} ${}(\\{count}[\|k]\G\\{scount}[\|k]){}$\1\5
${}\|a\K\\{count}[\|k]-\\{scount}[\|k];{}$\2\6
\&{else}\1\5
${}\|a\K\\{count}[\|k]+\|m-\\{scount}[\|k];{}$\2\6
${}\\{printf}(\.{"\ \%d:\%d"},\39\|a,\39\|k);{}$\6
${}\\{fprintf}(\\{math\_file},\39\.{",\%d"},\39\|a);{}$\6
\4${}\}{}$\2\6
\\{printf}(\.{"\\n"});\6
\\{fflush}(\\{stdout});\6
${}\\{fprintf}(\\{math\_file},\39\.{"\}\\n"}){}$;\par
\U22.\fi

\M{25}\B\X25:Print statistics\X${}\E{}$\6
$\\{printf}(\.{"All\ done!\ Final\ tot}\)\.{als\ (mod\ \%u):\\n"},\39\|m);{}$\6
\&{for} ${}(\|k\K\|w+\|w-\T{1};{}$ ${}\|k\Z\|n;{}$ ${}\|k\PP){}$\5
${}\{{}$\1\6
${}\\{printf}(\.{"\ \%d:\%d"},\39\\{count}[\|k],\39\|k);{}$\6
\4${}\}{}$\2\6
\\{printf}(\.{"\\n"});\6
\\{close\_it}(\,);\par
\U1.\fi

\N{1}{26}Checkpointing. {\mc POLYNUM} issues the special command \PB{%
\\{sync}}~255
when it wants to pause for breath and shore up its knowledge.
Therefore, if we see that instruction,  we must immediately
dump all the counters into a temporary file.
A special variant of this program is able to read that file
and reconstitute all the data, as if there had been no break in
the action. (See the change file \.{polyslave-restart.ch} for details.)

\Y\B\4\X26:Go into special shutdown mode\X${}\E{}$\6
${}\{{}$\1\6
\\{close\_it}(\,);\6
\\{printf}(\.{"Checkpoint\ stop:\ Af}\)\.{ter\ processing\ with\ }\)\.{all\
desired\ moduli,\\}\)\.{n"});\6
\\{printf}(\.{"\ please\ resume\ with}\)\.{\ polynum-restart\ and}\)\.{\
polyslave-restart.\\}\)\.{n"});\6
${}\\{sprintf}(\\{filename},\39\.{"\%.90s-\%u.dump"},\39\\{base\_name},\39%
\|m);{}$\6
${}\\{out\_file}\K\\{fopen}(\\{filename},\39\.{"wb"});{}$\6
\&{if} ${}(\R\\{out\_file}){}$\1\5
\\{panic}(\.{"I\ can't\ open\ the\ du}\)\.{mp\ file"});\2\6
\X27:Dump all information needed to restart\X;\6
\\{exit}(\T{1});\6
\4${}\}{}$\2\par
\U23.\fi

\M{27}\B\X27:Dump all information needed to restart\X${}\E{}$\6
$\\{dump\_data}[\T{0}]\K\|n;{}$\6
${}\\{dump\_data}[\T{1}]\K\|w;{}$\6
${}\\{dump\_data}[\T{2}]\K\|m;{}$\6
${}\\{dump\_data}[\T{3}]\K\\{slave\_size};{}$\6
${}\\{dump\_data}[\T{4}]\K\\{prev\_row};{}$\6
\&{if} ${}(\\{fwrite}(\\{dump\_data},\39{}$\&{sizeof}(\&{unsigned} \&{int})${},%
\39\T{5},\39\\{out\_file})\I\T{5}){}$\1\5
\\{panic}(\.{"Bad\ write\ at\ beginn}\)\.{ing\ of\ dump"});\2\6
\&{if} ${}(\\{fwrite}(\\{scount},\39\&{sizeof}(\&{counter}),\39\|n+\T{1},\39%
\\{out\_file})\I\|n+\T{1}){}$\1\5
\\{panic}(\.{"Couldn't\ dump\ the\ s}\)\.{ubtotals"});\2\6
\&{if} ${}(\\{fwrite}(\\{count},\39\&{sizeof}(\&{counter}),\39\\{slave\_size},%
\39\\{out\_file})\I\\{slave\_size}){}$\1\5
\\{panic}(\.{"Couldn't\ dump\ the\ c}\)\.{ounters"});\2\6
${}\\{printf}(\.{"[\%u\ bytes\ written\ o}\)\.{n\ file\ \%s.]\\n"},\39%
\\{ftell}(\\{out\_file}),\39\\{filename}){}$;\par
\U26.\fi

\M{28}\B\X5:Global variables\X${}\mathrel+\E{}$\6
\&{unsigned} \&{int} \\{dump\_data}[\T{5}];\C{ parameters needed to restart }\6
\&{FILE} ${}{*}\\{out\_file}{}$;\par
\fi

\M{29}For the record, here are three shell scripts called \.{nums},
\.{slaves}, and \.{slaves-restart}, which were used to
run {\mc POLYNUM} and {\mc POLYSLAVE} when $n=47$:
\bigskip
\hbox{\qquad\tt\hbox to 9em{nums\hfil}\vtop{\halign{\.{#}\hfil\cr
\#!/bin/sh\cr
if [ \$\# -ne 3 ]; then\cr
\quad  echo "Usage: nums width configs counts"\cr
\quad  exit 255\cr
fi\cr
\noalign{\smallskip}
time polynum 47 \$1 \$2 \$3 /home/tmp/poly47-\$1\cr
slaves \$1\cr
while [ \$? = 1 ]; do\cr
\quad  mv /home/tmp/poly47-\$1.dump /home/tmp/poly47-\$1.dump\~\cr
\quad  time polynum-restart 47 \$1 \$2 \$3 /home/tmp/poly47-\$1\cr
\quad  slaves-restart \$1\cr
done\cr
}}}
\bigskip
\hbox{\qquad\tt\hbox to 9em{slaves\hfil}\vtop{\halign{\.{#}\hfil\cr
\#!/bin/sh\cr
for m in 2147483648 2147483647 2147483645; do\cr
\quad  time polyslave /home/tmp/poly47-\$1 \$m\cr
done\cr
}}}
\bigskip
\hbox{\qquad\tt\hbox to 9em{slaves-restart\hfil}\vtop{\halign{\.{#}\hfil\cr
\#!/bin/sh\cr
for m in 2147483648 2147483647 2147483645; do\cr
\quad  cp /home/tmp/poly47-\$1-\$m.dump /home/tmp/poly47-\$1-\$m.dump\~\cr
\quad  time polyslave-restart /home/tmp/poly47-\$1 \$m\cr
done\cr
}}}
\bigskip
And here is the {\it Mathematica\/} script used to convert modular numbers
to multiprecise integers:
$$
\vbox{\halign{\.{#}\hfil\cr
(* for Chinese Remainders, say for example\cr
\quad   chinese[\{13,17,19\}]\cr
\quad   x=cdecode[\{1,2,3\}]\cr
and x (= 4031) will satisfy Mod[x,13]=1, Mod[x,17]=2, Mod[x,19]=3 *)\cr
\noalign{\smallskip}
chinese[l\_]:=Block[\{\},chinmod=Apply[Times,l];\cr
\quad chinlist=Table[(chinmod/l[[k]])PowerMod[chinmod/l[[k]],-1,l[[k]]],\cr
\qquad \{k,Length[l]\}]]\cr
cdecode[l\_]:=Mod[chinlist.l,chinmod]\cr
\noalign{\smallskip}
m=2\^31\cr
chinese[\{m,m-1,m-3\}]\cr
fn[a\_,b\_]:="poly47-"<>a<>"-"<>ToString[m-b]<>".m"\cr
squash[a\_,w\_]:=Block[\{\},Get[fn[a,0]];Get[fn[a,1]];Get[fn[a,3]];\cr
\quad Do[q[h,w]= cdecode[\{p[h,w,m],p[h,w,m-1],p[h,w,m-3]\}],\{h,w,48-w\}];\cr
\quad Save["poly47-"<>a<>".m",q];\cr
\quad Clear[q]]\cr
}}$$


\fi

\N{1}{30}Index.
\fi

\inx
\fin
\con
