\input cwebmac
\input epsf
\font\celtica=celtica13 \font\celticb=celticb13
\datethis

\N{1}{1}Intro. This quick-and-dirty program prepares \TeX\ files for use with
the weird fonts {\mc CELTICA} and {\mc CELTICB}. You can use it to print
amazing pictures that look like stylized Celtic knots. Sometimes the
resulting pictures look very elegant.
(Plug: You can see the author's most elegant example, so far,
in Chapter 46 of the book {\sl Selected
Papers on Fun and Games}, published by CSLI in 2010.)

The ``knots'' we print consist of one or more closed loops in which
no three points are concurrent. Therefore we can draw them in such a way
that the paths go alternately over-under-over-under, etc., whenever
they cross. The loops consist of segments that cut a square grid
either at corner points or at midpoints between adjacent corners.
So we can think of the total picture as composed of square tiles, where
each tile has eight possible entry or exit points, numbered thus:
$$\centerline{\epsfbox{celtic-picture.1}}$$

There are three kinds of tiles, each representable as a sequence
of four characters:

\smallskip\item{$\bullet$} A blank tile, represented by four blanks.
\smallskip\item{$\bullet$} A tile with a single segment from $i$ to~$j$,
where $i<j$, represented by $ij$ and two blanks.
\smallskip\item{$\bullet$} A tile with segments from $i$ to~$j$ and
from $i'$ to~$j'$, where $i<j$ and $i'<j'$ and $i<i'$, represented by $iji'j'$.

\smallskip\noindent The segments from $i$ to $j$ are also subject
to several additional restrictions:
\smallskip\item{$\bullet$} $i$ and $j$ cannot be adjacent on the periphery.
\smallskip\item{$\bullet$} $i$ and $j$ cannot both be even.
\smallskip\noindent Thus, for example, if $i=0$ the only possibilities
for $j$ are 3 and~5. But if $i=1$ we can have $j=3$, 4, 5, 6,~or~7.
It follows that 14 different one-segment tiles are legal,
and there are 47 with two segments.

The input to this program, for an $m\times n$ picture,
consists of $m$ lines of $5n$ characters, where each line contains
$n$ tile specifications separated by blanks and followed by a period.
For example, the simple $3\times4$ input
$$\def\\.{}
\vbox{\tt\obeyspaces\halign{\\#\cr
.35   57   35   57  .\cr
.13   1537 1537 17  .\cr
.     13   17       .\cr}}$$
yields the nice little picture `$\vcenter{
\begingroup\celtica
\catcode`\?=\active \def?#1?{\hskip#1em}
\catcode`\-=\active \def-#1#2#3{\celtica\char'#1#2#3}
\catcode`\+=\active \def+#1#2#3{\celticb\char'#1#2#3}
\offinterlineskip\baselineskip=1em
\let\par=\cr \obeylines \halign{#\hfil
-067-077-067-077
-051+005-012+061
?1?-051+061
}\endgroup
}$'.

The rules are illustrated more fully by the following $6\times7$ example,
which was used in the author's initial tests:
$$\def\\.{}
\vbox{\tt\obeyspaces\halign{\\#\cr
.                    35   57       .\cr
.     35   57   35   1537 1357 57  .\cr
.     1435 1657 15   1435 1637 17  .\cr
.35   1725 0514 16   1325 0537 57  .\cr
.13   1735 1725 03   17   13   17  .\cr
.     13   17                      .\cr}}$$
From that input (on \PB{\\{stdin}}), the output of this program (on \PB{%
\\{stdout}})
is a \TeX\ file that prints a ``poodle'':
$$\advance\abovedisplayskip-8pt\vcenter{
\begingroup\celtica
\catcode`\?=\active \def?#1?{\hskip#1em}
\catcode`\-=\active \def-#1#2#3{\celtica\char'#1#2#3}
\catcode`\+=\active \def+#1#2#3{\celticb\char'#1#2#3}
\offinterlineskip\baselineskip=1em
\let\par=\cr \obeylines \halign{#\hfil
?4?-067-077
?1?-067-077-067-002+102-077
?1?-341+326-055+340-226+061
-067-137+153-057+315-233-077
-051+107-137+045+061-051+061
?1?-051+061
}\endgroup
}$$

\fi

\M{2}Whenever a tile contains a segment through some boundary point,
the neighboring tiles must also contain a segment through that common point.
For example, tile `\.{35}' can be used only if its neighbor to the right
uses its point~\.7, and only if its neighbor below uses its point~\.1.
More significantly, if a tile uses a corner point, it has three neighbors
that touch the same corner, and all three of them must use that corner.
Paths therefore always make an `X' crossing, at right angles, whenever they
pass through a corner of the grid.

All regions of the final illustration are filled in with black, if they
don't lie completely outside of all paths.

\fi

\M{3}OK, let's get going. This program ought to be fun, once we get through
the tedious details of preparing font tables and of
reading/checking the input.

\Y\B\4\D$\\{maxm}$ \5
\T{100}\C{ at most this many rows; mustn't exceed 4096 }\par
\B\4\D$\\{maxn}$ \5
\T{100}\C{ at most this many columns; mustn't exceed 4096 }\par
\B\4\D$\\{bufsize}$ \5
$\T{5}*\\{maxn}+{}$\T{2}\par
\Y\B\8\#\&{include} \.{<stdio.h>}\6
\8\#\&{include} \.{<stdlib.h>}\6
\&{char} \\{buf}[\\{bufsize}]${},{}$ \\{entry}[\T{8}];\6
\&{int} \|a[\\{maxm}][\\{maxn}];\C{ the input }\6
\&{int} \|b[\\{maxm}][\\{maxn}];\C{ endpoints touched in each tile }\6
\&{int} \\{codetable}[\T{\^7778}];\C{ mapping from tiles to font positions }\6
\&{char} \\{bw}[\\{maxm}][\\{maxn}][\T{8}];\C{ black/white coloring of regions
}\6
\&{int} \\{inout}[\\{maxm}][\\{maxn}][\T{8}];\C{ inside/outside coloring of
regions }\7
\X11:Declare the magic tables\X;\7
\\{main}(\,)\1\1\2\2\6
${}\{{}$\1\6
\&{register} \&{int} \|i${},{}$ \|j${},{}$ \|k${},{}$ \\{ii}${},{}$ %
\\{jj}${},{}$ \\{kk}${},{}$ \|m${},{}$ \|n${},{}$ \|s${},{}$ \|t;\7
\X4:Initialize \PB{\\{codetable}}\X;\6
\X5:Read the input into \PB{\|a}, and check it for consistency\X;\6
\X8:Do the black/white coloring\X;\6
\X9:Do the inside/outside coloring\X;\6
\X13:Produce the output\X;\6
\4${}\}{}$\2\par
\fi

\M{4}(Begin tedium.)
Fonts {\mc CELTICA} and {\mc CELTICB} have a peculiar encoding scheme,
not terribly systematic, governed by a mapping from tile specs (in
hexadecimal) to character positions (in octal).

\Y\B\4\X4:Initialize \PB{\\{codetable}}\X${}\E{}$\6
\&{for} ${}(\|i\K\T{1};{}$ ${}\|i<\T{\^7778};{}$ ${}\|i\PP){}$\1\5
${}\\{codetable}[\|i]\K{-}\T{1};{}$\2\6
${}\\{codetable}[\T{\^1537}]\K\T{0}{}$;\5
${}\\{codetable}[\T{0}]\K\T{\~40};{}$\6
${}\\{codetable}[\T{\^0300}]\K\T{\~44},\39\\{codetable}[\T{\^0500}]\K\T{\~46},%
\39\\{codetable}[\T{\^1300}]\K\T{\~50},\39\\{codetable}[\T{\^1400}]\K\T{\~52},%
\39\\{codetable}[\T{\^1500}]\K\T{\~54},\39\\{codetable}[\T{\^1600}]\K\T{\~56},%
\39\\{codetable}[\T{\^1700}]\K\T{\~60},\39\\{codetable}[\T{\^2500}]\K\T{\~62},%
\39\\{codetable}[\T{\^2700}]\K\T{\~64},\39\\{codetable}[\T{\^3500}]\K\T{\~66},%
\39\\{codetable}[\T{\^3600}]\K\T{\~70},\39\\{codetable}[\T{\^3700}]\K\T{\~72},%
\39\\{codetable}[\T{\^4700}]\K\T{\~74},\39\\{codetable}[\T{\^5700}]\K\T{\~76},%
\39\\{codetable}[\T{\^1357}]\K\T{\~100},\39\\{codetable}[\T{\^1735}]\K\T{%
\~104},\39\\{codetable}[\T{\^0513}]\K\T{\~110},\39\\{codetable}[\T{\^2735}]\K%
\T{\~114},\39\\{codetable}[\T{\^1457}]\K\T{\~120},\39\\{codetable}[\T{\^1736}]%
\K\T{\~124},\39\\{codetable}[\T{\^0357}]\K\T{\~130},\39\\{codetable}[\T{%
\^1725}]\K\T{\~134},\39\\{codetable}[\T{\^1347}]\K\T{\~140},\39\\{codetable}[%
\T{\^1635}]\K\T{\~144},\39\\{codetable}[\T{\^0514}]\K\T{\~150},\39%
\\{codetable}[\T{\^2736}]\K\T{\~154},\39\\{codetable}[\T{\^0347}]\K\T{\~160},%
\39\\{codetable}[\T{\^1625}]\K\T{\~164},\39\\{codetable}[\T{\^0314}]\K\T{%
\~170},\39\\{codetable}[\T{\^2536}]\K\T{\~174},\39\\{codetable}[\T{\^0547}]\K%
\T{\~200},\39\\{codetable}[\T{\^1627}]\K\T{\~204},\39\\{codetable}[\T{\^0315}]%
\K\T{\~210},\39\\{codetable}[\T{\^2537}]\K\T{\~214},\39\\{codetable}[\T{%
\^1547}]\K\T{\~220},\39\\{codetable}[\T{\^1637}]\K\T{\~224},\39\\{codetable}[%
\T{\^0537}]\K\T{\~230},\39\\{codetable}[\T{\^1527}]\K\T{\~234},\39%
\\{codetable}[\T{\^1437}]\K\T{\~240},\39\\{codetable}[\T{\^1536}]\K\T{\~244},%
\39\\{codetable}[\T{\^0316}]\K\T{\~250},\39\\{codetable}[\T{\^0325}]\K\T{%
\~254},\39\\{codetable}[\T{\^2547}]\K\T{\~260},\39\\{codetable}[\T{\^1647}]\K%
\T{\~264},\39\\{codetable}[\T{\^0527}]\K\T{\~270},\39\\{codetable}[\T{\^1425}]%
\K\T{\~274},\39\\{codetable}[\T{\^1436}]\K\T{\~300},\39\\{codetable}[\T{%
\^0536}]\K\T{\~304},\39\\{codetable}[\T{\^0317}]\K\T{\~310},\39\\{codetable}[%
\T{\^1325}]\K\T{\~314},\39\\{codetable}[\T{\^3547}]\K\T{\~320},\39%
\\{codetable}[\T{\^1657}]\K\T{\~324},\39\\{codetable}[\T{\^0517}]\K\T{\~330},%
\39\\{codetable}[\T{\^1327}]\K\T{\~334},\39\\{codetable}[\T{\^1435}]\K\T{%
\~340},\39\\{codetable}[\T{\^3657}]\K\T{\~344},\39\\{codetable}[\T{\^0516}]\K%
\T{\~350},\39\\{codetable}[\T{\^0327}]\K\T{\~354},\39\\{codetable}[\T{\^1425}]%
\K\T{\~360},\39\\{codetable}[\T{\^3647}]\K\T{\~364}{}$;\par
\U3.\fi

\M{5}(More tedium. I do try to check carefully for errors, because the task
of preparing the input is even more tedious than the task of writing
this code.)

The rows are numbered from 0 to \PB{$\|m-\T{1}$}, and the columns from 0 to %
\PB{$\|n-\T{1}$}.

\Y\B\4\X5:Read the input into \PB{\|a}, and check it for consistency\X${}\E{}$\6
\&{for} ${}(\|m\K\T{0};{}$  ; ${}\|m\PP){}$\5
${}\{{}$\1\6
\&{if} ${}(\R\\{fgets}(\\{buf},\39\\{bufsize},\39\\{stdin})){}$\1\5
\&{break};\2\6
\&{for} ${}(\|j\K\T{0};{}$  ; ${}\|j\PP){}$\5
${}\{{}$\1\6
${}\|k\K\T{5}*\|j;{}$\6
\&{if} ${}(\|j\E\|n\W\|m>\T{0}){}$\5
${}\{{}$\1\6
${}\\{fprintf}(\\{stderr},\39\.{"Missing\ `.'\ at\ the\ }\)\.{end\ of\ row\ %
\%d!\\n"},\39\|m);{}$\6
${}\\{exit}({-}\T{1});{}$\6
\4${}\}{}$\2\6
\X6:Parse the entry for row \PB{\|m} and column \PB{\|j}\X;\6
\&{if} ${}(\\{buf}[\|k+\T{4}]\E\.{'.'}){}$\5
${}\{{}$\1\6
\&{if} ${}(\|m\E\T{0}){}$\1\5
${}\|n\K\|j+\T{1};{}$\2\6
\&{else} \&{if} ${}(\|n\I\|j+\T{1}){}$\5
${}\{{}$\1\6
${}\\{fprintf}(\\{stderr},\39\.{"Premature\ `.'\ in\ ro}\)\.{w\ \%d!\\n"},\39%
\|m);{}$\6
${}\\{exit}({-}\T{2});{}$\6
\4${}\}{}$\2\6
\&{break};\6
\4${}\}{}$\5
\2\&{else} \&{if} ${}(\\{buf}[\|k+\T{4}]\I\.{'\ '}){}$\5
${}\{{}$\1\6
${}\\{fprintf}(\\{stderr},\39\.{"Tile\ spec\ in\ row\ \%d}\)\.{,\ col\ \%d\ not%
\ followe}\)\.{d\ by\ blank!\\n"},\39\|m,\39\|j);{}$\6
${}\\{exit}({-}\T{5});{}$\6
\4${}\}{}$\2\6
\4${}\}{}$\2\6
\&{continue};\6
\4\\{badentry}:\5
${}\\{fprintf}(\\{stderr},\39\.{"Bad\ entry\ (\%s)\ in\ r}\)\.{ow\ \%d\ and\
column\ \%d!}\)\.{\\n"},\39\\{entry},\39\|m,\39\|j);{}$\6
${}\\{exit}({-}\T{3});{}$\6
\4${}\}{}$\2\6
\&{if} ${}(\|m\E\T{0}){}$\5
${}\{{}$\1\6
${}\\{fprintf}(\\{stderr},\39\.{"There\ was\ no\ input!}\)\.{\\n"});{}$\6
${}\\{exit}({-}\T{4});{}$\6
\4${}\}{}$\2\6
${}\\{fprintf}(\\{stderr},\39\.{"OK,\ I've\ successful}\)\.{ly\ read\ \%d\ rows%
\ and\ }\)\.{\%d\ columns.\\n"},\39\|m,\39\|n);{}$\6
\X7:Check for consistency\X;\par
\U3.\fi

\M{6}\B\X6:Parse the entry for row \PB{\|m} and column \PB{\|j}\X${}\E{}$\6
\&{for} ${}(\\{jj}\K\T{0};{}$ ${}\\{jj}<\T{4};{}$ ${}\\{jj}\PP){}$\1\5
${}\\{entry}[\\{jj}]\K\\{buf}[\T{5}*\|j+\\{jj}];{}$\2\6
\&{if} ${}(\\{entry}[\T{0}]\E\.{'\ '}){}$\5
${}\{{}$\1\6
\&{if} ${}(\\{entry}[\T{1}]\I\.{'\ '}\V\\{entry}[\T{2}]\I\.{'\ '}\V\\{entry}[%
\T{3}]\I\.{'\ '}){}$\1\5
\&{goto} \\{badentry};\2\6
\&{else}\1\5
${}\|a[\|m][\|j]\K\T{0};{}$\2\6
\4${}\}{}$\5
\2\&{else}\5
${}\{{}$\1\6
\&{if} ${}(\\{entry}[\T{0}]<\.{'0'}\V\\{entry}[\T{0}]>\.{'7'}){}$\1\5
\&{goto} \\{badentry};\2\6
\&{if} ${}(\\{entry}[\T{1}]<\.{'0'}\V\\{entry}[\T{1}]>\.{'7'}){}$\1\5
\&{goto} \\{badentry};\2\6
${}\|a[\|m][\|j]\K((\\{entry}[\T{0}]-\.{'0'})\LL\T{12})+((\\{entry}[\T{1}]-%
\.{'0'})\LL\T{8});{}$\6
${}\|b[\|m][\|j]\K(\T{1}\LL(\\{entry}[\T{0}]-\.{'0'}))+(\T{1}\LL(\\{entry}[%
\T{1}]-\.{'0'}));{}$\6
\&{if} ${}(\\{entry}[\T{2}]\E\.{'\ '}){}$\5
${}\{{}$\1\6
\&{if} ${}(\\{entry}[\T{3}]\I\.{'\ '}){}$\1\5
\&{goto} \\{badentry};\2\6
\4${}\}{}$\5
\2\&{else}\5
${}\{{}$\1\6
\&{if} ${}(\\{entry}[\T{2}]<\.{'0'}\V\\{entry}[\T{2}]>\.{'7'}){}$\1\5
\&{goto} \\{badentry};\2\6
\&{if} ${}(\\{entry}[\T{3}]<\.{'0'}\V\\{entry}[\T{3}]>\.{'7'}){}$\1\5
\&{goto} \\{badentry};\2\6
${}\|a[\|m][\|j]\MRL{+{\K}}((\\{entry}[\T{2}]-\.{'0'})\LL\T{4})+(\\{entry}[%
\T{3}]-\.{'0'});{}$\6
${}\|b[\|m][\|j]\MRL{+{\K}}(\T{1}\LL(\\{entry}[\T{2}]-\.{'0'}))+(\T{1}\LL(%
\\{entry}[\T{3}]-\.{'0'}));{}$\6
\4${}\}{}$\2\6
\4${}\}{}$\2\6
\&{if} ${}(\\{codetable}[\|a[\|m][\|j]]<\T{0}){}$\5
${}\{{}$\1\6
${}\\{fprintf}(\\{stderr},\39\.{"Sorry,\ \%s\ isn't\ a\ l}\)\.{egal\ tile\ (row%
\ \%d,\ c}\)\.{ol\ \%d)!\\n"},\39\\{entry},\39\|m,\39\|j);{}$\6
${}\\{exit}({-}\T{4});{}$\6
\4${}\}{}$\2\par
\U5.\fi

\M{7}\B\D$\\{eqbit}(\|k,\\{kk})$ \5
$(((\|i\GG\|k)\XOR(\\{ii}\GG\\{kk}))\AND\T{1}{}$)\par
\Y\B\4\X7:Check for consistency\X${}\E{}$\6
$\|t\K\T{0};{}$\6
\&{for} ${}(\|j\K\T{0};{}$ ${}\|j\Z\|m;{}$ ${}\|j\PP){}$\1\6
\&{for} ${}(\\{jj}\K\T{0};{}$ ${}\\{jj}<\|n;{}$ ${}\\{jj}\PP){}$\5
${}\{{}$\1\6
${}\|i\K(\|j>\T{0}\?\|b[\|j-\T{1}][\\{jj}]:\T{0});{}$\6
${}\\{ii}\K(\|j<\|m\?\|b[\|j][\\{jj}]:\T{0});{}$\6
\&{if} ${}(\\{eqbit}(\T{4},\39\T{2})+\\{eqbit}(\T{5},\39\T{1})+\\{eqbit}(\T{6},%
\39\T{0})){}$\5
${}\{{}$\1\6
${}\\{fprintf}(\\{stderr},\39\.{"Inconsistent\ tiles\ }\)\.{\%04x/\%04x\ (row\ %
\%d,\ c}\)\.{ol\ \%d)!\\n"},\39\|j>\T{0}\?\|a[\|j-\T{1}][\\{jj}]:\T{0},\39\|a[%
\|j][\\{jj}],\39\|j,\39\\{jj});{}$\6
${}\|t\PP;{}$\6
\4${}\}{}$\2\6
\4${}\}{}$\2\2\6
\&{for} ${}(\\{jj}\K\T{0};{}$ ${}\\{jj}\Z\|n;{}$ ${}\\{jj}\PP){}$\1\6
\&{for} ${}(\|j\K\T{0};{}$ ${}\|j<\|m;{}$ ${}\|j\PP){}$\5
${}\{{}$\1\6
${}\|i\K(\\{jj}>\T{0}\?\|b[\|j][\\{jj}-\T{1}]:\T{0});{}$\6
${}\\{ii}\K(\\{jj}<\|n\?\|b[\|j][\\{jj}]:\T{0});{}$\6
\&{if} ${}(\\{eqbit}(\T{2},\39\T{0})+\\{eqbit}(\T{3},\39\T{7})+\\{eqbit}(\T{4},%
\39\T{6})){}$\5
${}\{{}$\1\6
${}\\{fprintf}(\\{stderr},\39\.{"Inconsistent\ tiles\ }\)\.{\%04x,\%04x\ (row\ %
\%d,\ c}\)\.{ol\ \%d)!\\n"},\39\\{jj}>\T{0}\?\|a[\|j][\\{jj}-\T{1}]:\T{0},\39%
\|a[\|j][\\{jj}],\39\|j,\39\\{jj});{}$\6
${}\|t\PP;{}$\6
\4${}\}{}$\2\6
\4${}\}{}$\2\2\6
\&{if} (\|t)\5
${}\{{}$\1\6
${}\\{fprintf}(\\{stderr},\39\.{"Sorry,\ I\ can't\ go\ o}\)\.{n\ (errs=\%d).%
\\n"},\39\|t);{}$\6
${}\\{exit}({-}\T{69});{}$\6
\4${}\}{}$\2\par
\U5.\fi

\M{8}OK, now the fun begins: We've got decent input, and we want to figure out
how to typeset it.

The given loops partition the plane into regions, and the key idea is to
assign ``colors'' to each region of each tile. We use two different
bicolorings: (1) Regions are either black or white, where the color changes
at each boundary edge between regions. (2) Regions are either inside or
outside of the total picture. These two colorings are related by the
fact that outside regions are always white.

A blank tile has only one region. A tile $ij$ has two. And a tile
$iji'j'$ has either three or four, depending on whether $ij$ and $i'j'$
intersect. We unify these cases by considering eight subregions
along the boundary, namely
$0\,.\,.\,1$,
$1\,.\,.\,2$, \dots,
$7\,.\,.\,0$, some of which are known to be identical.

The black/white coloring is easily done in one pass. (This code is
in fact overkill.)

\Y\B\4\X8:Do the black/white coloring\X${}\E{}$\6
\&{for} ${}(\|i\K\T{0};{}$ ${}\|i<\|m;{}$ ${}\|i\PP){}$\1\6
\&{for} ${}(\|j\K\T{0};{}$ ${}\|j<\|n;{}$ ${}\|j\PP){}$\5
${}\{{}$\1\6
${}\\{bw}[\|i][\|j][\T{0}]\K(\|i\E\T{0}\?\T{0}:\\{bw}[\|i-\T{1}][\|j][%
\T{5}]);{}$\6
\&{for} ${}(\|k\K\T{1};{}$ ${}\|k<\T{8};{}$ ${}\|k\PP){}$\1\5
${}\\{bw}[\|i][\|j][\|k]\K(\|b[\|i][\|j]\AND(\T{1}\LL\|k)\?\T{1}:\T{0})\XOR%
\\{bw}[\|i][\|j][\|k-\T{1}];{}$\2\6
\4${}\}{}$\2\2\par
\U3.\fi

\M{9}The inside/outside coloring is trickier, because connectivity to the
outside can twist around, and because three-region tiles behave
differently from four-region tiles.

The following algorithm is essentially a depth-first search to find
all subregions that are connected to the upper left corner. A~stack
is maintained within the data structure. At the end of the process,
\PB{\\{inout}[\|i][\|j][\|k]} is nonzero if and only if that subregion is
outside.

\Y\B\4\D$\\{pop}(\|i,\|j,\|k)$ \5
$\|i\K\|s\GG\T{16},\39\|j\K(\|s\GG\T{4})\AND\T{\^fff},\39\|k\K\|s\AND\T{\^7},%
\39\|s\K\\{inout}[\|i][\|j]{}$[\|k]\par
\B\4\D$\\{push}(\\{ii},\\{jj},\\{kk})$ \6
${}\{{}$\5
\1\&{if} ${}(\\{inout}[\\{ii}][\\{jj}][(\\{kk})\AND\T{\^7}]\E\T{0}){}$\1\5
${}\\{inout}[\\{ii}][\\{jj}][(\\{kk})\AND\T{\^7}]\K\|s,\39\|s\K((\\{ii})\LL%
\T{16})+((\\{jj})\LL\T{4})+((\\{kk})\AND\T{\^7}){}$;\5
\2${}\}{}$\2\par
\Y\B\4\X9:Do the inside/outside coloring\X${}\E{}$\6
$\\{inout}[\T{0}][\T{0}][\T{0}]\K{-}\T{1};{}$\6
\&{for} ${}(\|s\K\T{0};{}$ ${}\|s\G\T{0};{}$ \,)\5
${}\{{}$\1\6
${}\\{pop}(\|i,\39\|j,\39\|k);{}$\6
\X10:Push all unseen neighbors of subregion \PB{[\|i][\|j][\|k]} onto the stack%
\X;\6
\4${}\}{}$\2\par
\U3.\fi

\M{10}The neighbors of a subregion within a tile are either in an adjacent tile
or in the same tile.

\Y\B\4\X10:Push all unseen neighbors of subregion \PB{[\|i][\|j][\|k]} onto the
stack\X${}\E{}$\6
\&{switch} (\|k)\5
${}\{{}$\1\6
\4\&{case} \T{0}:\5
\&{case} \T{1}:\5
\&{if} ${}(\|i>\T{0}){}$\1\5
${}\\{push}(\|i-\T{1},\39\|j,\39\T{5}-\|k){}$;\5
\2\&{break};\6
\4\&{case} \T{2}:\5
\&{case} \T{3}:\5
\&{if} ${}(\|j<\|n-\T{1}){}$\1\5
${}\\{push}(\|i,\39\|j+\T{1},\39\T{9}-\|k){}$;\5
\2\&{break};\6
\4\&{case} \T{4}:\5
\&{case} \T{5}:\5
\&{if} ${}(\|i<\|m-\T{1}){}$\1\5
${}\\{push}(\|i+\T{1},\39\|j,\39\T{5}-\|k){}$;\5
\2\&{break};\6
\4\&{case} \T{6}:\5
\&{case} \T{7}:\5
\&{if} ${}(\|j>\T{0}){}$\1\5
${}\\{push}(\|i,\39\|j-\T{1},\39\T{9}-\|k){}$;\5
\2\&{break};\6
\4${}\}{}$\2\6
\&{if} ${}((\|b[\|i][\|j]\AND(\T{1}\LL\|k))\E\T{0}){}$\1\5
${}\\{push}(\|i,\39\|j,\39\|k+\T{7}){}$;\C{ move counterclockwise }\2\6
${}\\{kk}\K(\|k+\T{1})\AND\T{\^7};{}$\6
\&{if} ${}((\|b[\|i][\|j]\AND(\T{1}\LL\\{kk}))\E\T{0}){}$\1\5
${}\\{push}(\|i,\39\|j,\39\\{kk}){}$;\C{ move clockwise }\2\6
\X12:Check neighbors in three-region tiles\X;\par
\U9.\fi

\M{11}A tile that contains two nonintersecting segments consists of a
middle region and two others. The middle region needs to be identified
so that we can ``jump'' from one of its edges to the other.

Three-region tiles are characterized by having \PB{\\{codetable}[\|a[\|i][%
\|j]]}
between \PB{\T{\~100}} and \PB{\T{\~164}}, inclusive. When that happens, we can
pack the
necessary logic into a magic four-byte table, which contains
the four endpoints $\{i,j,i',j'\}$ in the correct clockwise
order for processing.

\Y\B\4\X11:Declare the magic tables\X${}\E{}$\6
\&{char} \\{magic}[\T{56}]${}\K\{\3{-1}\T{1},\39\T{3},\39\T{5},\39\T{7},\3{-1}%
\39\T{7},\39\T{1},\39\T{3},\39\T{5},\3{-1}\39\T{1},\39\T{3},\39\T{5},\39\T{0},%
\3{-1}\39\T{7},\39\T{2},\39\T{3},\39\T{5},\3{-1}\39\T{1},\39\T{4},\39\T{5},\39%
\T{7},\3{-1}\39\T{7},\39\T{1},\39\T{3},\39\T{6},\3{-1}\39\T{0},\39\T{3},\39%
\T{5},\39\T{7},\3{-1}\39\T{7},\39\T{1},\39\T{2},\39\T{5},\3{-1}\39\T{1},\39%
\T{3},\39\T{4},\39\T{7},\3{-1}\39\T{6},\39\T{1},\39\T{3},\39\T{5},\3{-1}\39%
\T{1},\39\T{4},\39\T{5},\39\T{0},\3{-1}\39\T{7},\39\T{2},\39\T{3},\39\T{6},%
\3{-1}\39\T{0},\39\T{3},\39\T{4},\39\T{7},\3{-1}\39\T{6},\39\T{1},\39\T{2},\39%
\T{5}\}{}$;\par
\A22.
\U3.\fi

\M{12}\B\X12:Check neighbors in three-region tiles\X${}\E{}$\6
$\\{kk}\K\\{codetable}[\|a[\|i][\|j]]-\T{\~100};{}$\6
\&{if} ${}(\\{kk}<\T{\~70}\W\\{kk}\G\T{0}){}$\5
${}\{{}$\1\6
\&{if} ${}(\|k\E\\{magic}[\\{kk}+\T{1}]){}$\1\5
${}\\{push}(\|i,\39\|j,\39\\{magic}[\\{kk}]+\T{7});{}$\2\6
\&{if} ${}(\|k\E((\\{magic}[\\{kk}]+\T{7})\AND\T{\^7})){}$\1\5
${}\\{push}(\|i,\39\|j,\39\\{magic}[\\{kk}+\T{1}]);{}$\2\6
\&{if} ${}(\|k\E\\{magic}[\\{kk}+\T{3}]){}$\1\5
${}\\{push}(\|i,\39\|j,\39\\{magic}[\\{kk}+\T{2}]+\T{7});{}$\2\6
\&{if} ${}(\|k\E((\\{magic}[\\{kk}+\T{2}]+\T{7})\AND\T{\^7})){}$\1\5
${}\\{push}(\|i,\39\|j,\39\\{magic}[\\{kk}+\T{3}]);{}$\2\6
\4${}\}{}$\2\par
\U10.\fi

\M{13}And at last, when every subregion has been colored in both of the
bicolorings, we come to the {\it denouement\/}:
Typesetting can proceed.

Suppose \PB{\\{codetable}[\|a[\|i][\|j]]} is \PB{\|k}. Then the tile in
row~$i$, column~$j$
is typeset in font {\mc CELTICA} or {\mc CELTICB},
depending on whether
its subregion 0 is white or black, respectively; that is, depending on whether
\PB{\\{bw}[\|i][\|j][\T{0}]} is 0 or~1, respectively.

A~blank tile, when \PB{$\|k\K\T{\~40}$}, is typeset only if it's inside.
(Character \PB{\T{\~40}} is `$\vcenter{\hbox{\celtica\char'040}}$'.)

A two-region tile is typeset from $k$ if both regions are inside,
from $k+1$ if one region is outside. (For example, character
\PB{\T{\~44}} is `$\vcenter{\hbox{\celtica\char'044}}$'; character
\PB{\T{\~45}} is `$\vcenter{\hbox{\celtica\char'045}}$' in {\mc CELTICA} and
`$\vcenter{\hbox{\celticb\char'045}}$' in {\mc CELTICB}.)

A three-region or four-region tile is typeset from $k$, $k+1$, $k+2$,
or $k+3$, depending on the inside/outside configurations; details
are worked out below.

\Y\B\4\X13:Produce the output\X${}\E{}$\6
\X14:Publish the preamble\X;\6
\&{for} ${}(\|i\K\T{0};{}$ ${}\|i<\|m;{}$ ${}\|i\PP){}$\1\5
\X16:Publish row \PB{\|i}\X;\2\6
\X15:Publish the postamble\X;\par
\U3.\fi

\M{14}\B\X14:Publish the preamble\X${}\E{}$\6
\\{printf}(\.{"\%\%\ begin\ output\ of\ }\)\.{CELTIC-PATHS\\n"});\6
\\{printf}(\.{"\\\\font\\\\celtica=cel}\)\.{tica13\ \\\\font\\\\celti}\)%
\.{cb=celticb13\\n\\n"});\6
\\{printf}(\.{"\\\\begingroup\\\\celti}\)\.{ca\\n"});\6
\\{printf}(\.{"\\\\catcode`\\\\|=\\\\act}\)\.{ive\ \\\\def|\#1|\{\\\\hski}\)%
\.{p\#1em\}\\n"});\6
\\{printf}(\.{"\\\\catcode`\\\\-=\\\\act}\)\.{ive\ \\\\def-\#1\#2\#3\{\\\\c}\)%
\.{eltica\\\\char'\#1\#2\#3\}}\)\.{\\n"});\6
\\{printf}(\.{"\\\\catcode`\\\\+=\\\\act}\)\.{ive\ \\\\def+\#1\#2\#3\{\\\\c}\)%
\.{elticb\\\\char'\#1\#2\#3\}}\)\.{\\n"});\6
\\{printf}(\.{"\\\\offinterlineskip\\}\)\.{\\baselineskip=1em\\n"});\6
\\{printf}(\.{"\\\\let\\\\par=\\\\cr\ \\\\o}\)\.{beylines\ \\\\halign\{\#\\}\)%
\.{\\hfil\\n"});\par
\U13.\fi

\M{15}\B\X15:Publish the postamble\X${}\E{}$\6
\\{printf}(\.{"\}\\\\endgroup\\n"});\par
\U13.\fi

\M{16}\B\X16:Publish row \PB{\|i}\X${}\E{}$\6
${}\{{}$\1\6
${}\|s\K\T{0}{}$;\C{ \PB{\|s} holds the number of accumulated blanks }\6
\&{for} ${}(\|j\K\T{0};{}$ ${}\|j<\|n;{}$ ${}\|j\PP){}$\1\5
\X17:Typeset tile $(i,j)$\X;\2\6
\\{printf}(\.{"\\n"});\6
\4${}\}{}$\2\par
\U13.\fi

\M{17}\B\X17:Typeset tile $(i,j)$\X${}\E{}$\6
${}\{{}$\1\6
${}\\{kk}\K\\{codetable}[\|a[\|i][\|j]];{}$\6
${}\\{ii}\K\\{bw}[\|i][\|j][\T{0}];{}$\6
\&{if} ${}(\\{kk}\G\T{\~100}){}$\1\5
\X21:Handle a tile with three or four regions\X\2\6
\&{else} \&{if} ${}(\\{kk}\E\T{\~40}){}$\1\5
\X18:Handle a blank tile\X\2\6
\&{else} \&{if} ${}(\\{kk}\E\T{0}){}$\1\5
\X20:Handle tile \.{1537}\X\2\6
\&{else}\1\5
\X19:Handle a two-region tile\X;\2\6
\&{if} (\|s)\5
${}\{{}$\1\6
${}\\{printf}(\.{"|\%d|"},\39\|s);{}$\6
${}\|s\K\T{0};{}$\6
\4${}\}{}$\2\6
${}\\{printf}(\.{"\%c\%03o"},\39\\{ii}\?\.{'+'}:\.{'-'},\39\\{kk});{}$\6
\4${}\}{}$\2\par
\U16.\fi

\M{18}\B\X18:Handle a blank tile\X${}\E{}$\6
${}\{{}$\1\6
\&{if} (\\{inout}[\|i][\|j][\T{0}])\5
${}\{{}$\C{ normal case, blank and outside }\1\6
${}\|s\PP{}$;\5
\&{continue};\6
\4${}\}{}$\2\6
\4${}\}{}$\2\par
\U17.\fi

\M{19}\B\X19:Handle a two-region tile\X${}\E{}$\6
${}\{{}$\1\6
\&{for} ${}(\|k\K\T{0};{}$ ${}\|k<\T{8};{}$ ${}\|k\PP){}$\1\6
\&{if} (\\{inout}[\|i][\|j][\|k])\5
${}\{{}$\1\6
${}\\{kk}\PP{}$;\5
\&{break};\6
\4${}\}{}$\2\2\6
\4${}\}{}$\2\par
\U17.\fi

\M{20}The fonts treat \.{1537} as a special case in which all sixteen
combinations of black/white backgrounds are permissible.
Only four of them can actually occur in the output of this program,
because adjacent regions cannot both be `outside'.

\Y\B\4\X20:Handle tile \.{1537}\X${}\E{}$\6
${}\{{}$\1\6
${}\\{kk}\K(\\{inout}[\|i][\|j][\T{1}]\?\T{1}:\T{0})+(\\{inout}[\|i][\|j][%
\T{3}]\?\T{8}:\T{0})+(\\{inout}[\|i][\|j][\T{5}]\?\T{4}:\T{0})+(\\{inout}[\|i][%
\|j][\T{7}]\?\T{2}:\T{0});{}$\6
\4${}\}{}$\2\par
\U17.\fi

\M{21}In the most complex case, we walk clockwise around the edge of the tile
and
note the pattern of four inside/outside regions that we see. Four patterns
are possible (either 0000, 1000, 0010, or 1010 in {\mc CELTICA}, and
either 0000, 0001, 0100, or 0101 in {\mc CELTICB}); they cause us to
add 0, 1, 2, or 3, respectively to the character code \PB{\\{kk}}.

\Y\B\4\X21:Handle a tile with three or four regions\X${}\E{}$\6
${}\{{}$\1\6
${}\|t\K(\\{inout}[\|i][\|j][\T{0}]\?\T{8}:\T{0});{}$\6
\&{for} ${}(\|k\K\T{1},\39\\{jj}\K\T{4};{}$ \\{jj}; ${}\|k\PP){}$\1\6
\&{if} ${}(\|b[\|i][\|j]\AND(\T{1}\LL\|k)){}$\5
${}\{{}$\1\6
${}\|t\MRL{+{\K}}(\\{inout}[\|i][\|j][\|k]\?\\{jj}:\T{0});{}$\6
${}\\{jj}\MRL{{\GG}{\K}}\T{1};{}$\6
\4${}\}{}$\2\2\6
${}\\{kk}\MRL{+{\K}}\\{offset}[\|t];{}$\6
\4${}\}{}$\2\par
\U17.\fi

\M{22}\B\X11:Declare the magic tables\X${}\mathrel+\E{}$\6
\&{char} \\{offset}[\T{16}]${}\K\{\T{0},\39\T{1},\39\T{2},\39\T{0},\39\T{2},\39%
\T{3},\39\T{0},\39\T{0},\39\T{1},\39\T{0},\39\T{3},\39\T{0},\39\T{0},\39\T{0},%
\39\T{0},\39\T{0}\}{}$;\par
\fi

\N{1}{23}Index.
\fi

\inx
\fin
\con
