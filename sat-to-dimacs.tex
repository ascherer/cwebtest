\input cwebmac
\hypertextrue\srcloctrue

\N[1 sat-to-dimacs.w]{1}{1}Intro. This is a filter that inputs the format used
by {\mc SAT0},
{\mc SAT1}, etc., and outputs the well-known DIMACS format for
satisfiability problems.

DIMACS format begins with zero or more optional comment lines, indicated by
their first character `\.c'. The next line should say `\.p \.{cnf} $n$ $m$',
where $n$ is the number of variables and $m$ is the number of clauses.
Then comes a string of $m$ ``clauses,'' which are sequences of
nonzero integers of absolute value $\le n$, followed by zero.
A literal for the $k$th variable is represented by $k$; its complement
is represented by $-k$.

SAT format is more flexible, more symbolic, and more complicated; it is
explained in the programs cited above. I hacked this program from {\mc SAT3}.

\Y\B\8\#\&{include} \.{<stdio.h>}\6
\8\#\&{include} \.{<stdlib.h>}\6
\8\#\&{include} \.{<string.h>}\6
\8\#\&{include} \.{"gb\_flip.h"}\6
\8\#\&{include} \.{<time.h>}\6
\&{time\_t} \\{myclock};\6
\&{typedef} \&{unsigned} \&{int} \&{uint};\C{ a convenient abbreviation }\6
\&{typedef} \&{unsigned} \&{long} \&{long} \&{ullng};\C{ ditto }\7
\X4:Type definitions\X;\6
\X2:Global variables\X;\7
\\{main}(\&{int} \\{argc}${},\39{}$\&{char} ${}{*}\\{argv}[\,]){}$\1\1\2\2\6
${}\{{}$\1\6
\&{register} \&{uint} \|c${},{}$ \|h${},{}$ \|i${},{}$ \|j${},{}$ \|k${},{}$ %
\|l${},{}$ \|p${},{}$ \|q${},{}$ \|r${},{}$ \\{level}${},{}$ \\{kk}${},{}$ %
\\{pp}${},{}$ \\{qq}${},{}$ \\{ll};\7
\X3:Process the command line\X;\6
\X7:Initialize everything\X;\6
\X8:Input the clauses\X;\6
\&{if} (\\{verbose})\1\5
\X20:Report the successful completion of the input phase\X;\2\6
${}\\{myclock}\K\\{time}(\T{0});{}$\6
${}\\{printf}(\.{"c\ file\ created\ by\ S}\)\.{AT-TO-DIMACS\ \%s"},\39%
\\{ctime}({\AND}\\{myclock}));{}$\6
\X21:Output the clauses\X;\6
\4${}\}{}$\2\par
\fi

\M[39 sat-to-dimacs.w]{2}\B\X2:Global variables\X${}\E{}$\6
\&{int} \\{random\_seed}${}\K\T{0}{}$;\C{ seed for the random words of \PB{%
\\{gb\_rand}} }\6
\&{int} \\{verbose}${}\K\T{1}{}$;\C{ level of verbosity }\6
\&{int} \\{hbits}${}\K\T{8}{}$;\C{ logarithm of the number of the hash lists }\6
\&{int} \\{buf\_size}${}\K\T{1024}{}$;\C{ must exceed the length of the longest
input line }\par
\A6.
\U1.\fi

\M[45 sat-to-dimacs.w]{3}On the command line one can say
\smallskip
\item{$\bullet$}
`\.v$\langle\,$integer$\,\rangle$' to enable various levels of verbose
output on \PB{\\{stderr}};
\item{$\bullet$}
`\.h$\langle\,$positive integer$\,\rangle$' to adjust the hash table size;
\item{$\bullet$}
`\.b$\langle\,$positive integer$\,\rangle$' to adjust the size of the input
buffer; and/or
\item{$\bullet$}
`\.s$\langle\,$integer$\,\rangle$' to define the seed for any random numbers
that are used.

\Y\B\4\X3:Process the command line\X${}\E{}$\6
\&{for} ${}(\|j\K\\{argc}-\T{1},\39\|k\K\T{0};{}$ \|j; ${}\|j\MM){}$\1\6
\&{switch} (\\{argv}[\|j][\T{0}])\5
${}\{{}$\1\6
\4\&{case} \.{'v'}:\5
${}\|k\MRL{{\OR}{\K}}(\\{sscanf}(\\{argv}[\|j]+\T{1},\39\.{"\%d"},\39{\AND}%
\\{verbose})-\T{1}){}$;\5
\&{break};\6
\4\&{case} \.{'h'}:\5
${}\|k\MRL{{\OR}{\K}}(\\{sscanf}(\\{argv}[\|j]+\T{1},\39\.{"\%d"},\39{\AND}%
\\{hbits})-\T{1}){}$;\5
\&{break};\6
\4\&{case} \.{'b'}:\5
${}\|k\MRL{{\OR}{\K}}(\\{sscanf}(\\{argv}[\|j]+\T{1},\39\.{"\%d"},\39{\AND}%
\\{buf\_size})-\T{1}){}$;\5
\&{break};\6
\4\&{case} \.{'s'}:\5
${}\|k\MRL{{\OR}{\K}}(\\{sscanf}(\\{argv}[\|j]+\T{1},\39\.{"\%d"},\39{\AND}%
\\{random\_seed})-\T{1}){}$;\5
\&{break};\6
\4\&{default}:\5
${}\|k\K\T{1}{}$;\C{ unrecognized command-line option }\6
\4${}\}{}$\2\2\6
\&{if} ${}(\|k\V\\{hbits}<\T{0}\V\\{hbits}>\T{30}\V\\{buf\_size}\Z\T{0}){}$\5
${}\{{}$\1\6
${}\\{fprintf}(\\{stderr},\39\.{"Usage:\ \%s\ [v<n>]\ [h}\)\.{<n>]\ [b<n>]\
[s<n>]\ <}\)\.{\ foo.dat\\n"},\39\\{argv}[\T{0}]);{}$\6
${}\\{exit}({-}\T{1});{}$\6
\4${}\}{}$\2\par
\U1.\fi

\N[72 sat-to-dimacs.w]{1}{4}The I/O wrapper. The following routines read the
input and absorb it into
temporary data areas from which all of the ``real'' data structures
can readily be initialized. My intent is to incorporate these routines in all
of the SAT-solvers in this series. Therefore I've tried to make the code
short and simple, yet versatile enough so that almost no restrictions are
placed on the sizes of problems that can be handled. These routines are
supposed to work properly unless there are more than
$2^{32}-1=4$,294,967,295 occurrences of literals in clauses,
or more than $2^{31}-1=2$,147,483,647 variables or clauses.

In these temporary tables, each variable is represented by four things:
its unique name; its serial number; the clause number (if any) in which it has
most recently appeared; and a pointer to the previous variable (if any)
with the same hash address. Several variables at a time
are represented sequentially in small chunks of memory called ``vchunks,''
which are allocated as needed (and freed later).

\Y\B\4\D$\\{vars\_per\_vchunk}$ \5
\T{341}\C{ preferably $(2^k-1)/3$ for some $k$ }\par
\Y\B\4\X4:Type definitions\X${}\E{}$\6
\&{typedef} \&{union} ${}\{{}$\1\6
\&{char} \\{ch8}[\T{8}];\6
\&{uint} \\{u2}[\T{2}];\6
\&{long} \&{long} \\{lng};\2\6
${}\}{}$ \&{octa};\6
\&{typedef} \&{struct} \&{tmp\_var\_struct} ${}\{{}$\1\6
\&{octa} \\{name};\C{ the name (one to eight ASCII characters) }\6
\&{uint} \\{serial};\C{ 0 for the first variable, 1 for the second, etc. }\6
\&{int} \\{stamp};\C{ \PB{\|m} if positively in clause \PB{\|m}; \PB{${-}\|m$}
if negatively there }\6
\&{struct} \&{tmp\_var\_struct} ${}{*}\\{next}{}$;\C{ pointer for hash list }\2%
\6
${}\}{}$ \&{tmp\_var};\7
\&{typedef} \&{struct} \&{vchunk\_struct} ${}\{{}$\1\6
\&{struct} \&{vchunk\_struct} ${}{*}\\{prev}{}$;\C{ previous chunk allocated
(if any) }\6
\&{tmp\_var} \\{var}[\\{vars\_per\_vchunk}];\2\6
${}\}{}$ \&{vchunk};\par
\A5.
\U1.\fi

\M[109 sat-to-dimacs.w]{5}Each clause in the temporary tables is represented by
a sequence of
one or more pointers to the \PB{\&{tmp\_var}} nodes of the literals involved.
A negated literal is indicated by adding~1 to such a pointer.
The first literal of a clause is indicated by adding~2.
Several of these pointers are represented sequentially in chunks
of memory, which are allocated as needed and freed later.

\Y\B\4\D$\\{cells\_per\_chunk}$ \5
\T{511}\C{ preferably $2^k-1$ for some $k$ }\par
\Y\B\4\X4:Type definitions\X${}\mathrel+\E{}$\6
\&{typedef} \&{struct} \&{chunk\_struct} ${}\{{}$\1\6
\&{struct} \&{chunk\_struct} ${}{*}\\{prev}{}$;\C{ previous chunk allocated (if
any) }\6
\&{tmp\_var} ${}{*}\\{cell}[\\{cells\_per\_chunk}];{}$\2\6
${}\}{}$ \&{chunk};\par
\fi

\M[124 sat-to-dimacs.w]{6}\B\X2:Global variables\X${}\mathrel+\E{}$\6
\&{char} ${}{*}\\{buf}{}$;\C{ buffer for reading the lines (clauses) of \PB{%
\\{stdin}} }\6
\&{tmp\_var} ${}{*}{*}\\{hash}{}$;\C{ heads of the hash lists }\6
\&{uint} \\{hash\_bits}[\T{93}][\T{8}];\C{ random bits for universal hash
function }\6
\&{vchunk} ${}{*}\\{cur\_vchunk}{}$;\C{ the vchunk currently being filled }\6
\&{tmp\_var} ${}{*}\\{cur\_tmp\_var}{}$;\C{ current place to create new \PB{%
\&{tmp\_var}} entries }\6
\&{tmp\_var} ${}{*}\\{bad\_tmp\_var}{}$;\C{ the \PB{\\{cur\_tmp\_var}} when we
need a new \PB{\&{vchunk}} }\6
\&{chunk} ${}{*}\\{cur\_chunk}{}$;\C{ the chunk currently being filled }\6
\&{tmp\_var} ${}{*}{*}\\{cur\_cell}{}$;\C{ current place to create new elements
of a clause }\6
\&{tmp\_var} ${}{*}{*}\\{bad\_cell}{}$;\C{ the \PB{\\{cur\_cell}} when we need
a new \PB{\&{chunk}} }\6
\&{ullng} \\{vars};\C{ how many distinct variables have we seen? }\6
\&{ullng} \\{clauses};\C{ how many clauses have we seen? }\6
\&{ullng} \\{nullclauses};\C{ how many of them were null? }\6
\&{ullng} \\{cells};\C{ how many occurrences of literals in clauses? }\par
\fi

\M[139 sat-to-dimacs.w]{7}\B\X7:Initialize everything\X${}\E{}$\6
\\{gb\_init\_rand}(\\{random\_seed});\6
${}\\{buf}\K{}$(\&{char} ${}{*}){}$ \\{malloc}${}(\\{buf\_size}*\&{sizeof}(%
\&{char}));{}$\6
\&{if} ${}(\R\\{buf}){}$\5
${}\{{}$\1\6
${}\\{fprintf}(\\{stderr},\39\.{"Couldn't\ allocate\ t}\)\.{he\ input\ buffer\
(buf}\)\.{\_size=\%d)!\\n"},\39\\{buf\_size});{}$\6
${}\\{exit}({-}\T{2});{}$\6
\4${}\}{}$\2\6
${}\\{hash}\K{}$(\&{tmp\_var} ${}{*}{*}){}$ \\{malloc}${}(\&{sizeof}(\&{tmp%
\_var})\LL\\{hbits});{}$\6
\&{if} ${}(\R\\{hash}){}$\5
${}\{{}$\1\6
${}\\{fprintf}(\\{stderr},\39\.{"Couldn't\ allocate\ \%}\)\.{d\ hash\ list\
heads\ (h}\)\.{bits=\%d)!\\n"},\39\T{1}\LL\\{hbits},\39\\{hbits});{}$\6
${}\\{exit}({-}\T{3});{}$\6
\4${}\}{}$\2\6
\&{for} ${}(\|h\K\T{0};{}$ ${}\|h<\T{1}\LL\\{hbits};{}$ ${}\|h\PP){}$\1\5
${}\\{hash}[\|h]\K\NULL{}$;\2\par
\A13.
\U1.\fi

\M[155 sat-to-dimacs.w]{8}The hash address of each variable name has $h$ bits,
where $h$ is the
value of the adjustable parameter \PB{\\{hbits}}.
Thus the average number of variables per hash list is $n/2^h$ when there
are $n$ different variables. A warning is printed if this average number
exceeds 10. (For example, if $h$ has its default value, 8, the program will
suggest that you might want to increase $h$ if your input has 2560
different variables or more.)

All the hashing takes place at the very beginning,
and the hash tables are actually recycled before any SAT-solving takes place;
therefore the setting of this parameter is by no means crucial. But I didn't
want to bother with fancy coding that would determine $h$ automatically.

\Y\B\4\X8:Input the clauses\X${}\E{}$\6
\&{while} (\T{1})\5
${}\{{}$\1\6
\&{if} ${}(\R\\{fgets}(\\{buf},\39\\{buf\_size},\39\\{stdin})){}$\1\5
\&{break};\2\6
${}\\{clauses}\PP;{}$\6
\&{if} ${}(\\{buf}[\\{strlen}(\\{buf})-\T{1}]\I\.{'\\n'}){}$\5
${}\{{}$\1\6
${}\\{fprintf}(\\{stderr},\39\.{"The\ clause\ on\ line\ }\)\.{\%lld\ (%
\%.20s...)\ is\ t}\)\.{oo\ long\ for\ me;\\n"},\39\\{clauses},\39\\{buf});{}$\6
${}\\{fprintf}(\\{stderr},\39\.{"\ my\ buf\_size\ is\ onl}\)\.{y\ \%d!\\n"},\39%
\\{buf\_size});{}$\6
${}\\{fprintf}(\\{stderr},\39\.{"Please\ use\ the\ comm}\)\.{and-line\ option\
b<ne}\)\.{wsize>.\\n"});{}$\6
${}\\{exit}({-}\T{4});{}$\6
\4${}\}{}$\2\6
\X9:Input the clause in \PB{\\{buf}}\X;\6
\4${}\}{}$\2\6
\&{if} ${}((\\{vars}\GG\\{hbits})\G\T{10}){}$\5
${}\{{}$\1\6
${}\\{fprintf}(\\{stderr},\39\.{"There\ are\ \%lld\ vari}\)\.{ables\ but\ only\
\%d\ ha}\)\.{sh\ tables;\\n"},\39\\{vars},\39\T{1}\LL\\{hbits});{}$\6
\&{while} ${}((\\{vars}\GG\\{hbits})\G\T{10}){}$\1\5
${}\\{hbits}\PP;{}$\2\6
${}\\{fprintf}(\\{stderr},\39\.{"\ maybe\ you\ should\ u}\)\.{se\ command-line\
opti}\)\.{on\ h\%d?\\n"},\39\\{hbits});{}$\6
\4${}\}{}$\2\6
${}\\{clauses}\MRL{-{\K}}\\{nullclauses};{}$\6
\&{if} ${}(\\{clauses}\E\T{0}){}$\5
${}\{{}$\1\6
${}\\{fprintf}(\\{stderr},\39\.{"No\ clauses\ were\ inp}\)\.{ut!\\n"});{}$\6
${}\\{exit}({-}\T{77});{}$\6
\4${}\}{}$\2\6
\&{if} ${}(\\{vars}\G\T{\^80000000}){}$\5
${}\{{}$\1\6
${}\\{fprintf}(\\{stderr},\39\.{"Whoa,\ the\ input\ had}\)\.{\ \%llu\
variables!\\n"},\39\\{cells});{}$\6
${}\\{exit}({-}\T{664});{}$\6
\4${}\}{}$\2\6
\&{if} ${}(\\{clauses}\G\T{\^80000000}){}$\5
${}\{{}$\1\6
${}\\{fprintf}(\\{stderr},\39\.{"Whoa,\ the\ input\ had}\)\.{\ \%llu\ clauses!%
\\n"},\39\\{cells});{}$\6
${}\\{exit}({-}\T{665});{}$\6
\4${}\}{}$\2\6
\&{if} ${}(\\{cells}\G\T{\^100000000}){}$\5
${}\{{}$\1\6
${}\\{fprintf}(\\{stderr},\39\.{"Whoa,\ the\ input\ had}\)\.{\ \%llu\
occurrences\ of}\)\.{\ literals!\\n"},\39\\{cells});{}$\6
${}\\{exit}({-}\T{666});{}$\6
\4${}\}{}$\2\par
\U1.\fi

\M[205 sat-to-dimacs.w]{9}\B\X9:Input the clause in \PB{\\{buf}}\X${}\E{}$\6
\&{for} ${}(\|j\K\|k\K\T{0};{}$  ; \,)\5
${}\{{}$\1\6
\&{while} ${}(\\{buf}[\|j]\E\.{'\ '}){}$\1\5
${}\|j\PP{}$;\C{ scan to nonblank }\2\6
\&{if} ${}(\\{buf}[\|j]\E\.{'\\n'}){}$\1\5
\&{break};\2\6
\&{if} ${}(\\{buf}[\|j]<\.{'\ '}\V\\{buf}[\|j]>\.{'\~'}){}$\5
${}\{{}$\1\6
${}\\{fprintf}(\\{stderr},\39\.{"Illegal\ character\ (}\)\.{code\ \#\%x)\ in\
the\ cla}\)\.{use\ on\ line\ \%lld!\\n"},\39\\{buf}[\|j],\39\\{clauses});{}$\6
${}\\{exit}({-}\T{5});{}$\6
\4${}\}{}$\2\6
\&{if} ${}(\\{buf}[\|j]\E\.{'\~'}){}$\1\5
${}\|i\K\T{1},\39\|j\PP;{}$\2\6
\&{else}\1\5
${}\|i\K\T{0};{}$\2\6
\X10:Scan and record a variable; negate it if \PB{$\|i\E\T{1}$}\X;\6
\4${}\}{}$\2\6
\&{if} ${}(\|k\E\T{0}){}$\5
${}\{{}$\1\6
${}\\{fprintf}(\\{stderr},\39\.{"(Empty\ line\ \%lld\ is}\)\.{\ being\ ignored)%
\\n"},\39\\{clauses});{}$\6
${}\\{nullclauses}\PP{}$;\C{ strictly speaking it would be unsatisfiable }\6
\4${}\}{}$\2\6
\&{goto} \\{clause\_done};\6
\4\\{empty\_clause}:\5
\X17:Remove all variables of the current clause\X;\6
\4\\{clause\_done}:\5
${}\\{cells}\MRL{+{\K}}\|k{}$;\par
\U8.\fi

\M[226 sat-to-dimacs.w]{10}We need a hack to insert the bit codes 1 and/or 2
into a pointer value.

\Y\B\4\D$\\{hack\_in}(\|q,\|t)$ \5
(\&{tmp\_var} ${}{*})(\|t\OR{}$(\&{ullng}) \|q)\par
\Y\B\4\X10:Scan and record a variable; negate it if \PB{$\|i\E\T{1}$}\X${}\E{}$%
\6
${}\{{}$\1\6
\&{register} \&{tmp\_var} ${}{*}\|p;{}$\7
\&{if} ${}(\\{cur\_tmp\_var}\E\\{bad\_tmp\_var}){}$\1\5
\X11:Install a new \PB{\&{vchunk}}\X;\2\6
\X14:Put the variable name beginning at \PB{\\{buf}[\|j]} in \PB{$\\{cur\_tmp%
\_var}\MG\\{name}$} and compute its hash code \PB{\|h}\X;\6
\X15:Find \PB{$\\{cur\_tmp\_var}\MG\\{name}$} in the hash table at \PB{\|p}\X;\6
\&{if} ${}(\|p\MG\\{stamp}\E\\{clauses}\V\|p\MG\\{stamp}\E{-}\\{clauses}){}$\1\5
\X16:Handle a duplicate literal\X\2\6
\&{else}\5
${}\{{}$\1\6
${}\|p\MG\\{stamp}\K(\|i\?{-}\\{clauses}:\\{clauses});{}$\6
\&{if} ${}(\\{cur\_cell}\E\\{bad\_cell}){}$\1\5
\X12:Install a new \PB{\&{chunk}}\X;\2\6
${}{*}\\{cur\_cell}\K\|p;{}$\6
\&{if} ${}(\|i\E\T{1}){}$\1\5
${}{*}\\{cur\_cell}\K\\{hack\_in}({*}\\{cur\_cell},\39\T{1});{}$\2\6
\&{if} ${}(\|k\E\T{0}){}$\1\5
${}{*}\\{cur\_cell}\K\\{hack\_in}({*}\\{cur\_cell},\39\T{2});{}$\2\6
${}\\{cur\_cell}\PP,\39\|k\PP;{}$\6
\4${}\}{}$\2\6
\4${}\}{}$\2\par
\U9.\fi

\M[248 sat-to-dimacs.w]{11}\B\X11:Install a new \PB{\&{vchunk}}\X${}\E{}$\6
${}\{{}$\1\6
\&{register} \&{vchunk} ${}{*}\\{new\_vchunk};{}$\7
${}\\{new\_vchunk}\K{}$(\&{vchunk} ${}{*}){}$ \\{malloc}(\&{sizeof}(%
\&{vchunk}));\6
\&{if} ${}(\R\\{new\_vchunk}){}$\5
${}\{{}$\1\6
${}\\{fprintf}(\\{stderr},\39\.{"Can't\ allocate\ a\ ne}\)\.{w\ vchunk!%
\\n"});{}$\6
${}\\{exit}({-}\T{6});{}$\6
\4${}\}{}$\2\6
${}\\{new\_vchunk}\MG\\{prev}\K\\{cur\_vchunk},\39\\{cur\_vchunk}\K\\{new%
\_vchunk};{}$\6
${}\\{cur\_tmp\_var}\K{\AND}\\{new\_vchunk}\MG\\{var}[\T{0}];{}$\6
${}\\{bad\_tmp\_var}\K{\AND}\\{new\_vchunk}\MG\\{var}[\\{vars\_per%
\_vchunk}];{}$\6
\4${}\}{}$\2\par
\U10.\fi

\M[261 sat-to-dimacs.w]{12}\B\X12:Install a new \PB{\&{chunk}}\X${}\E{}$\6
${}\{{}$\1\6
\&{register} \&{chunk} ${}{*}\\{new\_chunk};{}$\7
${}\\{new\_chunk}\K{}$(\&{chunk} ${}{*}){}$ \\{malloc}(\&{sizeof}(\&{chunk}));\6
\&{if} ${}(\R\\{new\_chunk}){}$\5
${}\{{}$\1\6
${}\\{fprintf}(\\{stderr},\39\.{"Can't\ allocate\ a\ ne}\)\.{w\ chunk!%
\\n"});{}$\6
${}\\{exit}({-}\T{7});{}$\6
\4${}\}{}$\2\6
${}\\{new\_chunk}\MG\\{prev}\K\\{cur\_chunk},\39\\{cur\_chunk}\K\\{new%
\_chunk};{}$\6
${}\\{cur\_cell}\K{\AND}\\{new\_chunk}\MG\\{cell}[\T{0}];{}$\6
${}\\{bad\_cell}\K{\AND}\\{new\_chunk}\MG\\{cell}[\\{cells\_per\_chunk}];{}$\6
\4${}\}{}$\2\par
\U10.\fi

\M[274 sat-to-dimacs.w]{13}The hash code is computed via ``universal hashing,''
using the following
precomputed tables of random bits.

\Y\B\4\X7:Initialize everything\X${}\mathrel+\E{}$\6
\&{for} ${}(\|j\K\T{92};{}$ \|j; ${}\|j\MM){}$\1\6
\&{for} ${}(\|k\K\T{0};{}$ ${}\|k<\T{8};{}$ ${}\|k\PP){}$\1\5
${}\\{hash\_bits}[\|j][\|k]\K\\{gb\_next\_rand}(\,){}$;\2\2\par
\fi

\M[281 sat-to-dimacs.w]{14}\B\X14:Put the variable name beginning at \PB{%
\\{buf}[\|j]} in \PB{$\\{cur\_tmp\_var}\MG\\{name}$} and compute its hash code %
\PB{\|h}\X${}\E{}$\6
$\\{cur\_tmp\_var}\MG\\{name}.\\{lng}\K\T{0};{}$\6
\&{for} ${}(\|h\K\|l\K\T{0};{}$ ${}\\{buf}[\|j+\|l]>\.{'\ '}\W\\{buf}[\|j+\|l]%
\Z\.{'\~'};{}$ ${}\|l\PP){}$\5
${}\{{}$\1\6
\&{if} ${}(\|l>\T{7}){}$\5
${}\{{}$\1\6
${}\\{fprintf}(\\{stderr},\39\.{"Variable\ name\ \%.9s.}\)\.{..\ in\ the\
clause\ on\ }\)\.{line\ \%lld\ is\ too\ lon}\)\.{g!\\n"},\39\\{buf}+\|j,\39%
\\{clauses});{}$\6
${}\\{exit}({-}\T{8});{}$\6
\4${}\}{}$\2\6
${}\|h\MRL{{\XOR}{\K}}\\{hash\_bits}[\\{buf}[\|j+\|l]-\.{'!'}][\|l];{}$\6
${}\\{cur\_tmp\_var}\MG\\{name}.\\{ch8}[\|l]\K\\{buf}[\|j+\|l];{}$\6
\4${}\}{}$\2\6
\&{if} ${}(\|l\E\T{0}){}$\1\5
\&{goto} \\{empty\_clause};\C{ `\.\~' by itself is like `true' }\2\6
${}\|j\MRL{+{\K}}\|l;{}$\6
${}\|h\MRL{\AND{\K}}(\T{1}\LL\\{hbits})-\T{1}{}$;\par
\U10.\fi

\M[297 sat-to-dimacs.w]{15}\B\X15:Find \PB{$\\{cur\_tmp\_var}\MG\\{name}$} in
the hash table at \PB{\|p}\X${}\E{}$\6
\&{for} ${}(\|p\K\\{hash}[\|h];{}$ \|p; ${}\|p\K\|p\MG\\{next}){}$\1\6
\&{if} ${}(\|p\MG\\{name}.\\{lng}\E\\{cur\_tmp\_var}\MG\\{name}.\\{lng}){}$\1\5
\&{break};\2\2\6
\&{if} ${}(\R\|p){}$\5
${}\{{}$\C{ new variable found }\1\6
${}\|p\K\\{cur\_tmp\_var}\PP;{}$\6
${}\|p\MG\\{next}\K\\{hash}[\|h],\39\\{hash}[\|h]\K\|p;{}$\6
${}\|p\MG\\{serial}\K\\{vars}\PP;{}$\6
${}\|p\MG\\{stamp}\K\T{0};{}$\6
\4${}\}{}$\2\par
\U10.\fi

\M[307 sat-to-dimacs.w]{16}The most interesting aspect of the input phase is
probably the ``unwinding''
that we might need to do when encountering a literal more than once
in the same clause.

\Y\B\4\X16:Handle a duplicate literal\X${}\E{}$\6
${}\{{}$\1\6
\&{if} ${}((\|p\MG\\{stamp}>\T{0})\E(\|i>\T{0})){}$\1\5
\&{goto} \\{empty\_clause};\2\6
\4${}\}{}$\2\par
\U10.\fi

\M[316 sat-to-dimacs.w]{17}An input line that begins with `\.{\~\ }' is
silently treated as a comment.
Otherwise redundant clauses are logged, in case they were unintentional.
(One can, however, intentionally
use redundant clauses to force the order of the variables.)

\Y\B\4\X17:Remove all variables of the current clause\X${}\E{}$\6
\&{while} (\|k)\5
${}\{{}$\1\6
\X18:Move \PB{\\{cur\_cell}} backward to the previous cell\X;\6
${}\|k\MM;{}$\6
\4${}\}{}$\2\6
\&{if} ${}((\\{buf}[\T{0}]\I\.{'\~'})\V(\\{buf}[\T{1}]\I\.{'\ '})){}$\1\5
${}\\{fprintf}(\\{stderr},\39\.{"(The\ clause\ on\ line}\)\.{\ \%lld\ is\
always\ sati}\)\.{sfied)\\n"},\39\\{clauses});{}$\2\6
\&{else} \&{if} ${}(\\{vars}\E\T{0}){}$\1\5
${}\\{printf}(\.{"c\ \%s"},\39\\{buf}+\T{2}){}$;\C{ retain opening comments }\2%
\6
${}\\{nullclauses}\PP{}$;\par
\U9.\fi

\M[331 sat-to-dimacs.w]{18}\B\X18:Move \PB{\\{cur\_cell}} backward to the
previous cell\X${}\E{}$\6
\&{if} ${}(\\{cur\_cell}>{\AND}\\{cur\_chunk}\MG\\{cell}[\T{0}]){}$\1\5
${}\\{cur\_cell}\MM;{}$\2\6
\&{else}\5
${}\{{}$\1\6
\&{register} \&{chunk} ${}{*}\\{old\_chunk}\K\\{cur\_chunk};{}$\7
${}\\{cur\_chunk}\K\\{old\_chunk}\MG\\{prev}{}$;\5
\\{free}(\\{old\_chunk});\6
${}\\{bad\_cell}\K{\AND}\\{cur\_chunk}\MG\\{cell}[\\{cells\_per\_chunk}];{}$\6
${}\\{cur\_cell}\K\\{bad\_cell}-\T{1};{}$\6
\4${}\}{}$\2\par
\Us17\ET24.\fi

\M[340 sat-to-dimacs.w]{19}Here I must omit `\PB{\\{free}(\\{old\_vchunk})}'
from the code that's usually
in this section, because the variable data will be used later.

\Y\B\4\X19:Move \PB{\\{cur\_tmp\_var}} backward to the previous temporary
variable\X${}\E{}$\6
\&{if} ${}(\\{cur\_tmp\_var}>{\AND}\\{cur\_vchunk}\MG\\{var}[\T{0}]){}$\1\5
${}\\{cur\_tmp\_var}\MM;{}$\2\6
\&{else}\5
${}\{{}$\1\6
\&{register} \&{vchunk} ${}{*}\\{old\_vchunk}\K\\{cur\_vchunk};{}$\7
${}\\{cur\_vchunk}\K\\{old\_vchunk}\MG\\{prev}{}$;\C{ and don't \PB{\\{free}(%
\\{old\_vchunk})} }\6
${}\\{bad\_tmp\_var}\K{\AND}\\{cur\_vchunk}\MG\\{var}[\\{vars\_per%
\_vchunk}];{}$\6
${}\\{cur\_tmp\_var}\K\\{bad\_tmp\_var}-\T{1};{}$\6
\4${}\}{}$\2\par
\U22.\fi

\M[352 sat-to-dimacs.w]{20}\B\X20:Report the successful completion of the input
phase\X${}\E{}$\6
$\\{fprintf}(\\{stderr},\39\.{"(\%lld\ variables,\ \%l}\)\.{ld\ clauses,\ \%llu%
\ lit}\)\.{erals\ successfully\ r}\)\.{ead)\\n"},\39\\{vars},\39\\{clauses},\39%
\\{cells}){}$;\par
\U1.\fi

\N[356 sat-to-dimacs.w]{1}{21}The output phase. I had to input everything first
because DIMACS format
specifies the number of variables and clauses right at the beginning.

\Y\B\4\X21:Output the clauses\X${}\E{}$\6
\X22:Show the variable names as comments\X;\6
${}\\{printf}(\.{"p\ cnf\ \%lld\ \%lld\\n"},\39\\{vars},\39\\{clauses});{}$\6
\X23:Translate all the temporary cells into the simple DIMACS form\X;\6
\X25:Check consistency\X;\par
\U1.\fi

\M[365 sat-to-dimacs.w]{22}This section is optional, but I'm including it today
while I remember
how to provide it.

\Y\B\4\X22:Show the variable names as comments\X${}\E{}$\6
\&{for} ${}(\|c\K\\{vars};{}$ \|c; ${}\|c\MM){}$\5
${}\{{}$\1\6
\X19:Move \PB{\\{cur\_tmp\_var}} backward to the previous temporary variable\X;%
\6
${}\\{printf}(\.{"c\ \%.8s\ ->\ \%d\\n"},\39\\{cur\_tmp\_var}\MG\\{name}.%
\\{ch8},\39\|c);{}$\6
\4${}\}{}$\2\par
\U21.\fi

\M[374 sat-to-dimacs.w]{23}\B\X23:Translate all the temporary cells into the
simple DIMACS form\X${}\E{}$\6
\&{for} ${}(\|c\K\\{clauses};{}$ \|c; ${}\|c\MM){}$\5
${}\{{}$\1\6
\X24:Translate the cells for the literals of clause \PB{\|c}\X;\6
\\{printf}(\.{"\ 0\\n"});\6
\4${}\}{}$\2\par
\U21.\fi

\M[380 sat-to-dimacs.w]{24}\B\D$\\{hack\_out}(\|q)$ \5
(((\&{ullng}) \|q)${}\AND\T{\^3}{}$)\par
\B\4\D$\\{hack\_clean}(\|q)$ \5
((\&{tmp\_var} ${}{*})({}$(\&{ullng}) \|q${}\AND{-}\T{4}){}$)\par
\Y\B\4\X24:Translate the cells for the literals of clause \PB{\|c}\X${}\E{}$\6
\&{for} ${}(\|i\K\T{0};{}$ ${}\|i<\T{2};{}$ ${}\|j\PP){}$\5
${}\{{}$\1\6
\X18:Move \PB{\\{cur\_cell}} backward to the previous cell\X;\6
${}\|i\K\\{hack\_out}({*}\\{cur\_cell});{}$\6
${}\|p\K\\{hack\_clean}({*}\\{cur\_cell})\MG\\{serial};{}$\6
${}\\{printf}(\.{"\ \%s\%d"},\39\|i\AND\T{1}\?\.{"-"}:\.{""},\39\|p+\T{1});{}$\6
\4${}\}{}$\2\par
\U23.\fi

\M[391 sat-to-dimacs.w]{25}\B\X25:Check consistency\X${}\E{}$\6
\&{if} ${}(\\{cur\_cell}\I{\AND}\\{cur\_chunk}\MG\\{cell}[\T{0}]\V\\{cur%
\_chunk}\MG\\{prev}\I\NULL\V\\{cur\_tmp\_var}\I{\AND}\\{cur\_vchunk}\MG\\{var}[%
\T{0}]\V\\{cur\_vchunk}\MG\\{prev}\I\NULL){}$\5
${}\{{}$\1\6
${}\\{fprintf}(\\{stderr},\39\.{"This\ can't\ happen\ (}\)\.{consistency\ check%
\ fa}\)\.{ilure)!\\n"});{}$\6
${}\\{exit}({-}\T{14});{}$\6
\4${}\}{}$\2\par
\U21.\fi

\N[400 sat-to-dimacs.w]{1}{26}Index.


\fi


\inx
\fin
\con
