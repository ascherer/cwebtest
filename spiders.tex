\input cwebmac
\srcloctrue
\datethis
\input epsf
\def\dts{\mathinner{\ldotp\ldotp}}
\let\from=\gets
\def\bit#1{\\{bit}[#1]}
\def\losub#1{^{\vphantom\prime}_{#1}} % for contexts like $x'_n+x\losub n$


\N[8 spiders.w]{1}{1}Introduction. The purpose of this program is to implement
a pretty
algorithm that has a very pleasant theory. But I apologize
at the outset that the algorithm seems to be rather subtle, and
I have not been able to think of any way to explain it to dummies.
Readers who like discrete mathematics and computer science are
encouraged to persevere nonetheless.

An overview of the relevant theory appears in a paper called ``Deconstructing
coroutines,'' by D.~E. Knuth and F.~Ruskey, but this program tries to be
self-contained. Earlier versions of the ideas were embedded in now-obsolete
programs called {\mc KODA-RUSKEY} and {\mc LI-RUSKEY}, written in June, 2001.

\Y\B\8\#\&{include} \.{<stdio.h>}\6
\X4:Global variables\X\6
\X13:Subroutines\X\7
\&{int} \\{main}(\&{int} \\{argc}${},\39{}$\&{char} ${}{*}\\{argv}[\,]){}$\1\1%
\2\2\6
${}\{{}$\1\6
\X5:Local variables\X;\6
\X3:Parse the command line\X;\6
\X7:Initialize the data structures\X;\6
\X24:Generate the answers\X;\6
\&{return} \T{0};\6
\4${}\}{}$\2\par
\fi

\M[33 spiders.w]{2}Given a digraph that is {\it totally acyclic}, in the sense
that it has
no cycles when we ignore the arc directions, we want to find all ways to
label its vertices with 0s and 1s in such a way that
$x\to y$ implies $\bit x\le\bit y$.
Moreover, we want to list all such labelings as a Gray path,
changing only one bit at a time. The algorithm below does this, with
an extra proviso: Given a designated ``root'' vertex~$v$, $\bit v$
begins at 0 and changes exactly once.

For brevity, a totally acyclic digraph is called a {\it tad}, and a
connected tad is called a {\it spider}.

The simple three-vertex spider with $x\to y\from z$ has only five such
labelings, and they form a Gray path in essentially only one way,
namely $(000,010,011,111,110)$. This example shows that we cannot require
the Gray path to end at a convenient prespecified labeling like $11\ldots1$;
and the dual graph, obtained by reversing all the arrows and complementing
all the bits, shows that we can't require the path to start at $00\ldots0$.
[Generalizations of this example, in which the vertices are
$\{x_1,x_2,\ldots,x_n\}$ and each arc is either $x_{k-1}\to x_k$ or
$x_{k-1}\from x_k$, have solutions related to continued fractions.
Interested readers will enjoy working out the details.]

It is convenient to describe a tad by using a variant
of right-Polish notation, where a dot means ``put a new node on the stack''
and where a \.+ or \.- sign means ``draw an arc $x\from y$ (\.+) or
$x\to y$ (\.-) and remove $y$ from the stack,'' if $x$ and~$y$ are the
top stack elements. For example, a digraph with four vertices and no arcs is
represented by `\.{....}'; the digraph $1\to2\from3$ is `\.{...+-}', and
$2\to1\from 3$ is `\.{..+.+}', numbering the dots from left to right. This
numbering corresponds to {\it preorder\/} of the forest that is obtained if we
ignore arc directions.

The Polish notation implicitly specifies a hierarchical order, if the
\.+ and \.- operations make $y$ a child of~$x$. Using this tree structure,
each node of the digraph defines a subtree, and that subtree is a spider.
The Gray path we
construct is obtained by judiciously combining the Gray paths obtained
from the spiders.

\fi

\M[73 spiders.w]{3}In the following program, the parent of node $k$ is \PB{%
\\{par}[\|k]}, and the
arc between $k$ and its parent goes toward \PB{\\{par}[\|k]} if \PB{$\\{sign}[%
\|k]\K\T{1}$},
toward~$k$ if \PB{$\\{sign}[\|k]\K\T{0}$}.
The spider corresponding to~$k$ consists of nodes $k$ through
\PB{\\{scope}[\|k]}, inclusive. The Gray path corresponding to this spider
will be called $G_k$.

While we build the data structures, we might as well compute also \PB{%
\\{rchild}[\|k]}
and \PB{\\{lsib}[\|k]}, the rightmost child and left sibling of node~\PB{\|k}.
Then we
have a triply linked tree.

\Y\B\4\D$\\{maxn}$ \5
\T{100}\C{ limit on number of vertices }\par
\B\4\D$\\{abort}(\|f,\|d,\|n)$ \5
${}\{{}$\5
\1${}\\{fprintf}(\\{stderr},\39\|f,\39\|d){}$;\5
${}\\{exit}({-}\|n){}$;\5
${}\}{}$\2\par
\Y\B\4\X3:Parse the command line\X${}\E{}$\6
${}\{{}$\1\6
\&{register} \&{char} ${}{*}\|c;{}$\7
\&{if} ${}(\\{argc}<\T{2}\V\\{argc}>\T{3}\V(\\{argc}\E\T{3}\W\\{sscanf}(%
\\{argv}[\T{2}],\39\.{"\%d"},\39{\AND}\\{verbose})\I\T{1})){}$\1\5
${}\\{abort}(\.{"Usage:\ \%s\ graphspec}\)\.{ification\ [verbosity}\)\.{]\\n"},%
\39\\{argv}[\T{0}],\39\T{1});{}$\2\6
\&{for} ${}(\|c\K\\{argv}[\T{1}],\39\|j\K\|n\K\T{0};{}$ ${}{*}\|c;{}$ ${}\|c%
\PP){}$\1\6
\&{switch} ${}({*}\|c){}$\5
${}\{{}$\1\6
\4\&{case} \.{'.'}:\5
\&{if} ${}(\|n\E\\{maxn}-\T{1}){}$\1\5
${}\\{abort}(\.{"Sorry,\ I\ can\ only\ h}\)\.{andle\ \%d\ vertices!\\n}\)\.{"},%
\39\\{maxn}-\T{1},\39\T{2});{}$\2\6
${}\\{stack}[\|j\PP]\K\PP\|n{}$;\5
\&{break};\6
\4\&{case} \.{'+'}:\5
\&{case} \.{'-'}:\5
\&{if} ${}(\|j<\T{2}){}$\1\5
${}\\{abort}(\.{"Parsing\ error:\ `\%s'}\)\.{\ should\ start\ with\ `}\)\.{.'!%
\\n"},\39\|c,\39\T{3});{}$\2\6
${}\|j\MM,\39\|k\K\\{stack}[\|j],\39\|l\K\\{stack}[\|j-\T{1}];{}$\6
${}\\{sign}[\|k]\K({*}\|c\E\.{'+'}\?\T{1}:\T{0});{}$\6
${}\\{par}[\|k]\K\|l,\39\\{lsib}[\|k]\K\\{rchild}[\|l],\39\\{rchild}[\|l]\K%
\|k;{}$\6
${}\\{scope}[\|k]\K\|n;{}$\6
\&{break};\6
\4\&{default}:\5
${}\\{abort}(\.{"Parsing\ error:\ `\%s'}\)\.{\ should\ start\ with\ `}\)\.{.'\
or\ `+'\ or\ `-'!\\n"},\39\|c,\39\T{4});{}$\6
\4${}\}{}$\2\2\6
${}\\{scope}[\T{0}]\K\|n,\39\\{sign}[\T{0}]\K\T{1},\39\\{rchild}[\T{0}]\K%
\\{stack}[\MM\|j];{}$\6
\&{for} ${}(\|k\K\|n;{}$ ${}\|j\G\T{0};{}$ ${}\|j\MM){}$\5
${}\{{}$\1\6
${}\|l\K\\{stack}[\|j],\39\\{scope}[\|l]\K\|k,\39\|k\K\|l-\T{1};{}$\6
\&{if} ${}(\|j>\T{0}){}$\1\5
${}\\{lsib}[\|l]\K\\{stack}[\|j-\T{1}];{}$\2\6
\4${}\}{}$\2\6
\4${}\}{}$\2\par
\U1.\fi

\M[113 spiders.w]{4}\B\X4:Global variables\X${}\E{}$\6
\&{int} \\{par}[\\{maxn}];\C{ the parent of $k$ }\6
\&{int} \\{sign}[\\{maxn}];\C{ 0 if $\PB{\\{par}}[k]\to k$,\quad 1 if $\PB{%
\\{par}}[k]\from k$ }\6
\&{int} \\{scope}[\\{maxn}];\C{ rightmost element of the spider $k$ }\6
\&{int} \\{stack}[\\{maxn}];\C{ vertices whose \PB{\\{scope}} is not yet set }\6
\&{int} \\{rchild}[\\{maxn}]${},{}$ \\{lsib}[\\{maxn}];\C{ tree links for
traversal }\6
\&{int} \\{verbose};\C{ controls the amount of output }\par
\As9, 12, 18, 25\ETs32.
\U1.\fi

\M[121 spiders.w]{5}\B\X5:Local variables\X${}\E{}$\6
\&{register} \&{int} \|j${},{}$ \|k${},{}$ \|l${}\K\T{0}{}$;\C{ heavily-used
miscellaneous indices }\6
\&{int} \|n;\C{ size of the input graph }\par
\U1.\fi

\M[125 spiders.w]{6}Consider the example spider
$$\epsfbox{deco.5}$$
in which all arcs are directed upward; it could be written
\.{....+-.--..+-..-+} in Polish notation. Vertex~1 is the root.
A nonroot vertex~$k$ is
called {\it positive\/} if $\PB{\\{par}}[k]\to k$ and {\it negative\/}
if $\PB{\\{par}}[k]\from k$;
thus $\{2,3,5,6,9\}$ are positive in this example, and $\{4,7,8\}$ are
negative.

We write $j\to^* k$ if there is a directed path from $j$ to $k$.
Removing all vertices $j$ such that $j\to^* k$ disconnects spider~$k$
into a number of pieces having positive roots; in our example, removing
$\{1,8\}$ leaves three components rooted at $\{2,6,9\}$. We call these roots
the set of {\it positive vertices near\/}~$k$, and denote that set by~$U_k$.
Similarly, the {\it negative vertices near\/}~$k$ are obtained when we remove
all $j$ such that $k\to^* j$; the set of resulting roots, denoted by~$V_k$,
is $\{4,7,8\}$ in our example.

Why are the sets $U_k$ and $V_k$ so important? Because
the labelings of the $k$th spider for which $\bit k=0$ are
precisely those that we obtain by setting $\bit j=0$ for all $j\to^* k$
and then labeling each spider $u$ for $u\in U_k$.
Similarly, all labelings for which $\bit k=1$ are obtained by
setting $\bit j=1$ for all $k\to^* j$ and labeling each spider
$v$ for $v\in V_k$.

Thus if $n_k$ denotes the number of labelings of spider~$k$,
we have $n_k=\prod_{u\in U_k}n_u\,+\,\prod_{v\in V_k}n_v$.

\smallskip

Every positive child of $k$ appears in $U_k$, and every negative
child appears in $V_k$. These are called the {\it principal\/}
elements of~$U_k$ and~$V_k$. Every nonprincipal member of~$U_k$ is a member
of $U_v$ for some unique principal vertex~$v\in V_k$. Similarly, every
nonprincipal member of~$V_k$ is a member of $V_u$ for some unique
principal vertex $u\in U_k$. For example, 9~is a nonprincipal member of~$U_1$
and it also belongs to~$U_8$; 4~is a nonprincipal member of~$V_1$ and it
also belongs to~$V_2$.

If $k$ is a root of the given digraph, we say that $k$'s parent is~0.
This dummy vertex 0 is assumed to have arcs to all such~$k$, and it follows
that $U_0$ is the collection of those root vertices; the total number
of labelings is therefore $\prod_{u\in U_0}n_u$.
According to this convention, the root vertices
are considered to be positive. We also regard 0 as negative.

For example, the sample spider above has the following characteristics:
$$\vbox{\halign{$\hfil#\hfil$\quad&
$\hfil#\hfil$\quad&
$\hfil#\hfil$\quad&
$\hfil#\hfil$\quad&
$\hfil#\hfil$\quad&
$\hfil#\hfil$\quad&
$\hfil#\hfil$\quad&
$\hfil#\hfil$\quad&
$\hfil#+{}$&$#\hfil={}$&\hfil#\cr
k&\PB{\\{sign}}[k]&\PB{\\{scope}}[k]&\PB{\\{par}}[k]&\PB{\\{rchild}}[k]&\PB{%
\\{lsib}}[k]&U_k&V_k&
\multispan3{\hfil$n_k$\hfil}\cr
\noalign{\vskip2pt}
0&1&9&0&1& &\{1\}&\{4,7,8\}\cr
1&0&9&0&8&0&\{2,6,9\}&\{4,7,8\}&48&12&60\cr
2&0&5&1&5&0&\{3,5\}&\{4\}&6&2&8\cr
3&0&4&2&4&0&\emptyset&\{4\}&1&2&3\cr
4&1&4&3&0&0&\emptyset&\emptyset&1&1&2\cr
5&0&5&2&0&3&\emptyset&\emptyset&1&1&2\cr
6&0&7&1&7&2&\emptyset&\{7\}&1&2&3\cr
7&1&7&6&0&0&\emptyset&\emptyset&1&1&2\cr
8&1&9&1&9&6&\{9\}&\emptyset&2&1&3\cr
9&0&9&8&0&0&\emptyset&\emptyset&1&1&2\cr}}$$

\fi

\M[197 spiders.w]{7}We don't want to compute the sets $U_1$, \dots, $U_n$
explicitly,
because the total number of elements $\vert U_1\vert+\cdots+\vert U_n\vert$
can be $\Omega(n^2)$ in cases like $\..^{n/2}(\.{.+})^{n/2}\.+^{n/2-1}$.
But luckily for us, there is a nice way to represent all of those sets
implicitly, computing the representation in linear time.

Suppose $u$ is a positive vertex, not a root, so that $u\from v_1$ where
$v_1$ is $u$'s parent and $v_1\ne0$. If $v_1$ is negative, let $v_2$ be the
parent of $v_1$, and continue until reaching a positive vertex~$v_j$.
We call $v_j$ the {\it positive progenitor\/} of $v_1$; it is also
the positive progenitor of  $v_2$, \dots,~$v_{j-1}$, and itself.
By definition, $u\in U_k$ if and only if $k\in\{v_1,\ldots,v_j\}$.
It follows that $U_k=U_{k'}\cap\bigl[k\dts\PB{\\{scope}}[k]\bigr]$ when $k'$ is
the positive progenitor of~$k$.

We can therefore represent all the sets $U_k$ by linking their elements
together explicitly whenever $k$ is a positive vertex; such sets
$U_k$ are disjoint. Then if we compute \PB{\\{umax}[\|k]} for {\it every\/}
vertex~$k$, namely the index of the largest element of $U_k$, the set $U_k$
will consist of \PB{\\{umax}[\|k]}, \PB{\\{prev}[\\{umax}[\|k]]}, \PB{\\{prev}[%
\\{prev}[\\{umax}[\|k]]]}, etc.,
proceeding until reaching an element less than~$k$.

One pass through the forest in preorder suffices to compute the \PB{\\{prev}}
values.
A second pass in reverse postorder suffices to compute each \PB{\\{umax}},
because postorder traverses nodes in order of their scopes.

A similar idea works, of course, for $V_1$, \dots, $V_n$, using
{\it negative\/} progenitors.

\Y\B\4\X7:Initialize the data structures\X${}\E{}$\6
\&{for} ${}(\|j\K\T{1};{}$ ${}\|j\Z\|n;{}$ ${}\|j\PP){}$\5
${}\{{}$\1\6
${}\|k\K\\{par}[\|j];{}$\6
\&{if} ${}(\\{sign}[\|j]\E\T{0}){}$\5
${}\{{}$\1\6
${}\\{ppro}[\|j]\K\|j,\39\\{npro}[\|j]\K\\{npro}[\|k];{}$\6
\&{if} (\|k)\1\5
${}\\{prev}[\|j]\K\\{umax}[\\{ppro}[\|k]],\39\\{umax}[\\{ppro}[\|k]]\K\|j;{}$\2%
\6
\&{else}\1\5
${}\\{prev}[\|j]\K\\{lsib}[\|j]{}$;\C{ special case when $j$ is a root }\2\6
\4${}\}{}$\5
\2\&{else}\5
${}\{{}$\1\6
${}\\{npro}[\|j]\K\|j,\39\\{ppro}[\|j]\K\\{ppro}[\|k];{}$\6
${}\\{prev}[\|j]\K\\{vmax}[\\{npro}[\|k]],\39\\{vmax}[\\{npro}[\|k]]\K\|j;{}$\6
\4${}\}{}$\2\6
\4${}\}{}$\2\6
\X8:Fill in all \PB{\\{umax}} and \PB{\\{vmax}} links, traversing in reverse
postorder\X;\par
\As10, 11\ETs16.
\U1.\fi

\M[240 spiders.w]{8}Tree traversal is great fun, when it works.

\Y\B\4\X8:Fill in all \PB{\\{umax}} and \PB{\\{vmax}} links, traversing in
reverse postorder\X${}\E{}$\6
$\\{lsib}[\T{0}]\K{-}\T{1}{}$;\C{ sentinel }\6
${}\\{ptr}[\T{0}]\K\\{vmax}[\T{0}];{}$\6
${}\\{umax}[\T{0}]\K\\{rchild}[\T{0}];{}$\6
\&{for} ${}(\|j\K\\{rchild}[\T{0}];{}$  ; \,)\5
${}\{{}$\1\6
\&{if} ${}(\\{sign}[\|j]\E\T{0}){}$\5
${}\{{}$\1\6
${}\\{ptr}[\|j]\K\\{umax}[\|j]{}$;\C{ this pointer will run through $U_j$ }\6
${}\|k\K\\{npro}[\|j],\39\|l\K\\{ptr}[\|k];{}$\6
\&{while} ${}(\|l>\\{scope}[\|j]){}$\1\5
${}\|l\K\\{prev}[\|l];{}$\2\6
${}\\{ptr}[\|k]\K\|l;{}$\6
\&{if} ${}(\|l>\|j){}$\1\5
${}\\{vmax}[\|j]\K\|l;{}$\2\6
\4${}\}{}$\5
\2\&{else}\5
${}\{{}$\1\6
${}\\{ptr}[\|j]\K\\{vmax}[\|j]{}$;\C{ this pointer will run through $V_j$ }\6
${}\|k\K\\{ppro}[\|j],\39\|l\K\\{ptr}[\|k];{}$\6
\&{while} ${}(\|l>\\{scope}[\|j]){}$\1\5
${}\|l\K\\{prev}[\|l];{}$\2\6
${}\\{ptr}[\|k]\K\|l;{}$\6
\&{if} ${}(\|l>\|j){}$\1\5
${}\\{umax}[\|j]\K\|l;{}$\2\6
\4${}\}{}$\2\6
\&{if} (\\{rchild}[\|j])\1\5
${}\|j\K\\{rchild}[\|j]{}$;\C{ now we move to the next node }\2\6
\&{else}\5
${}\{{}$\1\6
\&{while} ${}(\R\\{lsib}[\|j]){}$\1\5
${}\|j\K\\{par}[\|j];{}$\2\6
${}\|j\K\\{lsib}[\|j];{}$\6
\&{if} ${}(\|j<\T{0}){}$\1\5
\&{break};\2\6
\4${}\}{}$\2\6
\4${}\}{}$\2\par
\U7.\fi

\M[268 spiders.w]{9}The sample spider leads, for example, to the following
values:
$$\vbox{\halign{$\hfil#\hfil$\quad&
$\hfil#\hfil$\quad&
$\hfil#\hfil$\quad&
$\hfil#\hfil$\quad&
$\hfil#\hfil$\quad&
$\hfil#\hfil$\cr
k&\PB{\\{ppro}}[k]&\PB{\\{npro}}[k]&\PB{\\{prev}}[k]&\PB{\\{umax}}[k]&\PB{%
\\{vmax}}[k]\cr
\noalign{\vskip2pt}
0&0&0&0&1&8\cr
1&1&0&0&9&8\cr
2&2&0&0&5&4\cr
3&3&0&0&0&4\cr
4&3&4&0&0&0\cr
5&5&0&3&0&0\cr
6&6&0&2&0&7\cr
7&6&7&4&0&0\cr
8&1&8&7&9&0\cr
9&9&8&6&0&0\cr}}$$

\Y\B\4\X4:Global variables\X${}\mathrel+\E{}$\6
\&{int} \\{ppro}[\\{maxn}]${},{}$ \\{npro}[\\{maxn}];\C{ progenitors }\6
\&{int} \\{prev}[\\{maxn}];\C{ previous element in the same progenitorial list
}\6
\&{int} \\{ptr}[\\{maxn}];\C{ current element in such a list }\6
\&{int} \\{umax}[\\{maxn}]${},{}$ \\{vmax}[\\{maxn}];\C{ rightmost elements in
$U_k$, $V_k$ }\par
\fi

\M[294 spiders.w]{10}Aha! We can begin to see how to get the desired Gray path.
The well-known
reflected Gray code for mixed-radix number systems tells us how to obtain a
path $P_k$ of length $\prod_{u\in U_k}n_u$ for the labelings of
spider~$k$ with $\bit k=0$, as
well as a path $Q_k$ of length $\prod_{v\in V_k}n_v$ for the labelings with
$\bit k=1$. All we have to do is figure out a way to end $P_k$ with
a labeling that differs only in $\bit k$ from the starting point of $Q_k$;
then we can let $G_k$ be `$P_k, Q_k$'. And indeed, such a labeling
is fairly obvious: It consists of the last labeling of spider~$u$ for
each positive child $u$ of~$k$, and the first labeling of spider~$v$ for
each negative child~$v$.

If $j\in U_k$, the reflected code in $P_k$ involves traversing $G_j$
a total of
$$\delta_{jk}=
\prod_{\scriptstyle u<j_{\mathstrut}\atop\scriptstyle u\in U_k}n_u$$
times, in alternating directions. Similarly, if $j\in V_k$, the path $G_j$
is traversed
$$\delta_{jk}=
\prod_{\scriptstyle v<j_{\mathstrut}\atop\scriptstyle v\in V_k}n_v$$
times in $Q_k$. We need to know whether these numbers are even or odd,
in order to figure out how $P_k$ should begin and $Q_k$ should end,
because we know how $P_k$ ends and $Q_k$ begins.

The numbers $\delta_{jk}$ can get as large as $2^n$, and we don't want to
mess with $n$-bit arithmetic if we don't have~to.
The trick is to compute two more tables, \PB{\\{ueven}[\|k]} and \PB{\\{veven}[%
\|k]},
which point to the smallest elements $u\in U_k$ and $v\in V_k$ such that
$n_u$ and $n_v$ are even. (If no such elements exist,
\PB{\\{ueven}[\|k]} and/or \PB{\\{veven}[\|k]} are set to \PB{\\{maxn}},
representing $\infty$.)
These tables give us the information we need about $\delta_{jk}$.

We set \PB{\\{ueven}[\T{0}]} artificially to $\infty$. This has the effect of
keeping each component of the digraph independent.

While we're computing the \PB{\\{ueven}} and \PB{\\{veven}} tables, we might as
well
also compute \PB{\\{umin}} and \PB{\\{vmin}}, a counterpart of \PB{\\{umax}}
and \PB{\\{vmax}} that
will prove useful later.

\Y\B\4\X7:Initialize the data structures\X${}\mathrel+\E{}$\6
\&{for} ${}(\|k\K\T{0};{}$ ${}\|k\Z\|n;{}$ ${}\|k\PP){}$\1\5
${}\\{ueven}[\|k]\K\\{veven}[\|k]\K\\{umin}[\|k]\K\\{vmin}[\|k]\K\\{maxn};{}$\2%
\6
\&{for} ${}(\|j\K\|n;{}$ ${}\|j>\T{0};{}$ ${}\|j\MM){}$\5
${}\{{}$\1\6
${}\|k\K\\{ppro}[\|j];{}$\6
\&{if} ${}(\\{umin}[\|k]\Z\\{scope}[\|j]){}$\1\5
${}\\{umin}[\|j]\K\\{umin}[\|k];{}$\2\6
\&{if} ${}(\\{ueven}[\|k]\Z\\{scope}[\|j]){}$\1\5
${}\\{ueven}[\|j]\K\\{ueven}[\|k];{}$\2\6
${}\|k\K\\{npro}[\|j];{}$\6
\&{if} ${}(\\{vmin}[\|k]\Z\\{scope}[\|j]){}$\1\5
${}\\{vmin}[\|j]\K\\{vmin}[\|k];{}$\2\6
\&{if} ${}(\\{veven}[\|k]\Z\\{scope}[\|j]){}$\1\5
${}\\{veven}[\|j]\K\\{veven}[\|k];{}$\2\6
${}\|l\K(\\{ueven}[\|j]<\\{maxn})\XOR(\\{veven}[\|j]<\\{maxn}){}$;\C{ $l=n_j%
\bmod2$ }\6
${}\|k\K\\{par}[\|j];{}$\6
\&{if} ${}(\\{sign}[\|j]\E\T{0}){}$\5
${}\{{}$\1\6
${}\\{umin}[\\{ppro}[\|k]]\K\|j;{}$\6
\&{if} ${}(\|l\E\T{0}){}$\1\5
${}\\{ueven}[\\{ppro}[\|k]]\K\|j;{}$\2\6
\4${}\}{}$\5
\2\&{else}\5
${}\{{}$\1\6
${}\\{vmin}[\\{npro}[\|k]]\K\|j;{}$\6
\&{if} ${}(\|l\E\T{0}){}$\1\5
${}\\{veven}[\\{npro}[\|k]]\K\|j;{}$\2\6
\4${}\}{}$\2\6
\4${}\}{}$\2\6
${}\\{ueven}[\T{0}]\K\\{maxn}{}$;\par
\fi

\M[354 spiders.w]{11}Another thing we'll need to know is \PB{\\{umaxbit}[\|k]},
the value of
\PB{\\{bit}[\\{umax}[\|k]]} when \PB{\\{bit}[\|k]} changes from 0 to~1. And of
course
the dual value \PB{\\{vmaxbit}[\|k]} will be equally important.

\Y\B\4\X7:Initialize the data structures\X${}\mathrel+\E{}$\6
\&{for} ${}(\|k\K\|n;{}$ ${}\|k>\T{0};{}$ ${}\|k\MM){}$\5
${}\{{}$\1\6
${}\|l\K\\{par}[\|k];{}$\6
\&{if} ${}(\|k\E\\{umax}[\|l]){}$\1\5
${}\\{umaxbit}[\|l]\K\T{1};{}$\2\6
\&{else}\5
${}\{{}$\1\6
${}\|j\K\\{umax}[\|k];{}$\6
\&{if} ${}(\|j\W\\{umax}[\|l]\E\|j){}$\5
${}\{{}$\1\6
\&{if} ${}(\\{ueven}[\|k]<\|j){}$\1\5
${}\\{umaxbit}[\|l]\K\\{umaxbit}[\|k]{}$;\C{ $\delta_{jk}$ is even }\2\6
\&{else}\1\5
${}\\{umaxbit}[\|l]\K\T{1}\XOR\\{umaxbit}[\|k];{}$\2\6
\4${}\}{}$\2\6
\4${}\}{}$\2\6
\&{if} ${}(\|k\E\\{vmax}[\|l]){}$\1\5
${}\\{vmaxbit}[\|l]\K\T{0};{}$\2\6
\&{else}\5
${}\{{}$\1\6
${}\|j\K\\{vmax}[\|k];{}$\6
\&{if} ${}(\|j\W\\{vmax}[\|l]\E\|j){}$\5
${}\{{}$\1\6
\&{if} ${}(\\{veven}[\|k]<\|j){}$\1\5
${}\\{vmaxbit}[\|l]\K\\{vmaxbit}[\|k]{}$;\C{ $\delta_{jk}$ is even }\2\6
\&{else}\1\5
${}\\{vmaxbit}[\|l]\K\T{1}\XOR\\{vmaxbit}[\|k];{}$\2\6
\4${}\}{}$\2\6
\4${}\}{}$\2\6
\4${}\}{}$\2\par
\fi

\M[379 spiders.w]{12}For the record, our example spider has the following
additional
characteristics (including some that we'll introduce later):
$$\vbox{\halign{$\hfil#\hfil$\enspace&
$\hfil#\hfil$\enspace&
$\hfil#\hfil$\enspace&
$\hfil#\hfil$\enspace&
$\hfil#\hfil$\enspace&
$\hfil#\hfil$\enspace&
$\hfil#\hfil$\enspace&
$\hfil#\hfil$\enspace&
$\hfil#\hfil$\enspace&
$\hfil#\hfil$\cr
k&\PB{\\{bstart}}[k]&\PB{\\{umin}}[k]&\PB{\\{ueven}}[k]&\PB{\\{umaxbit}}[k]&%
\PB{\\{umaxscope}}[k]&
\PB{\\{vmin}}[k]&\PB{\\{veven}}[k]&\PB{\\{vmaxbit}}[k]&\PB{\\{vmaxscope}}[k]\cr
\noalign{\vskip2pt}
0& &1&\infty& & &4&4\cr
1&1&2&2&0&9&4&4&0&9\cr
2&2&3&5&1&5&4&4&1&4\cr
3&3&\infty&\infty&0&3&4&4&0&4\cr
4&4&\infty&\infty&0&4&\infty&\infty&0&4\cr
5&5&\infty&\infty&0&5&\infty&\infty&0&5\cr
6&6&\infty&\infty&0&6&7&7&0&7\cr
7&7&\infty&\infty&0&7&\infty&\infty&0&7\cr
8&8&9&9&1&9&\infty&\infty&0&8\cr
9&9&\infty&\infty&0&9&\infty&\infty&0&9\cr
}}$$

\Y\B\4\X4:Global variables\X${}\mathrel+\E{}$\6
\&{int} \\{umin}[\\{maxn}]${},{}$ \\{vmin}[\\{maxn}];\C{ the smallest guys in
$U_k$, $V_k$ }\6
\&{int} \\{ueven}[\\{maxn}]${},{}$ \\{veven}[\\{maxn}];\C{ the smallest even
guys in $U_k$, $V_k$ }\6
\&{int} \\{umaxbit}[\\{maxn}]${},{}$ \\{vmaxbit}[\\{maxn}];\C{ significant
transition bits }\6
\&{int} \\{bit}[\\{maxn}];\C{ the current labeling }\par
\fi

\M[412 spiders.w]{13}A somewhat subtle point arises here, and it provides an
important
simplification: Suppose $j$ is a negative child of~$k$,
and $\PB{\\{ueven}}[k]\ge j$. Then the initial bits of spider~$j$ in the
sequence
for spider~$k$ are the same as the transition bits of spider~$j$.
The reason is that $\delta_{ij}+\delta_{ik}$ is even for all
$i\in U_j$.

Using this principle, we can write recursive procedures so that \PB{%
\\{setfirst}(\T{0})}
computes the very first setting the bit table, in $O(n)$ steps.
(This bound on the running time comes from the fact that each procedure sets
the bits of a subspider using a number of steps bounded by a constant times the
number of bits being set. Formally, if $T_n\le
a+(b+T_{n_1})+\cdots+(b+T_{n_t})$ where $n_1+\cdots+n_t=n-1$, then
it follows by induction that $T_n\le(a+b)n-b$.)

The first labeling of our example spider uses the first labeling
of subspider~2, the last labeling of subspider~6, and the first labeling
of subspider~8, so it is
$\PB{\\{bit}}[1]\ldots\PB{\\{bit}}[9]=000001100$.

Recursion is lots of fun too. Why do I sometimes prefer traversal?

\Y\B\4\X13:Subroutines\X${}\E{}$\6
\&{void} \\{setlast}(\&{register} \&{int} \|k);\C{ see below }\6
\&{void} \\{setmid}(\&{register} \&{int} \|k${},\39{}$\&{int} \|b);\C{ ditto }\7
\&{void} \\{setfirst}(\&{register} \&{int} \|k)\1\1\2\2\6
${}\{{}$\1\6
\&{register} \&{int} \|j;\7
${}\\{bit}[\|k]\K\T{0};{}$\6
\&{for} ${}(\|j\K\\{rchild}[\|k];{}$ \|j; ${}\|j\K\\{lsib}[\|j]){}$\1\6
\&{if} ${}(\\{sign}[\|j]\E\T{0}){}$\5
${}\{{}$\1\6
\&{if} ${}(\\{ueven}[\|k]\G\|j){}$\1\5
\\{setfirst}(\|j);\C{ $\delta_{jk}$ is odd }\2\6
\&{else}\1\5
\\{setlast}(\|j);\2\6
\4${}\}{}$\5
\2\&{else} \&{if} ${}(\\{ueven}[\|k]\G\|j){}$\1\5
${}\\{setmid}(\|j,\39\T{0}){}$;\C{ by the subtle point }\2\6
\&{else}\1\5
\\{setfirst}(\|j);\C{ $\delta_{ik}$ is even for all $i\in U_j$ }\2\2\6
\4${}\}{}$\2\7
\&{void} \\{setlast}(\&{register} \&{int} \|k)\1\1\2\2\6
${}\{{}$\1\6
\&{register} \&{int} \|j;\7
${}\\{bit}[\|k]\K\T{1};{}$\6
\&{for} ${}(\|j\K\\{rchild}[\|k];{}$ \|j; ${}\|j\K\\{lsib}[\|j]){}$\1\6
\&{if} ${}(\\{sign}[\|j]\E\T{1}){}$\5
${}\{{}$\1\6
\&{if} ${}(\\{veven}[\|k]\G\|j){}$\1\5
\\{setlast}(\|j);\C{ $\delta_{jk}$ is odd }\2\6
\&{else}\1\5
\\{setfirst}(\|j);\2\6
\4${}\}{}$\5
\2\&{else} \&{if} ${}(\\{veven}[\|k]\G\|j){}$\1\5
${}\\{setmid}(\|j,\39\T{1}){}$;\C{ by the subtle point }\2\6
\&{else}\1\5
\\{setlast}(\|j);\C{ $\delta_{ik}$ is even for all $i\in V_j$ }\2\2\6
\4${}\}{}$\2\7
\&{void} \\{setmid}(\&{register} \&{int} \|k${},\39{}$\&{int} \|b)\1\1\2\2\6
${}\{{}$\1\6
\&{register} \&{int} \|j;\7
${}\\{bit}[\|k]\K\|b;{}$\6
\&{for} ${}(\|j\K\\{rchild}[\|k];{}$ \|j; ${}\|j\K\\{lsib}[\|j]){}$\1\6
\&{if} ${}(\\{sign}[\|j]\E\T{0}){}$\1\5
\\{setlast}(\|j);\5
\2\&{else}\1\5
\\{setfirst}(\|j);\2\2\6
\4${}\}{}$\2\par
\A19.
\U1.\fi

\N[467 spiders.w]{1}{14}The active list. Reflected Gray code is nicely
generated by a process
based on a list of elements that are alternately active and passive.
(See, for example, Algorithm 7.2.1.1L in {\sl The Art of Computer
Programming}.) A slight generalization of that notion works admirably for the
problem faced here: We maintain a so-called {\it active list\/} $L$, whose
elements are alternately awake and asleep. Elements are occasionally inserted
into~$L$ and/or deleted from~$L$ according to the following protocol:
\smallskip
\itemitem{1)} Find the largest node $k\in L$ that is awake, and wake up
all elements of $L$ that exceed~$k$.
\itemitem{2)} If \PB{$\\{bit}[\|k]\K\T{0}$}, set $\PB{\\{bit}}[k]\gets1$ and
$L\gets(L\setminus U'_k)\cup V'_k$; otherwise set $\PB{\\{bit}}[k]\gets0$ and
$L\gets(L\setminus V'_k)\cup U'_k$. Here $U'_k$ and $V'_k$ denote the
{\it principal elements\/} of $U_k$ and $V_k$, namely the positive
and negative children of~$k$.
\itemitem{3)} Put $k$ to sleep.
\smallskip\noindent
The process stops when all elements of $L$ are asleep in step 1. In such a
case, waking them up and repeating the process will run through the
bit labelings again, but in reverse order.

The elements of $L$ are the positive vertices $k$ for which \PB{$\\{bit}[%
\\{par}[\|k]]\K\T{0}$}
and the negative vertices $k$ for which \PB{$\\{bit}[\\{par}[\|k]]\K\T{1}$}.
For example, the initial active list for the example spider
is $L=\{1,2,3,5,6,7,9\}$. All elements are awake at the beginning.

\def\p#1{\overline#1}
We can conveniently represent $L$ as a list of elements $k$, with
subscripts to indicate the current setting of \PB{\\{bit}[\|k]}. Then $k_0$ is
always followed in the list by sublists for the spiders of~$U_k$, and
$k_1$ is always followed by sublists for the spiders of~$V_k$. With
these conventions, the initial active list is
$$1_0\quad 2_0\quad 3_0\quad 5_0\quad 6_1\quad 7_1\quad 9_0\,.$$
Since $9_0$ is awake, we complement \PB{\\{bit}[\T{9}]}, and $L$ becomes
$$1_0\quad 2_0\quad 3_0\quad 5_0\quad 6_1\quad 7_1\quad \p9_1\,.$$
The bar over 9 indicates that this node is now asleep.

The next step complements \PB{\\{bit}[\T{7}]} and wakes up 9; thus the first
few steps take place as follows:
$$\vbox{\halign{$#\hfil$&
\quad\smash{\lower.5\baselineskip\hbox{$\cdots$ complement $\bit#$\hfil}}\cr
1_0\quad 2_0\quad 3_0\quad 5_0\quad 6_1\quad 7_1\quad 9_0&9\cr
1_0\quad 2_0\quad 3_0\quad 5_0\quad 6_1\quad 7_1\quad \p9_1&7\cr
1_0\quad 2_0\quad 3_0\quad 5_0\quad 6_1\quad \p7_0\quad 9_1&9\cr
1_0\quad 2_0\quad 3_0\quad 5_0\quad 6_1\quad \p7_0\quad \p9_0&6\cr
1_0\quad 2_0\quad 3_0\quad 5_0\quad \p6_0\quad 9_0&9\cr
1_0\quad 2_0\quad 3_0\quad 5_0\quad \p6_0\quad \p9_1&5\cr
1_0\quad 2_0\quad 3_0\quad \p5_1\quad 6_0\quad 9_1&9\cr
1_0\quad 2_0\quad 3_0\quad \p5_1\quad 6_0\quad \p9_0&6\cr
1_0\quad 2_0\quad 3_0\quad \p5_1\quad \p6_1\quad 7_0\quad 9_0\cr}}$$
Notice that 7 disappears from $L$ when \PB{\\{bit}[\T{6}]} becomes 0, but it
comes back again when \PB{\\{bit}[\T{6}]} reverts to~1. Soon \PB{\\{bit}[%
\T{3}]} will
change to~1, and $4_0$ will enter the fray.

The most dramatic change will occur after the first $n_2n_6n_9=48$ labelings,
when \PB{\\{bit}[\T{1}]} changes:
$$\vbox{\halign{$#\hfil$&
\quad\smash{\lower.5\baselineskip\hbox{$\cdots$ complement $\bit#$\hfil}}\cr
1_0\quad \p2_1\quad \p4_0\quad \p6_1\quad \p7_1\quad \p9_0&1\cr
\p1_1\quad 4_0\quad 7_1\quad 8_0\quad 9_0&9\cr
\p1_1\quad 4_0\quad 7_1\quad 8_0\quad \p9_1&8\cr
\p1_1\quad 4_0\quad 7_1\quad \p8_1&7\cr
\qquad\vdots&8\cr
\p1_1\quad \p4_1\quad \p7_1\quad \p8_0\quad 9_1&9\cr
\p1_1\quad \p4_1\quad \p7_1\quad \p8_0\quad \p9_0\cr}}$$
And finally the whole list is asleep; all 60 labelings have been generated.

\fi

\M[534 spiders.w]{15}Using the active list protocol, the average amount of work
per bit change is
only $O(1)$ when amortized over the entire computation, even if we do a
sequential search for~$k$ in step~(1) and recopy all elements greater than~$k$
in step~(2). Our implementation goes beyond the notion of amortization,
however; after $O(n)$ steps of initialization, the program below does at most
a bounded number of operations between bit changes. Thus it is actually {\it
loopless}, in the sense defined by Gideon Ehrlich [{\sl Journal of the ACM\/
\bf20} (1973), 500--513].

The extra contortions that we need to go through in order to achieve
looplessness are usually ill-advised, because they actually cause the
total execution time to be longer than it would be with a more straightforward
algorithm. But hey, looplessness carries an academic cachet.
So we might as well treat this task as a
challenging exercise that might help us to sharpen our algorithmic wits.

(There may actually be a loopless algorithm for this problem that does not
slow down the total execution time. For example, the loopless implementation
in the program {\mc LI-RUSKEY} is quite fast, but it sometimes needs
$\Omega(n^2)$ steps for initialization and $\Omega(n^2)$ space for tables.
The existence of such a fast implementation suggests that
totally acyclic digraphs might well have additional properties that
yield improvements over the approach taken here; readers are encouraged to
find a better way.)

The first step we shall take toward a loopless algorithm is to introduce
``focus pointers,'' as in Ehrlich's Algorithm 7.2.1.1L. Usually \PB{$\\{focus}[%
\|k]\K\|k$},
except when $k$ is asleep and the successor of~$k$ is awake. In the latter
case, \PB{\\{focus}[\|k]} is the largest $j<k$ such that $j$ is awake.

The active list will be doubly linked, with $k$ preceded by \PB{\\{left}[\|k]}
and
followed by \PB{\\{right}[\|k]}. We make the list circular by letting \PB{%
\\{left}[\T{0}]} be its
rightmost element and \PB{\\{right}[\T{0}]} the leftmost. Then, for example,
\PB{\\{focus}[\\{left}[\T{0}]]} will be the element~$k$ needed in step~(1) of
the protocol.
We can wake up all elements to $k$'s right by setting
\PB{$\\{focus}[\\{left}[\T{0}]]\K\\{left}[\T{0}]$}, and we can put $k$ to sleep
by setting
\PB{$\\{focus}[\|k]\K\\{focus}[\\{left}[\|k]]$}, \PB{$\\{focus}[\\{left}[\|k]]%
\K\\{left}[\|k]$}.

\fi

\M[572 spiders.w]{16}Now let's focus on the implementation of step (2), which
is the
heart of the computation.

A positive child $j$ of $k$ is called ``simple'' if $V_j$ is empty; a
negative child is called simple if $U_j$ is empty. Same-sign siblings always
enter or leave the active list as a unit. Therefore if $j$ and $j'$ have the
same sign and if $j=\PB{\\{lsib}}[j']$ is simple,  we will have $\PB{%
\\{right}}[j]=j'$ and
$\PB{\\{left}}[j']=j$ whenever they are inserted or deleted. These links can be
established as part of the initialization. On the other hand when $j$
cannot be combined with its right neighbor, we compute \PB{\\{bstart}[\|j]},
the
leftmost sibling that forms a block with~$j$. (Possibly \PB{$\\{bstart}[\|j]\K%
\|j$}.)

The following preprocessing steps establish the initial values of \PB{%
\\{bstart}},
\PB{\\{left}}, and \PB{\\{right}}. They also compute two further quantities
that sometimes
turn out to be indispensable: \PB{\\{umaxscope}[\|k]} is the largest node that
is
forced to be in the active list at a transition point
when \PB{$\\{bit}[\|k]\K\T{0}$}, and \PB{\\{vmaxscope}[\|k]} is
the corresponding quantity when \PB{$\\{bit}[\|k]\K\T{1}$}.

\Y\B\4\X7:Initialize the data structures\X${}\mathrel+\E{}$\6
\&{for} ${}(\|k\K\|n;{}$ \|k; ${}\|k\MM){}$\5
${}\{{}$\1\6
${}\|j\K\\{lsib}[\|k];{}$\6
\&{if} (\|j)\1\5
${}\\{left}[\|k]\K\|j,\39\\{right}[\|j]\K\|k;{}$\2\6
\&{else}\1\5
\X17:Compute the \PB{\\{bstart}} links for \PB{\|k}'s family\X;\2\6
${}\|j\K\\{umax}[\|k];{}$\6
\&{if} ${}(\R\|j){}$\1\5
${}\\{umaxscope}[\|k]\K\|k;{}$\2\6
\&{else}\1\5
${}\\{umaxscope}[\|k]\K(\\{umaxbit}[\|k]\E\T{1}\?(\\{vmax}[\|j]\?\\{vmax}[\|j]:%
\|j):\\{umaxscope}[\|j]);{}$\2\6
${}\|j\K\\{vmax}[\|k];{}$\6
\&{if} ${}(\R\|j){}$\1\5
${}\\{vmaxscope}[\|k]\K\|k;{}$\2\6
\&{else}\1\5
${}\\{vmaxscope}[\|k]\K(\\{vmaxbit}[\|k]\E\T{0}\?(\\{umax}[\|j]\?\\{umax}[\|j]:%
\|j):\\{vmaxscope}[\|j]);{}$\2\6
\4${}\}{}$\2\par
\fi

\M[604 spiders.w]{17}\B\X17:Compute the \PB{\\{bstart}} links for \PB{\|k}'s
family\X${}\E{}$\6
\&{for} ${}(\|j\K\|l\K\|k;{}$ \|j; ${}\|j\K\\{right}[\|j]){}$\5
${}\{{}$\1\6
\&{if} ${}(\\{right}[\|j]\W\\{sign}[\|j]\E\\{sign}[\\{right}[\|j]]\W\3{-1}((%
\\{sign}[\|j]\E\T{0}\W\R\\{vmax}[\|j])\V(\\{sign}[\|j]\E\T{1}\W\R\\{umax}[%
\|j]))){}$\1\5
\&{continue};\2\6
${}\\{bstart}[\|j]\K\|l,\39\|l\K\\{right}[\|j];{}$\6
\4${}\}{}$\2\par
\U16.\fi

\M[611 spiders.w]{18}\B\X4:Global variables\X${}\mathrel+\E{}$\6
\&{int} \\{left}[\\{maxn}]${},{}$ \\{right}[\\{maxn}];\C{ neighbors in the
active list }\6
\&{int} \\{bstart}[\\{maxn}];\C{ start of a block }\6
\&{int} \\{umaxscope}[\\{maxn}]${},{}$ \\{vmaxscope}[\\{maxn}];\C{ extreme
nodes when \PB{\\{bit}[\|k]} changes }\6
\&{int} \\{flag}[\\{maxn}];\C{ nonzero when an insertion or deletion is needed
}\6
\&{int} \\{focus}[\\{maxn}];\C{ pointers that encode wakefulness }\par
\fi

\M[618 spiders.w]{19}When \PB{\\{bit}[\|k]} changes from 0 to 1, we want to
delete $k$'s positive blocks of
children from the active list and insert the negative ones. The rightmost
block is addressed by \PB{\\{rchild}[\|k]}, and we get to the others by
following
\PB{\\{bstart}} and \PB{\\{lsib}} links. Our algorithm is supposed to be
loopless, so we
can't do all this updating at once. Therefore we do only the rightmost step,
and we plant a warning in the data structure so that subsequent steps
will be performed before the missing information is needed. All nodes are
awake while waiting to be inserted or deleted, so the focus pointers are
unaffected by these delayed actions.

The \PB{\\{fixup}} subroutine is the basic mechanism by which nodes
enter or leave the active list. This subroutine not only inserts
or deletes a block of children, it also inserts a flag so that
the previous block will be fixed in due time.

\Y\B\4\X13:Subroutines\X${}\mathrel+\E{}$\6
\&{void} \\{fixup}(\&{register} \&{int} \|k${},\39{}$\&{register} \&{int} \|l)%
\1\1\2\2\6
${}\{{}$\1\6
\&{register} \&{int} \|i${},{}$ \|j;\7
${}\\{flag}[\|l]\K\T{0};{}$\6
\&{if} ${}(\|k>\T{0}){}$\1\5
\X20:Insert block $k$ before $l$ and \PB{\&{return}}\X\2\6
\X21:Delete block $k$ before $l$\X;\6
\4${}\}{}$\2\par
\fi

\M[642 spiders.w]{20}Once the process has gotten started, \PB{\\{left}[\|j]}
and \PB{\\{right}[\|k]} will
already have the correct values, unchanged from the time block $k$ was
previously deleted. But we don't make use of this fact, because we don't
want to worry about presetting \PB{\\{left}[\|j]} and \PB{\\{right}[\|k]} when
the
action list is initialized.

\Y\B\4\X20:Insert block $k$ before $l$ and \PB{\&{return}}\X${}\E{}$\6
${}\{{}$\1\6
${}\|j\K\\{bstart}[\|k],\39\|i\K\\{lsib}[\|j];{}$\6
${}\\{left}[\|j]\K\\{left}[\|l],\39\\{right}[\\{left}[\|l]]\K\|j;{}$\6
${}\\{left}[\|l]\K\|k,\39\\{right}[\|k]\K\|l;{}$\6
\&{if} (\|i)\5
${}\{{}$\1\6
\&{if} ${}(\\{sign}[\|k]\E\T{1}){}$\5
${}\{{}$\1\6
\&{if} ${}(\\{sign}[\|i]\E\T{0}){}$\5
${}\{{}$\1\6
\&{if} ${}(\\{vmin}[\|i]<\\{maxn}){}$\1\5
${}\|j\K\\{vmin}[\|i];{}$\2\6
${}\|i\K{-}\|i{}$;\C{ the next fix will be a deletion }\6
\4${}\}{}$\5
\2\&{else}\1\5
${}\|j\K\\{umin}[\|i];{}$\2\6
\4${}\}{}$\5
\2\&{else}\5
${}\{{}$\1\6
\&{if} ${}(\\{sign}[\|i]\E\T{1}){}$\5
${}\{{}$\1\6
\&{if} ${}(\\{umin}[\|i]<\\{maxn}){}$\1\5
${}\|j\K\\{umin}[\|i];{}$\2\6
${}\|i\K{-}\|i{}$;\C{ the next fix will be a deletion }\6
\4${}\}{}$\5
\2\&{else}\1\5
${}\|j\K\\{vmin}[\|i];{}$\2\6
\4${}\}{}$\2\6
${}\\{flag}[\|j]\K\|i;{}$\6
\4${}\}{}$\2\6
\&{return};\6
\4${}\}{}$\2\par
\U19.\fi

\M[670 spiders.w]{21}A block being deleted might be preceded by a simple block
of the
other sign that wants to be inserted. In that case we insert the latter
in place of the former.

\Y\B\4\X21:Delete block $k$ before $l$\X${}\E{}$\6
$\|k\K{-}\|k,\39\|j\K\\{bstart}[\|k],\39\|i\K\\{lsib}[\|j];{}$\6
\&{if} ${}(\\{left}[\|l]\I\|k){}$\1\5
${}\\{printf}(\.{"Oops,\ fixup(\%d,\%d)\ }\)\.{is\ confused!\\n"},\39{-}\|k,\39%
\|l){}$;\C{ can't happen }\2\6
\&{if} ${}(\|i\W\\{sign}[\|i]\I\\{sign}[\|k]){}$\5
${}\{{}$\1\6
\&{if} ${}((\\{sign}[\|i]\E\T{0}\W\\{vmax}[\|i]\E\T{0})\V(\\{sign}[\|i]\E\T{1}%
\W\\{umax}[\|i]\E\T{0})){}$\1\5
\X22:Replace block $k$ by block $i$ and \PB{\&{return}}\X;\2\6
\4${}\}{}$\2\6
${}\\{left}[\|l]\K\\{left}[\|j],\39\\{right}[\\{left}[\|j]]\K\|l;{}$\6
\&{if} (\|i)\5
${}\{{}$\1\6
\&{if} ${}(\\{sign}[\|k]\E\T{0}){}$\5
${}\{{}$\1\6
\&{if} ${}(\\{sign}[\|i]\E\T{1}){}$\1\5
${}\|j\K\\{umin}[\|i];{}$\2\6
\&{else}\1\5
${}\|j\K\\{vmin}[\|i],\39\|i\K{-}\|i{}$;\C{ the next fix will be another
deletion }\2\6
\4${}\}{}$\5
\2\&{else}\5
${}\{{}$\1\6
\&{if} ${}(\\{sign}[\|i]\E\T{0}){}$\1\5
${}\|j\K\\{vmin}[\|i];{}$\2\6
\&{else}\1\5
${}\|j\K\\{umin}[\|i],\39\|i\K{-}\|i{}$;\C{ the next fix will be another
deletion }\2\6
\4${}\}{}$\2\6
${}\\{flag}[\|j]\K\|i;{}$\6
\4${}\}{}$\2\par
\U19.\fi

\M[694 spiders.w]{22}\B\X22:Replace block $k$ by block $i$ and \PB{\&{return}}%
\X${}\E{}$\6
${}\{{}$\1\6
${}\\{left}[\|l]\K\|i,\39\\{right}[\|i]\K\|l;{}$\6
${}\|k\K\\{bstart}[\|i],\39\\{left}[\|k]\K\\{left}[\|j],\39\\{right}[\\{left}[%
\|k]]\K\|k;{}$\6
${}\|i\K\\{lsib}[\|k];{}$\6
\&{if} (\|i)\5
${}\{{}$\1\6
\&{if} ${}(\\{sign}[\|k]\E\T{0}){}$\5
${}\{{}$\1\6
\&{if} ${}(\\{sign}[\|i]\E\T{1}){}$\5
${}\{{}$\1\6
\&{if} ${}(\\{umin}[\|i]<\\{maxn}){}$\1\5
${}\|k\K\\{umin}[\|i];{}$\2\6
${}\|i\K{-}\|i{}$;\C{ the next fix will be another deletion }\6
\4${}\}{}$\5
\2\&{else}\1\5
${}\|k\K\\{vmin}[\|i];{}$\2\6
\4${}\}{}$\5
\2\&{else}\5
${}\{{}$\1\6
\&{if} ${}(\\{sign}[\|i]\E\T{0}){}$\5
${}\{{}$\1\6
\&{if} ${}(\\{vmin}[\|i]<\\{maxn}){}$\1\5
${}\|k\K\\{vmin}[\|i];{}$\2\6
${}\|i\K{-}\|i{}$;\C{ the next fix will be another deletion }\6
\4${}\}{}$\5
\2\&{else}\1\5
${}\|k\K\\{umin}[\|i];{}$\2\6
\4${}\}{}$\2\6
${}\\{flag}[\|k]\K\|i;{}$\6
\4${}\}{}$\2\6
\&{return};\6
\4${}\}{}$\2\par
\U21.\fi

\M[716 spiders.w]{23}How does the active list get there in the first place? We
compute it from the \PB{\\{bit}} table, as follows.

\Y\B\4\X23:Launch the active list\X${}\E{}$\6
\\{setfirst}(\T{0});\C{ compute the initial setting of $\PB{\\{bit}}[1]\ldots%
\PB{\\{bit}}[n]$ }\6
\&{for} ${}(\|l\K\|k\K\T{0};{}$ ${}\|k\Z\|n;{}$ ${}\|k\PP){}$\5
${}\{{}$\1\6
${}\\{focus}[\|k]\K\|k;{}$\6
\&{if} ${}(\\{sign}[\|k]\E\\{bit}[\\{par}[\|k]]){}$\1\5
${}\\{right}[\|l]\K\|k,\39\\{left}[\|k]\K\|l,\39\|l\K\|k;{}$\2\6
\4${}\}{}$\2\6
${}\\{right}[\|l]\K\T{0},\39\\{left}[\T{0}]\K\|l{}$;\C{ link in the rightmost
node of the active list }\par
\U24.\fi

\N[727 spiders.w]{1}{24}Doing it. The time has come to construct the loopless
implementation in
practice, as we have been doing so far in theory.

Of course the printout in each step does involve a loop. This printout
is suppressed if \PB{$\\{verbose}<\T{0}$}.

\Y\B\4\X24:Generate the answers\X${}\E{}$\6
\X23:Launch the active list\X;\6
\&{if} ${}(\\{verbose}>\T{1}){}$\1\5
\X33:Print out the results of initialization\X;\2\6
\&{while} (\T{1})\5
${}\{{}$\1\6
${}\\{count}\PP;{}$\6
\&{if} ${}(\\{verbose}\G\T{0}){}$\1\5
\X30:Print out all the current bits\X;\2\6
\X26:Set \PB{\|k} to the rightmost nonsleeping node of the active list\X;\6
\&{if} (\|k)\5
${}\{{}$\1\6
\&{if} (\\{flag}[\|k])\1\5
${}\\{fixup}(\\{flag}[\|k],\39\|k);{}$\2\6
\&{if} ${}(\\{bit}[\|k]\E\T{0}){}$\1\5
\X28:Move forward, setting \PB{$\\{bit}[\|k]\K\T{1}$}\X\2\6
\&{else}\1\5
\X29:Move backward, setting \PB{$\\{bit}[\|k]\K\T{0}$}\X;\2\6
\4${}\}{}$\5
\2\&{else} \&{if} (\\{been\_there\_and\_done\_that})\1\5
\&{break};\2\6
\&{else}\5
${}\{{}$\1\6
${}\\{printf}(\.{"...\%d\ so\ far;\ now\ w}\)\.{e\ generate\ in\ revers}\)\.{e:%
\\n"},\39\\{count});{}$\6
${}\\{been\_there\_and\_done\_that}\K\T{1};{}$\6
\&{continue};\6
\4${}\}{}$\2\6
\X27:Put \PB{\|k} to sleep\X;\6
\4${}\}{}$\2\6
${}\\{printf}(\.{"Altogether\ \%d/2\ lab}\)\.{elings.\\n"},\39\\{count}){}$;\par
\U1.\fi

\M[753 spiders.w]{25}\B\X4:Global variables\X${}\mathrel+\E{}$\6
\&{int} \\{count};\C{ the number of labelings found so far }\6
\&{int} \\{been\_there\_and\_done\_that};\C{ have we reached all-asleep state
before? }\par
\fi

\M[757 spiders.w]{26}\B\X26:Set \PB{\|k} to the rightmost nonsleeping node of
the active list\X${}\E{}$\6
$\|j\K\\{left}[\T{0}],\39\|k\K\\{focus}[\|j],\39\\{focus}[\|j]\K\|j{}$;\par
\U24.\fi

\M[760 spiders.w]{27}At this point we know that all nodes greater than $k$ are
awake and
that \PB{$\\{flag}[\|k]\K\T{0}$}.

\Y\B\4\X27:Put \PB{\|k} to sleep\X${}\E{}$\6
$\|j\K\\{left}[\|k],\39\\{focus}[\|k]\K\\{focus}[\|j],\39\\{focus}[\|j]\K%
\|j{}$;\par
\U24.\fi

\M[766 spiders.w]{28}\B\X28:Move forward, setting \PB{$\\{bit}[\|k]\K\T{1}$}%
\X${}\E{}$\6
${}\{{}$\1\6
${}\\{bit}[\|k]\K\T{1},\39\|j\K\\{rchild}[\|k];{}$\6
\&{if} (\|j)\5
${}\{{}$\1\6
\&{if} ${}(\\{sign}[\|j]\E\T{0}){}$\5
${}\{{}$\C{ we want to delete \PB{$\|j\K\\{umax}[\|k]$} }\1\6
${}\|l\K\\{vmin}[\|j];{}$\6
\&{if} ${}(\|l<\\{maxn}){}$\1\5
${}\\{fixup}({-}\|j,\39\|l);{}$\2\6
\&{else}\1\5
${}\\{fixup}({-}\|j,\39\\{right}[\|j]){}$;\C{ $j$ ends a simple block }\2\6
\4${}\}{}$\5
\2\&{else}\5
${}\{{}$\C{ we want to insert \PB{$\|j\K\\{vmax}[\|k]$} }\1\6
${}\|l\K\\{umin}[\|j];{}$\6
\&{if} ${}(\|l<\\{maxn}){}$\1\5
${}\\{fixup}(\|j,\39\|l);{}$\2\6
\&{else}\1\5
${}\\{fixup}(\|j,\39\\{right}[\\{umaxscope}[\|k]]){}$;\C{ $j$ ends a simple
block }\2\6
\4${}\}{}$\2\6
\4${}\}{}$\2\6
\4${}\}{}$\2\par
\U24.\fi

\M[782 spiders.w]{29}\B\X29:Move backward, setting \PB{$\\{bit}[\|k]\K\T{0}$}%
\X${}\E{}$\6
${}\{{}$\1\6
${}\\{bit}[\|k]\K\T{0},\39\|j\K\\{rchild}[\|k];{}$\6
\&{if} (\|j)\5
${}\{{}$\1\6
\&{if} ${}(\\{sign}[\|j]\E\T{1}){}$\5
${}\{{}$\C{ we want to delete \PB{$\|j\K\\{vmax}[\|k]$} }\1\6
${}\|l\K\\{umin}[\|j];{}$\6
\&{if} ${}(\|l<\\{maxn}){}$\1\5
${}\\{fixup}({-}\|j,\39\|l);{}$\2\6
\&{else}\1\5
${}\\{fixup}({-}\|j,\39\\{right}[\|j]){}$;\C{ $j$ ends a simple block }\2\6
\4${}\}{}$\5
\2\&{else}\5
${}\{{}$\C{ we want to insert \PB{$\|j\K\\{umax}[\|k]$} }\1\6
${}\|l\K\\{vmin}[\|j];{}$\6
\&{if} ${}(\|l<\\{maxn}){}$\1\5
${}\\{fixup}(\|j,\39\|l);{}$\2\6
\&{else}\1\5
${}\\{fixup}(\|j,\39\\{right}[\\{vmaxscope}[\|k]]){}$;\C{ $j$ ends a simple
block }\2\6
\4${}\}{}$\2\6
\4${}\}{}$\2\6
\4${}\}{}$\2\par
\U24.\fi

\M[798 spiders.w]{30}\B\X30:Print out all the current bits\X${}\E{}$\6
${}\{{}$\1\6
\&{for} ${}(\|k\K\T{1};{}$ ${}\|k\Z\|n;{}$ ${}\|k\PP){}$\1\5
${}\\{putchar}(\.{'0'}+\\{bit}[\|k]);{}$\2\6
\&{if} ${}(\\{verbose}>\T{0}){}$\1\5
\X31:Print the active list in symbolic form\X;\2\6
\\{putchar}(\.{'\\n'});\6
\4${}\}{}$\2\par
\U24.\fi

\M[805 spiders.w]{31}Here I recompute what the active list should be, and
compare it
to the current links. Discrepancies are noted only if no flagged nodes
follow.

Sleeping nodes are enclosed in parentheses;
an exclamation point is printed before a node that is flagged.

\Y\B\4\X31:Print the active list in symbolic form\X${}\E{}$\6
${}\{{}$\1\6
\&{for} ${}(\|k\K\\{left}[\T{0}];{}$  ; ${}\|k\MM){}$\5
${}\{{}$\1\6
\&{for} ${}(\|j\K\|k,\39\|k\K\\{focus}[\|k];{}$ ${}\|j>\|k;{}$ ${}\|j\MM){}$\5
${}\{{}$\1\6
${}\\{asleep}[\|j]\K\T{1};{}$\6
\&{if} (\\{flag}[\|j])\1\5
${}\\{printf}(\.{"\\nOops,\ flag[\%d]\ is}\)\.{\ wrong!\\n"},\39\|j);{}$\2\6
\4${}\}{}$\2\6
\&{if} ${}(\|k\E\T{0}){}$\1\5
\&{break};\2\6
${}\\{asleep}[\|k]\K\T{0};{}$\6
\4${}\}{}$\2\6
\&{for} ${}(\|k\K\T{1},\39\|j\K\T{0};{}$ ${}\|k\Z\\{left}[\T{0}];{}$ ${}\|k%
\PP){}$\1\6
\&{if} ${}(\\{sign}[\|k]\E\\{bit}[\\{par}[\|k]]){}$\5
${}\{{}$\1\6
\&{if} (\\{asleep}[\|k])\1\5
${}\\{printf}(\.{"\ (\%d)"},\39\|k);{}$\2\6
\&{else} \&{if} (\\{flag}[\|k])\1\5
${}\\{printf}(\.{"\ !\%d"},\39\|k);{}$\2\6
\&{else}\1\5
${}\\{printf}(\.{"\ \%d"},\39\|k);{}$\2\6
\&{if} ${}((\|k\I\\{right}[\|j]\V\\{left}[\|k]\I\|j)\W\|k>\|l){}$\1\5
\\{printf}(\.{"[oops]"});\2\6
${}\|j\K\|k;{}$\6
\4${}\}{}$\2\2\6
\4${}\}{}$\2\par
\U30.\fi

\M[832 spiders.w]{32}\B\X4:Global variables\X${}\mathrel+\E{}$\6
\&{int} \\{asleep}[\\{maxn}];\C{ sleeping (or inactive) nodes }\par
\fi

\M[835 spiders.w]{33}Finally, we print even more stuff when the user calls for
an
exceptional level of absolute verbosity.

\Y\B\4\X33:Print out the results of initialization\X${}\E{}$\6
${}\{{}$\1\6
\&{for} ${}(\|k\K\T{0};{}$ ${}\|k\Z\|n;{}$ ${}\|k\PP){}$\5
${}\{{}$\1\6
${}\\{printf}(\.{"\%d(\%c):\ scope=\%d,\ p}\)\.{ar=\%d,\ rchild=\%d,\ ls}\)%
\.{ib=\%d,"},\39\|k,\39\\{sign}[\|k]\?\.{'-'}:\.{'+'},\39\\{scope}[\|k],\39%
\\{par}[\|k],\39\\{rchild}[\|k],\39\\{lsib}[\|k]);{}$\6
${}\\{printf}(\.{"\ ppro=\%d,\ npro=\%d,\ }\)\.{prev=\%d,\ bstart=\%d\\n}\)%
\.{"},\39\\{ppro}[\|k],\39\\{npro}[\|k],\39\\{prev}[\|k],\39\\{bstart}[%
\|k]);{}$\6
${}\\{printf}(\.{"\ umin=\%d,\ ueven=\%d,}\)\.{\ umax=\%d,\ umaxbit=\%d}\)\.{,\
umaxscope=\%d\\n"},\39\\{umin}[\|k],\39\\{ueven}[\|k],\39\\{umax}[\|k],\39%
\\{umaxbit}[\|k],\39\\{umaxscope}[\|k]);{}$\6
${}\\{printf}(\.{"\ vmin=\%d,\ veven=\%d,}\)\.{\ vmax=\%d,\ vmaxbit=\%d}\)\.{,\
vmaxscope=\%d\\n"},\39\\{vmin}[\|k],\39\\{veven}[\|k],\39\\{vmax}[\|k],\39%
\\{vmaxbit}[\|k],\39\\{vmaxscope}[\|k]);{}$\6
\4${}\}{}$\2\6
\4${}\}{}$\2\par
\U24.\fi

\N[852 spiders.w]{1}{34}Index.
\fi

\inx
\fin
\con
