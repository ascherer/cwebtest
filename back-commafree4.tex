\input cwebmac
\srcloctrue

\datethis


\N[6 back-commafree4.w]{1}{1}Intro. This is a somewhat nontrivial example of
backtrack programming,
intended to exhibit basic ideas as well as perhaps to solve an open problem.

The problem is to find the largest possible comma-free code of length~4
on an $m$-letter alphabet, where $m=5$. It was solved for $m\le4$
in the 60s, so I have some hope. On the other hand, the running time
might also be worse than exponential in~$m$, so I must keep my fingers crossed.

The theory was set out by Golomb, Gordon, and Welch in {\sl Canadian
Journal of Mathematics\/ \bf10} (1958), 202--209. There are $m^4$
four-letter words, and $m^2$ of them are ``cyclic'' (like \.{dodo}).
The others are divided into $(m^4-m^2)/4$ equivalence classes,
where a word is considered equivalent to its cyclic shifts. Our
problem is to choose as many codewords as possible such that,
if $x_1x_2x_3x_4$ and $x_5x_6x_7x_8$ are codewords,
then $x_2x_3x_4x_5$ and $x_3x_4x_5x_5$ and $x_4x_5x_6x_7$ are {\it not\/}
codewords. In particular, when $x_1x_2x_3x_4=x_5x_6x_7x_8$ this condition
tells us that we can choose at most one codeword from each equivalence class.

The number of equivalence classes is 3, 18, 60, and 150 when $m=2$, 3,
4, and~5. This maximum code size is achievable for $m=2$ and $m=3$;
but a computer program by Lee Laxdal in 1963 showed that the largest sets
for $m=4$ have 57 codewords, not 60.

I think there's an interesting way to reproduce that result (namely
by the methods below), and perhaps to go further.

Note (written after completing this program): I haven't time to follow up
on an afterthought that might be able to help crack the case \PB{$\|m\K\T{5}$}.
Please see the detailed remarks near the beginning of
\PB{$\X40:Find \PB{\|x} that maximizes the ``kill'' potential~\PB{\|s}\X$}.

\Y\B\4\D$\\{maxm}$ \5
\T{7}\C{ unfortunately however it turns out that \PB{$\|m\K\T{5}$} is already
too slow }\par
\B\4\D$\\{maxmmmm}$ \5
$(\\{maxm}*\\{maxm}*\\{maxm}*\\{maxm}{}$)\par
\B\4\D$\\{maxclasses}$ \5
$((\\{maxmmmm}-\\{maxm}*\\{maxm})/\T{4}{}$)\par
\B\4\D$\\{ustacksize}$ \5
\T{1000000}\par
\B\4\D$\|o$ \5
$\\{mems}\PP{}$\par
\B\4\D$\\{oo}$ \5
$\\{mems}\MRL{+{\K}}{}$\T{2}\par
\Y\B\8\#\&{include} \.{<stdio.h>}\6
\8\#\&{include} \.{<stdlib.h>}\6
\&{int} \|m${},{}$ \\{goal};\C{ command-line parameters }\6
\&{typedef} \&{unsigned} \&{long} \&{long} \&{ull};\7
\X28:Type definitions\X;\7
\&{ull} \\{mems};\C{ 8-byte memory accesses }\6
\&{ull} \\{nodes};\C{ size of search tree }\6
\&{ull} \\{thresh}${}\K\T{10000000000}{}$;\C{ reporting time }\6
\&{ull} ${}\\{profile}[\\{maxclasses}+\T{3}]{}$;\C{ width of search tree levels
}\6
\&{int} \\{count};\C{ number of solutions found }\6
\&{int} \\{maxstack}${},{}$ \\{maxpoison};\C{ how big have these lists gotten?
}\6
\&{int} \\{vbose};\C{ level of verbosity }\7
\X6:Global variables\X;\6
\X4:Subroutines\X;\7
\\{main}(\&{int} \\{argc}${},\39{}$\&{char} ${}{*}\\{argv}[\,]){}$\1\1\2\2\6
${}\{{}$\1\6
\&{register} \&{int} \|a${},{}$ \|b${},{}$ \|c${},{}$ \|d${},{}$ \|i${},{}$ %
\|j${},{}$ \|k${},{}$ \|l${},{}$ \|p${},{}$ \|q${},{}$ \|r${},{}$ \|s${},{}$ %
\\{pp}${},{}$ \|x${},{}$ \|y${},{}$ \\{yy}${},{}$ \|z${},{}$ \\{zz}${},{}$ %
\\{alpha}${},{}$ \\{ploc}${},{}$ \\{cls}${},{}$ \\{slack};\7
\X3:Process the command line\X;\6
\X10:Initialize the tables\X;\6
\X32:Backtrack through all solutions\X;\6
${}\\{fprintf}(\\{stderr},\39\.{"Altogether\ \%d\ solut}\)\.{ions\ (\%lld\
mems,\ \%ll}\)\.{d\ nodes)\\n"},\39\\{count},\39\\{mems},\39\\{nodes});{}$\6
${}\\{fprintf}(\\{stderr},\39\.{"(maxstack=\%d,maxpoi}\)\.{son=\%d)\\n"},\39%
\\{maxstack},\39(\\{maxpoison}-\\{poison})/\T{2});{}$\6
\&{if} (\\{vbose})\1\5
\X2:Print the profile\X;\2\6
\4${}\}{}$\2\par
\fi

\M[73 back-commafree4.w]{2}\B\X2:Print the profile\X${}\E{}$\6
${}\{{}$\1\6
${}\\{fprintf}(\\{stderr},\39\.{"Profile:\ \ \ \ \ \ \ \ \ \ 1}\)\.{\\n"});{}$\6
\&{for} ${}(\|k\K\T{2};{}$ ${}\|k\Z\\{maxl}+\T{1};{}$ ${}\|k\PP){}$\1\5
${}\\{fprintf}(\\{stderr},\39\.{"\%19lld\\n"},\39\\{profile}[\|k]);{}$\2\6
\4${}\}{}$\2\par
\U1.\fi

\M[79 back-commafree4.w]{3}\B\X3:Process the command line\X${}\E{}$\6
\&{if} ${}(\\{argc}<\T{3}\V\\{sscanf}(\\{argv}[\T{1}],\39\.{"\%d"},\39{\AND}%
\|m)\I\T{1}\V\\{sscanf}(\\{argv}[\T{2}],\39\.{"\%d"},\39{\AND}\\{goal})\I%
\T{1}){}$\5
${}\{{}$\1\6
${}\\{fprintf}(\\{stderr},\39\.{"Usage:\ \%s\ m\ goal\ [p}\)\.{rofile]\
[verbose]\ [e}\)\.{xtraverbose]\\n"},\39\\{argv}[\T{0}]);{}$\6
${}\\{exit}({-}\T{1});{}$\6
\4${}\}{}$\2\6
\&{if} ${}(\|m<\T{2}\V\|m>\\{maxm}){}$\5
${}\{{}$\1\6
${}\\{fprintf}(\\{stderr},\39\.{"m\ should\ be\ between}\)\.{\ 2\ and\ \%d,\
not\ \%d!\\n}\)\.{"},\39\\{maxm},\39\|m);{}$\6
${}\\{exit}({-}\T{2});{}$\6
\4${}\}{}$\2\6
${}\\{vbose}\K\\{argc}-\T{3}{}$;\par
\U1.\fi

\M[91 back-commafree4.w]{4}\B\X4:Subroutines\X${}\E{}$\6
\&{void} \\{confusion}(\&{char} ${}{*}\|s){}$\1\1\2\2\6
${}\{{}$\1\6
${}\\{fprintf}(\\{stderr},\39\.{"I'm\ confused\ (\%s)!\\}\)\.{n"},\39\|s);{}$\6
${}\\{exit}({-}\T{666});{}$\6
\4${}\}{}$\2\par
\As5, 18, 19, 20, 21, 31\ETs42.
\U1.\fi

\N[97 back-commafree4.w]{1}{5}Infrastructure. This program keeps its state
information in an
array \PB{\\{mem}} of 16-bit quantities. Everything in \PB{\\{mem}} also has a
16-bit address. Thus I can undo the changes made at one level
if I maintain a stack that holds 32-bit items, $(\rm address,oldvalue)$.

In this implementation I use a ``stamp'' to see whether the
former \PB{\\{mem}} value has already been saved. This makes the
cost $2r+5$~mems when we store $r>0$ times into a location,
rather than $5r$~mems; so it saves time if $r$ is usually greater than~1.

(With a change file I can try removing the stamps, to see what
happens in practice.)

\Y\B\4\X4:Subroutines\X${}\mathrel+\E{}$\6
\&{inline} \&{void} \\{store}(\&{register} \&{short} \|a${},\39{}$\&{register} %
\&{short} \|x)\1\1\2\2\6
${}\{{}$\1\6
\&{if} ${}(\|o,\39\\{stamp}[\|a]\I\\{curstamp}){}$\5
${}\{{}$\1\6
${}\|o,\39\\{stamp}[\|a]\K\\{curstamp};{}$\6
${}\\{oo},\39\\{undo}[\\{uptr}\PP]\K(\|a\LL\T{16})+\\{mem}[\|a];{}$\6
\&{if} ${}(\\{uptr}>\\{maxstack}){}$\5
${}\{{}$\1\6
${}\\{maxstack}\K\\{uptr};{}$\6
\&{if} ${}(\\{maxstack}\G\\{ustacksize}){}$\5
${}\{{}$\1\6
${}\\{fprintf}(\\{stderr},\39\.{"Stack\ overflow\ (ust}\)\.{acksize=\%d)!\\n"},%
\39\\{ustacksize});{}$\6
${}\\{exit}({-}\T{9});{}$\6
\4${}\}{}$\2\6
\4${}\}{}$\2\6
\4${}\}{}$\2\6
${}\|o,\39\\{mem}[\|a]\K\|x;{}$\6
\4${}\}{}$\2\7
\&{inline} \&{void} \\{unstore}(\&{register} \&{int} \\{ptr})\1\1\2\2\6
${}\{{}$\1\6
\&{register} \&{unsigned} \&{int} \|x;\7
\&{while} ${}(\\{uptr}>\\{ptr}){}$\5
${}\{{}$\1\6
${}\|o,\39\|x\K\\{undo}[\MM\\{uptr}];{}$\6
${}\|o,\39\\{mem}[\|x\GG\T{16}]\K\|x\AND\T{\^ffff}{}$;\C{ everything snaps back
to its former value }\6
\4${}\}{}$\2\6
\4${}\}{}$\2\par
\fi

\M[134 back-commafree4.w]{6}\B\X6:Global variables\X${}\E{}$\6
\&{unsigned} ${}\\{mem}[\T{1}\LL\T{16}];{}$\6
\&{unsigned} \&{int} \\{undo}[\\{ustacksize}];\6
\&{int} \\{uptr};\C{ this many entries are currently in \PB{\\{undo}} }\6
\&{unsigned} \&{int} ${}\\{stamp}[\T{1}\LL\T{16}];{}$\6
\&{unsigned} \&{int} \\{curstamp};\par
\As9, 14, 24\ETs29.
\U1.\fi

\M[141 back-commafree4.w]{7}Here's what we do when we need a new current stamp.

Incidentally, my first draft of this step had a noteworthy bug: I knew that
\PB{\\{curstamp}} won't actually overflow in this program; still, this program
is
intended as a model for more complicated ones in which overflow really
might happen. So I somewhat grudgingly added code to cover the case of
overflow. But I noticed that I could save one (1) line, aha, by
leaving \PB{$\\{curstamp}\K\T{0}$} after overflow, and setting all the
remaining
stamps to \PB{\T{\^ffffffff}}. Reason: The purpose is to have \PB{$\\{stamp}[%
\|x]\I\\{curstamp}$} for
all~\PB{\|x} after bumping. Fallacy: Later on, when \PB{\\{curstamp}} rises for
the second time to
\PB{\T{\^ffffffff}}, that invariant relation might no longer hold. A better
invariant
is to say that \PB{$\\{stamp}[\|x]<\\{curstamp}$} for all~\PB{\|x} after
bumping.

\Y\B\4\X7:Bump \PB{\\{curstamp}}\X${}\E{}$\6
\&{if} ${}(\PP\\{curstamp}\E\T{0}){}$\5
${}\{{}$\C{ whoops, \PB{\\{curstamp}} has overflowed! cancel all stamps }\1\6
\&{for} ${}(\|k\K\T{0};{}$ ${}\|k<(\T{1}\LL\T{16});{}$ ${}\|k\PP){}$\1\5
${}\|o,\39\\{stamp}[\|k]\K\T{0};{}$\2\6
${}\\{curstamp}\K\T{1};{}$\6
\4${}\}{}$\2\par
\U32.\fi

\M[160 back-commafree4.w]{8}This program also makes use of numerous sequential
lists, kept in~\PB{\\{mem}},
for which the basic operations are to append a new item, or to remove
an item given its name.

Suppose \PB{\|p} points to the first empty spot at the current end of a list.
To append a new item~$x$, knowing that it isn't already present,
the operation is \PB{${*}\|p\K\|x$} followed by \PB{$\|i[\|x]\K\|p\PP$}, where %
\PB{\|i} is an
inverse list. To delete item $x$, knowing that it is present,
we set \PB{$\|q\K\|i[\|x]$} and test if \PB{$\MM\|p\E\|q$}. If so, we're done.
If not,
we set \PB{$\|y\K{*}\|p$} and \PB{${*}\|q\K\|y$} and \PB{$\|i[\|y]\K\|q$}.

\fi

\N[171 back-commafree4.w]{1}{9}Memory layout. Let $\alpha$ be a four-letter
word $abcd$. We will
think of $\alpha$ as the radix-$m$ integer $(abcd)_m$,
although it will also be represented in memory as the hexadecimal
integer $(abcd)_{16}$ since that's convenient for programming
and debugging.

\Y\B\4\X6:Global variables\X${}\mathrel+\E{}$\6
\&{int} ${}\\{hexconv}[\T{1}\LL\T{16}]{}$;\C{ table to convert from hex to
radix $m$ }\6
\&{int} \\{mconv}[\\{maxmmmm}];\C{ and vice versa (used only in diagnostics) }%
\par
\fi

\M[181 back-commafree4.w]{10}\B\X10:Initialize the tables\X${}\E{}$\6
\&{for} ${}(\|a\K\T{0};{}$ ${}\|a<\|m;{}$ ${}\|a\PP){}$\1\6
\&{for} ${}(\|b\K\T{0};{}$ ${}\|b<\|m;{}$ ${}\|b\PP){}$\1\6
\&{for} ${}(\|c\K\T{0};{}$ ${}\|c<\|m;{}$ ${}\|c\PP){}$\1\6
\&{for} ${}(\|d\K\T{0};{}$ ${}\|d<\|m;{}$ ${}\|d\PP){}$\1\5
${}\|o,\39\\{hexconv}[(\|a\LL\T{12})+(\|b\LL\T{8})+(\|c\LL\T{4})+\|d]\K((\|a*%
\|m+\|b)*\|m+\|c)*\|m+\|d,\39\\{mconv}[((\|a*\|m+\|b)*\|m+\|c)*\|m+\|d]\K(\|a%
\LL\T{12})+(\|b\LL\T{8})+(\|c\LL\T{4})+\|d{}$;\2\2\2\2\par
\As17, 25\ETs30.
\U1.\fi

\M[187 back-commafree4.w]{11}During the backtrack process, every word $\alpha$
will be in
one of three states, which I'll call green, red, and blue.
(i)~Green words have been chosen (tentatively) to be in the code.
(ii)~Red words have been ruled out, either because they're incompatible
with the words that are currently green or because all solutions
in which they're green have already been considered. (iii)~The remaining
words, feeling blue, are still in limbo.
The current state of word $\alpha$ is in \PB{\\{mem}}$[\alpha]$.

\Y\B\4\D$\\{green}$ \5
\T{2}\par
\B\4\D$\\{red}$ \5
\T{0}\par
\B\4\D$\\{blue}$ \5
\T{1}\par
\fi

\M[200 back-commafree4.w]{12}Let $p_1(\alpha)=(a000)_m$, $p_2(%
\alpha)=(ab00)_m$, $p_3(\alpha)=(abc0)_m$
be the three proper prefixes of~$\alpha=abcd$, and let
$s_1(\alpha)=(d000)_m$, $s_2(\alpha)=(cd00)_m$, $s_3(\alpha)=(bcd0)_m$
be the (shifted) proper suffixes.

Every blue word $\alpha=abcd$ appears in six lists, corresponding
to those suffixes.
For example, $p_1(\alpha)$ corresponds to the list of all blue words that begin
with~$a$, and $s_2(\alpha)$ corresponds to the list of all blue words that
end with $cd$. List $p_1(\alpha)$ begins at \PB{\\{mem}} location
$\PB{\\{p1offset}}+p_1(\alpha)$; its current pointer is at \PB{\\{mem}}
location
$\PB{\\{p1offset}}+m^4+p_1(\alpha)$; and $\alpha$'s location in the list is
maintained in \PB{\\{mem}} location $\PB{\\{p1offset}}-m^4+\alpha$. Similar
conventions apply to the other five lists, whose offsets are
\PB{\\{p2offset}}, \dots,~\PB{\\{s3offset}}.

\Y\B\4\D$\\{p1}(\|x)$ \5
$(\|o,\39\\{hexconv}[(\|x)\AND\T{\^f000}]{}$)\par
\B\4\D$\\{p2}(\|x)$ \5
$(\|o,\39\\{hexconv}[(\|x)\AND\T{\^ff00}]{}$)\par
\B\4\D$\\{p3}(\|x)$ \5
$(\|o,\39\\{hexconv}[(\|x)\AND\T{\^fff0}]{}$)\par
\B\4\D$\\{s1}(\|x)$ \5
$(\|o,\39\\{hexconv}[((\|x)\AND\T{\^000f})\LL\T{12}]{}$)\par
\B\4\D$\\{s2}(\|x)$ \5
$(\|o,\39\\{hexconv}[((\|x)\AND\T{\^00ff})\LL\T{8}]{}$)\par
\B\4\D$\\{s3}(\|x)$ \5
$(\|o,\39\\{hexconv}[((\|x)\AND\T{\^0fff})\LL\T{4}]{}$)\par
\Y\B\4\X12:Create empty prefix and suffix lists\X${}\E{}$\6
\&{for} ${}(\|a\K\T{0};{}$ ${}\|a<\|m;{}$ ${}\|a\PP){}$\5
${}\{{}$\1\6
${}\|o,\39\\{mem}[\\{p1offset}+\\{mmmm}+\|a*\|m*\|m*\|m]\K\\{p1offset}+\|a*\|m*%
\|m*\|m;{}$\6
${}\|o,\39\\{mem}[\\{s1offset}+\\{mmmm}+\|a*\|m*\|m*\|m]\K\\{s1offset}+\|a*\|m*%
\|m*\|m;{}$\6
\&{for} ${}(\|b\K\T{0};{}$ ${}\|b<\|m;{}$ ${}\|b\PP){}$\5
${}\{{}$\1\6
${}\|o,\39\\{mem}[\\{p2offset}+\\{mmmm}+(\|a*\|m+\|b)*\|m*\|m]\K\\{p2offset}+(%
\|a*\|m+\|b)*\|m*\|m;{}$\6
${}\|o,\39\\{mem}[\\{s2offset}+\\{mmmm}+(\|a*\|m+\|b)*\|m*\|m]\K\\{s2offset}+(%
\|a*\|m+\|b)*\|m*\|m;{}$\6
\&{for} ${}(\|c\K\T{0};{}$ ${}\|c<\|m;{}$ ${}\|c\PP){}$\5
${}\{{}$\1\6
${}\|o,\39\\{mem}[\\{p3offset}+\\{mmmm}+((\|a*\|m+\|b)*\|m+\|c)*\|m]\K%
\\{p3offset}+((\|a*\|m+\|b)*\|m+\|c)*\|m;{}$\6
${}\|o,\39\\{mem}[\\{s3offset}+\\{mmmm}+((\|a*\|m+\|b)*\|m+\|c)*\|m]\K%
\\{s3offset}+((\|a*\|m+\|b)*\|m+\|c)*\|m;{}$\6
\4${}\}{}$\2\6
\4${}\}{}$\2\6
\4${}\}{}$\2\par
\U17.\fi

\M[237 back-commafree4.w]{13}The following instructions serve to populate the
lists at the beginning.
Here \PB{\\{alpha}} is the $m$-ary number, and \PB{\|x} is its hexadecimal
representation;
\PB{\\{del}} is a truncated version of \PB{\\{alpha}}, identifying where the
list
begins within its offset region.

\Y\B\4\D$\\{insert}(\|x,\\{alpha},\\{del},\\{off})$ \6
${}\{{}$\1\6
${}\\{ploc}\K(\\{del})+(\\{off}){}$;\C{ the list begins here }\6
${}\|o,\39\|p\K\\{mem}[\\{ploc}+\\{mmmm}]{}$;\C{ \PB{\|p} is its current
pointer }\6
${}\|o,\39\\{mem}[\|p]\K\|x{}$;\C{ we put the hex form into the list }\6
${}\|o,\39\\{mem}[(\\{off})-\\{mmmm}+\\{alpha}]\K\|p{}$;\C{ and remember it in
the inverse list }\6
${}\|o,\39\\{mem}[\\{ploc}+\\{mmmm}]\K\|p+\T{1}{}$;\C{ and update the list size
}\6
\4${}\}{}$\2\par
\Y\B\4\X13:Put \PB{\\{alpha}} into all of its prefix and suffix lists\X${}\E{}$%
\6
$\\{insert}(\|x,\39\\{alpha},\39\\{p1}(\|x),\39\\{p1offset});{}$\6
${}\\{insert}(\|x,\39\\{alpha},\39\\{p2}(\|x),\39\\{p2offset});{}$\6
${}\\{insert}(\|x,\39\\{alpha},\39\\{p3}(\|x),\39\\{p3offset});{}$\6
${}\\{insert}(\|x,\39\\{alpha},\39\\{s1}(\|x),\39\\{s1offset});{}$\6
${}\\{insert}(\|x,\39\\{alpha},\39\\{s2}(\|x),\39\\{s2offset});{}$\6
${}\\{insert}(\|x,\39\\{alpha},\39\\{s3}(\|x),\39\\{s3offset}){}$;\par
\U16.\fi

\M[258 back-commafree4.w]{14}The cyclic equivalence classes of words are given
identification numbers from
0 to $(m^4-m^2)/4-1$. The number of $\alpha$'s class is precomputed in
the \PB{\\{cl}} table.

Every blue word also appears in the list of all currently blue words
in its class. That list, for class $k$, begins at \PB{\\{mem}} location
$\PB{\\{cloffset}}+4k$. Its pointer is in \PB{\\{mem}} location
$\PB{\\{cloffset}}+m^4+4k$, and $\alpha$'s position in that list
is maintained in \PB{\\{mem}} location $\PB{\\{cloffset}}-m^4+\alpha$.

\Y\B\4\X6:Global variables\X${}\mathrel+\E{}$\6
\&{int} \\{mmmm};\C{ $m^4$ }\6
\&{int} \\{maxl};\C{ $(m^4-m^2)/4$ }\6
\&{int} \\{p1offset}${},{}$ \\{p2offset}${},{}$ \\{p3offset}${},{}$ %
\\{s1offset}${},{}$ \\{s2offset}${},{}$ \\{s3offset}${},{}$ \\{cloffset};\6
\&{char} \\{aa}[\T{5}];\C{ the working array for Algorithm 7.2.1.1F }\6
\&{int} \\{cl}[\\{maxmmmm}];\C{ class identification numbers }\6
\&{int} \\{clrep}[\\{maxclasses}];\C{ class representatives (as hexadecimal
numbers) }\6
\&{int} \\{classes};\C{ the number of classes found so far }\par
\fi

\M[277 back-commafree4.w]{15}Here I use Algorithm 7.2.1.1F to find the prime
binary strings.

\Y\B\4\X15:Compute the classes and initialize their lists\X${}\E{}$\6
\4\\{f1}:\5
${}\|o,\39\\{aa}[\T{0}]\K{-}\T{1},\39\|j\K\T{1},\39\\{classes}\K\T{0};{}$\6
\4\\{f2}:\5
\&{if} ${}(\|j\E\T{4}){}$\1\5
\X16:Visit the prime string $a_1\ldots a_4$\X;\2\6
\4\\{f3}:\5
\&{for} ${}(\|j\K\T{4};{}$ \|o${},\39\\{aa}[\|j]\E\|m-\T{1};{}$ ${}\|j\MM){}$\1%
\5
;\2\6
\4\\{f4}:\5
\&{if} (\|j)\5
${}\{{}$\1\6
${}\|o,\39\\{aa}[\|j]\PP;{}$\6
\4\\{f5}:\5
\&{for} ${}(\|k\K\|j+\T{1};{}$ ${}\|k\Z\T{4};{}$ ${}\|k\PP){}$\1\5
${}\\{oo},\39\\{aa}[\|k]\K\\{aa}[\|k-\|j];{}$\2\6
\&{goto} \\{f2};\6
\4${}\}{}$\2\6
\&{if} ${}(\\{classes}\I\\{maxl}){}$\1\5
\\{confusion}(\.{"classes"});\2\par
\U17.\fi

\M[290 back-commafree4.w]{16}\B\X16:Visit the prime string $a_1\ldots a_4$\X${}%
\E{}$\6
${}\{{}$\1\6
\&{for} ${}(\|i\K\T{0};{}$ ${}\|i<\T{4};{}$ ${}\|i\PP){}$\5
${}\{{}$\1\6
\&{for} ${}(\|x\K\T{0},\39\|k\K\T{0};{}$ ${}\|k<\T{4};{}$ ${}\|k\PP){}$\1\5
${}\|x\K(\|x\LL\T{4})+\\{aa}[\T{1}+((\|i+\|k)\MOD\T{4})];{}$\2\6
${}\|o,\39\\{alpha}\K\\{hexconv}[\|x];{}$\6
\&{if} ${}(\|i\E\T{0}){}$\1\5
${}\\{clrep}[\\{classes}]\K\|x;{}$\2\6
\&{if} ${}(\|x\I\T{\^0100}\W\|x\I\T{\^1000}){}$\5
${}\{{}$\C{ see below }\1\6
${}\|o,\39\\{mem}[\\{alpha}]\K\\{blue};{}$\6
${}\|o,\39\\{mem}[\|i+\T{4}*\\{classes}+\\{cloffset}]\K\|x;{}$\6
${}\|o,\39\\{mem}[\\{cloffset}-\\{mmmm}+\\{alpha}]\K\|i+\T{4}*\\{classes}+%
\\{cloffset};{}$\6
\X13:Put \PB{\\{alpha}} into all of its prefix and suffix lists\X;\6
\4${}\}{}$\2\6
${}\|o,\39\\{cl}[\\{alpha}]\K\\{classes};{}$\6
\4${}\}{}$\2\6
${}\|o,\39\\{mem}[\\{cloffset}+\\{mmmm}+\T{4}*\\{classes}]\K\\{cloffset}+\T{4}*%
\\{classes}+(\\{classes}\?\T{4}:\T{2});{}$\6
${}\\{classes}\PP;{}$\6
\4${}\}{}$\2\par
\U15.\fi

\M[309 back-commafree4.w]{17}Notice that we've already allocated $22m^4$ cells
of \PB{\\{mem}}.
Fortunately that's only 52822 when $m=7$, comfortably less than 65536.

\Y\B\4\X10:Initialize the tables\X${}\mathrel+\E{}$\6
$\\{mmmm}\K\|m*\|m*\|m*\|m;{}$\6
\&{for} ${}(\|k\K\T{0};{}$ ${}\|k<\\{mmmm};{}$ ${}\|k\PP){}$\1\5
${}\|o,\39\\{mem}[\|k]\K\\{red};{}$\2\6
${}\\{maxl}\K(\\{mmmm}-\|m*\|m)\GG\T{2};{}$\6
\&{if} ${}(\\{goal}<\\{maxl}-\|m*(\|m-\T{1})\V\\{goal}>\\{maxl}){}$\5
${}\{{}$\1\6
${}\\{fprintf}(\\{stderr},\39\.{"The\ goal\ should\ be\ }\)\.{between\ \%d\ and%
\ \%d,\ n}\)\.{ot\ \%d!\\n"},\39\\{maxl}-\|m*(\|m-\T{1}),\39\\{maxl},\39%
\\{goal});{}$\6
${}\\{exit}({-}\T{3});{}$\6
\4${}\}{}$\2\6
${}\\{p1offset}\K\T{2}*\\{mmmm},\39\\{p2offset}\K\T{5}*\\{mmmm},\39\\{p3offset}%
\K\T{8}*\\{mmmm};{}$\6
${}\\{s1offset}\K\T{11}*\\{mmmm},\39\\{s2offset}\K\T{14}*\\{mmmm},\39%
\\{s3offset}\K\T{17}*\\{mmmm};{}$\6
${}\\{cloffset}\K\T{20}*\\{mmmm};{}$\6
\X12:Create empty prefix and suffix lists\X;\6
\X15:Compute the classes and initialize their lists\X;\par
\fi

\M[327 back-commafree4.w]{18}Diagnostic information is helped by a symbolic
indication of
each list name.

\Y\B\4\X4:Subroutines\X${}\mathrel+\E{}$\6
\&{void} \\{decode\_list}(\&{int} \|p)\1\1\2\2\6
${}\{{}$\1\6
\&{register} \&{int} \|x${},{}$ \\{alpha};\7
${}\\{alpha}\K\|p\MOD\\{mmmm},\39\|x\K\\{mconv}[\\{alpha}];{}$\6
\&{switch} ${}(\|p/\\{mmmm}){}$\5
${}\{{}$\1\6
\4\&{case} \T{2}:\5
${}\\{fprintf}(\\{stderr},\39\.{"\%x..."},\39\|x\GG\T{12}){}$;\5
\&{break};\6
\4\&{case} \T{5}:\5
${}\\{fprintf}(\\{stderr},\39\.{"\%02x.."},\39\|x\GG\T{8}){}$;\5
\&{break};\6
\4\&{case} \T{8}:\5
${}\\{fprintf}(\\{stderr},\39\.{"\%03x."},\39\|x\GG\T{4}){}$;\5
\&{break};\6
\4\&{case} \T{11}:\5
${}\\{fprintf}(\\{stderr},\39\.{"...\%x"},\39\|x\GG\T{12}){}$;\5
\&{break};\6
\4\&{case} \T{14}:\5
${}\\{fprintf}(\\{stderr},\39\.{"..\%02x"},\39\|x\GG\T{8}){}$;\5
\&{break};\6
\4\&{case} \T{17}:\5
${}\\{fprintf}(\\{stderr},\39\.{".\%03x"},\39\|x\GG\T{4}){}$;\5
\&{break};\6
\4\&{case} \T{20}:\5
${}\\{fprintf}(\\{stderr},\39\.{"[\%04x]"},\39\\{clrep}[\\{alpha}/\T{4}]){}$;\5
\&{break};\6
\4\&{default}:\5
${}\\{fprintf}(\\{stderr},\39\.{"???"});{}$\6
\4${}\}{}$\2\6
\4${}\}{}$\2\par
\fi

\M[346 back-commafree4.w]{19}We will actually maintain all of these lists only
when none of
the list elements is green. As soon as a green element appears in a list
that begins at location~\PB{\|p}, we set the corresponding pointer to
$p+m^4-1$, as if the list length were~$-1$, and call the list ``closed.''
That pointer value serves
as a sentinel to tell us that we needn't update the list any further.

\Y\B\4\X4:Subroutines\X${}\mathrel+\E{}$\6
\&{void} \\{print\_list}(\&{int} \|p)\1\1\2\2\6
${}\{{}$\1\6
\&{register} \&{int} \|q${},{}$ \|r;\7
${}\|q\K\\{mem}[\|p+\\{mmmm}];{}$\6
${}\\{fprintf}(\\{stderr},\39\.{"List\ "});{}$\6
\\{decode\_list}(\|p);\6
${}\\{fprintf}(\\{stderr},\39\.{":"});{}$\6
\&{if} ${}(\|q<\|p){}$\1\5
${}\\{fprintf}(\\{stderr},\39\.{"\ ."});{}$\2\6
\&{else}\1\6
\&{for} ${}(\|r\K\|p;{}$ ${}\|r<\|q;{}$ ${}\|r\PP){}$\1\5
${}\\{fprintf}(\\{stderr},\39\.{"\ \%04x"},\39\\{mem}[\|r]);{}$\2\2\6
${}\\{fprintf}(\\{stderr},\39\.{"\\n"});{}$\6
\4${}\}{}$\2\par
\fi

\M[365 back-commafree4.w]{20}\B\X4:Subroutines\X${}\mathrel+\E{}$\6
\&{void} \\{print\_all\_lists}(\&{void})\1\1\2\2\6
${}\{{}$\1\6
\&{register} \&{int} \|a${},{}$ \|b${},{}$ \|c${},{}$ \|k;\7
\&{for} ${}(\|a\K\T{0};{}$ ${}\|a<\|m;{}$ ${}\|a\PP){}$\5
${}\{{}$\1\6
${}\\{print\_list}(\\{p1offset}+\|a*\|m*\|m*\|m);{}$\6
${}\\{print\_list}(\\{s1offset}+\|a*\|m*\|m*\|m);{}$\6
\&{for} ${}(\|b\K\T{0};{}$ ${}\|b<\|m;{}$ ${}\|b\PP){}$\5
${}\{{}$\1\6
${}\\{print\_list}(\\{p2offset}+(\|a*\|m+\|b)*\|m*\|m);{}$\6
${}\\{print\_list}(\\{s2offset}+(\|a*\|m+\|b)*\|m*\|m);{}$\6
\&{for} ${}(\|c\K\T{0};{}$ ${}\|c<\|m;{}$ ${}\|c\PP){}$\5
${}\{{}$\1\6
${}\\{print\_list}(\\{p3offset}+((\|a*\|m+\|b)*\|m+\|c)*\|m);{}$\6
${}\\{print\_list}(\\{s3offset}+((\|a*\|m+\|b)*\|m+\|c)*\|m);{}$\6
\4${}\}{}$\2\6
\4${}\}{}$\2\6
\4${}\}{}$\2\6
\&{for} ${}(\|k\K\T{0};{}$ ${}\|k<\\{maxl};{}$ ${}\|k\PP){}$\1\5
${}\\{print\_list}(\\{cloffset}+\T{4}*\|k);{}$\2\6
\4${}\}{}$\2\par
\fi

\N[383 back-commafree4.w]{1}{21}Sanity checking. Whenever data structures are
highly redundant,
we want to monitor them carefully until we're confident that our
program isn't messing them up. The \PB{\\{sanity}} routine is intended to
nip bugs in the bud.

\Y\B\4\D$\\{sanity\_checking}$ \5
\T{0}\C{ set this to 1 if you suspect a bug }\par
\Y\B\4\X4:Subroutines\X${}\mathrel+\E{}$\6
\&{void} \\{sanity}(\&{void})\1\1\2\2\6
${}\{{}$\1\6
\&{register} \&{int} \|a${},{}$ \|b${},{}$ \|c${},{}$ \|j${},{}$ \|k${},{}$ %
\|p${},{}$ \|q${},{}$ \|r${},{}$ \|x${},{}$ \\{alpha};\7
\X22:Check that the relevant lists contain only blue words\X;\6
\X23:Check that all blue words appear in all relevant lists\X;\6
\4${}\}{}$\2\par
\fi

\M[397 back-commafree4.w]{22}Here we check also the inverse property of the
inverse lists.

\Y\B\4\D$\\{bluecheck}(\\{loc},\\{off})$ \6
${}\{{}$\1\6
\&{for} ${}(\|p\K\\{loc}+\\{off},\39\|q\K\\{mem}[\\{loc}+\\{mmmm}];{}$ ${}\|p<%
\|q;{}$ ${}\|p\PP){}$\5
${}\{{}$\1\6
${}\|x\K\\{mem}[\|p],\39\\{alpha}\K\\{hexconv}[\|x];{}$\6
\&{if} ${}(\\{mem}[\\{alpha}]\I\\{blue}){}$\1\5
${}\\{fprintf}(\\{stderr},\39\.{"\ \%04x\ in\ list\ \%d\ is}\)\.{\ \%s!\\n"},%
\39\|x,\39\\{loc}+\\{off},\39\\{mem}[\\{alpha}]\E\\{green}\?\.{"green"}:%
\.{"red"});{}$\2\6
\&{if} ${}(\\{mem}[\\{off}-\\{mmmm}+\\{alpha}]\I\|p){}$\1\5
${}\\{fprintf}(\\{stderr},\39\.{"\ inverse\ index\ for\ }\)\.{\%04x\ in\ list\ %
\%d\ shou}\)\.{ld\ be\ \%d,\ not\ \%d!\\n"},\39\|x,\39\\{loc}+\\{off},\39\|p,%
\39\\{mem}[\\{off}-\\{mmmm}+\\{alpha}]);{}$\2\6
\4${}\}{}$\2\6
\4${}\}{}$\2\par
\Y\B\4\X22:Check that the relevant lists contain only blue words\X${}\E{}$\6
\&{for} ${}(\|a\K\T{0};{}$ ${}\|a<\|m;{}$ ${}\|a\PP){}$\5
${}\{{}$\1\6
${}\\{bluecheck}(\|a*\|m*\|m*\|m,\39\\{p1offset});{}$\6
${}\\{bluecheck}(\|a*\|m*\|m*\|m,\39\\{s1offset});{}$\6
\&{for} ${}(\|b\K\T{0};{}$ ${}\|b<\|m;{}$ ${}\|b\PP){}$\5
${}\{{}$\1\6
${}\\{bluecheck}((\|a*\|m+\|b)*\|m*\|m,\39\\{p2offset});{}$\6
${}\\{bluecheck}((\|a*\|m+\|b)*\|m*\|m,\39\\{s2offset});{}$\6
\&{for} ${}(\|c\K\T{0};{}$ ${}\|c<\|m;{}$ ${}\|c\PP){}$\5
${}\{{}$\1\6
${}\\{bluecheck}(((\|a*\|m+\|b)*\|m+\|c)*\|m,\39\\{p3offset});{}$\6
${}\\{bluecheck}(((\|a*\|m+\|b)*\|m+\|c)*\|m,\39\\{s3offset});{}$\6
\4${}\}{}$\2\6
\4${}\}{}$\2\6
\4${}\}{}$\2\6
\&{for} ${}(\|k\K\T{0};{}$ ${}\|k<\\{maxl};{}$ ${}\|k\PP){}$\1\5
${}\\{bluecheck}(\T{4}*\|k,\39\\{cloffset}){}$;\2\par
\U21.\fi

\M[426 back-commafree4.w]{23}Here we check green words also, to make sure that
their lists have
been ``closed.''

\Y\B\4\D$\\{appcheck}(\|x,\\{alpha},\\{del},\\{off})$ \6
${}\{{}$\1\6
${}\|p\K(\\{del})+(\\{off}){}$;\C{ the list starts here }\6
${}\|q\K\\{mem}[\|p+\\{mmmm}]{}$;\C{ the list currently ends here }\6
\&{if} ${}(\|q\I\|p-\T{1}){}$\5
${}\{{}$\C{ it's not closed }\1\6
\&{if} ${}(\\{mem}[\\{alpha}]\E\\{green}){}$\5
${}\{{}$\1\6
${}\\{fprintf}(\\{stderr},\39\.{"\ list\ "}){}$;\5
\\{decode\_list}(\|p);\6
${}\\{fprintf}(\\{stderr},\39\.{"\ for\ \%x\ isn't\ close}\)\.{d!\\n"},\39%
\|x);{}$\6
\4${}\}{}$\5
\2\&{else}\5
${}\{{}$\1\6
${}\|r\K\\{mem}[\\{off}-\\{mmmm}+\\{alpha}]{}$;\C{ where \PB{\|x} supposedly is
}\6
\&{if} ${}(\|r<\|p\V\|r\G\|q){}$\5
${}\{{}$\1\6
${}\\{fprintf}(\\{stderr},\39\.{"\ \%x\ isn't\ in\ list\ "},\39\|x);{}$\6
\\{decode\_list}(\|p);\5
${}\\{fprintf}(\\{stderr},\39\.{"!\\n"});{}$\6
\4${}\}{}$\2\6
\4${}\}{}$\2\6
\4${}\}{}$\2\6
\4${}\}{}$\2\par
\Y\B\4\X23:Check that all blue words appear in all relevant lists\X${}\E{}$\6
\&{for} ${}(\\{alpha}\K\T{0};{}$ ${}\\{alpha}<\\{mmmm};{}$ ${}\\{alpha}\PP){}$%
\1\6
\&{if} ${}(\\{mem}[\\{alpha}]\I\\{red}){}$\5
${}\{{}$\1\6
${}\|x\K\\{mconv}[\\{alpha}];{}$\6
${}\\{appcheck}(\|x,\39\\{alpha},\39\\{p1}(\|x),\39\\{p1offset});{}$\6
${}\\{appcheck}(\|x,\39\\{alpha},\39\\{p2}(\|x),\39\\{p2offset});{}$\6
${}\\{appcheck}(\|x,\39\\{alpha},\39\\{p3}(\|x),\39\\{p3offset});{}$\6
${}\\{appcheck}(\|x,\39\\{alpha},\39\\{s1}(\|x),\39\\{s1offset});{}$\6
${}\\{appcheck}(\|x,\39\\{alpha},\39\\{s2}(\|x),\39\\{s2offset});{}$\6
${}\\{appcheck}(\|x,\39\\{alpha},\39\\{s3}(\|x),\39\\{s3offset});{}$\6
${}\\{appcheck}(\|x,\39\\{alpha},\39\T{4}*\\{cl}[\\{alpha}],\39%
\\{cloffset});{}$\6
\4${}\}{}$\2\2\par
\U21.\fi

\N[460 back-commafree4.w]{1}{24}The poison list. We also need to maintain
another interesting data
structure, which remembers forbidden suffix-prefix pairs. Whenever
we've chosen $abcd$ to be a codeword, we want to remember that
no future codeword that ends with $a$ can be simultaneously present
with a codeword that begins with $bcd$. The first time either of those
events happens, we want to redden all of the blue words on the other side.
So we'll add the item `$a/bcd$' to the current poison list when $abcd$ becomes
green.

The suffix-prefix pairs $ab/cd$ and $abc/d$ are forbidden in the same way.

The poison list is kept within \PB{\\{mem}}, beginning at location \PB{%
\\{poison}};
its pointer is in \PB{\\{mem}} location \PB{$\\{poison}-\T{1}$}. Each item on
this list occupies
two consecutive \PB{\\{mem}} positions, one pointing a the suffix and the other
to the corresponding prefix.

An item will be removed from the poison list whenever we discover that one
or both of the lists that it points to has become empty.

\Y\B\4\X6:Global variables\X${}\mathrel+\E{}$\6
\&{int} \\{poison};\par
\fi

\M[482 back-commafree4.w]{25}\B\X10:Initialize the tables\X${}\mathrel+\E{}$\6
$\\{poison}\K\T{22}*\\{mmmm}+\T{1};{}$\6
${}\\{mem}[\\{poison}-\T{1}]\K\\{poison}{}$;\C{ empty list }\par
\fi

\M[486 back-commafree4.w]{26}When the following code is performed, \PB{\\{pp}}
will point to the
current end of the poison list.

\Y\B\4\X26:Delete entry \PB{\|p} from the poison list\X${}\E{}$\6
$\\{pp}\MRL{-{\K}}\T{2};{}$\6
\&{if} ${}(\|p\I\\{pp}){}$\5
${}\{{}$\1\6
${}\|o,\39\\{store}(\|p,\39\\{mem}[\\{pp}]);{}$\6
${}\\{store}(\|p+\T{1},\39\\{mem}[\\{pp}+\T{1}]){}$;\C{ \PB{\\{mem}[\\{pp}]}
and \PB{$\\{mem}[\\{pp}+\T{1}]$} are in the same octabyte }\6
\4${}\}{}$\2\par
\Us39\ET40.\fi

\M[496 back-commafree4.w]{27}\B\X27:Append \PB{\|y} and \PB{\|z} to the poison
list\X${}\E{}$\6
$\\{store}(\\{pp},\39\|y),\39\\{store}(\\{pp}+\T{1},\39\|z),\39\\{mems}\MM;{}$\6
${}\\{pp}\MRL{+{\K}}\T{2};{}$\6
\&{if} ${}(\\{pp}>\\{maxpoison}){}$\1\5
${}\\{maxpoison}\K\\{pp}{}$;\C{ the poison list can't overflow }\2\par
\U38.\fi

\N[501 back-commafree4.w]{1}{28}Lists outside of \PB{\\{mem}}. Some of the
state information is sufficiently
static and well-behaved that we needn't keep it in \PB{\\{mem}}. For example,
at every level we have a ``frame'' of vital data, telling for instance
what class was involved, what word was chosen in that class, and how big
the undo stack was when our data structures were stable.

\Y\B\4\X28:Type definitions\X${}\E{}$\6
\&{typedef} \&{struct} \&{frame\_struct} ${}\{{}$\1\6
\&{int} \\{cls};\C{ the class number involved at this level }\6
\&{int} \|x;\C{ the word of that class, or $-1$ if this class not included }\6
\&{int} \\{slack};\C{ how many classes are we still allowed to omit? }\6
\&{int} \\{uptr};\C{ how many items should we retain on \PB{\\{undo}} after
backtracking? }\2\6
${}\}{}$ \&{frame};\par
\U1.\fi

\M[515 back-commafree4.w]{29}We also maintain a list called \PB{\\{clfree}}, of
the classes not yet chosen.
It will contain \PB{$\\{maxl}-\|l$} items when we're on level~\PB{\|l}; but I
maintain the
redundant variable \PB{\\{freeptr}}, so that I don't have to worry about
transient moments as \PB{\|l} is changing. There's also
an inverse list \PB{\\{ifree}}.

\Y\B\4\X6:Global variables\X${}\mathrel+\E{}$\6
\&{frame} ${}\\{move}[\\{maxclasses}+\T{1}]{}$;\C{ basic info at each level }\6
\&{int} \\{clfree}[\\{maxclasses}];\C{ which classes are free? }\6
\&{int} \\{freeptr};\C{ the number of free classes }\6
\&{int} \\{ifree}[\\{maxclasses}];\C{ where is this class in \PB{\\{free}}? }%
\par
\fi

\M[527 back-commafree4.w]{30}\B\X10:Initialize the tables\X${}\mathrel+\E{}$\6
\&{for} ${}(\|k\K\T{0},\39\\{freeptr}\K\\{maxl};{}$ ${}\|k<\\{maxl};{}$ ${}\|k%
\PP){}$\1\5
${}\\{oo},\39\\{clfree}[\|k]\K\\{ifree}[\|k]\K\|k{}$;\2\par
\fi

\M[530 back-commafree4.w]{31}\B\X4:Subroutines\X${}\mathrel+\E{}$\6
\&{void} \\{print\_moves}(\&{int} \\{lev})\1\1\2\2\6
${}\{{}$\1\6
\&{register} \&{int} \|l;\7
\&{for} ${}(\|l\K\T{1};{}$ ${}\|l\Z\\{lev};{}$ ${}\|l\PP){}$\1\6
\&{if} ${}(\\{move}[\|l].\|x\G\T{0}){}$\1\5
${}\\{printf}(\.{"\ \%04x"},\39\\{move}[\|l].\|x);{}$\2\6
\&{else}\1\5
${}\\{printf}(\.{"\ (\%04x)"},\39\\{clrep}[\\{move}[\|l].\\{cls}]);{}$\2\2\6
\\{printf}(\.{"\\n"});\6
\4${}\}{}$\2\par
\fi

\N[539 back-commafree4.w]{1}{32}Doing it. OK, we're (hopefully) ready now for
the big climax.

\Y\B\4\X32:Backtrack through all solutions\X${}\E{}$\6
$\|l\K\T{1};{}$\6
${}\\{cls}\K\T{0},\39\|x\K\T{\^0001},\39\\{slack}\K\\{maxl}-\\{goal};{}$\6
${}\\{nodes}\K\\{profile}[\T{1}]\K\T{1};{}$\6
${}\\{uptr}\K\T{0};{}$\6
\&{goto} \\{tryit};\6
\4\\{enter}:\5
${}\\{profile}[\|l]\PP,\39\\{nodes}\PP;{}$\6
\X33:Report the current profile, if \PB{$\\{mems}\G\\{thresh}$}\X;\6
\&{if} ${}(\|l>\\{maxl}){}$\1\5
\X46:Report a solution and \PB{\&{goto} \\{backup}}\X;\2\6
\4\\{choose}:\5
\&{if} (\\{sanity\_checking})\1\5
\\{sanity}(\,);\2\6
\X34:Choose a potential move \PB{\|x}\X;\6
\4\\{tryit}:\5
\&{if} ${}(\\{vbose}>\T{1}){}$\5
${}\{{}$\1\6
\&{if} ${}(\|x\G\T{0}){}$\1\5
${}\\{fprintf}(\\{stderr},\39\.{"Level\ \%d,\ trying\ \%0}\)\.{4x\ (\%lld\
mems)\\n"},\39\|l,\39\|x,\39\\{mems});{}$\2\6
\&{else}\1\5
${}\\{fprintf}(\\{stderr},\39\.{"Level\ \%d,\ trying\ (\%}\)\.{04x)\ (\%lld\
mems)\\n"},\39\|l,\39\\{clrep}[\\{cls}],\39\\{mems});{}$\2\6
\4${}\}{}$\2\6
${}\|o,\39\\{move}[\|l].\\{uptr}\K\\{uptr}{}$;\5
\X7:Bump \PB{\\{curstamp}}\X;\6
\&{if} ${}(\|x\G\T{0}){}$\1\5
\X36:Update the data structures for new codeword \PB{\|x}\X\2\6
\&{else} \&{if} ${}(\\{slack}\E\T{0}\V\|l\E\T{1}){}$\1\5
\&{goto} \\{backup};\C{ disallow slack move at root level }\2\6
\&{else}\1\5
${}\\{slack}\MM;{}$\2\6
${}\|o,\39\\{move}[\|l].\|x\K\|x,\39\\{move}[\|l].\\{cls}\K\\{cls}{}$;\C{ both
in same octabyte, so 1 mem }\6
${}\|o,\39\\{move}[\|l].\\{slack}\K\\{slack};{}$\6
\X44:Update the free list\X;\6
${}\|l\PP{}$;\5
\&{goto} \\{enter};\6
\4\\{try\_again}:\5
${}\|o,\39\\{unstore}(\\{move}[\|l].\\{uptr});{}$\6
\&{if} ${}(\\{vbose}>\T{1}){}$\1\5
${}\\{fprintf}(\\{stderr},\39\.{"Level\ \%d,\ forbiddin}\)\.{g\ \%04x\ (\%lld\
mems)\\n}\)\.{"},\39\|l,\39\|x,\39\\{mems});{}$\2\6
\X7:Bump \PB{\\{curstamp}}\X;\6
\X43:Update the data structures to forbid \PB{\|x}\X;\6
\&{goto} \\{choose};\6
\4\\{backup}:\5
\&{if} ${}(\MM\|l){}$\5
${}\{{}$\1\6
${}\|o,\39\|x\K\\{move}[\|l].\|x,\39\\{cls}\K\\{move}[\|l].\\{cls};{}$\6
\X45:Downdate the free list\X;\6
\&{if} ${}(\|x<\T{0}){}$\1\5
\&{goto} \\{backup};\C{ omitting a class was the last resort }\2\6
${}\\{slack}\K\\{move}[\|l].\\{slack}{}$;\C{ this mem is covered at \PB{\\{try%
\_again}} }\6
\&{goto} \\{try\_again};\6
\4${}\}{}$\2\par
\U1.\fi

\M[581 back-commafree4.w]{33}\B\X33:Report the current profile, if \PB{$%
\\{mems}\G\\{thresh}$}\X${}\E{}$\6
\&{if} ${}(\\{mems}\G\\{thresh}){}$\5
${}\{{}$\1\6
${}\\{thresh}\MRL{+{\K}}\T{10000000000};{}$\6
${}\\{fprintf}(\\{stderr},\39\.{"After\ \%lld\ mems:"},\39\\{mems});{}$\6
\&{for} ${}(\|k\K\T{2};{}$ ${}\|k\Z\|l;{}$ ${}\|k\PP){}$\1\5
${}\\{fprintf}(\\{stderr},\39\.{"\ \%lld"},\39\\{profile}[\|k]);{}$\2\6
${}\\{fprintf}(\\{stderr},\39\.{"\\n"});{}$\6
\4${}\}{}$\2\par
\U32.\fi

\M[589 back-commafree4.w]{34}Two strategies are used to decide which
possibility to explore next.
First we try to find a free class that has the smallest remaining
number of blue words. If that number is 0 or~1, we choose it.

Otherwise we look at the poison list, and look for a choice that
is guaranteed to redden as many blue words as possible. If such
a choice exceeds a certain threshold, we make it, under the
assumption that this choice will soon be proved unsatisfactory,
thus giving us a new red word at comparatively little cost.

(In a sense, our goal is to make words red, because we are done
when enough of them have become red.)

The threshold is a compile-time parameter, because I'm going to be
running this program only a few times.

\Y\B\4\D$\\{sthresh}$ \5
\T{1}\C{ must be positive }\par
\Y\B\4\X34:Choose a potential move \PB{\|x}\X${}\E{}$\6
\X35:Find a class \PB{\\{cls}} with fewest blue words~\PB{\|r}\X;\6
\&{if} ${}(\|r>\T{1}){}$\5
${}\{{}$\1\6
\X40:Find \PB{\|x} that maximizes the ``kill'' potential~\PB{\|s}\X;\6
\&{if} ${}(\|s\G\\{sthresh}){}$\5
${}\{{}$\1\6
${}\\{oo},\39\\{cls}\K\\{cl}[\\{hexconv}[\|x]];{}$\6
\&{goto} \\{done};\6
\4${}\}{}$\2\6
\4${}\}{}$\2\6
\&{if} ${}(\|r\E\T{0}){}$\1\5
${}\|x\K{-}\T{1};{}$\2\6
\&{else}\1\5
${}\|o,\39\|x\K\\{mem}[\T{4}*\\{cls}+\\{cloffset}];{}$\2\6
\\{done}:\par
\U32.\fi

\M[620 back-commafree4.w]{35}\B\X35:Find a class \PB{\\{cls}} with fewest blue
words~\PB{\|r}\X${}\E{}$\6
\&{if} ${}(\\{vbose}>\T{3}){}$\1\5
${}\\{fprintf}(\\{stderr},\39\.{"\ class\ sizes"});{}$\2\6
\&{for} ${}(\|r\K\T{5},\39\|k\K\T{0};{}$ ${}\|k<\\{freeptr};{}$ ${}\|k\PP){}$\5
${}\{{}$\1\6
${}\|c\K\\{clfree}[\|k];{}$\6
${}\|o,\39\|j\K\\{mem}[\T{4}*\|c+\\{cloffset}+\\{mmmm}]-(\T{4}*\|c+%
\\{cloffset});{}$\6
\&{if} ${}(\\{vbose}>\T{3}){}$\1\5
${}\\{fprintf}(\\{stderr},\39\.{"\ \%04x:\%d"},\39\\{clrep}[\|c],\39\|j);{}$\2\6
\&{if} ${}(\|j<\|r){}$\5
${}\{{}$\1\6
${}\|r\K\|j,\39\\{cls}\K\|c;{}$\6
\&{if} ${}(\|r\E\T{0}){}$\1\5
\&{break};\2\6
\4${}\}{}$\2\6
\4${}\}{}$\2\6
\&{if} ${}(\\{vbose}>\T{3}){}$\1\5
${}\\{fprintf}(\\{stderr},\39\.{"\\n"}){}$;\2\par
\U34.\fi

\M[634 back-commafree4.w]{36}I should perhaps have added the phrase `or \PB{%
\&{goto} \\{try\_again}}' to the title
of this module, because we might find a reason to reject~\PB{\|x}.

\Y\B\4\X36:Update the data structures for new codeword \PB{\|x}\X${}\E{}$\6
${}\{{}$\1\6
\X37:Make \PB{\|x} green and close the lists it's in\X;\6
${}\|o,\39\\{pp}\K\\{mem}[\\{poison}-\T{1}]{}$;\C{ temporarily keep poison list
size in a register }\6
\X38:Append \PB{\|x}'s innards to the poison list\X;\6
\X39:Update the poison list, or \PB{\&{goto} \\{try\_again}}\X;\6
${}\\{store}(\\{poison}-\T{1},\39\\{pp});{}$\6
\4${}\}{}$\2\par
\U32.\fi

\M[646 back-commafree4.w]{37}\B\D$\\{closelist}(\\{del},\\{off})$ \6
${}\{{}$\1\6
${}\|p\K(\\{del})+(\\{off});{}$\6
${}\|o,\39\|q\K\\{mem}[\|p+\\{mmmm}];{}$\6
\&{if} ${}(\|q\I\|p-\T{1}){}$\1\5
${}\\{store}(\|p+\\{mmmm},\39\|p-\T{1});{}$\2\6
\4${}\}{}$\2\par
\Y\B\4\X37:Make \PB{\|x} green and close the lists it's in\X${}\E{}$\6
$\|o,\39\\{alpha}\K\\{hexconv}[\|x];{}$\6
${}\\{store}(\\{alpha},\39\\{green});{}$\6
${}\\{closelist}(\\{p1}(\|x),\39\\{p1offset});{}$\6
${}\\{closelist}(\\{p2}(\|x),\39\\{p2offset});{}$\6
${}\\{closelist}(\\{p3}(\|x),\39\\{p3offset});{}$\6
${}\\{closelist}(\\{s1}(\|x),\39\\{s1offset});{}$\6
${}\\{closelist}(\\{s2}(\|x),\39\\{s2offset});{}$\6
${}\\{closelist}(\\{s3}(\|x),\39\\{s3offset});{}$\6
${}\|o,\39\|k\K\\{cl}[\\{alpha}];{}$\6
${}\\{closelist}(\T{4}*\|k,\39\\{cloffset});{}$\6
\&{for} ${}(\|r\K\\{cloffset}+\T{4}*\|k;{}$ ${}\|r<\|q;{}$ ${}\|r\PP){}$\1\6
\&{if} ${}(\|o,\39\\{mem}[\|r]\I\|x){}$\1\5
\\{redden}(\\{mem}[\|r]);\2\2\par
\U36.\fi

\M[665 back-commafree4.w]{38}\B\X38:Append \PB{\|x}'s innards to the poison
list\X${}\E{}$\6
$\|y\K\\{p1}(\|x)+\\{s1offset};{}$\6
${}\|z\K\\{s3}(\|x)+\\{p3offset};{}$\6
\X27:Append \PB{\|y} and \PB{\|z} to the poison list\X;\6
${}\|y\K\\{p2}(\|x)+\\{s2offset};{}$\6
${}\|z\K\\{s2}(\|x)+\\{p2offset};{}$\6
\X27:Append \PB{\|y} and \PB{\|z} to the poison list\X;\6
${}\|y\K\\{p3}(\|x)+\\{s3offset};{}$\6
${}\|z\K\\{s1}(\|x)+\\{p1offset};{}$\6
\X27:Append \PB{\|y} and \PB{\|z} to the poison list\X;\par
\U36.\fi

\M[676 back-commafree4.w]{39}This section is more or less the heart of the
entire program.
Or maybe it's better described as the ``nerve center.'' Anyway,
it enforces the comma-free condition, when you translate all of
this esoteric logic into common sense terms.

After an entry on the poison list causes words to become red, previous
entries of that list might become deletable. We don't go back and delete
them, however; we can do that later, when we traverse the poison list again.

When I first wrote this part, I thought the case \PB{$\\{yy}<\|y\W\\{zz}<\|z$}
would be
impossible. But I first saw it occur legitimately when codewords \.{0001}
and \.{1011} had been chosen at levels 1 and~2, then \.{0011} was tried
at level~3. In that case, lists \.{.001} and \.{1...} were both closed.

\Y\B\4\X39:Update the poison list, or \PB{\&{goto} \\{try\_again}}\X${}\E{}$\6
\&{for} ${}(\|p\K\\{poison};{}$ ${}\|p<\\{pp};{}$ \,)\5
${}\{{}$\1\6
${}\|o,\39\|y\K\\{mem}[\|p],\39\|z\K\\{mem}[\|p+\T{1}]{}$;\C{ \PB{\|y} and \PB{%
\|z} are pointers to lists }\6
${}\|o,\39\\{yy}\K\\{mem}[\|y+\\{mmmm}];{}$\6
\&{if} ${}(\\{yy}\E\|y){}$\1\5
\&{goto} \\{delete};\C{ delete poison list entry that cites an empty list }\2\6
${}\|o,\39\\{zz}\K\\{mem}[\|z+\\{mmmm}];{}$\6
\&{if} ${}(\\{zz}\E\|z){}$\1\5
\&{goto} \\{delete};\C{ delete poison list entry that cites an empty list }\2\6
\&{if} ${}(\\{yy}<\|y\W\\{zz}<\|z){}$\1\5
\&{goto} \\{try\_again};\2\6
\&{if} ${}(\\{yy}>\|y\W\\{zz}>\|z){}$\5
${}\{{}$\1\6
${}\|p\MRL{+{\K}}\T{2}{}$;\5
\&{continue};\6
\4${}\}{}$\2\6
\&{if} ${}(\\{yy}<\|y){}$\5
${}\{{}$\1\6
\&{if} ${}(\\{vbose}>\T{3}){}$\5
${}\{{}$\1\6
${}\\{fprintf}(\\{stderr},\39\.{"\ killing\ "}){}$;\5
\\{decode\_list}(\|z);\5
${}\\{fprintf}(\\{stderr},\39\.{"\\n"});{}$\6
\4${}\}{}$\2\6
${}\\{store}(\|z+\\{mmmm},\39\|z){}$;\C{ make list \PB{\|z} empty }\6
\&{for} ${}(\|r\K\|z;{}$ ${}\|r<\\{zz};{}$ ${}\|r\PP){}$\1\5
${}\|o,\39\\{redden}(\\{mem}[\|r]);{}$\2\6
\4${}\}{}$\5
\2\&{else}\5
${}\{{}$\1\6
\&{if} ${}(\\{vbose}>\T{3}){}$\5
${}\{{}$\1\6
${}\\{fprintf}(\\{stderr},\39\.{"\ killing\ "}){}$;\5
\\{decode\_list}(\|y);\5
${}\\{fprintf}(\\{stderr},\39\.{"\\n"});{}$\6
\4${}\}{}$\2\6
${}\\{store}(\|y+\\{mmmm},\39\|y){}$;\C{ make list \PB{\|y} empty }\6
\&{for} ${}(\|r\K\|y;{}$ ${}\|r<\\{yy};{}$ ${}\|r\PP){}$\1\5
${}\|o,\39\\{redden}(\\{mem}[\|r]);{}$\2\6
\4${}\}{}$\2\6
\4\\{delete}:\5
\X26:Delete entry \PB{\|p} from the poison list\X;\6
\4${}\}{}$\2\par
\U36.\fi

\M[717 back-commafree4.w]{40}Items in the poison list pair a suffix with a
prefix, where the suffix list
has \PB{$\\{yy}-\|y$} elements and the prefix class has \PB{$\\{zz}-\|z$}
elements. The larger of
these lists can be killed off by choosing \PB{\|x} in the other list.

(Here's an unimplemented
afterthought, which potentially could be more powerful; unfortunately I
have no time to explore it: Suppose we've chosen a word $abcd$. Hence we will
have made an entry for ${*}{*}{*}a$ and $bcd{*}$ in the poison list, and it
will be nice if the $bcd{*}$ list is nonempty because any such word will
wipe out all of ${*}{*}{*}a$. However, that latter list may {\it already\/}
be empty, in which case we'll drop this item from the poison list.
The new idea is that any item $bcdz$ on list $bcd{*}$ will {\it also\/}
wipe out the entire list $z{*}{*}{*}$, which might well be huge for
at least one~$z$.
Similarly, any item $cdyz$ on list $cd{*}{*}$ will wipe out
$yz{*}{*}$; any item $zabc$ on list ${*}abc$ will wipe out ${*}{*}{*}z$;
etc. These six cases are not currently considered, because they aren't
easily detected within the present data structures.)

\Y\B\4\X40:Find \PB{\|x} that maximizes the ``kill'' potential~\PB{\|s}\X${}%
\E{}$\6
$\|s\K\T{0};{}$\6
\&{if} ${}(\\{vbose}>\T{3}){}$\1\5
${}\\{fprintf}(\\{stderr},\39\.{"\ kills"});{}$\2\6
${}\|o,\39\\{pp}\K\\{mem}[\\{poison}-\T{1}]{}$;\C{ temporarily keep poison list
size in a register }\6
\&{for} ${}(\|p\K\\{poison};{}$ ${}\|p<\\{pp};{}$ \,)\5
${}\{{}$\1\6
${}\|o,\39\|y\K\\{mem}[\|p],\39\|z\K\\{mem}[\|p+\T{1}];{}$\6
${}\|o,\39\\{yy}\K\\{mem}[\|y+\\{mmmm}];{}$\6
\&{if} ${}(\\{yy}\E\|y){}$\1\5
\&{goto} \\{deleet};\2\6
${}\|o,\39\\{zz}\K\\{mem}[\|z+\\{mmmm}];{}$\6
\&{if} ${}(\\{zz}\E\|z){}$\1\5
\&{goto} \\{deleet};\2\6
\&{if} ${}(\\{yy}<\|y\V\\{zz}<\|z){}$\1\5
\\{confusion}(\.{"poison\ remnant"});\2\6
\&{if} ${}(\\{vbose}>\T{3}){}$\5
${}\{{}$\1\6
${}\\{fprintf}(\\{stderr},\39\.{"\ \%d"},\39\\{yy}-\|y);{}$\6
\\{decode\_list}(\|y);\6
${}\\{fprintf}(\\{stderr},\39\.{"|"});{}$\6
\\{decode\_list}(\|z);\6
${}\\{fprintf}(\\{stderr},\39\.{"\%d"},\39\\{zz}-\|z);{}$\6
\4${}\}{}$\2\6
\&{if} ${}(\\{yy}-\|y\G\\{zz}-\|z){}$\5
${}\{{}$\1\6
\&{if} ${}(\\{yy}-\|y>\|s){}$\1\5
${}\|o,\39\|s\K\\{yy}-\|y,\39\|x\K\\{mem}[\|z];{}$\2\6
\4${}\}{}$\5
\2\&{else} \&{if} ${}(\\{zz}-\|z>\|s){}$\1\5
${}\|o,\39\|s\K\\{zz}-\|z,\39\|x\K\\{mem}[\|y];{}$\2\6
${}\|p\MRL{+{\K}}\T{2};{}$\6
\&{continue};\6
\4\\{deleet}:\5
\X26:Delete entry \PB{\|p} from the poison list\X;\6
\4${}\}{}$\2\6
\&{if} ${}(\\{vbose}>\T{3}){}$\1\5
${}\\{fprintf}(\\{stderr},\39\.{"\\n"});{}$\2\6
${}\\{store}(\\{poison}-\T{1},\39\\{pp}){}$;\par
\Q1.
\U34.\fi

\M[766 back-commafree4.w]{41}Here I want to mention some code that was deleted
from an earlier version,
because it introduced a sneaky bug. I'd once had a mechanism in the \PB{%
\&{frame}} structure
designed to break symmetry by confining solutions to ``restricted growth
sequences.''

Namely, I had originally said this:
\smallskip
{\leftskip=10pt\rightskip=10pt

One of the frame elements, called \PB{\\{rgs}}, is used to avoid repeating
choices that are equivalent under symmetry. The term `\PB{\\{rgs}}' stands
for ``restricted growth string''; it names the largest digit seen so
far among the chosen codewords. We shall insist that the codeword for
level~1 is either \.{0001} or \.{0010}. Henceforth the next digit
greater than~\.1 must be \.2, and the next digit greater than~\.2
must be~\.3, etc.

(Subtle(?)\ point: If the class is, say, \.{0312} and \PB{$\\{rgs}\K\T{1}$},
we will allow only the representatives \.{1203} and \.{2031}, not
\.{0312} or \.{3120}, because \.2 must precede~\.3.)

}
\smallskip\noindent And the code below was activated just before
making~\PB{\|x} green, if \PB{\\{rgs}} was $m-2$ or less.

However, I'd missed an even more subtle point. This \PB{\\{rgs}} logic must
be done earlier, in the move-choosing sections: Those sections should
not choose an~\PB{\|x} that will be rejected on \PB{\\{rgs}} grounds. For if
they do,
\PB{\|x} will be permanently blacklisted (actually redlisted) even after
the \PB{\\{rgs}} condition is no longer relevant. Instead, when choosing an
\PB{\|x} on the basis of a killing strategy, one must find an \PB{\|x} that
is \PB{\\{rgs}}-legitimate. And when choosing an element of a class with few
blue elements, one must be sure that the element is \PB{\\{rgs}}-safe.

I could have corrected the logic. But the program would have been
quite a lot more complicated, and I already knew from experience that
I wouldn't be able to solve the case \PB{$\|m\K\T{5}$} even with the faulty
strategy that eliminated too much.

Furthermore, the best way to break the symmetry in hard cases is
probably to prespecify the first few levels of the search tree.

So I've opted to record here the lesson that I learned, but {\it not\/} to use
any form of the following rejected-rejection code:

\Y\B\4\X41:Reject \PB{\|x} if it violates the restricted growth condition\X${}%
\E{}$\6
${}\{{}$\1\6
\&{for} ${}(\|i\K\T{12};{}$ ${}\|i\G\T{0};{}$ ${}\|i\MRL{-{\K}}\T{4}){}$\5
${}\{{}$\1\6
${}\|y\K(\|x\GG\|i)\AND\T{\^f};{}$\6
\&{if} ${}(\|y>\\{rgs}){}$\5
${}\{{}$\1\6
\&{if} ${}(\|y>\\{rgs}+\T{1}){}$\1\5
\&{goto} \\{try\_again};\2\6
${}\\{rgs}\K\|y;{}$\6
\4${}\}{}$\2\6
\4${}\}{}$\2\6
\4${}\}{}$\2\par
\fi

\M[821 back-commafree4.w]{42}\B\D$\\{rem}(\\{alpha},\\{del},\\{off})$ \6
${}\{{}$\1\6
${}\|p\K(\\{del})+(\\{off});{}$\6
${}\|o,\39\|q\K\\{mem}[\|p+\\{mmmm}]-\T{1};{}$\6
\&{if} ${}(\|q\G\|p){}$\5
${}\{{}$\C{ list is nonempty }\1\6
${}\\{store}(\|p+\\{mmmm},\39\|q){}$;\C{ shorten the list }\6
${}\|o,\39\|t\K\\{mem}[(\\{off})-\\{mmmm}+\\{alpha}]{}$;\C{ where does \PB{\|x}
appear? }\6
\&{if} ${}(\|t\I\|q){}$\5
${}\{{}$\1\6
${}\|o,\39\|y\K\\{mem}[\|q]{}$;\C{ get item to move }\6
${}\\{store}(\|t,\39\|y){}$;\C{ put \PB{\|y} in place of \PB{\|x} }\6
${}\|o,\39\\{store}((\\{off})-\\{mmmm}+\\{hexconv}[\|y],\39\|t){}$;\C{ update
the inverse index }\6
\4${}\}{}$\2\6
\4${}\}{}$\2\6
\4${}\}{}$\2\par
\Y\B\4\X4:Subroutines\X${}\mathrel+\E{}$\6
\&{inline} \&{void} \\{redden}(\&{register} \&{int} \|x)\1\1\2\2\6
${}\{{}$\1\6
\&{register} \&{int} \\{alpha}${},{}$ \|p${},{}$ \|q${},{}$ \|t${},{}$ \|y;\7
\&{if} ${}(\\{vbose}>\T{2}){}$\1\5
${}\\{fprintf}(\\{stderr},\39\.{"\ red\ \%04x\\n"},\39\|x);{}$\2\6
${}\|o,\39\\{alpha}\K\\{hexconv}[\|x];{}$\6
\&{if} ${}(\\{mem}[\\{alpha}]\I\\{blue}){}$\1\5
\\{confusion}(\.{"redden"});\2\6
${}\\{store}(\\{alpha},\39\\{red});{}$\6
${}\\{rem}(\\{alpha},\39\\{p1}(\|x),\39\\{p1offset}){}$;\C{ remove \PB{\|x}
from its \PB{\\{p1}} list }\6
${}\\{rem}(\\{alpha},\39\\{p2}(\|x),\39\\{p2offset}){}$;\C{ remove \PB{\|x}
from its \PB{\\{p2}} list }\6
${}\\{rem}(\\{alpha},\39\\{p3}(\|x),\39\\{p3offset}){}$;\C{ remove \PB{\|x}
from its \PB{\\{p3}} list }\6
${}\\{rem}(\\{alpha},\39\\{s1}(\|x),\39\\{s1offset}){}$;\C{ remove \PB{\|x}
from its \PB{\\{s1}} list }\6
${}\\{rem}(\\{alpha},\39\\{s2}(\|x),\39\\{s2offset}){}$;\C{ remove \PB{\|x}
from its \PB{\\{s2}} list }\6
${}\\{rem}(\\{alpha},\39\\{s3}(\|x),\39\\{s3offset}){}$;\C{ remove \PB{\|x}
from its \PB{\\{s3}} list }\6
${}\|o,\39\|t\K\\{cl}[\\{alpha}];{}$\6
${}\\{rem}(\\{alpha},\39\T{4}*\|t,\39\\{cloffset}){}$;\C{ remove \PB{\|x} from
its class list }\6
\4${}\}{}$\2\par
\fi

\M[853 back-commafree4.w]{43}\B\X43:Update the data structures to forbid \PB{%
\|x}\X${}\E{}$\6
\\{redden}(\|x);\par
\U32.\fi

\M[856 back-commafree4.w]{44}\B\X44:Update the free list\X${}\E{}$\6
$\\{freeptr}\MM;{}$\6
${}\|o,\39\|p\K\\{ifree}[\\{cls}];{}$\6
\&{if} ${}(\|p\I\\{freeptr}){}$\5
${}\{{}$\1\6
${}\|o,\39\|y\K\\{clfree}[\\{freeptr}];{}$\6
${}\|o,\39\\{clfree}[\|p]\K\|y;{}$\6
${}\|o,\39\\{ifree}[\|y]\K\|p;{}$\6
${}\|o,\39\\{clfree}[\\{freeptr}]\K\\{cls};{}$\6
${}\|o,\39\\{ifree}[\\{cls}]\K\\{freeptr};{}$\6
\4${}\}{}$\2\par
\U32.\fi

\M[867 back-commafree4.w]{45}\B\X45:Downdate the free list\X${}\E{}$\6
$\\{freeptr}\PP{}$;\par
\U32.\fi

\M[870 back-commafree4.w]{46}And we close with a happy moment.

\Y\B\4\X46:Report a solution and \PB{\&{goto} \\{backup}}\X${}\E{}$\6
${}\{{}$\1\6
${}\\{count}\PP;{}$\6
${}\\{printf}(\.{"\%d:"},\39\\{count});{}$\6
\\{print\_moves}(\\{maxl});\6
\&{goto} \\{backup};\6
\4${}\}{}$\2\par
\U32.\fi

\N[881 back-commafree4.w]{1}{47}Index.
\fi

\inx
\fin
\con
