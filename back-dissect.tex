\input cwebmac
\datethis
\def\adj{\mathrel{\!\mathrel-\mkern-8mu\mathrel-\mkern-8mu\mathrel-\!}}


\N{1}{1}Intro. This is an experimental program in which I try to cut a square
into a given number of pieces, in such a way that the pieces can be reassembled
to fill another given shape. (Everything is done pixelwise, without
``diagonal cuts.'') The pieces can be rotated but not flipped over.

I don't insist that the pieces be internally connected.
With change files I can add further restrictions.

The input on \PB{\\{stdin}} is a sequence of lines containing
periods and asterisks, where the asterisks mark usable positions.
The number of asterisks should be a perfect square.

The desired number of pieces is a command-line parameter.

\Y\B\4\D$\\{maxn}$ \5
\T{32}\C{ maximum number of input lines and characters per line }\par
\B\4\D$\\{maxd}$ \5
\T{7}\C{ maximum number of pieces }\par
\B\4\D$\\{bufsize}$ \5
$\\{maxn}+{}$\T{5}\C{ size of the input buffer }\par
\Y\B\8\#\&{include} \.{<stdio.h>}\6
\8\#\&{include} \.{<stdlib.h>}\6
\X25:Type definitions\X\7
\&{int} \|d;\C{ command-line parameter: the number of colors }\6
\&{char} \\{buf}[\\{bufsize}];\6
\&{int} \\{maxrow};\C{ largest row number used in the shape }\6
\&{int} \\{maxcol};\C{ largest column number used in the shape }\6
\&{char} ${}\\{aname}[\\{maxn}*\\{maxn}][\T{8}]{}$;\C{ symbolic names of the
cells in the square }\6
\&{char} ${}\\{bname}[\\{maxn}*\\{maxn}][\T{8}]{}$;\C{ symbolic names of the
cells in the shape }\6
\&{int} ${}\\{site}[\\{maxn}*\\{maxn}]{}$;\C{ where the cells are in the shape
}\6
\&{int} \\{vbose};\C{ level of verbosity }\7
\X11:Global variables\X;\6
\X33:Subroutines\X;\7
\\{main}(\&{int} \\{argc}${},\39{}$\&{char} ${}{*}\\{argv}[\,]){}$\1\1\2\2\6
${}\{{}$\1\6
\&{register} \&{int} \|a${},{}$ \|b${},{}$ \\{dd}${},{}$ \|i${},{}$ \|j${},{}$ %
\|k${},{}$ \|l${},{}$ \\{ll}${},{}$ \\{lll}${},{}$ \|m${},{}$ \|n${},{}$ %
\\{nn}${},{}$ \\{slack};\7
\X2:Process the command line\X;\6
\X3:Input the shape\X;\6
\X6:Find all solutions\X;\6
\X41:Print statistics about the run\X;\6
\4${}\}{}$\2\par
\fi

\M{2}\B\X2:Process the command line\X${}\E{}$\6
\&{if} ${}(\\{argc}<\T{2}\V\\{sscanf}(\\{argv}[\T{1}],\39\.{"\%d"},\39{\AND}%
\|d)\I\T{1}){}$\5
${}\{{}$\1\6
${}\\{fprintf}(\\{stderr},\39\.{"Usage:\ \%s\ d\ [verbos}\)\.{e]\ [extra\
verbose]\ <}\)\.{\ foo.dots\\n"},\39\\{argv}[\T{0}]);{}$\6
${}\\{exit}({-}\T{1});{}$\6
\4${}\}{}$\2\6
\&{if} ${}(\|d<\T{2}\V\|d>\\{maxd}){}$\5
${}\{{}$\1\6
${}\\{fprintf}(\\{stderr},\39\.{"The\ number\ of\ piece}\)\.{s\ should\ be\
between\ }\)\.{2\ and\ \%d,\ not\ \%d!\\n"},\39\\{maxd},\39\|d);{}$\6
${}\\{exit}({-}\T{2});{}$\6
\4${}\}{}$\2\6
${}\\{vbose}\K\\{argc}-\T{2}{}$;\par
\U1.\fi

\M{3}\B\D$\\{place}(\|i,\|j)$ \5
$((\|i)*\\{maxn}+(\|j){}$)\par
\Y\B\4\X3:Input the shape\X${}\E{}$\6
\&{for} ${}(\|i\K\\{nn}\K\T{0};{}$  ; ${}\|i\PP){}$\5
${}\{{}$\1\6
\&{if} ${}(\R\\{fgets}(\\{buf},\39\\{bufsize},\39\\{stdin})){}$\1\5
\&{break};\2\6
\&{if} ${}(\|i\G\\{maxn}){}$\5
${}\{{}$\1\6
${}\\{fprintf}(\\{stderr},\39\.{"Recompile\ me:\ I\ all}\)\.{ow\ at\ most\ \%d\
lines\ }\)\.{of\ input!\\n"},\39\\{maxn});{}$\6
${}\\{exit}({-}\T{3});{}$\6
\4${}\}{}$\2\6
\X4:Input row \PB{\|i} of the shape\X;\6
\4${}\}{}$\2\6
${}\\{maxrow}\K\|i-\T{1};{}$\6
\&{if} ${}(\\{maxrow}<\T{0}){}$\5
${}\{{}$\1\6
${}\\{fprintf}(\\{stderr},\39\.{"There\ was\ no\ input!}\)\.{\\n"});{}$\6
${}\\{exit}({-}\T{666});{}$\6
\4${}\}{}$\2\6
${}\\{fprintf}(\\{stderr},\39\.{"OK,\ I've\ got\ a\ shap}\)\.{e\ with\ \%d\
lines\ and\ }\)\.{\%d\ cells.\\n"},\39\|i,\39\\{nn});{}$\6
\&{for} ${}(\|n\K\T{1};{}$ ${}\|n*\|n<\\{nn};{}$ ${}\|n\PP){}$\1\5
;\C{ the shape has \PB{\\{nn}} asterisks }\2\6
\&{if} ${}(\|n*\|n\I\\{nn}){}$\5
${}\{{}$\1\6
${}\\{fprintf}(\\{stderr},\39\.{"The\ number\ of\ cells}\)\.{\ should\ be\ a\
positiv}\)\.{e\ perfect\ square!\\n"});{}$\6
${}\\{exit}({-}\T{4});{}$\6
\4${}\}{}$\2\6
\&{for} ${}(\|i\K\T{0};{}$ ${}\|i<\|n;{}$ ${}\|i\PP){}$\1\6
\&{for} ${}(\|j\K\T{0};{}$ ${}\|j<\|n;{}$ ${}\|j\PP){}$\1\5
${}\\{sprintf}(\\{aname}[\\{place}(\|i,\39\|j)],\39\.{"\%02da\%02d"},\39\|i,\39%
\|j);{}$\2\2\6
${}\\{complement}\K\\{place}(\|n-\T{1},\39\|n-\T{1}){}$;\par
\U1.\fi

\M{4}\B\X4:Input row \PB{\|i} of the shape\X${}\E{}$\6
\&{for} ${}(\|j\K\T{0};{}$ ${}\\{buf}[\|j]\W\\{buf}[\|j]\I\.{'\\n'};{}$ ${}\|j%
\PP){}$\5
${}\{{}$\1\6
\&{if} ${}(\\{buf}[\|j]\E\.{'*'}){}$\5
${}\{{}$\1\6
\&{if} ${}(\|j>\\{maxcol}){}$\5
${}\{{}$\1\6
${}\\{maxcol}\K\|j;{}$\6
\&{if} ${}(\|j\G\\{maxn}){}$\5
${}\{{}$\1\6
${}\\{fprintf}(\\{stderr},\39\.{"Recompile\ me:\ I\ all}\)\.{ow\ at\ most\ \%d\
column}\)\.{s\ of\ input!\\n"},\39\\{maxn});{}$\6
${}\\{exit}({-}\T{5});{}$\6
\4${}\}{}$\2\6
\4${}\}{}$\2\6
${}\\{site}[\\{nn}\PP]\K\\{place}(\|i,\39\|j);{}$\6
${}\\{sprintf}(\\{bname}[\\{place}(\|i,\39\|j)],\39\.{"\%02db\%02d"},\39\|i,\39%
\|j);{}$\6
\4${}\}{}$\2\6
\4${}\}{}$\2\par
\U3.\fi

\N{1}{5}The algorithm. Let's consider a special case of the problem, in order
to build some intuition and clarify the concepts. Suppose the input is
$$\vcenter{\halign{\tt#\hfil\cr
****\cr *..*\cr .***\cr}}$$
and we want to cut the eight cells specified by these asterisks
into $d=2$ pieces that can be assembled into a $3\times3$ square.
You can probably see one way to do the job: Break off the
two cells in the leftmost column, rotate them $90^\circ$, and
stick them into the ``jaws'' of the remaining seven. How can we
get a computer to discover this?

A solution to the general problem can be regarded as a way to color
the square with $d$ colors. The cells of the square are $(i,j)$ for
$0\le i,j<n$, and each of these cells is supposed to be
mapped into a distinct position $(i',j')$ of the other shape,
by rotation and shifting. The amount of
rotation and shift must be the same for all cells of the same color.
In the example, we can color the cells
$$\vcenter{\halign{\tt#\hfil\cr
111\cr221\cr111\cr}}$$
and map those of color \.1 by shifting one space right; those of
color~\.2 are mapped by, say, rotating $90^\circ$ clockwise
about the square's center, then shifting one space left. The result is
$$\vcenter{\halign{\tt#\hfil\cr
2111\cr 2..1\cr .111\cr}}$$
as desired. These two color labelings are written to \PB{\\{stdout}}.

There's always a set of allowable shift amounts, $(a_0,b_0)$, $(a_1,b_1)$,
\dots, $(a_{m-1},b_{m-1})$; these are the ways to shift the square so that
it overlaps the other shape in at least one cell. Our example problem
has $m=29$ such shifts, namely
$\bar2\bar2$,
$\bar2\bar1$,
$\bar20$,
$\bar21$,
$\bar22$,
$\bar23$,
$\bar1\bar2$,
$\bar1\bar1$,
$\bar10$,
$\bar11$,
$\bar12$,
$\bar13$,
$0\bar2$,
$0\bar1$,
$00$,
$01$,
$02$,
$03$,
$1\bar2$,
$1\bar1$,
$10$,
$11$,
$12$,
$13$,
$2\bar1$,
$20$,
$21$,
$22$,
$23$. (Here $\bar2$ stands for $-2$, and so on; our program uses
row-and-column coordinates $(i,j)$, so the left coordinate of a shift
refers to shifting downward and the
right coordinate refers to shifting rightward.
This list of acceptable shifts includes all values $(a,b)$ with
$-2\le a\le2$ and $-2\le b\le 3$ {\it except\/} for $2\bar2$. The
latter is omitted, because shifting the square down~2 and left~2 does
not intersect with the other shape.)

Each mapping can be specified by a pair $(s,t)$ where $0\le s<m$
and $0\le t<4$, meaning ``rotate $90t$ degrees clockwise, then
shift by $(a_s,b_s)$.'' The outer loop of the algorithm below
runs through all possible mappings $(s_1,t_1)$, \dots, $(s_d,t_d)$,
and tries to solve the corresponding bipartite matching problem
that involves those maps. For example, if $(s_1,t_1)$ is the map
``shift right~1'' and $(s_2,t_2)$ is the map ``rotate~90 and shift
left 1,'' the bipartite graph for which a perfect matching describes
a solution to the example problem has the edges
$$
\.{0a0}\adj\.{0b1},\quad
\.{0a1}\adj\.{0b2},\quad
\.{0a2}\adj\.{0b3},\quad
\.{1a2}\adj\.{1b3},\quad
\.{2a0}\adj\.{2b1},\quad
\.{2a1}\adj\.{2b2},\quad
\.{2a2}\adj\.{2b3}
$$
for color \.1 and
$$
\.{0a0}\adj\.{0b1},\quad
\.{0a2}\adj\.{2b1},\quad
\.{1a0}\adj\.{0b0},\quad
\.{1a1}\adj\.{1b0}
$$
for color \.2. (Here \.{0a0} stands for the cell in row~0 and column~0
of the square, while \.{0b0} stands for the cell in row~0 and column~0
of the other shape.) Notice that the edge $\.{0a0}\adj\.{0b1}$ occurs
{\it twice}, once for each color; this leads to {\it another\/} solution:
$$\vcenter{\halign{\tt#\hfil\cr
211\cr221\cr111\cr}}\qquad
\vcenter{\halign{\tt#\hfil\cr
2211\cr 2..1\cr .111\cr}}$$

\fi

\M{6}We save a factor of roughly $d!$ by assuming that
$$(s_1,t_1)\le (s_2,t_2)\le\cdots\le (s_d,t_d),\qquad\rm lexicographically,$$
because a permutation of the colors doesn't
change the solution. Furthermore we gain another factor of~4
by assuming that $t_1=0$, because rotation is a symmetry of the square.

(I could actually have written $(s_1,t_1)<\cdots<(s_d,t_d)$,
with `$<$' instead of~`$\le$';
the case $(s_k,t_k)=(s_{k+1},t_{k+1})$ won't occur in a solution
for minimum~$d$, because colors $k$ and $k+1$ could be merged in such
a case. However, equality might arise in
extensions of this problem that involve further constraints.
For example, we might
require color classes to be connected, or to have a bounded size.)

Most of the matching problems that arise are obviously unsolvable,
because they have isolated vertices. And most of those that
remain are quite easy to solve, because many vertices have degree~1
and their partner is forced. The algorithm looks at all $m+d-1\choose d$
sets of shifts with $s_1\le\cdots\le s_d$, and explores further only if those
shifts cover all cells of the given shape. In the latter case,
$4^{d-1}$ choices of $t_2$, \dots,~$t_d$ are considered, and the
matching process is inaugurated only if those rotations cover
all cells of the square.

For example, the only shifts that cover more than four cells of the shape
in our toy problem are 00, 01, and 02. At least one of these is needed, because
we need to cover nine cells with two shifts. Thus $m+d-1\choose d$ is
not a scary number of subproblems to consider.

\Y\B\4\X6:Find all solutions\X${}\E{}$\6
\X8:Generate the table of legal shifts\X;\6
\&{while} (\T{1})\5
${}\{{}$\1\6
\X9:If the shape isn't covered by $\{s_1,\ldots,s_d\}$, \PB{\&{goto} %
\\{shapenot}}\X;\6
${}\\{counta}\PP;{}$\6
\X7:Run through all sequences of shifts, $(t_2,\ldots,t_d)$\X;\6
\4\\{shapenot}:\5
\&{for} ${}(\|k\K\|d;{}$ ${}\|s[\|k]\E\|m-\T{1};{}$ ${}\|k\MM){}$\1\5
;\2\6
\&{if} ${}(\|k\E\T{0}){}$\1\5
\&{break};\2\6
\&{for} ${}(\|j\K\|s[\|k]+\T{1};{}$ ${}\|k\Z\|d;{}$ ${}\|k\PP){}$\1\5
${}\|s[\|k]\K\|j;{}$\2\6
\4${}\}{}$\2\par
\U1.\fi

\M{7}\B\X7:Run through all sequences of shifts, $(t_2,\ldots,t_d)$\X${}\E{}$\6
\&{for} ${}(\|k\K\T{2};{}$ ${}\|k\Z\|d;{}$ ${}\|k\PP){}$\1\5
${}\|t[\|k]\K\T{0};{}$\2\6
\&{while} (\T{1})\5
${}\{{}$\1\6
\&{for} ${}(\|k\K\T{2};{}$ ${}\|k\Z\|d;{}$ ${}\|k\PP){}$\1\6
\&{if} ${}(\|s[\|k]\E\|s[\|k-\T{1}]\W\|t[\|k]\E\|t[\|k-\T{1}]){}$\1\5
\&{goto} \\{squarenot};\2\2\6
\X10:If the square isn't covered by $\{(s_1,t_1),\ldots,(s_d,t_d)\}$, \PB{%
\&{goto} \\{squarenot}}\X;\6
${}\\{countb}\PP;{}$\6
\X12:Check for a perfect matching\X;\6
\4\\{squarenot}:\5
\&{for} ${}(\|k\K\|d;{}$ ${}\|t[\|k]\E\T{3};{}$ ${}\|k\MM){}$\1\5
${}\|t[\|k]\K\T{0};{}$\2\6
\&{if} ${}(\|k\E\T{1}){}$\1\5
\&{break};\2\6
${}\|t[\|k]\PP;{}$\6
\4${}\}{}$\2\par
\U6.\fi

\M{8}\B\X8:Generate the table of legal shifts\X${}\E{}$\6
\&{for} ${}(\|m\K\T{0},\39\|a\K\T{1}-\|n;{}$ ${}\|a\Z\\{maxrow};{}$ ${}\|a%
\PP){}$\1\6
\&{for} ${}(\|b\K\T{1}-\|n;{}$ ${}\|b\Z\\{maxcol};{}$ ${}\|b\PP){}$\5
${}\{{}$\1\6
\&{for} ${}(\|k\K\T{0},\39\|i\K(\|a<\T{0}\?{-}\|a:\T{0});{}$ ${}\|i<\|n\W\|a+%
\|i\Z\\{maxrow};{}$ ${}\|i\PP){}$\1\6
\&{for} ${}(\|j\K(\|b<\T{0}\?{-}\|b:\T{0});{}$ ${}\|j<\|n\W\|b+\|j\Z%
\\{maxcol};{}$ ${}\|j\PP){}$\1\6
\&{if} ${}(\\{bname}[\\{place}(\|a+\|i,\39\|b+\|j)][\T{0}]){}$\1\5
${}\\{bcover}[\|m][\|k\PP]\K\\{place}(\|a+\|i,\39\|b+\|j);{}$\2\2\2\6
\&{if} (\|k)\5
${}\{{}$\1\6
\&{if} ${}(\\{vbose}>\T{1}){}$\1\5
${}\\{fprintf}(\\{stderr},\39\.{"\ S[\%d]=(\%d,\%d)\\n"},\39\|m,\39\|a,\39%
\|b);{}$\2\6
${}\\{shift}[\|m]\K\\{place}(\|a,\39\|b),\39\\{bcovered}[\|m\PP]\K\|k;{}$\6
\4${}\}{}$\2\6
\4${}\}{}$\2\2\6
\&{if} (\\{vbose})\1\5
${}\\{fprintf}(\\{stderr},\39\.{"There\ are\ \%d\ legal\ }\)\.{shifts.\\n"},\39%
\|m){}$;\2\par
\U6.\fi

\M{9}\B\X9:If the shape isn't covered by $\{s_1,\ldots,s_d\}$, \PB{\&{goto} %
\\{shapenot}}\X${}\E{}$\6
\&{for} ${}(\\{slack}\K{-}\\{nn},\39\|k\K\T{1};{}$ ${}\|k\Z\|d;{}$ ${}\|k%
\PP){}$\1\5
${}\\{slack}\MRL{+{\K}}\\{bcovered}[\|s[\|k]];{}$\2\6
\&{if} ${}(\\{slack}<\T{0}){}$\1\5
\&{goto} \\{shapenot};\2\6
\&{for} ${}(\|k\K\T{0};{}$ ${}\|k<\\{nn};{}$ ${}\|k\PP){}$\1\5
${}\\{blen}[\\{site}[\|k]]\K\T{0};{}$\2\6
\&{for} ${}(\|k\K\T{1};{}$ ${}\|k\Z\|d;{}$ ${}\|k\PP){}$\5
${}\{{}$\1\6
\&{for} ${}(\|i\K\T{0},\39\|j\K\|s[\|k];{}$ ${}\|i<\\{bcovered}[\|j];{}$ ${}\|i%
\PP){}$\5
${}\{{}$\1\6
${}\|l\K\\{bcover}[\|j][\|i];{}$\6
\&{if} ${}(\R\\{blen}[\|l]){}$\1\5
${}\\{blen}[\|l]\K\T{1};{}$\2\6
\&{else}\5
${}\{{}$\1\6
\&{if} ${}(\R\\{slack}){}$\1\5
\&{goto} \\{shapenot};\2\6
${}\\{slack}\MM;{}$\6
${}\\{blen}[\|l]\PP;{}$\6
\4${}\}{}$\2\6
\4${}\}{}$\2\6
\4${}\}{}$\2\par
\U6.\fi

\M{10}While we make the second check for coverage, we also build the table of
edges.
Each edge is represented by its color and the position of the neighbor.

\Y\B\4\D$\\{pack}(\|c,\|p)$ \5
$(((\|c)\LL\T{16})+(\|p){}$)\par
\Y\B\4\X10:If the square isn't covered by $\{(s_1,t_1),\ldots,(s_d,t_d)\}$, %
\PB{\&{goto} \\{squarenot}}\X${}\E{}$\6
\&{for} ${}(\|k\K\T{0};{}$ ${}\|k<\\{nn};{}$ ${}\|k\PP){}$\1\5
${}\\{blen}[\\{site}[\|k]]\K\T{0};{}$\2\6
\&{for} ${}(\|i\K\T{0};{}$ ${}\|i<\|n;{}$ ${}\|i\PP){}$\1\6
\&{for} ${}(\|j\K\T{0};{}$ ${}\|j<\|n;{}$ ${}\|j\PP){}$\1\5
${}\\{alen}[\\{place}(\|i,\39\|j)]\K\T{0};{}$\2\2\6
\&{for} ${}(\\{slack}\K{-}\\{nn},\39\|k\K\T{1};{}$ ${}\|k\Z\|d;{}$ ${}\|k%
\PP){}$\1\5
${}\\{slack}\MRL{+{\K}}\\{bcovered}[\|s[\|k]];{}$\2\6
\&{for} ${}(\|k\K\T{1};{}$ ${}\|k\Z\|d;{}$ ${}\|k\PP){}$\5
${}\{{}$\1\6
\&{for} ${}(\|i\K\T{0},\39\|j\K\|s[\|k];{}$ ${}\|i<\\{bcovered}[\|j];{}$ ${}\|i%
\PP){}$\5
${}\{{}$\1\6
${}\|l\K\\{bcover}[\|j][\|i];{}$\6
${}\\{ll}\K\|l-\\{shift}[\|j];{}$\6
\&{if} ${}(\|t[\|k]\AND\T{1}){}$\5
${}\{{}$\1\6
\&{register} \&{int} \|q${}\K\\{ll}/\\{maxn},{}$ \|r${}\K\\{ll}\MOD\\{maxn};{}$%
\7
${}\\{ll}\K\\{place}(\|r,\39\|n-\T{1}-\|q){}$;\C{ rotate clockwise }\6
\4${}\}{}$\2\6
\&{if} ${}(\|t[\|k]\AND\T{2}){}$\1\5
${}\\{ll}\K\\{complement}-\\{ll};{}$\2\6
\&{if} (\\{alen}[\\{ll}])\5
${}\{{}$\1\6
\&{if} ${}(\R\\{slack}){}$\1\5
\&{goto} \\{squarenot};\2\6
${}\\{slack}\MM;{}$\6
\4${}\}{}$\2\6
${}\\{aa}[\\{ll}][\\{alen}[\\{ll}]\PP]\K\\{pack}(\|k,\39\|l);{}$\6
${}\\{bb}[\|l][\\{blen}[\|l]\PP]\K\\{pack}(\|k,\39\\{ll});{}$\6
\4${}\}{}$\2\6
\4${}\}{}$\2\par
\U7.\fi

\M{11}\B\X11:Global variables\X${}\E{}$\6
\&{char} ${}\\{alen}[\\{maxn}*\\{maxn}]{}$;\C{ how many moves remain at this
cell in the square }\6
\&{char} ${}\\{blen}[\\{maxn}*\\{maxn}]{}$;\C{ how many moves remain at this
cell in the shape }\6
\&{int} ${}\\{aa}[\\{maxn}*\\{maxn}][\\{maxd}]{}$;\C{ moves for the square }\6
\&{int} ${}\\{bb}[\\{maxn}*\\{maxn}][\\{maxd}]{}$;\C{ moves for the shape }\6
\&{int} ${}\\{shift}[\T{4}*\\{maxn}*\\{maxn}]{}$;\C{ offsets in the shifts }\6
\&{int} \\{complement};\C{ offset used for 180-degree rotation }\6
\&{int} ${}\\{bcover}[\T{4}*\\{maxn}*\\{maxn}][\\{maxn}*\\{maxn}]{}$;\C{ cells
covered by the shifts }\6
\&{int} ${}\\{bcovered}[\T{4}*\\{maxn}*\\{maxn}]{}$;\C{ how many cells are
covered }\6
\&{int} ${}\|s[\\{maxd}+\T{1}]{}$;\C{ the current sequence of shifts }\6
\&{int} ${}\|t[\\{maxd}+\T{1}]{}$;\C{ the current sequence of rotations }\par
\As22\ET28.
\U1.\fi

\N{1}{12}Prematching.
When we've managed to jump through all those hoops, we're left with
a perfect matching problem. And most of the time that matching problem
is quite trivial; so we might as well throw out the easy cases before
trying to do anything fancy.

In most cases some of the moves turn out to be forced, because a
cell of the square has only one possible shape-mate or vice versa.
We start by making all of those no-brainer moves.

\Y\B\4\X12:Check for a perfect matching\X${}\E{}$\6
\&{if} ${}(\\{vbose}>\T{1}){}$\1\5
\X13:Display the matching problem on \PB{\\{stderr}}\X;\2\6
\X14:Make forced moves from the square, or \PB{\&{goto} \\{done}}\X;\6
${}\\{countc}\PP;{}$\6
\X16:Make forced moves from the shape, or \PB{\&{goto} \\{done}}\X;\6
\X18:Make all remaining forced moves\X;\6
${}\\{countd}\PP;{}$\6
\X23:Find all perfect matchings in the remaining bigraph\X;\6
\\{done}:\par
\U7.\fi

\M{13}\B\X13:Display the matching problem on \PB{\\{stderr}}\X${}\E{}$\6
${}\{{}$\1\6
${}\\{fprintf}(\\{stderr},\39\.{"\ Trying\ to\ match"});{}$\6
\&{for} ${}(\|k\K\T{1};{}$ ${}\|k\Z\|d;{}$ ${}\|k\PP){}$\1\5
${}\\{fprintf}(\\{stderr},\39\.{"\ \%d\^\%d"},\39\|s[\|k],\39\|t[\|k]);{}$\2\6
${}\\{fprintf}(\\{stderr},\39\.{":\\n"});{}$\6
\&{for} ${}(\|i\K\T{0};{}$ ${}\|i<\|n;{}$ ${}\|i\PP){}$\1\6
\&{for} ${}(\|j\K\T{0};{}$ ${}\|j<\|n;{}$ ${}\|j\PP){}$\5
${}\{{}$\1\6
${}\\{fprintf}(\\{stderr},\39\.{"\ \ \%s\ --"},\39\\{aname}[\\{place}(\|i,\39%
\|j)]);{}$\6
\&{for} ${}(\|k\K\T{0};{}$ ${}\|k<\\{alen}[\\{place}(\|i,\39\|j)];{}$ ${}\|k%
\PP){}$\1\5
${}\\{fprintf}(\\{stderr},\39\.{"\ \%s.\%d"},\39\\{bname}[\\{aa}[\\{place}(\|i,%
\39\|j)][\|k]\AND\T{\^ffff}],\39\\{aa}[\\{place}(\|i,\39\|j)][\|k]\GG%
\T{16});{}$\2\6
${}\\{fprintf}(\\{stderr},\39\.{"\\n"});{}$\6
\4${}\}{}$\2\2\6
\4${}\}{}$\2\par
\U12.\fi

\M{14}\B\X14:Make forced moves from the square, or \PB{\&{goto} \\{done}}\X${}%
\E{}$\6
\&{for} ${}(\\{acount}\K\|i\K\T{0};{}$ ${}\|i<\|n;{}$ ${}\|i\PP){}$\1\6
\&{for} ${}(\|j\K\T{0};{}$ ${}\|j<\|n;{}$ ${}\|j\PP){}$\5
${}\{{}$\1\6
\&{if} ${}(\\{alen}[\\{place}(\|i,\39\|j)]>\T{1}){}$\1\5
${}\\{apos}[\\{place}(\|i,\39\|j)]\K\\{acount},\39\\{alist}[\\{acount}\PP]\K%
\\{place}(\|i,\39\|j);{}$\2\6
\&{else}\5
${}\{{}$\1\6
${}\|l\K\\{aa}[\\{place}(\|i,\39\|j)][\T{0}]\AND\T{\^ffff};{}$\6
\&{if} ${}(\R\\{blen}[\|l]){}$\1\5
\&{goto} \\{done};\C{ that position of the shape is already taken }\2\6
${}\\{acolor}[\\{place}(\|i,\39\|j)]\K\\{bcolor}[\|l]\K\\{aa}[\\{place}(\|i,\39%
\|j)][\T{0}]\GG\T{16};{}$\6
\&{if} ${}(\\{blen}[\|l]\E\T{1}){}$\1\5
${}\\{blen}[\|l]\K\T{0};{}$\2\6
\&{else}\1\5
\X15:Remove all other edges that go to shape position \PB{\|l}\X;\2\6
\4${}\}{}$\2\6
\4${}\}{}$\2\2\par
\U12.\fi

\M{15}Premature optimization is the root of all evil in programming.
Yet I couldn't resist trying to make this program efficient in special cases.

The removal of edges might reduce \PB{\\{alen}} to 1 for square cells that are
in the \PB{\\{alist}}, thus forcing further moves.
I won't worry about that until later.

\Y\B\4\X15:Remove all other edges that go to shape position \PB{\|l}\X${}\E{}$\6
${}\{{}$\1\6
\&{for} ${}(\|k\K\T{0};{}$ ${}\|k<\\{blen}[\|l];{}$ ${}\|k\PP){}$\5
${}\{{}$\1\6
${}\\{ll}\K\\{bb}[\|l][\|k]\AND\T{\^ffff};{}$\6
\&{if} ${}(\\{ll}\I\\{place}(\|i,\39\|j)){}$\5
${}\{{}$\1\6
\&{register} \&{int} \\{opp}${}\K(\\{bb}[\|l][\|k]\AND\T{\^ffff0000})+\|l{}$;%
\C{ the opposite version of this edge }\7
${}\\{dd}\K\\{alen}[\\{ll}]-\T{1},\39\\{alen}[\\{ll}]\K\\{dd};{}$\6
\&{if} ${}(\R\\{dd}){}$\1\5
\&{goto} \\{done};\2\6
\&{for} ${}(\|a\K\T{0};{}$ ${}\\{aa}[\\{ll}][\|a]\I\\{opp};{}$ ${}\|a\PP){}$\1\5
;\2\6
\&{if} ${}(\|a>\\{dd}){}$\1\5
\\{debug}(\.{"ahi"});\2\6
\&{if} ${}(\|a\I\\{dd}){}$\1\5
${}\\{aa}[\\{ll}][\|a]\K\\{aa}[\\{ll}][\\{dd}];{}$\2\6
\4${}\}{}$\2\6
\4${}\}{}$\2\6
${}\\{blen}[\|l]\K\T{0};{}$\6
\4${}\}{}$\2\par
\U14.\fi

\M{16}\B\X16:Make forced moves from the shape, or \PB{\&{goto} \\{done}}\X${}%
\E{}$\6
\&{if} (\\{acount})\5
${}\{{}$\1\6
\&{for} ${}(\\{bcount}\K\|i\K\T{0};{}$ ${}\|i<\\{nn};{}$ ${}\|i\PP){}$\5
${}\{{}$\1\6
${}\|l\K\\{site}[\|i];{}$\6
\&{if} ${}(\R\\{blen}[\|l]){}$\1\5
\&{continue};\C{ we've been forced to match this cell already }\2\6
\&{if} ${}(\\{blen}[\|l]>\T{1}){}$\1\5
${}\\{bpos}[\|l]\K\\{bcount},\39\\{blist}[\\{bcount}\PP]\K\|l;{}$\2\6
\&{else}\5
${}\{{}$\1\6
${}\\{ll}\K\\{bb}[\|l][\T{0}]\AND\T{\^ffff};{}$\6
\&{if} ${}(\R\\{alen}[\\{ll}]){}$\1\5
\&{goto} \\{done};\C{ that position of the square is already taken }\2\6
${}\\{acolor}[\\{ll}]\K\\{bcolor}[\|l]\K\\{bb}[\|l][\T{0}]\GG\T{16};{}$\6
${}\\{acount}\MM;{}$\6
\X17:Make square cell \PB{\\{ll}} inactive\X;\6
\4${}\}{}$\2\6
\4${}\}{}$\2\6
\&{if} ${}(\\{acount}\I\\{bcount}){}$\1\5
\\{debug}(\.{"count\ mismatch"});\2\6
\4${}\}{}$\2\par
\U12.\fi

\M{17}\B\X17:Make square cell \PB{\\{ll}} inactive\X${}\E{}$\6
$\|j\K\\{apos}[\\{ll}];{}$\6
\&{if} ${}(\|j\I\\{acount}){}$\1\5
${}\\{lll}\K\\{alist}[\\{acount}],\39\\{alist}[\|j]\K\\{lll},\39\\{apos}[%
\\{lll}]\K\|j;{}$\2\6
\&{if} ${}(\\{alen}[\\{ll}]\I\T{1}){}$\5
${}\{{}$\1\6
\&{for} ${}(\|k\K\T{0};{}$ ${}\|k<\\{alen}[\\{ll}];{}$ ${}\|k\PP){}$\5
${}\{{}$\1\6
${}\\{lll}\K\\{aa}[\\{ll}][\|k]\AND\T{\^ffff};{}$\6
\&{if} ${}(\\{lll}\I\|l){}$\5
${}\{{}$\1\6
\&{register} \&{int} \\{opp}${}\K(\\{aa}[\\{ll}][\|k]\AND\T{\^ffff0000})+%
\\{ll}{}$;\C{ the opposite version of this edge }\7
${}\\{dd}\K\\{blen}[\\{lll}]-\T{1},\39\\{blen}[\\{lll}]\K\\{dd};{}$\6
\&{if} ${}(\R\\{dd}){}$\1\5
\&{goto} \\{done};\2\6
\&{for} ${}(\|b\K\T{0};{}$ ${}\\{bb}[\\{lll}][\|b]\I\\{opp};{}$ ${}\|b\PP){}$\1%
\5
;\2\6
\&{if} ${}(\|b>\\{dd}){}$\1\5
\\{debug}(\.{"bhi"});\2\6
\&{if} ${}(\|b\I\\{dd}){}$\1\5
${}\\{bb}[\\{lll}][\|b]\K\\{bb}[\\{lll}][\\{dd}];{}$\2\6
\4${}\}{}$\2\6
\4${}\}{}$\2\6
${}\\{alen}[\\{ll}]\K\T{0};{}$\6
\4${}\}{}$\2\par
\Us16\ET21.\fi

\M{18}Beware: I'm using \PB{\\{acount}} and \PB{\\{bcount}} in a somewhat
tricky way here:
The old \PB{\\{acount}} is kept in \PB{\\{bcount}} so that a change can be
detected.
(Again I apologize for weak resistance.)

\Y\B\4\X18:Make all remaining forced moves\X${}\E{}$\6
\&{while} (\\{acount})\5
${}\{{}$\1\6
\&{for} ${}(\|i\K\T{0};{}$ ${}\|i<\\{acount};{}$ ${}\|i\PP){}$\1\6
\&{if} ${}(\\{alen}[\\{ll}\K\\{alist}[\|i]]\E\T{1}){}$\1\5
\X19:Force a move from \PB{\\{ll}}\X;\2\2\6
\&{for} ${}(\|i\K\T{0};{}$ ${}\|i<\\{acount};{}$ ${}\|i\PP){}$\1\6
\&{if} ${}(\\{blen}[\|l\K\\{blist}[\|i]]\E\T{1}){}$\1\5
\X21:Force a move from \PB{\|l}\X;\2\2\6
\&{if} ${}(\\{acount}\E\\{bcount}){}$\1\5
\&{break};\2\6
${}\\{bcount}\K\\{acount};{}$\6
\4${}\}{}$\2\par
\U12.\fi

\M{19}\B\X19:Force a move from \PB{\\{ll}}\X${}\E{}$\6
${}\{{}$\1\6
${}\\{acount}\MM;{}$\6
\&{if} ${}(\|i<\\{acount}){}$\1\5
${}\\{lll}\K\\{alist}[\\{acount}],\39\\{alist}[\|i]\K\\{lll},\39\\{apos}[%
\\{lll}]\K\|i\MM;{}$\2\6
${}\|l\K\\{aa}[\\{ll}][\T{0}]\AND\T{\^ffff};{}$\6
${}\\{acolor}[\\{ll}]\K\\{bcolor}[\|l]\K\\{aa}[\\{ll}][\T{0}]\GG\T{16};{}$\6
\X20:Make shape cell \PB{\|l} inactive\X;\6
\4${}\}{}$\2\par
\U18.\fi

\M{20}\B\X20:Make shape cell \PB{\|l} inactive\X${}\E{}$\6
$\|j\K\\{bpos}[\|l];{}$\6
\&{if} ${}(\|j<\\{acount}){}$\1\5
${}\\{lll}\K\\{blist}[\\{acount}],\39\\{blist}[\|j]\K\\{lll},\39\\{bpos}[%
\\{lll}]\K\|j;{}$\2\6
\&{if} ${}(\\{blen}[\|l]\I\T{1}){}$\5
${}\{{}$\1\6
\&{for} ${}(\|k\K\T{0};{}$ ${}\|k<\\{blen}[\|l];{}$ ${}\|k\PP){}$\5
${}\{{}$\1\6
${}\\{lll}\K\\{bb}[\|l][\|k]\AND\T{\^ffff};{}$\6
\&{if} ${}(\\{lll}\I\\{ll}){}$\5
${}\{{}$\1\6
\&{register} \&{int} \\{opp}${}\K(\\{bb}[\|l][\|k]\AND\T{\^ffff0000})+\|l{}$;%
\C{ the opposite version of this edge }\7
${}\\{dd}\K\\{alen}[\\{lll}]-\T{1},\39\\{alen}[\\{lll}]\K\\{dd};{}$\6
\&{if} ${}(\R\\{dd}){}$\1\5
\&{goto} \\{done};\2\6
\&{for} ${}(\|a\K\T{0};{}$ ${}\\{aa}[\\{lll}][\|a]\I\\{opp};{}$ ${}\|a\PP){}$\1%
\5
;\2\6
\&{if} ${}(\|a>\\{dd}){}$\1\5
\\{debug}(\.{"chi"});\2\6
\&{if} ${}(\|a\I\\{dd}){}$\1\5
${}\\{aa}[\\{lll}][\|a]\K\\{aa}[\\{lll}][\\{dd}];{}$\2\6
\4${}\}{}$\2\6
\4${}\}{}$\2\6
\4${}\}{}$\2\par
\U19.\fi

\M{21}\B\X21:Force a move from \PB{\|l}\X${}\E{}$\6
${}\{{}$\1\6
${}\\{acount}\MM;{}$\6
\&{if} ${}(\|i<\\{acount}){}$\1\5
${}\\{lll}\K\\{blist}[\\{acount}],\39\\{blist}[\|i]\K\\{lll},\39\\{bpos}[%
\\{lll}]\K\|i\MM;{}$\2\6
${}\\{ll}\K\\{bb}[\|l][\T{0}]\AND\T{\^ffff};{}$\6
${}\\{acolor}[\\{ll}]\K\\{bcolor}[\|l]\K\\{bb}[\|l][\T{0}]\GG\T{16};{}$\6
\X17:Make square cell \PB{\\{ll}} inactive\X;\6
\4${}\}{}$\2\par
\U18.\fi

\M{22}\B\X11:Global variables\X${}\mathrel+\E{}$\6
\&{int} ${}\\{alist}[\\{maxn}*\\{maxn}],{}$ ${}\\{blist}[\\{maxn}*\\{maxn}]{}$;%
\C{ list of cells not yet matched }\6
\&{int} ${}\\{apos}[\\{maxn}*\\{maxn}],{}$ ${}\\{bpos}[\\{maxn}*\\{maxn}]{}$;%
\C{ inverses of those lists }\6
\&{int} \\{acount}${},{}$ \\{bcount};\C{ the lengths of those lists }\6
\&{int} ${}\\{acolor}[\\{maxn}*\\{maxn}],{}$ ${}\\{bcolor}[\\{maxn}*%
\\{maxn}]{}$;\C{ color patterns in a solution }\6
\&{unsigned} \&{long} \&{long} \\{count};\C{ the number of solutions }\6
\&{unsigned} \&{long} \&{long} \\{counta}${},{}$ \\{countb}${},{}$ %
\\{countc}${},{}$ \\{countd}${},{}$ \\{counte};\C{ the number of times we
reached key points }\par
\fi

\N{1}{23}Matching. Sometimes we actually have real work to do.

At first I didn't think the problem would often be challenging.
So I just used brute-force backtracking, \`a~la Algorithm 7.2.2B.

But a surprising number of large subproblems arose. So I'm now implementing
a version of the original dancing links algorithm,
hacked from {\mc DANCE}.

\Y\B\4\X23:Find all perfect matchings in the remaining bigraph\X${}\E{}$\6
\&{if} ${}(\\{acount}\E\T{0}){}$\1\5
\X40:Print a solution\X\2\6
\&{else}\5
${}\{{}$\1\6
\X27:Local variables\X;\6
${}\\{counte}\PP;{}$\6
\&{if} ${}(\\{vbose}>\T{1}){}$\1\5
\X24:Display the remaining matching problem on \PB{\\{stderr}}\X;\2\6
\X29:Initialize for dancing\X;\6
\X31:Dance\X;\6
\4${}\}{}$\2\par
\U12.\fi

\M{24}\B\X24:Display the remaining matching problem on \PB{\\{stderr}}\X${}%
\E{}$\6
${}\{{}$\1\6
${}\\{fprintf}(\\{stderr},\39\.{"\ which\ reduces\ to:\\}\)\.{n"});{}$\6
\&{for} ${}(\|i\K\T{0};{}$ ${}\|i<\\{acount};{}$ ${}\|i\PP){}$\5
${}\{{}$\1\6
${}\\{fprintf}(\\{stderr},\39\.{"\ \ \%s\ --"},\39\\{aname}[\\{alist}[%
\|i]]);{}$\6
\&{for} ${}(\|k\K\T{0};{}$ ${}\|k<\\{alen}[\\{alist}[\|i]];{}$ ${}\|k\PP){}$\1\5
${}\\{fprintf}(\\{stderr},\39\.{"\ \%s.\%d"},\39\\{bname}[\\{aa}[\\{alist}[%
\|i]][\|k]\AND\T{\^ffff}],\39\\{aa}[\\{alist}[\|i]][\|k]\GG\T{16});{}$\2\6
${}\\{fprintf}(\\{stderr},\39\.{"\\n"});{}$\6
\4${}\}{}$\2\6
\4${}\}{}$\2\par
\U23.\fi

\M{25}The {\mc DANCE} program was developed to solve exact cover problems,
and bipartite matching is a particularly easy case of that problem:
Every column to be covered is a primary column, and every row specifies
exactly two primary columns.

Each column of the exact cover matrix is represented by a \&{column} struct,
and each row is represented as a linked list of \&{node} structs. There's one
node for each nonzero entry in the matrix.

More precisely, the nodes are linked circularly within each row, in
both directions. The nodes are also linked circularly within each column;
the column lists each include a header node, but the row lists do not.
Column header nodes are part of a \&{column} struct, which
contains further info about the column.

Each node contains six fields. Four are the pointers of doubly linked lists,
already mentioned; the fifth points to the column containing the node;
the sixth ties this node to the dissection problem we're solving.

\Y\B\4\X25:Type definitions\X${}\E{}$\6
\&{typedef} \&{struct} \&{node\_struct} ${}\{{}$\1\6
\&{struct} \&{node\_struct} ${}{*}\\{left},{}$ ${}{*}\\{right}{}$;\C{
predecessor and successor in row }\6
\&{struct} \&{node\_struct} ${}{*}\\{up},{}$ ${}{*}\\{down}{}$;\C{ predecessor
and successor in column }\6
\&{struct} \&{col\_struct} ${}{*}\\{col}{}$;\C{ the column containing this node
}\6
\&{int} \\{info};\C{ square position, shape position, and color of this edge }%
\2\6
${}\}{}$ \&{node};\par
\A26.
\U1.\fi

\M{26}Each \&{column} struct contains five fields:
The \PB{\\{head}} is a node that stands at the head of its list of nodes;
the \PB{\\{len}} tells the length of that list of nodes, not counting the
header;
the \PB{\\{name}} is a user-specified identifier;
\PB{\\{next}} and \PB{\\{prev}} point to adjacent columns, when this
column is part of a doubly linked list.

As backtracking proceeds, nodes
will be deleted from column lists when their row has been blocked by
other rows in the partial solution.
But when backtracking is complete, the data structures will be
restored to their original state.

\Y\B\4\X25:Type definitions\X${}\mathrel+\E{}$\6
\&{typedef} \&{struct} \&{col\_struct} ${}\{{}$\1\6
\&{node} \\{head};\C{ the list header }\6
\&{int} \\{len};\C{ the number of non-header items currently in this column's
list }\6
\&{char} ${}{*}\\{name}{}$;\C{ symbolic identification of the column, for
printing }\6
\&{struct} \&{col\_struct} ${}{*}\\{prev},{}$ ${}{*}\\{next}{}$;\C{ neighbors
of this column }\2\6
${}\}{}$ \&{column};\par
\fi

\M{27}One \PB{\&{column}} struct is called the root. It serves as the head of
the
list of columns that need to be covered, and is identifiable by the fact
that its \PB{\\{name}} is empty.

\Y\B\4\D$\\{root}$ \5
\\{col\_array}[\T{0}]\C{ gateway to the unsettled columns }\par
\Y\B\4\X27:Local variables\X${}\E{}$\6
\&{register} \&{column} ${}{*}\\{cur\_col};{}$\6
\&{register} \&{node} ${}{*}\\{cur\_node}{}$;\par
\As32\ET38.
\U23.\fi

\M{28}\B\D$\\{max\_cols}$ \5
$(\T{2}*\\{maxn}*\\{maxn}{}$)\par
\B\4\D$\\{max\_nodes}$ \5
$(\\{maxn}*\\{maxn}*\\{maxn}*\\{maxn}*\\{maxd}{}$)\par
\Y\B\4\X11:Global variables\X${}\mathrel+\E{}$\6
\&{column} ${}\\{col\_array}[\\{max\_cols}+\T{2}]{}$;\C{ place for column
records }\6
\&{node} \\{node\_array}[\\{max\_nodes}];\C{ place for nodes }\6
\&{column} ${}{*}\\{acol}[\\{maxn}*\\{maxn}],{}$ ${}{*}\\{bcol}[\\{maxn}*%
\\{maxn}];{}$\6
\&{node} ${}{*}\\{choice}[\\{maxn}*\\{maxn}]{}$;\C{ the row and column chosen
on each level }\par
\fi

\M{29}\B\X29:Initialize for dancing\X${}\E{}$\6
\&{for} ${}(\|i\K\T{0};{}$ ${}\|i<\\{acount};{}$ ${}\|i\PP){}$\5
${}\{{}$\1\6
${}\\{ll}\K\\{alist}[\|i],\39\|l\K\\{blist}[\|i];{}$\6
${}\\{acol}[\\{ll}]\K{\AND}\\{col\_array}[\|i+\|i+\T{1}],\39\\{col\_array}[\|i+%
\|i+\T{1}].\\{name}\K\\{aname}[\\{ll}];{}$\6
${}\\{bcol}[\|l]\K{\AND}\\{col\_array}[\|i+\|i+\T{2}],\39\\{col\_array}[\|i+%
\|i+\T{2}].\\{name}\K\\{bname}[\|l];{}$\6
\4${}\}{}$\2\6
${}\\{root}.\\{prev}\K{\AND}\\{col\_array}[\\{acount}+\\{acount}];{}$\6
${}\\{root}.\\{prev}\MG\\{next}\K{\AND}\\{root};{}$\6
\&{for} ${}(\\{cur\_col}\K\\{col\_array}+\T{1};{}$ ${}\\{cur\_col}\Z\\{root}.%
\\{prev};{}$ ${}\\{cur\_col}\PP){}$\5
${}\{{}$\1\6
${}\\{cur\_col}\MG\\{head}.\\{up}\K\\{cur\_col}\MG\\{head}.\\{down}\K{\AND}%
\\{cur\_col}\MG\\{head};{}$\6
${}\\{cur\_col}\MG\\{len}\K\T{0};{}$\6
${}\\{cur\_col}\MG\\{prev}\K\\{cur\_col}-\T{1},\39(\\{cur\_col}-\T{1})\MG%
\\{next}\K\\{cur\_col};{}$\6
\4${}\}{}$\2\6
\&{for} ${}(\\{cur\_node}\K\\{node\_array},\39\|i\K\T{0};{}$ ${}\|i<%
\\{acount};{}$ ${}\|i\PP){}$\5
${}\{{}$\1\6
${}\\{ll}\K\\{alist}[\|i];{}$\6
\&{for} ${}(\|k\K\T{0};{}$ ${}\|k<\\{alen}[\\{ll}];{}$ ${}\|k\PP){}$\1\5
\X30:Create the node for the \PB{\|k}th edge from \PB{\\{ll}}\X;\2\6
\4${}\}{}$\2\par
\U23.\fi

\M{30}\B\X30:Create the node for the \PB{\|k}th edge from \PB{\\{ll}}\X${}\E{}$%
\6
${}\{{}$\1\6
\&{register} \&{column} ${}{*}\\{ccol};{}$\7
${}\|l\K\\{aa}[\\{ll}][\|k]\AND\T{\^ffff};{}$\6
${}\|j\K((\\{aa}[\\{ll}][\|k]\GG\T{16})\LL\T{24})+(\|l\LL\T{12})+\\{ll};{}$\6
${}\\{ccol}\K\\{acol}[\\{ll}];{}$\6
${}\\{cur\_node}\MG\\{left}\K\\{cur\_node}\MG\\{right}\K\\{cur\_node}+\T{1};{}$%
\6
${}\\{cur\_node}\MG\\{col}\K\\{ccol},\39\\{cur\_node}\MG\\{info}\K\|j;{}$\6
${}\\{cur\_node}\MG\\{up}\K\\{ccol}\MG\\{head}.\\{up},\39\\{ccol}\MG\\{head}.%
\\{up}\MG\\{down}\K\\{cur\_node};{}$\6
${}\\{ccol}\MG\\{head}.\\{up}\K\\{cur\_node},\39\\{cur\_node}\MG\\{down}\K{%
\AND}\\{ccol}\MG\\{head};{}$\6
${}\\{ccol}\MG\\{len}\PP;{}$\6
${}\\{cur\_node}\PP;{}$\6
${}\\{ccol}\K\\{bcol}[\|l];{}$\6
${}\\{cur\_node}\MG\\{left}\K\\{cur\_node}\MG\\{right}\K\\{cur\_node}-\T{1};{}$%
\6
${}\\{cur\_node}\MG\\{col}\K\\{ccol},\39\\{cur\_node}\MG\\{info}\K\|j;{}$\6
${}\\{cur\_node}\MG\\{up}\K\\{ccol}\MG\\{head}.\\{up},\39\\{ccol}\MG\\{head}.%
\\{up}\MG\\{down}\K\\{cur\_node};{}$\6
${}\\{ccol}\MG\\{head}.\\{up}\K\\{cur\_node},\39\\{cur\_node}\MG\\{down}\K{%
\AND}\\{ccol}\MG\\{head};{}$\6
${}\\{ccol}\MG\\{len}\PP;{}$\6
${}\\{cur\_node}\PP;{}$\6
\4${}\}{}$\2\par
\U29.\fi

\M{31}Our strategy for generating all exact covers will be to repeatedly
choose always the column that appears to be hardest to cover, namely the
column with shortest list, from all columns that still need to be covered.
And we explore all possibilities via depth-first search.

The neat part of this algorithm is the way the lists are maintained.
Depth-first search means last-in-first-out maintenance of data structures;
and it turns out that we need no auxiliary tables to undelete elements from
lists when backing up. The nodes removed from doubly linked lists remember
their former neighbors, because we do no garbage collection.

The basic operation is ``covering a column.'' This means removing it
from the list of columns needing to be covered, and ``blocking'' its
rows: removing nodes from other lists whenever they belong to a row of
a node in this column's list.

\Y\B\4\X31:Dance\X${}\E{}$\6
$\\{level}\K\T{0};{}$\6
\4\\{forward}:\5
\X37:Set \PB{\\{best\_col}} to the best column for branching\X;\6
\\{cover}(\\{best\_col});\6
${}\\{cur\_node}\K\\{choice}[\\{level}]\K\\{best\_col}\MG\\{head}.\\{down};{}$\6
\4\\{advance}:\6
\&{if} ${}(\\{cur\_node}\E{\AND}(\\{best\_col}\MG\\{head})){}$\1\5
\&{goto} \\{backup};\2\6
\&{if} ${}(\\{vbose}>\T{1}){}$\1\5
${}\\{fprintf}(\\{stderr},\39\.{"L\%d:\ \%s\ \%s\\n"},\39\\{level},\39\\{cur%
\_node}\MG\\{col}\MG\\{name},\39\\{cur\_node}\MG\\{right}\MG\\{col}\MG%
\\{name});{}$\2\6
\X35:Cover all other columns of \PB{\\{cur\_node}}\X;\6
\&{if} ${}(\\{root}.\\{next}\E{\AND}\\{root}){}$\1\5
\X39:Record solution and \PB{\&{goto} \\{recover}}\X;\2\6
${}\\{level}\PP;{}$\6
\&{goto} \\{forward};\6
\4\\{backup}:\5
\\{uncover}(\\{best\_col});\6
\&{if} ${}(\\{level}\E\T{0}){}$\1\5
\&{goto} \\{done};\2\6
${}\\{level}\MM;{}$\6
${}\\{cur\_node}\K\\{choice}[\\{level}]{}$;\5
${}\\{best\_col}\K\\{cur\_node}\MG\\{col};{}$\6
\4\\{recover}:\5
\X36:Uncover all other columns of \PB{\\{cur\_node}}\X;\6
${}\\{cur\_node}\K\\{choice}[\\{level}]\K\\{cur\_node}\MG\\{down}{}$;\5
\&{goto} \\{advance};\par
\U23.\fi

\M{32}\B\X27:Local variables\X${}\mathrel+\E{}$\6
\&{register} \&{int} \\{level};\6
\&{register} \&{column} ${}{*}\\{best\_col}{}$;\C{ column chosen for branching
}\par
\fi

\M{33}When a row is blocked, it leaves all lists except the list of the
column that is being covered. Thus a node is never removed from a list
twice.

\Y\B\4\X33:Subroutines\X${}\E{}$\6
\\{cover}(\|c)\1\1\6
\&{column} ${}{*}\|c;\2\2{}$\6
${}\{{}$\5
\1\&{register} \&{column} ${}{*}\|l,{}$ ${}{*}\|r;{}$\6
\&{register} \&{node} ${}{*}\\{rr},{}$ ${}{*}\\{nn},{}$ ${}{*}\\{uu},{}$ ${}{*}%
\\{dd};{}$\6
\&{register} \|k${}\K\T{1}{}$;\C{ updates }\7
${}\|l\K\|c\MG\\{prev}{}$;\5
${}\|r\K\|c\MG\\{next};{}$\6
${}\|l\MG\\{next}\K\|r{}$;\5
${}\|r\MG\\{prev}\K\|l;{}$\6
\&{for} ${}(\\{rr}\K\|c\MG\\{head}.\\{down};{}$ ${}\\{rr}\I{\AND}(\|c\MG%
\\{head});{}$ ${}\\{rr}\K\\{rr}\MG\\{down}){}$\1\6
\&{for} ${}(\\{nn}\K\\{rr}\MG\\{right};{}$ ${}\\{nn}\I\\{rr};{}$ ${}\\{nn}\K%
\\{nn}\MG\\{right}){}$\5
${}\{{}$\1\6
${}\\{uu}\K\\{nn}\MG\\{up}{}$;\5
${}\\{dd}\K\\{nn}\MG\\{down};{}$\6
${}\\{uu}\MG\\{down}\K\\{dd}{}$;\5
${}\\{dd}\MG\\{up}\K\\{uu};{}$\6
${}\|k\PP;{}$\6
${}\\{nn}\MG\\{col}\MG\\{len}\MM;{}$\6
\4${}\}{}$\2\2\6
\4${}\}{}$\2\par
\As34\ET42.
\U1.\fi

\M{34}Uncovering is done in precisely the reverse order. The pointers thereby
execute an exquisitely choreo\-graphed dance which returns them almost
magically to their former state.

\Y\B\4\X33:Subroutines\X${}\mathrel+\E{}$\6
\\{uncover}(\|c)\1\1\6
\&{column} ${}{*}\|c;\2\2{}$\6
${}\{{}$\5
\1\&{register} \&{column} ${}{*}\|l,{}$ ${}{*}\|r;{}$\6
\&{register} \&{node} ${}{*}\\{rr},{}$ ${}{*}\\{nn},{}$ ${}{*}\\{uu},{}$ ${}{*}%
\\{dd};{}$\7
\&{for} ${}(\\{rr}\K\|c\MG\\{head}.\\{up};{}$ ${}\\{rr}\I{\AND}(\|c\MG%
\\{head});{}$ ${}\\{rr}\K\\{rr}\MG\\{up}){}$\1\6
\&{for} ${}(\\{nn}\K\\{rr}\MG\\{left};{}$ ${}\\{nn}\I\\{rr};{}$ ${}\\{nn}\K%
\\{nn}\MG\\{left}){}$\5
${}\{{}$\1\6
${}\\{uu}\K\\{nn}\MG\\{up}{}$;\5
${}\\{dd}\K\\{nn}\MG\\{down};{}$\6
${}\\{uu}\MG\\{down}\K\\{dd}\MG\\{up}\K\\{nn};{}$\6
${}\\{nn}\MG\\{col}\MG\\{len}\PP;{}$\6
\4${}\}{}$\2\2\6
${}\|l\K\|c\MG\\{prev}{}$;\5
${}\|r\K\|c\MG\\{next};{}$\6
${}\|l\MG\\{next}\K\|r\MG\\{prev}\K\|c;{}$\6
\4${}\}{}$\2\par
\fi

\M{35}\B\X35:Cover all other columns of \PB{\\{cur\_node}}\X${}\E{}$\6
$\\{cover}(\\{cur\_node}\MG\\{right}\MG\\{col}){}$;\par
\U31.\fi

\M{36}We included \PB{\\{left}} links, thereby making the rows doubly linked,
so
that columns would be uncovered in the correct LIFO order in this
part of the program. (The \PB{\\{uncover}} routine itself could have done its
job with \PB{\\{right}} links only.) (Think about it.)

(Thus the present implementation is overkill, for the special
case of bipartite matching.)

\Y\B\4\X36:Uncover all other columns of \PB{\\{cur\_node}}\X${}\E{}$\6
$\\{uncover}(\\{cur\_node}\MG\\{left}\MG\\{col}){}$;\par
\U31.\fi

\M{37}\B\X37:Set \PB{\\{best\_col}} to the best column for branching\X${}\E{}$\6
$\\{minlen}\K\\{max\_nodes};{}$\6
\&{if} ${}(\\{vbose}>\T{2}){}$\1\5
${}\\{fprintf}(\\{stderr},\39\.{"Level\ \%d:"},\39\\{level});{}$\2\6
\&{for} ${}(\\{cur\_col}\K\\{root}.\\{next};{}$ ${}\\{cur\_col}\I{\AND}%
\\{root};{}$ ${}\\{cur\_col}\K\\{cur\_col}\MG\\{next}){}$\5
${}\{{}$\1\6
\&{if} ${}(\\{vbose}>\T{2}){}$\1\5
${}\\{fprintf}(\\{stderr},\39\.{"\ \%s(\%d)"},\39\\{cur\_col}\MG\\{name},\39%
\\{cur\_col}\MG\\{len});{}$\2\6
\&{if} ${}(\\{cur\_col}\MG\\{len}<\\{minlen}){}$\1\5
${}\\{best\_col}\K\\{cur\_col},\39\\{minlen}\K\\{cur\_col}\MG\\{len};{}$\2\6
\4${}\}{}$\2\6
\&{if} ${}(\\{vbose}>\T{2}){}$\1\5
${}\\{fprintf}(\\{stderr},\39\.{"\ branching\ on\ \%s(\%d}\)\.{)\\n"},\39%
\\{best\_col}\MG\\{name},\39\\{minlen}){}$;\2\par
\U31.\fi

\M{38}\B\X27:Local variables\X${}\mathrel+\E{}$\6
\&{register} \&{int} \\{minlen};\6
\&{register} \&{int} \|j${},{}$ \|k${},{}$ \|x;\par
\fi

\M{39}\B\X39:Record solution and \PB{\&{goto} \\{recover}}\X${}\E{}$\6
${}\{{}$\1\6
\&{if} ${}(\\{vbose}>\T{1}){}$\1\5
${}\\{fprintf}(\\{stderr},\39\.{"(a\ good\ dance)\\n"});{}$\2\6
\&{for} ${}(\|k\K\T{0};{}$ ${}\|k\Z\\{level};{}$ ${}\|k\PP){}$\5
${}\{{}$\1\6
${}\|j\K\\{choice}[\|k]\MG\\{info};{}$\6
${}\\{acolor}[\|j\AND\T{\^fff}]\K\\{bcolor}[(\|j\GG\T{12})\AND\T{\^fff}]\K\|j%
\GG\T{24};{}$\6
\4${}\}{}$\2\6
\X40:Print a solution\X;\6
\&{goto} \\{recover};\6
\4${}\}{}$\2\par
\U31.\fi

\M{40}\B\X40:Print a solution\X${}\E{}$\6
${}\{{}$\1\6
\&{register} \&{int} \.{OK}${}\K\T{1}{}$;\C{ this (declaration facilitates
change files) }\7
\&{if} (\.{OK})\5
${}\{{}$\1\6
${}\\{count}\PP;{}$\6
${}\\{printf}(\.{"Solution\ \%lld,\ from}\)\.{"},\39\\{count});{}$\6
\&{for} ${}(\|k\K\T{1};{}$ ${}\|k\Z\|d;{}$ ${}\|k\PP){}$\1\5
${}\\{printf}(\.{"\ \%d\^\%d"},\39\|s[\|k],\39\|t[\|k]);{}$\2\6
\\{printf}(\.{":\\n"});\6
\&{for} ${}(\|i\K\T{0};{}$ ${}\|i<\|n\V\|i\Z\\{maxrow};{}$ ${}\|i\PP){}$\5
${}\{{}$\1\6
\&{for} ${}(\|j\K\T{0};{}$ ${}\|j<\|n;{}$ ${}\|j\PP){}$\1\5
${}\\{printf}(\.{"\%c"},\39\|i<\|n\?\\{acolor}[\\{place}(\|i,\39\|j)]+\.{'0'}:%
\.{'\ '});{}$\2\6
\&{if} ${}(\|i\Z\\{maxrow}){}$\5
${}\{{}$\1\6
\\{printf}(\.{"\ \ "});\6
\&{for} ${}(\|j\K\T{0};{}$ ${}\|j\Z\\{maxcol};{}$ ${}\|j\PP){}$\1\5
${}\\{printf}(\.{"\%c"},\39\\{bname}[\\{place}(\|i,\39\|j)][\T{0}]\?\\{bcolor}[%
\\{place}(\|i,\39\|j)]+\.{'0'}:\.{'.'});{}$\2\6
\4${}\}{}$\2\6
\\{printf}(\.{"\\n"});\6
\4${}\}{}$\2\6
\4${}\}{}$\2\6
\4${}\}{}$\2\par
\Us23\ET39.\fi

\M{41}\B\X41:Print statistics about the run\X${}\E{}$\6
$\\{fprintf}(\\{stderr},\39\.{"\%lld\ solutions;\ run}\)\.{\ stats\ \%d,\%lld,%
\%lld,}\)\.{\%lld,\%lld,\%lld.\\n"},\39\\{count},\39\|m,\39\\{counta},\39%
\\{countb},\39\\{countc},\39\\{countd},\39\\{counte}){}$;\par
\U1.\fi

\M{42}\B\X33:Subroutines\X${}\mathrel+\E{}$\6
\&{void} \\{debug}(\&{char} ${}{*}\|s){}$\1\1\2\2\6
${}\{{}$\1\6
\\{fflush}(\\{stdout});\6
${}\\{fprintf}(\\{stderr},\39\.{"***\%s!\\n"},\39\|s);{}$\6
\4${}\}{}$\2\par
\fi

\N{1}{43}Index.
\fi

\inx
\fin
\con
