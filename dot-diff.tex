\input cwebmac
\datethis
\font\logo=logo10
\def\MF{{\logo META}\-{\logo FONT}}


\N{1}{1}Introduction. This program prepares a \MF\ file for a special-purpose
font
that will approximate a given picture. The input file (\PB{\\{stdin}}) is
assumed to be an EPS file output by Adobe Photoshop$^{\rm TM}$
on a Macintosh with the binary EPS option, containing $m$~rows of
$n$~columns each;
in Photoshop terminology the image is $m$~pixels high and $n$~pixels
wide, in grayscale mode, with a resolution of 72 pixels per inch.
The output file (\PB{\\{stdout}}) will be a sequence of $m$~lines like
$$\.{row 10; data "d01...53";}$$
this means that the pixel data for row 10 is the string of $n$~bits
$110100000001\ldots01010011$ encoded as a hexadecimal string of length $n/4$.

For simplicity, this program assumes that $m=512$ and $n=440$.

\Y\B\4\D$\|m$ \5
\T{512}\C{ this many rows }\par
\B\4\D$\|n$ \5
\T{440}\C{ this many columns }\par
\Y\B\8\#\&{include} \.{<stdio.h>}\6
\&{float} ${}\|a[\|m+\T{2}][\|n+\T{2}]{}$;\C{ darknesses: 0.0 is white, 1.0 is
black }\7
\X4:Global variables\X\6
\X12:Subroutines\X\7
${}\\{main}(\\{argc},\39\\{argv}){}$\1\1\6
\&{int} \\{argc};\6
\&{char} ${}{*}\\{argv}[\,];\2\2{}$\6
${}\{{}$\1\6
\&{register} \&{int} \|i${},{}$ \|j${},{}$ \|k${},{}$ \|l${},{}$ \\{ii}${},{}$ %
\\{jj}${},{}$ \|w;\6
\&{register} \&{float} \\{err};\6
\&{float} \\{zeta}${}\K\T{0.2},{}$ \\{sharpening}${}\K\T{0.9};{}$\7
\X2:Check for nonstandard \PB{\\{zeta}} and \PB{\\{sharpening}} factors\X;\6
\X3:Check the beginning lines of the input file\X;\6
\X5:Input the graphic data\X;\6
\X15:Translate input to output\X;\6
\X17:Spew out the answers\X;\6
\4${}\}{}$\2\par
\fi

\M{2}Optional command-line arguments allow the user to change the
\PB{\\{zeta}} and/or \PB{\\{sharpening}} parameters discussed below.

\Y\B\4\X2:Check for nonstandard \PB{\\{zeta}} and \PB{\\{sharpening}} factors%
\X${}\E{}$\6
\&{if} ${}(\\{argc}>\T{1}\W\\{sscanf}(\\{argv}[\T{1}],\39\.{"\%g"},\39{\AND}%
\\{zeta})\E\T{1}){}$\5
${}\{{}$\1\6
${}\\{fprintf}(\\{stderr},\39\.{"Using\ zeta\ =\ \%g\\n"},\39\\{zeta});{}$\6
\&{if} ${}(\\{argc}>\T{2}\W\\{sscanf}(\\{argv}[\T{2}],\39\.{"\%g"},\39{\AND}%
\\{sharpening})\E\T{1}){}$\1\5
${}\\{fprintf}(\\{stderr},\39\.{"\ \ and\ sharpening\ fa}\)\.{ctor\ \%g\\n"},%
\39\\{sharpening});{}$\2\6
\4${}\}{}$\2\par
\U1.\fi

\M{3}Macintosh conventions indicate the end of a line by the
ASCII $\langle\,$carriage return$\,\rangle$ character
(i.e., control-M, aka \.{\char`\\r}), but the \CEE/ library is
set up to work best with newlines (i.e., control-J, aka \.{\char`\\n}).
We aren't worried about efficiency, so we simply input one character
at a time. This program assumes Macintosh conventions.

The job here is to look for the sequence \.{Box:} in the input,
followed by 0, 0, the number of columns, and the number of rows.

\Y\B\4\D$\\{panic}(\|s)$ \6
${}\{{}$\1\6
${}\\{fprintf}(\\{stderr},\39\|s){}$;\5
${}\\{exit}({-}\T{1});{}$\6
\4${}\}{}$\2\par
\Y\B\4\X3:Check the beginning lines of the input file\X${}\E{}$\6
$\|k\K\T{0};{}$\6
\4\\{scan}:\6
\&{if} ${}(\|k\PP>\T{1000}){}$\1\5
\\{panic}(\.{"Couldn't\ find\ the\ b}\)\.{ounding\ box\ info!\\n"});\2\6
\&{if} ${}(\\{getchar}(\,)\I\.{'B'}){}$\1\5
\&{goto} \\{scan};\2\6
\&{if} ${}(\\{getchar}(\,)\I\.{'o'}){}$\1\5
\&{goto} \\{scan};\2\6
\&{if} ${}(\\{getchar}(\,)\I\.{'x'}){}$\1\5
\&{goto} \\{scan};\2\6
\&{if} ${}(\\{getchar}(\,)\I\.{':'}){}$\1\5
\&{goto} \\{scan};\2\6
\&{if} ${}(\\{scanf}(\.{"\%d\ \%d\ \%d\ \%d"},\39{\AND}\\{llx},\39{\AND}%
\\{lly},\39{\AND}\\{urx},\39{\AND}\\{ury})\I\T{4}\V\\{llx}\I\T{0}\V\\{lly}\I%
\T{0}){}$\1\5
\\{panic}(\.{"Bad\ bounding\ box\ da}\)\.{ta!\\n"});\2\6
\&{if} ${}(\\{urx}\I\|n\V\\{ury}\I\|m){}$\1\5
\\{panic}(\.{"The\ image\ doesn't\ h}\)\.{ave\ the\ corect\ width}\)\.{\ and\
height!\\n"});\2\par
\U1.\fi

\M{4}\B\X4:Global variables\X${}\E{}$\6
\&{int} \\{llx}${},{}$ \\{lly}${},{}$ \\{urx}${},{}$ \\{ury};\C{ bounding box
parameters }\par
\As8\ET11.
\U1.\fi

\M{5}After we've seen the bounding box, we look for \.{beginimage\char`\\r};
this will be followed by the pixel data, one character per byte.

\Y\B\4\X5:Input the graphic data\X${}\E{}$\6
$\|k\K\T{0};{}$\6
\4\\{skan}:\6
\&{if} ${}(\|k\PP>\T{10000}){}$\1\5
\\{panic}(\.{"Couldn't\ find\ the\ p}\)\.{ixel\ data!\\n"});\2\6
\&{if} ${}(\\{getchar}(\,)\I\.{'b'}){}$\1\5
\&{goto} \\{skan};\2\6
\&{if} ${}(\\{getchar}(\,)\I\.{'e'}){}$\1\5
\&{goto} \\{skan};\2\6
\&{if} ${}(\\{getchar}(\,)\I\.{'g'}){}$\1\5
\&{goto} \\{skan};\2\6
\&{if} ${}(\\{getchar}(\,)\I\.{'i'}){}$\1\5
\&{goto} \\{skan};\2\6
\&{if} ${}(\\{getchar}(\,)\I\.{'n'}){}$\1\5
\&{goto} \\{skan};\2\6
\&{if} ${}(\\{getchar}(\,)\I\.{'i'}){}$\1\5
\&{goto} \\{skan};\2\6
\&{if} ${}(\\{getchar}(\,)\I\.{'m'}){}$\1\5
\&{goto} \\{skan};\2\6
\&{if} ${}(\\{getchar}(\,)\I\.{'a'}){}$\1\5
\&{goto} \\{skan};\2\6
\&{if} ${}(\\{getchar}(\,)\I\.{'g'}){}$\1\5
\&{goto} \\{skan};\2\6
\&{if} ${}(\\{getchar}(\,)\I\.{'e'}){}$\1\5
\&{goto} \\{skan};\2\6
\&{if} ${}(\\{getchar}(\,)\I\.{'\\r'}){}$\1\5
\&{goto} \\{skan};\2\6
\X6:Input rectangular pixel data\X;\6
\&{if} ${}(\\{getchar}(\,)\I\.{'\\r'}){}$\1\5
\\{panic}(\.{"Wrong\ amount\ of\ pix}\)\.{el\ data!\\n"});\2\par
\U1.\fi

\M{6}Photoshop follows the conventions of photographers who consider
0~to be black and 1~to be white; but we follow the conventions of
computer scientists who tend to regard 0~as devoid of ink (white)
and 1~as full of ink (black).

We use the fact that global arrays are initially zero to assume that
there are all-white rows of 0s above, below, and to the left and right
of the input data.

\Y\B\4\X6:Input rectangular pixel data\X${}\E{}$\6
\&{for} ${}(\|i\K\T{1};{}$ ${}\|i\Z\\{ury};{}$ ${}\|i\PP){}$\1\6
\&{for} ${}(\|j\K\T{1};{}$ ${}\|j\Z\\{urx};{}$ ${}\|j\PP){}$\1\5
${}\|a[\|i][\|j]\K\T{1.0}-\\{getchar}(\,)/\T{255.0}{}$;\2\2\par
\U5.\fi

\N{1}{7}Dot diffusion. Our job is to go from eight-bit pixels to one-bit
pixels;
that is, from 256 shades of gray to an approximation that uses only
black and white. The method used here is called {\it dot diffusion\/}
(see [D.~E. Knuth, ``Digital halftones by dot diffusion,'' {\sl ACM
Transactions on Graphics\/ \bf6} (1987), 245--273]); it works as follows:
The pixels are divided into 64 classes, numbered from 0 to~63. We convert the
pixel values to 0s and 1s by assigning values first to all the pixels of
class~0, then to all the pixels of class~1, etc. The error incurred at each
step is distributed to the neighbors whose class numbers are higher. This is
done by means of precomputed tables \PB{\\{class\_row}}, \PB{\\{class\_col}}, %
\PB{\\{start}},
\PB{\\{del\_i}}, \PB{\\{del\_j}}, and \PB{\\{alpha}} whose function is easy to
deduce from the
following code segment.

\Y\B\4\X7:Choose pixel values and diffuse the errors in the buffer\X${}\E{}$\6
\&{for} ${}(\|k\K\T{0};{}$ ${}\|k<\T{64};{}$ ${}\|k\PP){}$\1\6
\&{for} ${}(\|i\K\\{class\_row}[\|k];{}$ ${}\|i\Z\|m;{}$ ${}\|i\MRL{+{\K}}%
\T{8}){}$\1\6
\&{for} ${}(\|j\K\\{class\_col}[\|k];{}$ ${}\|j\Z\|n;{}$ ${}\|j\MRL{+{\K}}%
\T{8}){}$\5
${}\{{}$\1\6
\X9:Decide the color of pixel \PB{$[\|i,\|j]$} and the resulting \PB{\\{err}}%
\X;\6
\&{for} ${}(\|l\K\\{start}[\|k];{}$ ${}\|l<\\{start}[\|k+\T{1}];{}$ ${}\|l%
\PP){}$\1\5
${}\|a[\|i+\\{del\_i}[\|l]][\|j+\\{del\_j}[\|l]]\MRL{+{\K}}\\{err}*\\{alpha}[%
\|l];{}$\2\6
\4${}\}{}$\2\2\2\par
\U15.\fi

\M{8}We will use the following model for estimating the effect of a given
bit pattern in the output: If a pixel is black, its darkness is~1.0;
if it is white but at least one of its four neighbors is black, its
darkness is~\PB{\\{zeta}}; if it is white and has four white neighbors, its
darkness is~0.0. Laserprinters of the 1980s tended to spatter toner
in a way that could be approximated roughly by taking \PB{$\\{zeta}\K\T{0.2}$}
in
this model. The value of \PB{\\{zeta}} should be between $-0.25$ and $+1.0$.

An auxiliary array \PB{\\{aa}} holds code values
\PB{\\{white}}, \PB{\\{gray}}, or~\PB{\\{black}} to facilitate computations in
this model.
All cells are initially \PB{\\{white}}; but when we decide to make a pixel
\PB{\\{black}}, we change its \PB{\\{white}} neighbors (if any) to \PB{%
\\{gray}}.

\Y\B\4\D$\\{white}$ \5
\T{0}\C{ code for a white pixel with all white neighbors }\par
\B\4\D$\\{gray}$ \5
\T{1}\C{ code for a white pixel with 1, 2, 3, or 4 black neighbors }\par
\B\4\D$\\{black}$ \5
\T{2}\C{ code for a black pixel }\par
\Y\B\4\X4:Global variables\X${}\mathrel+\E{}$\6
\&{char} ${}\\{aa}[\|m+\T{2}][\|n+\T{2}]{}$;\C{ \PB{\\{white}}, \PB{\\{gray}},
or \PB{\\{black}} status of final pixels }\par
\fi

\M{9}In this step the current pixel's final one-bit value is determined.
It presently is either \PB{\\{white}} or \PB{\\{gray}}; we either leave it as
is,
or we blacken it and gray its white neighbors, whichever minimizes the
magnitude of the error.

Potentially \PB{\\{gray}} values near the newly chosen pixel make this
calculation slightly tricky.
Notice, for example, that the very first black pixel to be
created will increase the local darkness of the image by \PB{$\T{1}+\T{4}%
\\{zeta}$}.
Suppose the original image is entirely black, so that \PB{\|a[\|i][\|j]} is~1.0
for \PB{$\T{1}\Z\|i\Z\|m$} and \PB{$\T{1}\Z\|j\Z\|n$}. If a pixel of class~0 is
set to \PB{\\{white}},
the error (i.e., the darkness that needs to be diffused to its
upperclass neighbors) is 1.0; but if it is set to \PB{\\{black}}, the error
is $-4zeta$. The algorithm will choose \PB{\\{black}} unless \PB{$\\{zeta}\G%
\T{.25}$}.

\Y\B\4\X9:Decide the color of pixel \PB{$[\|i,\|j]$} and the resulting \PB{%
\\{err}}\X${}\E{}$\6
\&{if} ${}(\\{aa}[\|i][\|j]\E\\{white}){}$\1\5
${}\\{err}\K\|a[\|i][\|j]-\T{1.0}-\T{4}*\\{zeta};{}$\2\6
\&{else}\5
${}\{{}$\C{ \PB{$\\{aa}[\|i][\|j]\E\\{gray}$} }\1\6
${}\\{err}\K\|a[\|i][\|j]-\T{1.0}+\\{zeta};{}$\6
\&{if} ${}(\\{aa}[\|i-\T{1}][\|j]\E\\{white}){}$\1\5
${}\\{err}\MRL{-{\K}}\\{zeta};{}$\2\6
\&{if} ${}(\\{aa}[\|i+\T{1}][\|j]\E\\{white}){}$\1\5
${}\\{err}\MRL{-{\K}}\\{zeta};{}$\2\6
\&{if} ${}(\\{aa}[\|i][\|j-\T{1}]\E\\{white}){}$\1\5
${}\\{err}\MRL{-{\K}}\\{zeta};{}$\2\6
\&{if} ${}(\\{aa}[\|i][\|j+\T{1}]\E\\{white}){}$\1\5
${}\\{err}\MRL{-{\K}}\\{zeta};{}$\2\6
\4${}\}{}$\2\6
\&{if} ${}(\\{err}+\|a[\|i][\|j]>\T{0}){}$\5
${}\{{}$\C{ black is better }\1\6
${}\\{aa}[\|i][\|j]\K\\{black};{}$\6
\&{if} ${}(\\{aa}[\|i-\T{1}][\|j]\E\\{white}){}$\1\5
${}\\{aa}[\|i-\T{1}][\|j]\K\\{gray};{}$\2\6
\&{if} ${}(\\{aa}[\|i+\T{1}][\|j]\E\\{white}){}$\1\5
${}\\{aa}[\|i+\T{1}][\|j]\K\\{gray};{}$\2\6
\&{if} ${}(\\{aa}[\|i][\|j-\T{1}]\E\\{white}){}$\1\5
${}\\{aa}[\|i][\|j-\T{1}]\K\\{gray};{}$\2\6
\&{if} ${}(\\{aa}[\|i][\|j+\T{1}]\E\\{white}){}$\1\5
${}\\{aa}[\|i][\|j+\T{1}]\K\\{gray};{}$\2\6
\4${}\}{}$\2\6
\&{else}\1\5
${}\\{err}\K\|a[\|i][\|j]{}$;\C{ keep it white or gray }\2\par
\U7.\fi

\N{1}{10}Computing the diffusion tables. The tables for dot diffusion could be
specified by a large number of boring assignment statements, but it is more
fun to compute them by a method that reveals some of the mysterious underlying
structure.

\Y\B\4\X10:Initialize the diffusion tables\X${}\E{}$\6
\X13:Initialize the class number matrix\X;\6
\X14:Compile ``instructions'' for the diffusion operations\X;\par
\U15.\fi

\M{11}\B\X4:Global variables\X${}\mathrel+\E{}$\6
\&{char} \\{class\_row}[\T{64}]${},{}$ \\{class\_col}[\T{64}];\C{ first row and
column for a given class }\6
\&{char} \\{class\_number}[\T{10}][\T{10}];\C{ the number of a given position }%
\6
\&{int} \\{kk}${}\K\T{0}{}$;\C{ how many classes have been done so far }\6
\&{int} \\{start}[\T{65}];\C{ the first instruction for a given class }\6
\&{int} \\{del\_i}[\T{256}]${},{}$ \\{del\_j}[\T{256}];\C{ relative location of
a neighbor }\6
\&{float} \\{alpha}[\T{256}];\C{ diffusion coefficient for a neighbor }\par
\fi

\M{12}The order of classes used here is the order in which pixels might be
blackened in a font for halftones based on dots in a $45^\circ$ grid.
In fact, it is precisely the pattern used in the font \.{ddith300},
discussed in the author's paper ``Fonts for Digital Halftones''
[Chapter~21 of {\sl Selected Papers on Digital Typography\/}].

\Y\B\4\X12:Subroutines\X${}\E{}$\6
\&{void} ${}\\{store}(\|i,\39\|j){}$\1\1\6
\&{int} \|i${},{}$ \|j;\2\2\6
${}\{{}$\1\6
\&{if} ${}(\|i<\T{1}){}$\1\5
${}\|i\MRL{+{\K}}\T{8}{}$;\5
\2\&{else} \&{if} ${}(\|i>\T{8}){}$\1\5
${}\|i\MRL{-{\K}}\T{8};{}$\2\6
\&{if} ${}(\|j<\T{1}){}$\1\5
${}\|j\MRL{+{\K}}\T{8}{}$;\5
\2\&{else} \&{if} ${}(\|j>\T{8}){}$\1\5
${}\|j\MRL{-{\K}}\T{8};{}$\2\6
${}\\{class\_number}[\|i][\|j]\K\\{kk};{}$\6
${}\\{class\_row}[\\{kk}]\K\|i{}$;\5
${}\\{class\_col}[\\{kk}]\K\|j;{}$\6
${}\\{kk}\PP;{}$\6
\4${}\}{}$\2\7
\&{void} ${}\\{store\_eight}(\|i,\39\|j){}$\1\1\6
\&{int} \|i${},{}$ \|j;\2\2\6
${}\{{}$\1\6
${}\\{store}(\|i,\39\|j){}$;\5
${}\\{store}(\|i-\T{4},\39\|j+\T{4}){}$;\5
${}\\{store}(\T{1}-\|j,\39\|i-\T{4}){}$;\5
${}\\{store}(\T{5}-\|j,\39\|i);{}$\6
${}\\{store}(\|j,\39\T{5}-\|i){}$;\5
${}\\{store}(\T{4}+\|j,\39\T{1}-\|i){}$;\5
${}\\{store}(\T{5}-\|i,\39\T{5}-\|j){}$;\5
${}\\{store}(\T{1}-\|i,\39\T{1}-\|j);{}$\6
\4${}\}{}$\2\par
\U1.\fi

\M{13}\B\X13:Initialize the class number matrix\X${}\E{}$\6
$\\{store\_eight}(\T{7},\39\T{2}){}$;\5
${}\\{store\_eight}(\T{8},\39\T{3}){}$;\5
${}\\{store\_eight}(\T{8},\39\T{2}){}$;\5
${}\\{store\_eight}(\T{8},\39\T{1});{}$\6
${}\\{store\_eight}(\T{1},\39\T{4}){}$;\5
${}\\{store\_eight}(\T{1},\39\T{3}){}$;\5
${}\\{store\_eight}(\T{1},\39\T{2}){}$;\5
${}\\{store\_eight}(\T{2},\39\T{3});{}$\6
\&{for} ${}(\|i\K\T{1};{}$ ${}\|i\Z\T{8};{}$ ${}\|i\PP){}$\1\5
${}\\{class\_number}[\|i][\T{0}]\K\\{class\_number}[\|i][\T{8}],\39\\{class%
\_number}[\|i][\T{9}]\K\\{class\_number}[\|i][\T{1}];{}$\2\6
\&{for} ${}(\|j\K\T{0};{}$ ${}\|j\Z\T{9};{}$ ${}\|j\PP){}$\1\5
${}\\{class\_number}[\T{0}][\|j]\K\\{class\_number}[\T{8}][\|j],\39\\{class%
\_number}[\T{9}][\|j]\K\\{class\_number}[\T{1}][\|j]{}$;\2\par
\U10.\fi

\M{14}The ``compilation'' in this step simulates going through the diffusion
process the slow way, recording the actions it performs. Then those actions
can all be done at high speed later.

\Y\B\4\X14:Compile ``instructions'' for the diffusion operations\X${}\E{}$\6
\&{for} ${}(\|k\K\T{0},\39\|l\K\T{0};{}$ ${}\|k<\T{64};{}$ ${}\|k\PP){}$\5
${}\{{}$\1\6
${}\\{start}[\|k]\K\|l{}$;\C{ \PB{\|l} is the number of instructions compiled
so far }\6
${}\|i\K\\{class\_row}[\|k]{}$;\5
${}\|j\K\\{class\_col}[\|k]{}$;\5
${}\|w\K\T{0};{}$\6
\&{for} ${}(\\{ii}\K\|i-\T{1};{}$ ${}\\{ii}\Z\|i+\T{1};{}$ ${}\\{ii}\PP){}$\1\6
\&{for} ${}(\\{jj}\K\|j-\T{1};{}$ ${}\\{jj}\Z\|j+\T{1};{}$ ${}\\{jj}\PP){}$\1\6
\&{if} ${}(\\{class\_number}[\\{ii}][\\{jj}]>\|k){}$\5
${}\{{}$\1\6
${}\\{del\_i}[\|l]\K\\{ii}-\|i{}$;\5
${}\\{del\_j}[\|l]\K\\{jj}-\|j{}$;\5
${}\|l\PP;{}$\6
\&{if} ${}(\\{ii}\I\|i\W\\{jj}\I\|j){}$\1\5
${}\|w\PP{}$;\C{ diagonal neighbors get weight 1 }\2\6
\&{else}\1\5
${}\|w\MRL{+{\K}}\T{2}{}$;\C{ orthogonal neighbors get weight 2 }\2\6
\4${}\}{}$\2\2\2\6
\&{for} ${}(\\{jj}\K\\{start}[\|k];{}$ ${}\\{jj}<\|l;{}$ ${}\\{jj}\PP){}$\1\6
\&{if} ${}(\\{del\_i}[\\{jj}]\I\T{0}\W\\{del\_j}[\\{jj}]\I\T{0}){}$\1\5
${}\\{alpha}[\\{jj}]\K\T{1.0}/\|w;{}$\2\6
\&{else}\1\5
${}\\{alpha}[\\{jj}]\K\T{2.0}/\|w;{}$\2\2\6
\4${}\}{}$\2\6
${}\\{start}[\T{64}]\K\|l{}$;\C{ at this point \PB{\|l} will be 256 }\par
\U10.\fi

\N{1}{15}Synthesis. Now we're ready to put the pieces together.

\Y\B\4\X15:Translate input to output\X${}\E{}$\6
\X10:Initialize the diffusion tables\X;\6
\&{if} (\\{sharpening})\1\5
\X16:Sharpen the input\X;\2\6
\X7:Choose pixel values and diffuse the errors in the buffer\X;\par
\U1.\fi

\M{16}Experience shows that dot diffusion often does a better job if
we apply a filtering operation that exaggerates the differences between
the intensities of a pixel and its neighbors:
$$a_{ij}\gets {a_{ij}-\alpha {\bar a}_{ij}\over 1-\alpha},$$
where ${\bar a}_{ij}={1\over9}\sum_{\delta=-1}^{+1}\sum_{\epsilon=-1}^{+1}
a_{(i+\delta)(j+\epsilon)}$ is the average value of $a_{ij}$ and its
eight neighbors. (See the discussion
in the {\sl Transactions on Graphics\/} paper cited earlier. The
parameter $\alpha$ is the \PB{\\{sharpening}} value, which must obviously
be less than~1.0.)

We could use a buffering scheme to apply this transformation in place,
but it's easier to store the new value of $a_{ij}$ in $a_{(i-1)(j-1)}$
and then shift everything back into position afterwards. The values
of $a_{i0}$ and $a_{0j}$ don't have to be restored to zero after this
step, because they will not be examined again.

\Y\B\4\X16:Sharpen the input\X${}\E{}$\6
${}\{{}$\1\6
\&{for} ${}(\|i\K\T{1};{}$ ${}\|i\Z\|m;{}$ ${}\|i\PP){}$\1\6
\&{for} ${}(\|j\K\T{1};{}$ ${}\|j\Z\|n;{}$ ${}\|j\PP){}$\5
${}\{{}$\5
\1\&{float} \\{abar};\7
${}\\{abar}\K(\|a[\|i-\T{1}][\|j-\T{1}]+\|a[\|i-\T{1}][\|j]+\|a[\|i-\T{1}][\|j+%
\T{1}]+\|a[\|i][\|j-\T{1}]+\3{-1}\|a[\|i][\|j]+\|a[\|i][\|j+\T{1}]+\|a[\|i+%
\T{1}][\|j-\T{1}]+\|a[\|i+\T{1}][\|j]+\|a[\|i+\T{1}][\|j+\T{1}])/\T{9.0};{}$\6
${}\|a[\|i-\T{1}][\|j-\T{1}]\K(\|a[\|i][\|j]-\\{sharpening}*\\{abar})/(\T{1.0}-%
\\{sharpening});{}$\6
\4${}\}{}$\2\2\6
\&{for} ${}(\|i\K\|m;{}$ ${}\|i>\T{0};{}$ ${}\|i\MM){}$\1\6
\&{for} ${}(\|j\K\|n;{}$ ${}\|j>\T{0};{}$ ${}\|j\MM){}$\1\5
${}\|a[\|i][\|j]\K(\|a[\|i-\T{1}][\|j-\T{1}]\Z\T{0.0}\?\T{0.0}:\|a[\|i-\T{1}][%
\|j-\T{1}]\G\T{1.0}\?\T{1.0}:\|a[\|i-\T{1}][\|j-\T{1}]);{}$\2\2\6
\4${}\}{}$\2\par
\U15.\fi

\M{17}Here I'm assuming that \PB{\|n} is a multiple of 4.

\Y\B\4\X17:Spew out the answers\X${}\E{}$\6
\&{for} ${}(\|i\K\T{1};{}$ ${}\|i\Z\|m;{}$ ${}\|i\PP){}$\5
${}\{{}$\1\6
${}\\{printf}(\.{"row\ \%d;\ data\ \\""},\39\|i);{}$\6
\&{for} ${}(\|j\K\T{1};{}$ ${}\|j\Z\|n;{}$ ${}\|j\MRL{+{\K}}\T{4}){}$\5
${}\{{}$\1\6
\&{for} ${}(\|k\K\T{0},\39\|w\K\T{0};{}$ ${}\|k<\T{4};{}$ ${}\|k\PP){}$\1\5
${}\|w\K\|w+\|w+(\\{aa}[\|i][\|j+\|k]\E\\{black}\?\T{1}:\T{0});{}$\2\6
${}\\{printf}(\.{"\%x"},\39\|w);{}$\6
\4${}\}{}$\2\6
\\{printf}(\.{"\\";\\n"});\6
\4${}\}{}$\2\par
\U1.\fi

\N{1}{18}Index.
\fi

\inx
\fin
\con
