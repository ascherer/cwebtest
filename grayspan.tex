\input cwebmac
\datethis




\N{1}{1}Intro. This program was written (somewhat hastily) in order to
experiment
with an algorithm that generates all spanning trees of a given graph,
changing only one edge at a time. Most of the basic ideas are adapted from
Malcolm Smith's M.S. thesis, ``Generating spanning trees'' (University
of Victoria, 1997), which also contains more complex variations that
guarantee better asymptotic performance. I intend to experiment with
those additional bells and whistles later.

The first command line argument is the name of a file that specifies
an undirected graph in Stanford GraphBase
{\mc SAVE\_GRAPH} format; the graph may have
repeated edges, but it must not contain loops. Additional command line
arguments are ignored except that they cause more verbose output.
The least verbose output contains only overall statistics about the total
number of spanning trees found and the total number of mems used.

\Y\B\4\D$\\{verbose}$ \5
$(\\{argc}>\T{2}{}$)\par
\B\4\D$\\{extraverbose}$ \5
$(\\{argc}>\T{3}{}$)\par
\B\4\D$\|o$ \5
$\\{mems}\PP{}$\par
\B\4\D$\\{oo}$ \5
$\\{mems}\MRL{+{\K}}{}$\T{2}\par
\B\4\D$\\{ooo}$ \5
$\\{mems}\MRL{+{\K}}{}$\T{3}\par
\B\4\D$\\{oooo}$ \5
$\\{mems}\MRL{+{\K}}{}$\T{4}\par
\B\4\D$\\{ooooo}$ \5
$\\{mems}\MRL{+{\K}}{}$\T{5}\par
\Y\B\8\#\&{include} \.{"gb\_graph.h"}\6
\8\#\&{include} \.{"gb\_save.h"}\6
\ATH\6
\&{double} \\{mems};\C{ memory references made }\6
\&{double} \\{count};\C{ trees found }\7
\X5:Subroutines\X\7
\\{main}(\&{int} \\{argc}${},\39{}$\&{char} ${}{*}\\{argv}[\,]){}$\1\1\2\2\6
${}\{{}$\1\6
\X3:Local variables\X;\6
\X2:Input the graph\X;\6
\X16:Initialize the algorithm\X;\6
\X7:Generate all spanning trees\X;\6
${}\\{printf}(\.{"Altogether\ \%.15g\ sp}\)\.{anning\ trees,\ using\ }\)\.{%
\%.15g\ mems.\\n"},\39\\{count},\39\\{mems});{}$\6
\\{exit}(\T{0});\6
\4${}\}{}$\2\par
\fi

\M{2}\B\X2:Input the graph\X${}\E{}$\6
\&{if} ${}(\\{argc}<\T{2}){}$\5
${}\{{}$\1\6
${}\\{fprintf}(\\{stderr},\39\.{"Usage:\ \%s\ foo.gb\ [[}\)\.{gory]\ details]%
\\n"},\39\\{argv}[\T{0}]);{}$\6
\\{exit}(\T{1});\6
\4${}\}{}$\2\6
${}\|g\K\\{restore\_graph}(\\{argv}[\T{1}]);{}$\6
\&{if} ${}(\R\|g){}$\5
${}\{{}$\1\6
${}\\{fprintf}(\\{stderr},\39\.{"Sorry,\ can't\ create}\)\.{\ the\ graph\ from\
file}\)\.{\ \%s!\ (error\ code\ \%d)}\)\.{\\n"},\39\\{argv}[\T{1}],\39\\{panic%
\_code});{}$\6
${}\\{exit}({-}\T{1});{}$\6
\4${}\}{}$\2\6
${}\|n\K\|g\MG\|n;{}$\6
\X4:Check the graph for validity and prepare it for action\X;\par
\U1.\fi

\M{3}\B\X3:Local variables\X${}\E{}$\6
\&{register} \&{Graph} ${}{*}\|g{}$;\C{ the graph we're dealing with }\6
\&{register} \&{int} \|n;\C{ the number of vertices }\6
\&{register} \&{int} \|k;\C{ current integer of interest }\6
\&{register} \&{Vertex} ${}{*}\|u,{}$ ${}{*}\|v,{}$ ${}{*}\|w{}$;\C{ current
vertices of interest }\6
\&{register} \&{Arc} ${}{*}\|e,{}$ ${}{*}\\{ee},{}$ ${}{*}\|f,{}$ ${}{*}%
\\{ff}{}$;\C{ current edges of interest }\par
\A8.
\U1.\fi

\N{1}{4}Graph preparation.
While we're checking to see that the graph meets certain minimal standards, we
might as well also compute the degree of each vertex, since our algorithm
will be using that information. We also ensure that the SGB ``edge trick''
works on our computer.

In this program we deviate from normal conventions of the Stanford GraphBase
by using a {\it doubly\/} linked list of arcs from each vertex \PB{\|v}.
Namely,
\PB{$\|v\MG\\{arcs}$} points to a header node \PB{\|h}, and the arcs from \PB{%
\|v} are
\PB{$\|h\MG\\{next}$}, \PB{$\|h\MG\\{next}\MG\\{next}$}, etc., until returning
to \PB{\|h} again. All
arc nodes \PB{\|e} in this list have \PB{$\|e\MG\\{next}\MG\\{prev}\K\|e\MG%
\\{prev}\MG\\{next}\K\|e$}. The
header node is distinguished by the property \PB{$\|h\MG\\{tip}\K\NULL$}.

The ``length'' of each edge is changed to an identifying number
for use in printouts.

\Y\B\4\D$\\{deg}$ \5
$\|u.{}$\|I\C{ utility field \PB{\|u} of each vertex holds its current degree }%
\par
\B\4\D$\\{prev}$ \5
$\|a.{}$\|A\C{ utility field \PB{\|a} of each arc holds its backpointer }\par
\B\4\D$\\{mate}(\|e)$ \5
$(\\{edge\_trick}\AND(\&{siz\_t})(\|e)\?(\|e)-\T{1}:(\|e)+\T{1}{}$)\par
\Y\B\4\X4:Check the graph for validity and prepare it for action\X${}\E{}$\6
\&{if} (\\{verbose})\1\5
${}\\{printf}(\.{"Graph\ \%s\ has\ the\ fo}\)\.{llowing\ edges:\\n"},\39\|g\MG%
\\{id});{}$\2\6
\&{for} ${}(\|v\K\|g\MG\\{vertices},\39\|k\K\T{0};{}$ ${}\|v<\|g\MG%
\\{vertices}+\|n;{}$ ${}\|v\PP){}$\5
${}\{{}$\1\6
${}\|f\K\\{gb\_virgin\_arc}(\,);{}$\6
${}\|f\MG\\{next}\K\|v\MG\\{arcs}{}$;\C{ the new header node }\6
\&{for} ${}(\|v\MG\\{deg}\K\T{0},\39\|e\K\|v\MG\\{arcs},\39\|v\MG\\{arcs}\K%
\|f;{}$ \|e; ${}\|v\MG\\{deg}\PP,\39\|f\K\|e,\39\|e\K\|e\MG\\{next}){}$\5
${}\{{}$\1\6
${}\|e\MG\\{prev}\K\|f;{}$\6
${}\|u\K\|e\MG\\{tip};{}$\6
\&{if} ${}(\|u\E\|v){}$\5
${}\{{}$\1\6
${}\\{fprintf}(\\{stderr},\39\.{"Oops,\ there's\ a\ loo}\)\.{p\ from\ \%s\ to\
itself!}\)\.{\\n"},\39\|v\MG\\{name});{}$\6
${}\\{exit}({-}\T{3});{}$\6
\4${}\}{}$\2\6
\&{if} ${}(\\{mate}(\|e)\MG\\{tip}\I\|v){}$\5
${}\{{}$\1\6
${}\\{fprintf}(\\{stderr},\39\.{"Oops:\ There's\ an\ ar}\)\.{c\ from\ \%s\ to\ %
\%s,\\n"},\39\|u\MG\\{name},\39\|v\MG\\{name});{}$\6
${}\\{fprintf}(\\{stderr},\39\.{"\ but\ the\ edge\ trick}\)\.{\ doesn't\ find\
the\ op}\)\.{posite\ arc!\\n"});{}$\6
${}\\{exit}({-}\T{4});{}$\6
\4${}\}{}$\2\6
\&{if} ${}(\|u>\|v){}$\5
${}\{{}$\1\6
${}\|e\MG\\{len}\K\\{mate}(\|e)\MG\\{len}\K\PP\|k;{}$\6
\&{if} (\\{verbose})\1\5
${}\\{printf}(\.{"\ \%d:\ \%s\ --\ \%s\\n"},\39\|k,\39\|v\MG\\{name},\39\|u\MG%
\\{name});{}$\2\6
\4${}\}{}$\2\6
\4${}\}{}$\2\6
${}\|v\MG\\{arcs}\MG\\{prev}\K\|f,\39\|f\MG\\{next}\K\|v\MG\\{arcs}{}$;\C{
complete the double linking }\6
\&{if} ${}(\|v\MG\\{deg}\E\T{0}){}$\5
${}\{{}$\1\6
${}\\{fprintf}(\\{stderr},\39\.{"Graph\ \%s\ has\ an\ iso}\)\.{lated\ vertex\ %
\%s!\\n"},\39\|g\MG\\{id},\39\|v\MG\\{name});{}$\6
${}\\{exit}({-}\T{5});{}$\6
\4${}\}{}$\2\6
\4${}\}{}$\2\par
\U2.\fi

\M{5}Here's something I might like to use when debugging.

\Y\B\4\X5:Subroutines\X${}\E{}$\6
\&{void} \\{print\_arcs}(\&{Vertex} ${}{*}\|v){}$\1\1\2\2\6
${}\{{}$\1\6
\&{register} \&{Arc} ${}{*}\|a;{}$\7
${}\\{printf}(\.{"Arcs\ leading\ from\ \%}\)\.{s:\\n"},\39\|v\MG\\{name});{}$\6
\&{for} ${}(\|a\K\|v\MG\\{arcs}\MG\\{next};{}$ ${}\|a\MG\\{tip};{}$ ${}\|a\K\|a%
\MG\\{next}){}$\1\5
${}\\{printf}(\.{"\ \%d\ (to\ \%s)\\n"},\39\|a\MG\\{len},\39\|a\MG\\{tip}\MG%
\\{name});{}$\2\6
\4${}\}{}$\2\par
\A17.
\U1.\fi

\N{1}{6}The method. Let $G$ be a graph with $n>1$ vertices.
The basic idea of Smith's algorithm is to generate all spanning trees of~$G$
in such a way that the first one includes a given {\it near-tree}, namely
a set of $n-2$ edges that don't contain a cycle. This task is easy if $n=2$:
We simply list all the edges.

If $n>2$ and if the near-tree is $\{e_1,\ldots,e_{n-2}\}$, we proceed as
follows: First form the graph $G\cdot e_1$ by shrinking edge~$e_1$ (making its
endpoints identical). All spanning trees of~$G$ that include~$e_1$ are
obtained by appending $e_1$ to a spanning tree of $G\cdot e_1$; so we
proceed recursively to generate all spanning trees of $G\cdot e_1$,
beginning with the near-tree $\{e_2,\ldots,e_{n-2}\}$. If no such
trees exist, we stop; in this case $G\cdot e_1$ is not connected,
so $G$ is not connected and it has no spanning trees.
Otherwise, suppose the last spanning tree found
for $G\cdot e_1$ is $f_1\ldots f_{n-2}$. Then we complete our task by
deleting edge $e_1$ and generating all spanning trees in the resulting
graph $G\setminus e_1$, starting with the near-tree $\{f_1,\ldots,f_{n-2}\}$.

\fi

\M{7}This program implements the recursion directly by maintaining an array
of edges $a_1\ldots a_n$. When we enter level~$l$, positions $a_1\ldots
a_{l-1}$ contain edges to include in the tree, and those edges have
been shrunk in the current graph. Position~$a_l$ is effectively blank,
and the remaining positions $a_{l+1}\ldots a_{n-1}$ contain the edges
of a near-tree that should be part of the next spanning tree generated.

We don't delete an edge that is a ``bridge,'' whose removal
would disconnect the current graph. When a non-bridge edge \PB{\|e} is deleted
at
level~\PB{\|l}, we set \PB{$\\{change\_e}\K\|e$}. If the previously
found spanning tree was $a_1\ldots a_{n-1}$, the next tree found
will be $a_1\ldots a_{l-1}a_{l+1}\ldots a_{n-1}e'$ for some new edge~$e'$;
thus it will differ from its predecessor by removing edge \PB{\\{change\_e}}
and replacing it with~$e'$.

It's convenient to keep array element $a_l$ in a utility field within
the vertex array, represented by \PB{\\{aa}(\|l)}. Another such utility field,
\PB{\\{del}(\|l)}, points to a stack of the edges deleted before coming
to a bridge; edges on this list are linked together via a \PB{\\{link}} field.

Experienced readers will not be shocked by the fact that this part of
the program has a \PB{\&{goto}} leading from one loop into another.

\Y\B\4\D$\\{aa}(\|l)$ \5
$(\|g\MG\\{vertices}+\|l)\MG\|z.{}$\|A\C{ the edge $a_l$ }\par
\B\4\D$\\{del}(\|l)$ \5
$(\|g\MG\\{vertices}+\|l)\MG\|y.{}$\|A\C{ the most recent edge deleted on level
\PB{\|l} }\par
\B\4\D$\\{link}$ \5
$\|b.{}$\|A\C{ points from one edge to another }\par
\Y\B\4\X7:Generate all spanning trees\X${}\E{}$\6
$\\{change\_e}\K\NULL;{}$\6
${}\|v\K\|g\MG\\{vertices}{}$;\C{ this instruction needed only if $n=2$ }\6
\&{for} ${}(\|l\K\T{1};{}$ ${}\|l<\|n-\T{1};{}$ ${}\|l\PP){}$\5
${}\{{}$\1\6
${}\|o,\39\\{del}(\|l)\K\NULL;{}$\6
\4\\{enter}:\5
${}\\{ooo},\39\|e\K\\{aa}(\|l+\T{1}),\39\|u\K\|e\MG\\{tip},\39\|v\K\\{mate}(%
\|e)\MG\\{tip};{}$\6
\&{if} ${}(\\{oo},\39\|u\MG\\{deg}>\|v\MG\\{deg}){}$\1\5
${}\|v\K\|u,\39\|e\K\\{mate}(\|e),\39\|u\K\|e\MG\\{tip};{}$\2\6
\X10:Shrink \PB{\|e} by changing \PB{\|u} to \PB{\|v}\X;\6
${}\|o,\39\\{aa}(\|l)\K\|e;{}$\6
\4${}\}{}$\2\6
\&{for} ${}(\|o,\39\|e\K\|v\MG\\{arcs}\MG\\{next};{}$ \|o${},\39\|e\MG%
\\{tip};{}$ \|o${},\39\|e\K\|e\MG\\{next}){}$\5
${}\{{}$\1\6
${}\|o,\39\\{aa}(\|l)\K\|e;{}$\6
\X9:Produce a new spanning tree by changing \PB{\\{change\_e}} to \PB{\|e}\X;\6
${}\\{change\_e}\K\|e;{}$\6
\4${}\}{}$\2\6
\&{for} ${}(\|l\MM;{}$ \|l; ${}\|l\MM){}$\5
${}\{{}$\1\6
${}\|e\K\\{aa}(\|l),\39\|u\K\|e\MG\\{tip},\39\|v\K\\{mate}(\|e)\MG\\{tip};{}$\6
\X11:Unshrink \PB{\|e} by restoring \PB{\|u}\X;\6
\X12:If \PB{\|e} is not a bridge, delete it, set \PB{$\\{change\_e}\K\|e$}, and
\PB{\&{goto} \\{enter}}\X;\6
\X15:Undelete all edges deleted since entering level \PB{\|l}\X;\6
\4${}\}{}$\2\par
\U1.\fi

\M{8}\B\X3:Local variables\X${}\mathrel+\E{}$\6
\&{register} \&{int} \|l;\C{ the current level }\6
\&{Arc} ${}{*}\\{change\_e}{}$;\C{ edge that may change next }\par
\fi

\M{9}\B\X9:Produce a new spanning tree by changing \PB{\\{change\_e}} to \PB{%
\|e}\X${}\E{}$\6
$\\{count}\PP;{}$\6
\&{if} (\\{verbose})\5
${}\{{}$\1\6
\&{if} ${}(\R\\{change\_e}\V\\{extraverbose}){}$\5
${}\{{}$\1\6
${}\\{printf}(\.{"\%.15g:"},\39\\{count});{}$\6
\&{for} ${}(\|k\K\T{1};{}$ ${}\|k<\|n;{}$ ${}\|k\PP){}$\1\5
${}\\{printf}(\.{"\ \%d"},\39\\{aa}(\|k)\MG\\{len});{}$\2\6
\&{if} ${}(\\{extraverbose}\W\\{change\_e}){}$\1\5
${}\\{printf}(\.{"\ (-\%d+\%d)\\n"},\39\\{change\_e}\MG\\{len},\39\|e\MG%
\\{len});{}$\2\6
\&{else}\1\5
\\{printf}(\.{"\\n"});\2\6
\4${}\}{}$\5
\2\&{else}\1\5
${}\\{printf}(\.{"\%.15g:\ -\%d+\%d\\n"},\39\\{count},\39\\{change\_e}\MG%
\\{len},\39\|e\MG\\{len});{}$\2\6
\4${}\}{}$\2\par
\U7.\fi

\M{10}To shrink an edge between \PB{\|u} and \PB{\|v}, we insert \PB{\|u}'s
adjacency list
into \PB{\|v}'s, changing all references to \PB{\|u} into references to~\PB{%
\|v};
those references occur in \PB{\\{tip}} fields of mates of the arcs in
\PB{\|u}'s list.

We also delete all former edges between \PB{\|u} and \PB{\|v},
since those would otherwise become loops. Those former edges are
linked together via their \PB{\\{link}} fields, so that we can restore them
later.

Note that \PB{$\|e\MG\\{tip}\K\|u$}, so \PB{\|e} appears in the \PB{\|v} list
while \PB{\\{mate}(\|e)} appears
in the \PB{\|u} list.

\Y\B\4\D$\\{delete}(\|e)$ \5
$\\{ee}\K\|e,\39\\{oooo},\39\\{ee}\MG\\{prev}\MG\\{next}\K\\{ee}\MG\\{next},\39%
\\{ee}\MG\\{next}\MG\\{prev}\K\\{ee}\MG{}$\\{prev}\par
\Y\B\4\X10:Shrink \PB{\|e} by changing \PB{\|u} to \PB{\|v}\X${}\E{}$\6
$\\{oo},\39\|k\K\|u\MG\\{deg}+\|v\MG\\{deg};{}$\6
\&{for} ${}(\|o,\39\|f\K\|u\MG\\{arcs}\MG\\{next},\39\\{ff}\K\NULL;{}$ \|o${},%
\39\|f\MG\\{tip};{}$ \|o${},\39\|f\K\|f\MG\\{next}){}$\1\6
\&{if} ${}(\|f\MG\\{tip}\E\|v){}$\1\5
${}\\{delete}(\|f),\39\\{delete}(\\{mate}(\|f)),\39\|k\MRL{-{\K}}\T{2},\39\|o,%
\39\|f\MG\\{link}\K\\{ff},\39\\{ff}\K\|f;{}$\2\6
\&{else}\1\5
${}\|o,\39\\{mate}(\|f)\MG\\{tip}\K\|v;{}$\2\2\6
${}\\{oo},\39\|e\MG\\{link}\K\\{ff},\39\|v\MG\\{deg}\K\|k;{}$\6
\&{if} (\\{extraverbose})\1\5
${}\\{printf}(\.{"level\ \%d:\ Shrinking}\)\.{\ \%d;\ now\ \%s\ has\ degr}\)%
\.{ee\ \%d\\n"},\39\|l,\39\|e\MG\\{len},\39\|v\MG\\{name},\39\|v\MG\\{deg});{}$%
\2\6
${}\|o,\39\\{ff}\K\|v\MG\\{arcs}{}$;\C{ now \PB{$\|f\K\|u\MG\\{arcs}$}; we will
merge the two lists }\6
${}\\{oooo},\39\|f\MG\\{prev}\MG\\{next}\K\\{ff}\MG\\{next},\39\\{ff}\MG%
\\{next}\MG\\{prev}\K\|f\MG\\{prev};{}$\6
${}\\{ooo},\39\|f\MG\\{next}\MG\\{prev}\K\\{ff},\39\\{ff}\MG\\{next}\K\|f\MG%
\\{next}{}$;\par
\U7.\fi

\M{11}Unshrinking uses the principle of ``dancing links,'' whereby we exploit
the fact that previously deleted nodes still have good information in
their \PB{\\{prev}} and \PB{\\{next}} fields, provided that we undelete in
reverse order.

\Y\B\4\D$\\{undelete}(\|e)$ \5
$\\{ee}\K\|e,\39\\{oooo},\39\\{ee}\MG\\{next}\MG\\{prev}\K\\{ee},\39\\{ee}\MG%
\\{prev}\MG\\{next}\K{}$\\{ee}\par
\Y\B\4\X11:Unshrink \PB{\|e} by restoring \PB{\|u}\X${}\E{}$\6
$\\{oo},\39\|f\K\|u\MG\\{arcs},\39\\{ff}\K\|v\MG\\{arcs};{}$\6
${}\\{ooo},\39\\{ff}\MG\\{next}\K\|f\MG\\{prev}\MG\\{next};{}$\6
${}\|o,\39\\{ff}\MG\\{next}\MG\\{prev}\K\\{ff};{}$\6
${}\\{ooo},\39\|f\MG\\{prev}\MG\\{next}\K\|f,\39\|f\MG\\{next}\MG\\{prev}\K%
\|f;{}$\6
\&{for} ${}(\|f\K\|f\MG\\{prev};{}$ \|o${},\39\|f\MG\\{tip};{}$ \|o${},\39\|f\K%
\|f\MG\\{prev}){}$\1\5
${}\|o,\39\\{mate}(\|f)\MG\\{tip}\K\|u;{}$\2\6
\&{for} ${}(\\{oo},\39\|f\K\|e\MG\\{link},\39\|k\K\|v\MG\\{deg};{}$ \|f; %
\|o${},\39\|f\K\|f\MG\\{link}){}$\1\5
${}\|k\MRL{+{\K}}\T{2},\39\\{undelete}(\\{mate}(\|f)),\39\\{undelete}(\|f);{}$%
\2\6
${}\\{oo},\39\|v\MG\\{deg}\K\|k-\|u\MG\\{deg};{}$\6
\&{if} (\\{extraverbose})\1\5
${}\\{printf}(\.{"level\ \%d:\ Unshrinki}\)\.{ng\ \%d;\ now\ \%s\ has\ de}\)%
\.{gree\ \%d\\n"},\39\|l,\39\|e\MG\\{len},\39\|v\MG\\{name},\39\|v\MG%
\\{deg}){}$;\2\par
\U7.\fi

\M{12}For bridge detection, we try a heuristic that often gives a quick answer
when the graph is sparse (namely, to test if \PB{\|u} has degree~1).
Or if \PB{$\|e\MG\\{link}\MG\\{link}\I\NULL$}, there was another edge between %
\PB{\|u} and \PB{\|v}.
Otherwise we resort to brute-force breadth-first, testing whether
one can get from \PB{\|u} to \PB{\|v} without \PB{\|e}.

When I put this algorithm in a book, I'll probably leave out the
two quick-try heuristics, in order to keep the algorithm shorter;
breadth-first search will resolve both cases without too much additional
calculation. But for now I'm trying to see how useful they are.

\Y\B\4\D$\\{bfs}$ \5
$\|v.{}$\|V\C{ link for the breadth-first search: nonnull if vertex seen }\par
\Y\B\4\X12:If \PB{\|e} is not a bridge, delete it, set \PB{$\\{change\_e}\K%
\|e$}, and \PB{\&{goto} \\{enter}}\X${}\E{}$\6
\&{if} ${}(\|o,\39\|u\MG\\{deg}\E\T{1}){}$\5
${}\{{}$\1\6
\&{if} (\\{extraverbose})\1\5
${}\\{printf}(\.{"level\ \%d:\ \%d\ is\ a\ b}\)\.{ridge\ with\ endpoint\ }\)\.{%
\%s\\n"},\39\|l,\39\|e\MG\\{len},\39\|u\MG\\{name});{}$\2\6
\&{goto} \\{bridge};\6
\4${}\}{}$\2\6
\&{if} ${}(\|o,\39\|e\MG\\{link}\MG\\{link}){}$\5
${}\{{}$\1\6
\&{if} (\\{extraverbose})\1\5
${}\\{printf}(\.{"level\ \%d:\ \%d\ is\ par}\)\.{allel\ to\ \%d\\n"},\39\|l,\39%
\|e\MG\\{len},\39\|e\MG\\{link}\MG\\{len}\I\|e\MG\\{len}\?\|e\MG\\{link}\MG%
\\{len}:\|e\MG\\{link}\MG\\{link}\MG\\{len});{}$\2\6
\&{goto} \\{nonbridge};\6
\4${}\}{}$\2\6
\&{for} ${}(\|o,\39\|u\MG\\{bfs}\K\|v,\39\|w\K\|u;{}$ ${}\|u\I\|v;{}$ \|o${},%
\39\|u\K\|u\MG\\{bfs}){}$\5
${}\{{}$\1\6
\&{for} ${}(\\{oo},\39\|f\K\|u\MG\\{arcs}\MG\\{next};{}$ \|o${},\39\|f\MG%
\\{tip};{}$ \|o${},\39\|f\K\|f\MG\\{next}){}$\1\6
\&{if} ${}(\|o,\39\|f\MG\\{tip}\MG\\{bfs}\E\NULL){}$\5
${}\{{}$\1\6
\&{if} ${}(\|f\MG\\{tip}\E\|v){}$\5
${}\{{}$\1\6
\&{if} ${}(\|f\I\\{mate}(\|e)){}$\1\5
\X13:Nullify all \PB{\\{bfs}} links and \PB{\&{goto} \\{nonbridge}}\X;\2\6
\4${}\}{}$\5
\2\&{else}\1\5
${}\\{oo},\39\|f\MG\\{tip}\MG\\{bfs}\K\|v,\39\|w\MG\\{bfs}\K\|f\MG\\{tip},\39%
\|w\K\|f\MG\\{tip};{}$\2\6
\4${}\}{}$\2\2\6
\4${}\}{}$\2\6
\&{if} (\\{extraverbose})\1\5
${}\\{printf}(\.{"level\ \%d:\ \%d\ is\ a\ b}\)\.{ridge\\n"},\39\|l,\39\|e\MG%
\\{len});{}$\2\6
\&{for} ${}(\|o,\39\|u\K\|e\MG\\{tip};{}$ ${}\|u\I\|v;{}$ \|o${},\39\|u\MG%
\\{bfs}\K\NULL,\39\|u\K\|w){}$\1\5
${}\|o,\39\|w\K\|u\MG\\{bfs};{}$\2\6
\&{goto} \\{bridge};\6
\4\\{nonbridge}:\5
${}\\{change\_e}\K\|e;{}$\6
\X14:Delete \PB{\|e} and \PB{\&{goto} \\{enter}}\X;\6
\4\\{bridge}:\par
\U7.\fi

\M{13}\B\X13:Nullify all \PB{\\{bfs}} links and \PB{\&{goto} \\{nonbridge}}%
\X${}\E{}$\6
${}\{{}$\1\6
\&{for} ${}(\|o,\39\|u\K\|e\MG\\{tip};{}$ ${}\|u\I\|v;{}$ \|o${},\39\|u\MG%
\\{bfs}\K\NULL,\39\|u\K\|w){}$\1\5
${}\|o,\39\|w\K\|u\MG\\{bfs};{}$\2\6
\&{goto} \\{nonbridge};\6
\4${}\}{}$\2\par
\U12.\fi

\M{14}\B\X14:Delete \PB{\|e} and \PB{\&{goto} \\{enter}}\X${}\E{}$\6
\&{if} (\\{extraverbose})\1\5
${}\\{printf}(\.{"level\ \%d:\ deleting\ }\)\.{\%d\\n"},\39\|l,\39\|e\MG%
\\{len});{}$\2\6
${}\\{ooo},\39\|e\MG\\{link}\K\\{del}(\|l),\39\\{del}(\|l)\K\|e;{}$\6
${}\\{delete}(\|e),\39\\{delete}(\\{mate}(\|e)),\39\\{oo},\39\|e\MG\\{tip}\MG%
\\{deg}\MM,\39\|v\MG\\{deg}\MM;{}$\6
\&{goto} \\{enter};\par
\U12.\fi

\M{15}\B\X15:Undelete all edges deleted since entering level \PB{\|l}\X${}\E{}$%
\6
\&{for} ${}(\|o,\39\|e\K\\{del}(\|l);{}$ \|e; \|o${},\39\|e\K\|e\MG\\{link}){}$%
\5
${}\{{}$\1\6
${}\\{oooo},\39\\{mate}(\|e)\MG\\{tip}\MG\\{deg}\PP,\39\|e\MG\\{tip}\MG\\{deg}%
\PP,\39\\{undelete}(\\{mate}(\|e)),\39\\{undelete}(\|e);{}$\6
\&{if} (\\{extraverbose})\1\5
${}\\{printf}(\.{"undeleting\ \%d\\n"},\39\|e\MG\\{len});{}$\2\6
\4${}\}{}$\2\par
\U7.\fi

\N{1}{16}Getting started. We're done, except for one embarrassing detail:
It is necessary to prime the pump by setting up the original
near-tree $a_2\ldots a_{n-1}$. For this purpose I'll use
depth-first search, since it seems a bit faster than the alternatives.
And I might as well check that the graph is connected.

\Y\B\4\D$\\{sentinel}$ \5
$(\|g\MG\\{vertices}{}$)\par
\Y\B\4\X16:Initialize the algorithm\X${}\E{}$\6
\&{for} ${}(\|v\K\|g\MG\\{vertices}+\T{1};{}$ ${}\|v<\|g\MG\\{vertices}+\|n;{}$
${}\|v\PP){}$\1\5
${}\|v\MG\\{bfs}\K\NULL;{}$\2\6
\&{for} ${}(\|k\K\|n-\T{1},\39\|o,\39\|w\K\|v\K\|g\MG\\{vertices},\39\|w\MG%
\\{bfs}\K\\{sentinel};{}$  ; \|o${},\39\|v\K\|w,\39\|w\K\|w\MG\\{bfs}){}$\5
${}\{{}$\1\6
\&{for} ${}(\\{oo},\39\|e\K\|v\MG\\{arcs}\MG\\{next};{}$ \|o${},\39\|u\K\|e\MG%
\\{tip};{}$ \|o${},\39\|e\K\|e\MG\\{next}){}$\1\6
\&{if} ${}(\|o,\39\|u\MG\\{bfs}\E\NULL){}$\5
${}\{{}$\1\6
${}\|o,\39\\{aa}(\|k)\K\|e,\39\|k\MM;{}$\6
\&{if} ${}(\|k\E\T{0}){}$\1\5
\&{goto} \\{connected};\2\6
${}\|o,\39\|u\MG\\{bfs}\K\|w,\39\|w\K\|u;{}$\6
\4${}\}{}$\2\2\6
\&{if} ${}(\|w\E\\{sentinel}){}$\1\5
\&{break};\2\6
\4${}\}{}$\2\6
\\{printf}(\.{"Oops,\ the\ graph\ isn}\)\.{'t\ connected!\\n"});\5
\\{exit}(\T{0});\6
\4\\{connected}:\6
\&{for} ${}(\|u\K\|g\MG\\{vertices};{}$ ${}\|u<\|g\MG\\{vertices}+\|n;{}$ ${}%
\|u\PP){}$\1\5
${}\|o,\39\|u\MG\\{bfs}\K\NULL;{}$\2\6
\&{if} (\\{extraverbose})\5
${}\{{}$\1\6
\\{printf}(\.{"Depth-first\ search\ }\)\.{yields\ the\ following}\)\.{\
spanning\ tree:\\n"});\6
\\{print\_a}(\|g);\6
\4${}\}{}$\2\6
\&{if} (\\{verbose})\1\5
${}\\{printf}(\.{"(\%.15g\ mems\ for\ ini}\)\.{tialization)\\n"},\39%
\\{mems}){}$;\2\par
\U1.\fi

\M{17}One final debugging aid.

\Y\B\4\X5:Subroutines\X${}\mathrel+\E{}$\6
\&{void} \\{print\_a}(\&{register} \&{Graph} ${}{*}\|g){}$\1\1\2\2\6
${}\{{}$\1\6
\&{register} \&{int} \|k;\7
\&{for} ${}(\|k\K\T{1};{}$ ${}\|k<\|g\MG\|n;{}$ ${}\|k\PP){}$\1\5
${}\\{printf}(\.{"\ a\%d=\%d\ (\%s\ --\ \%s)\\}\)\.{n"},\39\|k,\39\\{aa}(\|k)%
\MG\\{len},\39\\{aa}(\|k)\MG\\{tip}\MG\\{name},\39\\{mate}(\\{aa}(\|k))\MG%
\\{tip}\MG\\{name});{}$\2\6
\4${}\}{}$\2\par
\fi

\N{1}{18}Index.

\fi


\inx
\fin
\con
